\chapter{Окружающая среда и природа}

\section{Городск\'{а}я жизнь}
Вы никогда не думали о последствиях жизни в городе? Казалось бы, \explain{на первый взгляд}{at first sight}, что жизнь в больших экономических и культурных центрах имеет только преимущества, но дальнейшее рассмотр\'{е}ние показывает, что она имеет и \explain{недостатки}{недостаток: disadvantage}.

С \explain{положительной}{полож\'{и}тельный/-ая: positive} стороны, легче найти работу в городе, потому что там обычно много ресторанов, кафе, гостиниц, школ, библиотек, музеев и т.д. Кроме того, жители города имеют прекрасную возможность посетить множество культурных и развлекательных учреждений, таких как музеи, галереи, ночные клубы, дискотеки и многое другое.

С другой стороны, жителям города \explain{приходится}{приходиться/прийтись: have to} жить в загрязненной атмосфере из-за интенсивного автомобильного движения и \explain{промышленных}{industrial} \explain{предприятий}{предпри\'{я}тие: enterprise}. Это может \explain{вызвать}{to cause} \explain{заболевания}{disease} лёгких и проблемы с сердцем. Кроме того, городской образ жизни довольно \explain{напряженный}{intense}, \explain{поскольку}{since/because} приходится много работать, много ездить на автомобиле и, в результате, стоять в пробках...

В заключение, городская жизнь имеет некоторые преимущества. Тем не менее, она также \explain{может нанести ощутимый вред}{can cause significant damage}, так что местные власти должны сделать несколько важных решений, например, они должны \explain{запретить}{[запрещать] to ban} промышленные предприятия в городах и вблизи городов, которые загрязняют воздух и воду токсичными парами.

\section{Причины и последствия загрязнения окружающей среды}
\textit{Источник: \url{https://bit.ly/3NV3JeZ}}

Загрязнение окружающей среды в настоящее время является самой большой проблемой, с которой сегодня \explain{ст\'{а}лкивается}{faces, is facing} мир. Наприм\'{е}р, в Соединенных Штатах 40\% рек и 46\% озёр слишком загрязнен\'{ы} для рыбной л\'{о}вли, \ed{купания}{купание}{bathing} и водных организмов. Это \explain{неудивительно}{not surprising}, когда ежегодно в американские воды \explain{сбрасывается}{dumped} 1,2 триллиона галлонов \explain{неочищенных}{untreated} ливневых вод, промышленных \ed{отходов}{отх\'{о}ды}{waste} и неочищенных \ed{сточных вод}{ст\'{о}чные в\'{о}ды}{sewage}.

Одна треть верхнего \ed{слоя}{спой}{layer} \ed{почвы}{почва}{soil} в мире уже деградирована, и \explain{с учётом}{taking into account} нынешних темпов деградации почвы, вызванной неправильными методами ведения сельского хозяйства и промышленности, а также \ed{обезлесением}{обезлесение}{deforestation}, большая часть верхнего слоя почвы в мире может исчезнуть в течение следующих 60 лет.

Великий смог 1952 года унёс жизни 8000 человек в Лондоне. Это событие \explain{было вызвано}{was caused (by)} периодом холодной погоды \explain{в сочетании с}{in conjunction with} безветренными условиями, которые сформировали \explain{плотный}{dense} слой переносимых по воздуху загрязнителей, в основном от угольных электростанций, над городом.

Существует множество источников загрязнения, каждый из которых по-своему влияет на окружающую среду и живые организмы. В этой статье обсуждаются проблема загрязнения и последствия различных видов загрязнения.

\textbf{Причины.} Причины загрязнения не \explain{ограничиваются}{are limited} только выбросами \ed{ископаемого топлива}{ископ\'{а}емое т\'{о}пливо}{fossil fuel} и \ed{углерода}{углерод}{carbon}. Существует множество других типов загрязнения, включая химическое загрязнение \ed{водоёмов}{водоём}{reservoir} и п\'{о}чвы в результате неправильной утилизации и сельскохозяйственной деятельности, а также шумов\'{о}е и светов\'{о}е загрязнение, создаваемое городами и урбанизацией в результате роста населения.

\textbf{Загрязнение воздуха.}
Существует два типа \ed{загрязнителей}{загрязн\'{и}тель}{pollutant} воздуха: \ed{первичные}{перв\'{и}чный}{primary} и \ed{вторичные}{вторичный}{secondary}. Первичные загрязнители выбрасываются \explain{непосредственно}{directly} из их источника, в то время как вторичные загрязнители образуются, когда первичные загрязнители вступают в реакцию в атмосфере.

\ed{Сжигание}{сжиг\'{а}ние}{burning, combustion} ископаемого топлива для тр\'{а}нспорта и электричества производит как первичные, так и вторичные загрязнители и является одним из крупнейших источников загрязнения воздуха.

\ed{Выхлопные газы}{выхлопные газы}{traffic fumes; exhaust fumes} автомобилей содержат опасные газы и \explain{твёрдые частицы}{solid particles}, включая \ed{углеводороды}{углеводород}{hydrocarbon}, оксиды азота и монооксид углерода. Эти газы \explain{поднимаются}{rise} в атмосферу и \explain{вступают в реакцию}{react} с другими атмосферными газами, создавая ещё б\'{о}лее токсичные газы.

По данным Института Земли, интенсивное использование \ed{удобрений}{удобрение}{fertiliser} в с\'{е}льском хоз\'{я}йстве является основным источником загрязнения воздуха \ed{мелкими частицами}{мелкие частицы}{microparticles}, что \explain{затронуло}{affected} большую часть Европы, России, Китая и США. Считается, что уровень загрязнения, вызванного сельскохозяйственной деятельностью, \explain{превышает}{exceeds} все другие источники загрязнения воздуха мелкими частицами в этих странах.

\ed{Аммиак}{амми\'{а}к}{ammonia} -- это основной загрязнитель воздуха, \explain{образующийся}{emerging} в результате сельскохозяйственной деятельности. Аммиак попадает в воздух в виде газа из концентрированных отходов животноводства и полей, которые \explain{чрезм\'{е}рно}{excessively} уд\'{о}брены.

Затем этот \explain{газообразный}{gaseous} аммиак соединяется с другими загрязнителями, такими как оксиды и сульфаты азота, образующиеся в транспортных средствах и промышленных процессах, с образованием аэрозолей. Аэрозоли -- это \explain{крошечные}{tiny} частицы, которые могут \explain{проникать}{permeate} глубоко в лёгкие и вызывать сердечные и лёгочные заболевания.

Другие сельскохозяйственные загрязнители воздуха включают пестициды, \explain{гербициды}{herbicides} и фунгициды. Все это также способствует загрязнению воды.

\textbf{Загрязнение воды.}
Загрязнение \ed{питательными веществами}{пит\'{а}тельные веществ\'{а}}{nutrients} вызывается сточными водами и удобрениями. Выс\'{о}кие уровни питательных веществ в этих источниках попадают в водоемы и способствуют росту \ed{водорослей}{водоросли}{algae} и \ed{сорняков}{сорняк}{weed}, что может сделать воду \ed{непригодной}{непригодный}{unusable; unfit} для \ed{питья}{питьё}{drinking} и \explain{истощить}{deplete} кислород, что \explain{приведет к гибели}{will lead to death} водных организмов.

Пестициды и гербициды, \explain{применяемые}{used; applied} для сельскохозяйственных культур и жил\'{ы}х районов, концентрируются в почве и перен\'{о}сятся в грунтовые воды с дождевой водой и стоками. По этим причинам каждый раз, когда кто-то пробуривает \ed{скважину}{скважина}{well (water well)} на воду, её необходимо проверять на \explain{наличие}{availability} загрязняющих веществ.

Промышленные отходы являются одной из основных причин загрязнения воды, поскольку они создают первичные и вторичные загрязнители, включая \ed{серу}{сера}{sulfur}, \explain{свинец}{lead} и \explain{ртуть}{mercury}, нитраты и фосфаты, а также разливы нефти.

В \ed{развивающихся странах}{развив\'{а}ющиеся стр\'{а}ны}{developing countries} около 70\% \explain{твёрдых отходов}{solid waste} сбрасывается непосредственно в океан или море. Это вызывает серьёзные проблемы, включая причинение вреда и убийство \ed{морских существ}{морские существа}{sea creatures}, что \explain{в конечном итоге}{eventually} влияет на людей.

\textbf{Загрязнение земли и п\'{о}чвы.}
Загрязнение земель -- это разрушение зем\'{е}ль в результате деятельности человека и неправильного использования земельных ресурсов. Это происходит, когда люди наносят на почву химические вещества, такие как пестициды и гербициды, неправильно утилизируют отходы и \explain{безответственно}{irresponsibly} \explain{эксплуатируют}{exploit} полезные ископаемые при \ed{добыче}{добыча}{mining} полезных ископаемых.

Почва также загрязняется из-за протекающих подземных септиков, канализационных систем, вымывания вредных веществ со \ed{свалок}{свалка}{landfill} и прям\'{о}го сбр\'{о}са сточных вод промышленными \ed{предприятиями}{предприятие}{enterprise} в реки и океаны.

Дождь и наводнение могут переносить загрязнители с других уже загрязнённых зем\'{е}ль в п\'{о}чву в других местах.

Избыточное \explain{земледелие}{agriculture} и чрезм\'{е}рный \explain{в\'{ы}пас}{grazing} в результате сельскохозяйственной деятельности прив\'{о}дят к тому, что п\'{о}чва теряет свою питательную ценность и структуру, вызывая деградацию почвы, ещё один тип загрязнения почвы.

Свалки могут вымывать вредные вещества в почву и водные пути и создавать очень неприятные запахи, а также являются рассадниками \ed{грызун\'{о}в}{грыз\'{у}н}{rodent}, которые являются переносчиками болезней.

\textbf{Шум и световое загрязнение.}
Шум считается загрязнителем окружающей среды, вызываемым бытовыми источниками, общественными мероприятиями, коммерческой и промышленной деятельностью и транспортом.

Световое загрязнение вызвано длительным и чрезмерным использованием искусственного \ed{освещения}{освещение}{lighting} в ночное время, что может вызвать проблемы со здоровьем у людей и нарушить естественные циклы, \explain{в том числе}{including} деятельность \explain{дикой}{wild} природы. Источники светового загрязнения включают электронные рекламные щиты, ночн\'{ы}е спортивные площадки, уличные и автомобильные фонар\'{и}, городск\'{и}е парки, общественные места, аэроп\'{о}рты и жил\'{ы}е районы.

