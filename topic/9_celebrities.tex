\chapter{Знаменитости}

\section{Пётр Налич}
Пётр Андреевич Налич (род. 30 апреля 1981, Москва) --- российский певец и композитор. Пишет песни, музыку к спектаклям, фильмам и мультфильмам.

В первый раз имя Петр Налич \explain{прозвучало}{звучать/прозвучать: to sound (here: was heard)} в 2007 году, когда Петр снял клип на свою песню «Гитар» и выложил её в интернет. Клип вошёл в ТОП-20 самых просматриваемых российских клипов русской версии ютьюба.

С 2008 года в составе МКПН (Музыкальный коллектив Петр\'{а} Налича) пел и \explain{сочинял}{сочинять/сочинить: to compose} песни в стиле, который сам характеризует как «весёлые бабури». \explain{Представил}{представлять/представить: (here) to introduce} Россию на конкурсе Евровидение 2010 как \explain{победитель}{winner} отб\'{о}ра, \explain{прошедшего}{прошедший: past participle of пройти, meaning ``past''} 7 марта 2010 года. Пётр Налич стал первым артистом в истории российской музыкальной индустрии, выпустившим в Интернете альбом («Радость простых мелодий») с использованием системы Pay What You Want («Заплати, сколько хочешь»)

\textbf{1981--2006: детство, юность.}
Пётр Налич родился в Москве в семье архит\'{е}кторов Андрея Захидовича и Валентины Марковны. Старший брат Павел --- художник-оформитель. По \explain{происхождению}{происхождение: origin (here: by origin)} --- дед по отцовской линии --- лирический тенор и диктор югославской редакции Московского радио (с 1993 года --- «Голос России»), Захид Омерович Налич был бошняком из города Тузла в Боснии и Герцеговине.

\begin{fancyquotes}
    Мой дед --- босниец. Был до войны лирическим тенором. Имел ангажемент в Белградской опере. Но потом война, лагерь, там ему фашисты \explain{гортань}{larynx} сломали, больше он не пел. Потом он попал в Советский Союз и здесь всю жизнь работал на радио
\end{fancyquotes}

Налич рассказывал, что его «папа \explain{част\'{е}нько}{quite often} пел за столом... Это и цыганские песни, и романсы». Пётр Налич окончил детскую музыкальную школу им. Мясковского, обучался в музыкальном училище при Московской консерватории им. П. И. Чайковского, а также в студии «Орфей» \explain{под руководством}{under the direction of} Ирины Мухиной. В 2010 году поступил в РАМ имени Гнесиных на специальность «Академическое пение» (класс проф. В. Левко). Среднее образование получил в школе No 1278. Пётр рассказывал, что в школе пел в хард-роковой группе: «Когда голос у меня сломался, стал громким, я стал петь с удовольствием. Вообще я всю жизнь пою. Я музыкальную школу закончил, играли в хард-рок в школе, потом в институте продолжил заниматься музыкой. Чуть позже начал заниматься вокалом и только тогда понял, что пою не очень хорошо».

\textbf{2006--2007: начало карьеры.}
Творчество Петр\'{а} Налича \explain{приобрел\'{о}}{приобретать/приобрести: to gain; to acquire}
известность после того, как он опубликовал на YouTube самостоятельно сделанный клип
на собственную песню «Гитар».
Клип был выложен весной 2007 года. А осенью, в течение одного месяца, клип посмотрело 70000 человек.
Ссылку на р\'{о}лик \explain{пересыл\'{а}ли}{пересылать/переслать: to forward}
друг другу пользователи Livejournal, число просмотров прирастало в день на тысячу. Затем в нескольких \explain{печатных}{печатный/печатные: printed} \explain{изданиях}{издание: publication} появились интервью с Петр\'{о}м и статьи о нём. \explain{Публика}{the public} \explain{потребовала}{требовать/потребовать: to demand} \explain{явления}{явление, the audience demanded the appearance of the artist} артиста.

У Петр\'{а} на тот момент было написано около 40 песен и музыкальных композиций, все они были выложены в свободном \explain{д\'{о}ступе}{д\'{о}ступ: access} на его сайте --- энтузиасты собрали из них архив, который до сих пор можно найти в интернете. Этот материал \explain{лёг}{ложиться/лечь: to lie down (past tense: лёг, легл\'{а}, легл\'{о}, легл\'{и})} в основу репертуара, с которым Пётр 9 ноября 2007 года дал свой первый концерт в клубе «Апшу».

Из-за ажиот\'{а}жа многие не смогли попасть \explain{внутрь}{same as внутри: inside} переполненного зрителями \explain{помещения}{помещение: premises}. Концерт прошёл успешно, появились статьи, \explain{\'{о}тзывы}{\'{о}тзыв: review} в блогах и прессе. Тогда Пётр собрал группу музыкантов и зимой 2008 года дал ещё два концерта в клубе «Икра». Билеты на эти концерты были раскуплены задолго до события. Группа получила название «Музыкальный коллектив Петр\'{а} Налича» (сокращённо МКПН).

\begin{fancyquotes}
    ``Гитар'' я сочинил года четыре тому назад, а клип записали этой весной. Снимали у меня на даче по Щелковскому шоссе. Мы просто тусовались, веселились. У нас там стояла убитая ``копейка'' друга моего брата, которую он никак не мог забрать. Все уже \explain{злились}{злиться: to get angry}, что она стоит на участке. Ну и решили её \explain{употребить}{употреблать/употребить: to use; to consume}, раз уж она здесь. \explain{Набились}{?} в неё все... ну, а дальше вы знаете.
\end{fancyquotes}

\textbf{2007---2010: первый альбом, участие в конкурсе «Евровидение».}
В течение двух следующих лет, помимо концертов в Москве, МКПН посетил с гастролями Петербург, Екатеринбург, Нижний Новгород и другие большие города России. Летом 2008 года МКПН поддерживал российские спортивные команды на Чемпионате Европы по футболу и Олимпиаде 2008 в Пекине. Затем коллектив выпустил свой первый альбом --- «Радость простых мелодий», фильм-концерт «МКПН в Б1 Maximum» и макси-сингл «Море». В 2009 году группа выступила хедлайнером на международном фестивале «Sfinks» в Антверпене.

7 марта 2010 года «Музыкальный коллектив Петр\'{а} Налича» с песней «Lost and Forgotten» был выбран в качестве участника от России на конкурс «Евровидение». Выступление на конкурсе в Осло принесло 11-е место.

\textbf{2010---2012: второй и третий альбомы}
6 апреля 2010 года прошёл первый концерт, на котором Пётр Налич исполнял не свои песни, а классический оперный репертуар и романсы. С тех пор подобные концерты проводятся регулярно. В 2010 году группа выпустила второй студийный альбом «Весёлые Бабури». Продажи стартовали 9 октября в клубе «Арена», а с 11 октября --- в музыкальных магазинах города.

С 2011 года Пётр Налич принимает участие в спектаклях театра-студии оперы РАМ имени Гнесиных под руководством проф. Ю. А. Сперанского. Спел партии Рудольфа в опере «Богема» Дж. Пуччини и Ленского в спектакле по опере П. И. Чайковского «Евгений Онегин».

Осенью 2012 года вышел третий альбом коллектива --- «Золотая рыбка».


\textbf{2013 год: четвёртый и пятый альбомы.}
В апреле 2013 года был выложен в сети четвёртый альбом --- «Песни о любви и родине», записанный Петр\'{о}м Наличем в сопровождении оркестра Ю. Башмета «Новая Россия» (дирижёр Игорь Разумовский). 17 мая 2013 состоялся концерт-презентация альбома в сопровождении оркестра «Новая Россия» в Светлановском зале ММДМ. Основу альбома составили новые композиции, никогда ранее не исполнявшиеся, написанные для вокала в сопровождении классического симфонического оркестра. В конце 2013 года выходит пятый альбом «Кухня», сборник эскизов (2005---2013) --- первый в мире альбом, записанный \explain{целик\'{о}м}{entirely} на языке бабурси. Некоторые композиции из этого альбома впервые были исполнены на новогоднем концерте в Известия-Холл, 29 декабря 2013 года.

\textbf{2014 год.}
22 апреля 2014 года Пётр, являясь артистом Театра-студии РАМ имени Гнесиных, исполнил партию Германа в опере П. И. Чайковского «\explain{П\'{и}ковая дама}{Queen of spades}» в Государственном музее А. С. Пушкина (режиссёр-постановщик --- заслуженный деятель искусств России, профессор Ю. А. Сперанский, режиссёр --- профессор Е. Бабичева)

Также, весной 2014 года МКПН выпустил макси-сингл «Sugar lies», записанный на средства, собранные с помощью акционеров в системе краудфандинг на сайте Планета

3-4 октября 2014 года Пётр Налич принял участие в театрализованных онлайн-чтениях «Каренина. Живое издание»

\textbf{2015 год.}
15 февраля 2015 года Пётр исполнил несколько песен из новой программы на конференции TEDxMoscow. 27 февраля 2015 в московском клубе «Известия-холл» была предст\'{а}влена новая программа «Песни сказочных героев», исполненная в сопровождении небольшого симфонического оркестра и х\'{о}ра (дирижёр Фёдор Сухарников, хормейстер Алия Мухаметгалиева). В неё вошли композиции из альбома «Песни о любви и родине», а также новые композиции, ранее не исполнявшиеся на публике. В течение 2015 года Пётр дал ещё несколько подобных концертов с этим коллективом в Москве, Санкт-Петербурге и Нижнем Новгороде.

Объявлено, что Музыкальный коллектив Петр\'{а} Налича находится в бессрочном творческом отпуске.

Пётр закончил обучение в РАМ им. Гнесиных по специальности «Академическое пение» \explain{с красным дипломом}{with honours}. В качестве дипломного спектакля спел партию Рудольфа («Богема» Пуччини) на сцене театра-студии оперы РАМ им. Гнесиных.

Осенью 2015 года в Российском Академическом Молодёжном Театре состоялась премьера спектакля «Северная Одиссея». Режиссёр --- Екатерина Гранитова. Художник --- Росита Рауд. Пластика, танцы --- Альбертс Албертс. Спектакль поставлен по киносценарию Петр\'{а} Луцыка и Алексея Саморядова. Музыку к спектаклю написал Пётр Налич, и сам же исполняет её на сцене вместе с коллективом музыкантов. 9 декабря 2015 состоялась презентация нового диска с саундтреком к спектаклю «Северная Одиссея».

Кроме того, 26 декабря в театре Вахтангова на малой сцене состоялась премьера детского спектакля «Питер Пэн», музыку для которого также написал Пётр. Режиссёр-хореограф --- Александр Коручеков. Сценография --- Максим Обрезков. Художник по костюмам и гриму --- Мария Данилова. Хореограф --- Сергей Землянский. Художник по свету --- Нарек Туманян. В спектакле также звучит музыка The Beatles, Queen, группы «АукцЫон».

\textbf{2016 год.}
2 февраля 2016 на сцене Пермского ТЮЗа \explain{состоялась премьера}{premiered} спектакля по пьесе Евгения Шварца «Обыкновенное чудо». Режиссёр спектакля Максим Соколов для работы над пьесой приглашает звездную команду: композитора Петр\'{а} Налича, хореографа Артура Ощепкова, художника-постановщика Валентину Серебрянникову. Премьерные спектакли собрали полные залы.

16 апреля 2016 года в ЦКМ НАУ в городе Киеве и 28 апреля в московском Театре Эстрады Пётр представил «неожиданный концерт-презентацию» под названием «Утёсов и не только...». Основную программу концерта составили песни из репертуара Леонида Утёсова. Петр\'{а} сопровождал большой эстрадно-симфонический бэнд под управлением Фёдора Сухарникова. Также на концертах был презентован диск «Утёсов», выпущенный к 120-летнему юбилею певца.

Осенью 2016 года был выпущен диск «Паровоз», записанный на средства, собранные с помощью акционеров в системе краудфандинг на сайте Планета. «Для меня особенно важно, что этот альбом записывается на деньги тех, кто хочет его услышать! Спасибо вам за это!», --- отмечает в своем обращении к поклонникам Пётр. 3 ноября состоялся концерт-презентация в Театре Эстрады. Пётр Налич выступил в сопровождении небольшого эстрадно-симфонического оркестра и хора. Одной из особенностей нового альбома стало то, что все песни и композиции сочинены специально для эстрадно-симфонического оркестра.

Следует отметить, что начиная с 2016 года состав музыкантов кардинально меняется, как и исполняемая Петр\'{о}м музыка. Формируется коллектив, куда приглашаются музыканты, уровень профессионализма которых соответствует новым сложным ритмам и мелодиям. Первым знаковым в этом смысле полноценным концертом стал концерт в Клубе 16 тонн, 10 декабря 2016 года. Презентация бэндовой программы с доработанными эскизами из альбома «Кухня». Пётр Налич und da Bande. --- премьеры: Аэроплан, Gasquerigym dym, Woodwind, Get ma Ola, Dead Husband. Вокал --- Пётр Налич; Хор: Геннадий Ивлев, Анфиса Затонская, Ильфат Баязитов, Мария Григорьева; Музыканты: Алиса Мандрик --- клавиши; Лев Слепнер --- перкуссии; Олег Маряхин --- баритон-саксофон, сопрано-саксофон; Антон Залетаев --- тенор-саксофон, флейта, блок-флейта; Павел Бокий --- труба; Ирина Бокий --- альт-саксофон; Дмитрий Сланский --- барабаны; Сергей Сокулер --- бас-гитара; Сергей Коньков --- гитара.

Продолжается сотрудничество с Оперным театром-студией им. Ю. А. Сперанского. 21 ноября 2016 года Пётр впервые исполняет партию Тамино в опере Моцарта «Волшебная флейта» в постановке Ольги Тимофеевны Ивановой.



\textbf{2017 год.}
17 мая 2017 года состоялся премьерный показ нового спектакля по одноименному рассказу А. П. Чехова «Тина» Академии кинематографического и театрального искусства Н. С. Михалкова на сцене Театрального зала Московского международного Дома музыки. Режиссёр дипломной постановки --- актёр, режиссёр, педагог Александр Коручеков. Сценография и костюмы --- Максим Обрезков. Композитор --- Пётр Налич.

1 октября 2017 года в Москве состоялась мировая премьера оперы-променад «Пиковая дама», в которой Пётр исполнил партию Германа. Классическая опера П. И. Чайковского впервые предстала в иммерсивном формате. Музыку Петр\'{а} Чайковского дополнили произведения неоклассика Николы Мельникова.

Режиссёр --- Александр Легчаков, хореограф --- Олег Глушков, художник-постановщик --- Полина Бахтина, дирижёр-постановщик --- Андрей Рейн.


\section{Виктор Цой}
Виктор Робертович Цой (21 июня 1962 года, Ленинград --- 15 августа 1990 года, близ посёлка Кестерциемс, Латвийская ССР) --- советский рок-музыкант, автор песен и художник. Основатель и лидер рок-группы «Кино», в которой пел, играл на гитаре, писал музыку и стихи. Снялся в нескольких фильмах.

Виктор Цой родился единственным ребёнком в семье инженера корейского происхождения Роберта Максимовича Цоя и преподавательницы физкультуры Валентины Васильевны. Детство музыканта прошло в окрестностях Московского Парка Победы: он родился в роддоме на Кузнецовской улице (располагается внутри парка; сейчас это кардиоцентр), семья до 1990-х гг. жила в примечательном «генеральском доме» на углу Московского проспекта и улицы Бассейной (сейчас это памятник архитектуры). Некоторое время Виктор учился в близлежащей школе на улице Фрунзе, где работала его мама. В 1973 г. родители Цоя развелись, а через год повторно вступили в брак.

С 1974 по 1977 год посещал среднюю художественную школу, где возникла группа «Палата No. 6» во главе с Максимом Пашковым.
После исключения за неуспеваемость из художественного училища имени В. Серова поступил в СГПТУ--61 на специальность резчика по дереву.
В молодости был поклонником Михаила Боярского и Владимира Высоцкого, позднее Брюса Ли, имиджу которого начал подражать.
Увлекался восточными единоборствами и часто дрался «по-китайски» с Юрием Каспаряном.

\textbf{Смерть в автокатастрофе.}
15 августа 1990 года в 12 часов 28 минут Виктор Цой погиб в автокатастрофе. ДТП произошло на 35 километре трассы «Слока --- Талси» под Тукумсом в Латвии, в нескольких десятках километров от Риги. Согласно наиболее правдоподобной официальной версии, Цой заснул за рулём, после чего его «Москвич-2141» тёмно-синего цвета вылетел на встречную полосу и столкнулся с автобусом «Икарус» модели 250 (иногда этот автобус ошибочно идентифицируют как 280 модель.

{\it Столкновение автомобиля «Москвич-2141» тёмно-синего цвета с рейсовым автобусом «Икарус-280» произошло в 12 час. 28 мин. 15 августа 1990 г. на 35 км трассы Слока --- Талси. Автомобиль двигался по трассе со скоростью не менее 130 км/ч, водитель Цой Виктор Робертович не справился с управлением. Смерть В. Р. Цоя наступила мгновенно, водитель автобуса не пострадал. ...В. Цой был абсолютно трезв накануне гибели. Во всяком случае, он не употреблял алкоголь в течение последних 48 часов до смерти. Анализ клеток мозга свидетельствует о том, что он уснул за рулем, вероятно, от переутомления.

--- из милицейского протокола; по данным сайта kinoman.net}

19 августа он был похоронен на Богословском кладбище в Ленинграде.

\textbf{Прочие версии гибели}.
Создатели документального кино из цикла «Следствие вели...» предположили, что Цой мог попасть в аварию, когда решил переставить другой стороной кассету в своём магнитофоне, тем самым отвлекшись от движения у «слепого поворота» дороги. Речь в передаче шла о кассете с демозаписью последнего альбома. Гитарист Юрий Каспарян ещё в 2002 году опроверг информацию о наличии этой кассеты в автомобиле Цоя: «Пользуясь случаем, хочу развеять миф, что на месте аварии нашли кассету с демо ``Черного альбома''... Все было не так. Я специально приехал в Юрмалу с аппаратурой, с инструментами и мы делали аранжировки для нового альбома. Когда доделали, я забрал кассету и поехал в Петербург. Я приехал утром, вечером узнал о случившемся. И поехал обратно. И кассета все время была у меня в кармане».


\textbf{Творчество.}
В конце 1970-х --- начале 1980-х началось тесное общение между Алексеем Рыбиным из хард-роковой группы «Пилигримы» и Виктором Цоем, игравшим на бас-гитаре в группе «Палата № 6», оба они познакомились в гостях у Андрея Панова (Свина), на квартире которого часто собирались компании, а также репетировала его собственная панк-группа «Автоматические удовлетворители».

Виктор Цой и Алексей Рыбин в составе «Автоматических удовлетворителей» ездили в Москву и играли панк-рок-металл на подпольных концертах Артемия Троицкого. Во время аналогичного выступления в Ленинграде по случаю юбилея Андрея Тропилло произошло первое знакомство с Борисом Гребенщиковым

\textbf{Первый альбом.}
Летом 1981 года Виктор Цой, Алексей Рыбин и Олег Валинский основали группу «Гарин и Гиперболоиды», которая уже осенью была принята в члены Ленинградского рок-клуба. Вскоре Валинского забирают в армию, а группа, сменив название на «Кино», весной 1982 приступила к записи дебютного альбома. «Кино» под руководством Бориса Гребенщикова записывались на студии Андрея Тропилло в Доме Юного Техника, в записи принимали участие музыканты «Аквариума». Вскоре с ними же «Кино» дали свой первый электрический концерт в рок-клубе, всё выступление шло под драм-машину, а под песню «Когда-то ты был битником» из-за кулис на сцену выскочили БГ, Майк и Панкер. К лету альбом был полностью завершён, продолжительность его звучания составляла 45 минут, откуда и появилось название. Но позже из окончательного варианта была убрана песня «Я --- асфальт», которую можно найти в переиздании «45», где она прилагается в качестве бонус-трека. Запись получила некоторое распространение, о группе заговорили, начались квартирные концерты в Москве и Ленинграде. Вместе с будущим барабанщиком Зоопарка Валерием Кирилловым осенью этого же года «Кино» записывает в студии Андрея Кускова несколько песен, в том числе «Весна» и «Последний герой», вошедшие в сборник «Неизвестные песни Виктора Цоя» (всего четыре издания).

Тогда запись была забракована и распространения не получила, так как Цой забрал ленту себе.

19 февраля 1983 года проходит совместный электрический концерт «Кино» и «Аквариума», музыканты выступали с тёмным макияжем и в костюмах со стразами. При этом они исполняли «Электричку», «Троллейбус» и «Алюминиевые огурцы». В основной состав был приглашён Юрий Каспарян. Весной из-за разногласий с Цоем Алексей Рыбин покидает группу «Кино». Лето уходит на совместные репетиции с новым гитаристом. В результате этого Виктор Цой и Юрий Каспарян записали альбом «46», который вначале задумывался как демозапись «Начальника Камчатки». Алексей Вишня «скинул» запись нескольким друзьям на плёнку. «46» получил широкое распространение и был воспринят как полноценный альбом. Осенью 1983 года Виктор Цой лёг на обследование в психиатрическую больницу на Пряжке, где провёл полтора месяца, избегая призыва в армию. После выписки из психиатрической клиники он пишет песню «Транквилизатор». Весной выступил на втором фестивале рок-клуба, где группа «Кино» получила лауреатское звание, а песня «Я объявляю свой дом безъядерной зоной», открывшая фестиваль, признана лучшей антивоенной песней фестиваля 1984 года.



\textbf{Второй состав «Кино».}
Летом 1984 года в студии «Антроп» Андрея Тропилло начинается запись альбома «Начальник Камчатки», к которому, кроме Виктора, приложили свою руку БГ и Сергей Курёхин.

В феврале 1984 Виктор и Марьяна празднуют свадьбу. На свадьбу были приглашены Гребенщиков, Майк, Титов, Каспарян, Гурьянов и другие.

Весной 1985 «Кино» заработали ещё одно звание лауреата и засели в студию к А. Тропилло писать «Ночь». Работа над записью затянулась из-за желания создать новую музыку с новыми приёмами игры. Альбом никак не получался, Виктор бросил «Ночь» недоделанной и в студии Алексея Вишни занялся записью «Это не любовь», который получился всего за неделю с небольшим. К осени «Это не любовь» была сведена и удачно разошлась по стране, а в январе 1986 вышла «Ночь», среди песен которой были известные «Мама Анархия» и «Видели ночь». Параллельно с выходом пластинки растёт популярность Виктора Цоя, а в феврале на 4-м фестивале рок-клуба «Кино» получает диплом за лучшие тексты. 5 августа 1985 года у Цоя родился сын Саша.


Летом 1986 года Виктор работал в бане на проспекте Ветеранов, он там мыл помещения из брандспойта. Необходимо было приходить на один час в день, но это было время с 22 до 23 часов, что ему мешало, так как Цой проводил это время суток с группой.

Также летом все участники группы уезжают в Киев на съёмки фильма «Конец каникул» (режиссёр Сергей Лысенко), а чуть позже дают совместный концерт с «Аквариумом» и «Алисой» в ДК МИИТ в Москве, с этими же группами в США выходит «Красная волна». Осенью Сергей Фирсов приглашает Виктора работать кочегаром. Цой соглашается, и они оба начинают работать кочегарами в котельной «Камчатка», откуда выросли многие знаменитые рок-музыканты.

В ней Рашид Нугманов организовал съёмки короткометражки «Йя-Хха», там же проходят съёмки фильма «Рок» Алексея Учителя --- оба фильма при участии Цоя. Осень и зима проходят в Ялте на съёмках «Ассы» Сергея Соловьёва.

Весна 1987 богата концертными событиями: премьера «Ассы» в ДК МЭЛЗ, последнее участие на фестивале рок-клуба, где «Кино» получили приз «За творческое совершеннолетие».

На порто-студии «Yamaha MT44» «Кино» начинают записывать альбом «Группа крови». Осенью 1987 года Виктор улетает к Рашиду Нугманову в Алма-Ату на съёмки своего последнего фильма «Игла», в связи с этим «Кино» доработали «Группу крови» и на время прекратили концертную деятельность. В 1988 выходит «Игла» и «Группа крови», которые породили «киноманию».

Начинаются триумфальные гастроли по Советскому Союзу --- «Кино» собирают аншлаги на всех концертах.

16 ноября 1988 на мемориальном концерте памяти Александра Башлачёва публика ведёт себя крайне активно; по плану концерт должна была заканчивать песня Башлачёва «Время колокольчиков» (в записи), памяти которого был посвящён концерт, но по невыясненным причинам во время выступления Цоя (он играл на гитаре) внезапно включили «Время колокольчиков», Цой прекратил играть, не понимая откуда идёт звук, который он не производит и что вообще происходит. Администрация многократно объявляла, что всем н\'{у}жно расходиться, концерт окончен. Цой не уходил, он несколько раз подходил к выключенным микрофонам и проверял, работают ли. Потом разводил руками --- «не работает», и ходил по сцене туда и сюда с цветком, не уходя со сцены, но и не имея возможности петь и что-то сказать публике. Публика не расходилась, люди шумели, кричали, было видно, что что-то идёт не так. Создавалось впечатление, что некая злая воля решила прекратить концерт и включила финальную песню прямо во время выступления Цоя. Через 10 минут этого противостояния администрация включила микрофон. Цой, в очередной раз подойдя проверять микрофон, услышал что он включён, и объявил людям, что по непонятным причинам несвоевременно была включена финальная песня Саши Башлачева, но после этого петь и играть уже не очень удобно. После этого он стал собираться и публика потянулась к выходу.

Весной 1988 записывается черновик, а зимой окончательный вариант альбома «Звезда по имени Солнце», который решили выпустить осенью. Цой знакомится с Юрием Айзеншписом, который с 1989 стал продюсером «Кино», организовывая концертные туры и частые выступления на телевидении, после чего группа обретает всесоюзную популярность. В день 50-летия Цоя Александр Градский в эфире канала «Москва-24» рассказал, что в тот период Артемий Троицкий инспирировал письмо в Московский Горком, которое должно было настроить московских рок-музыкантов против Виктора Цоя.

На телевидении Виктор Цой дебютировал в программе «Взгляд», об этом рассказано в книге «Взгляд» --- битлы перестройки.

В начале 1989 группа «Кино» впервые едет за границу во Францию, где выпускают альбом «Последний герой». Летом Виктор с Юрием Каспаряном едут в США. Тем временем «Игла» выходит на второе место в прокате советских фильмов, а на кинофестивале «Золотой Дюк» в Одессе Виктора Цоя признают лучшим актёром СССР.

24 июня 1990 года прошёл последний концерт «Кино» в Москве на Большой спортивной арене Лужников. На этом концерте, впервые после московской Олимпиады-80 был зажжён огонь в Олимпийской чаше. После этого Цой с Каспаряном уединились на даче под Юрмалой, где на порто-студию начали записывать материал для нового альбома. Этот альбом, дописанный и сведённый музыкантами группы «Кино» уже после смерти Цоя, вышел в январе 1991 и получил символическое название «Чёрный альбом», с соответствующим оформлением обложки.

\newpage
\section{Елена Ваенга}
Ел\'{е}на В\'{а}енга (настоящее имя --- Ел\'{е}на Влад\'{и}мировна Хрулёва; род. 27 января 1977, Североморск, Мурманская область, РСФСР, СССР) --- российская \explain{эстрадная певица}{pop singer}, автор песен, актриса. Лауреат премий «Шансон года».

В\'{а}енга --- это название родного для Елены Хрулёвой города Североморска до 18 апреля 1951 года, а также реки недалеко от него. В основе названия и псевдонима --- саамское слово «\explain{олен\'{и}ха}{deer}» (кильд. вайонгг). Псевдоним придуман её матерью.

\textbf{Биография.} Родилась 27 января 1977 года в Североморске. \explainDetail{Петь}{петь}{to sing} и \explain{обучаться}{to study (+ dative)} музыке начала с трёх лет.

Мать Елены Ваенги по образованию химик, отец --- инженер, работали в посёлке Вьюжный на \explain{судоремонтном заводе}{shipyard (судно: vessel; ship, plural: суда)} «Нерпа», который обслуживает атомные \explain{подводные лодки}{подводная лодка: submarine}. Про отца и родной Север у Елены Ваенги есть песня:

{\it У меня глаза северных цветов,\\
И мне не нужны тропические страны.\\
Я всегда с тобой рядышком была.\\
Жаль, что ты уехал слишком рано.\\
Я вдруг поняла: все эти города\\
Я должна пройти, как в наказанье.\\
Но у меня есть дом, а у дома --- я,\\
А у Севера --- сиянье}

Дед Елены со стороны матери --- контр-адмирал Северного флота Василий Семёнович Журавель, он упоминается в книге «Знаменитые люди Санкт-Петербурга». Бабушка Надежда Георгиевна Журавель (её крёстная) (род. 1927). Про неё у Елены Ваенги есть песня: «Моя бабушка любит суши...». Родители отца --- \explain{коренные}{коренн\'{о}й ж\'{и}тель; коренн\'{ы}е ж\'{и}тели: indigenous} петербуржцы, \explain{пережили}{(переживать/пережить) to survive; to experience} блокаду Ленинграда. Дед по линии отца --- зенитчик, во время \explain{Великой Отечественной войны}{second world war} \explainDetail{воевал}{воевать/повоевать}{to fight} под Ораниенбаумом, а бабушка по линии отца была врачом в госпитале в блокадном Ленинграде.

У Елены Ваенги есть младшая сестра Татьяна, она работает в дипломатической сфере, знает несколько языков.

Гражданский муж Елены Ваенги \explain{на протяжении 16 лет}{for 16 years} с 1995 по 2011 год --- Иван Иванович Матвиенко (род. 1957) --- продюсер певицы, по национальности цыган, был женат, его дочь на 2 года старше Елены Ваенги, раньше Иван перегонял машины из Германии.

Племянник, Руслан Сулимовский --- директор её коллектива.

В ночь с пятницы на субботу 10 августа 2012 года Ваенга в родильном доме No. 16 Санкт-Петербурга родила сына Ивана. 30 сентября 2016 года Елена официально вышла замуж за Романа Садырбаева.

\textbf{Творческая \explain{деятельность}{activity}.} Первую песню «Голуби» написала в 9 лет, стала победительницей Всесоюзного конкурса молодых композиторов на Кольском полуострове. После школы приехала в Санкт-Петербург, где закончила музыкальное училище им. Н. А. Римского-Корсакова по классу фортепиано, получив диплом педагога-концертмейстера. Некоторое время преподавала музыку в школе. Факультативно занималась вокалом.

Елена Ваенга с детства мечтала стать актрисой, поэтому после музыкального училища поступила в Театральную академию (ЛГИТМИК) на курс Г. Тростянецкого, но проучилась лишь два месяца, так как её пригласили в Москву записывать первый альбом. Продюсером певицы стал Степан Разин. Под псевдонимом Нина она выпустила клип на песню «Длинные коридоры» (композиция была издана в 2011 году на сборнике «Живая струна»). Альбом был записан, но не вышел. Разочаровавшаяся в шоу-бизнесе певица сбежала от Разина и уехала в Санкт-Петербург. Тем временем её песни взяли в свой репертуар Александр Маршал («Невеста»), Татьяна Тишинская («А ты налей мне белого вина», «Мама, что ты плачешь», «Володенька», «Угостите даму сигаретой»), группы «Стрелки» («Тонкая веточка»), «Божья коровка» («Сердце моё», «Самая любимая моя») и другие известные исполнители. Эти песни распространил её бывший продюсер. Елена Ваенга приняла решение с ним не судиться.

В Санкт-Петербурге Ваенга узнала, что в Балтийском институте экологии, политики и права на кафедре театрального искусства набирает курс П. С. Вельяминов, и в 2000 году пошла учиться к нему. Закончив курс, получила диплом по специальности «драматическое искусство». Выступила в антрепризном спектакле «Свободная пара» в паре с однокурсником Андреем Родимовым (режиссёр Екатерина Шимилёва).

Концертирует с девятнадцати лет. Лауреат петербургского конкурса «Шлягер года 1998» с песней «Цыган», «Достойная песня 2002». Участник концертов-фестивалей «Весна романса» в БКЗ «Октябрьский», «Вольная песня над вольной Невой», «Невский бриз». Дала несколько сольных концертов в ДК имени М.Горького. Гастролирует по России и другим странам и каждый год, в конце января, даёт концерт в БКЗ «Октябрьский» по случаю своего дня рождения.

Настоящая популярность пришла к певице в 2005 году с выходом альбома «Белая птица», в котором было много хитов: «Желаю», «Аэропорт», «Тайга», «Шопен» и заглавная композиция, на которую вышел клип.

28 ноября 2009 года Елена Ваенга получила свой первый приз «Золотой граммофон» за песню «Курю».

4 декабря 2010 года Елена повторила свой успех, получив во второй раз премию «Золотой граммофон» за песню «Аэропорт». В том же году певица впервые стала лауреатом фестиваля «Песня года», исполнив композицию «Абсент». А 12 ноября она дала первый в своей концертной деятельности сольный концерт в Государственном Кремлёвском дворце, трансляция которого прошла на Первом канале 7 января 2011 года. В телеанонсе Елене Ваенга была дана следующая характеристика:
Елена Ваенга --- тонкая, художественная, мечтательная и романтичная натура. Музыкальная одарённость, природный темперамент, трудолюбие, жизнерадостность --- всё это составляющие её жизни и творчества... Несмотря на внешнюю хрупкость и молодость, за спиной у этой очаровательной девушки богатая творческая биография и не такая уж простая человеческая судьба. Жанр, в котором работает певица, с большим трудом определяет даже она сама: «На 50 процентов это фолк-рок, есть старинные баллады, городские романсы, шансон. Но границы между ними провести почти невозможно.»
--- анонс на Первом канале --- «Белая птица». Концерт Елены Ваенги
В 2011 году Елена Ваенга приняла участие в ежегодной церемонии национальной премии Шансон года в Кремле, на которой исполнила песни «Оловянное сердце» и «Девочка». Популярность певицы растёт. В январе этого же года она победила Леонида Агутина в телепередаче «Музыкальный ринг» на канале НТВ, набрав почти в пять раз больше голосов слушателей.

26 ноября 2011 года певица в третий раз получила премию «Золотой граммофон» за песню «Клавиши», но на концерте в Кремле исполнила композицию «Шопен». 21 декабря 2011 года певица в третий раз дала концерт в Кремле. В 2012 году на «Золотой граммофон» претендовали песни «Шопен» и «Где была».

В 2011 году Елена Ваенга дала 150 афишных концертов, гастролировала в США, Германии, Израиле.

Периодически играет в спектакле «Свободная пара», совместно с Андреем Родимовым.

В 2011 году Ваенга впервые попала в список самых успешных деятелей российского шоу-бизнеса, составленный Forbes, и заняла в нём девятую позицию, с годовым доходом более шести миллионов долларов.

В репертуаре певицы --- собственные песни, старинные и современные романсы, баллады и народные песни, а также песни на стихи классиков, таких, как Сергей Есенин («Задымился вечер») и Николай Гумилёв («Жираф», «Шут»). В 2012 году певица провела концертный тур по Украине и Германии. Однако на этом деятельность певицы оборвалась в связи с потерей голоса из-за механического повреждения связок. После выздоровления певица дала последние концерты в городах Средней Волги и ушла в отпуск.

В ноябре 2012 года певица вышла из декрета и возобновила концертную деятельность. По сведениям журнала Forbes, в 2012 году певица в списке самых успешных российских деятелей шоу-бизнеса заняла четырнадцатое место. Сама артистка это отрицает, также как и в прошлом году, утверждая, что её доход гораздо меньше. На данный момент артистка активно гастролирует.

В 2014 году Елена Ваенга стала одним из членов жюри шоу Первого канала «Точь-в-точь».

27 ноября 2015 года состоялся сольный концерт Ваенги в Государственном Кремлёвском дворце, где она выпустила новую программу и представила новый альбом.


\section{Почему Джордж Майкл взял себе такой псевдоним?}

Настоящее имя британского певца Джорджа Майкла --- Йоргос (Георгиос) Кириакос Панайоту. В 1982 году молодой дуэт «Wham!», в составе которого он выступал, выпустил свой первый официальный сингл «Wham Rap!». На его обложке, помимо второго вокалиста Эндрю Риджли, красовалось имя Джордж Панос. Джордж --- это производное от Георгиос, а Панос --- фамилия отца поп-звезды, которую тот взял себе, эмигрировав из Греции в Великобританию в 1950-х годах. Он сменил имя Кириакос Панайоту на Джек Панос.

«Это было, когда я изменил своё имя. Я всегда знал, что мне придётся изменить его, но они уже начали выпуск ``Wham Rap!'', а я всё ещё не выбрал себе псевдонима. Так что около двадцати тысяч экземпляров того первого релиза вышли с моим настоящий именем на обложке: Джордж Панос. И на этом этапе я понимал, что должен буду выбрать что-нибудь», --- позже рассказал в своём интервью Джордж Майкл.

Однако покорять мировые чарты с именем Джордж Панос певцу не хотелось, и он решил придумать себе более яркий псевдоним. Ему всегда нравилось имя Майкл, так звали брата его отца и школьного приятеля, также имевшего греческие корни. Поэтому певец принял решение взять себе фамилию Майкл. Все последующие хиты певец исполнял под псевдонимом Джордж Майкл.

«Я подумал, что это было бы отличным псевдонимом. Оно легко произносилось, и я не должен был отказываться от своего греческого происхождения полностью. Я не забывал об этом, хотя большинство людей решили, что это было еврейским именем», --- рассказал Джордж Майкл.