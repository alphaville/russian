% !TeX root = ../russian-vocab-2021-22.tex

\chapter{Новости}

\section{Ее можно считать жертвой}
\textit{Никита Абрамов, Наталья Гранина}\\
\url{https://lenta.ru/articles/2021/12/22/kinder/}

\textit{Отец девятилетней студентки МГУ \explainDetail{нап\'{а}л}{нападать/напасть}{attack} на людей в вузе. Что о семье Тепляковых думают педагоги?}

У девочки-вундеркинда Алисы Тепляковой, которая в восемь лет \explain{сдала ЕГЭ}{сдавать/сдать экзамен} и поступила на платное отделение психфака МГУ, началась первая сессия. Во вторник, 21 декабря, появилось видео, где ее отец Евгений Тепляков пытается прорваться через пост охраны факультета с криками, что он «хочет поговорить с преподавателями». Администрация факультета вынуждена была прятать их от агрессивного родителя. Еще в сентябре, когда стало известно, что Алиса станет студенткой МГУ, «Лента.ру» попросила прокомментировать это событие известных педагогов. Все они сомневались, что для психики маленького ребенка учеба в вузе будет \explainDetail{посильной}{посильный/-ая}{Within one's strength, abilities, powers (for a task); посильная задача} задачей. Возможно, \explainDetail{опасения}{опасение}{fear} начинают \explainDetail{сбываться}{сбываться/сбыться}{to come true}. Мы публикуем их мнения об Алисе и методах ее отца.

\subsection{Будет не столько студенткой, сколько подопытным объектом}
\textit{Леонид Кацва, автор учебников и пособий по истории России. Преподаватель московской школы № 1543}

Я смотрел видеоинтервью с Алисой Тепляковой. У нее, видимо, очень тренированная \explainDetail{память}{память (ж)}{memory}. Каких-то других качеств она не показала в выступлении. Разговаривает она как семилетка, уровень ее понимания ситуации --- типичный для маленького ребенка. У нас в школе на \explain{педсовете}{teachers' council} перед началом учебного года говорили об этом, многие считают, что вся эта история очень дурно пахнет. \explainDetail{Имеется в виду}{имеется в виду}{it means} не то что девочка сдала ЕГЭ --- \explain{вызубрить}{memorise} какие-то вещи по нескольким предметам на минимальный балл она могла, если у нее действительно вот такая память. Я видел, как она читает --- быстро, но \explain{судя}{judging} по всему общего \explainDetail{смысла}{смысл}{meaning} текста не понимает. Папа \explain{дрессировал}{trained} детей именно на скорочтение. А скорочтение --- это немного не про чтение в том смысле, как мы его понимаем.

У меня нет вопросов к папе. Он хочет доказать некую идею --- что можно в школе не учиться 11 лет, а \explainDetail{освоить}{осваивать/освоить}{to master} все за три года. К девочке у меня тоже вопросов нет. Потому что в данном случае она --- \explain{орудие}{tool} в руках папы, \explain{в какой-то мере}{to an extent} ее даже можно считать жертвой.

У меня есть вопрос к МГУ: \explainDetail{принять}{принимать/принять}{accept} девятилетнего ребенка на психфак --- это надо все же сильно постараться.  Но гораздо больше у меня вопросов к школе,  которая ее  выпустила. Я ничего про эту школу не знаю. Даже не знаю номера. Видимо, она была на домашней форме   \explainDetail{обучения}{обучение}{learning; education}, на уроки не ходила. Не знаю, как это было оформлено --- экстернат или домашнее обучение, этого не могу сказать. Но если она в восемь лет сдала экзамены за все годы обучения, то, \explain{грубо говоря}{roughly speaking}, девочка должна была с шести лет сдавать экзамены каждые два-три месяца. Это если их принимали.

Папа говорит совершенно открыто, что девочка не прочитала ни одного программного литературного произведения, что она знакомилась с художественными книгами в виде кратких пересказов. Я понимаю, что так некоторые дети и делают, даже в 17-летнем возрасте. Но \explain{все-таки}{even so} это принято скрывать, а не превращать в манифест.

Я 40 лет преподаю историю и, как говорится, зуб даю, что если ребенка начать спрашивать не на уровне тестов, кто командовал теми-то войсками, кто был генеральным секретарем тогда-то, министром, великим князем тогда-то, а начать спрашивать \explain{всерьёз}{seriously}, с причинно-следственными связями, с характеристиками событий, то не о чем будет говорить. Специалисты по естественным наукам и физике также замечают, что даже на уровне физиологии не может ребенок в таком возрасте эти дисциплины качественно осваивать.

У нас были вундеркинды. И я знаю случаи, когда в вуз приходил учиться 14-летний студент. Однако разница между 14 и 17 годами, когда \explain{пол\'{о}жено}{one should, one ought to, one is supposed to} сдавать ЕГЭ, \explain{на порядок}{by an order of magnitude} меньше.
Я уж не говорю о разнице между 17 годами и девятью. Поэтому я в данном случае вижу какую-то \explain{недобросовестность}{dishonesty} с разных сторон. И прежде всего --- школы.
Возможно, она просто решила \explain{подыграть}{play along} папе, не знаю почему. Либо просто отвязаться от этого папы. Потому что папа такой, что проще согласиться на его \explainDetail{условия}{условие}{condition; term}, чем объяснять ему, почему этого делать не стоит.
Но, с другой стороны, есть и контраргумент, почему это может быть не так. За Алисой --- на подходе очередь из ее братьев и сестер. \explainDetail{Причём}{причём}{moreover} если у Алисы имя обычное --- среди девочек школьного возраста Алисы встречаются, то у остальных детей в семье имена скандинавских богов, а не детей из России. И тут у меня \explain{ощущение}{sensation}, что психологическое состояние папы от старшего ребенка к младшим начало \explainDetail{усугубляться}{усугубляться/усугубиться}{get worse}.

На мой взгляд, тут широкое поле деятельности для \explain{Рособрнадзора}{Федеральная служба по надзору в сфере образования и науки (Рособрнадзор): Federal Service for Supervision in Education and Science}. Не думаю, что тут речь о \explainDetail{мошенничестве}{мошенничество}{fraud; cheating} при ЕГЭ --- ребенок с натренированной памятью мог рассчитывать на минимальные баллы, чтобы экзамен считался сданным. Но \explain{полноценное}{of full value} среднее образование она получить не могла. Девочка под папиным \explainDetail{внушением}{внушение}{suggestion} говорит: в школе 11 лет учатся, а в институте --- пять, значит, институт --- проще, я его окончу за два года. Эти слова ребенка \explain{цитируют}{quote} \explain{СМИ}{средства м\'{а}ссовой информации}. Предположим, она окончит институт в 11 лет. Вы пойдете на консультацию к такому специалисту-психологу? Я, честно говоря, \explainDetail{остерегусь}{остерегаться/остеречься}{beware}.

\begin{fancyquotes}
    Это моя гипотеза --- и кроме догадок она ни на чем не основана, --- что на психфак ее приняли не столько для того, чтобы обучать как полноценного студента, сколько для того, чтобы ставить своего рода эксперимент. То есть в этой ситуации она будет не столько студенткой, сколько подопытным объектом, потому что с точки зрения психологии в ее развитии есть какие-то аномалии --- скорее всего положительные, а может быть, и не только
\end{fancyquotes}

Папа говорит, что учител\'{я}, которые заставляют свободно читающего ребенка \explain{по слог\'{а}м}{syllable by syllable} \explainDetail{произносить}{произносить/произнести}{pronounce; (present tense) произнош\'{у}, произн\'{о}сишь, произн\'{о}сят} ма-ма мы-ла ра-му, --- преступники. У него все --- преступники, один он --- молодец. Совершенно понятно, что папа преследует какие-то цели. Не могу сказать, что они материальные, \explain{по-в\'{и}димому}{apparently}, он хочет \explainDetail{прославиться}{прославляться/прославиться}{become famous}, стать великим реформатором образования или кем-то еще \explain{в этом роде}{like that}. Но мне кажется, что эти эксперименты очень опасные.

В моей практике были дети, которые перескакивали через класс --- из шестого в восьмой, из восьмого в десятый. Таких случаев у меня было, если не ошибаюсь, три. Эти ситуации на состояние детей оказали скорее \explain{отрицательное влияние}{negative influence}, чем положительное. Ребята были развитые, скучали в классах по возрасту, но когда их перевели на год вперед, они совершенно потерялись. Мне кажется,  что так делать не надо.

У меня есть дети, которые очень одарены математически и учатся в математическом классе. Они становились призерами Всероссийских олимпиад, но это не повод считать, что во всем остальном дети так же одарены. Знаете, как говорил великий русский поэт Козьма Прутков: «Специалист подобен флюсу, полнота его односторонняя». \explainDetail{Допускаю}{допуск\'{а}ть/допуст\'{и}ть}{admit}, что талантливые дети могут оканчивать школу, \explain{допустим}{let's say}, не за 11 лет, а за девять. Но в то, что ребенок может окончить школу в 9 лет, --- не верю.

\subsection{Слишком умных учеников частенько боятся}
\textit{Леонид Перлов, почетный работник общего образования России, много лет преподавал географию в одной из лучших математических школ страны Лицей «Вторая школа»}

В обычных школах слишком умных учеников частенько просто боятся. Потому что учитель --- живой человек. Он понимает, когда у него не получается, и не понимает --- почему. А не получается просто потому, что он раньше мог не иметь дела с такими детьми. Или школьная администрация от него требует одно, а ребенку нужно совершенно другое. И как найти в этом приемлемую середину --- очень сложный вопрос.

С такими ребятами действительно трудно, \explain{ничуть}{not at all} не легче, чем с детьми с аутическим компонентом, с другими \explainDetail{особенностями}{особенность}{feature, singularity, characteristic, particulatiry} развития.

Просто здесь трудности другого рода. Учитель должен очень много знать не только в области своей математики, географии или литературы, а именно в области педагогики. Эти дети больше требуют, они иначе \explain{воспринимают}{perceive} действительность, спос\'{о}бны быстро анализировать действия того же самого учителя и показать ему, прав он или нет в той или ин\'{о}й ситуации. Им очень много надо от учителя, а учитель далеко не всегда в состоянии им это дать. Нужн\'{а} другая ман\'{е}ра общения с ребенком. И грань между \explainDetail{жесткостью}{жесткость}{harshness} и фамильярностью учителю помогает установить только опыт.

Педагогика --- не наука. Это синтез искусства и \explainDetail{ремесл\'{а}}{ремесл\'{о}}{craft, trade. Plural: ремёсла, (gen) ремёсел}. И в контакте с каждым конкретным учеником педагог работает так, как этому конкретному ученик\'{у} требуется. Естественно, если школа \explainDetail{предоставляет}{предоставлять/предоставить}{provide} педагогу такую \explain{возможность}{opportunity}, если он не \explain{вынужден}{compelled, forced} как большинств\'{о} учителей трудиться на полторы-две ставки. В моей «Второй школе» у учителей такая возможность есть.

Сейчас \explain{упор}{emphasis} делают на математической \explainDetail{одаренности}{одаренность}{talent, giftedness}, спортивной, музыкальной.
Да и \explain{собственно}{in fact} --- все. Других одаренностей стандарт не \explainDetail{предусматривает}{предусматривать/предусмотреть}{foresee}. А на самом деле этих одаренностей --- миллион. Ребенок вполне может быть талантлив в чем-то, чего пока еще не \explainDetail{проявил}{проявлять/проявить}{to show, to display, to evince, to manifest, to reveal; e.g., проявлять заб\'{о}ту (show concern); интерес (interest); себя (to prove oneself)}. И сам может о своей способности не догадываться.

Одна из задач квалифицированного педагога --- выявить эту одаренность. А вот что у ребенка здорово? Ну вот он \explain{дуб д\'{у}бом}{очень глупый} в математике и совершенно не интересуется химией. Но зато он пальцами чувствует, как из куска пластилина вылепить медведя. Его никто никогда этому не учил, но у него  прекрасно получается. Или, например, он педагогически одарён и обожает возиться с младшими своими товарищами. И у него отлично получается: они его слушают, они его обожают, они на нем виснут. Это одаренность? Думаю --- да.

Но школа сегодня не имеет задачи выявить талант у каждого. Главная задача школы --- выполнение стандарта. Все, наверное, слышали о федеральном государственном стандарте. \explainDetail{Подразумевается}{подразумеваться}{to be implied/meant}, что он --- \explain{некий}{a certain} \explain{эталон}{standard}, на который нужно равняться.
Для работы с детьми высоко мотивированными, грамотными, желающими учиться необходимо \explain{отклониться}{deviate} от этой нормы. Норма не рассчитана на повышенный уровень образования, в первую очередь она не может \explainDetail{удовлетворить}{удовлетворять/удовлетворить}{to satisfy, fulfill, gratify, suffice} требований со стороны ученика. Стандарты «отклонения» не приветствуют. Кроме того, отклонение в любую сторону --- хоть в сторону повышенных \explainDetail{потребностей}{потребность}{need (n.)} со стороны ученика, хоть в сторону работы с детьми с особенностями развития --- все это требует особой, \explain{соответствующей}{corresponding} квалификации учителей. Действующий профессиональный стандарт учителя подразумевает, что педагог обязан работать с любыми детьми в любых условиях. Хотя его никто и никогда не учил этому.

\begin{fancyquotes}
    Для родителей часто ребенок, \explain{скачущий}{galloping, prancing} \explain{со ступеньки на ступеньку}{step by step} в школе, побеждающий в олимпиадах, --- предмет гордости, повод свысока поглядывать на коллегу по работе или на соседей. Но ребенку эти успехи не всегда приносят радость. Рано или поздно родители начинают ему говорить: «Вот ты \explainDetail{з\'{а}нял}{занимать/занять}{occupy, take up, secure; note on stress: з\'{а}нял, занял\'{а}, з\'{а}няло, з\'{а}няли} второе место, а почему не первое? А ну-ка, поработай еще!»
\end{fancyquotes}

Дети, которые перепрыгивали через классы, были и 20 лет назад, и сто лет назад. Но ничего хорошего, как правило, из этого не выходит. Всему свое время, в том числе и детству. Думаю, что и на этот раз исключением эта девочка не станет. Конечно, для таких детей нужен особый подход. Ей нужны знания, соответствующие ее развитию и способностям. Но это вовсе не курсы ЕГЭ по русскому и математике. Подготовительные курсы к ЕГЭ --- это называется \explain{дрессировка}{training}. Медведь вон ездит в цирке на велосипеде. Но, во-первых, он не знает, что это неприятно. А во-вторых, совершенно не понимает, что для него --- медведя --- это нехорошо.

Все же взрослым нужно \explain{поаккуратнее}{more carefully} \explain{подходить к этим вопросам}{approach these questions} и в первую \'{о}чередь выяснить --- это им так кажется, или сам ребенок ощущает, что у него к чему-то талант и он готов в этом направлении развиваться. Очень часто ощущения родителей и детей не совпадают. Например, родители считают, что ребенок математически одаренный, а он мечтает играть на кларнете и в любую свободную минуту летит к инструменту, потому что это ему по-настоящему нравится. При этом он занимается математикой, олимпиадник и так далее, но только потому, что он \explain{послушный}{obedient} ребенок. У меня такие случаи были. Ребенок --- член команды Москвы по шахматам со всеми разрядами, подающий очень большие надежды. В девятом классе мальчик сказал родителям, что на \explain{юниорский}{junior} чемпионат он не поедет и эту страницу своей биографии закрыл. Он намерен поступать на мехмат МГУ, а значит, \explain{оставшиеся}{remaining} до окончания школы два года будет заниматься именно этим. Скандал был нереальный. Но парень, \explain{надо отдать ему должное}{we must give him his due}, \explain{выдержал}{survived}.


\subsection{Ведущая деятельность ребенка --- игра, отец заменяет ее учебой}
\textit{Александр Снегуров, \explain{заслуженный}{distinguished} учитель России, кандидат психологических наук}

Корреспонденты обратились ко мне за комментарием феномена, я высказал свое мнение. Это были корреспонденты телеканала «Россия 24». А потом мне сообщили, что девочку \explainDetail{опрос\'{и}ли}{опрашивать/опросить}{to question, to interview} --- есть ролик с выступлениями ее отца, который я не смог посмотреть. Так вот, ребенку задали ряд вопросов, и выяснилось, что она не знает каких-то тривиальных вещей, после чего выпуск сюжета \explainDetail{отмен\'{и}ли}{отменять/отменить}{cancel, revoke, abolish}.

Да, неудобно говорить о ее достижениях, когда она не знает обычных вещей. А я это допустил еще до ее опроса. Потому что тут \explain{налиц\'{о}}{obvious; present; on hand} диссонанс и \explain{нарушение}{violation}, так скажем, динамики по всем направлениям развития. Обязательно что-то будет отставать. В случае этих вундеркиндов это практически все время имеет место. Знаете ведь, как \explain{тесто}{dough} \explainDetail{замешивают}{замешивать}{knead}? А потом уже от замешанного теста можно взять кусочек и потянуть вверх. Мы можем вытянуть достаточно высок\'{о}. Так же и с детьми. Представим, что мы вытянули один \explain{показатель}{indicator} из общей \explainDetail{палитры}{пал\'{и}тра}{palette} развития ребенка. В данном случае --- сегмент ее интеллектуальной \explain{составляющей}{component}. А социализация и психологическое развитие, оставшиеся в этом \explainDetail{т\'{а}зике}{т\'{а}зик}{basin}, соответствуют возрасту. А мы-то хотим ориентироваться на высоту вытянутого сегмента, чтобы остальное ему соответствовало.

А потом удивляются, почему у таких детей стрессы и \explain{изломанные}{broken} судьбы, а в лучшем случае возвращение к типичному пейзажу в привычный \explain{ландшафт}{landscape} \explainDetail{сверстников}{сверстник}{peer}. Подобное ускоренное обучение \explainDetail{чревато}{чрев\'{а}тый}{fraught with (+\textit{твор.})} этими факторами. Я не буду говорить, что это дурно, просто обозначаю. Ты даёшь обучение, но надо понимать, что миновать вид детства не рекомендуется. Нужно понимать, что \explain{ведущая}{leading} деятельность ребенка --- игра, а в данном случае ее отец \explainDetail{заменяет}{заменять/заменить}{replace} ее учебой, но психика ребенка настроена на эту смену не сейчас.

Да он говорил, что она гуляет и играет, но это все сочетается, может быть, только на его взгляд. А может быть, это всего лишь имитация сочетания. А потом \explain{выявится}{comes to light} большой диссонанс, который обнаружится \explain{внез\'{а}пно}{suddenly}. Допуст\'{и}м, ребенок вдруг чего-то не захочет, например, жить на свете или находиться среди этих людей. Вдруг он может заявить своим родителям --- вы меня \explainDetail{измучили}{измучивать/измучить}{to weary, to exhaust, to tire out} и \explainDetail{достали}{достать}{(colloq.) to annoy sb badly}. Этого не стоит исключать. Я не говорю, что это может \explainDetail{произойти}{происходить/произойти}{happen} в обязательной перспективе, но исключать этого нельзя, и стоит иметь это постоянно в виду.

Я \explain{склонен}{inclined} к позиции, что ее \explain{натаскали}{coached} на сдачу ЕГЭ, при этом \explain{не отрицая}{not denying (отрицать)} все-таки ее интеллектуальных возможностей. Однако в таком случае игнорируется --- хотя не должен --- опыт взросления. Ряд школьных предметов основан именно на \explainDetail{постепенном}{постепенный}{gradual} взрослении, \explain{к примеру}{e.g.,}, литература и история. Ну что она, «Войну и мир» прочитала?

Она поступила на психфак, не \explainDetail{созрев}{созревать/созреть}{to mature [short past adverbial perfective participle of созреть]} и не испытав этапов взросления, которые нужно пройти человеку. Хоть и говорят, что мы, психологи, работаем с опытом другого человека, однако если у тебя есть собственный опыт, это точно не \explain{помешает}{will interfere}. Я не говорю, что каждый должен пройти через большую драму или катастрофу и тогда он сможет работать психологом. Но, \explain{разум\'{е}ется}{of course}, какие-то приблизительные ощущения и \explainDetail{переживания}{переживание}{experience} должны быть, чтобы работа была успешной.

Рост психологический и физиологический обязательно должен отразиться на психике и \explainDetail{сознании}{сознание}{consciousness} ребенка, чтобы была полноценная картина, а этого, по-моему, не произошло, ведь отец мог интеллектуально ее \explain{подгонять}{to fit, to adjust, to accommodate}. Но само созревание\dots{}  Да, может, у нее идут и эти процессы быстрее, но это еще не значит, что они соответствуют ее интеллектуальным \explainDetail{свершениям}{свершение}{accomplishment}. А там, может быть, и свершения интеллектуальные не по всем пунктам. Интересно было бы побеседовать с ребенком не в рамках ЕГЭ, а в рамках общей эрудиции и взглядов на мир. \explain{Сложилось}{to (be) form(ed), to develop, to turn out, to take shape. Example: Оп\'{а}сная ситу\'{а}ция слож\'{и}лась: a dangerous situation took shape} ли у нее \explain{мировоззрение}{world view}, а не \explain{набор}{set} фактов, который опирается на память. Но \explain{складывание}{folding} и формирование мировоззрения --- другая вещь. И жизненный опыт в том числе.

Они и в вузе хотят ускорить обучение, чтобы она окончила его в 11 лет. Задаюсь вопросом: работать она не может, а значит может быть провисание, которое, не исключено, может привести к экзистенциальному кризису: а зачем вы это сделали со мной? А что мне теперь делать, а где мое детство? Это один вариант. Другой --- она сама вернется в свою детскую парадигму, ну и как-то все в общем пейзаже сравняется.

Сам я \explain{сторонник}{supporter} отмены ЕГЭ, но это отдельная беседа. Этот экзамен проверяет память и приспособленность ребенка к определенным процедурам, а это отнюдь не весь спектр потенциала ребенка. У кого-то память не так хороша, у кого-то лучше. Много у меня опций критического характера по этому вопросу. Однако я за различные сценарии итоговой аттестации.

\begin{fancyquotes}
    Возможен и драматичный вариант. Редкий случай, что отец постоянно будет подкидывать ей поленья новых задач, чтобы она не \explain{простаивала}{to idle}. Но \explain{вряд}{hardly} ли это получится в той динамике, которую он задал. Однако до какого-то возраста он может это делать, потом же этот механизм у ребенка-вундеркинда даст \explain{сбой}{glitch, failure}
\end{fancyquotes}

\explainDetail{По поводу}{по п\'{о}воду чег\'{о}-либо}{concerning something} детей-вундеркиндов я считаю, что родители развивают уже имеющуюся \explainDetail{почву}{почва}{soil}. Можно сказать, что это \explain{дар}{gift} или наказание, но в любом случае это данность, которую родители заметили и н\'{а}чали развивать \explainDetail{доступными}{доступный}{accessible, available} способами. Это нетипичный пейзаж, который можно называть \explain{своеобразным}{своеобразный}{peculiar} нарушением или отклонением от нормы. \explainDetail{Обижать}{обижать}{to offend, to hurt sb's feelings} не станем, но примеры многих талантливых людей, \explain{ув\'{ы}}{alas}, подтверждают эту позицию.




\section[Воздух несвободы]{Воздух несвободы. Что заставляет российских ученых уезжать за границу?}
\textit{Ольга Просвирова}\\
\textit{Русская служба Би-би-си}\\
\url{https://www.bbc.com/russian/features-57028917}

\textit{Россия --- единственная из развитых стран, где несколько десятилетий \explain{подр\'{я}д}{in a row} уменьшается числ\'{о} учёных, заявил главный учёный секретарь Российской академии наук Николай Долгушкин. И если с первым утверждением легко не согласиться --- \explain{МВФ}{Международный валютный фонд (IMF)} отн\'{о}сит Россию к развивающимся странам, --- с\'{а}ми учёные не видят ничего удивительного в том, что высококвалифицированные специалисты уезжают из страны.}

Долгушкин рассказал, что одна из главных причин \explainDetail{сокращения}{сокращение}{reduction} численности ученых --- отъезд за рубеж: если в 2012 году из России уехали 12 тысяч высококвалифицированных специалистов, то сейчас, по словам главного ученого секретаря РАН, уезжают 70 тысяч в год.

Статистика РАН не очень \explain{соотносится}{correlates} с цифрами Росстата. Но этот факт скорее вызывает вопросы к ведению статистики в принципе.

Росстат считает, что в 2019 году - это самые \explain{свежие}{fresh (recent)} данные --- из России уехали 384 тысячи человек (с учетом детей и подростков до 14 лет --- 416 тыс.). Из них только у 62 тысяч было высшее образование. А докторов и кандидатов наук было всего 360.

Предварительно Росстат посчитал миграцию и за период с января по ноябрь 2020 года: в эти месяцы пандемии из страны уехали почти 440 тысяч человек. Большая часть уехавших просто вернулась к себе на родину в страны \explain{СНГ}{Содружество Независимых Государств: The Commonwealth of Independent States is a regional intergovernmental organization in Eastern Europe and Asia. It was formed following the dissolution of the Soviet Union in 1991. Member states: Armenia,Azerbaijan, Belarus, Kazakhstan, Kyrgyzstan, Moldova, Russia, Tajikistan, and Uzbekistan.}.


В этой статистике важно то, что более 60 тыс. человек уехали в страны дальнего \explainDetail{заруб\'{е}жья}{заруб\'{е}жье}{abroad} --- это рекордный показатель за последние девять лет. Годом ранее эта цифра была в районе 45 тысяч.

Росстат не указывает в предварительной статистике информацию об образовании уехавших, но даже на основании имеющихся данных трудно \explainDetail{предположить}{предполаг\'{а}ть/предполож\'{и}ть}{to assume, to suppose; to presume, to imply}, какой информацией руководствовался РАН, когда говорил о 70 тысячах уезжающих высококвалифицированных специалистов.

Вероятнее всего, в РАН считают не только кандидатов и докторов наук, но также и уезжающих \explainDetail{аспирантов}{аспирант}{graduate student} и \explain{выпускников}{выпускник}{graduate} университетов, планирующих заниматься наукой. Но даже в таком случае сложно объяснить способ их подсчёта.

Цифры Росстата также могут быть неточными, отмечают эксперты. Основной \explain{ист\'{о}чник информации}{source of information} Росстата --- данные о регистрации человека по месту \explainDetail{жительства}{жительство}{residence}. Далеко не все уезжающие россияне "выписываются" из квартир, а значит, не считаются эмигрантами.

Кроме того, издание «Проект» подсчитало, что российская статистика эмиграции в среднем расходится с зарубежной \explain{в шесть раз}{six times}. \explainDetail{Опираясь}{опираться/опереться}{rely} на данные 2017 года, издание заметило, что в 2017 году США \explain{насчитали}{counted} у себя в шесть раз больше прибывших россиян, чем в том же году зафиксировал Росстат.

\subsection{Абсолютно свободные люди}

Как бы то ни было, в Кремле с данными РАН спорить не стали. Пресс-секретарь президента Дмитрий Песков сказал, что не видит ничего трагичного в отъезде российских учёных: «Какие-то ученые уезжают, какие-то возвращаются».

Учёные, по словам Пескова, абсолютно свободные люди и работают в тех местах, где реализуются наиболее интересные проекты и создаются наиболее комфортные условия. И задача стран --- создавать такие условия.

\begin{wrapfigure}{l}{0.6\textwidth}
    \begin{center}
        \includegraphics[width=0.58\textwidth]{img/wall.png}
    \end{center}
    \caption{На деньги гранта лаборатории могут позволить себе неплохое обор\'{у}дование [equipment], но от протечек [протечка: leak] во время дождей оно не защищено.}
\end{wrapfigure}
Каковы бы ни были реальные цифры эмиграции учёных, некоторые из них в разговорах с Би-би-си \explain{признаются}{are recognised}, что с задачей создать комфортные условия власти в России не справляются.

Но какие условия нужны ученым для успешной работы?

Отвечая на этот вопрос, многие собеседники Би-би-си, как уехавшие из страны, так и продолжающие работать в России, сначала говорили о зарплатах, финансировании исследований, хорошем оборудовании, \explainDetail{отс\'{у}тствии}{отс\'{у}тствие}{absence} бюрократии, научной свободе. Но в итоге \explainDetail{сходились}{сходиться/сойтись}{converge [сходиться: схож\'{у}сь, сх\'{о}дишься, сх\'{о}дятся; сойтись: сойд\'{у}сь, сойдёшься, сойд\'{у}тся]} на том, что все эти факторы должны сойтись \explain{воедино}{together}: исправив лишь что-то одно, невозможно создать нормальные условия для работы ученых.

В начале 90-х, когда развитие науки \explain{мало кого}{a few (people) $\neq$ много кого: many people; also: мало где: a few places $\neq$ много где} интересовало, учёным часто не хватало на жизнь. Кто мог, уезжал за границу, если удавалось найти работу. В тот момент американский финансист Джордж Сорос (несколько его фондов впоследствии были признаны в России «нежелательными организациями») выделил на развитие российской науки более 100 млн долларов.

В 1993 году через Международный научный фонд, который \explain{возглавлял}{was headed} открывший структуру \explain{ДНК}{дезоксирибонуклеиновая кислота: DNA} нобелевский лауреат Джеймс Уотсон, 25 тысяч российских учёных получили единовременную выплату в 500 долларов --- большие по тем временам деньги, когда месячная зарплата учёного могла \explain{составлять}{be, constitute} 50 долларов.

На деньги Сороса лаборатории закупали реактивы и оборудование, отправляли учёных на международные научные конференции.

Были и другие иностранные фонды, готовые «подкармливать» российских ученых, существовали программы, которые позволяли получать иностранные деньги, работая в России.

Все это закончилось в середине \explain{нулевых}{2000's}.

\subsection{Небольшие деньги}
С тех времен начал\'{и}сь б\'{о}лее заметные \explainDetail{влож\'{е}ния}{влож\'{е}ние}{investment} самой России в науку: российский Фонд фундаментальных исследований
\explainDetail{давал}{давать/дать}{%
to give; reminder: [дав\'{а}ть: да\'{ю}, даёшь, да\'{ю}т; дав\'{а}л, дав\'{а}ла, дав\'{а}ло, дав\'{а}ли] [дать: дам, дашь, даст, дад\'{и}м, дад\'{и}те, дад\'{у}т; дал, дала, дало, д\'{а}ли]
}
небольшие гранты, которые, по воспоминаниям их получателей, «позволяли выживать». Тогда же появился проект 5-100 --- государственная инициатива, которая должна была приблизить российские университеты к мировым стандартам.



\begin{fancyquotes}
    Эти инвестиции не были трансформативными, --- говорит Сергей Ерофеев, профессор американского университета Ратгерс. --- Их было недостаточно, чтобы привести к каким-то реальным переменам. Некоторые декоративные перемены все же были возможны. \explain{Кое-где}{somewhere} --- например, в Высшей школе экономики и некоторых других университетах --- была концентрация хороших ресурсов. Но в целом это получилась совершенно косметическая программа.
\end{fancyquotes}


\begin{wrapfigure}{l}{0.6\textwidth}
    \begin{center}
        \includegraphics[width=0.58\textwidth]{img/bigmoney.png}
    \end{center}
    \caption{Благодаря грантам российские ученые могут получать конкурентные зарплаты, но других проблем государственные инициативы не решают.}
\end{wrapfigure}
Тогда же появились различные федеральные \explain{целевые}{target} программы, министерские \explain{конкурсы}{contests}. И к 2010 году, во время президентства Дмитрий Медведева, правительство учредило конкурс научных мегагрантов, рассчитывая привлечь \explain{ведущих}{leading} ученых в российские вузы.

Конкурс мегагрантов в основном создавался для поддержки \explainDetail{дорогостоящих}{дорогостоящий}{costly} исследований в стратегических направлениях науки, которые определялись \explainDetail{чиновниками}{чиновник}{official, officer}.

Так в стране \explain{укрепилась}{was strengthened} грантовая система финансирования научных исследований. \explainDetail{В отличие}{в отличие от}{unlike}, например, от американской грантовой системы, в которой деньги идут не на зарплаты ученым, а на проведение исследования, российские специалисты могли значительно улучшить собственное финансовое положение.

Попросивший не называть своего имени российский ученый, работающий в крупной московской лаборатории, рассказал: «На Западе сложно получить \explainDetail{постоянную позицию}{постоянная позиция}{permanent position} в университете, но если получил --- зарплата \explain{приличная}{decent}. У нас гораздо легче получить позицию, в образовательных учреждениях есть маленькая базовая зарплата, но это очень небольшие деньги, и жить на них \explain{затруднительно}{cumbersome, hard, straitened or awkward}, особенно в Москве».

\explainDetail{Вопрос}{вопрос}{issue, matter} зарплат в России решается \explain{исходя из}{based on} возможностей конкретного университета или \explain{НИИ}{научно-исследовательский институт}. Например, один из сотрудников подмосковного НИИ рассказал, что в последние годы начал получать \explain{надбавки}{allowances, perks} за публикацию научных статей в хороших журналах.

«С грантов мы можем получать нормальную зарплату, иногда даже больше, чем в Европе, но по большинству из них мы даже компьютер не можем купить, потому что деньги требуют тратить только на \explain{непосредственные}{immediate} нужды исследования по проекту», --- говорит ученый.

В России старший научный сотрудник получает в среднем 26 тысяч рублей (примерно 350 долларов), а профессор --- 36 тысяч в месяц.

По указу Владимира Путина, зарплата научных работников должна быть доведена до 200\% от средней по региону. Однако никакого дополнительного финансирования на достижение этой цели выделено не было. Чтобы достичь определенных президентом показателей, университеты начали переводить научных работников на половину \explainDetail{ставки}{ставка}{rate}.

Об этом впервые публично рассказала кандидат биологических наук из Института цитологии и генетики РАН Анастасия Проскурина. Во время встречи с Путиным она заявила, что вместо \explain{положенных}{due} 79 тысяч получает 25, а руководство института вместо повышения зарплат предложило ей перейти на \explain{полставки}{part-time}.


\subsection{Мыши в кармане}

Инвестиции в экспериментальное оборудование часто происходят \explain{без учёта того}{without taking into account}, где и как это оборудование будет установлено, потому что деньги выделяются на конкретные проекты.

\begin{wrapfigure}{l}{0.45\textwidth}
    \begin{center}
        \includegraphics[width=0.44\textwidth]{img/mouse.png}
    \end{center}
    \caption{Процесс закупок трансгенных мышей из-за границы не был очевиден ни сотрудникам лаборатории, ни руководству университета.}
\end{wrapfigure}
«Мы покупаем \explain{сравнительно}{relatively} дорогостоящее оборудование, но во время ливней или \explain{таяния}{melting} снега на него начинает \explain{течь}{flow} вода, потому что университет годами не может нормально залатать крышу. \explainDetail{Оснащение}{оснащение}{equipment} образовательных лабораторий, где студенты проходят практикум, \explain{несравнимо}{incomparable} с западными», --- поделился на условиях анонимности сотрудник крупной московской лаборатории.

Одни лаборатории, созданные на деньги мегагрантов, успешно существуют до сих пор, а некоторые, даже после полного оснащения, закрываются без дальнейшей финансовой поддержки.

В 2011 году нейробиолог Виктория Коржова закончила Санкт-Петербургский государственный университет и работала в лаборатории, которая тоже создавалась на деньги мегагранта.


"С одной сторон\'{ы}, очень хорошее финансирование, а с другой --- денег недостаточно. Все время, которое грант действовал, ушло на организацию самой лаборатории, ремонт помещений и закупку оборудования", --- рассказывает Коржова.

Но для исследовательской работы не всегда достаточно просто оборудовать рабочие места.

В лаборатории, где работала Коржова, например, должны были проводиться эксперименты на трансгенных мышах. «Большинств\'{о} трансгенных мышей производит одна международная компания, которая \explain{поставляет}{supplies} их в лаборатории по всему миру, --- рассказывает она. --- Наш \explain{заведующий}{manager} лабораторией, у которого были деньги на закупку, не знал, как именно привезти мышей в Россию, каков легальный процесс. Университет тоже не мог ему помочь. В какой-то момент мы уже, не совсем \explain{шутя}{jokingly}, обсуждали, сможем ли мы привезти их в кармане, что совершенно безумная идея».

Проблема решалась почти два года. Все это время ученые были \explain{вынуждены}{forced} придумывать компромиссные варианты, чтобы, как они говорят, получилось что-то похожее на \explainDetail{первоначальную}{первоначальная}{initial, original} \explainDetail{задумку}{зад\'{у}мка}{idea, plan}.

Одна из важных проблем в России, которая давно решена в странах, где активно развивается наука, --- это отсутствие \explainDetail{таможенных}{таможенный}{related to customs ($<$ там\'{о}жня: customs)} \explainDetail{льгот}{льг\'{о}та}{privilege, benefit, exemption, discount} для научных закупок, считает нейробиолог.

В других лабораториях страны ситуация иная.

«Нет такой страны, которая бы производила все нужные реактивы для ученых. Нормально, что некоторые реактивы покупаются в США или Европе. При этом везде есть способ быстро получить \explain{доставку}{доставка}{delivery}, --- рассказывает Коржова. --- Работая в Германии, я могла \explain{заказать}{to order} антитела и получить их максимум в течение недели, а обычно --- за два-три дня».

В России же, по ее словам, люди ждут месяцами.

«Антитела --- очень \explain{чувствительные}{sensitive} вещества, которые нужно обязательно хранить в холоде. А если процесс на таможне \explain{затягивается}{drags on}, то неизвестно, как они хранятся. Из-за этого еще и эксперименты нужно планировать на год вперед, что не очень реалистично. Приходится идти на компромиссы: я делаю не так, как мне хотелось бы сделать в идеале, а существую в рамках больших ограничений», --- говорит нейробиолог.

\subsection{Возвращаются единицы}

Российский кристаллограф, член Европейской академии наук Артем Оганов уехал из России в 1997 году.

Он так вспоминает об этом: "Когда я уезжал, это был 1998 год, зарплаты были, ну я не знаю, пятьдесят долларов в месяц. Это нереально было для существования. Доктора наук продавали стиральный порошок на улице. Я решил уехать. Сейчас времена изменились".


В конце 2014 года Оганов, работавший в США, отказался от гринкарты и вернулся в Россию. "Если при прочих равных выбор: жить у себя дома или не у себя дома (вот я могу здесь заниматься передовой наукой и здесь, я могу и здесь иметь нормальный уровень жизни, и здесь), так вот выбирать нужно всегда свой дом", - объяснял он.

\begin{wrapfigure}{l}{0.5\textwidth}
    \begin{center}
        \includegraphics[width=0.48\textwidth]{img/microscope.png}
    \end{center}
    \caption{В России ученым не хватает свободы, считают некоторые уехавшие за границу.}
\end{wrapfigure}
"Таких, кто полностью вернулся в Россию, прервав основные контракты за рубежом, единицы. Подавляющее большинств\'{о} наших соотечественников, которые успешно работают за рубежом, свои позиции в научно-образовательных организациях развитых стран не оставляют, - говорит Сергей Ерофеев, работающий в американском университете Ратгерс. - Они прекрасно понимают, что это их основная жизнь. Если есть возможность успешно работать с коллегами в России, получать какие-то новые возможности для исследований - они иногда ухватываются за эту возможность".

Многие действительно стараются сохранять и использовать профессиональные связи в России.


Директор Центра нелинейной физики в Австралийском университете Юрий Кившарь был одним из первых получателей мегагранта. "Мы создали лабораторию - там было человек пять, потом она выросла в центр. Начали набирать новых людей, стало человек сорок. А потом центр стал кафедрой, а кафедра - факультетом. Так появился Новый физтех - физико-технический факультет университета ИТМО", -рассказывает Кившарь.

Сам ученый уехал в Австралию еще из СССР, но связи с российскими коллегами сохранил.

"Когда мы уезжали, это было другое время, - уверен он. - Многие толковые ребята уезжали, лишь бы уехать. Это был отъезд если не насовсем, то надолго. Сейчас у меня есть студенты из Москвы, и в основном они говорят, что уехали на время. Они не собираются оставаться. Сейчас вообще появилось больше возможностей поехать куда-то. В том же ИТМО молодежь посылают за границу на стажировку на год-два, но многие возвращаются", - говорит Кившарь.

\subsection{Национальная идентичность и воздух свободы}

Аспирант астрофизического факультета Принстонского университета США Айк Акопян, до этого работавший на кафедре физики и астрофизики МФТИ, вспоминает, как российские друзья прислали ему ссылку на статью с заголовком: "Американские ученые [далее шли типичные российские имена и фамилии] открыли…"

"Все трое, естественно, работали в США. Вообще сравнивать объем американской и российской науки невозможно, - рассказывает Акопян. - Здесь к науке относятся как к части культуры, как к национальной идентичности. Из-за этого есть частное финансирование науки - то, чего практически нигде в мире нет".

\begin{wrapfigure}{l}{0.5\textwidth}
    \begin{center}
        \includegraphics[width=0.48\textwidth]{img/supercomputer.png}
    \end{center}
    \caption{Российские ученые, уехавшие работать в США, считают, что возможностей для развития карьеры здесь больше, а научным сотрудникам доступно более современное оборудование. На фото - суперкомпьютер, на пользование которым можно получить отдельный грант.}
\end{wrapfigure}
Акопян согласен, что в последние годы ситуация с финансированием науки в России начала улучшаться: "Я учился в России с 2010 года и уехал в 2016-м. За это время ситуация поменялась в лучшую сторону. Многие ученые начали получать огромное количество денег. Но финансы - не основная проблема".

Развития карьеры вне США, где в сфере астрофизики гранты на исследование выдает в том числе и NASA, Акопян не видит. Одним из основных достоинств он считает постоянное взаимодействие ученых.

"Когда открыли гравитационные волны, на следующий же день у нас была информация из первых рук: приехали ученые, которые рассказывали обо всех деталях открытия. Нам не приходится ждать всяких пресс-релизов, все становится открытым, ведется обсуждение. Когда миссия "Кассини" давала результаты - мы опять же всё узнавали от тех, кто занимался проектом".

Другой научный сотрудник университета США, переехавший за границу после окончания МФТИ, соглашается, что для нормальной работы ученых необходимо сочетание независимости и включенности в деятельность научного сообщества:


"Первое выражается в стабильной зарплате и возможности выбирать направления работы в широких рамках. Второе - в доступности участия в конференциях, семинарах, командировках. До сих пор один из важнейших критериев оценки ученых в Европе и США - это их репутация в международном научном сообществе. Обычно это дает гораздо более точное представление, чем формальные показатели".

"Самое главное, чего не хватает отечественным ученым, - воздуха. Того, что называется научной свободой. Страх сейчас проникает во все поры научного тела", - считает профессор Сергей Ерофеев.

С одной стороны, говорит он, руководство считает, что надо добиваться конкурентоспособности российской науки. Но есть и другая рука, которая главной своей задачей считает контроль.

"А контроль над обществом не предполагает свободы - даже научного мнения", - убежден ученый.

В 2018 году наука в России была официально объявлена национальным проектом. Нацпроект был разработан по следам майских указов Владимира Путина. Было обещано, что к 2024 году Россия должна войти в пятерку ведущих стран, осуществляющих научные исследования и разработки в областях, определяемых приоритетами научно-технологического развития.

Чиновники \explain{по-прежнему}{still} обещают создать \explain{привлекательные}{attractive} условия для работы ведущих российских и зарубежных ученых, а также увеличить финансирование научных проектов и разработок.


\section{Человечество решает умереть}
\textit{Ярослав Забалуев}\\
\url{https://lenta.ru/articles/2021/12/25/dontlookup/}

{\it Вышла комедия про дураков и апокалипсис с Ди Каприо, Стрип и Бланшетт. Зачем ее смотреть?}

\textit{На Netflix вышла новая комедия известного «Игрой на понижение» Адама Маккея «Не смотрите наверх» — хвастающая, возможно, самым звездным за последнее время актерским составом. «Лента.ру» рассказывает, почему фильм о конце света с Леонардо Ди Каприо в главной роли — идеальное кино для конца этого странного года.}

Астрономы Рэндалл Минди (Леонардо Ди Каприо) и Кейт Дибиаски (Дженнифер Лоуренс) обнаруживают, что к Земле мчится комета диаметром в десяток километров. Столкновение с планетой может привести к полному исчезновению не только человеческой, но и вообще всякой жизни. Рэндалл и Кейт грузятся в самолет и отправляются в Вашингтон, чтобы обсудить планы спасения Земли с высочайшими государственными чинами. Однако выясняется, что президент Дженни Орлин (Мэрил Стрип) куда больше увлечена живописным курением и секстингом с каким-то региональным отморозком. Главой администрации работает ее сын Джейсон (Джона Хилл), который в свою очередь в основном хвастается новой татуировкой дракона и упивается властью ходить на работу упоротым.

Минди и Дибиаски пытаются добиться огласки, выступив в популярном телешоу, однако Кейт заслуживает лишь волну мемов в интернете, а Рэндалл — недвусмысленные знаки внимания ведущей (Кейт Бланшетт), которую возбуждает, что скоро мы все умрем. В какой-то момент правительство США все же снарядит спасательную экспедицию, но сразу после старта развернет ракеты, поскольку на сцену выйдет Питер Ишервелл (Марк Рейланс) — визионер, объясняющий, что из столкновения с кометой тоже в теории можно извлечь пользу.

Фильмы и сериалы-катастрофы прошлого и нынешнего годов дали повод вновь заговорить о сверхъестественном чутье художников, прозревающих будущее без всяких на то логических объяснений. Вот и разработка «Не смотрите наверх» началась еще во вполне безмятежном ноябре 2019-го. Впрочем, никакой безмятежностью, разумеется, и не пахло — Дональд Трамп вовсю собирался на второй срок, а Америка все глубже погружалась в депрессию, лекарство от которой так и не придумали до сих пор. Тем не менее за без малого полтора года, которые занял путь картины к зрителю, в мире изменилось слишком многое, создав для «Не смотрите наверх» уникальный и куда более подходящий случаю контекст.

В сверхъестественной проницательности в данном случае стоит обвинять сценариста и режиссера Адама Маккея. Это автор удивительной судьбы. Шесть лет назад он не пожелал сидеть в комедийном жанровом гетто и после дилогии про Рона Бургугди («Телеведущий») бросился покорять новые территории. В итоге его «Игра на понижение» стала одним из самых остроумных фильмов про кризис 2008-го года и принесла Маккею «Оскар» за лучший адаптированный сценарий. Через три года, в 2018-м, Адам решил развить успех и одновременно взвинтить ставки — его «Власть» имела в своем центре ни много ни мало бывшего вице-президента Дика Чейни. Сатирический байопик абсолютного зла (именно такова трактовка Маккея) был воспринят чуть менее однозначно, несмотря на очередной актерский подвиг Кристиана Бейла. И вот в следующем своем проекте режиссер совместил едкий социальный комментарий с внешне легкомысленным задором своих ранних комедий.

\begin{fancyquotes}
    Кажется, что фильмы про летящие к Земле кометы вышли из моды на рубеже тысячелетий — нулевые показали, что над нами летают штуки и пострашнее
\end{fancyquotes}

Последней из больших голливудских картин на тему, конечно, был пропагандистский шедевр Майкла Бэя «Армагеддон». Это было кино, где все важные вещи говорили на фоне развевающегося звездно-полосатого флага, а Брюс Уиллис, наконец, смог погибнуть — но только под песню Aerosmith. Самым ярким послесловием к этому сюжету стала «Меланхолия» Ларса фон Триера, который с явным удовольствием разнес-таки нашу планету в труху. «Не смотрите наверх» отсылает к этим двум фильмам вполне прямолинейно. «Армагеддон» спародирован целыми фрагментами, а Уиллиса заменил Рон Перлман — и так даже смешнее. С «Меланхолией» у Маккея значительно более нежные отношения. Пародиями тут не пахнет, скорее уж речь идет о трепетном оммаже — при желании героинь Лоуренс и Бланшетт можно без особых поправок поместить в триеровский контекст.

Однако прямые и не очень аллюзии в данном случае отнюдь не самоцель. Маккей использует энергию предшественников, чтобы напитать ей совершенно авторское высказывание, сделанное, как водится, в сатирическом ключе. Режиссер владеет этим сложнейшим на самом деле жанром виртуозно и в «Не смотрите наверх» явно упивается возможностью не заботиться об исторической достоверности. На орехи тут достается абсолютно всем: озверевшим от самодовольства селебрити политикам, мямлям-ученым, не способным разговаривать человеческим языком, дурящим народ визионерам со своими дурацкими смартфонами… Перечислять мишени Маккея можно долго и с удовольствием, благо режиссер не придерживается более или менее никакой конкретной позиции — просто стреляет во все, что видит.

За без малого два с половиной часа от такого потока желчи и презрения можно было бы устать, но фокус в том, что «Не смотрите наверх» этих потоков на зрителя, в общем, не льет. Остроумие наблюдений и сарказм авторских комментариев здесь уравновешен удивительно человечной интонацией. У Маккея с его врожденной язвительностью и острым глазом нет ответов на вопрос «что делать?» За происходящим на экране балаганом сквозит растерянность умного человека, вынужденного, как обычно, пытаться хоть как-то достучаться до идиотов. Это, пожалуй, и правда самая адекватная эмоция в мире, где ученый, сообщающий в ток-шоу о гибели человечества, добивается лишь статуса «астронома, которого я бы трахнула». Зато об этом мире можно снять фильм, который под конец очередного безумного года дарит не столько депрессию, сколько радость и умиротворение. Если мы все равно скоро умрем, то нет ни одной причины отказываться от праздничного ужина.

Фильм «Не смотрите наверх» (Don't Look Up) вышел на Netflix 24 декабря



\section{Война на Украйне}
\subsection{Запад не поверил предупреждениям Владимира Путина}
% https://russian.rt.com/inotv/2022-02-28/AC-Zapad-ne-poveril-preduprezhdeniyam

Многие считают Владимира Путина нерациональным, однако российский президент сделал именно то, о чём \explainDetail{предупреждал}{предупреждать/предупредить}{warn, let know beforehand}, и специальная военная операция на Украине является ответом на \explain{отказ}{refusal} от расширения НАТО, считает американский политик и публицист, главный редактор журнала The American Conservative Патрик Бьюкенен. По его мнению, Путин не хочет войны с Западом, однако для этого Байдену нужно \explain{пересмотреть}{rethink, revise} своё видение Североатлантического альянса.

Когда Владимир Путин потребовал, чтобы США исключили Украину из числа будущих членов альянса НАТО, США ответили, что НАТО проводит политику «открытых дверей» и любая страна, включая Украину, может \explain{подать заявку}{apply, submit application} на членство и быть принятой в альянс. Как считает американский политик и публицист, главный редактор журнала The American Conservative Патрик Бьюкенен, не \explainDetail{сум\'{е}в}{ум\'{е}ть/сум\'{е}ть}{to be able, to manage, to succeed} получить \explain{удовлетворительный}{satisfactory} ответ на своё требование, Путин сделал именно то, о чём он предупреждал, и начал специальную военную операцию на Украине.
Как продолжает Бьюкенен, каким бы ни был характер российского президента, он подтвердил одно: ему можно верить, и когда Путин предупреждает, что он что-то сделает, он следует своим словам. И теперь Западу стоит ответить на два вопроса: как он дошёл до этого и куда теперь двигаться дальше?

\begin{fancyquotes}
    «Как мы дошли до такого положения, когда Россия, оказалась \explainDetail{прижатой}{приж\'{а}тый, -ая (прижать)}{to press} \explainDetail{спиной к стене}{прижатый спиной к стене}{with one's back against the wall}, а Соединенные Штаты, пододвигая НАТО все ближе к границам России, ещё больше загон\'{я}ли её в угол?», --- пишет Бьюкенен.
\end{fancyquotes}

Для этого автор статьи \explain{призывает}{calls} посмотреть в прошлое. В период с 1989-го по 1991 год Михаил Горбачёв позволил \explain{снести}{tear down} Берлинскую стену, \explain{воссоединить}{reunit} Германию и освободить все «\explain{порабощённые народы}{enslaved nations/people}» Восточной Европы. Также Горбачёв позволил Советскому Союзу \explain{распасться}{fall apart} на 15 независимых государств, \explain{отменил}{cancelled, abolished, put an end to} холодную войну в Европе, \explainDetail{устранив}{устран\'{и}ть}{eliminate} со стороны Москвы все причины для исторического \explainDetail{раскола}{раскол}{schism, split} мира.

Путин \explain{пришёл к власти}{came to power} в 1999 году после «катастрофического десятилетнего правления Бориса Ельцина, который \explain{букв\'{а}льно}{literally} \explain{втоптал}{trampled} Россию в землю». В том же 1999 году Путин \explain{наблюдал}{observed, watched, oversaw}, как Америка провела 78-дневную кампанию бомбардировок Сербии. В том же году три бывшие страны Варшавского договора: Чехия, Венгрия и Польша --- были приняты в НАТО.

\begin{fancyquotes}
    «Тогда \explain{справедливо}{righteously} \explain{возник}{past tense of возн\'{и}кнуть (arise)} вопрос: от кого эти страны должны были быть защищен\'{ы} американским оружием и альянсом НАТО?» --- пишет Бьюкенен.
\end{fancyquotes}

По его мнению, прямой ответ на этот вопрос был дан в 2004 году, когда Словения, Словакия, Литва, Латвия, Эстония, Румыния и Болгария были приняты в НАТО. Затем, в 2008 году, была принята Бухарестская декларация, которая поставила Грузию и Украину, граничащие с Россией, в очередь на членство в НАТО.

В том же году Грузия \explain{напала}{attacked} на Южную Осетию, что спровоцировало Россию на ответ. Американский \explain{истеблишмент}{establishment} объявил, что это была агрессивная война России, но впоследствии проведённое Евросоюзом расследование \explain{обвинило}{accused} в развязывании войны президента Грузии Михаила Саакашвили.

А потом, пишет Бьюкенен, был Крым. В 2014 году демократически избранный президент Украины Виктор Янукович был \explain{свергнут}{overthrown} в Киеве и \explain{заменён}{replaced} прозападным режимом. \explain{Вместо того чтобы}{instead of} потерять Севастополь, историческую военно-морскую базу России в Крыму, Путин \explainDetail{присоединил}{присоедин\'{я}ть/присоедин\'{и}ть}{to join, to link, to add, to connect, to attach; here: to annex}* полуостров к России. И всё это, пишет Патрик Бьюкенен, подводит нас к сегодняшнему дню.

\begin{fancyquotes}
    «\explain{Что бы мы ни думали о Путине}{whatever we think about Putin}, он --- не Сталин. Он не убивал миллионы людей и не создавал архипелаг ГУЛАГ. Он не нерационален, в чём \explainDetail{обвиняют}{обвин\'{я}ть/обвин\'{и}ть}{accuse} его некоторые учёные мужи. Он не хочет с нами войны, которая была бы более чем \explain{разрушительной}{destructive} для обеих сторон. Путин --- русский патриот, традиционалист, холодный и \explain{безжалостный}{ruthless} реалист, \explain{стремящийся}{aspiring} сохранить Россию как великую и уважаемую \explainDetail{державу}{держава}{power}, которой она когда-то была. И он верит, что она может снова стать такой», --- пишет Бьюкенен.
\end{fancyquotes}

Однако этого не может произойти, если расширение НАТО не \explainDetail{прекратится}{прекращ\'{а}ться/прекрат\'{и}ться}{stop} или если родственное России государство Украина станет частью военного альянса, который больше всего гордится тем, что выиграл холодную войну против страны, которой Путин служил всю свою жизнь.

\begin{fancyquotes}
    «Президент Джо Байден почти ежечасно обещает: «Мы не собираемся воевать на Украине». Почему же тогда он не готов исключить членство Украины в НАТО, которое потребовало бы от нас того, что, по слов\'{а}м самог\'{о} Байдена, мы, американцы, никогда не должны делать ради нашего же собственного выживания: начать войну с Россией», --- пишет главный редактор The American Conservative.
\end{fancyquotes}

\textit{*Крым вошёл в состав России после того, как за это \explainDetail{проголосовало}{голосов\'{а}ть/проголосов\'{а}ть}{to vote} подавляющее большинств\'{о} жителей полуострова на референдуме 16 марта 2014 года (прим. ИноТВ).}




\subsection{Ставки продолжают повышаться}
% https://novayagazeta.ru/articles/2022/02/27/stavki-prodolzhaiut-povyshatsia

\textit{В кризисе вокруг Украины теперь напомнили о ядерных силах России}

Президент РФ Владимир Путин приказал привести российские силы сдерживания в режим особого несения дежурства. Это демонстративное действие для сведения стран НАТО произошло в воскресенье во время встречи с министром обороны Сергеем Шойгу и начальником Генштаба Валерием Герасимовым.

На памяти опрошенных нами ветеранов такой случай были лишь однажды — в момент операции по присоединению Крыма, об этом президент Путин сам рассказал ВВС. В советское время такое происходило дважды в 1962 и 1983 годах. Однако подтверждений из других источников, что силы ядерного сдерживания были переведены в особый режим весной 2014 пока нет.

Основы госполитики в сфере ядерного сдерживания Путин утвердил в июне 2020 года. Согласно этому указу все ключевые решения, в том числе о переводе сил сдерживания в особый режим, принимает лично он. Причин, по которым вводится такой режим, несколько, и все они связываются с подготовкой других стран к применению против нашей страны ядерного оружия.


Однако, есть там и такое условие перехода на особый режим, которое соответствует нынешней ситуации на Украине, после начала там спецоперации ВС РФ


\begin{fancyquotes}
    «Развертывание государствами, которые рассматривают РФ в качестве потенциального противника, систем и средств противоракетной обороны, крылатых и баллистических ракет средней и меньшей дальности, высокоточного неядерного и гиперзвукового оружия, ударных беспилотных летательных аппаратов, оружия направленной энергии» — о таких шагах президент России говорил в своих речах неоднократно.
\end{fancyquotes}

Говоря языком военной разведки, перевод стратегических ядерных сил (СЯС) на особый режим осуществляется при вскрытии мероприятий противника по подготовке своих ядерных сил, а предотвратить кризис другим способом невозможно. Угроза разгрома ключевой для обороноспособности страны группировки войск также может привести к ответной угрозе применения ядерного оружия со стороны России. В этих случаях речь идет о вероятном нанесении ударов тактическим ядерным оружием.

Пока неизвестно, как ведут себя (и ведут ли как-то вообще) авианосные ударные группировки США, перебазировались ли стратегические бомбардировщики на аэродромы Англии и Италии, откуда они обычно летали в Ирак. Также не поступало информации о какой-то угрозе, нависшей над группировкой ВС РФ на Украине.

Скорее всего, перевод СЯС на особый режим является превентивной мерой в адрес стран НАТО, публичным предостережением от участия в военных действиях на стороне ВСУ. Тем более, что, по словам опрошенных нами авторитетных специалистов по стратегическим вооружениям, российские силы сдерживания всегда и постоянно находятся в состоянии полной готовности. Безо всяких об этом объявлений.

\subsection{«Никто не хочет к вам»}
\textit{Монолог IT-архитектора Андрея Ключко, четвертый день сидящего в подвале частного дома в Харькове}

Я нахожусь в Харькове, в северной части города, в доме моих родителей, в частном секторе. Все это время с 5 утра [24 февраля], когда начали ******* [бомбить], мы сидим в подвале. Первую ночь тут была вся семья, все собаки, и даже тетка приходила. Утром второго дня всех, кого можно было поместить в одну машину — трех женщин, трех собак и трех котов — мы отправили во Львов. Они уже там, но очень много людей едет, большие не пробки, но «тягучки»: на блокпостах их проверяют и отпускают. Кто остается во Львове, кто-то едет в Карпаты, кто-то в Польшу, там на границе большие очереди.

Сегодня [27 февраля] четвертый день, в районе 8-9 утра заехали [в город] ваши ребятки просто на джипах БМП, и следующие два часа их тут били и жгли — есть куча видео в телеграм-каналах. Все это сопровождается периодическими арт-обстрелами. То мы стреляем, а те отвечает, то те стреляют, а мы тут жухаемся. В городе много всего разбомбленного.

\textbf{Было ли страшно.}
Страха толком не было. Первые пару минут обстрела, когда было темно и вообще ни хрена непонятно, что происходит и кто где. Когда начинаешь понимать диспозицию сил, что происходит вокруг, какой тип оружия стреляет, что он может, чем больше ты понимаешь в боевых действиях, тем спокойнее их воспринимать.

\textbf{Как устроена жизнь}
У нас с отцом в подвале житейские условия плюс-минус хорошие. У нас и свет, газ и вода. В наличии еда, но есть сейчас как-то не хочется. До вчерашнего дня даже был wi-fi, но вчера обстреливали очень сильно Салтовку (знаменитый район Харькова, крупнейший жилмассив бывшего СССР — прим. «Новой»), и много, где пропал свет. У нас он остался, но у провайдера — нет, поэтому wi-fi больше нет, хотя я его успел в подвал провести.

Приходится ловить мобильный интернет, а для этого вылезать из подвала, а это небольшой стресс, поскольку очень много шума и стрельбы вокруг, и ничего непонятно. У нас дом расположен на возвышенности, а это частный сектор, и тут толком ничего нет, поэтому ночью хорошо слышно все вокруг, и кажется, что снаряды бьют прямо сюда. Но психологически мы к этому уже привыкли, просто надо сидеть и ждать. Хорошо, что есть хотя какая-то связь, вообще без нее было бы тяжело. А так видишь, что

\begin{fancyquotes}
    у нас есть военные успехи, и мораль на очень высоком уровне. У всех, кого я знаю.
\end{fancyquotes}

Большинство знакомых поуезжали или поотправляли своих на запад. Те, кто живут в жилмассивах, стараются сидеть в подвалах, куда-то закапываться. На Салтовке уже довольно много взорванных подъездов — на улице Бучмы сегодня точно был взорван дом напротив заправки, но надеюсь, людей там уже не было.

Я живу довольно далеко от центра, до ближайшей станции метро километров 5-6, и мы, конечно, туда не ходим. А те, кто в центре или ближе к метро, спускаются туда. Никакого гуманитарного ужаса в плане отсутствия еды и воды нет. Магазины закрыты, но в метро привозят еду, а те, кто сидят по подвалам, закупились в первый день.

\textbf{Как узнавать информацию.}
Мы сидим все в харьковских чатах, и когда слышим какие-то «прилеты» или «отлеты» ракет, то сверяем по чату. Раньше это было сообщество где-то сотни людей, которые играют в квесты, а теперь мы это чат харьковской переклички, люди по разным районам контролируют кто, что и где.

\begin{fancyquotes}
    Слухов особенно нет, да и я стараюсь не читать ничего, что похоже на «кум моего знакомого парня из СБУ сказал».
\end{fancyquotes}

Читаю официальные каналы, УНИАН и «Суспильное», и этого в принципе хватает для понимания. Большинство сообщений в этих каналах обнадеживающие и поддерживающие, поэтому настроение довольно высокое.

\textbf{Почему не уехал.}
Я с самого начала знал, что я никуда не поеду. Сложно это объяснить, но для меня даже вопрос такой не стоял. Я в принципе знал, что у родителей есть подвал, хотя никто в семье до последнего момента об этом не упоминал. И готовили [для жизни] мы его уже во время первого обстрела утром первого дня.

Но я знал, что даже если будет самый плохой сценарий — какой примерно и произошло — это дает какое-то [ощущение] тыла. Люди без такого тыла и понимания, что делать, больше боялись и больше были склонны переехать ближе к западной границе. Сыграло свою роль и то, что в моей семье много родственников живут в «ДНР» и «ЛНР». Они передали мне опыт сидения под обстрелами, распознавания разных видов вооружения. Мы все знали, что это часть жизни — но раньше это была часть жизни наших родственников, а теперь и нашей. И это не так страшно, но тем, кто так близко не соприкасался — а у меня мама и папа родом с территории Донецкой и Луганской областей, ныне контролируемой «ДНР» и «ЛНР» — наверное, было сложнее справляться со стрессом.

\textbf{Отношение к происходящему.} Честно? Спокойное. Просто хочется, чтобы это закончилось и дошло до нашей логической победы. Нам сейчас нужно воевать, в переговоры верится слабо. Я не понимаю, какие гарантии Россия может дать кому-либо, даже своим гражданам. У многих ощущение, что чем больше украинские вооруженные силы нанесут ущерба российским вооруженным силам, тем ближе конец этого всего, и это единственный путь. Но лично меня деморализовали подвижки нашего украинского руководства к каким-то переговорам. Это мое личное отношение.



\textbf{Жертвы среди мирного населения.} Они есть, но надо сказать, что когда мы видим попадания в девятиэтажки, что в Киеве, что на Салтовке, то эти дома, слава богу, по крайней мере частично были «самоэвакуированы». В каналах выкладывают информацию о погибших людях, но с большинством из них это случилось на улице. Это те, кто не успел добежать до бомбоубежища или по какой-то причине не пошел туда.

Есть такая ироничная точка зрения, что харьковчанам несколько все равно на происходящее. Из забавного я видел просьбу от наших официальных сил не ждать транспорта в связи с военной ситуацией. Я просто представил, что где-то сейчас стоят люди на остановках и ждут трамвая. Или вот мой приятель с Салтовки ответил на мой вопрос о том, как у него дела, так: «Да все в порядке, обычная салтовская история. Грабят магазины, бьем тех, кто грабит… Ну да, еще бомбят». Жертвы единичны или, возможно, их десятки, но как оценить не знаю. Про жертвы среди военных информации почти нет.

\textbf{Верилось ли, что может быть война?}
Я долгое время игнорировал [разговоры о будущей войне], потому что эта новостная тревога девальвируется. Путин что-то сказал, Песков что-то сказал, и я начал на всех злиться: «Путин, ну сколько можно, Байден, перестань, это просто какая-то херня». И я очень злился на все медиа — и украинские, и российские, потому что каждый чей-то вздох превращался в [алармистские] заголовки. Я думал, зачем нужен этот пустой звон, и хотелось сказать: «Заткнитесь все, ведь если что-то произойдет, мы узнаем первые». Потом люди начали куда-то собираться, совершенно случайно — в давно запланированные командировки и отпуска, начали покидать страну. Я понимаю, почему, но это немного раздражало. Я долгое время хранил скептически-саркастический взгляд: «Ну да, ну едьте, ну и сидите там». 90\% людей, которые сейчас находятся во Львове, выехали туда уже во время боевых действий, это все решается, и незачем было развивать эту панику до начала событий.

Я ожидал какого-то адища по линии соприкосновения «ДНР», «ЛНР» и Украины, но что начнут Харьков, Одессу, Херсон и Киев херачить — нет. Около «ДНР» и «ЛНР» стояли наши лучшие части, самые опытные, и они сильно зашиты в землю, они копали окопы все эти 8 лет, и я думал, что их оборону просто не прорвут. И все эти Совбезы, разрешения Совфеда [на ведение войны] меня не волновали, потому что я знал, что у нас есть армия, и если что, мы будем драться. То, что сейчас делал Путин, мы видели и перед Крымом. Ощущение было такое: «Ну да, и что? Так это уже было. Пусть входит — будем воевать». Слава богу, что мы не струсили, мы стоим, и у нас есть армия. Это круто видеть, что она вооружена и профессиональна, и это не какие-то солдатики в советских кирзовых сапогах, как было в 2014 году.


\textbf{Почему так мало эмоций} Есть заезженная фраза, что война идет уже 8 лет. И, действительно, если бы я записывал это аудиосообщение в 2014 году, то я бы, наверное, срывался на более высокие ноты. Сейчас, после всего, что произошло с моими родственниками на Донбассе, мы не то, что привыкли, но психика адаптировалась к этому агрессору, и ты воспринимаешь эту ситуацию более спокойно. Нет смысла что-то орать, махать кулаками и психовать. Отношение — крайне агрессивное и негативное, но оно уже так долго такое, что выражать его наружу, наверное, психика уже устала. Надо просто воевать и закончить с этим всем.

Но я могу сказать, что все, кто причастен к этой войне со стороны РФ, — мрази и ублюдки. Если они все умрут, я буду крайне счастлив, а все, что происходит здесь — это полный ******* [кошмар]. И одно дело защищать какие-то принципы, даже выдуманных нацистов на Донбассе бить, но

\begin{fancyquotes}
    кого вы бьете в Одессе, Киеве и Харькове? Каких нацистов? Это просто полный сюр!
\end{fancyquotes}

\textbf{Что, если Харьков оккупируют?}
Город не будет оккупирован ни в коем случае, судя по тому, что я вижу, грубо говоря, из окна. Сегодня, когда утром заезжали российские боевые машины с севера и ехали куда-то на скорости, это увидели все. Ведь до этого [происходило] по сути, на окраинах, а сегодня какие-то машины доезжали чуть ли не до центра, где их уничтожали. Было такое впечатление, что они ехали просто бесцельно, и это приводит в замешательство. Очевидно, что большие города-миллионники не захватываются сотней другой боевых машин, это делается не так, и непонятно, зачем это кому-то было нужно.

Чтобы оккупировать Харьков, нужно выбить отсюда украинские войска авиаударами — этого нет, потом заходить колоннами танков — этого тоже нет. И остаются люди, Харьков — полуторамилионный город, и людей здесь до сих пор до хрена. Это огромный город, нелояльный россиянам. Большая разница с Донецком в 2014 году, который был лоялен россиянам. Тогда люди из Донецка видели Крым, у них был пример, они хотели к вам, в Россию. Сейчас в Харькове все видели, что происходит эти 8 лет в «ДНР» и «ЛНР», весь этот ****** [ужас], людей, ездящих за пенсией к нам.

\begin{fancyquotes}
    Никто не хочет к вам. Что вы будете делать с этими людьми?
\end{fancyquotes}

Вырезать и убивать, а потом перевозить сюда из Донецка людей, которые должны будут стать харьковчанами? В этот бред не верится. Абсурдность всего того, что я сейчас произнес, говорит о том, что этого не произойдет. Но даже если бы нас тут сравнивали просто с землей, то я бы все равно оставался в подвале. Я бы не бежал никуда. Я бы просто ждал. Не знаю чего.

Мне трудно ответить на такой вопрос, потому что я вообще об этом не думал. Это как спросить: «Андрей, что ты будешь делать, если окажешься в марсианской экспедиции?» От меня это столь же далекая перспектива, как и оккупированный Харьков. И это касается практически каждого областного центра — я не верю, что они будет оккупированы и заняты, и даже вроде как занятый российской армией Херсон, уверен, будет освобожден в ближайшие дни.

\textbf{Обращение к обычным россиянам.} Я ничего не хочу сказать обычным россиянам, среди которых много моих знакомых. Мне кажется, они сами все знают.

В отборе на Евро-2000 Андрей Шевченко в Лужниках в стыковом матче забил знаменитый гол Филимонову за шиворот. Мне было примерно 10-11 лет, и я плакал у телевизора и был счастлив (кажется по голосу, что Ключко и сейчас начинает плакать — прим. «Новой»). И я ничего не хотел сказать обычным российским болельщикам, мне было на них все равно. Я хотел сказать тогда украинским болельщикам и сейчас — украинским гражданам, что мы ******** [классные] и мы победим.