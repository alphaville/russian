% \chapter{Искусственный интеллект}

\section{Искусственный интеллект для «чайников»}

\textit{Николай Воронин, научный корреспондент}

\textit{18 июля 2023 г.}

\textit{Источник: \url{https://www.bbc.com/}}

% https://www.bbc.com/russian/resources/idt-74697280-e684-43c5-a782-29e9d11fecf3



Вы уже разобрались в том, что такое искусственный интеллект?

За последние полгода огромной популярностью стали пользоваться чат-боты вроде ChatGPT и генераторы картинок типа Midjourney.

Однако эволюция искусственного интеллекта (AI или ИИ) и машинного обучения идет уже далеко не первый год.

В этом \ed{руководстве для начинающих}{руководство для начинающих}{beginner's guide; учебное пособие по какому-либо предмету. Примеры: Руководство по физике. Руководство по тренерской работе. Руководство по пользованию прибором.} мы познакомим вас не только с чат-ботами, но и с другими видами AI. Мы расскажем о том, какую роль в нашей жизни уже сейчас играют алгоритмы.

\textbf{Как происходит обучение AI?}

В основе работы любой умной машины лежит процесс, известный как обучение алгоритма, когда в компьютерную программу вкладывают для анализа огромный объем данных (иногда с поясняющей разметкой) и набор инструкций.

Инструкции могут выглядеть примерно так: «Отбери из этих фотографий те, на которых есть лица» или «Отсортируй эти звуки по заданным категориям».

Программа начинает «прочесывать» (анализировать) имеющуюся в ее распоряжении информацию в поисках закономерностей — в соответствии с поставленной задачей.

Получаемые в процессе обработки результаты необходимо постоянно корректировать, помечая ошибки программы --- например, «это не лицо» или эти два звука относятся к разным категориям». Однако на выходе, по итогам анализа всех имеющихся данных, мы получаем «умную» или, скорее, «натренированную» модель, неплохо справляющуюся с поставленной задачей. То есть алгоритм.

Чтобы лучше понять, как в результате такого обучения получаются различные виды AI, можно сравнить их с разными видами животных.

Как у животных под влиянием окружающей среды за миллионы лет эволюции развились те или иные характерные особенности, так и AI, миллионы раз проанализировавший одну и ту же базу данных, приобретает необходимые навыки, все лучше и лучше справляясь с поставленной задачей.

Давайте рассмотрим несколько видов AI, \ed{приспособленных}{приспособленный}{adapted; fit} для решения разного рода задач.

\textbf{Что такое чат-боты?}

\begin{wrapfigure}{l}{0.5\textwidth}
    \begin{fancyquotes}
        Чат-бот напоминает попугая, который может повторять услышанные слова и даже до некоторой степени соотносить их с контекстом, но значение этих слов (если допустить, что он вообще различает отдельные слова) попугаю известно лишь очень приблизительно.
    \end{fancyquotes}
\end{wrapfigure}

Чат-боты делают примерно то же самое, только на значительно более сложном уровне — и, кажется, вскоре могут полностью изменить наши отношения с письменной речью.

Но как чат-боты научились писать?

Чат-боты относятся к \ed{подвиду}{подвид}{subspecies, subtype} AI, известному как «большие языковые модели» (LLM), и их обучают на колоссальных объемах текста, совершенно немыслимых для человека.

LLM может сравнивать не только отдельные слова, но и целые предложения, а также анализировать, в каком контексте использованы те или иные слова и выражения в различных \ed{отрывках}{отрывок}{учсукзе}, находя их поиском по всей имеющейся в распоряжении алгоритма базе данных.

Используя эти миллиарды сравнений между словами и фразами, он может прочитать вопрос и сгенерировать ответ — вроде того, какой вы можете получить при обычном обмене текстовыми сообщениями на вашем телефоне.

Удивительная особенность больших языковых моделей заключается в том, что они могут изучать правила грамматики и определять значение слов из контекста самостоятельно, без помощи человека.

\begin{fancyquotes}
    «Думаю, что лет через 10 у нас будут чат-боты, которые станут экспертами в любой необходимой области знания. То есть с их помощью вы сможете задать свой вопрос опытному ``врачу,'' опытному ``учителю'', опытному ``юристу'' --- и получить от них квалифицированный ответ».\\

    \begin{flushright}
        Сэм Олтман — генеральный директор OpenAI,\\
        создатель ChatGPT
    \end{flushright}
\end{fancyquotes}

\textbf{Могу ли я поговорить с AI?}

Если вы когда-либо пользовались Alexa, Siri, Алисой или любым другим голосовым помощником, значит, вы уже разговаривали с AI.

Представьте себе кролика, длинные уши которого специально приспособлены для того, чтобы различать малейшие изменения звука.

Когда вы говорите с голосовым помощником, AI записывает вашу речь, фильтрует фоновый шум, раскладывает фразу на отдельные звуки и слоги --- а затем \ed{сопоставляет}{сопоставлять}{сравнивая, соотнести друг с другом для получения каких-либо выводов (to match up, to compare, to contrast)} каждый из них их с огромной базой данных --- библиотекой звуков того или иного языка.

На их основе AI воссоздает произнесенную фразу в текстовом виде, причем любые допущенные на этом этапе ошибки прослушивания можно скорректировать до того, как будет дан ответ.

Такой тип AI называют обработкой естественного языка (NLP).

Именно этот принцип лежит в основе технологий, которая позволяет вам подтвердить голосом по телефону банковскую транзакцию, просто сказав «да», или попросить мобильный телефон показать вам прогноз погоды на ближайшие несколько дней в городе, куда вы собираетесь поехать.


\textbf{Может ли AI распознавать изображения?}

Ваш телефон уже предлагал вам отобрать ваши фотографии в папки с названиями вроде «На пляже» или «На вечеринке»?

Если да, значит, вы уже пользовались AI, даже не осознавая этого. Алгоритм обнаружил на ваших снимках некие закономерности и сгруппировал похожие фото.

Такие программы обучают, давая алгоритму для анализа миллионы изображений, сопровожденных простым описанием.

Если вы дадите AI, распознающему изображения, достаточно картинок с пометкой «велосипед», в конце концов он начнет понимать, как выглядит велосипед и чем он отличается от лодки или автомобиля.

Иногда AI специально тренируют находить крошечные различия в похожих изображениях.

Именно так работает система распознавания лиц, способная рассчитать уникальные пропорции каждого лица, отличающие его от любого другого лица на планете.

Работающие по этому принципу алгоритмы научились анализировать медицинские снимки, выявляя \ed{злокачественные опухоли}{злокачественная опухоль}{malignant tumour} на самых ранних этапах формирования, поскольку за время, которое требуется консультанту для детального анализа одного изображения, AI способен обработать тысячи снимков.


\textbf{Как AI создает новые изображения?}

Не так давно появились модели AI, способные не только распознавать изображения, но и изменять на них линии и цвета, подобно хамелеону.

Такие алгоритмы, генерирующие изображения, обладают способностью на основе анализа миллионов изображений идентифицировать их отдельные элементы и даже сложные узоры, собирая из этих деталей новые картинки.

Например, вы можете попросить AI создать «фотографию» вымышленного события — вроде фото человека, идущего по поверхности Марса.

Вы можете задать изображению творческое направление: например, «Нарисуй портрет английского футбольного менеджера в стиле Пикассо».

AI-алгоритмы последнего поколения начинают процесс создания нового изображения с набора случайно окрашенных пикселей.

В этих случайных точках алгоритм ищет любой \ex{намек}{hint} на известный ему шаблон (то есть узор или закономерность построения линий), идентифицированный программой в ходе обучения — на основе таких шаблонов AI и создает новые изображения.

Постепенно шаблоны становятся все лучше. На полученное изображение накладываются все новые и новые слои, в каждом из которых алгоритм сохраняет точки, которые укладываются в искомый шаблон, и отбрасывает все остальные — пока, наконец, изображение не начнет отвечать поставленной задаче.

Получив все необходимые элементы изображения — «поверхность Марса», «астронавт» и «прогулка», — AI складывает их вместе, получая на выходе новую картинку.

Поскольку новое изображение построено из слоев случайных пикселей, в результате получается то, чего раньше никогда не было, но оно по-прежнему основано на миллиардах паттернов, полученных из исходных обучающих изображений.

Сейчас общество начинает задумываться о том, какие последствия это будет иметь для охраны авторских прав и насколько этично создавать новые произведений искусства на основе кропотливого труда настоящих художников, дизайнеров и фотографов.


\textbf{А что насчет беспилотных автомобилей?}

Разговор об автомобилях, не требующих водителя, ведутся уже на протяжении десятилетий и \ex{прочно}{firmly} \ex{закрепились}{established themselves} в массовом воображении благодаря произведениям научной фантастики.

Предназначенные для этой цели алгоритмы известны как программы автономного вождения, и для их работы автомобили необходимо \ex{оснастить}{equip} камерами, радарами и лазерными датчиками.

\begin{wrapfigure}{r}{0.5\textwidth}
    \begin{fancyquotes}
        Представьте себе стрекозу с обзором в 360 градусов и миниатюрными датчиками на крыльях, которые помогают ей маневрировать в процессе полета и постоянно корректировать свою \ed{траекторию}{траектория}{trajectory}.
    \end{fancyquotes}
\end{wrapfigure}
Точно таким же образом AI анализирует показатели датчиков автомобиля, обнаруживая вокруг себя различные объекты, выясняя, движутся ли они, и различая, что это — другой автомобиль, велосипед, пешеход или еще что-нибудь.

Анализ тысяч часов видеозаписи примеров идеального вождения научил AI в реальном мире мгновенно принимать на дороге правильные решения, чтобы управлять автомобилем, избегая \ed{столкновений}{столкновение}{действие по значению гл. сталкиваться, столкнуться (collision)}.

Долгие годы алгоритмы пытались найти закономерности в часто непредсказуемом характере водителей, но на сегодняшний день беспилотные автомобили проанализировали данные, собранные в ходе движения автомобилей по реальным дорогам, проехавших в общей сложности не один миллион километров. И в Сан-Франциско уже работают первые коммерческие автономные такси без водителя.

Автономное вождение также является \ed{наглядным}{наглядный}{недвусмысленно и доходчиво выражающий что-то своей внешностью (\textit{Наглядный пример}: illustrative example); предназначенный для визуального пояснения, служащий иллюстрацией (\textit{Наглядные пособия}: visual aids)} примером того, что новым технологиям приходится преодолевать далеко не только технические препятствия.

Перспектива полностью автоматизированного трафика будущего \ex{упирается в}{rests on} существующие законодательные нормы и правила техники безопасности, а также в совершенно понятное общественное беспокойство по поводу возможных последствий передачи управления автомобилем от живого человека компьютерному алгоритму.

\begin{fancyquotes}
    «Думаю, никто не будет спорить с тем, что безусловным приоритетом дорожного движения может быть только его безопасность. Особенно интересно слушать рассуждения на эту тему именно сейчас, когда люди и роботы-водители справляются с вождением примерно на одинаковом уровне. Однако, учитывая скорость, с которой совершенствуются алгоритмы, уже через несколько лет разговоры эти уйдут в прошлое. Потому что AI будет водить настолько лучше человека, что и обсуждать там будет совершенно нечего».\\

    \begin{flushright}
        Кайл Фогт — генеральный директор компании Cruise, занимающейся беспилотными автомобилями.
    \end{flushright}
\end{fancyquotes}

\textbf{Что AI знает обо мне?}

Некоторые AI просто работают с числами, собирая и комбинируя их в объеме, чтобы создать рой информации, результаты которого могут быть чрезвычайно ценными.

Скорее всего, уже существует несколько профилей вашей финансовой и социальной активности, особенно онлайн, которые можно использовать для прогнозирования вашего поведения.

Ваша карта лояльности супермаркета \ex{отслеживает}{keeps track} ваши привычки и вкусы в вашем еженедельном магазине. Кредитные агентства отслеживают, сколько у вас денег в банке и сколько вы должны по кредитным картам. Netflix и Amazon отслеживают, сколько часов контента вы просмотрели прошлым вечером. Ваши учетные записи в социальных сетях знают, сколько видео вы прокомментировали сегодня.

И не только вы, эти цифры существуют для всех, что позволяет моделям AI анализировать их в поисках социальных тенденций. Эти модели AI уже формируют вашу жизнь: от помощи в принятии решения о том, можете ли вы получить кредит или ипотеку, до влияния того, как ваш выбор рекламы в интернете влияет на ваши покупки.


\textbf{Сможет ли AI делать всё?}

Можно ли объединить некоторые из этих навыков в единую гибридную модель AI? Это именно то, что может делать одно из самых последних достижений в области AI. Он называется мультимодальным AI и позволяет просматривать различные типы данных, такие как изображения, текст, аудио или видео, и \ex{выявлять}{reveal} между ними новые закономерности.

Этот мультимодальный подход был одной из причин огромного \ed{скачка}{скачок}{leap} в возможностях ChatGPT, когда его модель AI была обновлена с GPT3.5, который обучался только на тексте, до GPT4, который также обучался с изображениями.

Идея единой модели AI, способной обрабатывать любые данные и, следовательно, выполнять любые задачи, от перевода между языками до разработки новых лекарств, известна как общий искусственный интеллект (ИИК).

Для некоторых это конечная цель всех исследований искусственного интеллекта; для других это путь ко всем этим научно-фантастическим \ed{антиутопиям}{антиутопия}{dystopia}, в которых мы высвобождаем интеллект настолько далеко за пределами нашего понимания, что мы больше не можем его контролировать.


\textbf{Как обучать AI?}

До недавнего времени ключевой процесс обучения большинства AI был известен как «обучение с учителем».

Огромные наборы обучающих данных были обозначены людьми, и AI попросили выяснить закономерности в данных.

Затем AI попросили применить эти шаблоны к некоторым новым данным и дать отзыв о их точности.

Например, представьте, что вы даете AI дюжину фотографий — шесть помечены как «автомобиль» и другие шесть помечены как «фургон».



Затем попросите AI разработать визуальный шаблон, который сортирует автомобили и фургоны на две группы.

Как вы думаете, что происходит, когда вы просите его классифицировать эту фотографию?



К сожалению, похоже, что AI думает, что это фургон. Он, оказывается, не такой уж и сообразительный.

Теперь вы показываете это.



И это говорит вам, что это машина.

Совершенно ясно, что пошло не так.

Из ограниченного количества изображений, на которых он обучался, AI решил, что цвет — лучший способ отделить автомобили от фургонов.

Но самое удивительное в программе AI то, что она пришла к этому решению сама, и мы можем помочь ей усовершенствовать процесс принятия решений.

Мы можем сказать ему, что он неправильно идентифицировал два новых объекта — это заставит его найти новый шаблон на изображениях.

Но что еще более важно, мы можем исправить смещение в наших обучающих данных, придав им более разнообразные изображения.

Эти два простых действия, взятые вместе — и в огромном масштабе — это то, как большинство систем AI были обучены принимать невероятно сложные решения.

\textbf{Как AI учится самостоятельно?}

Обучение с учителем — невероятно мощный инструмент для обучения, но многие недавние прорывы в области искусственного интеллекта стали возможными именно благодаря обучению без учителя.

Проще говоря, именно здесь использование сложных алгоритмов и огромных наборов данных означает, что AI может учиться без какого-либо руководства со стороны человека.

ChatGPT — самый известный пример.

Объем текста в интернете и в оцифрованных книгах настолько огромен, что за много месяцев ChatGPT смог самостоятельно научиться осмысленно комбинировать слова, а затем люди помогали настраивать его ответы.

Представьте, что у вас есть большая стопка книг на иностранном языке, возможно, некоторые из них с картинками.

В конце концов вы могли бы выяснить, что одно и то же слово появлялось на странице всякий раз, когда был рисунок или фотография дерева, и другое слово, когда была фотография дома.

И вы увидите, что рядом с этими словами часто было слово, которое могло означать «это» или, может быть, «этот» — и так далее.

ChatGPT провел такой тщательный анализ отношений между словами, чтобы построить огромную статистическую модель, которую затем можно использовать для прогнозирования и создания новых предложений.

Он опирается на огромную вычислительную мощность, которая позволяет AI запоминать огромное количество слов — по отдельности, группами, в предложениях и на разных страницах — а затем читать и сравнивать, как они используются, снова и снова за долю секунды. второй.

Быстрый прогресс, достигнутый моделями глубокого обучения в прошлом году, вызвал новую волну энтузиазма и беспокойства по поводу потенциала искусственного интеллекта, и нет никаких признаков того, что он замедлится.

Обещания и предупреждения научной фантастики, кажется, внезапно подкрались к нам, и мы обнаруживаем, что уже живем в мире, где AI начинает раскрывать свои странные нечеловеческие способности.

\begin{fancyquotes}
    «Ответ на то, как мы можем подготовить машины для этого этически сложного мира, заключается в том, как мы воспитываем наших собственных детей и готовим их к встрече с нашим сложным миром. Когда мы воспитываем детей, мы не знаем, с какими именно ситуациями они столкнутся. Мы не кормим их с ложки ответами на все возможные вопросы; скорее, мы учим их, как найти ответ самостоятельно».\\

    \begin{flushright}
        Мо Гавдат — писатель и бывший главный бизнес-директор Google X.
    \end{flushright}
\end{fancyquotes}




