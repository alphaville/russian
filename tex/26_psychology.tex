\chapter{Личность, Характер и Психология}
\section{Что такое думскроллинг}

\textit{и почему чтение плохих новостей пагубно влияет на ментальное здоровье?}

\textit{Источник: \url{https://www.elle.ru}}
% https://www.elle.ru/otnosheniya/psikho/chto-takoe-dumskrolling-i-pochemu-chtenie-plokhikh-novostei-pagubno-vliyaet-na-mentalnoe-zdorove/

\begin{fancyquotes}
    Постоянно просматриваете новости? Не выпускаете смартфон из рук? У нас для вас плохие новости — вы зависимы, и это пагубно влияет на ваше здоровье и умственные способности.
\end{fancyquotes}

Думскроллинг (от английского doom — «гибель, судьба, рок, Судный день» и scrolling — «прокрутка») — это склонность к просмотру и чтению плохих новостей, несмотря на то, что они вызывают негативные эмоции, удручают, огорчают и деморализуют. Термин стал относительно широко использоваться в начале пандемии и сейчас снова стал актуален на фоне нестабильной ситуации в мире. Так почему же мы не можем оторваться от плохих новостей и как это влияет на наше психоэмоциональное состояние?

Основная причина бесконтрольного думскроллинга — это боязнь пропустить важные новости. Беспокойный ум стремится понять, что происходит в мире и как это может коснуться лично нас. В таком случае думскроллинг дает ощущение контроля над ситуацией. Интуитивно мы пытаемся подготовиться к потенциальным угрозам. Принцип «предупрежден — значит вооружён» миллионы лет способствует выживанию людей как вида.

\textbf{Почему это вредно?}

Иллюзия контроля над ситуацией, по большому счету, не дает никаких преимуществ. Думскроллинг способствует развитию тревожности и стресса, повышается вероятность панических атак, снижается концентрация. Умные алгоритмы соцсетей предлагают все больше и больше плохих новостей, поиск и чтение пугающих статей превращается в зависимость, человек игнорирует собственные мысли и чувства. Впоследствии чтение негативных новостей может приводить и к ухудшению сна и истощению нервной системы.

\textbf{Как бороться с думскроллингом?}

Не пользуйтесь гаджетами перед сном. Не читайте новостей о коронавирусе, войнах, протестах и других тревожных явлениях на ночь. Если вам сложно контролировать это самостоятельно, то установите будильник и за 2-3 часа до сна переводите телефон в авиарежим.

Читайте только ту информацию, которая вам нужна, не переключайте внимание на другие новости. Перед тем, как читать новости и статьи в интернете, четко определите «цель визита», не обращайте внимание на предложенные статьи или кликбейтные заголовки.

Отвлекитесь от новостного контента. Попробуйте сместить фокус внимания с новостей на интересные статьи, интервью, рецензии.

Займитесь чем-то другим. Кино, музыка, встречи с друзьями — все это поможет провести вам время с куда большей пользой для нервной системы.

\section{Синдром спасителя}

\textit{Как добрые намерения скрывают эмоциональные недостатки}

\textit{Что скрывает навязчивое желание помочь ближнему, если вас об этом никто не просит}

\textit{Источник: \url{https://www.elle.ru}}
% https://www.elle.ru/otnosheniya/psikho/sindrom-spasitelya-kak-dobrye-namereniya-skryvayut-emocionalnye-nedostatki/

Проявления синдрома спасителя не всегда очевидны для окружающих и даже тех, кого непрошеные благодетели окружают заботой. Импровизированные супергерои всегда готовы помочь нерасторопным коллегам, спасти близких (и не очень) людей от жизненных невзгод и пагубных пристрастий. Именно такие светлые человечки рады круглосуточно наставлять подшефных по вопросам правильного питания, карьерного роста и токсичных отношений. Ведомые убеждением, что в этом их высшее предназначение, «спасители» возводят альтруизм в культ, в действительности скрывая за самоотдачей эгоизм и невротические изъяны.

Термин «спаситель» используется психологами с 1968 года, с тех пор, как доктор Стивен Карпман, ученик Эрика Берна и знаток транзактного анализа, раскрыл в опубликованной работе модель социального и психологического взаимодействия, названную в его честь «треугольник Карпмана» (он же — «треугольник судьбы» или Karpman drama triangle). В статье Fairy Tales and Script Drama Analysis американский ученый описал три привычные роли, которые мы часто играем в разных ситуациях: жертва, преследователь и спаситель, который вмешивается, как кажется, из желания помочь тому, кого обижают или недооценивают.

Как разъяснил Карпман, ролевая игра, схожая с мелодраматическим сюжетом про «героя, злодея и девицу в беде», раскрывает неочевидный мотив: спаситель заинтересован поддерживать жертву в ее зависимости от себя. Догадываетесь почему?

\textbf{Как из заботы получается созависимость}

«Нужда помогать другим движет личностью, способной реализоваться исключительно в опеке других, — объясняет Анн-Виктуар Русселе, парижский психолог и терапевт. — Такие люди считают своим долгом спасать других в ущерб себе, включая тех, кто в этом абсолютно не нуждается. Они сознательно вступают в созависимые отношения, считая, что не заслуживают любви партнера, но убеждая себя, что эта связь оправдана желанием избавить ее/его от проблем. Настойчивая услужливость имеет в корне нарциссический изъян, скрывающий неуверенность в себе и сопутствующую мотивацию: потребность поднять самооценку. „Спаситель“ становится лучше в собственных глазах, проецируя на ближних позитивные намерения и поступки».

Чего же следует ожидать от непрошеной заботы? «Спасаемые» увиливают от спасения, не предлагая взамен долгожданной компенсации, что, естественно, отзывается в душе супергероя горьким разочарованием. «Я делаю все для всех, но никто никогда ничего не делает для меня», — типичная жалоба отвергнутого спасителя.

«Последствия амбивалентного синдрома непременно дадут о себе знать, — пишут калифорнийские психологи Мэри Ламиа и Мэрилин Кригер в книге „Синдром Белого Рыцаря: спасение себя от потребности спасти других“ (The White Knight Syndrome: Rescuing Yourself from Your Need to Rescue Other). — В начале отношений спаситель кажется удовлетворенным своей самоотверженностью, но со временем становится все более несчастным и бессильным. Она/он буквально выдыхается, теряя смысл, интерес, энергию, ресурсы, что, в свою очередь, отражается на самооценке. Убедившись, что усилия напрасны, «белый рыцарь» выходит из игры эмоционально и психологически истощенным».

\textbf{В чем (опять) виноваты детские травмы}

Потребность спасать указывает на эмоциональный и психологический дисбаланс, ноги которого предсказуемо растут из воспитания, образования и внушенных ценностей. «Она/он спасает всех вокруг, стараясь быть „хорошей девочкой/мальчиком“, чтобы получить одобрение со стороны реального или внутреннего родителя и укрепить самооценку, — описывает природу травмы доктор Русселе. — Возможно, в детстве „спасителю“ приходилось помогать больной матери, заботиться о братьях и сестрах, с раннего возраста посвящая себя нуждам взрослых и опуская свои потребности в нижний ранг приоритетов».

Пресловутый дисбаланс имеет тенденцию преследовать «спасителя» до зрелости, поддерживая в ней/нем потребность окружить себя партнерами, друзьями и коллегами, о которых нужно будет заботиться. И эта иллюзорная стратегия предсказуемо обречена на неудачу, потому что суть всего успешного базируется на гармонии.

\textbf{Как избавиться от синдрома спасителя}

От самоотверженного поведения никто не застрахован, но, к счастью, существуют когнитивные методики, которые помогут скорректировать психологический разлад. «Чтобы избавиться от непреходящего стремления быть кому-то нянькой, бросьте все силы на прокачку самоуважения и любви к себе, — призывают Мэри Ламиа и Мэрилин Кригер. — «Спасителям» надо усилием воли сменить тактику, признав наконец, что их любят не за сервис, который они обеспечивают, а за то, кто они на самом деле».

Увязшим в «спасающем» паттерне непросто пойти на поправку — изменить отношение к себе и окружающему миру мешает страх перед возможным одиночеством. А что, если опекаемые отвернутся и забудут насовсем? А что, если вместе с заботами о других из жизни исчезнет смысл?

Чтобы заблокировать страхи, доктор Русселе советует свести помощь к «контрактному» формату, а попросту — договориться. «Если хотите подстраховать себя от разочарований, вместо того чтобы помогать без спроса, обсудите напрямую, чем вы можете быть полезны для конкретного человека. Так вы заранее поймете, готовы ли оказывать услуги без ожиданий безграничной признательности и вдобавок проявите реальную заботу о себе — в кои-то веки. К тому же это хорошая практика в соблюдении личных границ, на которые каждый из нас имеет право».

\section{Мой характер}
\textit{Источник: \url{https://ru4.ilovetranslation.com/yuYUND7JV3L=d/}}

О себе говорить приятно, но немного трудно. Приятно, потому что всем нравится говорить о своих интересах, вкусах и предпочтениях. Но это в то же время трудно, так как изучить человека, особенно себя самого, не так уж просто.

Прежде чем говорить о своем характере, хотелось бы сначала уточнить, что такое характер. Человек отличается от остальных своими качествами. Часто люди говорят, что я не такой как остальные. Но я не считаю, что я какой-то особенный. В темноте все кошки серые. Но если вы подойдете ближе и включите свет, вы увидите, что мне присущи определенные черты.

Но не будем вдаваться в подробности, и немного сократим рассказ. У меня хорошее чувство юмора, я ответственный, трудолюбивый и эмоциональный человек. Мне нравится творчество, и я ценю эту черту в других людях. Я не люблю ложь и чувствую, когда другие лгут.

Я стараюсь никогда не опаздывать и \explain{терпеть}{to brook} не могу, когда другие не приходят \explain{вовремя}{on time}. Я предпочитаю общаться с умными и вежливыми людьми. \explain{Досадно}{it's annoying}, когда тот, кому ты \explain{доверяешь}{доверять/доверить: to trust}, оказывается \explain{ненадежным}{ненадежный: unreliable} человеком.

Я стараюсь \explain{обращаться}{to treat / обратиться} с другими так, как я хотел бы, чтобы они обращались со мной. Я ищу человека со здоровым и сильным ум\'{о}м и телом. Человека, с которым интересно общаться, которому я могу доверять и на кого можно положиться.

Что касается моих интересов, мне нравится психология в плане общения с людьми, а также способа формирования мыслей наилучшим образом. Я очень люблю путешествовать, встречаться с новыми людьми, знакомиться с их традициями и обычаями, их культурой, смотреть достопримечательности. Мне также нравятся разные стили музыки, нравится ритмичная музыка, под которую можно танцевать.


\section{Психоанализ Зигмунда Фрейда}

\textit{Предпосылки и базовые идеи за 5 минут}

\textit{Источник: \url{https://bit.ly/3sfBWgd}}

Изучением психики человека уже не один десяток лет занимаются великие умы, но на многие вопросы ответов до сих пор нет. Что скрывается в глубинах человеческого существа? Почему события, произошедшие когда-то в детстве, по сей день оказывают влияние на людей? Что заставляет нас совершать одни и те же ошибки и мёртвой хваткой держаться за опостылевшие отношения? Где берут своё начало сновидения, и какая информация в них заложена? На эти и множество других вопросов, относительно психической реальности человека, может ответить революционный и поправший собой многие основы психологии психоанализ, созданный выдающимся австрийским учёным, неврологом и психиатром Зигмундом Фрейдом.

\textbf{Как появился психоанализ?}

В самом начале своей деятельности Зигмунд Фрейд успел поработать с выдающимися учёными своего времени – физиологом Эрнстом Брюкке, практикующим гипноз врачом Иосифом Брейером, неврологом Жаном-Маре Шарко и другими. Часть мыслей и идей, которые зародились на этом этапе, Фрейд развивал и в своих дальнейших научных трудах.

Если говорить более конкретно, то ещё молодого тогда Фрейда привлекло то, что некоторые симптомы истерии, проявлявшиеся у больных ею, не могли никак быть интерпретированы с физиологической точки зрения. К примеру, человек мог ничего не чувствовать в одной области тела, несмотря на то, что в соседних областях чувствительность сохранялась. Ещё одним доказательством того, что далеко не все психические процессы могут быть объяснены реакцией человеческой нервной системы или актом его сознания, было наблюдение за поведением людей, которые подвергались гипнозу.

Сегодня все понимают, что если находящемуся под гипнозом человеку внушить приказ что-либо выполнить, после своего пробуждения он бессознательно будет стремиться к его выполнению. А если поинтересоваться у него, почему он хочет это сделать, он сможет привести вполне адекватные объяснения своему поведению. Отсюда и получается, что психика человека имеет свойство самостоятельно создавать объяснения каким-то поступкам, даже если в них нет никакой необходимости.

В современность Зигмунда Фрейда само понимание того, что действиями людей могут управлять скрытые от их сознания причины, стало шокирующим откровением. До исследований Фрейда таких терминов как «подсознательное» или «бессознательное» не было вовсе. И его наблюдения стали отправной точкой в развитии психоанализа – анализа человеческой психики с позиции движущих ею сил, а также причин, последствий и воздействия на последующую жизнь человека и состояние его нервно-психического здоровья опыта, полученного им в прошлом.

\textbf{Базовые идеи психоанализа}

Теория психоанализа зиждется на том утверждении Фрейда, что в психической (если удобнее – душевной) природе человека не может быть непоследовательности и перерывов. Любая мысль, любое желание и любой поступок всегда имеет свою причину, обусловленную сознательным или бессознательным намерением. События, имевшие место в прошлом, влияют на будущие. И даже если человек убеждён, что какие-либо его душевные переживания не имеют оснований, всегда присутствуют скрытые связи между одними событиями и другими.

Исходя из этого, Фрейд разделял психику человека на три отдельные области: область сознания, область предсознания и область бессознательного.

\begin{enumerate}
    \item К области бессознательного относятся бессознательные инстинкты, никогда не доступные сознанию. Сюда же можно отнести вытесненные из сознания мысли, чувства и переживания, которые воспринимаются сознанием человека как не имеющие права на существование, грязные или запрещённые. Область бессознательного не подчиняется временным рамкам. Например, какие-то воспоминания из детства, вдруг снова попав в сознание, будут такими же интенсивными, как и в момент своего появления.
    \item К области предсознания относится часть области бессознательного, способная в любой момент стать доступной для сознания.
    \item Область сознания включает в себя всё то, что осознаёт человек в каждый момент своей жизни.
\end{enumerate}

Основными действующими силами человеческой психики, согласно идеям Фрейда, являются именно инстинкты – напряжения, которые направляют человека к какой-либо цели. И эти инстинкты включают в себя два главенствующих:

\begin{enumerate}
    \item Либидо, являющееся энергией жизни
    \item Агрессивная энергия, являющаяся инстинктом смерти
\end{enumerate}

Психоанализ рассматривает, по большей части, либидо, в основе которого лежит сексуальная природа. Оно представляет собой живую энергию, характеристики которой (появление, количество, перемещение, распределение) могут истолковать любые психические расстройства и особенности поведения, мыслей и переживаний индивида.

Личность человека, согласно психоаналитической теории, представлена тремя структурами:
\begin{enumerate}
    \item Оно (Ид)
    \item Я (Эго)
    \item Сверх-Я (Супер-Эго)
\end{enumerate}

Оно (Ид) является всем изначально заложенным в человеке – наследственностью, инстинктами. На Ид никак не влияют законы логики. Его характеристики — это хаотичность и неорганизованность. Но Ид воздействует на Я и Сверх-Я. Причём, его воздействие безгранично.

Я (Эго) является той частью личности человека, которая находится в тесном контакте с окружающими его людьми. Эго берёт своё начало из Ид с того самого момента, когда ребёнок начинает осознавать себя как личность. Ид питает Эго, а Эго защищает его, словно оболочка. То, как взаимосвязаны Эго и Ид, легко отобразить на примере потребности в сексе: Ид могло бы осуществить удовлетворение этой потребности посредством прямого сексуального контакта, но Эго решает, когда, где и при каких условиях этот контакт может быть реализован. Эго способно перенаправлять или сдерживать Ид, тем самым являясь гарантом обеспечения физического и душевного здоровья человека, а также его безопасности.

Сверх-Я (Супер-Эго) произрастает из Эго, являясь хранилищем моральных устоев и законов, ограничений и запретов, которые накладываются на личность. Фрейд утверждал, что Сверх-Я выполняет три функции, коими являются:
\begin{enumerate}
    \item Функция совести
    \item Функция самонаблюдения
    \item Функция, формирующая идеалы
\end{enumerate}

Оно, Я и Сверх-Я необходимы для совместного достижения одной цели – поддержания равновесия между стремлением, ведущим к увеличению удовольствия, и опасностью, возникающей от неудовольствия.

Возникшая в Оно энергия отражается в Я, а Сверх-Я определяет границы Я. Учитывая то, что требования Оно, Сверх-Я и внешней реальности, к которой должен приспособиться человек, нередко являются противоречивыми, это неизбежно приводит к внутриличностным конфликтам. Решение же конфликтов внутри личности происходит посредством нескольких способов:
\begin{enumerate}
    \item Сновидения
    \item Сублимация
    \item Компенсация
    \item Блокировка механизмами защиты
\end{enumerate}

Сновидения могут быть отражением желаний, не реализованных в реальной жизни. Сновидения, которые повторяются, могут быть указателями на определённую потребность, которая не была реализована, и которая может служить помехой на пути свободного самовыражения человека и его психологического роста.

Сублимация является перенаправлением энергии либидо на цели, одобряемые обществом. Нередко такими целями выступает творческая, социальная или интеллектуальная деятельность. Сублимация есть форма успешной защиты, а сублимированная энергия создаёт то, что все мы привыкли называть словом «цивилизация».

Состояние тревожности, которое возникает от неудовлетворённого желания, есть возможность нейтрализовать через прямое обращение к проблеме. Так, энергия, которая не может найти выхода, будет направлена на преодоление препятствий, на уменьшение последствий этих препятствий и на компенсацию того, чего не хватает. В качестве примера можно привести идеальный слух, который развивается у слепых или слабовидящих людей. Человеческая психика способна поступить аналогичным образом: к примеру, у человека, страдающего недостатком способностей, но имеющего сильнейшее желание достичь успеха, может развиться непревзойдённая работоспособность или беспримерная напористость.

Однако бывают и такие ситуации, в которых появившееся напряжение может быть искажено или отвергнуто особыми защитными механизмами, такими как гиперкомпенсация, регрессия, проекция, изоляция, рационализация, отрицание, подавление и другими. Например, неразделённую или потерянную любовь можно подавить («Не помню никакой любви»), отвергнуть («Да любви и не было»), рационализировать («Те отношения были ошибкой»), изолировать («Мне не нужна любовь»), спроецировать, приписав другим свои чувства («Люди не умеют любить по-настоящему»), гиперкомпенсировать («Я предпочитаю свободные отношения») и т.д.

\textbf{Краткое резюме}

Психоанализ Зигмунда Фрейда – это величайшая попытка прийти к пониманию и описанию тех составляющих психической жизни человека, которые до Фрейда были непостижимыми. Самим же термином «психоанализ» в настоящее время называют:

\begin{enumerate}
    \item Научную дисциплину
    \item Комплекс мероприятий по исследованию психических процессов
    \item Методику лечения нарушений невротического характера
\end{enumerate}


Работа Фрейда и его психоанализ даже сегодня нередко критикуются, однако те понятия, которые он ввёл (Ид, Эго, Супер-Эго, механизмы защиты, сублимация, либидо) понимаются и применяются в наше время как учёными, так и просто образованными людьми. Психоанализ нашёл своё отражение во многих науках (социологии, педагогике, этнографии, антропологии и других), а также в искусстве, литературе и даже кинематографе.
