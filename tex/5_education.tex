\chapter{Образование}

\section{Образование}

\textit{Источник: \url{https://bit.ly/3mQJ8N7}}

\textbf{Важность образования.}
Невозможно переоценить важность образования в современном мире. Образование стало ведущей силой технологического прогресса и тем самым всего развития человечества.

\textbf{Виды образования в России.}
В нашей стране есть разные виды образования: дошкольное, начальное, среднее и высшее.
Дошкольное образование охватывает ясли и детские сады. Там за детьми присматривают профессиональные няни и воспитатели. Часто их обучают читать и считать.

В России есть разные виды школ. Все школы начинаются с начального образования. Оно продолжается до 5 класса. Большинство учеников ходят в средние школы, другие идут в лицеи, гимназии, специализированные школы. Если ученики успешно оканчивают среднюю школу, они получают аттестат о среднем образовании. Он дает им возможность поступить в университет или академию, которые являются учреждениями высшего образования.

Учреждения высшего образования сейчас готовят специалистов, магистров и докторов. Диплом о высшем образовании позволяет найти более хорошую работу.

\textbf{Виды образования в других странах.}
В разных странах системы образования различны. В Британии три ступени образования. У Британцев есть начальная школа, средняя школа и высшее образование. Последнее включает профессиональное и собственно высшее образование.
В США система образования очень децентрализована. Это означает, что каждый штат имеет свои законы об образовании. Как правило, там есть начальные школы (6-11 лет), средние школы (11-15 лет) и старшие школы (9-12 классы). Есть несколько способов продолжить образование: университеты, колледжи, местные колледжи, технические и профессиональные школы.


\section[Что мешает хорошо обучаться]{Ученые из России и Швеции выяснили, что мешает хорошо обучаться}

\textit{Источник: \url{https://trends.rbc.ru/trends/education/62a849179a7947b0cc4329a4?from=mainpage}}

Ученые из МГУ, Сколтеха и Стокгольмского университета выяснили, от чего зависит способность к обучению. Рассказываем, что нужно об этом знать

\textbf{Что происходит}

В организме белок синтезируется с помощью рибосом в ядре и цитоплазме, либо в митохондриях. Первый способ изучен хорошо, второй — недостаточно.

Два года назад ученые из МГУ открыли два фермента — белковых соединения, которые участвуют в сборке рибосом в митохондриях. Затем ученые из Стокгольмского университета выделили митохондрии с недостроенными рибосомами, определили их структуру и показали процесс их сборки. Выяснилось, что в этих клетках были неактивными ферменты метилтрансферазы.

Чтобы проверить влияние неактивных ферментов на организм, ученые из МГУ вырастили мышей с такими ферментами. Затем провели эксперимент: посадили животное в ящик с несколькими выходами, где только один ведет в домашнюю клетку, остальные — в тупик. Если нормальная мышь запоминает правильный выход и в следующие разы бежит именно туда, то мышь с неактивными ферментами не запоминает правильный путь и каждый раз ищет его заново.

Таким образом, пришли к следующему выводу. Если ферменты метилтрансфераза неактивны, то работа митохондрий нарушается, и в клетки перестает поступать энергия. Из-за этого мышцы становятся слабее, а интеллектуальные способности снижаются.

\textbf{Что это значит}

Ученые из МГУ уже исследовали нарушения в работе митохондрий. В мае 2022 года они пришли к выводу, что нарушение работы этих органелл приводит к старению. Это происходит из-за того, что с возрастом появляется все больше нарушений в работе митохондрий, и клетки не успевают их устранять. Чтобы уменьшить число поломок и замедлить старение, согласно исследованиям ученых, нужно соблюдать диету.

По словам руководителя отдела разработки ДНК-тестов в компании MyGenetics Валерия Полуновского, чем лучше работают митохондрии в организме человека, тем лучше функционирует тело и мозг. А чтобы число митохондрий в клетках стало больше, нужно заниматься спортом.

