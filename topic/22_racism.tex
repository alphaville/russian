% \chapter{Расизм}

\section{Расизм: обзор}

\textit{Сокращенная версия}

\textit{\url{https://encyclopedia.ushmm.org/content/ru/article/racism-abridged-article}}


Расисты – это люди, которые верят, что врожденные, унаследованные биологические особенности человека определяют его поведение. Расистская доктрина утверждает, что национальная идентичность определяется чистотой крови. Расизм, включая расовый антисемитизм (предубеждение против евреев или ненависть к ним, основанные на ложных биологических теориях), всегда был неотъемлемой частью германского национал-социализма (нацизма). Нацисты воспринимали всю историю человечества как историю биологически обусловленной борьбы между людьми разных рас. После того, как нацисты пришли к власти, они провели Нюрнбергские законы 1935 года, в которых давалось якобы биологическое определение еврейства. В соответствии с нацистской расовой теорией, германцы и прочие национальности северной Европы были «арийцами», представителями высшей расы. В годы Второй мировой войны нацистские врачи проводили псевдонаучные медицинские эксперименты, пытаясь найти физическое доказательство превосходства арийцев и неполноценности всех остальных. Несмотря на то, что в ходе этих экспериментов было убито огромное число неарийских заключенных, нацисты не смогли найти никаких доказательств своей теории о расовых различиях между людьми.

Нацистский расизм стал причиной массовых убийств беспрецедентного масштаба. Во время Второй мировой войны нацистское руководство распорядилось провести так называемую «этническую чистку» на оккупированных территориях Польши и Советского Союза. Эта политика предполагала полное истребление всех так называемых «враждебных рас», включая геноцид европейских евреев и уничтожение политических и государственных лидеров славянских народов. Нацистские расисты рассматривали умственные и физические расстройства как биологическую угрозу чистоте арийской расы. После тщательного планирования немецкие врачи начали убивать неполноценных пациентов в лечебных учреждениях Германии; эта операция носила эвфемистическое название «Программа эвтаназии».

\newpage
\section{Создание образа врага}

\textit{\url{https://encyclopedia.ushmm.org/content/ru/article/defining-the-enemy}}

\begin{wrapfigure}{r}{0.6\textwidth}
    \begin{fancyquotes}
        «Я стала национал-социалистом потому, что идея национального сообщества \ex{воодушевляла}{inspired} меня. Я никогда раньше не подозревала, что столь значительное число немцев недостойны участия в этом сообществе».\\[1em]

        \begin{flushright}
            --- Послевоенные мемуары немецкой женщины, принимавшей активное участие в молодежных программах нацистов.
        \end{flushright}
    \end{fancyquotes}
\end{wrapfigure}
Одним из важнейших факторов создания сплоченной группы является определение тех, чей доступ следует исключить. Нацистские пропагандисты способствовали реализации политики режима, публично идентифицируя исключаемые группы, насаждая ненависть, культивируя безразличие и объясняя населению, почему эти группы стали париями. Нацистская пропаганда играла ключевую роль в распространении мифа о «народном сообществе» среди немцев, стремившихся к единству, национальной гордости и величию страны и к отказу от жесткого классового расслоения общества, имевшего место в прошлом. Но другой, более зловещий, аспект нацистского мифа заключался в том, что далеко не всех немцев будут приветствовать в рядах этого нового сообщества. Пропаганда содействовала определению тех, кто будет исключен из нового сообщества, и оправдывала меры против «изгоев общества»: евреев, цыган (рома и синти), гомосексуалистов, политических диссидентов, немцев, отнесенных к генетически низшей ступени и сочтенных опасными для «здоровья нации» (т. е. людей с психическими заболеваниями или интеллектуальными или физическими недостатками, эпилептиков, людей с врожденной глухотой и слепотой, хронических алкоголиков, наркоманов и других).

\textbf{Пропаганда против евреев.} Эксплуатируя ранее известные образы и стереотипы, нацистские пропагандисты изображали евреев, как «чуждую расу», питавшуюся соками принимающей нации, отравляющую ее культуру, захватившую ее экономику и поработившую ее рабочих и крестьян. Это полное ненависти отображение, хотя и не было чем-то новым или же уникальным изобретением нацистской партии, теперь стало имиджем, поддерживаемым государственной машиной. Когда нацистский режим после 1933 года усилил контроль над прессой и издательствами, пропагандисты начали вести целенаправленную пропаганду, ориентированную на различные аудитории, включая тех немцев, которые не являлись нацистами и не читали партийные газеты. Публичная демонстрация антисемитизма в нацистской Германии принимала разнообразные формы – от плакатов и газет до кинофильмов и выступлений по радио. Пропагандисты предлагали более тонкие антисемитские высказывания и точки зрения для образованных немцев из среднего класса, которым претили грубые карикатуры. Университетские профессора и религиозные лидеры придавали антисемитским темам респектабельность, включая их в свои лекции и церковные службы.


\textbf{Другие изгои.} Евреи не были единственной группой, исключенной из «народного сообщества». Пропаганда содействовала определению тех, кто будет исключен из нового сообщества, и оправдывала меры против «изгоев общества»: включая евреев, цыган (рома), гомосексуалистов, членов секты Свидетелей Иеговы, немцев, отнесенных к генетически низшей ступени и сочтенных опасными для «здоровья нации» (людей с психическими заболеваниями или интеллектуальными или физическими недостатками, эпилептиков, людей с врожденной глухотой и слепотой, хронических алкоголиков, наркоманов и других).

\textbf{Идентификация, изоляция и исключение.}  Пропаганда также помогала заложить фундамент для значительных антисемитских законов, опубликованных в Нюрнберге 15 сентября 1935 года. Эти законы были изданы после волны антисемитских бесчинств, осуществлявшихся нетерпеливыми радикальными элементами нацистской партии. Закон об охране германской крови и германской чести запрещал браки и внебрачные половые связи между евреями и лицами с «немецкой или связанной с нею кровью», а закон о гражданстве рейха определял евреев, как «подданных» государства, т. е. им был присвоен статус второго сорта.

Эти законы распространялись примерно на 450 тысяч «чистых евреев» (у которых все четыре родителя их родителей были евреями и которые исповедуют иудейскую религию) и 250 тысяч других лиц (включая крещеных евреев и лиц с примесью еврейской крови, у которых не все родители были евреями) – в общей сложности эти группы составляли немногим более одного процента населения Германии. В течение нескольких месяцев до принятия Нюрнбергских законов нацистская партийная пресса активно агитировала немцев против загрязнения расы, и даже присутствие евреев в публичных плавательных бассейнах стало важной темой для обсуждения.

\textbf{Контроль за учреждениями культуры.}  Благодаря контролю над учреждениями культуры (такими как музеи), осуществлявшемуся Имперской палатой культуры, нацисты создали новые возможности для распространения антисемитской пропаганды. Наиболее заметной стала выставка под названием «Вечный жид», привлекшая 412 300 посетителей – более 5 тысяч человек в день – во время ее проведения в Немецком музее в Мюнхене с ноября 1937 по январь 1938 года. Выставке сопутствовали специальные постановки Баварского государственного театра, вторившие антисемитской направленности выставки. Нацисты также ассоциировали евреев с «дегенеративным искусством», ставшим предметом параллельной выставки в Мюнхене, которую посетили два миллиона человек.

В одном из наиболее известных эпизодов фильма евреи уподобляются крысам, переносящим заразу, заполоняющим континент и пожирающим драгоценные ресурсы. Вечный жид отличается не только своей грубой и отталкивающей характеризацией евреев, усугубляемой неприглядными кадрами работы ритуального еврейского забойщика скота, но также настойчивым подчеркиванием чуждой природы восточно-европейских евреев. В одном из эпизодов фильма показано как «типичные» бородатые польские евреи сбривают бороды и превращаются в евреев «западного типа». Такие сцены «снятия масок» служили для того, чтобы продемонстрировать немецкой аудитории отсутствие различий между евреями, живущими в восточно-европейских гетто, и теми, кто живет в немецких городах.

Фильм Вечный жид завершается печально известной речью Гитлера, произнесенной в рейхстаге 30 января 1939 года: «Если международные еврейские финансисты в Европе и за ее пределами сумеют еще раз втянуть народы в мировую войну, то результатом войны будет не ... триумф еврейства, а уничтожение еврейской расы в Европе». Как представляется, это выступление провозглашало радикальное решение «еврейского вопроса» в виде надвигавшегося «окончательного решения» и послужило предвестием эпохи массовых убийств.

\textbf{Пропаганда геноцида.} Хотя большинство немцев не одобряли актов насилия над евреями, тем не менее неприязнь к евреям, легко подогреваемая в трудные времена, ощущалась не только верными сторонниками нацистской партии. Большинство немцев, по меньшей мере пассивно, соглашались с дискриминацией против евреев. В докладе, подготовленном подпольным наблюдателем для руководителей германской социал-демократической партии в изгнании, отмечается: «Сегодня повсеместным является ощущение, что евреи относятся к другой расе».

В периоды, предшествовавшие принятию новых мер против евреев, проводились пропагандистские кампании, создававшие атмосферу терпимости к актам насилия против евреев или использовавшие такие акты насилия (как тщательно рассчитанные, так и происходившие спонтанно), чтобы побудить людей к пассивности и согласию с антисемитскими законами и указами в качестве средства для восстановления общественного правопорядка. Пропаганда, демонизировавшая евреев, также готовила население Германии (в контексте национального чрезвычайного положения) к более жестким мерам, таким как массовая депортация и, в конечном счете, геноцид.

\textbf{Нацистская пропаганда в оккупированной Польше.} Нацистский режим распространял свою пропаганду, уподобляющую евреев паразитам или болезням, не только в Германии. В оккупированной Польше нацистская пропаганда подкрепляла политику помещения евреев в гетто, рисуя их как разносчиков инфекции, которых требовалось поместить в карантин, а затем германские власти действительно добились реализации этого пророчества, резко ограничив доступ обитателей гетто к пище, воде и лекарствам. В немецких учебных фильмах, показывавшихся польским детям в школе, евреи изображались как разносчики вшей и тифа. Генерал-губернатор Варшавы Людвиг Фишер сообщал о распространении «3000 больших плакатов, 7000 малых плакатов и 500 000 листовок», информирующих польское население об опасности для их здоровья, создаваемой заключенными в гетто евреями. Такое запугивание несомненно препятствовало оказанию публичной помощи евреям в гетто на территории оккупированной Германией Польши.


