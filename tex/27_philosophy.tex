\chapter{Философия}

\section{Слава богу, мы атеисты}

\textit{Источник: \url{https://4brain.ru/blog/slava-bogu-my-ateisty/}}

Чем лучше у человека развито мышление, тем более полно и объемно он воспринимает реальность, тем глубже он проникает в суть вещей с точки зрения науки. В этом плане мышление атеистов очень даже полезно.

Мы не призываем вас становиться атеистами, но рекомендуем изучать свой мозг, развивать мышление и учиться смотреть на все с разных точек восприятия, изучая события через научные концепции.

Предлагаем начать изучение этого вопроса с наиболее актуальной темы для каждого читателя – с точки зрения навыков, которые помогают оставаться на плаву в быстро меняющемся мире.

\begin{fancyquotes}
    \textbf{Некоторые цифры}\\

    {\Huge 55\%}\\
    людей сегодня проживают в городах (в 2000 году проживало менее 50\%).\\[1em]

    {\Huge 89\%}\\
    людей сегодня имеют доступ к Интернету (в 1998 году имели лишь 41\%).\\[1em]

    {\Huge 95\%}\\
    людей сегодня пользуются мобильными телефонами (в 1996 году пользовались лишь 60%).

\end{fancyquotes}

Например, куда нужно идти учиться и какую профессию выбирать, чтобы быть уверенным, что «завтра» она не окажется неактуальной и ненужной? Какое давать образование своим детям и как их воспитывать, чтобы они смогли достичь успеха и стали полноценными членами общества, когда вырастут?

От ответов на эти вопросы зависит качество вашей жизни, и, если вы хотите именно ту жизнь, о которой мечтаете, вы должны быть гибкими, а для этого нужно начать овладевать новыми навыками, которые будут являться современными на протяжении хотя бы нескольких ближайших десятилетий и, вполне возможно, даже дольше.

\newpage
\section{Постправда: полезно знать и понимать}

\textit{Источник: \url{https://4brain.ru/blog/postpravda-polezno-znat-i-ponimat/}}

Шокирующая статистика – более, чем у 80\% жителей России отсутствует или слабо развито критическое мышление. И это может быть правдой, но официальных подтверждений этому нет. Это могло бы быть постправдой, если бы разоблачение было в конце статьи или вышло бы намного позже. И, вероятно, вы поверили в первое предложение, если его идея схожа с вашими мыслями. Скорее всего, вы даже не удивились всем этим обстоятельствам. Ведь мы все привыкли жить в эпоху постправды, ярких заголовков и цепляющих фактов. Выходит, люди перестали распознавать ложь? Или, наоборот, они научились жить со своей правдой?

\textbf{Постправда – это не ложь}

А что же это?

Некоторые переводчики говорят, что вернее будет называть ее не постправдой, а пост-истиной. Но это только предположения, а вот что точно известно, так это то, что постправда не противоположна истине.

Все дело в том, что мы живем в мире постправды – месте, где исчезли общие объективные стандарты истины. Универсальных критериев для определения правды нет. Как нет и вероятности, что люди примут аргумент с реальными фактами, нежели тот, который им больше подходит по личным убеждениям. Все зависит от эмоций, которые вызывает тот или иной аргумент у человека.

На данный момент специалисты выделяют два типа постправды:

\begin{enumerate}
    \item несоответствие поступков и слов (например, общественные или политические деятели предлагают идеи, а затем не выполняют их или совершают противоположные действия);
    \item игнорирование фактов (независимый эксперт может дать разоблачающую аргументацию, но это не остановит людей распространять фейки по этой теме).
\end{enumerate}

Что же на это повлияло? Корпорация RAND провела исследования и возложила вину за распад истины и ценности фактов на следующие факторы:

\begin{enumerate}
    \item увеличение количества и расширение социальных сетей;
    \item перегруженность образовательной системы, которая не успевает за изменениями в информационной экосистеме;
    \item повышенная политическая и социальная поляризация;
    \item как результат предыдущих пунктов, растущая тенденция людей создавать свою собственную субъективную социальную реальность [Д. Кавана, М. Д. Рич, 2018].
\end{enumerate}

Приставка «пост» означает не столько хронологическое состояние (истина обнародуется позже), сколько ее отсутствие, ее понижение до уровня, когда она становится нерелевантной и вторичной по отношению к эмоциональному обращению к глубоким обидам и чувству незащищенности.

\textbf{В постправде главное внимание, которое на нее направлено}

Как же она его зарабатывает?

С каждым днем доступной человеку информации становится все больше, но проверяют ее все реже. Параллельно с ростом фактов стирается грань между массовым и личным: статья профессора университета и пост друга в соцсети могут иметь одинаковую ценность и приниматься на веру безусловно.

Далее публичная сфера начинает дробиться на множество сообществ по признаку своих убеждений. Человек находит своих единомышленников, они подтверждают его размышления, а автоматическая персонализированная выдача новостей закрепляет веру в его правоту. И люди используют массмедиа для подпитки своих предубеждений.

Поэтому важнее объективной правды становится реальность индивидуального и коллективного страха, наравне с разочарованностью, унынием и яростью. Внимание привлекают факты, которые задевают худшие страхи или перевернутую реальность [L. Haddad, 2016].

Лжецы пытаются применить волшебные инструменты: они умышленно говорят вещи, которые, как они знают, не являются правдой, для эффекта. Они добавляют остроумные шутки и преднамеренные преувеличения. Сторонники коммуникации постправды наслаждаются недосказанными вещами. Их блеф рассчитан не только на привлечение внимания общественности, он одновременно скрывает от людского взора действительно важные и пугающие факты (например, растущая безработица, рост смертности, милитаризация демократии и т.п.)

Также от более глубоких тенденций и важных явлений отвлекает частое использование «горячих» (экстренных) новостей. Яркие картинки, кричащие заголовки, шокирующие данные – это все признаки новостей, борющихся за внимание. Они распространяются по Сети подобно вирусам. Их цель – дезориентировать и дестабилизировать людей, чтобы привести к нужной идее. Одна из первых вещей, которую хочет сделать серийный лжец, – это подорвать доверие к поставщикам фактов, проверяющим его ложь. Это можно назвать газлайтингом, который применяется более сильными людьми, чтобы разрушить способность других людей анализировать информацию и выносить суждения [Д. Кин, 2018].

И ведь проблема не в том, что людям все равно или их настолько искусно обманывают. Дело в том, что круг внимания среднестатистических людей к истине иррационально мал. Вот исследования, подтверждающие эту мысль (мы же не хотим быть голословными):

\begin{enumerate}
    \item Журнал Science опубликовал исследование фейковых новостей в Твиттере. В нем были обнаружены 126 000 лживых твитов за последние 10 лет, которые в общей сложности были ретвитнуты более 4,5 миллионов раз. Согласно анализу, ложные истории доходят до 1500 человек в среднем в шесть раз быстрее, чем правдивые истории [ Gold, 2018].
    \item В Стэнфорде провели исследование, в котором принимали участие опытные интернет-пользователи и фактчекеры ведущих новостных изданий, историки и студенты. Оказалось, что люди упускают важные подсказки о том, что может попытаться повлиять на их мнение, потому что они заимствовали способы чтения из печатного мира. В Интернете же действуют другие правила. В то время как исторкии и студенты-«цифровые аборигены» поверили адресу и внешнему виду сайта, фактчекеры отправились узнавать подробности о сайте вне его страниц. Но в реальном мире так делают только профессионалы [S. Wineburg, S. McGrew, 2018].
\end{enumerate}

Выходит, лица, применяющие постправду, пользуются эмоциональной привязкой людей и воздействуют на нее, понимая, что они, скорее всего, не смогут найти достоверную информацию. Или же она не будет для них убедительной. И виноватыми в этом получаются сами «жертвы».

\textbf{Постправда существовала всегда}

Но почему о ней заговорили только сейчас?

В 1992 году американский драматург Стив Тесич опубликовал в журнале The Nation эссе. В нем автор описал свои наблюдения, как после интенсивного возмущения различными действиями правительства американское общество затихало и забывало о причине своего недавнего переживания. «Мы быстро превращаемся в нацию такого типа, – писал драматург, – о которой могли только мечтать адепты тоталитаризма. До сих пор всем диктаторам приходилось прилагать серьезные усилия, чтобы заглушить в обществе голос правды. Сейчас мы своими действиями показываем, что в этом уже нет необходимости, что нам удалось сформировать в себе некий психический механизм, который способен избавлять нас от осознания правды, какой бы важной она ни была. Будучи свободными людьми, мы по своей собственной воле приняли фундаментальное решение о том, что хотим жить в некоем мире постправды» [S. Tesich, 1992].

С тех пор постправда использовалась как окказионализм в различных публикациях, пока в 2016 году издательство Oxford Dictionaries не назвало лексему «post-truth» «словом года», т.е. наиболее актуальным, востребованным, отражающим дух времени понятием. В том году термин был столь популярен в связи с проведением референдума о выходе Великобритании из Евросоюза. Большинство британцев были уверены, что членство в ЕС обходится стране дорого, хотя институт фискальных исследований не раз доказывал, что названная сумма неверна. Но, несмотря на все цифры и документы, люди верили в фейк.

Это привлекло внимание публицистов и общественных деятелей и в какой-то степени стало самосбывающимся пророчеством: термин стал употребляться повсеместно и стал едва ли не основной характеристикой сообщений в средствах массовой информации. А издательства и лингвисты тем временем пытались дать ему максимально точное и емкое определение. В конце концов в новое слово вошло значение целого комплекса социокультурных феноменов, касающихся искажения восприятия или отражения реальности в медиа, выступлениях политиков, коммуникации обычных людей [И. Тузовский, 2020].

Но хоть слово появилось относительно недавно, само явление существовало многие сотни лет. О политике постправды говорил еще Платон. В своем диалоге «Государство» он отмечал, что всеми знаниями должны обладать только те, кто управляет государством. Простым людям необязательно обладать всеми знаниями, ведь легче ввести аудиторию в заблуждение и передать ограниченную информацию под определенным углом, чем представить разные точки зрения.

Для Платона люди, которые определяют, что истинно и что ложно, – это те, кто получил образование в области математики, физики, астрономии и музыки – универсальных форм понимания мира, которые не зависят от конкретных материальных условий. Остальные же должны знать только то, что соответствует их образу жизни и опыту, и в этих знаниях они должны быть уверены. Им следует рассказывать о вещах, которые позволят им спокойно жить, как раньше. В этом Платон видел формулу стабильного общества [S.Fuller, 2017].

Человечество всегда существовало рука об руку с искажением реальности, будь то демократический или тоталитарный режим. Будь то расцвет цивилизации или самое ее начало. Люди всегда верили только в то, во что им хотелось верить.

\textbf{Побочным эффектом постправды стал не только упадок традиционных ценностей}

На что же еще влияет постправда?

В книге On truth Саймон Блэкберн написал о великом американском прагматике Чарльзе Сандерсе Пирсе, который считал, что сомнение – это настолько неудобное положение, что люди сделают почти все, что угодно, лишь бы ухватиться за веру или убеждение, устраняющее его [S. Blackburn, 2018].

Люди всегда ищут то, что заставляет их чувствовать себя хорошо. В области идей то, что заставляет человека чувствовать себя хорошо, – это то, что он уже думает. Это изъян в познании, который заставляет людей искать и принимать доказательства в пользу утверждений, в которые они уже верят, избегая или отвергая доказательства против.

Можно предположить, что люди говорят что-то, чтобы передать правдивую информацию. Однако язык все чаще используется в публичных разговорах для обозначения личной позиции или идентичности, независимо от того, истинны ли сделанные заявления. А в цифровом мире, поглощенном фейковыми новостями и токсичными социальными сетями, заговоры и слухи распространяются быстрее, чем когда-либо, и их все труднее опровергнуть.

Значит ли это, что люди понимают, что люди намеренно делают ошибки и не испытывают при этом угрызений совести? По словам Сьюзен Краусс Уитборн, врача и автора журнала Psychology Today, унижение определяется как эмоция, которую люди испытывают, когда их статус принижается перед другими. Вы можете чувствовать раздражение на себя, когда совершаете ошибку или не знаете ответа, но если другие не будут свидетелями этого, это все, что вы будете чувствовать.

Обычно людям нужен кто-то еще под рукой, чтобы чувствовать себя униженным из-за ошибок. Но сейчас, в цифровую эпоху, правильный ответ находится «на кончиках пальцев». Если его можно найти в Интернете, то вы больше не можете найти оправданий для своей ошибки. Поскольку люди осознают, как легко быть правыми и как трудно ошибиться в той или иной части информации, они чувствуют себя еще более огорченными, когда у них нет ответов. Поэтому они занимают какую-то позицию и стараются придерживаться ее постоянно, чтобы не выглядеть глупо [S. K. Whitbourne, 2014].

Для того, чтобы «раскачать» эмоции, приверженцы постправды «нападают» на популярные мнения и добавляют страх вместо фактов и доказательств. Помимо разрушения ценностей честности, этики и ответственности, медиа и различные деятели перестали работать на исцеление болезней общества. Вместо этого они предлагают невыносимые идеи вроде построения огромных стен, выдворения миллионов из страны и отмены политики предшественников, что зачастую поддерживается общественностью [L. Haddad, 2016].

Примечательно то, что постправда стала просачиваться и в другие сферы жизни человека. Например, студент Стэнфорда делится сомнениями о правильности подхода к их образованию: «Вместо того, чтобы назначать чтения, которые предлагают разные мнения, и способствовать созданию среды, которая позволяла бы вести оживленную, но цивилизованную беседу, мои учителя знакомили нас только с точками зрения, одобренными крайне левыми политическими силами. Некоторые даже откровенно объясняли, что предложение альтернатив будет способствовать угнетению, поскольку даже простое обсуждение расходящихся мнений считалось опасным и равносильным разжиганию ненависти в классе» [W. J. Winograd, 2018].

Таким образом, с помощью продавления субъективных оценок реальности, сжимания сложных вопросов до уровня лозунгов, приравнивания идей к общей идеологии общество делится на группы «свой» и «не свой». И те, кто не поддаются этому искажению, становятся изгоями.

\textbf{Иногда постправда – это всего лишь шутка}

Что, и такое бывает?

Как уже было указано выше, термин «постправда» объединил в себе различные феномены, и из-за этого понятие стало достаточно широким и сейчас постоянно пополняется новыми вариациями из массовой культуры. Ниже мы привели основные примеры злонамеренного использования постправды:

\begin{enumerate}
    \item \textit{Манипуляции со статистикой}. Стоит помнить, что статистика – это интерпретация, данные собирают и обрабатывают люди. И они сами выбирают, что считать, как считать, какими числами делиться и какими словами их описывать. Все зависит от идеи, к которой надо прийти.
    \item \textit{Относительность чисел}. Наряду с выбором, какое значение необходимо «подсветить» дополнительно, перед людьми стоит выбор, с чем сравнивать числа, чтобы они выглядели пугающе большими или маленькими.
    \item \textit{Махинации с опросами}. Любое мнение можно перекрутить и переиначить согласно преследуемым целям. Например, чтобы собрать информацию об оппоненте, господине N, всего лишь нужно задать людям вопрос: «Есть ли что-нибудь, с чем вы не согласны или чего вы не одобряете в словах и поступках N даже при условии, что вы его поддерживаете?» Так почти у всех найдутся хоть какие-нибудь претензии.
    \item \textit{Работа с альтернативными объяснениями}. У одного и того же явления может быть несколько объяснений. В том числе псевдообъяснения. Они опровергают устоявшиеся мнения и выводы и дают новые противоречащие гипотезы.
    \item \textit{Использование фактоидов и фейков}. Это означает выдавать за факт непроверенные сведения или чье-то «авторитетное» мнение, которое ничем не подтверждается. Для фейкньюс же не требуется никаких источников, достаточно выпустить их на «экспертной» страничке.
    \item \textit{Применение дезинформации}. Зачастую она не является прямой ложью, а лишь паразитирует на эмоциональных вопросах и ситуациях. Т.е. «разогревает» тему, которая всех волнует. А разогрев часто сопровождается непроверенной информацией. Дело в том, что изначально слово использовалось для обозначения маскировки и распространения ложной информации, с целью выявления места ее утечки.
\end{enumerate}

Как видите, вариантов введения в заблуждение немало, и, к сожалению, все они используются в массовой культуре ежедневно и приводят к плачевным последствиям. Однако есть и другие, не настолько вредоносные, виды искажения правды:

\begin{enumerate}
    \item \textit{Фанфейки} (не путать с фанфиками) – это намеренное создание новостных сообщений, основанных на несуществующих (часто мистифицированных) фактах с целью развлечения аудитории. Основным мотивом является инфотейнмент, для этого часто используются пранки и троллинг. Они не могут долгое время оставаться на позиции правды, т.к. развлекательный момент наступает именно в момент разоблачения.
    \item \textit{Кликбейт} – это использование заголовков, искажающих содержание ради усиления эмоциональной привлекательности публикации.
    \item \textit{«Хайп»} или \textit{информационный шум} (т.е. множество однородных сообщений), связанных с определенной темой, ситуацией, событием или персоной. На волне хайпа публикации могут быть пересказывающими основную новость или намеренно провокационными [И. Тузовский, 2020].
\end{enumerate}

Итак, с постправдой мы многократно встречаемся каждый день, она множится и распространяется во все сферы нашей жизни уже сейчас. А достижения в области искусственного интеллекта и машинного обучения, несомненно, сделают завтрашние фейковые новости еще более убедительными. Пока большая часть искаженной информации безобидна, но остальное может иметь серьезное влияние на эмоциональное состояние, работоспособность, мечты и желания.

\textbf{Не логикой единой}

А что еще поможет избежать влияния постправды?

Естественно, в первую очередь хочется предположить, что для борьбы с искаженной правдой необходимо заниматься фактчекингом и научиться искать истину с помощью критического мышления.

Несомненно, даже такие базовые навыки, как искать первоисточник информации или брать поисковые запросы в кавычки, чтобы поисковые системы искали запрос целиком, помогут избежать «кроличьих нор». Но постправда так глубоко вошла в нашу жизнь, что предпринимаемые действия должны быть намного системнее и шире.

Например, в свое время Ганди (индийский политический и общественный деятель) создал неологизм «сатьяграха», означающий ненасильственное прямое действие. Цель сатьяграхи состояла в том, чтобы пробудить совесть угнетателей и придать их жертвам чувство моральной свободы. Сатьяграха, буквально переводимая как «крепко держаться истины», обязывала протестующих «всегда сохранять непредвзятость и быть всегда готовыми обнаружить, что то, что мы считали истиной, в конце концов было неправдой». Ганди рано осознал, что любое общество живет в эпоху постправды. «Мы никогда не будем мыслить одинаково, и мы всегда будем видеть истину фрагментарно и под разными углами зрения», – писал он. Его позиция ненасильственного гражданского неповиновения привела к освобождению Индии от власти Британии [P. Mishra, 2018].

Важная часть борьбы с влиянием постправды – научиться выявлять ее, замечать собственные предубеждения в повседневном мышлении, а также признать преимущество наличия постправды в мире. Ведь на самом деле она дает нам возможность стать более чувствительными к мнениям других людей и рассматривать жизненный опыт как подлинное представление реальности с точки зрения другого человека. Люди относятся к чувствам, как к слабым вещам, которые по своей природе затуманивают суждения. Однако именно принятие чужого мнения и эмоций может привести к сокращению искажения истины, т.к. многомерность мнений будет признана большей ценностью [K.Colombo, 2018].

А пока общество только приходит к осознанию наличия постправды в повседневности, вы уже можете начать избавляться от ее влияния. Например, вы можете подписаться на Telegram-каналы с альтернативными новостями (но помните, что они тоже могут быть надуманными) или заглядывать на другие ресурсы, созданные специально для борьбы с постправдой (например, SIFT – Сиднейская инициатива за правду).

Также важно знать «врага» в лицо, для этого рекомендуем вам познакомиться с книгами по теме:

\begin{enumerate}
    \item Ли МакИнтайр, «Пост-истина».
    \item Ральф Киз, «Эпоха постправды: нечестность и обман в современной жизни».
    \item Дэниел Левитин, «Путеводитель по лжи: Критическое мышление в эпоху постправды».
    \item Андрей Мовчан, «Россия в эпоху постправды».
\end{enumerate}
В глобальном же смысле человечеству необходимо восстановить объективность и контроль над данными. Для этого важно требовать от СМИ более высоких стандартов работы, т.к. именно они являются первым источником ежедневных новостей. А также каждому отдельному человеку нужно отказаться от хайпа на подогреваемых темах, перестать реагировать на кликбейты и теории заговоров. Отсутствие внимания сделает провокацию менее эффективной.

В каждом вопросе необходима осознанность. Наверняка правда отдельной личности не будет совпадать с публичной, но значительная доля критического мышления позволит вовремя остановить себя и не отреагировать на новость так, как этого ждут. А уверенность в фактах и глубокое познание себя помогут остаться при своем мнении вне зависимости от внешнего давления.

