\chapter{Сказки}
\section{Сестрица Алёнушка и братец Иванушка}
% https://deti-online.com/skazki/russkie-narodnye-skazki/sestrica-alyonushka-i-bratec-ivanushka/
% https://www.youtube.com/watch?v=UDaOREoItE8
Жили-были стар\'{и}к да стар\'{у}ха, у них был\'{а} дочка Алёнушка да сын\'{о}к Иванушка. Старик со старухой умерли. Остались Алёнушка да Иванушка одни-одинёшеньки. Пошла Алёнушка на работу и братца с собой взяла. Идут они по д\'{а}льнему пут\'{и}, по шир\'{о}кому п\'{о}лю, и захотелось Иванушке пить.
%
\begin{dialogue}
    \item Сестр\'{и}ца Алёнушка, я пить хочу!
    \item Подожди, братец, дойдем до кол\'{о}дца.
\end{dialogue}
%
Шли-шли, -- солнце высоко, колодец далёко, жар \explainDetail{донимает}{донимать}{(colloq.) to bother, harass}, пот выступ\'{а}ет. Сто\'{и}т коровье коп\'{ы}тце\footnote{hoof (diminutive of коп\'{ы}то)} полн\'{о} водицы.
%
\begin{dialogue}
    \item Сестрица Алёнушка, хлебну\footnote{(colloq.) to drink} я из копытца!
    \item Не пей, братец, телёночком станешь!
\end{dialogue}
%
Братец послушался, пошли дальше. Солнце высоко, колодец далёко, жар донимает, пот выступает. Сто\'{и}т лошадиное копытце полно водицы.
%
\begin{dialogue}
    \item Сестрица Алёнушка, напьюсь я из копытца!
    \item Не пей, братец, жеребёночком станешь!
\end{dialogue}
%
\explainDetail{Вздохнул}{вздых\'{а}ть/вздохн\'{у}ть}{sigh} Иванушка, опять пошли дальше. Идут, идут, -- солнце высоко, колодец далёко, жар донимает, пот выступает. Сто\'{и}т к\'{о}зье копытце полно водицы. Иванушка говорит:
%
\begin{dialogue}
    \item  Сестрица Алёнушка, мочи нет: напьюсь я из копытца!
    \item  Не пей, братец, козлёночком станешь!
\end{dialogue}
%
Не послушался Иванушка и нап\'{и}лся из к\'{о}зьего копытца. Нап\'{и}лся и стал козлёночком\dots Зовёт Алёнушка братца, а вместо Иванушки бежит за ней беленький козлёночек. Залилась\footnote{flooded} Алёнушка слез\'{а}ми, села на стож\'{о}к -- плачет, а козлёночек в\'{о}зле неё скачет. В ту пору ехал мимо купец:
%
\begin{dialogue}
    \item О чём, красная девица, плачешь?
\end{dialogue}
Рассказала ему Алёнушка про свою беду. Купец ей и говорит:
\begin{dialogue}
    \item Под\'{и} за меня замуж. Я тебя наряж\'{у} в златосеребро, и козлёночек будет жить с нами.
\end{dialogue}
Алёнушка под\'{у}мала, под\'{у}мала и пошла за купца замуж. Стали они жить-поживать, и козлёночек с ними живёт, ест-пьёт с Алёнушкой из одной чашки. Один раз купца не было д\'{о}ма. \explainDetail{Откуда не возьм\'{и}сь}{откуда не возьм\'{и}сь}{out of the blue} прих\'{о}дит \explain{в\'{е}дьма}{witch}: стала под Алёнушкино окошко и такто ласково начал\'{а} звать её куп\'{а}ться на реку\footnote{произношение: н\'{а}реку}. Привела ведьма Алёнушку на реку. \explainDetail{Кинулась}{кид\'{а}ться/к\'{и}нуться}{to throw oneself, to fling oneself, to dash, to rush, } на неё, привязала Алёнушке на шею камень и бр\'{о}сила её в в\'{о}ду. А сам\'{а} оборот\'{и}лась Алёнушкой, \explainDetail{нарядилась}{наряж\'{а}ться/наряд\'{и}ться}{to dress as someone, to imitate} в её пл\'{а}тье и пришла в её хор\'{о}мы. Никто ведьму не \explain{распозн\'{а}л}{recognised}. Купец вернулся -- и тот не распознал.

Одному козлёночку всё было ведомо. Повесил он голову, не пьет, не ест. Утром и вечером ходит по бережку около воды и зовёт:
\begin{dialogue}
    \item Алёнушка, сестрица моя! Выплынь, выплынь на бережок\dots
\end{dialogue}

Узнала об этом ведьма и стала просить мужа \explain{зарежь}{slaughter} да зарежь козлёнка.
Купцу жалко было козлёночка, \explain{привык}{got used to + \textit{дат.}} он к нему. А ведьма так пристаёт, так упрашивает, --- делать н\'{е}чего, купец согласился:
%
\begin{dialogue}
    \item Ну, зарежь его\dots
\end{dialogue}
%
Велела ведьма разложить костры высокие, греть котлы чугунные, точить ножи булатные.
Козлёночек проведал, что ему недолго жить, и говорит названому отцу:
%
\begin{dialogue}
    \item Перед смертью пусти меня на речку сходить, водицы испить, кишочки прополоскать.
    \item Ну, сходи.
\end{dialogue}
%
Побежал козлёночек на речку, стал на берегу и жалобнёхонько закричал:
%
\begin{dialogue}
    \item   Алёнушка, сестрица моя! Выплынь, выплынь на бережок.
    Костры горят высокие,
    Котлы кипят чугунные,
    Ножи точат \explain{булатные} {Bulat is a type of steel alloy known in Russia from medieval times; it was regularly mentioned in Russian legends as the material of choice for cold steel. This type of steel was used by the armies of nomadic peoples. Bulat steel was the main type of steel used for swords in the armies of Genghis Khan.},
    Хотят меня зарезати!
\end{dialogue}
%
%
Алёнушка из реки ему отвечает:
%
\begin{dialogue}
    \item Ах, братец мой Иванушка! Тяжёл камень на дно тянет,
    Шёлкова трава ноги спутала,
    Желты пески на груди легли.
\end{dialogue}
%
%
А ведьма ищет козлёночка, не может найти и посылает \explainDetail{слуг\'{у}}{слуг\'{а}}{servant}:
\begin{dialogue}
    \item Пойди найди козлёнка, приведи его ко мне.
\end{dialogue}
% 
%
Пошёл слуга на реку и видит: по берегу бегает козлёночек и жалобнёшенько зовёт:
\begin{dialogue}
    \item Алёнушка, сестрица моя! Выплынь, выплынь на бережок.
    Костры горят высокие,
    Котлы кипят чугунные,
    Ножи точат булатные,
    Хотят меня зарезати!
\end{dialogue}
%
%
А из реки ему отвечают:
\begin{dialogue}
    \item Ах, братец мой Иванушка!
    Тяжёл камень на дно тянет,
    Шелкова трава ноги спутала,
    Желты пески на груди легли.
\end{dialogue}
%
Слуг\'{а} побежал домой и рассказал купцу про то, что слышал на речке. Собрали народ, пошли на реку, закинули сети шелковые и вытащили Алёнушку на берег. Сняли камень с шеи, окунули её в ключевую воду, одели её в нарядное платье. Алёнушка ожила и стала краше, чем была.

А козлёночек от радости три раза перекинулся через голову и обернулся мальчиком Иванушкой.

Ведьму привязали к лошадиному \explainDetail{хвосту}{хвост}{tail}, и пустили в чистое поле.

\section{Маша и медведь}
Жили-были дедушка да бабушка. Была у них внучка Машенька. Собрались раз подружки в лес -- по грибы да по ягоды. Пришли звать с собой и Машеньку.
\begin{dialogue}
    \item Дедушка, бабушка, -- говорит Машенька, -- отпустите меня в лес с подружками!
\end{dialogue}
Дедушка с бабушкой отвечают:
%
\begin{dialogue}
    \item Иди, только смотри от подружек не отставай -- не то заблудишься.
\end{dialogue}
Пришли девушки в лес, стали собирать грибы да ягоды. Вот Машенька -- деревце за деревце, кустик за кустик -- и ушла далеко-далеко от подружек.

Стала она аукаться, стала их звать. А подружки не слышат, не отзываются.
Ходила, ходила Машенька по лесу -- совсем заблудилась.
Пришла она в самую глушь, в самую чащу. Видит-стоит избушка. Постучала Машенька в дверь -- не отвечают. Толкнула она дверь, дверь и открылась.
Вошла Машенька в избушку, села у окна на лавочку.

Села и думает:

\begin{fancyquotes}
    «Кто же здесь живёт? Почему никого не видно?..» А в той избушке жил большущий медведь. Только его тогда дома не было: он по лесу ходил. Вернулся вечером медведь, увидел Машеньку, обрадовался.
\end{fancyquotes}
%
\begin{dialogue}
    \item Ага, -- говорит, -- теперь не отпущу тебя! Будешь у меня жить. Будешь печку топить, будешь кашу варить, меня кашей кормить.
\end{dialogue}

Потужила Маша, погоревала, да ничего не поделаешь. Стала она жить у медведя в избушке.

Медведь на целый день уйдёт в лес, а Машеньке наказывает никуда без него из избушки не выходить.
%
\begin{dialogue}
    \item А если уйдёшь, -- говорит, -- всё равно поймаю и тогда уж съем!
\end{dialogue}
Стала Машенька думать, как ей от медведя убежать. Кругом лес, в какую сторону идти -- не знает, спросить не у кого\dots

Думала она, думала и придумала.

Приходит раз медведь из лесу, а Машенька и говорит ему:
\begin{dialogue}
    \item Медведь, медведь, отпусти меня на денёк в деревню: я бабушке да дедушке гостинцев снесу.
    \item Нет, -- говорит медведь, -- ты в лесу заблудишься. Давай гостинцы, я их сам отнесу!
\end{dialogue}
А Машеньке того и надо!

Напекла она пирожков, достала большой-пребольшой короб и говорит медведю:

\begin{dialogue}
    \item Вот, смотри: я в короб положу пирожки, а ты отнеси их дедушке да бабушке. Да помни: короб по дороге не открывай, пирожки не вынимай. Я на дубок влезу, за тобой следить буду!
    \item Ладно, -- отвечает медведь, -- давай короб! Машенька говорит:
    \item Выйди на крылечко, посмотри, не идёт ли дождик! Только медведь вышел на крылечко, Машенька сейчас же залезла в короб, а на голову себе блюдо с пирожками поставила.
\end{dialogue}

Вернулся медведь, видит -- короб готов. Взвалил его на спину и пошёл в деревню.

Идёт медведь между ёлками, бредёт медведь между берёзками, в овражки спускается, на пригорки поднимается. Шёл-шёл, устал и говорит:

\begin{fancyquotes}
    Сяду на пенёк, Съем пирожок! А Машенька из короба:
    Вижу, вижу! Не садись на пенёк, Не ешь пирожок! Неси бабушке, Неси дедушке!
\end{fancyquotes}

\begin{dialogue}
    \item Ишь какая глазастая, -- говорит медведь, -- всё видит! Поднял он короб и пошёл дальше. Шёл-шёл, шёл-шёл, остановился, сел и говорит:
\end{dialogue}

\begin{fancyquotes}
    Сяду на пенёк, Съем пирожок! А Машенька из короба опять: Вижу, вижу! Не садись на пенёк, Не ешь пирожок! Неси бабушке, Неси дедушке!
\end{fancyquotes}

Удивился медведь:

\begin{dialogue}
    \item Вот какая хитрая! Высоко сидит, далеко глядит! Встал и пошёл скорее.
\end{dialogue}
Пришёл в деревню, нашёл дом, где дедушка с бабушкой жили, и давай изо всех сил стучать в ворота:
\begin{dialogue}
    \item Тук-тук-тук! Отпирайте, открывайте! Я вам от Машеньки гостинцев принёс.
\end{dialogue}
А собаки почуяли медведя и бросились на него. Со всех дворов бегут, лают.

Испугался медведь, поставил короб у ворот и пустился в лес без оглядки.

Вышли тут дедушка да бабушка к воротам. Видят- короб стоит.
\begin{dialogue}
    \item Что это в коробе? -- говорит бабушка.
\end{dialogue}
А дедушка поднял крышку, смотрит и глазам своим не верит: в коробе Машенька сидит -- живёхонька и здоровёхонька.

Обрадовались дедушка да бабушка. Стали Машеньку обнимать, целовать, умницей называть.
