\chapter{Общество}

\section{Чайлдфри: а почему бы и нет?}

\textit{Источник: \url{https://4brain.ru/blog/chajldfri-a-pochemu-by-i-net/}}

«Дети --- цветы жизни, но пусть они лучше растут в чужом \ed{цветнике}{цветник}{flower garden}». Знакомо? Подобные мысли посещают \explain{время от времени}{from time to time} любых родителей, потому что дети --- это постоянные \explain{хл\'{о}поты}{chores}, расходы и проблемы. Хлопоты, конечно, могут быть приятными, проблемы --- вполне решаемыми, однако от всего этого, так или иначе, устаёшь.

А отдохнув, шагаешь в новый день, ведя за руку своего капризного и невыспавшегося, но такого милого \ed{карапуза}{карапуз}{little fellow} в детский садик, радуясь первым школьным успехам своего первоклассника, готовясь к выпускному вместе со своим уже выросшим ребёнком, провожая сына в армию и выдавая дочку замуж. «А как иначе?» --- спросит кто-то. Да очень просто!

Наша сегодняшняя тема --- чайлдфри во всех его видах и \ed{проявлениях}{проявление}{manifestation}. И для начала --- небольшой исторический экскурс.

\textbf{Что такое чайлдфри: немного истории}

Термин «чайлдфри» появился по историческим меркам сравнительно недавно, однако точное его происхождение установить все равно не удалось. \ed{Предположительно}{предположительно}{presumably}, этот термин впервые \explain{вошёл в оборот}{entered into circulation} в 70-е годы 20 века в рамках деятельности Национальной организации для не-родителей США, которая \explain{н\'{ы}не}{now} уже не существует [Е. Селивирова, 2010].

Более широкую известность это понятие обрело в 90-е годы, а именно после того, как школьная учительница Лесли Лафэйетт создала интернет-сообщество под названием The Childfree Network, целью которого была борьба с разными видами дискриминации бездетных людей и семей [Е. Алексахина, 2011].

\ed{Уточним}{уточн\'{и}м}{let us clarify}, что 30 лет назад тема была не то чтобы под запретом, а, скажем так, вызывала непонимание в обществе, что приводило к различным эксцессам. Уточн\'{и}м также, что русское слово «чайлдфри» является прямым переводом английского childfree, буквально означающего «без детей» либо «свободный от детей».

Данным термином обозначают людей, сознательно выбравших путь отказа от родительства при том, что их физическое и финансовое состояние вполне позволяет иметь детей. Термин \explain{был внедрён}{was introduced} \explain{в противов\'{е}с}{in opposition} понятию «childless», что означает «бездетный» и обычно употребляется в контексте «не способный иметь детей».

Сегодня термин «childfree» является широко распространённым, \ed{повсеместным}{повсеместный}{ubiquitous} и \ed{общеупотребимым}{общеупотребимый}{commonly used}. Во всяком случае, большинство людей знает, что этот термин обозначает. Если сказать простыми словами, чайлдфри --- это когда м\'{о}гут, но не хот\'{я}т. Однако «внутри» понятия childfree существует множество разных градаций и \ed{оттенков}{оттенок}{shade; tone} смысла, для которых придуманы отдельные термины. Это стоит обсудить подр\'{о}бнее.

\textbf{Виды и типы чайлдфри}

Для родителей со ст\'{а}жем, «обвешанных» заботами о подрастающем поколении и живущих в непрерывном потоке мыслей, чем накормить, во что одеть, куда отправить учиться и на какие гроши жить дальше, после того как детей одели, обули, накормили и сдали деньги в школу на очередные «шторы», типы чайлдфри, возможно, не сл\'{и}шком интересны.

Однако если в вашей жизни существуют ещё какие-то заботы, кроме как о д\'{е}тях и о том, как дожить до зарплаты, разбираться в типах childfree желательно уже для того, чтобы не попадать в неловкие ситуации, пытаясь комментировать чей-то образ жизни и образ мыслей.

В статье «Чайлдфри без паники: социологический взгляд» разобраны основные тренды данного явления на реальных примерах из жизни [Е. Селивирова, 2010]. Мы слишком углубляться не будем, и ограничимся общей характеристикой основных типов.

Основные типы чайлдфри:
\begin{enumerate}
    \item Реджекторы, они же чайлдхейтеры --- люди, которые не любят детей и относятся с некоторой долей брезгливости к детям и самой идее беременности, грудного вскармливания, замены памперсов и прочим неизбежностям, связанным с появлением в семье ребенка.
    \item Откладыватели --- люди, которые откладывают идею завести детей «на потом», когда закончат вуз, найдут работу, решат материальные и жилищные проблемы, купят машину, посмотрят мир, поживут для себя и так до бесконечности.
    \item Аффексьонадо --- предпочитают свободу и отдают себе отчет, что дети свободу изрядно ограничивают. Поэтому делают сознательный выбор в пользу свободы, в том числе свободы от детей.
    \item Отказники --- длительно колеблются, взвешивая «за» и «против» появления детей в семье и чаще всего отказываются от планов по деторождению либо «пропустив» детородный возраст, когда здоровье позволяет завести детей, либо найдя еще 100500 причин, почему им это не нужно.
\end{enumerate}

Это основные типы childfree, но это еще не все. В современном мире существуют различные субкультуры, представители которых не объявляют об отказе от намерения завести детей, однако детей все равно не заводят. Это, если можно так выразиться, «вторичные» чайлдфри, когда отсутствие детей не цель или средство, а прямое следствие их образа жизни и/или исповедуемых ценностей.

Субкультуры и молодежные движения с высоким процентом чайлдфри:

\begin{enumerate}
    \item Кидалт --- дословно «взрослый ребенок». Термин произошел от английского kidult, где kid означает «ребенок» и adult означает «взрослый». Такие «взрослые дети» активно познают мир, имеют множество хобби и пока не готовы покупать игрушки кому-то еще, кроме как себе.
    \item Синглтон --- человек, который предпочитает жить один, потому что ему так удобнее. Один --- это значит, один, без жены, мужа и, соответственно, детей.
    \item Твиксер --- человек, «зависший» между двумя состояниями: статусом тинейджера и статусом взрослого. Как правило, живет с родителями, перебивается случайными заработками, поэтому семью содержать не на что, да и привести потенциального партнера некуда. Термин распространен в США.
    \item Фурита --- примерно то же, что твиксер, но в Японии. Чаще употребляется применительно к молодым людям, решившим не поступать в университет и не получать высшее образование, а потому мало зарабатывающим и не могущим позволить себе семью и детей.
    \item Хикикомори, они же хикки --- в переводе с японского это означает «пребывание в уединении» и предполагает высокую степень добровольной социальной изоляции (не путать с карантином и принудительной самоизоляцией). Уединенный образ жизни мало способствует новым романтическим отношениям, созданию семьи и появлению детей.
    \item Поколение сатори --- в переводе с японского, это поколение, свободное от материальных желаний, которое довольствуется малым и не стремится зарабатывать деньги. Разумеется, такой образ жизни мало пригоден для семьи и вступает в противоречие с законным требованием законного партнера заботиться о материальном достатке семьи и содержании детей.
    \item Поколение NEET --- молодые люди, которые не работают и не учатся, поэтому с высокой степенью вероятности вряд ли дозреют до серьезных взрослых отношений и готовности нести ответственность за семью. Термин появился в Великобритании, обрел определенную популярность в странах Латинской Америки и… правильно --- в Японии!
\end{enumerate}

Как видим, немало молодежных течений, ведущих к чайлдфри, локализуется в Японии. Это не значит, что японцы в меньшей степени любят детей или в меньшей степени стремятся работать, чем другие народы. Причина, скорее, в том, что японцы более педантичны и склонны к детализации, поэтому именно там зародилось множество терминов, описывающих разные оттенки чайлдфри. В прочих странах обычно довольствуются либо общим определением чайлдфри, либо же используют один из уже придуманных кем-то терминов.

На самом деле, можно найти множество других терминов, названий и обозначений для разных вариантов градаций чайлдфри. Не будем приводить их все, потому что все они являются лишь вариациями вышеописанных трендов. А вот откуда взялся общий тренд чайлдфри и субкультуры, сторонники которых, скорее всего, так и не познают радость материнства и отцовства? Давайте разбираться.

\textbf{Истоки и причины движения чайлдфри}

В этой статье мы уже упоминали о Национальной организации для не-родителей США, которая ныне уже не существует [Е. Селивирова, 2010]. Можно ли сегодня сказать, что чайлдфри --- это общественное движение? Многие психологи и социологи полагают, что как раз сегодня есть все основания говорить именно о движении чайлдфри [Д. Клэйн, 2019].

Во-первых, по причине все растущей популярности данной идеи. Во-вторых, по причине растущего общественного интереса к этому явлению. И, наконец, потому что сегодня в процентном отношении реально стало больше людей, не желающих иметь детей и принимающих меры к тому, чтобы дети не появились.

Ученые называют разные статистические данные, что обусловлено разной методикой подсчета. Так, по данным Левада-центра в России порядка 2\% людей не хотят иметь детей, и примерно 9\% ожидают, что детей у них и не будет по разным причинам [Левада-центр, 2019]. Что касается конкретных данных по чайлдфри в России, десяток лет тому назад их насчитали чуть более трех с половиной тысяч [Е. Алексахина, 2011].

В США, по ситуации на 2014 год, порядка 15\% женщин в возрасте 40-44 лет так и не завели ни одного ребенка [Pew Research Center, 2015]. Это общее число без разделения на childfree (когда могут, но не хотят) и childless (когда хотят, но не могут).

В Европе к 2010 году наблюдался рост числа людей, не имеющих детей, относительно данных за 90-е годы 20 столетия. Так, если в 90-е таковых было меньше 10\%, в 2010 году практически во всей Западной Европе показатель уверено превысил отметку в 12\%. Больше всего childfree + childless в Испании (21.6\%), Австрии (21.54\%), Англии (20\%), Финляндии (19.89\%) и Ирландии (19\%) [OECD, 2010].

Более свежие исследования указывают на рост количества чайлдфри в обществе, хотя исследователи по-прежнему сталкиваются с некоторыми трудностями сепарации и разделения childfree и childless в ходе исследовательских программ [J. Neal, 2021].

Связаны эти трудности преимущественно с тем, что общество по-прежнему с трудом понимает людей, добровольно отказавшихся от продолжения рода. А ввиду того, что феномен группового подкрепления, так или иначе, заставляет приспосабливаться к общественному мнению, многие чайлдфри не желают себя идентифицировать именно таким образом. Нередки случаи, когда чайлдфри выдают себя за бесплодных и утверждают, что никакое лечение и никакие репродуктивные технологии им не помогают.

Тем не менее, многие вполне готовы идентифицировать себя именно как чайлдфри, и даже могут внятно сформулировать, почему они выбрали такой жизненный путь. К слову, если воздержаться от путанных методик, а просто на условиях анонимности «в лоб» спросить, хочет ли человек детей, складывается вполне определенная картина. Так, в ходе одного из недавних опросов выяснилось, что порядка 27\% современных молодых людей не желают иметь детей [А. Салькова, 2021]. Во всяком случае, на момент проведения опроса.

Немало интересного в ходе научных исследований выявил наш российский социолог Илья Ломакин из Лаборатории сравнительных социальных исследований НИУ ВШЭ [О. Соболевская, 2020].

Основные причины чайлдфри:
\begin{enumerate}
    \item Нежелание брать на себя ответственность за детей и иметь лишние проблемы в жизни.
    \item Стремление к свободе в распоряжении временем, денежными средствами, и сознательный выбор в пользу свободы, в том числе свободы от детей.
    \item Представление о детях как преграде на пути к самодостаточности, самореализации, карьере и т.д.
    \item Негативное отношение, брезгливость или отвращение к маленьким детям. Обычно имеет физическую биологическую основу --- кто-то не любит мышей, жуков и пауков, а кто-то --- детей.
    \item Боязнь необратимых последствий, потому что в случае, если родительство не принесет удовлетворения или принесет разочарование, «отыграть назад» эту ситуацию уже не получится.
\end{enumerate}

Это основные причины, почему люди становятся чайлдфри. Гораздо реже исследователи указывают на такие причины, как финансовые проблемы или отсутствие официального супруга, постоянного партнера или кого-то, с кем можно было бы разделить радость материнства или отцовства.

Возможно, в этом что-то есть, потому что по-настоящему «заточенные» на материнство женщины даже в отсутствие мужа часто рожают «для себя». А возможными материальными трудностями с учетом нашего менталитета молодежь зачастую «не заморачивается».

Как говорится, «дал Господь зайку --- подаст и лужайку», так что многие молодые люди готовы переложить материальные хлопоты на плечи государства, родителей, муниципальной власти, различных благотворительных организаций. Не все, конечно, но многие.

Совсем небольшой процент объясняет свое нежелание заводить детей глобальными причинами: глобальным потеплением, глобальным изменением климата, необходимостью бороться с перенаселением планеты и т.д. Правда ли они так думают или это отличный лайфхак, чтобы защититься от обвинений в эгоизме, однозначно сказать сложно.

К слову, желающих обвинить, переубедить и перевоспитать представителей чайлдфри по сей день много. Достаточно заглянуть на любой форум чайлдфри в Интернете, чтобы увидеть, сколько там комментаторов, не имеющих никакого отношения к чайлдфри, но желающих доказать этим людям, что они круто неправы. Почему? Давайте обсудим и это.

\textbf{Причины неприятия чайлдфри}

При том, что количество чайлдфри неуклонно растет, и число приверженцев этой идеологии становится больше, их все равно заметно меньше половины населения. А значит, в количественном плане они проигрывают, и в возможности отстоять свою позицию тоже.

Кстати, некоторые ученые предлагают идентифицировать чайлдфри на основе этой идентичности, которую люди принимают и проговаривают. Часть ученых полагает, что чайлдфри предполагает «социально-политическую мобилизацию», публичное отстаивание своих прав и убеждений [О. Соболевская, 2020].

Как бы там ни было, в меньшинстве отстаивать свои взгляды всегда сложнее, чем примкнув к большинству. Выше мы уже упоминали, что некоторые childfree, не желая навлекать на себя гнев общественного мнения, предпочитают идентифицировать себя как childless, неспособных завести детей по медицинским основаниям.

Так или иначе, сегодня масштабы неприятия чайлдфри такие, что этому даже посвящают специальные исследования. Например, «Кто такие чайлдфри и как им живется в Украине, Азербайджане и Центральной Азии» [Т. Ярмощук, Н. Мусави, А. Сафарзода, 2021].

И даже в научном мире, который, казалось бы, должен быть образцом беспристрастности, термин «чайлдфри» употребляется в заведомо негативном контексте. Чего только стоят заголовки наподобие «Коммуникативные стратегии в текстах, репрезентирующих идеологию childfree: на грани экстремизма» [Ю. Антонова, 2013].

Это при том, что, как мы помним, в 140-миллионной России, по ситуации на 2011 год, насчитали аж 3500 чайлдфри [Е. Алексахина, 2011]! Это же как нужно бояться новых веяний и любых перемен, чтобы все не слишком понятное и привычное объявлять экстремизмом?!

Ситуация пока что меняется медленно, о чем свидетельствует интервью главы ВЦИОМа Валерия Федорова под говорящим само за себя названием «К чайлдфри россияне относятся негативно» [Д. Филиппова, 2017].

Демократический Запад традиционно более терпимо относится ко всему новому, а научный мир готов изучать новые тренды без налета предвзятости. Интерес к теме чайлдфри наблюдается едва ли не с момента появления данного феномена.

Так, уже в конце 70-х годов 20 века канадская исследовательница Джейн Виверс исследовала семьи, сознательно отказавшиеся от деторождения, на предмет их мотивов, и подытожила свои наблюдения в монографии Childless By Choice («Бездетный по выбору») [J. Veevers, 1980].

Отдельное исследование посвящено изменениям в обществе, приведшим к росту идей чайлдфри, и исследованию отношения общества к новому на тот момент явлению. Результаты подытожены в книге Continuity and Change in Marriage and Family («Преемственность и перемены в браке и семье») [J. Veevers, 1990].

О причинах чайлдфри как явления мы уже поговорили. Поэтому остановимся на причинах все еще господствующего неприятия данного явления.

Почему общество настороженно относится к чайлдфри:
\begin{enumerate}
    \item Консерватизм --- большинство людей тяжело принимает любые перемены и изменения в социуме, в принципе.
    \item Традиции --- многолетняя пропаганда семейных ценностей принесла определенные плоды, и для многих дети являются самостоятельной ценностью независимо от того, насколько взрослые способны обеспечить детей, сделать их счастливыми, уделить им должное внимание.
    \item Конформизм --- это стремление наличествует на уровне инстинкта, и люди присоединяются к большинству даже тогда, когда это не сулит никаких выгод, а пребывание в меньшинстве не грозит никакой опасностью. Например, в лабораторных условиях, в которых проходили знаменитые эксперименты Аша по исследованию конформности.
    \item Природная агрессивность --- большинство людей агрессивно воспринимает все, что выходит за рамки их восприятия, известных и знакомых им явлений, в том числе людей, способных мыслить и действовать нестандартно.
    \item Теория «стакана воды» --- в социуме превалирует убеждение, что забота о пожилых людях --- это обязанность их детей, поэтому дети являются необходимым элементом нормальной семьи и успешной старости.
    \item Подспудная боязнь нехватки ресурсов --- заботу об одиноких стариках берет на себя государство, поэтому у остальных граждан есть заведомо негативное отношение к «халявщикам» и боязнь того, что лишняя нагрузка на систему социального обеспечения ударит по карману представителей традиционных семейных ценностей, всю жизнь платящих налоги и при этом решающих все свои проблемы самостоятельно.
    \item Зависть --- уже доказано, что большинство чайлдфри живут ничуть не менее, а зачастую и более счастливо, чем семьи с детьми. Получается, что к счастью можно прийти с меньшими хлопотами и энергозатратами, но не все это вовремя поняли.
\end{enumerate}

Да, на сегодняшний день есть немало подтвержденных случаев, когда люди, сознательно отказавшиеся от родительства много лет тому назад, полностью довольны жизнью и никак не сожалеют о своем выборе. Так, психологи Мичиганского государственного университета опросили порядка тысячи взрослых людей с целью выявить корреляцию между наличием или отсутствием детей и степенью удовлетворенности жизнью [C. Brooks, 2021].

Для этого использовалась так называемая «Шкала удовлетворенности жизнью». Выяснилось, что значимой разницы в показателях между чайлдфри и сторонниками традиционных семейных ценностей не наблюдается. Более того, оказалось, что чайлдфри в целом более либеральны и толерантны по отношению к окружающим.

Так или иначе, новые веяния прокладывают себе путь, и чайлдфри становится все больше, в том числе на высшем политическом уровне. В демократическом обществе политикум репрезентует общество, его настроения и тренды, в том числе в части идеологии чайлдфри. Это ни хорошо, ни плохо --- это так.

Как водится, четко разделить childfree и childless не всегда представляется возможным. Однако самого факта бездетности известного политика вполне достаточно, чтобы чайлдфри истолковали это как потенциальную поддержку своей позиции.

Вот только некоторые из известных политиков, которые так и не завели своих детей по разным причинам:
\begin{enumerate}
    \item Эммануэль Макрон, действующий президент Франции.
    \item Марк Рютте, глава правительства Нидерландов.
    \item Ангела Меркель, бывший канцлер Германии.
    \item Тереза Мэй, бывший премьер-министр Великобритании.
    \item Стефан Левен, бывший премьер-министр Швеции.
    \item Паоло Джентилони, бывший глава правительства Италии.
\end{enumerate}

Большинство «бывших» отошли от дел в силу преклонного возраста и прочих обстоятельств, однако достаточно длительное их пребывание на высших государственных должностях однозначно способствовало более толерантному принятию обществом идей чайлдфри.

Тема чайлдфри суверенно занимает свое место в литературе и искусстве. В 2020 году увидела свет книга, которую написала писательница Тала Тоцкая «Чайлдфри» (читать онлайн по ссылке) [Т. Тоцка, 2020].

Еще раньше современные музыкальные исполнители Noize MC и Монеточка порадовали поклонников клипом на свою песню «Чайлдфри»: \url{https://youtu.be/_l0LVFRuHMk}

Возможно, общими усилиями со временем наше общество станет более терпимым и поймет, что частная жизнь --- это личное право и личный выбор каждого. И это касается не только темы рождения детей.

