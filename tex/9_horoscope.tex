\chapter{Гороскоп и другие глупости}

\section{Согласно астрологам: каким трем знакам зодиака особенно повезет на этой неделе?}

\textit{Кто из знаков-храбрецов способен противостоять текущим астрологическим обстоятельствам и готов сам «переписать» судьбу, если она его не устраивает? Звезды вам благоволят!}

\textit{Источник: \url{https://www.elle.ru}}

\textbf{Телец}

Даже несмотря на сложное ретроградное время, именно Тельцы, согласно гороскопу немецких астрологов ELLE, на этой неделе смогут «воспарить» над многими тревожными событиями. Причина в том, что планета любви, Венера, в данный момент благоволит представителям этого знака и окутывает их, будто мягким облаком розовой ваты. Одинокие люди могут обрести большое счастье — особенно в повседневной жизни, держите глаза широко открытыми.

\textbf{Рак}

Если в отношении других знаков зодиака ретроградный Меркурий лишь вносит неразбериху во все коммуникации, то Раки — единственные, кто на этой неделе получат отличного партнера для переговоров. Наберитесь смелости и сформулируйте свои цели и требования. У вас все получится! Но будьте осторожны: даже если вы можете многого добиться на этой неделе, не забывайте и о других аспектах. Начните думать о том, что вы готовы дать взамен.

\textbf{Козерог}

Рожденные под знаком зодиака Козерог на этой неделе точно знают, как правильно использовать наработанное контакты. И даже ретроградный Меркурий сейчас выступает на вашей стороне — вам без особых усилий удастся извлечь максимум пользы для себя и близких из любой ситуации. Ищите союзников и доверяйте их профессионализму. Впереди неделя, которая может многое изменить навсегда! «Мечта, о которой вы мечтаете в одиночестве, — это всего лишь мечта. Мечта, о которой вы мечтаете вместе, — это реальность».



\section{Откуда взять энергию?}
Отсутствие энергии --- это первый признак приближающихся несчастий и болезней. В Аюрведе говорится, что если человек продвигается в духовной жизни, то это должно быть видно по двум признакам:

\begin{enumerate}[noitemsep]
    \item Человек с каждым днем становится все счастливей и счастливей.
    \item Его отношения с другими людьми улучшаются. Когда мы получаем тонкую энергию...

          Тонкую энергию мы получаем когда:
          \begin{itemize}[noitemsep, label=+]
              \item голодаем,
              \item \explainDetail{выполняем}{выполнять/выполнить}{to perform} дыхательные упражнения,
              \item \explain{уединяемся}{we stay alone},
              \item даем \explain{обет}{vow} молчания, на какое-то время.
              \item гуляем (или просто находимся) по берегу моря, по гор\'{а}м, \explain{созерцаем}{contemplate} красивые пейзажи природы,
              \item занимаемся \explainDetail{бескорыстно}{бескорыстный}{unselfish} творчеством,
              \item \explain{восхваляем}{praise} \explain{достойную}{worthy} личность, за его возвышенные качества и \explain{поступки}{deeds},
              \item смеемся, \explain{радуемся}{rejoice}, улыбаемся от души,
              \item бескорыстно кому-то помогаем,
              \item проявляем \explain{скромность}{modesty},
              \item молимся \explain{перед}{before + instr.} едой,
              \item едим продукты полные \textit{праной} (жизненной энергией) --- натуральные \explain{злаки}{cereals}, каши, \explain{топлённое масло}{ghee}, мед, фрукты, овощи,
              \item спим с 9-10 вечера, до двух часов ночи (в другое время нервная система не отдыхает, \explain{сколько бы}{as much as we may sleep} мы не спали).
              \item получаем \explain{сеанс}{session} хорошего массажа, от гармоничной личности. Или делаем самомассаж.
              \item обливаемся холодной водой, особенно по утрам и наиболее сильный эффект если мы при этом стоим босиком на земле.
              \item \explain{жертвуем}{sacrifice} своим временем, деньгами...
              \item видим за всем божественную \explainDetail{в\'{о}лю}{воля}{will}.
          \end{itemize}
\end{enumerate}


\textbf{Когда мы теряем энергию...}
К потери энергии приводят:
\begin{itemize}[noitemsep, label=--]
    \item уныние, недовольство судьбой, сожаление о прошлом и страх, неприятие будущего,
    \item  постановка и преследование эгоистичных целей,
    \item бесцельное существование,
    \item обиды (обида)
    \item переедание,
    \item бесконтрольное блуждание ума, неумение сконцентрироваться.
    \item когда мы ед\'{и}м жаренную или старую пищу, пищу приготовленную человеком в гневе или испытывающем другие отрицательные эмоции, при использовании микроволновой печи, продукты, содержащие консерванты, химические добавки, выращенные в искусственных условиях, с использованием химических \explainDetail{удобрений}{удобрение}{fertilizer},
    \item поедание пищи лишенной праны --- кофе, черный чай, белый сахар, белая мука, мясо, алкоголь,
    \item еда в спешке и на ходу,
    \item курение,
    \item \explain{пустые разговоры}{(lit.) void discussions},
    \item неправильное дыхание, например, слишком \explainDetail{ч\'{а}стое}{частый}{frequent} и глубокое,
    \item нахождение под прямыми \explainDetail{луч\'{а}ми}{луч}{ray} Солнца, с 12 до 4 дня, особенно в пустыне,
    \item беспорядочные \explainDetail{половые}{полов\'{о}й}{sexual} связи, секс без любви к партнёру,
    \item \explain{излишний}{unnecessary} сон, сон после 7 \'{у}тра, недостаток сна,
    \item \explain{напряж\'{е}ние}{tension; stress} ум\'{а} и т\'{е}ла,
    \item \explain{\'{а}лчность}{greed} и \explain{жадность}{stinginess}.
\end{itemize}

Восточная психология на 50\% состоит из пранаямы --- теории и практики определенных дыхательных техник, которые позволяют человеку быть всегда наполненным жизненной силой (Праной).

Как утверждают современные просветленные учителя йоги набраться праны мы можем через:
\begin{enumerate}
    \item \textbf{Элемент земли.} питаясь натуральной пищей, жить на природе, созерцать деревья, ходить босиком по земле. Недавно я общался с очень известным аюрведическим доктором, \explainDetail{защитившему}{защитить, защитивший (past act.)}{who defends} диссертацию по медицине, он утверждал, что если человек начинает жить на природе, \explain{вдали от}{away from} больших городов, которые \explain{вынуждают}{necessitate} ездить в метро, ходить по асфальту, то у такого человека быстро восстанавливается иммунитет и он начинает жить здоровой жизнью.

    \item \textbf{Элемент воды.} пить воду из колодцев или \explainDetail{ручьев}{(sing.) ручей, ручья, ручью, ручей, ручьём, ручье, (plur.) ручьи, ручьёв, ручьям, ручьи, ручьям, ручьями, ручьях}{brook; creek}. Плавать в реке или море. \explainDetail{Избегать}{избегать}{to avoid} пить \explain{кофеиносодержащие}{containing caffeine} напитки, алкоголь и соду.

    \item \textbf{Элемент огня.} нахождение на Солнце и употребление пищи содержащей Солнечный свет.

    \item  \textbf{Элемент воздуха.} это самый важный элемент получения Праны, через вдыхание чистого воздуха, особенно в горах, в лесу и на берегу моря. Курение и нахождение в местах большого \explain{скопления}{?} людей, лишает человека праны.

    \item \textbf{Элемент эфира.} культивируя \explain{позитивное}{положительный} мышление, \explainDetail{доброт\'{у}}{доброт\'{а}}{kindness}, хорошее настроение.
\end{enumerate}


И этот \explain{уровень}{level} считается базовым.

Ибо даже, если человек живет на природе и правильно питается, но при этом ходит раздражённый и злой, то наоборот, \explain{изл\'{и}шек}{surplus} Праны еще быстрее разрушит его.
С другой стороны гармоничный человек, то есть добродушный,
\explain{бесстрашный}{fearless}, может довольно долго протянуть в городе,
если он вынужден там жить.
Но даже такому человеку н\'{у}жно следить за питанием
и периодически «вырываться» на природу.

У нас каждую секунду есть выбор --- светить миру, приносить своей жизнью благо и счастье окружающим, улыбаться, \explainDetail{заб\'{о}титься}{(по)заб\'{о}титься (заб\'{о}чусь, заб\'{о}{}тишься, заб\'{о}- тятся)}{to care about} о других, служить бескорыстно, жертвовать, сдерживать низшие \explain{побуждения}{drives (сдерживать низшие побуждения: to control/restrain the lower drives)}, видеть в каждом человеке Учителя, в каждой ситуации видеть Божественное \explain{провидение}{providence}, которое создало эту ситуацию \explain{для того что бы}{so that; in order to} нас чему то научить, благодарить ...

Либо предъявлять претензии, \explain{обижаться}{to take offence}, \explain{жаловаться}{to complain}, \explain{завидовать}{to envy}, ходить с клинообразным выражением лица, погрузиться в свои проблемы, в зарабатывание денег для того что бы потратить их на \explain{удовлетворение}{satisfaction} чувств, \explain{проявлять}{to show; to demonstrate} агрессию.

В этом случае \explain{в независимости от того}{regardless of} сколько у человека денег, он будет несчастный и мрачный. И с каждым днем энергии будет все меньше и меньше. И для того что бы её где-то взять нужны будут искусственные стимуляторы: кофе, сигареты, алкоголь, ночные клубы, выяснение отношений с кем-то. Все это дает вначале подъём, но в итоге приводит к полному разрушению..

Простой регулярный вопрос себе: я свечу миру или поглощаю свет? Может быстро изменить ход наших мыслей и следовательно поступков. И быстро превратить нашу жизнь в красивое яркое \explain{сияние}{shining}, полное любви. И тогда вопросы, где взять энергию уже не \explain{возникают}{emerge}.

\begin{flushright}
    \it Р. Блект
\end{flushright}


Каждый раз, когда кто-то причинил тебе боль, не спеши гневаться. Эта \explain{боль}{(ж) pain; ache (e.g., головн\'{а}я боль: headache)} выпр\'{о}шена тобой у Вселенной. \explainDetail{Неосознанно}{неосознанно}{unconsciously}, ты сам \explain{привлёк}{past tense of привлечь} её.

Причиняющий боль - всего лишь \explain{кукла на верёвочках}{doll on a string (puppet)}, что управляема тобой...

``Будь осторожен с желаниями'' --- слова Вечности. Что несут они в себе?

Когда ты \explain{жаждешь}{to crave for} \explainDetail{исполнения}{исполнение}{fulfilment} одного из своих желаний, ты не задумываешься над тем, что для того, чтобы оно \explainDetail{исполнилось}{исполн\'{я}ть/исп\'{о}лнить}{to carry out}, тебе н\'{у}жно через что-то пройти, чего-то \explain{лишиться}{to lose; to be deprived of; recall that лишать/лишить means to deprive}, что-то приобрести... Как только желание сформировалось и укрепилось в сознании, всё вокруг начинает \explainDetail{перестраиваться}{перестраиваться/перестроиться}{to rebuild; to reorganize; to re-form; to restructure} для того, чтобы оно смогло исполниться. Уходят люди из твоей жизни, что мешают его исполнению, появляются новые, которые пом\'{о}гут, приходят те, что должны научить тебя увидеть дорогу к желанному. Иногда нужн\'{а} сила, без которой не пройти по этому пути, а силу дают боль и трудности. Ты привык к тому, что было ранее и не видишь к чему ведут болезненные изменения вокруг. Но ведь ты хочешь исполнения желаемого? Оно то, чего не было, но то, что должно родиться и это \explainDetail{уберёт}{убирать/убрать}{to clean/tidy; to take away; to remove (убир\'{а}йся отс\'{ю}да!=get lost!)} старое из твоей жизни, что не давало прийти Новому...

Рождение проходит через боль. За Ночью приходит День.

Должно Темноте \explain{сгуститься}{to thicken}, \explain{даб\'{ы}}{so that} Свет силой засиял...

Желая, ты сам, только ты, включаешь механизмы, что начиняют менять жизнь для нового рождения в ней. Ты притягиваешь всё для этого и боль, \explain{в том числе}{including}. Поэтому, помни, тот человек, что причинил тебе боль --- \explain{вызван тобой}{caused by you}. Он - кукла. Не \explain{гневись}{be angry} на него, а благодарностью одари за силу, за помощь в пути к Новому.

\begin{flushright}
    \it Аму Мом
\end{flushright}

Медитация для снятия эмоционального напряжения

Эмоциональное напряжение усиливает существующие в теле мышечные спазмы и повышает тонус нервной системы.

Это усиливает боль физическую и душевную. Нам кажется, что эмоции - это что-то неуправляемое.

Есть методика, выполни ее и забудь про: беспокойство, злость, апатию или другие истощающие эмоции.

\begin{enumerate}
    \item Прими удобное положение или отправься на \explainDetail{прогулку}{прогулка}{walk (n)} в одиночестве

    \item Сидя или гуляя начни \explainDetail{наблюдать}{наблюдать/понаблюдать}{to take care of; to observe; to watch} за своим дыханием, чтобы \explain{переключиться}{to switch} от мыслей к телу

    \item Наблюдая за дыханием \explainDetail{погрузи}{погрузи}{immerse; dip} все свое внимание в тело и почувствуй, где в теле \explainDetail{кроется}{крыться/покрыться}{to be concealed (кроюсь, кроешься, кроются)} напряжение

    \item Не думай, не анализируй и не пытайся понять, просто наблюдай.
          Наблюдай за тем, как начинается вдох и выдох, погружая свое внимание в тело

    \item Когда ты найдешь центр напряжения в теле, собери все внимание в этой части тела

    \item Дыши спокойной и с каждым выдохом \explainDetail{отпускай}{отпускать}{to let go} напряжение,
          если в это время ты \explainDetail{испытаешь}{испытывать/испытать}{to try} какие-то эмоции
          отпускай и их с дыханием

    \item Удели этой практики 10-15 минут и твое внутреннее состояние больше не будет \explain{неуправляемым}{uncontrolled} хаосом.
\end{enumerate}

Улыбаться полезно. Кроме выработки гормона радости для здоровья, это ещё и приятно.
\explainDetail{Искренняя}{искренний/яя/ее/ие}{sincere} улыбка посылает энергию любви,
которая \explainDetail{обладает}{обладать}{to have; to possess; to own} силой, чтобы \explainDetail{согревать}{согревать/согреть}{to warm up} и \explainDetail{исцелять}{исцелять/исцелить}{to heal}. Просто вспомните время, когда вы были \explain{расстроены}{upset} или больны физически, и кто-то, возможно, даже незнакомый, вам искренне улыбнулся --- и внез\'{а}пно вы почувствовали себя лучше.

Не всегда есть желание улыбаться? Упражнение «Улыбка Будды» поможет вам в любое время повысить своё эмоциональное состояние.

Выполняется очень просто. Сначала потренируйтесь перед зеркалом с отрытыми глазами.
Потом закройте глаза и запомните положение \explainDetail{мышц}{мышца}{muscle} лица,
чтобы повторить в любой обстановке.

\begin{enumerate}[noitemsep]
    \item Расположите свои губы так, чтобы \explainDetail{черт\'{а}}{черт\'{a} (ж)}{line},
          разделяющая их, располагалась строго горизонтально (параллельно полу).
    \item Совсем немного приподнимите вверх уголки губ.
\end{enumerate}

Посмотрите для наглядности на фотографию статуи Будды.

А теперь посидите с этой улыбкой хотя бы 5 минут и вы почувствуете ее действие.

«Надевайте» на свое лицо улыбку и вам станет легче жить.

Дополнение.

Очень хорошо с такой улыбкой \explainDetail{осознавать}{осознавать/осознать}{to realize} дыхание.
Следите за дыханием, говоря про себя: «Вдыхаю с улыбкой --- выдыхаю с улыбкой!»
