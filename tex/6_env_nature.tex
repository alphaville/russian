\chapter{Окружающая среда и природа}

\section{Городск\'{а}я жизнь}
Вы никогда не думали о последствиях жизни в городе? Казалось бы, \explain{на первый взгляд}{at first sight}, что жизнь в больших экономических и культурных центрах имеет только преимущества, но дальнейшее рассмотр\'{е}ние показывает, что она имеет и \explain{недостатки}{недостаток: disadvantage}.

С \explain{положительной}{полож\'{и}тельный/-ая: positive} стороны, легче найти работу в городе, потому что там обычно много ресторанов, кафе, гостиниц, школ, библиотек, музеев и т.д. Кроме того, жители города имеют прекрасную возможность посетить множество культурных и развлекательных учреждений, таких как музеи, галереи, ночные клубы, дискотеки и многое другое.

С другой стороны, жителям города \explain{приходится}{приходиться/прийтись: have to} жить в загрязненной атмосфере из-за интенсивного автомобильного движения и \explain{промышленных}{industrial} \explain{предприятий}{предпри\'{я}тие: enterprise}. Это может \explain{вызвать}{to cause} \explain{заболевания}{disease} лёгких и проблемы с сердцем. Кроме того, городской образ жизни довольно \explain{напряженный}{intense}, \explain{поскольку}{since/because} приходится много работать, много ездить на автомобиле и, в результате, стоять в пробках...

В заключение, городская жизнь имеет некоторые преимущества. Тем не менее, она также \explain{может нанести ощутимый вред}{can cause significant damage}, так что местные власти должны сделать несколько важных решений, например, они должны \explain{запретить}{[запрещать] to ban} промышленные предприятия в городах и вблизи городов, которые загрязняют воздух и воду токсичными парами.

\section[Загрязнение окружающей среды]{Причины и последствия загрязнения окружающей среды}
\textit{Источник: \url{https://bit.ly/3NV3JeZ}}

Загрязнение окружающей среды в настоящее время является самой большой проблемой, с которой сегодня \explain{ст\'{а}лкивается}{faces, is facing} мир. Наприм\'{е}р, в Соединенных Штатах 40\% рек и 46\% озёр слишком загрязнен\'{ы} для \ed{рыбной л\'{о}вли}{рыбная ловля}{fishing}, \ed{купания}{купание}{bathing} и водных организмов. Это \explain{неудивительно}{not surprising}, когда ежегодно в американские воды \explain{сбрасывается}{dumped} 1,2 триллиона галлонов \explain{неочищенных}{untreated} ливневых вод, промышленных \ed{отходов}{отх\'{о}ды}{waste} и неочищенных \ed{сточных вод}{ст\'{о}чные в\'{о}ды}{sewage}.

Одна треть верхнего \ed{слоя}{спой}{layer} \ed{почвы}{почва}{soil} в мире уже деградирована, и \explain{с учётом}{taking into account} нынешних темпов деградации почвы, вызванной неправильными методами ведения сельского хозяйства и промышленности, а также \ed{обезлесением}{обезлесение}{deforestation}, большая часть верхнего слоя почвы в мире может исчезнуть в течение следующих 60 лет.

Великий смог 1952 года унёс жизни 8000 человек в Лондоне. Это событие \explain{было вызвано}{was caused (by)} периодом холодной погоды \explain{в сочетании с}{in conjunction with} безветренными условиями, которые сформировали \explain{плотный}{dense} слой переносимых по воздуху загрязнителей, в основном от угольных электростанций, над городом.

Существует множество источников загрязнения, каждый из которых по-своему влияет на окружающую среду и живые организмы. В этой статье обсуждаются проблема загрязнения и последствия различных видов загрязнения.

\textbf{Причины.} Причины загрязнения не \explain{ограничиваются}{are limited} только выбросами \ed{ископаемого топлива}{ископ\'{а}емое т\'{о}пливо}{fossil fuel} и \ed{углерода}{углерод}{carbon}. Существует множество других типов загрязнения, включая химическое загрязнение \ed{водоёмов}{водоём}{reservoir} и п\'{о}чвы в результате неправильной утилизации и сельскохозяйственной деятельности, а также шумов\'{о}е и светов\'{о}е загрязнение, создаваемое городами и урбанизацией в результате роста населения.

\textbf{Загрязнение воздуха.}
Существует два типа \ed{загрязнителей}{загрязн\'{и}тель}{pollutant} воздуха: \ed{первичные}{перв\'{и}чный}{primary} и \ed{вторичные}{вторичный}{secondary}. Первичные загрязнители выбрасываются \explain{непосредственно}{directly} из их источника, в то время как вторичные загрязнители образуются, когда первичные загрязнители вступают в реакцию в атмосфере.

\ed{Сжигание}{сжиг\'{а}ние}{burning, combustion} ископаемого топлива для тр\'{а}нспорта и электричества производит как первичные, так и вторичные загрязнители и является одним из крупнейших источников загрязнения воздуха.

\ed{Выхлопные газы}{выхлопные газы}{traffic fumes; exhaust fumes} автомобилей содержат опасные газы и \explain{твёрдые частицы}{solid particles}, включая \ed{углеводороды}{углеводород}{hydrocarbon}, оксиды азота и монооксид углерода. Эти газы \explain{поднимаются}{rise} в атмосферу и \explain{вступают в реакцию}{react} с другими атмосферными газами, создавая ещё б\'{о}лее токсичные газы.

По данным Института Земли, интенсивное использование \ed{удобрений}{удобрение}{fertiliser} в с\'{е}льском хоз\'{я}йстве является основным источником загрязнения воздуха \ed{мелкими частицами}{мелкие частицы}{microparticles}, что \explain{затронуло}{affected} большую часть Европы, России, Китая и США. Считается, что уровень загрязнения, вызванного сельскохозяйственной деятельностью, \explain{превышает}{exceeds} все другие источники загрязнения воздуха мелкими частицами в этих странах.

\ed{Аммиак}{амми\'{а}к}{ammonia} -- это основной загрязнитель воздуха, \explain{образующийся}{emerging} в результате сельскохозяйственной деятельности. Аммиак попадает в воздух в виде газа из концентрированных отходов животноводства и полей, которые \explain{чрезм\'{е}рно}{excessively} уд\'{о}брены.

Затем этот \explain{газообразный}{gaseous} аммиак соединяется с другими загрязнителями, такими как оксиды и сульфаты азота, образующиеся в транспортных средствах и промышленных процессах, с образованием аэрозолей. Аэрозоли -- это \explain{крошечные}{tiny} частицы, которые могут \explain{проникать}{permeate} глубоко в лёгкие и вызывать сердечные и лёгочные заболевания.

Другие сельскохозяйственные загрязнители воздуха включают пестициды, \explain{гербициды}{herbicides} и фунгициды. Все это также способствует загрязнению воды.

\textbf{Загрязнение воды.}
Загрязнение \ed{питательными веществами}{пит\'{а}тельные веществ\'{а}}{nutrients} вызывается сточными водами и удобрениями. Выс\'{о}кие уровни питательных веществ в этих источниках попадают в водоемы и способствуют росту \ed{водорослей}{водоросли}{algae} и \ed{сорняков}{сорняк}{weed}, что может сделать воду \ed{непригодной}{непригодный}{unusable; unfit} для \ed{питья}{питьё}{drinking} и \explain{истощить}{deplete} кислород, что \explain{приведет к гибели}{will lead to death} водных организмов.

Пестициды и гербициды, \explain{применяемые}{used; applied} для сельскохозяйственных культур и жил\'{ы}х районов, концентрируются в почве и перен\'{о}сятся в грунтовые воды с дождевой водой и стоками. По этим причинам каждый раз, когда кто-то пробуривает \ed{скважину}{скважина}{well (water well)} на воду, её необходимо проверять на \explain{наличие}{availability} загрязняющих веществ.

Промышленные отходы являются одной из основных причин загрязнения воды, поскольку они создают первичные и вторичные загрязнители, включая \ed{серу}{сера}{sulfur}, \explain{свинец}{lead} и \explain{ртуть}{mercury}, нитраты и фосфаты, а также разливы нефти.

В \ed{развивающихся странах}{развив\'{а}ющиеся стр\'{а}ны}{developing countries} около 70\% \explain{твёрдых отходов}{solid waste} сбрасывается непосредственно в океан или море. Это вызывает серьёзные проблемы, включая причинение вреда и убийство \ed{морских существ}{морские существа}{sea creatures}, что \explain{в конечном итоге}{eventually} влияет на людей.

\textbf{Загрязнение земли и п\'{о}чвы.}
Загрязнение земель -- это разрушение зем\'{е}ль в результате деятельности человека и неправильного использования земельных ресурсов. Это происходит, когда люди наносят на почву химические вещества, такие как пестициды и гербициды, неправильно утилизируют отходы и \explain{безответственно}{irresponsibly} \explain{эксплуатируют}{exploit} полезные ископаемые при \ed{добыче}{добыча}{mining} полезных ископаемых.

Почва также загрязняется из-за протекающих подземных септиков, канализационных систем, вымывания вредных веществ со \ed{свалок}{свалка}{landfill} и прям\'{о}го сбр\'{о}са сточных вод промышленными \ed{предприятиями}{предприятие}{enterprise} в реки и океаны.

Дождь и наводнение могут переносить загрязнители с других уже загрязнённых зем\'{е}ль в п\'{о}чву в других местах.

Избыточное \explain{земледелие}{agriculture} и чрезм\'{е}рный \explain{в\'{ы}пас}{grazing} в результате сельскохозяйственной деятельности прив\'{о}дят к тому, что п\'{о}чва теряет свою питательную ценность и структуру, вызывая деградацию почвы, ещё один тип загрязнения почвы.

Свалки могут вымывать вредные вещества в почву и водные пути и создавать очень неприятные запахи, а также являются рассадниками \ed{грызун\'{о}в}{грыз\'{у}н}{rodent}, которые являются переносчиками болезней.

\textbf{Шум и световое загрязнение.}
Шум считается загрязнителем окружающей среды, вызываемым бытовыми источниками, общественными мероприятиями, коммерческой и промышленной деятельностью и транспортом.

Световое загрязнение вызвано длительным и чрезмерным использованием искусственного \ed{освещения}{освещение}{lighting} в ночное время, что может вызвать проблемы со здоровьем у людей и нарушить естественные циклы, \explain{в том числе}{including} деятельность \explain{дикой}{wild} природы. Источники светового загрязнения включают электронные рекламные щиты, ночн\'{ы}е спортивные площадки, уличные и автомобильные фонар\'{и}, городск\'{и}е парки, общественные места, аэроп\'{о}рты и жил\'{ы}е районы.


\section[Парниковый эффект]{Парниковый эффект: что надо знать о влиянии парниковых газов на Землю}

\textit{Источник: \url{https://trends.rbc.ru/trends/green/603766c39a794772017c8a13}}

О парниковом эффекте обычно говорят в связи с изменением климата. Действительно ли парниковый эффект вреден для нас и что нужно о нем знать?

\textbf{Что такое парниковый эффект?}
Парниковый эффект — это естественное явление, которое повышает температуру на нашей планете для комфортного существования.

Как он возникает? На нашу планету поступает солнечная радиация, которая нагревает поверхность. Излучение от солнца коротковолновое, поэтому парниковые газы, которые находятся вокруг Земли, свободно пропускают его. Какую-то незначительную часть солнечного света могут отразить обратно аэрозоли, которые находятся вместе с парниковыми газами в атмосфере Земли.

В свою очередь, когда планета нагревается, она отдает тепловую радиацию — инфракрасное излучение (длинные волны). Но так как излучение длинноволновое, то парниковые газы не дают полностью ему улететь в космос. Частично тепловому излучению все же удается обойти парниковые газы, но значительная доля отражается обратно, что и повышает температуру на Земле.

Первым, кто описал парниковый эффект, стал французский ученый Жан-Батист Жозеф Фурье в 1824 году, его же называют автором термина.


\textbf{Какие на Земле есть основные парниковые газы}

\textbf{Углекислый газ ($\rm{CO}_2$)}
читается важнейшим парниковым газом антропогенного происхождения. Углекислый газ возникает и естественным путем при круговороте углерода, но именно человек увеличил его концентрацию в атмосфере на 47\% с момента индустриальной революции.

\textbf{Метан ($\rm{CH}_4$)} — по своему парниковому эффекту метан считается даже сильнее, чем углекислый газ, но в атмосфере его заметно меньше. Естественные источники — болота и термитники. Антропогенное происхождение — свалки, сельское хозяйство, добыча угля и природного газа.

Закись азота ($\rm{N}_2\rm{O}$) образуется при сжигании твердых отходов и ископаемого топлива. Значительная часть N2O идет от сельского хозяйства.

Синтетические химические вещества, например, гидрофторуглероды, галогенированные углеводороды, гексафторид серы и другие синтетические газы. Основной источник — это химическая промышленность.

\textbf{Озон ($\rm{O}_3$)} — естественным образом встречается в стратосфере и тропосфере Земли и не вызывает значительного парникового эффекта. [2]

\textbf{Водяной пар} — по объему занимает первое место среди всех парниковых газов, однако прямые выбросы водяного пара влияют на парниковый эффект наименьшим образом. [3]

Сам по себе парниковый эффект — благо для нас, так как без него не было бы жизни на Земле. Если представить, что его не существует, средняя температура на Земле составляла бы $-18^\circ\rm{C}$, то есть реки и океаны всегда были бы замерзшими и нигде не росли растения. С его же помощью на нашей планете средняя температура достигает $+15^\circ\rm{C}$. [4]

Самый сильный парниковый эффект в Солнечной системе существует на Венере. Атмосфера планеты практически полностью состоит из углекислого газа, поэтому температура на поверхности Венеры достигает $475^\circ\rm{C}$.

\textbf{Причины парникового эффекта.}
Земля постоянно получает и отдает энергию. По закону сохранения энергии все это должно пребывать в радиационном балансе. Но человек своими действиями вывел систему из баланса. Когда объем парниковых газов увеличивается, они все чаще и чаще не позволяют теплу покинуть атмосферу Земли. Получается, что даже то инфракрасное излучение, которое когда-то улетало в космос, теперь частично остается с нами — глобальная температура повышается.

Ученые пришли к выводу, что средняя температура на Земле выросла на 1,1℃ с конца XIX века. Разница всего в 4℃ ранее приводила к ледниковым эпохам, поэтому эта цифра не такая уж и маленькая. Сложился научный консенсус, что в резком росте парниковых газов в атмосфере виновата хозяйственная деятельность человека.

Что усиливает парниковый эффект:
\begin{enumerate}
    \item выбросы производств;
    \item добыча полезных ископаемых;
    \item угольные электростанции;
    \item автомобильные выхлопы;
    \item экстенсивное сельское хозяйство;
    \item эксплуатация зданий;
    \item лесные пожары;
    \item вырубки лесов.
\end{enumerate}

Наибольший парниковый эффект вызывает сжигание топлива, его добыча и транспортировка, производство сырья (цемент, сталь и другие металлы), пищевая промышленность, захоронение и сжигание отходов. На них приходится примерно 70\% всех глобальных антропогенных выбросов.

Ученые вывели потенциал глобального потепления, который позволяет сравнить климатические эффекты парниковых газов за различные периоды времени. Например, 1 кг метана поглощает тепловое излучение в 84 раза лучше, чем 1 кг CO$_2$, если брать 20-летний период.

У газов разное время жизни, например, у метана оно составляет около 12 лет, у N$_2$O — 114 лет. Часть антропогенных выбросов углекислого газа удаляются из атмосферы в течении нескольких десятилетий, но значительная часть остается в атмосфере вплоть до нескольких тысячелетий.

\textbf{Последствия парникового эффекта.}
Изменение температуры прямо пропорционально радиационному воздействию. Ученые уже подсчитали, что если количество CO2 удвоится, это вызовет потепление от 1,5°C до 4,5$^{\circ}\rm{C}$ — это так называемая чувствительность климата. Уже сейчас концентрация углекислого газа в 1,5 раза выше доиндустриального уровня.

Некоммерческий исследовательский центр Oxford Economics опубликовал исследование о влиянии глобального потепления на экономику. Ученые взяли за основу показатель оптимальной температуры, при которой люди работают максимально производительно, а сельскохозяйственные культуры дают наибольший урожай. Эксперты определили этот показатель в 15$^{\circ}\rm{C}$. Государства, в которых среднегодовая температура ниже этого значения, могут получить небольшие преимущества от потепления. Страны с более жарким климатом, наоборот, понесут ущерб.

В ходе исследования специалисты из Oxford Economics проанализировали данные о положении в 203 развитых и развивающихся странах и спрогнозировали падение мирового ВВП на 20\% к 2100 году. Такой вывод основан на предположении, что средняя температура продолжит расти с такой же скоростью, что и сейчас (примерно на 0,2$^{\circ}\rm{C}$ в десятилетие). Выводы специалистов из Oxford Economics подтверждают результаты более раннего исследования, которое в 2015 году опубликовали ученые из Стэнфордского университета и Калифорнийского университета в Беркли.

По мнению экспертов из Oxford Economics, больше всего пострадает экономика Индии: ВВП на душу населения в стране упадет на 90\% к 2100 году, если выбросы парниковых газов в атмосферу не снизятся. Специалисты также предположили, каким мог бы быть этот показатель в разных странах, если бы средняя температура была на 1,1$^{\circ}\rm{C}$ ниже. Согласно прогнозу, он был бы значительно выше. Например, ВВП на душу населения в Нигерии мог бы быть на 35\% больше, чем сейчас.


\textbf{Пути решения.}
Существует множество путей решения проблемы, которые можно условно разделить на фантастические и реальные.

К фантастическим относится предложение распылить частички серебра в стратосфере, чтобы те отражали как можно больше солнечного света. Так Солнце не будет нагревать нашу планету, а та в свою очередь меньше будет отдавать тепла. По этой же причине некоторые ученые предлагают искусственно вызывать облака, так как они способны отражать солнечный свет, поступающий на Землю.

Что можно реально делать уже сейчас, чтобы парниковый эффект не навредил нам в будущем:
\begin{enumerate}
    \item сократить использование ископаемого топлива и переходить на возобновляемые источники энергии;
    \item повышать энергоэффективность и модернизировать технологий по сбережению энергии;
    \item заниматься устойчивым лесоуправлением и контролировать лесные пожары;
    \item переходить к экологически бережному сельскому хозяйству;
    \item восстанавливать почвенный покров, так как потеря гумуса напрямую влияет на парниковый эффект;
    \item отказаться от личного транспорта и переходить на велосипеды, общественный транспорт и электромобили.
\end{enumerate}
