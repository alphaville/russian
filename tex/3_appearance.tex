\chapter{Внешний вид}

\section{Что внешний вид способен рассказать о вас?}

\textit{Источник: \url{https://thewallmagazine.ru/what-does-the-look-tell-adout-you/}}

\textit{Автор: Анна Радионова}

\begin{flushright}
    \it Как-то во время индивидуальной беседы она втолковывала одной ученице, что внешность не имеет значение — куда важнее твоя личность. «Какую только чушь не приходится вдалбливать в детские головы», — думала Тесса, листая журнал.

    Джоан Роулинг. Случайная вакансия
\end{flushright}

Тема внешности во все времена являлась довольно деликатной и вызывающей жаркие споры среди представителей обоих полов. Кто-то делал выводы о человеке за первые 30 секунд знакомства и  придерживался мнения о том, что его визуальный образ способен как позитивно, так и негативно влиять на его жизнь. Иные, напротив, считали, что внутренний мир непременно важнее внешней оболочки, и глупо осуждать человека за независящие от него вещи. Каждый и по сей день придерживается той или иной точки зрения в большей степени, и что интересно, сторонники обеих правы.

Для того чтобы продолжать разговор о внешнем виде, стоит прояснить получившуюся путаницу, из-за которой происходит столкновение двух позиций, которые по своей сути не являются противоречащими. Следует понимать, что понятия внешности и внешнего вида в корне различны. Внешность – это то, что дано природой, тогда как внешний вид – то, что формирует сам человек и то, за что ответственен только он. А значит встречать и «судить» человека по его внешнему виду не только непредосудительный, но и весьма полезный навык, особенно, если в сферу профессиональных обязанностей входит подбор персонала, или просто есть желание чуть лучше разбираться в людях и  производить на окружающих приятное впечатление.

\begin{fancyquotes}
    Внешность – это то, что дано природой, тогда как внешний вид – то, что формирует сам человек и то, за что ответственен только он
\end{fancyquotes}

Существует ряд особенностей в манере одеваться, заметив которые, можно делать выводы о некоторых чертах характера человека, которого решили «встретить по одёжке».

Во-первых, необходимо смотреть на общую ухоженность человека: в каком состоянии ногти и волосы, аккуратно ли он одет, выглажена ли одежда, нет ли на ней пятен. Немаловажно и то, как он пахнет.[1] Конечно, по какому-либо одному пункту, вроде мятой рубашки, не стоит вешать на человека ярлык неряхи, однако, если есть два и более, то это может свидетельствовать о низкой самооценке и серьёзной внутренней дисгармонии. (Важно не приравнивать аккуратность и ухоженность к наличию у человека дорогих вещей известных брендов и возможности каждую неделю посещать салон красоты, так как это больше имеет отношение к материальному состоянию, нежели личным качествам человека.)

Что же касается следования модным тенденциям, безропотное их соблюдение говорит о том, что человек зависим от мнения окружающих и одежда помогает ему чувствовать себя частью сообщества.

Манера одеваться слишком сексуально и вызывающе может свидетельствовать о том, что у человека, предпочитающего такой стиль, как минимум, острый недостаток внимания, и подобные проявления – не что иное, как попытка выделиться из толпы и заявить о себе. Однако также это может быть свидетельством эмоционального и сексуального неблагополучия.

Люди, предпочитающие, кричаще яркие цвета и броские аксессуары, как правило, страдают от неуверенности в себе. Они остро нуждаются во внимании окружающих, что нередко может быть объяснено их низкой самооценкой.

Другая крайность – слишком унылые расцветки и чересчур консервативный покрой вещей говорят о застенчивости их обладателя. Такие люди предпочитают оставаться в тени, в стороне от внимания, стремясь таким образом защититься от проблем.

Идеально сидящая одежда и аккуратно причёсанные волосы (у женщин, как правило, собранные) говорят о том, что перед нами человек высокоорганизованный и дисциплинированный. Аккуратность может свидетельствовать и о ряде негативных черт характера, таких как излишняя жёсткость, суровость, неумение уступать и идти на компромисс.

Одежда, несоответствующая случаю, может свидетельствовать о том, что это, вероятнее всего, неуверенный в себе человек, стремящийся к вниманию и создающий себе бунтарский образ. Такие люди стремятся к тотальному контролю над ситуацией и роли лидера в коллективе.

Также многое может сказать о человеке его причёска:

Мужчины, пытающиеся скрыть лысину, зачёсывая волосы набок, или тонирующие седину, как правило, являются неуверенными и скрытными натурами, не готовыми откровенничать и выдавать всю правду о себе.

Женщины же, часто меняющие причёску и цвет волос, постоянно находятся в поиске индивидуальности и наиболее комфортного образа, что также может свидетельствовать о неуверенности в себе и низкой самооценке.

Необычные прически говорят о том, что их обладатель буквально кричит о своем желании быть замеченным. Тогда как чересчур консервативные прически «волосок к волоску» говорят о негибкости и неумении уступать.

Согласно последним исследованиям учёных, внешние данные, заложенные природой, тоже могут служить источником информации о человеке. Например, исследователи из Чехии утверждают, что нашли зависимость между чертами лица и уровнем интеллекта. [2]

По их мнению, для людей с более высоким уровнем интеллекта характерны далеко посаженные глаза, крупный нос, приподнятые уголки губ и заострённый подбородок.  Что интересно, по утверждению самих учёных, полученные данные применимы только к мужской аудитории, и уровень интеллекта женщины таким образом определить нельзя.

Учёные из Университета Эдинбурга пошли дальше и определили зависимость между симметричностью лица и прошлым человека. В ходе продолжительного исследования было выявлено, что люди, в чьём детстве не было серьёзных испытаний и потрясений, обладают значительно более симметричными чертами лица, чем люди, столкнувшиеся в раннем возрасте с трудностями. [3]

\begin{fancyquotes}
    В ходе продолжительного исследования было выявлено, что люди, в чьём детстве не было серьёзных испытаний и потрясений, обладают значительно более симметричными чертами лица, чем люди, столкнувшиеся в раннем возрасте с трудностями
\end{fancyquotes}

А в Университете Глазго попробовали пересмотреть традиционные методы оценивания физической формы человека. По мнению шотландских учёных, гораздо большее значение для здоровья имеет не количество жировой ткани в организме человека, а то, в каких местах она откладывается. Так, например, пухлые щёчки могут свидетельствовать о большей подверженности человека стрессам и инфекционным заболеваниям. [4]

Таким образом, не остается сомнений, что внешний вид – это не просто картинка, на которую не стоит обращать внимание. Напротив, стоит, но главное, не забывать, что внешность – это оружие, и оно выстрелит, только если внутри есть патроны.

[1-4]  \url{https://thewallmagazine.ru/what-does-the-look-tell-adout-you/}