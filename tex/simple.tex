\chapter{Остальные статьи}



\section{Покупки}
Если нам необходимо купить что-то, мы первым делом отправляемся в магазин. Существует огромное количество различных магазинов, в которых можно купить все, что угодно --- от продуктов питания до шурупов, болтов и гаек. Не трудно догадаться, какой тип магазина наиболее популярен. Без преувеличения можно сказать, что это супермаркеты и гастрономы. Человек питается ежедневно, поэтому обойти стороной такие магазины крайне сложно.

В каждом городе вы найдете такие магазины, как бакалеи, магазины одежды, булочные, мясные магазины. Я люблю ходить в цветочный магазин больше всего, потому что цветы - это мое хобби. Каждую неделю я хожу в антикварный магазин, потому что очень люблю оригинальные, старые вещи. Время от времени я посещаю магазин игрушек, чтобы купить игрушки своим племянникам и детям. Практически каждый месяц я захожу в магазин подарков для того, чтобы приобрести подарки на день рождения моим родственникам и друзьям.

Я вообще люблю тратить свое время на покупки, предпочтительно мне нравятся магазины с самообслуживанием. Ты можешь рассматривать какую-то вещь столько времени, сколько надо. Придирчивая продавщица не торопит тебя, ты сам себе хозяин. После всего этого, ты можешь спокойно направляться к кассе, где все покупки подсчитают и суммируют. В наше время по такому принципу работают не только супермаркеты, но и универмаги, магазины одежды и магазины бытовых товаров.






\section{Мой выходной день}

Меня зовут Андрей, мне 14 лет. Я хотел бы рассказать о своем выходном дне.
Воскресенье --- мой любимый день недели, потому что не н\'{у}жно вставать рано и идти в школу.
В воскресенье я встаю позже, чем обычно, заправляю постель, иду в ванную комнату, умываюсь и чищу зубы. Потом мы все вместе завтракаем. В воскресенье мама обычно печет блины, они у нее очень вкусные. Мы пьем чай или кофе и едим блинчики с джемом. После завтрака я помогаю маме убрать со стола и помыть посуду.
Потом я иду на улицу играть со своими друзьями. Зимой мы катаемся на лыжах. Летом мы ходим с папой в бассейн, особенно когда очень жарко.
В час дня мы обедаем.
После обеда мы немного отдыхаем. Папа идет к себе в кабинет и читает газеты. Мама смотрит телевизор, моя младшая сестренка спит. Я обычно иду в свою комнату, читаю книгу, слушаю музыку или делаю домашнюю работу на следующую неделю.
Вечером мы все вместе идем гулять в парк. Он очень большой и красивый. Там много разных аттракционов. Мы с сестрой очень любим кататься на них или просто гулять с нашими родителями. Мама и папа часто покупают нам сладкую вату или попкорн. Потом мы заходим в кафе и заказываем самую большую пиццу. Мы кушаем, разговариваем, смеемся.
Домой мы приходим в 10 часов вечера, \explain{довольные}{satisfied}, \explain{уставшие}{tired} и очень счастливые от того, что провели такой чудесный выходной.



\section{Библиотеки}
\textbf{Роль библиотек.}
Библиотеки всегда играли важную роль в развитии человечества. Библиотеки --- это хранилище нашего культурного наследия. На протяжении всей истории они были центрами науки и искусства.

\textbf{Первые библиотеки.}
Первые библиотеки появились на Древнем Востоке около 2500 г. до н.э. Они представляли собой собрания глиняных табличек, покрытых клинописью. В Древнем Египте существовало множество библиотек, включая крупнейшую библиотеку древнего мира в Александрии.

\textbf{Современные библиотеки.}
В современных библиотеках вы можете найти книги, газеты, периодические издания, документы, ноты, карты, CD и DVD-диски. Здесь есть библиотекари, готовые помочь найти нужную информацию, и читальный зал --- тихое и удобное место для работы с библиотечными ресурсами.
В некоторых библиотеках вы также можете получить доступ к их электронным ресурсам или интернету. Сегодня библиотеки есть практически в каждом городе, поселке или деревне. Школы, университеты и крупные организации часто имеют собственные библиотеки.

\textbf{Моя школьная библиотека.}
В моей школе также есть библиотека. Она не очень большая, но в ней много интересных и познавательных книг. Здесь можно найти \red{как} учебную, \explainDetail{так и}{как... так и...}{both... and...} художественную литературу.
В нашей библиотеке есть \explain{просторный}{spacious} читальный зал, где иногда проходят уроки литературы. Наша школа организует встречи с писателями и поэтами нашего города, которые обычно проходят в библиотеке.

Мне нравится брать книги в моей школьной библиотеке. Хотя мы живем в век компьютерных технологий, и почти все книги можно найти в Интернете, полежать на диване с книгой в руках для меня настоящие удовольствие.
Мне особенно нравится читать научную фантастику, русскую классику и адаптированные английские книги.

\section{Встреча Трампа и Кима: театральщины и уступок не будет}

Американский лидер Дональд Трамп согласился на встречу с главой КНДР Ким Чен Ыном, и это рациональный поступок без всякой театральщины. С помощью санкций на Северную Корею оказали колоссальное давление, Ким Чен Ын обещал приостановить испытания и заявил о желании встретиться с Трампом, сообщил директор ЦРУ США Майкл Помпео.

Но Трамп ни на какие уступки Пхеньяну не пойдет. Ким Чен Ын должен позволить США проводить военные учения на Корейском полуострове и гарантировать, что на встрече будет обсуждаться вопрос об окончательной денуклеаризации КНДР.

Глава ЦРУ подчеркнул, что Пхеньян уже дал согласие на выполнение этих условий, это достижение, потому что это серьезная уступка в обмен на проведение двусторонних переговоров. Вашингтон на себя обязательств не брал.

Помпео напомнил о статусе Трампа и Ким Чен Ына. Они принимают важные решения на высшем уровне, передает ТАСС .

В четверг начальник Управления национальной безопасности при президенте Южной Кореи Чон Ый Ён заявил, что Трамп готов провести встречу с Ким Чен Ыном до мая 2018 года. Пресс-секретарь Белого дома Сара Сандерс сообщила, что Трамп примет приглашение встретиться с лидером КНДР, но сроки и место еще не известны. Но встречи не будет, пока Пхеньян не начнет сворачивать ядерные разработки.

Сеул пока предал гласности не все заявления по поводу США, которые сделал Ким Чен Ын 5 марта в Пхеньяне. Северокорейский лидер заявил, что готов к откровенному диалогу с Вашингтоном, в случае его развития он остановит работу ядерного центра в Йонбене в 80 км к северу от Пхеньяна.


\section{ТВ сегодня}
Телевизор стал частью нашей жизни уже много лет назад. Смотря телевизор, мы узнаем много информации. Самая важная информация, поступающая из телевизора, -- это, безусловно, новости. Благодаря новостям, мы остаёмся в курсе событий как внутренней политики, так и внешней. Однако не все передачи приносят нам пользу. \explain{Скорее всего}{likely}, наоборот --- они только препятствуют нашему развитию, делая нас немощными и малоактивными. Поэтому сказать, что телевизор --- это 100\% благо невозможно.

Так чего же прин\'{о}сит больше «ящик» с развлекательными программами --- \explain{п\'{о}льзы}{польза: use; utility} или вреда? Дать \explain{однозначный}{unambiguous} ответ нельзя. Что будет, если мы вообще не будем включать телевизор? Мы не будем знать, что \explain{происх\'{о}дит}{happens; takes place} в мире и в стране. Можно, конечно, узнавать новости с помощью интернетa, но не многие люди привыкли это делать. Особенно это касается людей \red{за} 50 лет.

Давайте рассмотрим другую ситуацию. Человек целый день лежит на диване или сидит в кресле, и \explain{сутками напролёт}{day in day out; constantly} смотрит в «ящик». Кому понравится такой образ жизни \explain{лежеб\'{о}ки}{лежебока: couch potato}? Ответ: никому! Однако такое \explain{отношение}{attitude} в жизни также нанесет вред главному герою --- ленивому телезрителю. Почему? Во-первых, он пассивен, что в дальнейшем приведёт к ухудш\'{е}нию его здоровья и качества жизни. Во-вторых, он смотрит все \explain{подряд}{in a row}, переключая каналы с одного на другой. Мы все хорошо знаем, что 90\% информации по телевидению не являются ни обучающими, ни познавательными. Они нацелены на то, чтобы просто \explain{привлечь}{(привлекать): to attract} внимание зрителя, не говоря уже о бесполезной рекламе.



\section{Песков рассказал, чем займется Путин на каникулах}
Работать и заниматься спортом намерен в новогодние девятидневные праздники президент России Владимир Путин. А вот что будет на столе у президента, его пресс-секретарь Дмитрий Песков ответить не смог.

Он только отметил, что Путин весьма скромен в плане пищевых предпочтений, он предпочитает абсолютно скромную и здоровую еду. "Остальное, --- цитирует Пескова "Интерфакс", --- это личная жизнь президента, вряд ли стоит туда вмешиваться".
Полноценных каникул, отметил Дмитрий Песков, у президента не будет --- его ждут телефонные звонки по оперативным вопросам и ежедневные доклады силовиков, спецслужб.

"По многу раз в день он общается и с членами кабинета. У него же практически не бывает полноценных выходных", --- продолжил Дмитрий Песков.

Свободное время Владимир Путин использует для занятий любимыми видами спорта.

"Он и в хоккей играет, вы знаете, и плавает. И делает это, когда возможность представляется, на ежедневной основе", --- подчеркнул Дмитрий Песков.

Кроме того, Путин будет смотреть на Новый год свое телеобращение с близкими людьми.


\section{Мои планы на будущее}
Меня зовут Анна и я ученица одиннадцатого класса, поэтому в этом году я оканчиваю школу. Я хочу поделиться с вами своими планами на будущее. Конечно, я и мои друзья часто \explain{обсуждаем}{обсуждать/обсудить: to discuss + винительный падеж} такие темы, как наши планы на будущее. Некоторые из них будут поступать в институт или университет, или, возможно, в какой-нибудь колледж. Другие же хотят начать трудовую деятельность после окончания школы.

Что касается меня, я не хотела бы работать сразу же после окончания школы. Я хотела бы получить качественное образование, поэтому я старалась хорошо учиться, и могу \explain{с гордостью}{proudly (гордость: pride)} \explain{утверждать}{(утвердить) to claim}, что я \explain{дост\'{и}гла}{достигать/достичь: to achieve} в этом успеха.

Я бы сказала, что я творческий человек \explain{по своей природе}{by nature} и интересные идеи по оформлению интерь\'{е}ра и моделирования одежды очень часто рождаются у меня в голове, так что я хотела бы стать модельером. Но сначала мне н\'{у}жно \explain{поступить в Университет}{to go to university} технологии и дизайна и в течение всего года я \explain{усердно}{diligently} готовлюсь к \explain{вступительным экзаменам}{вступительные экзамены: entrance exams}. Мой репетитор помогает мне в этом.

Я думаю, что мода --- это очень популярная и \explain{развивающаяся}{developing (from: развиваться/развиться); past part. развивающийся} \explain{промышленность}{industry}. И ясное дело, что мода \explain{влияет}{влиять/повлиять (+на): to affect} на миллионы людей во всем мире. На мой взгляд, большинство людей, особенно женщины, стараются следовать модным тенденциям, чтобы быть стильными и привлекательными. Но я уверенна, что \explain{в любом случае}{anyway; in any case}, важно придерживаться золотой середины в моде, чтобы не выглядеть как манекен из бутика модной одежды.

Я выписываю специализированную литературу о моде и её истории; также я стараюсь следить за показами мод известных дизайнеров. Это очень интересно и \explain{захватывающе}{exciting} для меня, и я чувствую, что смогу \explain{добиться успеха}{to achieve success (добиваться/добиться)} в этой области. Я мечтаю стать известным дизайнером и создать свою собственную уникальную линию одежды. Я думаю, что моя одежда будет запоминающейся и красивой. Я постараюсь сделать всё возможное для этого.

Также эта профессия даст мне возможность \explain{бывать/побывать}{to visit (also: посещать/посетить} в разных странах и \explain{ознакомиться}{ознакомляться/ознакомиться: to familiarize} с их культурой и историей. Для того, чтобы комфортно себя чувствовать за границей, я изучаю два иностранных языка --- английский и французский. Ещё я мечтаю посетить главные столицы моды --- Париж, Лондон и Милан. А также \explain{надеюсь}{надеться (надеюсь, -еешься, -ются); понадеяться}, что смогу познакомиться с известными дизайнерами, а также \explain{перенять}{перенимать/перенять: to adopt; to immitate} их опыт и профессионализм.

Конечно, я хочу, чтобы у меня был\'{а} дружная семья в будущем. Наша семья будет построена на \explain{дов\'{е}рии}{дов\'{е}рие}, уважении и \explain{любв\'{и}}{любовь, любв\'{и}, любви, любовь, люб\'{о}вью, любви.}. У моей будущей семьи будут свои традиции и, конечно, традиции моей семьи также будут продолжены.
Конечно, это еще только планы, но я должна стараться, чтобы мои мечты \explain{осуществились}{осуществляться/oсуществиться: to come true}. Я думаю, что у меня всё пол\'{у}чится.

\section{Путин -- о дочерях, любви и воспитании}

Президент России стал героем нового фильма журналиста Андрея Кондрашова ``Путин''. Фильм опубликован в Интернете на странице Дмитрия Киселева в социальной сети ``Одноклассники''.

В частности, Владимир Путин рассказал об отношениях в семье. По воспоминаниям президента, родители окружили его атмосферой любви. Они помогали ему развиваться и достигать успехов. ``Они были заточены на это, на мое будущее, жизненные успехи и результаты'', --- отметил Путин.

По словам Путина, его отец был строгим, но ни разу в жизни не употребил крепкое слово в присутствии сына. Хотя, по воспоминанию учительницы Путина, отец называл мальчика разгильдяем после того, как он стал заниматься спортом. ``Они --- люди простые, из рабочей среды, у них были свои представления об основах воспитания'', --- прокомментировал Путин. Родители хотели, чтобы он научился музыке, и купили ему баян.
К своим дочерям Путин, по его признанию, не всегда относился требовательно. Но то, как они относятся к труду, его радует. ``Они стараются, работают очень много. Они --- трудоголики'', --- сказал президент.

Дочери советуются с Путиным, как можно улучшить жизнь россиян. ``Как уж они будут строить свою личную жизнь --- это уже в значительной степени находится в их руках'', --- сказал Путин.




\section{Откуда взять энергию?}
Отсутствие энергии --- это первый признак приближающихся несчастий и болезней. В Аюрведе говорится, что если человек продвигается в духовной жизни, то это должно быть видно по двум признакам:

\begin{enumerate}[noitemsep]
    \item Человек с каждым днем становится все счастливей и счастливей.
    \item Его отношения с другими людьми улучшаются. Когда мы получаем тонкую энергию...

          Тонкую энергию мы получаем когда:
          \begin{itemize}[noitemsep, label=+]
              \item голодаем,
              \item \explainDetail{выполняем}{выполнять/выполнить}{to perform} дыхательные упражнения,
              \item \explain{уединяемся}{we stay alone},
              \item даем \explain{обет}{vow} молчания, на какое-то время.
              \item гуляем (или просто находимся) по берегу моря, по гор\'{а}м, \explain{созерцаем}{contemplate} красивые пейзажи природы,
              \item занимаемся \explainDetail{бескорыстно}{бескорыстный}{unselfish} творчеством,
              \item \explain{восхваляем}{praise} \explain{достойную}{worthy} личность, за его возвышенные качества и \explain{поступки}{deeds},
              \item смеемся, \explain{радуемся}{rejoice}, улыбаемся от души,
              \item бескорыстно кому-то помогаем,
              \item проявляем \explain{скромность}{modesty},
              \item молимся \explain{перед}{before + instr.} едой,
              \item едим продукты полные \textit{праной} (жизненной энергией) --- натуральные \explain{злаки}{cereals}, каши, \explain{топлённое масло}{ghee}, мед, фрукты, овощи,
              \item спим с 9-10 вечера, до двух часов ночи (в другое время нервная система не отдыхает, \explain{сколько бы}{as much as we may sleep} мы не спали).
              \item получаем \explain{сеанс}{session} хорошего массажа, от гармоничной личности. Или делаем самомассаж.
              \item обливаемся холодной водой, особенно по утрам и наиболее сильный эффект если мы при этом стоим босиком на земле.
              \item \explain{жертвуем}{sacrifice} своим временем, деньгами...
              \item видим за всем божественную \explainDetail{в\'{о}лю}{воля}{will}.
          \end{itemize}
\end{enumerate}


\textbf{Когда мы теряем энергию...}
К потери энергии приводят:
\begin{itemize}[noitemsep, label=--]
    \item уныние, недовольство судьбой, сожаление о прошлом и страх, неприятие будущего,
    \item  постановка и преследование эгоистичных целей,
    \item бесцельное существование,
    \item обиды (обида)
    \item переедание,
    \item бесконтрольное блуждание ума, неумение сконцентрироваться.
    \item когда мы едим жаренную или старую пищу, пищу приготовленную человеком в гневе или испытывающем другие отрицательные эмоции, при использовании микроволновой печи, продукты, содержащие консерванты, химические добавки, выращенные в искусственных условиях, с использованием химических \explainDetail{удобрений}{удобрение}{fertilizer},
    \item поедание пищи лишенной праны --- кофе, черный чай, белый сахар, белая мука, мясо, алкоголь,
    \item еда в спешке и на ходу,
    \item курение,
    \item \explain{пустые разговоры}{(lit.) void discussions},
    \item неправильное дыхание, например, слишком \explainDetail{ч\'{а}стое}{частый}{frequent} и глубокое,
    \item нахождение под прямыми \explainDetail{луч\'{а}ми}{луч}{ray} Солнца, с 12 до 4 дня, особенно в пустыне,
    \item беспорядочные \explainDetail{половые}{полов\'{о}й}{sexual} связи, секс без любви к партнёру,
    \item \explain{излишний}{unnecessary} сон, сон после 7 \'{у}тра, недостаток сна,
    \item \explain{напряж\'{е}ние}{tension; stress} ум\'{а} и т\'{е}ла,
    \item \explain{\'{а}лчность}{greed} и \explain{жадность}{stinginess}.
\end{itemize}

Восточная психология на 50\% состоит из пранаямы --- теории и практики определенных дыхательных техник, которые позволяют человеку быть всегда наполненным жизненной силой (Праной).

Как утверждают современные просветленные учителя йоги набраться праны мы можем через:
\begin{enumerate}
    \item \textbf{Элемент земли.} питаясь натуральной пищей, жить на природе, созерцать деревья, ходить босиком по земле. Недавно я общался с очень известным аюрведическим доктором, \explainDetail{защитившему}{защитить, защитивший (past act.)}{who defends} диссертацию по медицине, он утверждал, что если человек начинает жить на природе, \explain{вдали от}{away from} больших городов, которые \explain{вынуждают}{necessitate} ездить в метро, ходить по асфальту, то у такого человека быстро восстанавливается иммунитет и он начинает жить здоровой жизнью.

    \item \textbf{Элемент воды.} пить воду из колодцев или \explainDetail{ручьев}{(sing.) ручей, ручья, ручью, ручей, ручьём, ручье, (plur.) ручьи, ручьёв, ручьям, ручьи, ручьям, ручьями, ручьях}{brook; creek}. Плавать в реке или море. \explainDetail{Избегать}{избегать}{to avoid} пить \explain{кофеиносодержащие}{containing caffeine} напитки, алкоголь и соду.

    \item \textbf{Элемент огня.} нахождение на Солнце и употребление пищи содержащей Солнечный свет.

    \item  \textbf{Элемент воздуха.} это самый важный элемент получения Праны, через вдыхание чистого воздуха, особенно в горах, в лесу и на берегу моря. Курение и нахождение в местах большого \explain{скопления}{?} людей, лишает человека праны.

    \item \textbf{Элемент эфира.} культивируя \explain{позитивное}{положительный} мышление, \explainDetail{доброт\'{у}}{доброт\'{а}}{kindness}, хорошее настроение.
\end{enumerate}


И этот \explain{уровень}{level} считается базовым.

Ибо даже, если человек живет на природе и правильно питается, но при этом ходит раздражённый и злой, то наоборот, \explain{изл\'{и}шек}{surplus} Праны еще быстрее разрушит его.
С другой стороны гармоничный человек, то есть добродушный,
\explain{бесстрашный}{fearless}, может довольно долго протянуть в городе,
если он вынужден там жить.
Но даже такому человеку н\'{у}жно следить за питанием
и периодически «вырываться» на природу.

У нас каждую секунду есть выбор --- светить миру, приносить своей жизнью благо и счастье окружающим, улыбаться, \explainDetail{заб\'{о}титься}{(по)заб\'{о}титься (заб\'{о}чусь, заб\'{о}{}тишься, заб\'{о}- тятся)}{to care about} о других, служить бескорыстно, жертвовать, сдерживать низшие \explain{побуждения}{drives (сдерживать низшие побуждения: to control/restrain the lower drives)}, видеть в каждом человеке Учителя, в каждой ситуации видеть Божественное \explain{провидение}{providence}, которое создало эту ситуацию \explain{для того что бы}{so that; in order to} нас чему то научить, благодарить ...

Либо предъявлять претензии, \explain{обижаться}{to take offence}, \explain{жаловаться}{to complain}, \explain{завидовать}{to envy}, ходить с клинообразным выражением лица, погрузиться в свои проблемы, в зарабатывание денег для того что бы потратить их на \explain{удовлетворение}{satisfaction} чувств, \explain{проявлять}{to show; to demonstrate} агрессию.

В этом случае \explain{в независимости от того}{regardless of} сколько у человека денег, он будет несчастный и мрачный. И с каждым днем энергии будет все меньше и меньше. И для того что бы её где-то взять нужны будут искусственные стимуляторы: кофе, сигареты, алкоголь, ночные клубы, выяснение отношений с кем-то. Все это дает вначале подъём, но в итоге приводит к полному разрушению..

Простой регулярный вопрос себе: я свечу миру или поглощаю свет? Может быстро изменить ход наших мыслей и следовательно поступков. И быстро превратить нашу жизнь в красивое яркое \explain{сияние}{shining}, полное любви. И тогда вопросы, где взять энергию уже не \explain{возникают}{emerge}.

\begin{flushright}
    \it Р. Блект
\end{flushright}


Каждый раз, когда кто-то причинил тебе боль, не спеши гневаться. Эта \explain{боль}{(ж) pain; ache (e.g., головн\'{а}я боль: headache)} выпр\'{о}шена тобой у Вселенной. \explainDetail{Неосознанно}{неосознанно}{unconsciously}, ты сам \explain{привлёк}{past tense of привлечь} её.

Причиняющий боль - всего лишь \explain{кукла на верёвочках}{doll on a string (puppet)}, что управляема тобой...

``Будь осторожен с желаниями'' --- слова Вечности. Что несут они в себе?

Когда ты \explain{жаждешь}{to crave for} \explainDetail{исполнения}{исполнение}{fulfilment} одного из своих желаний, ты не задумываешься над тем, что для того, чтобы оно \explainDetail{исполнилось}{исполн\'{я}ть/исп\'{о}лнить}{to carry out}, тебе н\'{у}жно через что-то пройти, чего-то \explain{лишиться}{to lose; to be deprived of; recall that лишать/лишить means to deprive}, что-то приобрести... Как только желание сформировалось и укрепилось в сознании, всё вокруг начинает \explainDetail{перестраиваться}{перестраиваться/перестроиться}{to rebuild; to reorganize; to re-form; to restructure} для того, чтобы оно смогло исполниться. Уходят люди из твоей жизни, что мешают его исполнению, появляются новые, которые пом\'{о}гут, приходят те, что должны научить тебя увидеть дорогу к желанному. Иногда нужн\'{а} сила, без которой не пройти по этому пути, а силу дают боль и трудности. Ты привык к тому, что было ранее и не видишь к чему ведут болезненные изменения вокруг. Но ведь ты хочешь исполнения желаемого? Оно то, чего не было, но то, что должно родиться и это \explainDetail{уберёт}{убирать/убрать}{to clean/tidy; to take away; to remove (убир\'{а}йся отс\'{ю}да!=get lost!)} старое из твоей жизни, что не давало прийти Новому...

Рождение проходит через боль. За Ночью приходит День.

Должно Темноте \explain{сгуститься}{to thicken}, \explain{даб\'{ы}}{so that} Свет силой засиял...

Желая, ты сам, только ты, включаешь механизмы, что начиняют менять жизнь для нового рождения в ней. Ты притягиваешь всё для этого и боль, \explain{в том числе}{including}. Поэтому, помни, тот человек, что причинил тебе боль --- \explain{вызван тобой}{caused by you}. Он - кукла. Не \explain{гневись}{be angry} на него, а благодарностью одари за силу, за помощь в пути к Новому.

\begin{flushright}
    \it Аму Мом
\end{flushright}

Медитация для снятия эмоционального напряжения

Эмоциональное напряжение усиливает существующие в теле мышечные спазмы и повышает тонус нервной системы.

Это усиливает боль физическую и душевную. Нам кажется, что эмоции - это что-то неуправляемое.

Есть методика, выполни ее и забудь про: беспокойство, злость, апатию или другие истощающие эмоции.

\begin{enumerate}
    \item Прими удобное положение или отправься на \explainDetail{прогулку}{прогулка}{walk (n)} в одиночестве

    \item Сидя или гуляя начни \explainDetail{наблюдать}{наблюдать/понаблюдать}{to take care of; to observe; to watch} за своим дыханием, чтобы \explain{переключиться}{to switch} от мыслей к телу

    \item Наблюдая за дыханием \explainDetail{погрузи}{погрузи}{immerse; dip} все свое внимание в тело и почувствуй, где в теле \explainDetail{кроется}{крыться/покрыться}{to be concealed (кроюсь, кроешься, кроются)} напряжение

    \item Не думай, не анализируй и не пытайся понять, просто наблюдай.
          Наблюдай за тем, как начинается вдох и выдох, погружая свое внимание в тело

    \item Когда ты найдешь центр напряжения в теле, собери все внимание в этой части тела

    \item Дыши спокойной и с каждым выдохом \explainDetail{отпускай}{отпускать}{to let go} напряжение,
          если в это время ты \explainDetail{испытаешь}{испытывать/испытать}{to try} какие-то эмоции
          отпускай и их с дыханием

    \item Удели этой практики 10-15 минут и твое внутреннее состояние больше не будет \explain{неуправляемым}{uncontrolled} хаосом.
\end{enumerate}

Улыбаться полезно. Кроме выработки гормона радости для здоровья, это ещё и приятно.
\explainDetail{Искренняя}{искренний/яя/ее/ие}{sincere} улыбка посылает энергию любви,
которая \explainDetail{обладает}{обладать}{to have; to possess; to own} силой, чтобы \explainDetail{согревать}{согревать/согреть}{to warm up} и \explainDetail{исцелять}{исцелять/исцелить}{to heal}. Просто вспомните время, когда вы были \explain{расстроены}{upset} или больны физически, и кто-то, возможно, даже незнакомый, вам искренне улыбнулся --- и внез\'{а}пно вы почувствовали себя лучше.

Не всегда есть желание улыбаться? Упражнение «Улыбка Будды» поможет вам в любое время повысить своё эмоциональное состояние.

Выполняется очень просто. Сначала потренируйтесь перед зеркалом с отрытыми глазами.
Потом закройте глаза и запомните положение \explainDetail{мышц}{мышца}{muscle} лица,
чтобы повторить в любой обстановке.

\begin{enumerate}[noitemsep]
    \item Расположите свои губы так, чтобы \explainDetail{черт\'{а}}{черт\'{a} (ж)}{line},
          разделяющая их, располагалась строго горизонтально (параллельно полу).
    \item Совсем немного приподнимите вверх уголки губ.
\end{enumerate}

Посмотрите для наглядности на фотографию статуи Будды.

А теперь посидите с этой улыбкой хотя бы 5 минут и вы почувствуете ее действие.

«Надевайте» на свое лицо улыбку и вам станет легче жить.

Дополнение.

Очень хорошо с такой улыбкой \explainDetail{осознавать}{осознавать/осознать}{to realize} дыхание.
Следите за дыханием, говоря про себя: «Вдыхаю с улыбкой --- выдыхаю с улыбкой!»


