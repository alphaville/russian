\chapter{Личность, Характер и Психология}
\section{Что такое думскроллинг}

\textit{и почему чтение плохих новостей пагубно влияет на ментальное здоровье?}

\textit{Источник: \url{https://www.elle.ru}}
% https://www.elle.ru/otnosheniya/psikho/chto-takoe-dumskrolling-i-pochemu-chtenie-plokhikh-novostei-pagubno-vliyaet-na-mentalnoe-zdorove/

\begin{fancyquotes}
    Постоянно просматриваете новости? Не выпускаете смартфон из рук? У нас для вас плохие новости  ---  вы зависимы, и это пагубно влияет на ваше здоровье и умственные способности.
\end{fancyquotes}

Думскроллинг (от английского doom  ---  «гибель, судьба, рок, Судный день» и scrolling  ---  «прокрутка»)  ---  это склонность к просмотру и чтению плохих новостей, несмотря на то, что они вызывают негативные эмоции, удручают, огорчают и деморализуют. Термин стал относительно широко использоваться в начале пандемии и сейчас снова стал актуален на фоне нестабильной ситуации в мире. Так почему же мы не можем оторваться от плохих новостей и как это влияет на наше психоэмоциональное состояние?

Основная причина бесконтрольного думскроллинга  ---  это боязнь пропустить важные новости. Беспокойный ум стремится понять, что происходит в мире и как это может коснуться лично нас. В таком случае думскроллинг дает ощущение контроля над ситуацией. Интуитивно мы пытаемся подготовиться к потенциальным угрозам. Принцип «предупрежден  ---  значит вооружён» миллионы лет способствует выживанию людей как вида.

\textbf{Почему это вредно?}

Иллюзия контроля над ситуацией, по большому счету, не дает никаких преимуществ. Думскроллинг способствует развитию тревожности и стресса, повышается вероятность панических атак, снижается концентрация. Умные алгоритмы соцсетей предлагают все больше и больше плохих новостей, поиск и чтение пугающих статей превращается в зависимость, человек игнорирует собственные мысли и чувства. Впоследствии чтение негативных новостей может приводить и к ухудшению сна и истощению нервной системы.

\textbf{Как бороться с думскроллингом?}

Не пользуйтесь гаджетами перед сном. Не читайте новостей о коронавирусе, войнах, протестах и других тревожных явлениях на ночь. Если вам сложно контролировать это самостоятельно, то установите будильник и за 2-3 часа до сна переводите телефон в авиарежим.

Читайте только ту информацию, которая вам нужна, не переключайте внимание на другие новости. Перед тем, как читать новости и статьи в интернете, четко определите «цель визита», не обращайте внимание на предложенные статьи или кликбейтные заголовки.

Отвлекитесь от новостного контента. Попробуйте сместить фокус внимания с новостей на интересные статьи, интервью, рецензии.

Займитесь чем-то другим. Кино, музыка, встречи с друзьями  ---  все это поможет провести вам время с куда большей пользой для нервной системы.

\newpage
\section{Синдром спасителя}

\textit{Как добрые намерения скрывают эмоциональные недостатки}

\textit{Что скрывает навязчивое желание помочь ближнему, если вас об этом никто не просит}

\textit{Источник: \url{https://www.elle.ru}}
% https://www.elle.ru/otnosheniya/psikho/sindrom-spasitelya-kak-dobrye-namereniya-skryvayut-emocionalnye-nedostatki/

Проявления синдрома спасителя не всегда очевидны для окружающих и даже тех, кого непрошеные благодетели окружают заботой. Импровизированные супергерои всегда готовы помочь нерасторопным коллегам, спасти близких (и не очень) людей от жизненных невзгод и пагубных пристрастий. Именно такие светлые человечки рады круглосуточно наставлять подшефных по вопросам правильного питания, карьерного роста и токсичных отношений. Ведомые убеждением, что в этом их высшее предназначение, «спасители» возводят альтруизм в культ, в действительности скрывая за самоотдачей эгоизм и невротические изъяны.

Термин «спаситель» используется психологами с 1968 года, с тех пор, как доктор Стивен Карпман, ученик Эрика Берна и знаток транзактного анализа, раскрыл в опубликованной работе модель социального и психологического взаимодействия, названную в его честь «треугольник Карпмана» (он же  ---  «треугольник судьбы» или Karpman drama triangle). В статье Fairy Tales and Script Drama Analysis американский ученый описал три привычные роли, которые мы часто играем в разных ситуациях: жертва, преследователь и спаситель, который вмешивается, как кажется, из желания помочь тому, кого обижают или недооценивают.

Как разъяснил Карпман, ролевая игра, схожая с мелодраматическим сюжетом про «героя, злодея и девицу в беде», раскрывает неочевидный мотив: спаситель заинтересован поддерживать жертву в ее зависимости от себя. Догадываетесь почему?

\textbf{Как из заботы получается созависимость}

«Нужда помогать другим движет личностью, способной реализоваться исключительно в опеке других,  ---  объясняет Анн-Виктуар Русселе, парижский психолог и терапевт.  ---  Такие люди считают своим долгом спасать других в ущерб себе, включая тех, кто в этом абсолютно не нуждается. Они сознательно вступают в созависимые отношения, считая, что не заслуживают любви партнера, но убеждая себя, что эта связь оправдана желанием избавить ее/его от проблем. Настойчивая услужливость имеет в корне нарциссический изъян, скрывающий неуверенность в себе и сопутствующую мотивацию: потребность поднять самооценку. „Спаситель“ становится лучше в собственных глазах, проецируя на ближних позитивные намерения и поступки».

Чего же следует ожидать от непрошеной заботы? «Спасаемые» увиливают от спасения, не предлагая взамен долгожданной компенсации, что, естественно, отзывается в душе супергероя горьким разочарованием. «Я делаю все для всех, но никто никогда ничего не делает для меня»,  ---  типичная жалоба отвергнутого спасителя.

«Последствия амбивалентного синдрома непременно дадут о себе знать,  ---  пишут калифорнийские психологи Мэри Ламиа и Мэрилин Кригер в книге „Синдром Белого Рыцаря: спасение себя от потребности спасти других“ (The White Knight Syndrome: Rescuing Yourself from Your Need to Rescue Other).  ---  В начале отношений спаситель кажется удовлетворенным своей самоотверженностью, но со временем становится все более несчастным и бессильным. Она/он буквально выдыхается, теряя смысл, интерес, энергию, ресурсы, что, в свою очередь, отражается на самооценке. Убедившись, что усилия напрасны, «белый рыцарь» выходит из игры эмоционально и психологически истощенным».

\textbf{В чем (опять) виноваты детские травмы}

Потребность спасать указывает на эмоциональный и психологический дисбаланс, ноги которого предсказуемо растут из воспитания, образования и внушенных ценностей. «Она/он спасает всех вокруг, стараясь быть „хорошей девочкой/мальчиком“, чтобы получить одобрение со стороны реального или внутреннего родителя и укрепить самооценку,  ---  описывает природу травмы доктор Русселе.  ---  Возможно, в детстве „спасителю“ приходилось помогать больной матери, заботиться о братьях и сестрах, с раннего возраста посвящая себя нуждам взрослых и опуская свои потребности в нижний ранг приоритетов».

Пресловутый дисбаланс имеет тенденцию преследовать «спасителя» до зрелости, поддерживая в ней/нем потребность окружить себя партнерами, друзьями и коллегами, о которых нужно будет заботиться. И эта иллюзорная стратегия предсказуемо обречена на неудачу, потому что суть всего успешного базируется на гармонии.

\textbf{Как избавиться от синдрома спасителя}

От самоотверженного поведения никто не застрахован, но, к счастью, существуют когнитивные методики, которые помогут скорректировать психологический разлад. «Чтобы избавиться от непреходящего стремления быть кому-то нянькой, бросьте все силы на прокачку самоуважения и любви к себе,  ---  призывают Мэри Ламиа и Мэрилин Кригер.  ---  «Спасителям» надо усилием воли сменить тактику, признав наконец, что их любят не за сервис, который они обеспечивают, а за то, кто они на самом деле».

Увязшим в «спасающем» паттерне непросто пойти на поправку  ---  изменить отношение к себе и окружающему миру мешает страх перед возможным одиночеством. А что, если опекаемые отвернутся и забудут насовсем? А что, если вместе с заботами о других из жизни исчезнет смысл?

Чтобы заблокировать страхи, доктор Русселе советует свести помощь к «контрактному» формату, а попросту  ---  договориться. «Если хотите подстраховать себя от разочарований, вместо того чтобы помогать без спроса, обсудите напрямую, чем вы можете быть полезны для конкретного человека. Так вы заранее поймете, готовы ли оказывать услуги без ожиданий безграничной признательности и вдобавок проявите реальную заботу о себе  ---  в кои-то веки. К тому же это хорошая практика в соблюдении личных границ, на которые каждый из нас имеет право».

\newpage
\section{Мой характер}
\textit{Источник: \url{https://ru4.ilovetranslation.com/yuYUND7JV3L=d/}}

О себе говорить приятно, но немного трудно. Приятно, потому что всем нравится говорить о своих интересах, вкусах и предпочтениях. Но это в то же время трудно, так как изучить человека, особенно себя самого, не так уж просто.

Прежде чем говорить о своем характере, хотелось бы сначала уточнить, что такое характер. Человек отличается от остальных своими качествами. Часто люди говорят, что я не такой как остальные. Но я не считаю, что я какой-то особенный. В темноте все кошки серые. Но если вы подойдете ближе и включите свет, вы увидите, что мне присущи определенные черты.

Но не будем вдаваться в подробности, и немного сократим рассказ. У меня хорошее чувство юмора, я ответственный, трудолюбивый и эмоциональный человек. Мне нравится творчество, и я ценю эту черту в других людях. Я не люблю ложь и чувствую, когда другие лгут.

Я стараюсь никогда не опаздывать и \explain{терпеть}{to brook} не могу, когда другие не приходят \explain{вовремя}{on time}. Я предпочитаю общаться с умными и вежливыми людьми. \explain{Досадно}{it's annoying}, когда тот, кому ты \explain{доверяешь}{доверять/доверить: to trust}, оказывается \explain{ненадежным}{ненадежный: unreliable} человеком.

Я стараюсь \explain{обращаться}{to treat / обратиться} с другими так, как я хотел бы, чтобы они обращались со мной. Я ищу человека со здоровым и сильным ум\'{о}м и телом. Человека, с которым интересно общаться, которому я могу доверять и на кого можно положиться.

Что касается моих интересов, мне нравится психология в плане общения с людьми, а также способа формирования мыслей наилучшим образом. Я очень люблю путешествовать, встречаться с новыми людьми, знакомиться с их традициями и обычаями, их культурой, смотреть достопримечательности. Мне также нравятся разные стили музыки, нравится ритмичная музыка, под которую можно танцевать.

\newpage
\section{Психоанализ Зигмунда Фрейда}

\textit{Предпосылки и базовые идеи за 5 минут}

\textit{Источник: \url{https://bit.ly/3sfBWgd}}

Изучением психики человека уже не один десяток лет занимаются великие умы, но на многие вопросы ответов до сих пор нет. Что скрывается в глубинах человеческого существа? Почему события, произошедшие когда-то в детстве, по сей день оказывают влияние на людей? Что заставляет нас совершать одни и те же ошибки и мёртвой хваткой держаться за опостылевшие отношения? Где берут своё начало сновидения, и какая информация в них заложена? На эти и множество других вопросов, относительно психической реальности человека, может ответить революционный и поправший собой многие основы психологии психоанализ, созданный выдающимся австрийским учёным, неврологом и психиатром Зигмундом Фрейдом.

\textbf{Как появился психоанализ?}

В самом начале своей деятельности Зигмунд Фрейд успел поработать с выдающимися учёными своего времени – физиологом Эрнстом Брюкке, практикующим гипноз врачом Иосифом Брейером, неврологом Жаном-Маре Шарко и другими. Часть мыслей и идей, которые зародились на этом этапе, Фрейд развивал и в своих дальнейших научных трудах.

Если говорить более конкретно, то ещё молодого тогда Фрейда привлекло то, что некоторые симптомы истерии, проявлявшиеся у больных ею, не могли никак быть интерпретированы с физиологической точки зрения. К примеру, человек мог ничего не чувствовать в одной области тела, несмотря на то, что в соседних областях чувствительность сохранялась. Ещё одним доказательством того, что далеко не все психические процессы могут быть объяснены реакцией человеческой нервной системы или актом его сознания, было наблюдение за поведением людей, которые подвергались гипнозу.

Сегодня все понимают, что если находящемуся под гипнозом человеку внушить приказ что-либо выполнить, после своего пробуждения он бессознательно будет стремиться к его выполнению. А если поинтересоваться у него, почему он хочет это сделать, он сможет привести вполне адекватные объяснения своему поведению. Отсюда и получается, что психика человека имеет свойство самостоятельно создавать объяснения каким-то поступкам, даже если в них нет никакой необходимости.

В современность Зигмунда Фрейда само понимание того, что действиями людей могут управлять скрытые от их сознания причины, стало шокирующим откровением. До исследований Фрейда таких терминов как «подсознательное» или «бессознательное» не было вовсе. И его наблюдения стали отправной точкой в развитии психоанализа – анализа человеческой психики с позиции движущих ею сил, а также причин, последствий и воздействия на последующую жизнь человека и состояние его нервно-психического здоровья опыта, полученного им в прошлом.

\textbf{Базовые идеи психоанализа}

Теория психоанализа зиждется на том утверждении Фрейда, что в психической (если удобнее – душевной) природе человека не может быть непоследовательности и перерывов. Любая мысль, любое желание и любой поступок всегда имеет свою причину, обусловленную сознательным или бессознательным намерением. События, имевшие место в прошлом, влияют на будущие. И даже если человек убеждён, что какие-либо его душевные переживания не имеют оснований, всегда присутствуют скрытые связи между одними событиями и другими.

Исходя из этого, Фрейд разделял психику человека на три отдельные области: область сознания, область предсознания и область бессознательного.

\begin{enumerate}
    \item К области бессознательного относятся бессознательные инстинкты, никогда не доступные сознанию. Сюда же можно отнести вытесненные из сознания мысли, чувства и переживания, которые воспринимаются сознанием человека как не имеющие права на существование, грязные или запрещённые. Область бессознательного не подчиняется временным рамкам. Например, какие-то воспоминания из детства, вдруг снова попав в сознание, будут такими же интенсивными, как и в момент своего появления.
    \item К области предсознания относится часть области бессознательного, способная в любой момент стать доступной для сознания.
    \item Область сознания включает в себя всё то, что осознаёт человек в каждый момент своей жизни.
\end{enumerate}

Основными действующими силами человеческой психики, согласно идеям Фрейда, являются именно инстинкты – напряжения, которые направляют человека к какой-либо цели. И эти инстинкты включают в себя два главенствующих:

\begin{enumerate}
    \item Либидо, являющееся энергией жизни
    \item Агрессивная энергия, являющаяся инстинктом смерти
\end{enumerate}

Психоанализ рассматривает, по большей части, либидо, в основе которого лежит сексуальная природа. Оно представляет собой живую энергию, характеристики которой (появление, количество, перемещение, распределение) могут истолковать любые психические расстройства и особенности поведения, мыслей и переживаний индивида.

Личность человека, согласно психоаналитической теории, представлена тремя структурами:
\begin{enumerate}
    \item Оно (Ид)
    \item Я (Эго)
    \item Сверх-Я (Супер-Эго)
\end{enumerate}

Оно (Ид) является всем изначально заложенным в человеке – наследственностью, инстинктами. На Ид никак не влияют законы логики. Его характеристики  ---  это хаотичность и неорганизованность. Но Ид воздействует на Я и Сверх-Я. Причём, его воздействие безгранично.

Я (Эго) является той частью личности человека, которая находится в тесном контакте с окружающими его людьми. Эго берёт своё начало из Ид с того самого момента, когда ребёнок начинает осознавать себя как личность. Ид питает Эго, а Эго защищает его, словно оболочка. То, как взаимосвязаны Эго и Ид, легко отобразить на примере потребности в сексе: Ид могло бы осуществить удовлетворение этой потребности посредством прямого сексуального контакта, но Эго решает, когда, где и при каких условиях этот контакт может быть реализован. Эго способно перенаправлять или сдерживать Ид, тем самым являясь гарантом обеспечения физического и душевного здоровья человека, а также его безопасности.

Сверх-Я (Супер-Эго) произрастает из Эго, являясь хранилищем моральных устоев и законов, ограничений и запретов, которые накладываются на личность. Фрейд утверждал, что Сверх-Я выполняет три функции, коими являются:
\begin{enumerate}
    \item Функция совести
    \item Функция самонаблюдения
    \item Функция, формирующая идеалы
\end{enumerate}

Оно, Я и Сверх-Я необходимы для совместного достижения одной цели – поддержания равновесия между стремлением, ведущим к увеличению удовольствия, и опасностью, возникающей от неудовольствия.

Возникшая в Оно энергия отражается в Я, а Сверх-Я определяет границы Я. Учитывая то, что требования Оно, Сверх-Я и внешней реальности, к которой должен приспособиться человек, нередко являются противоречивыми, это неизбежно приводит к внутриличностным конфликтам. Решение же конфликтов внутри личности происходит посредством нескольких способов:
\begin{enumerate}
    \item Сновидения
    \item Сублимация
    \item Компенсация
    \item Блокировка механизмами защиты
\end{enumerate}

Сновидения могут быть отражением желаний, не реализованных в реальной жизни. Сновидения, которые повторяются, могут быть указателями на определённую потребность, которая не была реализована, и которая может служить помехой на пути свободного самовыражения человека и его психологического роста.

Сублимация является перенаправлением энергии либидо на цели, одобряемые обществом. Нередко такими целями выступает творческая, социальная или интеллектуальная деятельность. Сублимация есть форма успешной защиты, а сублимированная энергия создаёт то, что все мы привыкли называть словом «цивилизация».

Состояние тревожности, которое возникает от неудовлетворённого желания, есть возможность нейтрализовать через прямое обращение к проблеме. Так, энергия, которая не может найти выхода, будет направлена на преодоление препятствий, на уменьшение последствий этих препятствий и на компенсацию того, чего не хватает. В качестве примера можно привести идеальный слух, который развивается у слепых или слабовидящих людей. Человеческая психика способна поступить аналогичным образом: к примеру, у человека, страдающего недостатком способностей, но имеющего сильнейшее желание достичь успеха, может развиться непревзойдённая работоспособность или беспримерная напористость.

Однако бывают и такие ситуации, в которых появившееся напряжение может быть искажено или отвергнуто особыми защитными механизмами, такими как гиперкомпенсация, регрессия, проекция, изоляция, рационализация, отрицание, подавление и другими. Например, неразделённую или потерянную любовь можно подавить («Не помню никакой любви»), отвергнуть («Да любви и не было»), рационализировать («Те отношения были ошибкой»), изолировать («Мне не нужна любовь»), спроецировать, приписав другим свои чувства («Люди не умеют любить по-настоящему»), гиперкомпенсировать («Я предпочитаю свободные отношения») и т.д.

\textbf{Краткое резюме}

Психоанализ Зигмунда Фрейда – это величайшая попытка прийти к пониманию и описанию тех составляющих психической жизни человека, которые до Фрейда были непостижимыми. Самим же термином «психоанализ» в настоящее время называют:

\begin{enumerate}
    \item Научную дисциплину
    \item Комплекс мероприятий по исследованию психических процессов
    \item Методику лечения нарушений невротического характера
\end{enumerate}


Работа Фрейда и его психоанализ даже сегодня нередко критикуются, однако те понятия, которые он ввёл (Ид, Эго, Супер-Эго, механизмы защиты, сублимация, либидо) понимаются и применяются в наше время как учёными, так и просто образованными людьми. Психоанализ нашёл своё отражение во многих науках (социологии, педагогике, этнографии, антропологии и других), а также в искусстве, литературе и даже кинематографе.

\newpage
\section{Номофобия: позвони мне, позвони...}

\textit{Источник: \url{https://4brain.ru/blog/nomofobiya-pozvoni-mne-pozvoni/}}

Когда-то весь мир был театром, а люди в нем – актерами. Сейчас весь мир превратился в Интернет, и Интернет поместился в телефон, а люди стали просто юзерами.

Насколько это хорошо или плохо, что с этим делать и нужно ли что-то с этим делать в принципе, вы сможете ответить, если пройдете наши программы «Когнитивистика» и «Психическая саморегуляция». А наша сегодняшняя тема – номофобия.

\textbf{Что такое номофобия: немного истории}

Слово «номофобия» пришло к нам из английского языка. Термин «nomophobia» происходит от выражения no mobile-phone phobia, что означает страх остаться на какое-то время без мобильного телефона. Впервые термин «номофобия» был введен в статье Nomophobia is the fear of being out of mobile phone contact  ---  and it’s the plague of our 24/7 age («Номофобия это боязнь остаться вне связи с мобильным телефоном – и это чума нашего века 24/7») [YouGov, 2008].

Тогда и была поднята новая на тот момент проблема – номофобия, а статья подводила итоги исследования, проведенного по заказу UK Post Office. В опросе приняли участие более двух тысяч человек, и оказалось, что 48\% женщин и 58\% мужчин испытывают беспокойство, если остаются без мобильной связи по какой-либо причине (забыли телефон, села батарея, нет сети и прочее) [YouGov, 2008].

Можно сказать, что номофобия – это зависимость от телефона, потому что без телефона номофоб себя ощущает крайне некомфортно. Считается, что номофобия появилась в распоряжении людей одновременно с мобильным телефоном, хотя ее истоки отчетливо прослеживаются в намного более раннем периоде. Будет правильнее сказать, что боязнь отойти далеко от телефона появилась одновременно с телефонами.

С появлением первых стационарных телефонов в серьезных организациях секретарши крупных и средних начальников боялись выскочить на 5 минут в «дамскую комнатку», потому что по закону подлости именно в эти 5 минут звонил их непосредственный начальник.

В «доцифровую» эпоху в разных конторах от ЖЭКа и почты до городской справочной и центральной прачечной сидящая «на телефоне» сотрудница должна была попросить кого-то подежурить «у аппарата», пока у нее перекур, чтобы не оставить без ответа звонки, поступившие в данный отрезок времени.

Если же подменить «человека на телефоне» было некому, решившийся покинуть свой «боевой пост» в страхе молился, чтобы никто не позвонил в тот момент, пока он курит или заваривает себе чай, потому что не взятая после третьего гудка трубка приравнивалась к отсутствию на рабочем месте без уважительных причин.

Такая «ноуфонфобия» поддерживалась жесткой трудовой дисциплиной и санкциями за прогулы, прописанными в Трудовом законодательстве. Сейчас большинство организаций обзаводятся цифровыми каналами связи, а не дозвонившийся по телефону может оставить сообщение в чате и дождаться ответа в асинхронном режиме.

Поэтому сегодня «телефонобоязнь» распространена, разве что, в совсем консервативных организациях, где пока нет системы контроля продуктивности, основанной на конкретных показателях работы, а присутствие на рабочем месте остается единственным мерилом ценности сотрудника.

Зато появилась другая напасть: теперь люди неуютно себя чувствуют, если забыли дома мобильник, когда собирались в магазин за хлебом. Или если долго беседуют с домочадцами, и уже полчаса как не проверяли сообщения «в телефоне». А если человек уехал на работу и обнаружил отсутствие гаджета в кармане уже сидя за рулем, это полностью уважительная причина, чтобы вернуться домой за смартфоном.

На самом деле, если вы ждете важного звонка или сообщения, такое поведение полностью оправдано. И если ваша работа обязывает вас быть всегда на связи, сидите ли вы за компьютером, за рулем или вышли покурить, телефон лучше держать при себе. Равно как нежелательно оставлять свой смартфон без присмотра, если там собраны все ваши приложения и аккаунты с сохраненными паролями для входа. Хотя тут лучше бы продумать дополнительные меры кибербезопасности.

А вот если ничего важнее «котиков» в соцсетях и сплетен от подружек у вас сегодня не предвидится, а вы все равно ни на минуту не можете отвлечься от телефона, тут впору говорить про киберлофинг, фаббинг, а то и про номофобию. Как понять, что у человека номофобия – боязнь остаться без телефона, и как отличить ее от объективной необходимости быть на связи? Как отличить номофобию или зависимость от телефона от простого желания быть в курсе событий? Давайте поговорим об этом.


\textbf{Номофобия: симптомы и последствия}

Итак, каковы же симптомы номофобии? Психоневрологи выделяют несколько четких сигналов, позволяющих считать, что тяга к телефону обрела болезненный характер [Н. Гартман, 2019].

Топ-5 признаков номофобии:

\begin{enumerate}
    \item Волнение, нарастающее по мере снижения уровня заряда аккумулятора.
    \item Постоянная проверка наличия новых писем, сообщений, оповещений.
    \item Постоянное обновление ленты новостей и чтение одних и тех же новостей «по кругу».
    \item Постоянное присутствие в соцсетях, мессенджерах и на прочих онлайн-платформах для общения.
    \item Сильная боязнь испортить гаджет.
\end{enumerate}

Все эти признаки часто сопровождаются явными физическими проявлениями, такими как волнение, тревога и даже нервный тик и тремор в конечностях, если в какой-то момент вдруг не окажется под рукой гаджета, а человек не сможет быстро вспомнить, где он его оставил [Н. Гартман, 2019].

Особо мнительные и психически неустойчивые люди подвержены еще более сильным физиологическим реакциям: усиленное сердцебиение, потоотделение, панические атаки, потеря ориентации в пространстве [Н. Копылова, 2021].

В очень тяжелых случаях возможны и долгосрочные последствия номофобии:
\begin{enumerate}
    \item Усталость и бессонница.
    \item Ослабление коммуникативных навыков.
    \item Склонность к социопатии и нелюдимость.
    \item Грубость и агрессия.
    \item Снижение когнитивных способностей (память, интеллект, концентрация внимания).
    \item Эмоциональная холодность.
    \item Неспособность выразить свои чувства словами.
\end{enumerate}

Почему так происходит? Почему современные люди часто впадают в жесткую зависимость от гаджета? Давайте разберемся и с этим.

\textbf{Номофобия: причины}

На самом деле, тему зависимости от телефона и прочих гаджетов наиболее ярко описывает шутливый диалог, когда юноша сообщает своему приятелю, что купил крутейший смартфон, который умнее человека. В ответ на сомнения и возражения, что такого не может быть, счастливый обладатель умного устройства поясняет, что такое вполне может быть, ведь не зря же он отдал за смартфон 300 тысяч рублей. В этот момент приятель соглашается, что тогда да, смартфон может быть умнее человека, только дело тут не в смартфоне…

Справедливости ради отметим, что в номофобию иногда впадают и вполне образованные люди, а не только малограмотные особи, которых ничего не интересует, кроме пустопорожней болтовни и бессмысленных сообщений с кучей орфографических ошибок. Психотерапевты уже достаточно глубоко исследовали тему и готовы поделиться своим видением причин данного явления [В. Холманских, О. Демьянова, 2020].

Топ-5 причин номофобии:
\begin{enumerate}
    \item Нерешенные личные проблемы, от которых можно отвлечься с помощью телефона.
    \item Сложности с построением отношений и навыками коммуникации в офлайне.
    \item Желание презентовать себя в лучшем свете в виртуальном пространстве, что невозможно без помощи электронных инструментов.
    \item Желание чувствовать себя важным, нужным и осведомленным в любой момент времени.
    \item Страх изоляции – социальной, информационной, прочей.
\end{enumerate}

Заметим, что с началом пандемии страх изоляции обрел реальные очертания. Когда все и везде переходят на «удаленку», выключенный гаджет или телефон вне зоны доступа сродни изоляции от событий внешнего мира. И это касается абсолютно всех, а не только тех, у кого есть проблемы с отношениями и коммуникацией «в реале».

Однако и тут следует знать меру, потому что за 5 минут, которые вам необходимы, чтобы заварить чай, в жизни обычного офисного служащего вряд ли произойдет нечто судьбоносное, что требует сиюминутной реакции и не подождет вашего ответа, пока вы спокойно допьете чашечку горячего чая с бубликом или конфетой.

Для описания причин номофобии иногда используют такой термин, как «эскапизм», он же «эскепизм» или «эскейпизм». Под эскапизмом подразумевается сознательное избегание всего неприятного и рутинного, что есть в этой жизни, в том числе путем «зависания» в телефоне. Тогда каждый раз, когда у человека по какой-либо причине исчезает потенциальная возможность «погрузиться» в телефон, он испытывает беспокойство [В. Холманских, О. Демьянова, 2020].

Такой поход имеет право на жизнь в качестве одной из причин номофобии. Во-первых, причин номофобии гораздо больше, чем просто стремление постоянно отвлекаться от скуки бытия. Во-вторых, термин «эскапизм» гораздо шире, чем просто зависание в телефоне. Сюда же относится бегство от действительности путем погружения в творчество, чтение, размышления, духовные практики, какую-либо иную реальность, отличную от существующей вокруг.

Так ли все плохо с номофобией и может ли быть от нее какая-то польза? Давайте посмотрим.

\textbf{Номофобия: польза и вред}

Думается, вред от избыточной привязанности к телефону вполне очевиден из всего вышеизложенного. Тревожность, беспокойство, неприятные физиологические реакции, а в перспективе снижение памяти, концентрации внимания и прочие проблемы – это вполне достаточные основания говорить о том, что номофобия вредна.

Эти выводы подтверждают и целевые научные исследования среди различных социальных групп. Можем рекомендовать по теме номофобии статьи, опубликованные в ведущих научных изданиях.

Например, статью «Номофобия и опасность для здоровья: использование смартфонов и зависимость среди студентов вузов» («Nomophobia and Health Hazards: Smartphone Use and Addiction Among University Students») [A. Daei et al., 2019].

Или, к примеру, статью, посвященную «Последствиям чрезмерного использования мобильного телефона и психологическим рискам среди штатных медсестер» («Effects of Excessive Use of Mobile Phone and Psychological Hazards among Staff Nurses») [O. Swami et al., 2021]. Скажем прямо, что номофобия в той или иной степени затронула многие социальные группы, а наносимый данной фобией вред стал причиной пристального внимания ученых и медиков.

Однако есть исследователи, которые считают, что «Нет больше фобии о номофобии» («No more phobia about nomophobia») [CityU, 2019]. Так, группа южнокорейских ученых пришла к выводу, что страдающие номофобией люди сильнее вовлечены в работу и в большей степени переживают за результат, нежели те, кто покидает рабочий чат минута в минуту с окончанием рабочего дня. Как следствие, номофобы способны выполнить больший объем работы и добиться лучших результатов.

Но и в этом случае ученые признают, что все хорошо в меру, потому что перманентное пребывание на связи в режиме «25/8» чревато эмоциональным выгоранием и далее постепенным снижением продуктивности. Так или иначе, вреда от номофобии явно больше, чем пользы, поэтому от нее нужно вовремя избавляться, а лучше вовсе не доводить себя до такого состояния.

\textbf{Как победить номофобию?}

Сегодня можно найти массу советов по цифровому детоксу в целом, и борьбе с номофобией в частности. В большинстве случаев рекомендуется сразу переходить к ограничительным мерам: не пользоваться телефоном какое-то время (час, день, неделю), просматривать сообщения строго определенное количество раз в течение суток, удалить наиболее отвлекающие приложения, закладки, аккаунты в соцсетях.

Однако гораздо продуктивнее подход, рекомендующий сначала разобраться в причинах собственной номофобии, а уже потом переходить к каким-то действиям [Н. Копылова, 2021]. Точнее, сначала следует убедиться, что у вас есть признаки номофобии.

Выше мы уже говорили о том, как распознать номофобию, однако можем вам облегчить задачу, предложив ответить «да» или «нет» на следующие утверждения:
\begin{enumerate}
    \item Перед тем, как лечь спать, вы проверяете телефон.
    \item Первым делом после пробуждения вы проверяете телефон.
    \item Вы никогда не выключаете телефон.
    \item Вы паникуете, если осталось менее 30\% заряда аккумулятора.
    \item Вы возвращаетесь, если забыли телефон дома, даже если вышли ненадолго.
    \item Вы носите телефон с собой везде, даже дома из комнаты в комнату.
    \item Вы стараетесь отвечать на все письма и сообщения моментально.
    \item Во время живой беседы и любого другого офлайн-занятия вы постоянно прерываетесь, чтобы проверить телефон.
    \item Вы нервничаете, если исчезает мобильная сеть или вай-фай.
\end{enumerate}

Каждый ответ «да» повышает вероятность того, что вас настигла номофобия. Если таких ответов больше трех, пора задуматься, что именно вас пугает в том, чтобы остаться без телефона на какое-то время. Вы ждете какое-то мегаважное сообщение? Однако вряд ли в вашей жизни так уж много ситуаций, когда должны свершиться какие-то суперсобытия.

Вы боитесь, что вас не найдет начальник, когда вы очень нужны? Основной массив офисной работы вполне допускает асинхронный режим общения и решения проблем. Лучше обговорить заранее, что вы будете перезванивать или отвечать на сообщение не позднее, чем в течение часа, чтобы не отвлекаться от переговоров с заказчиками. Заодно выиграете время на все случаи жизни, чтобы обдумать ответ.

Ваша жена устраивает вам сцены ревности, если вы не берете трубку сразу, как только она вам позвонила? Это достаточный повод, чтобы задуматься, насколько здоровые отношения вам удалось построить в семье, и принять меры, чтобы их оздоровить, пока не поздно.

И только после того, как вы справитесь с первопричиной вашего беспокойства, борьба с номофобией как таковой сможет принести результат. Итак, как же избавиться от телефонной зависимости?

Топ-7 способов справиться с номофобией:
\begin{enumerate}
    \item Заранее заряжать телефон, чтобы он не разрядился в неподходящий момент, и брать с собой пауэр-банк, если планируется длительное пребывание вне помещения.
    \item Завершать свои рабочие дела так, чтобы минимизировать вероятность внеплановых звонков в нерабочее время.
    \item Найти интересные увлекательные занятия в повседневной жизни (спорт, танцы, общение с друзьями), от которых не захочется отвлекаться на телефон.
    \item Освоить навыки тайм-менеджмента и выделить время, когда вы будете заниматься исключительно работой, домашними делами или хобби, не отвлекаясь на соцсети и уведомления.
    \item Отключить уведомления о событиях, которые не требуют мгновенной реакции (обновления в соцсетях, сообщения о рассылках и т.д.)
    \item Выделить своему телефону персональное место в квартире, на столе, на тумбочке, которое будет занимать только он, и больше никто и ничто.
    \item Разнообразить способы получения информации.
\end{enumerate}

По поводу последнего пункта уточним, что время можно узнать, посмотрев на часы на руке, книгу можно почитать бумажную, а не электронную, в выходной можно сходить в кино и не искать новый фильм на торрент-трекерах.

Стоит ли вам удалять все подряд приложения из телефона, чтобы они вас не отвлекали, решайте сами. В условиях санкций и ограниченной доступности обновлений это может оказаться не лучшим шагом, если вдруг окажется, что то или другое приложение вам все-таки нужно.

Мы уже говорили о том, что все хорошо в меру, и борьба с номофобией тоже. Избыточная требовательность к себе и беспокойство по поводу того, что вы неидеально распоряжаетесь своим временем, ничуть не полезнее беспокойства по поводу того, чтобы всегда быть «на связи».

Если же вам не удается справиться с проблемой своими силами, есть смысл обратиться к психологу или психотерапевту. На сегодняшний день наработан достаточно объемный опыт психологической помощи людям, впавшим в какую-либо специфическую зависимость. В медицине расстройства такого рода носят название «Специфические (изолированные) фобии» и обозначаются кодом F40.2 [classinform, 2021].

При необходимости может быть назначено медикаментозное лечение. Обращаем ваше внимание, что диагностику и лечение может проводить только врач. Однако вы вполне можете поддержать свой организм в минимальных дозировках такими средствами, как настойка валерианы и витамины группы В [kb, 2019].

И, конечно, оптимальным вариантом будет профилактика и недопущение такого явления, как номофобия в свою жизнь. Отрадно, что этой теме начали уделять внимание в действующей системе образования уже на уровне средней школы.

Например, посвящен теме номофобии проект-исследование «Насколько мы зависимы от телефона» [Ю. Левкина, И. Рубель, 2020]. В ходе реализации данного проекта авторы попытались не только выявить масштабы зависимости от гаджетов среди школьников 7-11 классов, но и выработать меры, позволяющие соблюдать баланс между цифровой активностью и обычным общением, научиться пользоваться телефоном без ущерба для учебного процесса и не мешая окружающим.

Такой подход дает надежду, что со временем тема номофобии станет не столь актуальной, как сейчас, а люди, беря в руки телефон, будут пользоваться цифровыми благами мира и не чувствовать себя беспомощными, если Google вдруг не сможет ответить на их вопрос прямо сейчас.

Впрочем, весь мир театр – как был, так и остался. И пусть этот театр иногда виртуальный, по-настоящему активным людям это не мешает играть главную роль на подмостках своей жизни, не впадая в зависимость от чего бы то ни было.

Мы желаем, чтобы пользование гаджетами было для вас исключительно удобно, комфортно и с пользой для дела. Мы приглашаем вас на наши программы «Когнитивистика» и «Психическая саморегуляция». И просим вас ответить на один вопрос по теме статьи...

\newpage
\section{Профессия психолог: особенности и преимущества специальности}

\textit{Источник: \url{https://4brain.ru/}}

Человек по своей природе склонен даже в спокойное время постоянно испытывать разные внутренние психологические проблемы, такие как трудности в семье, недопонимание с близкими, неуверенность, чувство одиночества, страхи или сезонные депрессии. Что же говорить о периодах каких-то глобальных кризисов, происходящих в последнее время во всем мире…

Людей все больше охватывают страх и паника за свою жизнь, многие оказываются в стрессовых состояниях, которые крайне негативно влияют на здоровье, а со временем могут развиваться в разные фобии.

Имея огромное желание справиться со своими внутренними терзаниями, люди начинают понимать, что обычные беседы «по душам» им уже не помогают, а глубокие внутренние проблемы сможет решить только профессиональный психолог [need4study.com, 2018].

Во многих западных странах уже очень давно введена практика частной психологии. Многие влиятельные лица и знаменитости, а также простые люди давно имеют своих личных психологов, с которыми регулярно консультируются по тем или иным вопросам.

\textbf{Психолог: суть профессии}
«Психология» переводится с греческого языка как «наука о душе». С лингвистической точки зрения термины «психика», «душа» означают одно и тоже. Но со временем, в процессе развития культуры и науки, значения этих понятий разошлись.

Сегодня психология является в первую очередь дисциплиной, которая имеет научное определение, задачи, цели, объекты и все то, что делает ее наукой.

Профессия психолог нужна для изучения личностных черт людей, а также для понимания особенностей мышления и взаимосвязи с окружающими. Профессиональные специалисты помогают людям разными эффективными психологическими приемами как в личных отношениях, так и в профессиональной деятельности.

Роль психолога заключается в том, чтобы помочь человеку пройти через разные сложные ситуации в жизни и найти потенциал для того, чтобы двигаться дальше, используя для этого все необходимые техники. Опытный специалист даст совет как лучше поступить, и при этом он обязательно прислушается к индивидуальным потребностям и ценностям каждого клиента.

Положительные качества консультации в том, что человек, пришедший на нее, не просто пассивно принимает назначенные врачом методики, но и сам активно участвует в том, что происходит. Он исследует и познает собственные трудности и пытается выразить свои ощущения/мысли для дальнейшего тщательного анализа.

Психолог помогает прояснить какие-то события, определить связь между ними и поведением самого человека, старается узнать, как эти связи влияют на будущее и как отражались в прошлом. Другими словами, он помогает избавиться от ненужных переживаний, стать увереннее в себе при достижении поставленных целей и сделать произошедшие изменения максимально устойчивыми.

Хороший психолог всегда старается создать наиболее приемлемые условия для самовыражения человека. Он может большую часть времени просто выслушивать, но при этом всегда подмечает какие-то важные детали, которые обычно ускользают от простого взгляда. Поэтому психолог способен максимально прояснить и выразить свое видение на ту или иную проблему. Он может находить взаимосвязи между мыслями, чувствами и поступками людей, тем самым оказывая огромную помощь в достижении поставленных задач перед человеком.

Чем может помочь профессиональный психолог? Наиболее значимые пункты:
\begin{enumerate}
    \item лучше понимать своих близких;
    \item обрести гармонию со своим внутренним «я»;
    \item быстро находить выход из разных напряженных ситуаций;
    \item справляться с возникающими трудностями;
    \item  решать проблемы в личных отношениях;
    \item справляться с внутренней болью;
    \item избегать конфликтов;
    \item повышать качество жизни путем внутреннего самопознания;
    \item повышать самооценку и уверенность в собственных силах;
    \item бороться с тревогами, фобиями и депрессиями;
    \item преодолевать возрастные кризисы;
    \item решать трудности переходного возраста в подростковом периоде;
    \item избегать разногласий в воспитании детей и в отношениях с родителями;
    \item решать проблемы с коллегами в коллективе или с начальством;
    \item справляться с горем при потере/смерти близкого человека [my-self.ru, 2020].
\end{enumerate}

Многие люди выбирают психологию в качестве своей основной работы.

\textbf{Профессия психолог: специализации}

Итак, психолог. Эта профессия в последние годы стала очень популярной. Существует много разных вариантов деятельности для тех, кто получил психологическое образование, т.к. перед дипломированными специалистами всегда открыто больше дверей и возможностей.

Все больше специалистов этой области находят себя в разных современных направлениях, таких как бизнес-тренерство и коучинг. Часто они могут останавливать свой выбор на HR-службах. Наиболее популярными направлениями в последнее время являются следующие специальности.

\textbf{Клинический психолог}

Большинство людей данного профиля выбирает работу в сфере здравоохранения. Это может быть профессия клинического психолога или специалиста по судебной медицине. Сюда же относятся психоаналитики, различные консультанты служб психологической поддержки по телефону.

Деятельность клинического специалиста-психолога заключается в оказании помощи людям с разного рода зависимостями (алкоголизм, курение, наркотики). Сюда же входят нтернет-зависимость, игромания, психогенное переедание и т.д.

Клинический психолог работает с людьми, имеющими разные психологические отклонения/уязвимости, и может оказывать поддержку в реабилитационных службах, помогая тем, кто запутался в жизненных обстоятельствах, перенес тяжелые психотравмы, а также тяжелобольным и ВИЧ-инфицированным.

\textbf{Школьный психолог}

Очень востребованной является профессия педагог-психолог, особенно в частных учебных заведениях. Школьные специалисты способны помочь детям приспособиться к новым условиям за короткое время с наименьшими потерями для психики.

Школьный специалист на основе определенных тестов видит насколько ребенок готов к обучению. Также он проводит персональные беседы с трудными подростками с целью выявления отклонений.

В работу психолога, работающего с детьми, входит проведение различных тренингов, которые в дальнейшем помогают детям подобрать нужную для себя профессию.

Данный специалист оказывает большую помощь в развитии нормальных отношений в любых детских и подростковых коллективах.

\textbf{Семейный психолог}

Очень часто на консультацию к психологам приходят люди, имеющие проблемы в семье. Целью деятельности этого специалиста является работа с людьми, находящимися в долгих отношениях. Основная задача психолога здесь – помочь партнерам понять друг друга и постараться устранить все возможные непонимания и проблемы.

Консультации у опытного профессионала пригодятся на всю жизнь – люди смогут качественно проработать отношения, преодолеть собственный кризис, а также решить проблемы с другими членами семьи, включая детей и родителей, чтобы вывести отношения на новый уровень.

Хороший специалист по семейным вопросам отлично знаком с общей клинической картиной, понимает психологию развития, учитывает все тонкости отношений и применяет диагностику в совокупности со специальными методами терапии.

\textbf{Спортивный психолог}

Психология очень хорошо развита в спортивной среде. Здесь работа специалистов заключается в том, чтобы настроить спортсмена на победу и вселить уверенность в собственных силах. А также подготовить к соревнованиям без страха и лишних тревог.

Специалист проводит большую психологическую работу и помогает избавиться от неуверенности, поддерживает морально и обучает специальным методикам, способным быстро снимать стресс [www.kp.ru, 2020].

\textbf{Пенитенциарный психолог}

Работа данного специалиста заключается во взаимодействии с заключенными.

Целью деятельности этих психологов является оценка психического состояния осужденных, предупреждение любых попыток суицида, максимальная психологическая поддержка, а также помощь в адаптации к тюремным условиям жизни.

Суть профессии заключается в том, чтобы помогать заключенным изменить взгляды и отношения к жизни, а также подготовить их к выходу на свободу.

Хороший специалист данной сферы должен обладать смелостью, доброжелательностью, умением держать дистанцию, широким кругозором, харизмой, эмпатией и состраданием, а также желанием помогать людям [studika.ru, 2020].

\textbf{Психологи МЧС и МВД}

Специалисты данной области проводят психологическую оценку при приеме сотрудников на работу в силовые структуры, поскольку большинство людей, особенно молодой контингент, нуждаются в моральной поддержке. Ведь работа, так или иначе, связана с насилием.

Работа психологов необходима людям, пришедшим в структуры МВД после того, когда они не сумели найти работу по специальности, а здесь прошли сокращенные курсы за 6 месяцев и получили в руки оружие. Главная задача такого специалиста заключается в том, чтобы убедить сотрудника в правильности своего решения, успокоить его и настроить на дальнейшую работу.

Психологи МВД занимаются изучением социально-психологического климата в коллективе и дают качественные рекомендации по его улучшению. Также они консультируют и помогают адаптироваться военным и людям, вернувшимся недавно из горячих точек [moluch.ru, 2019].

Цель психологов МЧС состоит в том, чтобы помочь людям справиться с сильными переживаниями и горем, возникающими после гибели близких людей или при чрезвычайных ситуациях.

Их задача – выводить людей из острого стрессового состояния, общаться с пострадавшими и успокаивать родственников. Иногда приходится работать и вести переговоры даже с потенциальными самоубийцами.

Другое направление заключается в том, чтобы реабилитировать сотрудников МЧС, проводить с ними большую работу по недопущению профессионального выгорания и развитию моральной устойчивости [httpsvc.ru].

\textbf{Корпоративные психологи}

Здесь специалисты занимаются подбором персонала в коллектив, а также обучением сотрудников разным психологическим методам воздействия на клиентов с целью привлечения большего их количества, развития умения проводить продуктивные переговоры и развивать в себе сильные лидерские качества.

Также одной из задач корпоративного психолога является разрешение конфликтов в компании, улучшение взаимодействия между сотрудниками, мотивация и повышение их стрессоустойчивости.

Как ни странно, но сюда же относятся и вопросы физического оздоровления, которыми также занимаются специалисты. Например, направление психосоматологии сегодня очень хорошо решает эти вопросы, поскольку эмоциональное напряжение в коллективе, а также любые затяжные конфликты могут приводить к росту больничных. В таких случаях на помощь приглашаются квалифицированные корпоративные психологи [trends.rbc.ru, 2019].

Сегодня многие стараются получить профессию психолога. Как стать хорошим специалистом, какие качества для этого нужны и к чему должен быть готов человек при получении профессии?

\textbf{Как стать психологом: особенности обучения}

Чтобы стать психологом, необходимо получить соответствующий диплом о высшем образовании. Для этого после окончания 11 класса следует поступать в университет.

При выборе профессии психолога, чтобы определиться с университетом, нужно отдавать предпочтение престижным учебным заведениям, расположенным в крупных городах.

О том, какие предметы сдавать при поступлении в вуз лучше узнать заранее. Учеба в вузе продлится как минимум 4 года. Если появится желание поступать в аспирантуру или магистратуру, то учиться придется более 6 лет.

На факультете психологии основными предметами являются анатомия, социология, философия, логика, антропология, физиология центральной нервной системы и высшей нервной деятельности, математические действия в психологии и высшая математика. Изучается психология личности и сознания, осваиваются общепрофессиональные дисциплины, история психологии, психология труда, профильная психология и другие направления.

Обучение в высшем заведении включает в себя не только теорию, но и практику – как учебную, так и производственную. Глубина и масштаб полученных знаний будут напрямую зависеть от выбранного вуза, желаний и способностей самого человека, а также технологий обучения, поскольку программы в разных университетах могут кардинально различаться [aif.ru, 2018].

К учебе следует подходить ответственно и серьезно, хорошо подготавливаясь к экзаменам и зачетам, поскольку потом при устройстве на работу некоторые работодатели могут поинтересоваться оценками в дипломе. Окончившие университет студенты-отличники могут претендовать на самые престижные и высокие должности не только в государственных, но и в коммерческих фирмах, где зарплата психолога будет значительно выше.

Начинающий специалист, недавно окончивший высшее учебное заведение, должен постоянно увеличивать объем знаний и качественно улучшать свои профессиональные навыки.

Также обучиться профессии психолога можно на базе 9 классов. Тем, кто решил остановить свой выбор на этом варианте, сложные экзамены типа ЕГЭ сдавать не придется.

В большинство колледжей принимают по среднему баллу в аттестате, но может быть назначено дополнительное собеседование. Такие нюансы всегда зависят от самого учебного заведения.

Предметы, являющиеся обязательными для сдачи экзаменов на факультет Психологии – математика и русский язык, а также на выбор самого учащегося два любых предмета среди биологии, химии и обществознания.

Процесс обучения по специальности Психология в колледже длится от 1 года 10 месяцев до двух лет. Преимущество решивших поступить на психологический факультет после 9 класса в том, что при поступлении в высшее учебное заведение на тот же факультет они могут сразу же перейти на второй либо на третий курс, имея при этом за плечами не очень много опыта [vyuchit.work, 2018].

Чем более качественное будет получено образование, тем выше у человека будет заработная плата.

\textbf{Доходы специалистов}

Доход дипломированного психолога напрямую зависит от его специализации, места работы, уровня квалификации и наличия опыта. Консультирующие психологи, как правило, сначала начинают с невысоких расценок и постепенно повышают их по мере роста своего опыта.

В среднем, практикующий специалист в столице может за один час персональной консультации получать от 2 500 до 4 000 рублей.

В государственных учреждениях средняя зарплата психолога составляет примерно от 45 000 до 90 000 рублей в месяц.

В коммерческих организациях уровень доходов может быть значительно выше [aif.ru, 2020].

\textbf{Профессиональные требования}

Психология – это сфера, где в первую очередь важны именно личные качества специалиста. Далеко не каждый способен стать хорошим практикующим психологом, ведь это не просто прослушал курс и начал искать клиентов. Для того, чтобы успешно работать и помогать людям, необходимо обладать особенными качествами характера, без которых стать профессионалом будет очень трудно.

Психолог должен иметь высокий эмоциональный интеллект, уметь хорошо слушать, а самое главное слышать, быть терпеливым и уметь сопереживать пришедшему на консультацию человеку. В такие моменты клиенты ждут, что им будут сочувствовать в их проблемах и переживаниях.

Специалист этого профиля должен быть очень наблюдательным. Иногда действия и слова могут сильно различаться, а способность улавливать такие изменения всегда будет на руку специалисту.

К примеру, когда человек на консультации говорит, что спокоен, но при этом постоянно теребит свои пальцы, часы, телефон или что-то другое – это напрямую указывает на его нервозность и эмоциональную неустойчивость.

Хороший психолог должен понимать чувства других людей и правильно интерпретировать их действия, подбирая «ключи» к каждому конкретному человеку. Он должен уметь располагать к себе, быть открытым и коммуникабельным, поскольку ему придется много общаться с людьми вне зависимости от своего настроения.

Если при беседе с незнакомыми людьми такой специалист испытывает неловкость, если он не любит длительных разговоров и новых знакомств, ему лучше поискать себя в какой-нибудь другой сфере.

Психолог должен быть тактичным и уметь хранить чужие секреты. Хоть психология и не является медицинской наукой, но правило «врачебной тайны» здесь никто не отменял: хороший психолог никогда и ни при каких обстоятельствах не должен выносить на публику подробности своей работы с людьми.

Также он не имеет права никого осуждать, критиковать, сравнивать с другими или в целом проявлять бестактность по отношению к окружающим. Он должен быть внимательным к переживаниям других людей, и, одновременно с этим, ему нужно уметь абстрагироваться от чужих бед.

Отличный специалист обязан быть непредвзятым по отношению к другим. Ведь за долгие годы работы ему не раз придется встречать людей, чьи взгляды будут кардинально отличаться от его намерений. Поэтому профессиональный психолог не имеет права позволять своим личным убеждениям как-то влиять на работу и на отношение к людям в целом.

Задача настоящего специалиста – помочь человеку самому найти ответы на вопросы и разобраться со своими проблемами, а не навязывать свою точку зрения. Так что он должен быть беспристрастным, тактичным и ответственным.

Психолог не должен работать по каким-то шаблонам, поскольку все люди разные, и то, что помогает одному, может оказаться совершенно бесполезным для другого. А значит, он должен уметь сочетать различные методики и адаптировать их под каждого клиента.

Без этих качеств достойно выполнять обязанности психолога будет слишком сложно, ну а если человек обнаружил у себя подобные качества, пусть даже находящиеся в зачаточном состоянии, ему имеет смысл задуматься о получении этой профессии [aif.ru, 2020].

Работа психолога является концентрированным опытом взаимоотношений с разными людьми. И в этом есть свои плюсы и минусы.

\textbf{Плюсы профессии}

Преимуществом профессии психолога являются умения:
\begin{enumerate}
    \item дающие возможность осознать и многое понять про себя и свою жизнь;
    \item разбираться с жизнью других людей;
    \item помогать другим людям справляться с трудностями;
    \item обрести спокойствие внутри себя;
    \item познать причины поведения людей и их поступков;
    \item формировать философское отношение к внешним событиям;
    \item вести частную практику и быть независимым от работодателей.
\end{enumerate}

Несомненным плюсом профессии психолог является постоянный поиск, стремление росту и духовному развитию. Ведь если у человека внутри ничего этого нет, то как он что-то может дать другому?

Чтобы окончательно разобраться, подходит вам профессия психолога или нет, нужно знать и о минусах этой деятельности.

\textbf{Минусы профессии}

Большим минусом является почти постоянное состояние нервного напряжения. Психологам приходится иметь дело со страхами, отчаянием, депрессией, постоянно разбираться с причинами тяжелых переживаний, а также отыскивать и находить пути избавления от этих проблем.

При этом бывает очень трудно себя контролировать, чтобы не принимать близко к сердцу чужие негативные эмоции.

Минус состоит еще и в большой ответственности, т.к т рекомендаций психолога напрямую будут зависеть поступки и душевное состояние клиента. Чрезмерное чувство ответственности специалиста может являться причиной развития своеобразных страхов совершить что-то непоправимое или неправильное.

Недостатком профессии психолог также является интенсивное общение с незнакомыми людьми. Не каждый человек способен постоянно и помногу быть в контакте с окружающими.

К минусам профессии относится время от времени возникающая усталость, присущая психологам при больших душевных затратах и личных вложениях, которых требует работа. Ведь приходится думать о своих клиентах и после приема, постоянно переживать за них, разговаривать по телефону, чтобы оказать необходимую поддержку и помощь.

Но обычно такая усталость приятна и дорога. Ведь если специалист является востребованным, значит, он может принести пользу и хорошо знает свое дело. Такую усталость нельзя ни с чем сравнить.

Выбирая профессию психолога, человек осознает, что обратного пути у него больше нет. Он становится психологом везде и навсегда, поскольку невозможно не использовать в жизни те знания и опыт, которые он имеет, особенно когда наблюдаешь за собственными детьми, или, например, при общении с любимым человеком.

И порой бывает очень грустно уметь разбираться в каких-то вещах больше, чем другие, к тому же это может отдалять от некоторых близких людей [psynavigator.ru, 2019]

В профессии психолога, несомненно, есть как положительные, так и отрицательные стороны, но все же плюсы этой деятельности реально перевешивают. Т.к. у психолога есть отличная возможность помогать другим людям, спасать их от каких-то неправильных, а порой и глупых решений, и направлять все силы исключительно в правильное русло.

Желаем вам удачи и предлагаем поучаствовать в небольшом \ed{опросе}{опрос}{survey}: Пользуетесь ли вы услугами психолога?

\newpage
\section{Что делать, если случился нервный срыв?}

\textit{И как не довести себя до крайней степени стресса}

\textit{Источник: \url{https://lenta.ru/articles/2022/11/07/nerves/}}

Многие знают, что для сохранения ментального здоровья нужно читать и смотреть меньше новостей, много спать и сбалансировано питаться. Но не у всех получается следовать\footnote{следовать чему; e.g., он следует моему совету: he is following my advice} этим советам, а рекомендация «\explain{не нервничать}{do not be nervous}» кому-то и вовсе кажется насмешкой. «Лента.ру» вместе с психологами и эндокринологами разобралась в причинах возникновения нервного срыва и в том, как предупредить его появление, чтобы не \explain{испортить}{spoil; ruin} жизнь себе и близким, а главное --- не попасть в больницу.

\textbf{Что такое нервный срыв}

Специалисты \explain{расходятся во мнениях}{disagree} о точной формулировке понятия «нервный срыв». Если говорить прост\'{ы}м язык\'{о}м, то нервный срыв  ---  это термин, который люди употребляют в ситуациях, когда человек перестаёт \explain{справляться}{to cope with} с эмоциональной \ed{нагрузкой}{нагрузка}{То, что нагружено, приходится на что-н., выполняется кем-чем-н. Пример: Большая нагрузка вагона. | Нагрузка электрической сети в вечерние часы.} и «ломается». Это \explain{острая}{acute} фаза стресса, проявляющаяся в виде невротических и депрессивных расстройств.

Психолог, психотерапевт, член Международной профессиональной ассоциации психологов Елизавета Деменштейн обратила внимание, что понятие «нервный срыв» не зафиксировано в Международной статистической классификации болезней именно как болезнь. Однако если обратиться к врачам с ж\'{а}лобами на «нервный срыв», они помогут и \explain{назначат лечение}{prescribe treatment}.

Нервный срыв может проявляться по-разному. У кого-то возникает состояние ступора, эмоции пропадают, возникает ощущение, что всё происходящее нереально, наступает упадок сил и всё время хочется лежать. Другие впад\'{а}ют в возбуждение с кр\'{и}ками, слез\'{а}ми, ист\'{е}риками, агр\'{е}ссией и желанием \explain{крушить}{С силой ломать, уничтожать} всё вокруг.

«В большинстве случаев организм человека успевает адекватно реагировать на последствия стресса, который уже давно стал привычным явлением. Но если нагрузка слишком велика, может наступить нервный срыв  ---  состояние, с которым психика не справляется»,  ---  пояснил психолог, кандидат психологических наук Валерий Гут. Нервный срыв, по его словам, является лишь эпизодом \ed{затяжного}{затяжн\'{о}й}{Очень продолжительный, затянувшийся. \textit{Затяжная болезнь. | Затяжной кризис капитализма. | Затяжной прыжок (прыжок с долго не раскрываемым парашютом).}} стресса.

Психолог калифорнийского университета Риверсайд Мэтью Чанг уверен, что люди не \explain{справл\'{я}ются с}{cope with} эмоциональным напряжением, потому что большинству из них нравится \explain{предсказуемость}{predictability} и рутина. \explain{Их устраивает}{they are satisfied}, что один день похож на другой, так как это дает ощущение уверенности и стабильности. Но зачаст\'{у}ю в череду спокойных б\'{у}дней врываются перемены и р\'{у}шат \explain{устоявшийся}{established} порядок жизни, заставляют нервничать. Причём нервное потрясение м\'{о}гут вызвать любые яркие события  ---  как положительные, так и отрицательные.

Нервный срыв может быть связан и с \ed{переутомлением}{переутомление}{overwork}, хронической усталостью, напряжённым периодом на работе или в семейных отношениях, рассказала психолог Деменштейн. В некоторых случаях даже диета может стать \ed{спусковым механизмом}{спусковой механизм}{trigger mechanism} для срыва, пояснила она.

Психологи, психиатры и психотерапевты должны выяснить, из-за какого длительного стресса случился нервный срыв или на фоне какого психического заболевания мог развиться сам стресс. Таких болезней может быть много: депрессия, посттравматическое расстройство и так далее.

Нервный срыв может произойти у любого человека, в том числе у тех, кто не страдает психическими заболеваниями, отметила в своих исследованиях доцент кафедры психиатрии и психолог Университета Цинциннати Мария Эспинола. Но, по её словам, люди с психическими \ed{расстройствами}{расстройство}{Заболевание, нарушающее нормальные функции какого-н. органа. \textit{Расстройство нервной деятельности.} Disorder} переживают нервные срывы чаще.

\begin{fancyquotes}
    Хронический стресс и провоцирующий фактор  ---  это составляющие нервного срыва\\

    \begin{flushright}
        Елизавета Деменштейн\\
        психолог, психотерапевт
    \end{flushright}
\end{fancyquotes}

Когнитивно-поведенческий психолог онлайн-школы психологических профессий «Психодемия» Александра Титарева рассказала, что за нервным срывом могут скрываться:
\begin{enumerate}
    \item расстройства, связанные со стрессом (расстройство адаптации, посттравматическое стрессовое расстройство);
    \item тревожные и связанные со стрессом расстройства (генерализованное тревожное расстройство);
    \item аффективные (биполярное или депрессивное расстройство) и многие другие расстройства.
\end{enumerate}

Так что же такое нервный срыв? Сотрудники факультета психологии Государственного университета Уэйна в США дают ему следующее определение: нервный срыв  ---  это состояние повышенного нервного напряжения, длящееся короткий промежуток времени. Ему присущи симптомы, схожие с симптомами тревожного расстройства и депрессии. Но эти заболевания, в отличие от нервного срыва, имеют длительный характер. Нервный срыв часто случается на фоне провоцирующего события  ---  финансовой потери, ссоры в семье, потери близкого и других.

\begin{framed}
    \begin{center}
        {
            \Huge
            26\%
        }

        {
            \Large
            американцев в ходе опроса признались, что хотя бы раз чувствовали себя на грани нервного срыва
        }
    \end{center}
\end{framed}


\newpage
\textbf{Почему случается нервный срыв}

Главные причины нервного срыва как части долгого стресса.
\begin{enumerate}
    \item Психическое расстройство, провоцирующее срыв (депрессия, биполярное расстройство, посттравматическое расстройство, беспокойство, расстройство адаптации, возникающей из-за реакции на стрессовое событие или череду психотравмирующих ситуаций).
    \item \explain{Истощение}{exhaustion} организма из-за хронических болезней (сбоев в работе сердечно-сосудистой системы, неврологических заболеваний).
    \item Тяжёлый \explain{график работы}{work schedule} без возможности отдохнуть в полноценном отпуске.
    \item Длительная тяжёлая \explain{обстановка}{situation} в семье.
    \item Известия о тяжёлых событиях (смерть близких, развод, увольнение и другие).
\end{enumerate}

\begin{framed}
    \begin{center}
        \Large
        В ходе исследования, проведенного зарубежными социологами, стало известно, что самыми частыми причинами срыва называют проблемы в отношениях и воспитание ребенка в одиночку (у женщин)
    \end{center}
\end{framed}

Спровоцировать нервный срыв могут даже такие факторы, как нарушение сна и \explain{компульсивное переедание}{binge-eating}.

Сохранение режима сна  ---  одно из условий поддержания физического и ментального здоровья. Хроническое недосыпание, а также кошмары, регулярные пробуждения среди ночи, нарушение циркадных ритмов также могут стать причиной нервного срыва, пояснил медицинский психолог, психоаналитик Олег Долгицкий. При этом одни могут бесконечно спать (но просыпаются не отдохнувшими), а другие, наоборот, спят мало, урывками, становятся \explain{чересчур}{То же, что слишк\'{о}м. \textit{Чересчур горячий суп. | Чересчур много говорил. | Это уж чересчур!}} бодрыми и активными. Обычно таких людей объединяет то, что они не могут получать удовольствие в своей жизни, добавил он.

Сочетание нескольких или всех \ed{перечисленных}{перечисленные}{listed} симптомов с отсутствием удовольствия и наличием психосоматической болезни может приводить к накалу эмоциональных состояний, к нервным срывам, которые говорят о том, что у человека уже закончились силы и ему пора на отдых.


\textbf{Как понять, есть ли нервный срыв у меня}

Невролог центра здоровья Verba Mayr Елена Сопова обратила внимание, что признаков нервного срыва может быть множество в зависимости от причины нервного напряжения и силы его проявления. Но главный из них  ---  это отсутствие способности организма нормально функционировать. Итак, при нервном срыве могут проявляться следующие симптомы.

\begin{enumerate}
    \item Эмоциональные: тревога, тоска, раздражительность, \explain{навязчивые мысли}{intrusive thoughts}, \explain{перепады настроения}{mood swings}.
    \item Психические: появление страха смерти, ощущение \ed{беспомощности}{беспомощность}{helplessness}.
    \item \ed{Сбой}{Сбой}{failure} в работе иммунной системы: частые простудные заболевания, \explain{обострение}{exacerbation} герпетической инфекции.
    \item Снижение когнитивных функций (памяти, внимания). Это может значительно снизить работоспособность, \explain{сказаться на}{affect} возможности обслуживать себя и отразиться на качестве жизни в целом.
    \item Сбой работы \ed{желудочно-кишечного тракта}{желудочно-кишечный тракт}{GI tract} (боли в животе, нарушения стула, метеоризм).
    \item Расстройства вегетативной нервной системы: нарушение работы сердца (аритмия, тахикардия), сухость во рту, повышенная потливость, тремор.
    \item Изменение аппетита и веса.
    \item Проблемы со сном.
\end{enumerate}

Психолог Гут добавил, что при нервном срыве у человека могут появиться мысли о смерти и желание \explain{причинить}{to cause} себе вред. Тело реагирует на стресс слабостью, мышечным перенапряжением (стиснутые зубы, «деревянная» спина).

Во время нервного срыва человек понимает, что теряет над собой контроль: не может сдержать слёзы, выступающие \explain{без в\'{е}ского п\'{о}вода}{without a good reason}, становится раздражительным, замечает частую смену настроения и ухудшение концентрации, у него повышается уровень тревожности, добавила Деменштейн. В общем, организм всеми возможными способами подаёт сигналы о том, что ему необходима помощь.

Психиатр Дельвена Томас предупредила, что человек в период срыва может полностью изолироваться от общества. Со стороны может казаться, что больной не интересуется никем и ничем вокруг из-за скуки, но на деле таким образом он \ed{б\'{о}рется}{бор\'{о}ться}{to fight (борюсь, б\'{о}решься, б\'{о}рется, б\'{о}рются)} за своё психическое здоровье и, в \'{о}бщем-то, жизнь.

«В таком состоянии люди не могут быть рядом с другими из-за своей психической \ed{неустойчивости}{неустойчивость}{instability},  ---  конкретизировала Томас.  ---  Общим сигналом нервного срыва является уход из привычного окружения. Человек перестает общаться с семьей и друзьями и больше не считается полностью функциональным».

% <------ HERE -------->
Научные сотрудники Университета Колумбии в США в ходе тестирования 102 человек установили, что у людей с сопутствующими психическими заболеваниями нервный срыв проявлялся по-разному.

Так, участники эксперимента с паническими атаками, усугубленными нервным срывом, часто ощущали \explain{уд\'{у}шье}{suffocation} и страх смерти. Люди с нервным срывом и аффективным расстройством вели себя очень агрессивно и признавались в постоянном желании кричать. У тех, у кого нервный срыв стал результатом других сопутствующих заболеваний и тревожных расстройств, ярких симптомов было меньше.

Кстати, учёные из Великобритании в\'{ы}яснили, что на женщин и мужчин стресс влияет по-разному. Так, количество женщин, исп\'{ы}тывающих стресс, связанный с работой, оказалось на 50 процентов больше, чем у мужчин того же возраста. Это происходит из-за того, что женщины много работают, стр\'{о}ят карьеру, но в то же время за ними сохраняются \ed{обязанности}{обязанность}{Определённый круг действий, возложенных на кого-н. и безусловных для выполнения. \textit{Права и обязанности граждан. | Служебные обязанности. | Воинская обязанность. | Общественная обязанность. | Исполняющий обязанности директора (т. е. ещё не утверждённый в должности, работающий временно).}} \ed{следить за}{следить за \textit{чем}}{take care of; look after} детьми и домом.

\begin{framed}
    \begin{center}
        {
            \Huge
            у 25процентов
        }

        {
            \Large
            женщин рано или поздно развивается депрессия
        }
    \end{center}
\end{framed}

\textbf{Чем нервный срыв отличается от панической атаки.}
Иногда нервный срыв можно спутать с панической атакой из-за некоторых схожих симптомов. Но паническая атака  ---  это совершенно другое состояние, отмечают психологи. Постоянные панические атаки могут привести к тревожному расстройству.

Панической атаке, как и нервному срыву, присуще сильное чувство страха, ощущение того, что непременно должно произойти что-то плохое. Люди при панической атаке боятся потерять контроль над происходящим и даже умереть. При атаке, которая обычно накрывает внезапно, возникает повышенное потоотделение, тремор рук, учащенное сердцебиение. Могут наблюдаться проблемы с ЖКТ и головная боль.

Панические атаки короче, чем нервные срывы, и, когда они проходят, человек ощущает сильный стресс и усталость. Панические атаки очень пугают своей внезапностью и множеством физических симптомов, которых больше, чем при нервном срыве.

Большая разница между панической атакой и нервным срывом заключается в том, что при первой человек порой не может даже подняться с постели, настолько ему плохо, а при втором энергия может бить через край.

\begin{fancyquotes}
    Возможно, кому-то знакомо ощущение помутненного рассудка, когда кружится голова и энергия буквально бьет в мозг. Это как ядерная бомба, которую сбросили с самолета. Если она ударится о землю, будет взрыв\\

    \begin{flushright}
        Валерий Гут
        \\
        психолог
    \end{flushright}
\end{fancyquotes}

\textbf{Гормоны щитовидной железы и нервный срыв.}
Эндокринолог Тодд Ниппольдт в статье «Заболевания щитовидной железы: могут ли они повлиять на настроение человека?» однозначно ответил на заданный в заглавии вопрос. «Да, заболевания щитовидной железы могут влиять на настроение, в первую очередь вызывая тревогу или депрессию. Как правило, чем тяжелее заболевание щитовидной железы, тем сильнее меняется настроение»,  ---  пояснил он.

Эндокринолог Екатерина Асташова рассказала «Ленте.ру», что при недостатке гормонов щитовидной железы  ---  гипотиреозе  ---  обменные процессы в организме замедляются, и нервная система угнетается. При гипотиреозе нарушается эмоциональное состояние: у людей отмечается подавленное, тоскливое настроение, приступы тревоги.

\begin{fancyquotes}
    Избыток гормонов щитовидной железы  ---  гипертиреоз или тиреотоксикоз  ---  также вызывает проблемы со здоровьем. Обменные процессы ускоряются, появляется повышенная двигательная и психическая активность. Человек может беспричинно плакать, раздражаться, постоянно находиться на взводе\\

    \begin{flushright}
        Екатерина Асташова
        \\
        эндокринолог
    \end{flushright}
\end{fancyquotes}

Избыток гормонов щитовидной железы встречается у двух процентов женщин и 0,2 процента мужчин, добавила эндокринолог, эксперт по превентивной и anti-age медицине Европейского медицинского центра Камиля Табеева. Недостаток гормонов диагностируют у 4,6 процента женщин.

\begin{fancyquotes}
    У 21 женщины из 1000 и 19 мужчин из 1000 наблюдается недостаток гормонов щитовидной железы\\

    \begin{flushright}
        Камиля Табеева\\
        эндокринолог
    \end{flushright}
\end{fancyquotes}

Табеева отметила, что у таких людей есть риск развития депрессии. Зачастую пациенты наблюдаются с этой болезнью, не зная, что имеют дефицит гормонов, который можно компенсировать с помощью таблеток.

После диагностики заболевания щитовидной железы, которое может протекать в том числе и с переходом из тиреотоксикоза в гипотиреоз, эндокринолог подберет лечение и план наблюдения. И через три-шесть недель эмоциональное состояние пациента придет в норму, резюмировала эндокринолог Асташова.

\textbf{Последствия нервного срыва.}
Существует четкая связь между стрессом и ухудшением состояния здоровья. Люди, находящиеся в постоянном стрессе, чаще болеют как психически, так и физически, предупредил психолог Долгицкий.

\begin{fancyquotes}
    Если человек длительное время находится в состоянии стресса, у него на 80 процентов повышается риск возникновения бронхиальной астмы, язвенного колита, эссенциальной гипертензии, нейродермита, ревматоидного артрита, язвенной болезни желудка и язвы двенадцатиперстной кишки\\

    \begin{flushright}
        Олег Долгицкий\\
        медицинский психолог, психоаналитик
    \end{flushright}
\end{fancyquotes}

Тело всегда реагирует на сильный стресс, заявил судебный психиатр Мэттью Чанг. «Мозг посылает сигнал нервной системе в начале стрессовой ситуации о выбросе гормонов стресса  ---  адреналина, норадреналина и кортизола»,  ---  отметил он.

Этот каскад химических реакций побуждает тело перенаправлять приток крови к крупным мышцам и сердцу, что в свою очередь учащает дыхание, повышает пульс и кровяное давление и подготавливает тело к психологической и физической реакции. Скелетные мышцы также активируются, напрягаются, чтобы помочь защититься от возможных травм или подготовить тело к выходу из критической ситуации.

Если реакция «бей и беги», возникающая при стрессе, в том числе и нервном срыве, срабатывает слишком часто, это может привести к проблемам:

\begin{enumerate}
    \item болезням сердца;
    \item постоянно высокому кровяному давлению;
    \item повышенному уровню сахара в крови и снижению чувствительности к инсулину;
    \item ослаблению иммунной системы;
    \item повышенному риску ожирения (особенно в области живота);
    \item депрессии и беспокойствам;
    \item головным болям;
    \item бессоннице.
\end{enumerate}

«Стресс, который продолжается в течение длительного времени, может быть опасным для жизни,  ---  добавил Чанг.  ---  Любой, кто испытывает хронический стресс, должен принять меры, чтобы снизить его уровень и позволить телу вернуться в нормальное состояние».

\textbf{Как не допустить нервный срыв.}
Психолог Гут, сравнивший нервный срыв с падающей на землю бомбой, отметил, что, пока «бомба» не приземлилась, есть всего несколько секунд, чтобы ее «поймать» и предотвратить удар. Нужно научиться «подставлять сетку» и не давать снаряду разорваться. Это позволит не разрушить жизнь гневом, истериками и скандалами.

Для профилактики стрессов в общем и нервных срывов в частности, нужно соблюдать определенные правила, настаивает нутрициолог Дарья Ермилова. Главным, по ее словам, является соблюдение циркадных ритмов и гигиены сна. «Спать нужно ложиться строго до 23:00, а просыпаться около 7:00,  ---  рассказала она.  ---  Днем нужно проводить максимальное количество времени при дневном освещении, а в вечернее время использовать мягкий свет, избегая синего излучения от техники и гаджетов».

Питание, направленное на снижение воспалительных процессов, поможет поддержать нервную систему, добавила нутрициолог. Из рациона нужно убрать трансжиры, чрезмерное количество углеводов, сахара, молочных продуктов.

\begin{fancyquotes}
    Нужно есть больше мяса нежирных сортов (курица, кролик, телятина, индейка), рыбы и морепродуктов, овощей (белокочанная капуста, брокколи, морковь, тыква), зелени, ягод (особенно черной смородины), а также круп и орехов\\

    \begin{flushright}
        Дарья Ермилова\\
        нутрициолог
    \end{flushright}
\end{fancyquotes}

Адекватная физическая нагрузка повышает чувствительность к инсулину и снижает уровень воспаления в организме, рассказала Ермилова. Это может быть простая ходьба, но не менее 10 тысяч шагов в день, аэробные нагрузки, бег. За физическую активность сойдет даже интенсивная уборка дома.

Большую роль в поддержании психического здоровья играет отказ от вредных привычек: курения, употребления алкоголя и запрещенных веществ.

\begin{framed}
    \begin{center}
        \Large

        Медитация, или дыхательные практики  ---  это простой и приятный способ восстановить баланс парасимпатической и симпатической нервных систем, снизить воспаление в организме и уровень стресса
    \end{center}
\end{framed}

Психолог Титарева добавила, что профилактикой нервного срыва является снижение повседневного стресса. Для этого нужно отказаться от части нагрузки, контролировать режим работы и отдыха, следить за питанием, расслабляться, слушая пение птиц или наблюдая за аквариумными рыбками.

\begin{fancyquotes}
    Подойдут любые способы, помогающие вернуться в спокойное состояние: прогулка на свежем воздухе, общение с любимым питомцем, которого можно потискать, сбор пазлов или картин по номерам\\

    \begin{flushright}
        Александра Титарева\\
        психолог
    \end{flushright}

\end{fancyquotes}

Также необходимо разобраться в причинах, которые привели к нервному срыву, чтобы избежать этого состояния в дальнейшем, отметил психолог Гут. В этом поможет ведение дневника, который станет ценным инструментом проживания и анализа ситуаций и событий.

Нервный срыв  ---  это всегда острое состояние, требующее немедленных действий. Техники самопомощи помогут с ним справиться, но лучше начать заботиться о себе и своем психическом здоровье заранее и не допустить кризиса.

\textbf{Как помочь себе при нервном срыве.}
Профилактика срыва  ---  это лучшая помощь себе. Но если ее методы не сработали, то можно прибегнуть к действенным методам самопомощи.

В момент нервного срыва происходит огромный отток энергии. И первое, что нужно сделать, это затормозить отвечающую за стресс систему. Умывание холодной водой, мышечная релаксация и простые физические упражнения (приседания или отжимания) снимут острую фазу, отметил психолог Гут.

После этого нужно создать условия, в которых человек будет чувствовать себя комфортно и безопасно. Добиться этого ощущения помогают еда и сон.

\begin{fancyquotes}
    Также облегчить состояние помогает массаж, с его помощью можно снять зажимы и расслабить мышцы шеи, плеч, которые чаще всего излишне напряжены во время стресса\\

    \begin{flushright}
        Елизавета Деменштейн\\психолог
    \end{flushright}
\end{fancyquotes}

\textbf{Как помочь другому человеку при нервном срыве.}
При нервном срыве нужно помочь человеку успокоиться и переключить его на какое-либо занятие или просто отвлечь беседой. Сделать это нужно здесь и сейчас, чтобы он в порыве нервного напряжения не навредил себе и окружающим.

«Важно понимать, что человек при нервном срыве деградирует в реакциях и поведении до уровня ребенка. Поэтому необходимы забота, деликатность и мягкая настойчивость»,  ---  рассказывал о срыве в своем канале психолог Артем Толоконин.

Для того чтобы облегчить проявление нервного срыва у другого человека, он посоветовал предложить больному сесть или лечь. При этом нужно быть рядом с ним и дать своим спокойным поведением понять, что все хорошо.

При возможности человека в нервном перевозбуждении нужно обнять. Так, по мнению Толоконина, его внутренний ребенок успокоится, так как закроет потребность в тепле и близости.

\textbf{Лечение нервного срыва.}
Нервные срывы часто влекут негативные последствия. Так, психосоматические нарушения могут стать причиной повышения давления, болей в желудке, головокружений. Также по цепочке может ухудшиться основное психическое заболевание, повлекшее нервный срыв. Иногда даже может быть затронута работа некоторых отделов головного мозга.

Постоянное нервное напряжение мешает сохранению отношений в семье, общению с коллегами и друзьями, что грозит проблемами дома и на работе, а также социальной изоляцией и в перспективе  ---  ухудшением финансового положения.

«Не стоит забывать, что часто в нашей стране последствия нервного срыва принято «гасить» психоактивными веществами. Однако в долгосрочной перспективе такое «самолечение» приводит к усилению тревоги и подавленности, ночным кошмарам и даже суициду»,  ---  предостерегла Титарева.

\begin{framed}
    \begin{center}
        За психологической поддержкой всегда нужно обращаться к специалисту. В беседе с психологом или психотерапевтом необходимо делать упор на причине, вызвавшей нервный срыв, но при этом не отвергать эмоции. Без выявления причины велик риск повторного срыва, даже после курса поддерживающей терапии
    \end{center}
\end{framed}

«Если вы начинаете чувствовать, что стресс становится слишком сильным, поговорите со своим терапевтом. Он может направить вас к психологу или психиатру. Терапевт также может назначить лечение физических симптомов»,  ---  рассказал врач Дэн Бреннан.

По его словам, правильное лечение нервного срыва зависит главным образом от его причины и индивидуальных особенностей пациента. Некоторые методы лечения включают:

\begin{enumerate}
    \item изменение образа жизни;
    \item сокращение количества ежедневных обязательств;
    \item соблюдение здоровой диеты;
    \item отдых при первой необходимости;
    \item практика медитации;
    \item проведение большего времени на природе.
\end{enumerate}

Стрессы и психологические заболевания лечат и с помощью медикаментов. Если обратиться к врачу, он может назначить антидепрессанты или успокаивающие препараты, чтобы облегчить симптомы нервного срыва. Если стресс вызывает бессонницу, могут прописать снотворное, пояснил Бреннан. Нарушения сна могут усугубить стресс и тревогу, которые в свою очередь только усугубят бессонницу  ---  образуется замкнутый круг. Снотворные могут помочь разорвать цикл бессонницы и уменьшить стресс.

Психотерапия поможет справиться с нервным срывом и снизить риск его повторения. Разговор с профессионалом поможет обдумать мысли и найти решения, которые уменьшат стресс и тревогу.

В очень тяжелых случаях может быть уместно даже стационарное лечение, но обычно это возникает тогда, когда нервный срыв может быть сопряжен с риском для жизни человека (суицидальная попытка), прокомментировал психолог Долгицкий. «Тогда в соответствии со ст. 29 закона "О психиатрической помощи и гарантиях прав граждан при ее оказании" пациента госпитализируют,  ---  пояснил психолог.  ---  Это становится возможным в том случае, если человек несет непосредственную опасность для себя или окружающих во время нервного срыва».

Также, по его словам, нужно госпитализировать пациента, если известно, что без профессиональной помощи пострадает его здоровье. Например, когда человек падает в голодные обмороки, теряет сознание или наносит себе повреждения, подвел итог Долгицкий.

\newpage
\section{Квалиа: как объяснить то, что чувствуешь}

\textit{Источник: \url{https://4brain.ru/blog/kvalia-kak-obyasnit-to-chto-chuvstvuesh/}}

Субъективность чувственного опыта – это большая проблема в биологии и философии. Восприятие вещей и явлений – индивидуальное переживание каждого человека. Возможно ли объяснить, что такое цвет, каков на вкус шоколад, как ощущается голод, другому человеку так, чтобы он понял, что мы хотим ему донести?

Анализ информации всегда субъективен. Даже решение математических задач может быть совершено разными подходами и алгоритмами, хотя, казалось бы, математика – точная наука. То же касается остальных сфер: от компьютерных игр и художественной литературы до выполнения повседневных бытовых задач и взглядов на них.

Чтобы расширить возможности мозга для улучшения восприятия чужого опыта, научиться смотреть на вещи и явления с разных сторон, приглашаем вас на онлайн-программу «Когнитивистика». Кстати, ориентироваться в пространстве информации экологично и формировать собственную точку зрения помогает хорошо развитое критическое мышление, но ему часто мешают внешние воздействия: пропаганда, манипуляции, фальсификация фактов. И чтобы вы могли укрепить этот важный навык, приглашаем вас на программу «Критическое мышление».

А мы переходим к основной теме.

\textbf{Что такое квалиа?}

Феномен квалиа (мн. ч.) простыми словами – это обозначение свойств психических состояний и сознательный опыт в целом [Вестник ВГУ, 2018]. Термин ввел в 1929 году философ Кларенс Льюис. Это «сырые чувства» и сознательные ощущения, которые могут дополнять или исключать друг друга одновременно. Дискуссии о природе и реальности квалиа включают одновременно сводимость и несводимость с онтологическими основаниями различных теорий.

Чтобы предметно понять, что такое квалиа, рассмотрим их на примере изучения цвета.

Предположим, есть ученый, который изучает цвет, но от рождения он слеп. Он знает, какие рецепторы отвечают за цветовосприятие, как импульс проходит в мозг и как перерабатывается. Но, напомним, сам цвет слепой ученый не видит. Если в результате лечения он прозреет и увидит то, что изучал, собственными глазами, насколько его восприятие цветов изменится? В данном случае знание о цвете можно разделить на два событийных класса:


\begin{enumerate}
    \item
          Опыт от первого лица, тот, который проживает каждый из нас индивидуально. В данном случае это возможность видеть цвет собственными глазами – квале (ед. ч.) цвета.
    \item Опыт от третьего лица, как бы со стороны. В данном случае это исследование цвета ученым без личного его обозревания.
\end{enumerate}

Выходит, квалиа – это опыт от первого лица. С другой стороны, то, что третье лицо знает от своего первого лица о цвете, звуке, тактильных ощущениях или через шрифт Брайля – тоже квалиа, но для него, а не для второго человека, который опыт первого воспримет все равно по-своему.

Квалиа, личный опыт переживания, сложно передать словами, его можно только пережить. Интересный пример: оттенки цвета. Так, многие мужчины видят, предположим красный в его оттенках, но значения им не придают. Если на полке в магазине лежат две футболки красного цвета, но не точь-в-точь одинаковых, а одна, например, ярче другой, большинству представителей сильного пола это не так важно. Другое дело – женщина в маникюрном салоне. Она может пересмотреть палитру из дести оттенков красного и не выбрать ни один, потому что того самого нет. Это касается восприятия и других цветов.

Еще один пример разности восприятий – так называемое перевернутое квале. Так, в вашем понимании красный для внутреннего взора другого человека – зеленое квале [Science Direct, 2012]. Эксперименты показали: разница между слуховым и зрительным восприятием действительно существует из-за разности анатомического строения органов чувств и их тонкой настройки.

А как объяснить голод? Для одного – это ощущение пустоты в желудке и томное потягивание, для другого – состояние через четыре часа после еды, для третьего – еще что-то. Первый не поймет третьего в разговоре о голоде, у них будут совершенно разные ассоциации, потому что личный опыт восприятия разный. Это и есть квалиа.

Как и голод, сложно объяснить другие персональные переживания:
\begin{enumerate}
    \item вкус;
    \item запах;
    \item цвет;
    \item весь спектр чувств (любовь, тоска, гнев и т.д.);
    \item тактильные физические ощущения (боль, зуд, жар и т.д.)
\end{enumerate}

Использовать «квалиа» в качестве синонима ощущений ошибочно. У квале шире спектр восприятия, в то время как ощущения ограничены набором сенсорных модальностей в конкретный момент при заданных обстоятельствах. Восприятие цвета, звука, скорости у всех разное. Причем не только у людей, но и у животных. Для летучих мышей и кошек квали скорости разное.

При некоторых болезнях мозга, когда человек не различает лица и движение, не непосредственные квалиа он не получает.


\textbf{Проблема изучения квалиа}

Как уже удалось выяснить, квалиа – это приватный опыт от первого лица, что вызывает проблему их изучения. В данном случае получается, что ученый – это третье лицо; соответственно, даже при установке всех коррелятов сознания он не увидит того, что испытывает участник эксперимента. Для исследователей доступна только косвенная корреляция результатов изучения по поведению испытуемого и отчетам используемого в экспериментах оборудования.

Парадокс квалиа: они вреде бы реально существуют в восприятии каждого человека, но, с другой стороны, в реальности их как будто нет, ведь то, что субъективно, в объективной реальности не существует [Вики Чтение, 2022].

Квалиа невозможно отследить с точки зрения официальной науки, поэтому некоторые люди впадают в мистику или дуализм, либо вовсе отрицают наличие индивидуального внутреннего опыта, как Дэниел Деннет [Level One, 2022]. Философ считает, что он относится к устаревшей терминологии в бинарной метафизике, а внутри нас нет никакого эфемерного «я». Деннет понимает квалиа как множественные информационные потоки, борющиеся за доступ к мозгу – который победит, тот и определит поведение человека.

Проблему квалиа сознания физическим подходом объясняет австралийский философ Дэвид Чалмерс. Он описывает сложность сознательной деятельности человека как исчерпывающую теорию психики. В нем преобладают описания нейронов, поведения, причинно-следственных связей с позиции третьего лица.

Согласно взглядам Чалмерса, все сущее является физическим или производным от него, квалиа – не исключение. Сознание не автономно, а тождественно мозгу и является следствием его функционирования. Этого же мнения придерживался упомянутый ранее Дэниел Деннет, но в отличие от него Чалмерс само понятие не отрицает, только объясняет его иначе [Д. Чалмерс, 2013]. По его теории, квалиа – фундаментальный элемент вселенной, поэтому теорию сознания он предлагает рассматривать с упором на физические свойства организма, а не биологические. При этом философ понимает, что квалиа невозможно пощупать или увидеть, это энергетическая форма жизни, поэтому Чалмерс называет ее «натуралистический дуализм» [Level One, 2022].

\textbf{Эксперимент Мэри}

В литературе по философии квалиа приводится множество мысленных экспериментов разного рода. Один из самых известных – случай Мэри, философского мысленного эксперимента Фрэнка Джэксона [Стэнфордская энциклопедия философии, 1997].

Девушку заключили в черно-белую комнату, которую она никогда не покидала. Соответственно других цветов, кроме черного и белого, она не видела, всю информацию получала из черно-белого телевизора, книг, мониторов. Кожу и волосы ее покрывали краской, чтобы никакие оттенки не появлялись на глазах. Со временем Мэри узнает об аспектах цветов и цветного зрения, знает все физические факты в этой области, становится мировым авторитетом по цветам и их восприятию. Но о том, как эти самые цвета воспринимаются собственными глаза, она не знает.

Однажды Мэри освобождается из комнаты и попадает в реальный цветной мир. Она выходит в сад, полный растений, видит красную розу и восклицает: «Значит, вот каково это — ощущать красный цвет… И это, – добавляет она, глядя на траву, – каково это – ощущать зелень».

В момент, когда Мэри оказывается в цветном мире, она делает важное открытие: собственными глазами видит то, о чем знала лишь в теории – обладала физической информацией, знала о цвете все, но впервые получила собственный опыт переживания того, о чем только в теории знала.

Одно из объяснений этого явления заключается в том, что существует сфера субъективных качеств, относящихся к категории феноменальных. Их природу внутри себя открывает Мэри после освобождения, и подвергается новым для себя цветовым переживаниям, ведь знания о цвете были объективными, а не субъективными.

Данный эксперимент нельзя объяснить физикалисту. Так, для критика ощущение цвета станет физическим качеством, но Мэри знала о нем, находясь в черно-белой комнате. Налицо противоречие. Однако несогласные с явлением квалиа могут объяснить подобное знание цвета ноу-хау. Мэри, после получения субъективного ощущения красного, зеленого и других цветов, ничего нового о них не узнала, только обрела способность распознавать глазами цветные вещи и представлять себе цветное пространство [Стэнфордская энциклопедия философии, 1997].

\textbf{Философский зомби}

Ранее упомянутый философ Дэвид Чалмерс, отвергающий теорию внутреннего «я», предложил концепт абстрактного человека – «философского зомби», противоположный опыту Мэри.

Итак, человек не способен переживать квалиа – он лишен внутренней сознательной жизни, но успешно имитирует поведение обычного человека. «Зомби» действует в «ментальной темноте»: не понимает суть вещей и явлений, с которыми имеет дело, но придерживается прописанного для него распорядка и правил. Существу разрешены героические поступки, но только в имитации их моральности, потому что чувственный опыт ему недоступен [David J. Chalmers, 1996].

Понятие «философский зомби» стало аргументом в спорах о проблемах сознания. Оно объясняет, что его феноменальные и субъективные аспекты не всегда сводятся к функциональным фактам. По сути, «философский зомби» – это наш двойник, который полностью повторяет нас, у него есть наши психические особенности, он таким же образом перерабатывает информацию, говорит, но наших феноменальных свойств и квалиа – способности испытывать чувственный опыт. Он не видит мир субъективно и по-своему качественно, т.е. красная роза для него – это объективно красная роза, без каких-либо собственных переживаний по ее поводу:

\begin{fancyquotes}
    Поcмотрите это видео: \url{https://youtu.be/mN6aTSHpjm0}.
\end{fancyquotes}

Результаты опыта не столь однозначны: выводы эксперимента могут быть признаны верными только в том случае, когда экспериментатор владеет тотальными знаниями об исследуемом человеке, а это невозможно, поскольку исследователь руководствуется отчасти собственными соображениями по поводу интерпретации результатов.

В философии невозможность оперировать смыслами не всегда зависит от внутреннего восприятия человека. Он может превратиться в «зомби» под действием внешних факторов, в частности социума, если не имеет собственных осознаний, взглядов, мнения и понимания происходящего.

\textbf{Китайская комната и нейросети}

Еще один мысленный эксперимент американского философа Джона Серля «Китайская комната» показывает, что создание сильного искусственного интеллекта, подобного человеческому мозгу, невозможно. В нем описан человек, который не знает китайского, но помещен в комнату с кубиками, на которых изображены китайские иероглифы.

Испытуемому дали инструкцию с описанием иероглифов и руководство по их выбору для ответа на поступающие извне вопросы. Через какое-то время человек овладевает искусством выбора кубиков до такой степени, что китайцы, которые задавали ему вопросы, думают, что общаются с носителем языка. Но человек, используя иероглифы, их по-прежнему не понимает, а владеет лишь алгоритмом синтаксиса без осознания смысла ответов.

Выходит, если искусственный интеллект будет развит до самосознания в нашем понимании, мы не сможем проверить, действительно ли он переживает квалиа или только имитирует их.

Философ Джон Серль в своем труде «Сознание, мозг и программы», в котором описан опыт с китайской комнатой, много внимания уделяет биологическому аспекту человека и утверждает, что машины не обладают способностями к аналитическим и лингвистическим философским традициям, их «сознание» лишено субъективной компоненты квале. Выходит, компьютеры не способны к проявлению чувств, которые формируют человеческое сознание.

\textbf{Рамки мышления и речь}

Непередаваемые субъективные ощущения, коими считаются квалиа, вызывают проблему связи языка и мышления. Так, с помощью речи мы выражаем собственные мысли и ощущения, но в этих же рамках и понимаем других; иными словами, наше мышление ограничено собственным восприятием происходящего и сути вещей [Frontiersin, 2021].

Австрийско-британский философ Людвиг Витгенштей утверждал, что границы языка, на котором мы думаем, определяют рамки воспринимаемого нами мира [Koç University, 2020]. В случае, если какая-то концепция в языке отсутствует, понимать ее становится затруднительно. Отсюда проблемы понимания между разными группами языков, например, восточных и европейских стран. Такая точка зрения стала гипотезой лингвистического детерминизма.

Подобное мнение в конце 19 века высказал антрополог Франц Боас. Он предположил, что люди воспринимают мир в рамках своих культур.

Гипотеза лингвистического детерминизма в настоящее время является спорной. Исследования подтверждают: язык влияет на мышление созданием семантических структур, основанных на чувственном опыте. Иными словами, расширить горизонт понимания и восприятия можно, для этого необходимо развивать когнитивные способности мозга [Cambridge Core, 2018]. Это делает описание квалиа еще более сложным.

\textbf{Искусство с точки зрения квалиа}

Парадокс квалиа, который касается личного опыта каждого человека и живого существа, служит основой для захватывающих литературных сюжетов в жанре фантастики, художественных холстов.

В 2021 году в России Посольством республики Польша был проведен конкурс на тему «Мир в 2121 году», посвященный 100-летию фантаста Станислава Лема. Одним из его финалистов стал начинающий автор Никита Тропинин с рассказом «Квалиа», написанном в «лемовской» философии [Мир Фантастики, 2021]. В рассказе идет речь об изобретении метода искусственного наведения чувственных ощущений, в то числе оргазма.

О квалиа в произведениях искусства рассказывает Михайлов Игорь Феликсович – кандидат философских наук, старший научный сотрудник Института философии РАН [Философия науки и техники, 2018]. Интересный пример – знаменитая картина Леонардо да Винчи «Портрет госпожи Лизы дель Джокондо».

Представьте, что вам удалось попасть в Лувр, пробраться сквозь толпу зрителей и увидеть знаменитый холст с женщиной с томной улыбкой и темно-русыми волосами в складчатом платье, сидящей на фоне зеленовато-голубовато-розоватого пейзажа. Вы запечатлели в памяти ее образ в разной степени детальности. Но вот вы ушли из Лувра. Где теперь картина? В музее или в вашей (и не только) памяти или еще где-то?

Позитивно мыслящий человек, полагающийся на здравый смысл, утвердительно ответит, что полотно в музее, и посмеется над философами, потому что в миру картины остаются на своих местах, пока их не отправят на другую выставку, на реставрацию или украдут. А химик или физик задумаются, ведь, глядя на Мону Лизу, по большому счету, мы видим форму органических молекул холста и краски. Возможно, после реставрации от оригинального полотна уже мало что осталось? Однако воспринимаем мы, глядя на картины, субъективный образ изображения, который и остается в нашей памяти.

В случае с картиной Леонардо да Винчи создал физико-химическую матрицу, которая при контакте с нашим когнитивным аппаратом создала зрительную иллюзию, которую философы и назвали квалиа.

Великое произведение искусства – это объективная матрица или субъективная иллюзия? В данном случае картина имеет сложную онтологическую структуру как когнитивно-социальный комплекс.

\textbf{Квалиа в играх}

Тема сложности восприятия зрительных и слуховых образов обыгрывается не только в фантастических фильмах, книгах и картинах. Компьютерные игры последних версий изобилуют качественной графикой, многие из них потрясают воображения и позволяют получить новый интересный опыт передвижения, цвета, пространства и т.д.

Популярная игра action-adventure с открытым миром и элементами выживания No Man’s Sky погружает игрока в мир космических и фантастических ощущений. В ней, кстати, используются квалиа разных персонажей, из них даже можно создавать внутренние объекты для игры.

Как создать живой корабль в No Man’s Sky из разрозненных квалиа, читайте в подробной инструкции опытного игрока.

\textbf{Чувственный мир квалиа: манга}

Тема квалиа широко используется в восточной культуре. Так, в Японии создан цикл комиксов манга, которые, судя по отзывам и рецензиям, очень приближены к реальной жизни, причем, по мнению не только японцев, но и русских читателей.

Манга о квалиа описывает жизнь людей в разрезе их ощущений – необычная глубина для подобного драматического жанра. Самые известные и успешные из них:

\begin{enumerate}
    \item Квалиа снега (Yuki No Shita No Qualia). Это манга о гомофобе и женоненавистнике Акио и его кохай-гее Оохаши Юми. Между ними завязывается дружба, Акио предстоит понять – так ли плохи геи, возможно, к ним нужно изменить отношение?
    \item Пурпурная квалиа (Murasakiiro no Qualia). Юкари Марий – девушка с глазами пурпурного цвета, она видит все живые существа как механизмы. Эта способность мешает жить Юкари, но отключить такое видение девушка не может. И все же героиня уверена, что такое зрение помогает разглядеть таланты в людях. Подруга Манабу вскоре увидит невероятные способности Юкари.
\end{enumerate}

Большей популярностью пользуется манга «Квалиа снега», также известная как «Квалиа под снегом». Несколько глав комикса читаются на одном дыхании и востребованы, судя по всему, откровенностью и реалистичностью сюжета. Для знакомства с жанром рекомендуем начать с манга «Пурпурная квалиа», который тоже состоит из нескольких глав.

Предупреждаем: даже если вы не фанат комиксов, эти истории затянут вас в чтение!

\textbf{Актуальность изучения квалиа}

Ставшие предметом вечных споров квалиа для врачей, биологов, философов – не просто интерес. Вероятно, знание и понимание этого явления помогут в решении насущных задач, таких как лечение дальтонизма, ахроматопсии, улучшение коммуникаций в разных сферах жизни.

Несмотря на опровержение возможности создания полноценного искусственного интеллекта, исследователи не опускают руки и ищут способы изучить этот феномен, чтобы сделать нейросети умнее и функциональнее, даже, возможно, человечнее.

Для людей, не относящихся к ученому сообществу, понимание квалиа и умение с ними обращаться – это способ расширить когнитивные способности для более осознанной, интересной жизни в гармонии с собой и окружающими.

На нашей программе «Когнитивистика» вы сделаете первые шаги к расширению способностей мозга мыслить, понимать окружающих и происходящие события, уметь решать бытовые и сложные задачи. А научиться ориентироваться в потоках информации, улавливать нужную и анализировать ее в океане манипуляций, пропаганды и попыток повлиять на мнение и восприятие обывателей поможет программа «Критическое мышление».


\newpage
\section{Прокрастинатор: как понять, что это именно вы?}

\textit{Источник: \url{https://4brain.ru/blog/prokrastinator-kak-ponjat-chto-eto-imenno-vy/}}

Если вы заинтересовались названием этой статьи, значит, вам знакома ситуация, когда сроки уже поджимают, а вы все так и не сделали то, что вам поручено, либо вы уже давно планируете реализовать какую-то идею, но пока не сдвинулись с места ни на йоту… Что же, давайте мы попробуем вам объяснить, что происходит.

Во-первых, придется вам сообщить, что вы метите в прокрастинаторы. Конечно, эта новость (хотя, возможно, и не новость) может вас огорчить, но не спешите печалиться, ведь после окончания статьи вы поймете это ясно и хотя бы будете знать, какая у вас проблема, ведь знание, принятие – это уже немалая дистанция на пути исправления.

Во-вторых, давайте познакомимся с тремя главными героями вашего сознания:

\begin{enumerate}
    \item Рациональный человечек, принимающий решения, ответственный, справляющийся с поставленными задачами в срок, умеющий планировать, понимающий сущность слова «вовремя».
    \item Милая обезьянка, которой интересно все вокруг – сколько и какие спутники у планеты Марс (их два, если что – Фобос и Деймос); как приготовить низкокалорийную лазанью; у какой страны Россия закупает апельсины и т.п. Обезьянка делает вас более эрудированными, ее девизом по жизни является «Просто и весело», она ищет все, что может завлечь и развлечь ее.
    \item Монстр атаки – большое страшное чудище, вгоняющее вас в состояние тревоги, страха, приходящее, когда сроки вот-вот догорят, а работой еще и не пахнет…
\end{enumerate}

Это три основных героя в сознании практически каждого человека. От того, кто из них более часто выходит на авансцену, зависит степень прокрастинации человека [T. Urban, 2016]. Вы могли догадаться, что если большую часть времени с вами пребывает обезьянка, а монстр атаки ваш частый гость, то вероятность того, что вы прокрастинатор, увеличивается.

В-третьих, если все же вы поняли, что история с прокрастинацией про вас, и вам не нужно в этом убеждаться, советуем вам прочитать нашу статью «Прокрастинация. Причины и 10 способов с ней справиться». В случае, если вы замечаете отсутствие дисциплинированности, неорганизованность, частые срывы дедлайнов, низкую продуктивность, если вы нуждаетесь в переменах, рекомендуем вам пройти нашу онлайн-программу «Лучшие техники тайм-менеджмента», где за 5 недель вы изучите избранные техники управления личным временем, целеполагания, планирования и много других полезных знаний.

И теперь давайте немного порассуждаем о том, что такое прокрастинация, чем она вредна, откуда появляется, а затем научимся распознавать симптомы этого нежелательного явления и узнаем, как распознать прокрастинатора.

\textbf{Немного о сущности прокрастинации}

Если подытожить описание сознания с рациональным человечком, обезьянкой и монстром атаки, можно заключить простыми словами: прокрастинация – это постоянное откладывание дел на потом.

Вам нужно написать курсовую? Да впереди ведь еще целых два месяца, можно и подождать. Начальник требует отчет по второму полугодию за несколько дней? Отлично, приступим за день, ведь дел там на пару минут. Отец попросил прикрутить ручку к двери? Да ведь не к спеху – живем же так уже неделю, и ничего…

Чувствуете привкус лени, неэффективного управления временем, попустительства, безответственности, неорганизованности? Это все ближайшие спутники прокрастинации. Именно в совокупности все эти чувства убеждают нас внутри: «Постой, чего торопиться, это можно сделать и попозже… Ведь как-то жили без этого до сих пор, и еще проживем…» Однако жизнь не стоит на месте, и тот, кто не успевает гнаться за ней, увы, рискует остаться позади.

Не стоит путать прокрастинацию с ленью. Последняя приводит к нежеланию делать что-либо вообще, тем не менее при прокрастинации человек не теряет желания делать что-либо, он просто заменяет то, что необходимо делать (как правило, скучно и трудно), на то, что ему делать хочется (как правило, это весело, незатруднительно и доступно).

В чем заключаются причины прокрастинации:

\begin{enumerate}
    \item отсутствие волевых стимулов и мотивации;
    \item страх изменений;
    \item неумение планировать;
    \item отсутствие навыка принятия решений;
    \item паттерн успеха от действий в последний момент;
    \item перфекционизм.
\end{enumerate}

Здесь мы перечислили основные причины, сопровождающие прокрастинацию в ее нехронической форме. Человек может откладывать на потом какую-то обязанность ввиду незначительного вознаграждения за свою работу. Если он при этом не обладает силой воли, то вероятность того, что он станет прокрастинатором, увеличивается. К тому же люди зачастую боятся каких-либо перемен, причем удивительно, что не только неудач, но и успеха.

Навык планирования – один из важнейших в нашей быстрой и разносторонней жизни, требующей хотя бы основ организованности и системности. Кстати, прокрастинация может стать обычным состоянием личности, если ей удается справляться с делами, т.е. добиваться успеха, несмотря на постоянное откладывание. Последняя причина связана с перфекционистами – людьми, всегда готовыми придраться к какой-то мелочи, попытаться довести ее до конца, оттого зачастую оттягивающими конечный вариант [И. Рудевич, 2021].

О более глубоких причинах прокрастинации задумались ученые из Рурского университета в Бохуме. В 2018 году в журнале Psychological Science была опубликована работа четырех ученых, описывающая исследование, связанное с биологическими предпосылками этого феномена. Ученые обследовали 264 мужчины и женщины при помощи магнитно-резонансной томографии на предмет объема различных участков головного мозга и функциональных связей между ними. В дополнение все участники проходили опрос с указанием самостоятельно оценить уровень контроля своих действий.

Была найдена закономерность: участники с меньшим контролем действий имели большую амигдалу (миндалевидное тело), а функциональная связь между миндалевидным телом и так называемой передней поясной корой оказалась менее выраженной. Как раз-таки эти две области головного мозга и отвечают за контроль действий, выбор стратегии поведения при оценке ситуации, что было выяснено в предыдущих научных работах [C. Schlüter, 2018]. Таким образом, люди с описанными физиологическими особенностями испытывают больший страх и тревожность за предстоящее дело, что, скорее всего, становится причиной откладывания на потом.

Допустим, человек осознал, что он прокрастинатор. Но что в этом плохого? Когда выше мы давали определение прокрастинации, мы отметили только один ее аспект – постоянное откладывание. По сути, в этом нет ничего ужасающего – да, мы сдвигаем час Х, но в конце концов ведь выполняем то, что нужно. Однако не спешите делать выводы – прокрастинация не так проста, как кажется на первый взгляд. Главной ее особенностью является появление психологических проблем.

Когда человек откладывает значимые дела на потом, он испытывает стресс от гнетущей ответственности и осознания того, что ему все же придется это делать. Фактически, он прекрасно понимает, что не способен справиться с приоритетностью задач, что влечет появление проблем. Отсюда может развиться чувство вины, также негативно отражающееся на психологическом здоровье. Поэтому следует предотвращать прокрастинацию, бороться с ней.

А сейчас мы расскажем о том, какие существует методы оценки прокрастинации, и начнем с образа типичного прокрастинатора.

\textbf{Психологический портрет прокрастинатора}

Есть прокрастинация, посещающая нас в ежедневной рутине, когда мы можем отложить какое-то неважное дело на потом, но есть прокрастинация, угрожающая нашему благополучию – продуктивности на работе, стабильности в личной жизни, общению с друзьями и т.д. К тому же часто для работодателей бывает полезно распознать потенциального прокрастинатора. Каковы же его главные черты? Как понять, что вы имеете дело именно с человеком, страдающим прокрастинацией?

Вот некоторые черты, на которые следует обращать внимание:

\textbf{Проблемное отношение к изменениям}

В причинах прокрастинации мы указали страх изменений, и это напрямую связано с хронической чертой человека – боязнью перемен. Чтобы перейти из одного проекта в другой, прокрастинатору требуется определенный период, т.к. по завершении первого он чувствует тревогу и отсутствие целей. Поэтому на протяжении какого-то времени ему бывает необходимо справиться с этим состоянием, отвлечь себя на что-то другое – что-нибудь более легкое и непринужденное для ума.

Однако этот период «праздных размышлений» приводит к потерям производительности, к тому же прокрастинатор своим примером может демотивировать коллег по цеху. Поэтому необходимо бороться со своим неумением встречать перемены и браться за новые идеи.

\textbf{Потеря в трех соснах}


Или нежелание эти сосны считать и изучать. Все же справившись со своей проблемой столкновения с изменениями, прокрастинатор приступает к новому проекту. Но ни тут-то было: «Работы непочатый край – к чему подступить, за что взяться, неужели вся эта работа валится на одного человека? Ну нет, это можно и нужно отложить, чтобы разбираться медленно, постепенно…»

Так думает прокрастинатор. Однако ему срочно нужно собраться и шаг за шагом понять, в чем состоит суть проекта, какие у него цели, структура, этапы; ему нужно провести анализ, разбить все на логические блоки. Это и приведет к появлению стимулов и даже вовлечет работника в работу.

\textbf{Неустойчивый список дел}

«Так, вчера мне нужно было отвезти ребенка на плавание, сегодня у меня работа с новыми заказчиками, завтра снова плавание, но заказчики тоже завтра, к тому же нужно приготовить вкусный ужин, ведь родители мужа придут уже сегодня! Ой, а еще я же записалась на стрижку в пол-одиннадцатого завтра утром, тогда кто отведет ребенка на плавание…», – и в этот же момент звонит телефон, начальник напоминает об электронном письме, что должно было быть отправлено вчера вечером, а подруга пишет о приглашении на день рождения на послезавтра… «Зато какой интересный сериал я посмотрела вчера перед сном!»

Узнаете свои мысли? Надеемся, что нет, потому что такой сбивчивый, бессистемный, чересчур гибкий график человека приводит к стрессу. Когда вы, как белка в колесе, пытаетесь успеть по всем фронтам, у вас может быть две причины – либо вы взяли на себя слишком много ролей, т.е. вы не умеете говорить «нет» и отказываться от того, что вам сейчас не нужно, либо вы прокрастинируете. В принципе, оба варианта исправимы. Тем не менее не стоит запускать свою жизнь на такой неподъемный уровень.

\textbf{Акцент на мелких задачах вместо важных и более масштабных целей}

Помните, мы указывали на то, что прокрастинация отличается от лени, главным образом, тем, что прокрастинаторы не пассивны, они осуществляют какие-то действия. Их проблема в том, что эти действия не носят характера важных и необходимых, скорее это бывают незначительные, простые в исполнении дела. Безусловно, в этом смещении приоритетов большую роль играют страхи, указанные выше.

Поэтому нужно просто убедить себя в том, что вы никуда не денетесь от этой работы – рано или поздно вам нужно будет ее выполнить, и для вашей же нервной системы – чем раньше, тем лучше.

\textbf{Постоянные опоздания}

Еще одна черта прокрастинатора – это неумение правильно спланировать свое время, и здесь прямое попадание в любителей опаздывать. Непунктуальные люди думают, что времени еще много, хотя на самом деле уже пора собираться. То же самое происходит и со страдающими синдромом откладывания – вместо того, чтобы приступать к проекту, они полагают, что времени еще очень много, а затем не успевают по всем дедлайнам [business.com, 2020].

\textbf{Исследование Пирса Стила (Pierse Steel)}

В 2007 году американский исследователь-психолог и публичный оратор в области мотивации и прокрастинации из Университета Калгари Пирс Стил опубликовал научную работу, в которой подробно описал черты, свойственные прокрастинаторам. Среди них ученый выделил следующие:

\begin{enumerate}
    \item низкий уровень сознательности, причем сознательность в данном случае включает дисциплинированность, организованность, ориентированность на результат, концентрацию внимания; данная черта наиболее выраженно служит оценкой при определении прокрастинатора;
    \item импульсивность – качество, толкающее человека на руководство, прежде всего, эмоциями (как правило, запредельными) и состоянием, переживаемым в данный момент, т.е. импульсивность – это всегда планирование в краткосрочную; о последствиях, как правило, импульсивные люди думают после;
    \item низкий уровень личной эффективности отражает отсутствие веры в свои возможности, в достижение поставленных целей;
    \item низкая самооценка – стремление занизить собственную значимость, оценить свои возможности ниже, чем они есть на деле; к сожалению, когда человек откладывает дела на потом и возвращается к ним только к истечению сроков, зачастую это влечет неудачи, что в свою очередь вызывает неверие в свои силы, чувство вины, неуверенность и т.д.;
    \item невротическое состояние – склонность к переживанию негативных эмоций и психологических кризисов; эту черту не всегда относят к главным в контексте описания прокрастинаторов, однако в ряде случаев невротизм может привести к синдрому откладывания;
    \item стремление к острым ощущениям – большая проблема прокрастинаторов, связанная с недостатком (по их мнению) интереса и превалированием скуки в жизни; отсюда поиск легких, забавных, веселых способов времяпровождения вместо усердной и монотонной работы;
    \item низкий уровень благожелательности как результат негодования по поводу того, что прокрастинаторы получают какие-то задачи от начальства; бывает так, что прокрастинация – это акт протеста, нежелание подчиняться [P. Steel, 2007].
\end{enumerate}

Помимо вышеотмеченных качеств важным следует признать особенное устройство мозга прокрастинаторов, в результате которого такие люди более ориентированы на настоящее, нежели на будущее, поэтому они мастера перекладывать решение задач на потом.

Другое исследование итальянских ученых было также посвящено прокрастинации. В ходе изучения этого феномена было выяснено, что в организме каждого человека существует ген, отвечающий за выпуск дофамина – важнейшего трансмиттера, играющего значительную роль в способности личности контролировать свое поведение.

Оказалось, что люди, имеющие определенную версию данного гена (своеобразный генетический код), обладают исходным уровнем дофамина, позволяющим осуществлять ровный, стабильный контроль. Однако есть и категория людей с другой версией данного гена. Такие люди обладают меньшим исходным уровнем дофамина, и дополнительный выброс происходит только в стрессовых ситуациях, поэтому прокрастинаторам необходимо испытывать крайнее эмоциональное напряжение, чтобы приобрести контроль над ситуацией.

Тем не менее стоит отметить, что любой механизм в пределах гормональной системы носит комплексный характер, и тот же самый дофамин способствует импульсивности, которая в свою очередь является одним из наиболее ярких характеристик прокрастинации. Поэтому «уровень продвинутости прокрастинатора» зависит от конкретного набора индивидуальных факторов [F. Di Nocera, 2017].



\begin{fancyquotes}
    Узнаете ли вы себя в данном описании? Делитесь комментариями. Если же вы хотите узнать побольше о типичном прокрастинаторе, а также о том, как перестать постоянно откладывать на потом, рекомендуем к просмотру следующий видеоролик: \url{https://youtu.be/pW8xp0_JpFs}
\end{fancyquotes}

Далее мы расскажем о том, какие исследования посвящались оценке прокрастинации и какие методы существуют сегодня в среде ученых-психологов.

\textbf{Тесты на прокрастинацию}

Сам феномен прокрастинации кажется не столь сложным, однако методы его оценки всегда вызывали проблемы у психологов и других заинтересованных лиц. Исследования по данному вопросу опираются в основном на различные опросники, базирующиеся на методе субъективной оценки себя, но каждое по-своему определяет теоретические критерии, в соответствии с которыми человека следует отнести к прокрастинаторам.

Одной из ранних попыток оценить прокрастинацию у личности считается тест Манна (Decisional Procrastination Questionnaire, DPQ), состоящий из пяти выражений, подчеркивающих черту откладывания принятия решений, к примеру, «Я трачу очень много времени на незначительные вопросы до того как принять конечное решение» [L. Mann, 1998]. Другой тест, подготовленный Генри Шоувенбургом в 1995 году, под названием Academic Procrastination State Inventory (APSI) включает уже 23 вопроса/утверждения и основывается на различных научных работах по прокрастинации. Например, одно из утверждений звучало так: «Забываю готовить материал по учебе» [J.R. Ferrari, 1995].

Тем временем развивались и другие инструменты для оценивания прокрастинации как ежедневной тенденции, и здесь в пример можно поставить работу С. Лэя – General Procrastination Scale (GPS), предлагающую опрос из 20 вопросов, в числе которых: «Обычно я откладываю работу, которую мне необходимо сделать» [C.H. Lay, 1986]. Наподобие теста С. Лэя были созданы и другие тесты – Adult Inventory of Procrastination (AIP), включающий 15 выражений наподобие «Мне не удается справляться с делами вовремя» [W. McCown, 1989], а также Aitken Procrastination Inventory (API) – так называемый тест М. Айткен, одно из 19 выражений которого звучит так: «Даже если мне известно, что работа должна быть выполнена, я никогда не приступаю к ее выполнению в нужный момент» [M. Aitken, 1982].

Несмотря на некую схожесть между тестами GPS и AIP, было выявлено, что они могут определять различные виды прокрастинации – возбуждения и избегания – отличающихся друг от друга тем, что первая практикуется, когда человек откладывает в силу поиска новых и более интересных факторов (arousal procrastination), а вторая – когда человек отступает от действий, руководствуясь страхом неудачи и неуверенностью [P. Steel, 2010].

Выше мы уже упоминали имя Пирса Стила, и настало время рассказать о другом его вкладе в изучение прокрастинации. Ученому удалось вывести шкалы по оценке «чистой» прокрастинации, используя мета-анализ на основе предыдущих исследований, а также факторный анализ (Procrastination Scale Items, PSI, и Irrational Procrastination Scale, IPS).

Выяснилось, что прокрастинация представляется единым цельным феноменом (так называемая общая прокрастинация) с большой классификацией, поэтому некоторые отдельные критерии, использованные в предыдущих опросниках, были просто опущены (например, состояние спешки или оперативность). Также Стила интересовало противоречивое свойство прокрастинаторов: с одной стороны, они понимают, что это навредит им в будущем, с другой стороны, ничего не меняют [Н. Клепикова, И. Кормачева, 2019].

Наибольшего успеха добилась шкала иррациональной прокрастинации (IPS) в силу значимости фактора негативной оценки себя, что и является главным отрицательным последствием такого явления как откладывание на потом. Ниже вы можете пройти данный тест.

\textbf{ТЕСТ (Шкала иррациональной прокрастинации Пирса)}

Пройдите данный тест и узнайте, насколько прокрастинация стала привычной для вас.

Во всех вопросах выберите ответ, в наибольшей мере описывающий ваше отношение к указанному выражению.

\textit{(1) Я откладываю дела в такой мере, что это нельзя назвать оправданным}

\begin{enumerate}
    \item Очень редко
    \item Редко
    \item Иногда
    \item Часто
    \item Очень часто
\end{enumerate}

\textit{(2) Я выполняю свои обязанности тогда, когда к ним действительно нужно приступить}
\begin{enumerate}
    \item Очень часто
    \item Часто
    \item Иногда
    \item Редко
    \item Очень редко
\end{enumerate}

\textit{(3) Зачастую я жалею о том, что не приступил(а) к делам раньше}
\begin{enumerate}
    \item Нет
    \item Скорее нет
    \item Средне
    \item Скорее да
    \item Да
\end{enumerate}

\textit{(4) Есть некоторые моменты моей жизни, решение которых я откладываю, хотя знаю, что этого не следует делать}
\begin{enumerate}
    \item Нет
    \item Скорее нет
    \item Средне
    \item Скорее да
    \item Да
\end{enumerate}

\textit{(5) Если что-то срочно необходимо сделать, я приступлю к выполнению этого, отложив более мелкие задачи на потом}
\begin{enumerate}
    \item Да
    \item Скорее да
    \item Средне
    \item Скорее нет
    \item Нет
\end{enumerate}

\textit{(6) Я откладываю вещи на слишком долгий период, от чего страдают мои благополучие и производительность}
\begin{enumerate}
    \item Нет
    \item Скорее нет
    \item Средне
    \item Скорее да
    \item Да
\end{enumerate}

\newpage
\textit{(7) В конце дня я думаю о том, что мог(ла) бы потратить время более правильно}
\begin{enumerate}
    \item Очень редко
    \item Редко
    \item Иногда
    \item Часто
    \item Очень часто
\end{enumerate}

\textit{(8) Я трачу время с умом}
\begin{enumerate}
    \item Да
    \item Скорее да
    \item Средне
    \item Скорее нет
    \item Нет
\end{enumerate}

\textit{(9) Когда мне следует выполнять одно дело, я делаю другое}
\begin{enumerate}
    \item Очень редко
    \item Редко
    \item Иногда
    \item Часто
    \item Очень часто
\end{enumerate}


\textit{Результаты:}

\begin{itemize}
    \item Менее 19 – Поздравляем! Вы относитесь к группе людей, знакомых с прокрастинацией очень плохо. Ваш девиз по жизни «Делу – время, потехе – час».
    \item 20-23 – Отличный результат! Вы едва знакомы с прокрастинацией. Продолжайте в том же духе!
    \item 24-31 – Средний результат. Вы входите в 50\% людей, которые прокрастинируют от случая к случаю. Главное – контролируйте этот процесс, будьте бдительны с таким врагом как прокрастинация!
    \item 32-36 – К сожалению, вы прокрастинатор. Попробуйте научиться контролировать эту привычку, научитесь приступать к делам сразу же.
    \item 37 и выше – Ваше второе имя – «Завтра». Вы настолько поглощены прокрастинацией, что, скорее всего, едва завершили этот тест, не отложив его на потом. Самое время заняться собой!
\end{itemize}

Надеемся, что ваш результат удовлетворителен. Если же нет, советуем обратиться к нашей онлайн-программе «Психическая саморегуляция», где за 6 недель вы научитесь бороться со стрессом, появляющимся в результате разных причин, в том числе и прокрастинации, а также обретете большую уверенность, избавитесь от страхов и научитесь большему контролю своих эмоций.

\newpage
\textbf{Итоги}

Как видно, тесты по прокрастинации созданы в большом количестве и разнообразии учеными, психологами, и сегодня они применяются в различных исследованиях, а также на обывательском уровне, когда человек желает проверить себя на уровень прокрастинации.

Не забывайте, что если прокрастинация пробирается активной поступью в вашу жизнь, ничего кроме хаоса, разрушения и стресса в долгосрочной перспективе она не принесет. Поэтому будьте наготове и сражайте врага разумными доступными методами.

Желаем вам успехов!


\newpage
\section{Опасность примирительного секса для отношений}

\textit{Психолог Дарья Суслова назвала примирительный секс опасным для отношений }

\textit{Источник: \url{https://lenta.ru/news/2023/05/31/quarrel/}}


Примирительный секс может быть опасным для отношений, так как многие с помощью него сглаживают проблемы вместо того, чтобы решать их. Клинический психолог Дарья Суслова в беседе с «Лентой.ру» назвала привычки, приводящие к ссорам и расставанию.

Врач объяснила, что некоторым людям с появлением любимого человека хочется делить с ним все — печали, радости, увлечения, друзей. Однако рано или поздно это начинает надоедать одному из партнеров. По ее словам, растворяться в партнере больше склонны женщины. «Многие мужчины решаются на развод, аргументируя его тем, что прежде яркая личность рядом с ним превратилась в человека фактически без интересов и друзей. Причем им невдомек, что таким способом любимая девушка пыталась продемонстрировать, насколько дороги ей эти отношения», — комментирует психолог.

Еще одна \explain{п\'{а}губная прив\'{ы}чка}{bad habit} --- стремление думать и говорить за того, кто рядом, продолжила Суслова. Это проявляется, когда человек говорит за двоих, употребляя местоимение «мы». Она \ed{пояснила}{пояснять/пояснить}{to explain}, что за этой привычкой может скрываться стремление \explain{мерить партнера по своей мерке}{\textit{lit.} to measure one's partner according to one's own measures}, рассуждая о его взглядах на жизнь, отвечая за него на вопросы. «Это та самая анекдотичная ситуация, когда один в паре решает, какие фильмы, рестораны нравятся обоим», — иронично отметила психолог.

\begin{fancyquotes}
    Иногда люди и вовсе \ed{проецируют}{проецировать}{project} на других свои комплексы по поводу внешности, веса или выбора одежды. Появляется привычка критиковать спутника жизни даже за попытки найти свой стиль или самовыражения. \explain{До пор\'{ы} до времени}{for the time being} человек может не замечать, что навязывает другому свое мнение.

    \begin{flushright}
        Дарья Суслова,\\
        клинический психолог
    \end{flushright}
\end{fancyquotes}

Еще одной привычкой, которая \explain{грозит}{threatens} \ed{разрывом отношений}{разрыв отношений}{a break up}, Суслова назвала избегание проблем или их отрицание. Психолог подчеркнула, что ни один из партнеров не обязан соглашаться с любым высказыванием или точкой зрения любимого человека. Однако выслушивать того, кто рядом, и обсуждать острые вопросы необходимо.

\ed{Неумение}{неумение}{inability} искренне прощать грозит ухудшением отношений в паре, предупредила специалистка. «\ed{Руководствуясь какими-то причинами}{рук\'{о}водствуясь какими-то причинами}{for some reason}, вы можете сказать любимому человеку, что простили его, но в глубине души продолжите испытывать боль и \explain{раз за разом}{every now and again} возвращаться к неприятным воспоминаниям. От этого страдают оба: ваш партнёр думает, что неприятный инцидент \explain{исчерпан}{\textit{here:} is over}, вы продолжаете прокручивать в голове болезненные события», — разъяснила Суслова.


\begin{fancyquotes}
    Обиды будут накапливаться и со временем начнут напоминать о себе при каждом конфликте. Два-три подобных эпизода и даже искреннее желание выстроить конструктивный разговор о проблемах в паре превратится в обмен\footnote{обмен + чем?} \ed{обвинениями}{обвинение}{accusation}. Если чувствуете, что слукавили и не смогли забыть причину обид, обсудите ее с партнером.

    \begin{flushright}
        Дарья Суслова,\\
        клинический психолог
    \end{flushright}
\end{fancyquotes}

Психолог напомнила, что сравнивать партнера с другими людьми — плохая идея. По ее мнению, человеку может казаться, что подобными сравнениями реально мотивировать спутника жизни на перемены, но ничего, кроме злости, такие сравнения не вызывают.

\begin{fancyquotes}
    Склонность критиковать партнера и сравнивать его с другими обижает и утомляет, даже когда у вас нет злого умысла. Если такая привычка стала постоянным спутником отношений, то можно говорить об эмоциональной манипуляции. Для здоровых отношений важно принимать любимого человека таким, какой он есть, а не пытаться переделать его.

    \begin{flushright}
        Дарья Суслова,\\
        клинический психолог
    \end{flushright}
\end{fancyquotes}

Опасно для отношений сглаживать острые углы с помощью секса, уверяет Суслова. «Секс после ссоры называют make-up sex. "Макияжный" он лишь потому, что скрывает проблемы, когда видимого решения для них нет. Разногласия никуда не исчезнут, а вот навыки вербального примирения и умения полноценно выстраивать диалог — да», — предупредила она. Психолог заключила, что ничего плохого в сексе нет, однако проблемы начинаются тогда, когда его используют вместо конструктивного обсуждения причин конфликта или их разбора на сеансе со специалистом.

Ранее тренер назвал факторы, снижающие либидо и ухудшающие отношения. Для поддержания сексуальной активности он посоветовал заниматься спортом.




\newpage
\section{Уроки хладнокровия}

\textit{Источник: \url{https://reallanguage.club}}

% https://reallanguage.club/russkie-teksty-prodvinutogo-urovnya-s-audio/uroki-xladnokroviya/

\textit{Автор: Алексей Овчинников}

\textbf{Note:} C1-level text!

Обида, гнев, страх и многие другие эмоции каждый день сопровождают наше сознание, лишая \ex{рассудок}{judgement} трезвости и \ed{здравомыслия}{здравомыслие}{reasonableness}. \ed{Порою}{порою}{sometimes}, «встав не с той ноги», можно \ex{з\'{а}просто}{just like that, easily} \ex{поругаться}{(\textit{colloq.}) to quarrel}, например, с \ed{начальством}{начальство}{(\textit{colloq.}) chief, boss}, а это совсем уже не нужно и только во вред. Эмоции вызываются поступками, действиями, но, чаще всего, словами. И теми же поступками, действиями и словами можно \ex{противопоставить}{to constrast} негативу позитив.

Вас толкнули в метро? О, как хочется этому \ed{гаду}{гад}{(\textit{figurative, derogatory}) creep, repulsive person, vile creature, scoundrel, cad, asshole, bastard} пяткой, да по ноге наступить, а потом ещё и \ex{растереть}{to grind, to rub}, как \ex{ок\'{у}рок}{cigarette butt}! Зачем? Сначала вы получите от этого облегчение, а потом ещё и погавкаетесь с ним, может, и подерётесь, и доберётесь до пункта назначения в препротивнейшем настроении.

Попробуйте-ка, вместо этого, улыбнуться ему и сказать «Ой, простите, пожалуйста, я такой неловкий», как будто не он виноват, а вы, да ещё сделайте это с юмором, \ed{этаким}{\'{э}такий}{such} сарказмом, только не переборщите, пусть ваш голос звучит просто и дружелюбно. И, поверьте, неуклюжий «джентльмен» удивится, \ex{смутится}{will get embarrassed}, а может и покраснеет и, в свою очередь, тоже извинится и вы вместе посмеётесь над ним и вашим остроумием.

Вот представьте, опять-таки, общественный транспорт — автобус только что остановился, открылась \ex{передняя дверь}{front door} с турникетом, чтобы впустить садящихся, а потом уже (чтобы не было зайцев) откроются остальные двери, чтобы выпустить сходящих. Но какая-то нетерпеливая «\ex{кошёлка}{(\textit{colloq.}) ugly woman}» уже жмёт на звонок, требуя непременно открыть ту дверь, у которой она стоит, причём, жмёт, не переставая, как будто, если прямо сейчас не сойдёт, то начнётся Третья Мировая! От трескучего звона уже звенит в ушах. Весь автобус уже вожделеет двинуть тяжёлым портфелем по темечку источник раздражителя.

«Тётенька, а вы можете на звонке \ed{частушки}{частушка}{ditty} наиграть? А я вам подпою --- давайте?», --- когда я это выкрикнул, весь автобус грохнул со смеху, а тётка так смутилась, что не ответила ни слова и, назло всем, продолжала жать чёртову кнопку, но нам уже было всё равно, источник раздражения превратился в источник хорошего настроения, все смеялись от души.

А когда дверь таки открылась, товарищ, ехавший со мной, \ex{заразительно}{contagiously} часто зааплодировал, многие его поддержали. Остальную часть пути в автобусе как будто праздник наступил --- ехали, громко разговаривая о своём, но только на «будьте любезны» и, непременно, шутя!

Это \ex{наглядный пример}{visual example} поведения в условиях поступков и действий. Надо просто, вопреки логике, поступить не так, как хочется, а совсем наоборот. Но, всё-таки, гораздо чаще нам приходится сталкиваться с эмоциональным срывом в \ex{словесной}{словесный}{verbal, oral} форме.

Чаще всего это происходит на работе или в семье. Здесь всё далеко не так просто. Когда обижают словом, то не получится \ex{отшутиться}{laugh it off}, ведь мы видим этого человека \ex{каждый божий день}{every single day}, а если это начальник, то и вовсе за шутку можно и \ed{под сокращение попасть}{попасть под сокращение}{get laid off}. Как же быть? Здесь вместо шутки оружием должна служить серьёзность.

Вот пример. «Кровопийца! Все нервы мои истрепал! Куда твои родители смотрят?! И как таких паразитов вообще в школу пускают?!», --- это часто приходится слышать школьникам и студентам. В этом случае беседующие не равны статусом и учащемуся придётся подстраиваться под преподавателя.

Разберём гневную тираду. Такие слова, как «кровопийца» и «паразит» являются не несущими смысл, а скорее скрывающими его своей яркостью. Эти слова выражают эмоции, но из-за них смысл сказанного не доходит в полной мере до адресата. Учащийся просто обижается, а учащаяся и заплакать может, и истерику закатить.

Переведём высказывание, выбросив весь «мусор», получится «Своим плохим воспитанием (или успеваемостью) ты меня нервируешь». Этот вариант даже на \ex{выговор}{reprimand, scolding} не тянет и звучит совсем не так обидно, да и значение его --- лишь констатация разницы в воспитании (или знаниях) преподавателя и учащегося.

Если научить учащегося, в первую очередь, вот такой вот простой психологии --- переводить гневные выкрики в спокойные замечания, то учащийся скорее поймёт всю вредность профессии преподавателя и просто перестанет его раздражать, или научится, хотя бы, не реагировать на обидные слова. А вот этого --- ой, как не хватает современному обществу.

Теперь спроецируем ситуацию на двух равных по статусу людей. В ответ на «гневные \ed{в\'{о}пли}{вопль}{scream}» можно и «по фэйсу» схлопотать, так что стоит не только не воспринимать обиду (как известно, обидчик сам будет не рад отсутствию вашей реакции), но и стараться самим не обижать.

Вот ещё пример. Дело было в моей редакции (я работаю в газете). «Что за хрень ты тут понаписал, неужели ты думаешь, что твои \ex{кар\'{а}кули}{scribbles} появятся на наших страницах?!», --- это я сорвался на молодом репортёре, а он, повесив голову, повернулся и стал уходить.

Я его вернул и извинился (да-да --- извинился, хотя он у меня в подчинении, а не я у него!) и тут же перевёл свою же фразу на другой язык: «Твоя статья слишком велика и тема не соответствует рубрике, да и над почерком стоит поработать, ведь мне это читать предстоит». Тогда он всё понял и через полчаса выдал отличный материал!

Отнеситесь к обиде как к критике и постарайтесь сами не обижать, а лучше критикуйте. Удачи.



\newpage
\section{Сила эмпатии}

\textit{Донни Эбенштейн}

\textit{Источник: \texttt{https://4brain.ru/blog/сила-эмпатии-донни-эбенштейн/}}

Ты едешь ранним \ed{погожим}{погожий}{lovely (of weather), fine} утром по шоссе и \ex{тебе никак не обогнать}{you can't overtake} новичка с буквой «У» в левом верхнем углу, который едет настолько медленно, что можно выйти из машины, не останавливаясь, и так же не спеша её догнать. Ты начинаешь злиться, настроение портится, мозг занят не \ed{любованием}{любование}{admiration} красотами загородной природы и не \ed{предвкушением}{предвкушение}{anticipation} встречи с семьей, а подбором нелестных эпитетов по поводу \ed{раздражителя}{раздражитель}{irritant, trigger, stimulus} впереди.

Или подруга в очередной раз пришла в гости жаловаться на начальство, которое не ценит, не повышает заработную плату и заставляет задерживаться после окончания рабочего времени. Ты, \ex{негодуя}{(being) indignant}, советуешь бросить эту работу и искать себе новую, уверяя подругу, что та --- высококлассный специалист. (Хотя в действительности понятия не имеешь, что именно она представляет из себя на работе.)

\begin{center}
    \Large
    Как связаны эти две ситуации?
\end{center}

В обоих случаях человек \ex{ставит во главу угла}{prioritises} себя и свои интересы и не задаётся вопросом: что может думать о сложившейся ситуации другая сторона --- в первом случае начинающий водитель, во втором начальство подруги. Чтобы найти выход из неприятных ситуаций, иногда достаточно просто включить воображение и представить себя на месте другого участника конфликта. Начать думать \ex{в подобном ключе}{in such a way} для человека, к этому не привыкшего, может быть довольно сложно. Но стоит начать, и многие жизненные ситуации не \ed{дорастают}{дорастать (до чего)}{to grow as much as...; доростаю, -ешь, -ют} до конфликта. Если водитель из ситуации №1 вспомнил бы, что его дочь полгода назад села за руль и периодически \ed{переживает}{переживать/пережить}{here: to be worried; it also means to experience, to go through; to endure, to suffer, to ache; to survive, to outlive} из-за \ed{несдержанности}{несдержанность}{lack of restraint} других более опытных водителей на дороге, которые сигналят и подрезают её, то с большой вероятностью он бы просто завернул в магазин по дороге или терпеливо ждал прерывистой линии, чтобы совершить безопасный манёвр.

Именно умение взглянуть на ситуацию с другой стороны называется \textit{эмпатией}. Автор \ed{подмечает}{подмечать/подметить}{to notice}, что одно из важных принципов парадигмы верхнего и нижнего этажа (обеих сторон конфликта) --- изменить свой угол зрения, не отказываясь от собственного видения. Эбенштейн называет это «перемещение перспективы». Анализируя опыт, который он приобрел, возглавляя свою компанию, которая занималась поддержкой в проведении переговоров, достижении соглашений и разрешении конфликтов, автор призывает нас помнить несколько постулатов:

\begin{enumerate}
    \item Мы все иногда заходим в тупик.
    \item То, в какие именно \ed{застойные}{застойный}{stagnant} ситуации вы попадёте, \ex{обусловлено}{is due to} вашей личностью.
    \item Возможно, вы не сразу поймёте, в чём, собственно дело.
    \item У проблемы не всегда существует чёткое и объективно правильное решение.
    \item Гибкое мышление позволяет нам найти несколько вариантов.
    \item Если вы принимаете на время чужую точку зрения, это не означает, что вы должны отказаться от собственной.
    \item Решение должно соответствовать вашим личным особенностям.
    \item Какой-то выход есть всегда.
    \item Умение приходит с опытом.
\end{enumerate}

Купить и прочитать книгу «Сила эмпатии» можно на сайте издательства.

\clearpage

\section{О Гневе}

\textit{Сенека, О гневе, Книга I}

\textit{Источник: \url{https://ancientrome.ru/antlitr/t.htm?a=1354021917}}

\textit{Аудиокнига в Ютубе \url{https://youtu.be/O1V3R_pVbe0?si=ByPpBX1xvSbkylD9}}


1. (1) Ты про­сил меня напи­сать, Новат, как справ­лять­ся с гне­вом. Прось­ба твоя пока­за­лась мне осно­ва­тель­ной: нам сле­ду­ет боять­ся гне­ва боль­ше, чем всех про­чих чувств, вол­ну­ю­щих нашу душу, как само­го отвра­ти­тель­но­го и само­го \ex{неукро­ти­мо}{indomitable}-буй­но­го. В самом деле, осталь­ные чув­ства наши хоть вре­ме­на­ми быва­ют спо­кой­ны­ми и мир­ны­ми, но гнев --- это все­гда \ex{воз­буж­де­ние}{excitement, arousal}, раз­дра­же­ние, взрыв; \ed{неисто­вая}{не\'{и}стовый}{furious, violent, frantic (не\'{и}стовый гнев)}, нече­ло­ве­че­ская жаж­да ору­жия, кро­ви, каз­ни; гне­ву без­раз­лич­но, что ста­нет­ся с ним самим, лишь бы навредить дру­го­му; он \ed{лезет пря­мо на рож\'{о}н}{лезть на рож\'{о}н}{to ask for trouble}, горя жела­ни­ем \ed{ото­мстить}{мстить/ото­мстить}{to revenge oneself, to avenge (to take vengeance for)} во что бы то ни ста­ло, хотя бы ценой жиз­ни \ed{мсти­те­ля}{мсти­те­ль}{avenger}.

(2) Вот отче­го неко­то­рые из муд­рых мужей назы­ва­ли гнев крат­ковре­мен­ным \ed{поме­ша­тель­ст­вом}{поме­ш\'{а}­тель­ст­во}{madness, insanity}. И вер­но: гнев так же не вла­де­ет собой, не забо­тит­ся о \ed{при­ли­чи­ях}{при­ли­чи­е}{decency, propriety, decorum}, не пом­нит род­ства; так же \ed{упо­рен}{упо­рный}{persistent, insistent} и \ed{целе­устрем­лён}{целеустремлённый}{goal-driven, goal-oriented, focused (on a goal)} в том, за что взял­ся, так же наглу­хо закрыт для сове­тов и дово­дов разу­ма; так же воз­буж­да­ет­ся по само­му пусто­му пово­ду; неспо­со­бен \ex{отли­чать}{to distinguish} спра­вед­ли­вое и истин­ное;
чело­век во гне­ве подо­бен \ed{руша­ще­му­ся}{руша­щеийся (рушиться)}{collapsing} дому, кото­рый сам \ex{раз­ва­ли­ва­ет­ся}{is falling apart} на кус­ки над тела­ми тех, кого \ex{зада­вил}{crushed}.
(3) При­смот­рись, как ведут себя люди, одер­жи­мые гне­вом, и ты убедишь­ся, что \ex{они не в сво­ем ум\'{е}}{they are out of their minds}. Есть несколь­ко вер­ных при­зна­ков, поз­во­ля­ю­щих отли­чить \ed{буй­но­по­ме­шан­ных}{буй­но­по­ме­шан­ный}{тот, кто находится в состоянии буйного помешательства}: выра­же­ние лица бес­страш­ное и угро­жаю­щее; \ex{чело}{\textit{высок.}, \textit{поэт.} лоб} \ex{нахму­ре­но}{frowning}; взор дик и мра­чен; поход­ка тороп­ли­ва, руки в бес­пре­стан­ном дви­же­нии; цвет лица необыч­ный; дыха­ние \ed{пре­р\'{ы}­ви­стое}{пре­р\'{ы}­ви­стый}{intermittent} и \ed{уча­щён­ное}{учащённый}{quickened}. Тако­вы же отли­чи­тель­ные при­зна­ки раз­гне­ван­ных:
(4) гла­за горят и свер­ка­ют; все лицо чрез­вы­чай­но крас­ное, ибо \ex{кипя­щая кровь}{boiling blood} под­ни­ма­ет­ся из самой глу­би­ны серд­ца; губы тря­сут­ся, зубы \ex{стис­ну­ты}{squeezed}, воло­сы шеве­лят­ся и вста­ют дыбом; дыха­ние выры­ва­ет­ся со сви­стом и \ed{шипе­ни­ем}{шипе­ни­е}{hissing sound}; \ed{суста­вы}{суста­в}{(\textit{anat.}) joint} тре­щат, выво­ра­чи­ва­ясь; он стонет, \ex{рычит}{growls}, речь его пре­ры­ва­ет­ся и пол­на мало­по­нят­ных слов; он то и дело хло­па­ет в ладо­ши, \ex{топа­ет нога­ми}{stomps his feet}; все тело его дро­жит от воз­буж­де­ния, гро­зя вели­ким гне­вом; страш­ное и \ex{оттал­ки­ваю­щее}{repulsive} лицо --- \ed{иска­жён­ное}{искажённый}{distorted}, \ex{рас­пух­шее}{распухший}{swollen}; не зна­ешь, чего боль­ше в этом \ed{поро­ке}{поро­к}{vice; flaw, defect} --- дур­но­го или \ed{без­образ­но­го}{без­образ­ный}{ugly, hideous}. (5) Дру­гие поро­ки мож­но скрыть, пре­да­ва­ясь им без свиде­те­лей; гнев не спря­чешь --- он весь отра­жа­ет­ся на лице, и чем он силь­нее, тем замет­нее.

Раз­ве ты не заме­чал, что у всех живот­ных напа­де­ние \ex{пред­ва­ря­ет­ся}{is preceded by} осо­бы­ми при­зна­ка­ми и тело их \ex{утра­чи­ва­ет}{loses} обыч­ный спо­кой­ный облик, на гла­зах още­ти­ни­ва­ясь и ста­но­вясь более диким? (6) У каба­н\'{а} высту­па­ет из пасти \ex{пена}{foam}, он начи­на­ет с гром­ким \ed{ляз­гом}{лязг}{clank, clang, clack} точить \ed{клы­ки}{клык}{canine tooth} друг о дру­га; быки тря­сут рога­ми в возду­хе и взры­ва­ют копы­та­ми песок, львы рычат, рас­сер­жен­ные змеи \ed{наду­ва­ют}{наду­вать/надуть}{inflate} шеи, беше­ные соба­ки глядят мрач­но. Самый страш­ный и опас­ный от при­ро­ды зверь в при­сту­пе гне­ва начи­на­ет выглядеть еще более диким и гроз­ным. (7) Я знаю, дру­гие чув­ства тоже труд­но скрыть: и \ex{похоть}{lust}, и \ex{испуг}{fright}, и \ex{дер­зость}{insolense, rudeness} выда­ют себя извест­ны­ми при­зна­ка­ми, их мож­но уга­дать. Ни одно более или менее силь­ное внут­рен­нее воз­буж­де­ние не может не отра­зить­ся на лице. В чем же тогда отли­чие гне­ва? Про­чие чув­ства быва­ют замет­ны­ми; гнев же выда­ет­ся настоль­ко, что не замет­но уже ниче­го дру­го­го.

2. (1) Хочешь, посмот­ри на раз­ру­ши­тель­ные дей­ст­вия гне­ва: ни одна \ex{чум\'{а}}{plague, pestilence} не обо­шлась чело­ве­че­ско­му роду так доро­го. Ты увидишь \ed{окро­вав­лен­ные}{окро­вав­лен­ный}{bloody} тру­пы; при­готов­лен­ные яды; пото­ки гря­зи, кото­рой обли­ва­ют друг дру­га \ex{истец и ответ­чик}{plaintiff and defendant}; \ex{раз­ва­ли­ны горо­дов}{city ruins}; \ex{истреб­лен­ные пле­ме­на}{exterminated tribes}; голо­вы вождей, пус­кае­мые с молот­ка; \ed{факе­лы}{ф\'{а}ке­л}{torch}, под­жи­гаю­щие \ed{кров­ли}{кров­ля}{roof}; пожа­ры, рас­про­стра­ня­ю­щи­е­ся за город­ские сте­ны, и целые огром­ные стра­ны, \ex{пылаю­щие}{blazing} в непри­я­тель­ском огне. (2) Посмот­ри, что оста­лось от зна­ме­ни­тей­ших неко­гда государств: мы с трудом раз­ли­ча­ем послед­ние следы их фун­да­мен­тов --- их раз­ру­шил гнев. Посмот­ри на пусты­ни, тяну­щи­е­ся на мно­гие и мно­гие мили без еди­но­го жите­ля: их опу­сто­шил гнев.
Посмот­ри, сколь­ко вели­ких вождей оста­лось в памя­ти потом­ков при­ме­ра­ми \ed{зло­счаст­ной}{злосчастный}{    (\textit{dated}) ill-starred, ill-fated} судь­бы: одно­го гнев \ed{зако­лол}{заколоть}{to slaughter (animal), to kill by stabbing} на соб­ст­вен­ной посте­ли, дру­го­го \ed{сра­зил}{сразить}{to hit} за \ed{пир­ше­ст­вен­ным сто­лом}{пир­ше­ст­вен­ный сто­л}{feast table} вопре­ки свя­щен­ным зако­нам тра­пезы, третье­го рас­тер­зал в кло­чья на гла­зах запол­нен­но­го наро­дом фору­ма, где, каза­лось бы, место одним лишь законам\footnote{Речь идет о гибе­ли пре­то­ра Азел­ли­о­на на фору­ме. Ср.: Аппи­ан. Граж­дан­ские вой­ны. I. 6. 54.}, кровь чет­вер­то­го про­лил соб­ст­вен­ный сын, пято­му пере­ре­за­ла цар­ст­вен­ное гор­ло раб­ская рука, шесто­го \ex{рас­пя­ли на кре­сте}{was crucified}. (3) До сих пор я гово­рю лишь об отдель­ных слу­ча­ях; но, может быть, тебе захо­чет­ся, оста­вив тех, кого гнев \ex{испе­пе­лил}{incinerated} пооди­ноч­ке, взгля­нуть на целые собра­ния, выре­зан­ные мечом; на тол­пы плеб­са, пере­би­тые сол­да­та­ми; на целые наро­ды, при­го­во­рен­ные к смер­ти и обре­чен­ные сов­мест­ной гибе­ли... (лаку­на) ...

(4) слов­но они пре­не­бре­га­ют нашей заботой или пре­зи­ра­ют наш авто­ри­тет. Как ты дума­ешь, отче­го народ гне­ва­ет­ся на гла­ди­а­то­ров, и гне­ва­ет­ся так неспра­вед­ли­во, оби­жа­ясь, что они гиб­нут без осо­бой охоты? Народ реша­ет, что его пре­зи­ра­ют, и вот уже все --- лица, жесты, пыл --- обли­ча­ет в нем не зри­те­ля, а про­тив­ни­ка. (5) Но что бы это ни было, это не гнев, а некое подо­бие гне­ва, как дети, когда упа­дут и уши­бут­ся, хотят, чтобы высек­ли зем­лю; они часто сами не зна­ют, отче­го гне­ва­ют­ся, про­сто гне­ва­ют­ся, без при­чи­ны и без обиды; впро­чем, они все­гда име­ют в виду какую-нибудь вооб­ра­жае­мую обиду и жаж­ду вооб­ра­жае­мо­го нака­за­ния обид­чи­ка. Сде­лай­те вид, буд­то вы их бье­те, и они пове­рят; затем сде­лай­те вид, что вы пла­че­те, и про­си­те про­ще­ния, и они успо­ко­ят­ся, так как нев­сам­де­лиш­ная боль про­хо­дит от удо­вле­тво­ре­ния нев­сам­де­лиш­ным мще­ни­ем.

3. (1) Нам воз­ра­зят: «Мы часто гне­ва­ем­ся не на тех, кто при­чи­нил нам зло, а на тех, кто еще толь­ко соби­ра­ет­ся это сде­лать; сле­до­ва­тель­но, гнев рож­да­ет­ся не из обиды». Вер­но, мы гне­ва­ем­ся на тех, кто соби­ра­ет­ся при­чи­нить нам зло, но самая мысль об этом уже есть зло, и соби­раю­щий­ся нане­сти обиду уже есть обид­чик.

(2) «Но гнев не есть так­же и жаж­да возмездия\footnote{Опре­де­ле­ние гне­ва у Посидо­ния: Ira est cupiditas ulciscendae iniuriae.}, --- воз­ра­зят нам, --- посколь­ку самые бес­по­мощ­ные часто гне­ва­ют­ся на самых могу­ще­ст­вен­ных, но не жаж­дут воз­мездия, ибо не могут на него наде­ять­ся». Преж­де все­го, мы гово­ри­ли, что гнев есть жаж­да воз­мездия, а не воз­мож­ность осу­ще­ст­вить его; желать могут и те, кто сде­лать не может. Кро­ме того, нет чело­ве­ка, сто­я­ще­го так низ­ко, чтобы он не мог наде­ять­ся как-нибудь нака­зать дру­го­го, даже если тот зани­ма­ет самое высо­кое поло­же­ние; что-что, а вредить все люди уме­ют непло­хо. (3) Ари­сто­телев­ское опре­де­ле­ние гне­ва близ­ко к наше­му; он гово­рит, что гнев есть жела­ние воздать болью за боль\footnote{Ари­сто­тель. О душе. 403a 30: «Диа­лек­тик опре­де­лил бы гнев как стрем­ле­ние ото­мстить за оскорб­ле­ние (букв.: стрем­ле­ние воздать болью за боль)… рас­суж­даю­щий же о при­ро­де --- как кипе­ние кро­ви око­ло серд­ца» (пер. П. С. Попо­ва).}. Чем отли­ча­ет­ся наше опре­де­ле­ние от это­го, объ­яс­нять дол­го. Но про­тив обо­их выдви­га­ет­ся обыч­но такое воз­ра­же­ние: что, мол, дикие зве­ри гне­ва­ют­ся, не будучи оби­же­ны, и не отто­го, что стре­мят­ся при­чи­нить кому-то боль или добить­ся спра­вед­ли­во­го нака­за­ния; ибо даже если в резуль­та­те имен­но так и полу­ча­ет­ся, то цель у них иная. (4) На это сле­ду­ет отве­тить, что дикие зве­ри вооб­ще не зна­ют гне­ва, рав­но как и все про­чие живот­ные, за исклю­че­ни­ем чело­ве­ка. Ибо гнев, будучи вра­гом разу­ма, не рож­да­ет­ся там, где нет разу­ма. У диких зве­рей быва­ет воз­буж­де­ние, бешен­ство, сви­ре­пость, склон­ность напа­дать на вся­ко­го; но гне­ва у них не быва­ет, так же как и склон­но­сти к рос­ко­ши, хотя в иных стра­стях и удо­воль­ст­ви­ях они быва­ют даже раз­нуздан­нее, чем чело­век. (5) Не верь тому, кто гово­рит:

Вепрь свой гнев поза­был; лань не ищет спа­се­нья в бег­стве, И на рога­тых коров напа­дать пере­ста­ли медведи4.

«Гне­вом» поэт назы­ва­ет их воз­буж­де­ние, готов­ность бро­сить­ся, напасть; в дей­ст­ви­тель­но­сти, они не более спо­соб­ны гне­вать­ся, чем про­щать. (6) У бес­сло­вес­ных живот­ных нет чело­ве­че­ских чувств, хотя есть неко­то­рые похо­жие побуж­де­ния. В про­тив­ном слу­чае, если бы они уме­ли любить и нена­видеть, меж­ду ними суще­ст­во­ва­ли бы друж­ба и враж­да, раздо­ры и согла­сия. Кое-какие следы таких чувств у них, прав­да, мож­но обна­ру­жить, одна­ко вооб­ще-то все дур­ное и хоро­шее может жить лишь в чело­ве­че­ской груди. (7) Нико­му, кро­ме чело­ве­ка, недо­ступ­ны бла­го­ра­зу­мие, пред­у­смот­ри­тель­ность, усер­дие, рас­суди­тель­ность, зато живот­ные, обде­лен­ные чело­ве­че­ски­ми доб­ро­де­те­ля­ми, сво­бод­ны так­же и от поро­ков. Они совер­шен­но непо­хо­жи на чело­ве­ка ни внешне, ни внут­ренне: цар­ст­вен­ное и гла­вен­ст­ву­ю­щее начало5 в них устро­е­но по-дру­го­му. Так, у них есть голос, но неяс­ный, нечле­но­раздель­ный, неспо­соб­ный про­из­но­сить сло­ва; у них есть язык, но он мало­по­дви­жен и не может про­из­во­дить доста­точ­но раз­но­об­раз­ные зву­ки; и точ­но так же гла­вен­ст­ву­ю­ще­му нача­лу в них недо­ста­ет тон­ко­сти и чет­ко­сти. Оно вос­при­ни­ма­ет обра­зы вещей, кото­рые при­зы­ва­ют его к опре­де­лен­ным дей­ст­ви­ям, но обра­зы эти смут­ные и рас­плыв­ча­тые. (8) От это­го живот­ные, хоть и при­хо­дят порой в силь­ней­шее смя­те­ние или воз­буж­де­ние, все же не зна­ют стра­ха и тре­во­ги, печа­ли и гне­ва, но спо­соб­ны испы­ты­вать лишь нечто отда­лен­но их напо­ми­наю­щее. Вот отче­го их настро­е­ния быст­ро про­хо­дят, сме­ня­ясь про­ти­во­по­лож­ны­ми, и живот­ное, толь­ко что буше­вав­шее яро­стью или поте­ряв­шее голо­ву от стра­ха, вдруг начи­на­ет мир­но пастись, и после безум­ных воплей и скач­ков сра­зу ложит­ся и тихо засы­па­ет.

4. (1) Что такое гнев, мы объ­яс­ни­ли доста­точ­но подроб­но. Как он отли­ча­ет­ся от такой вещи, как гнев­ли­вость, я думаю, понят­но: как пья­ный отли­ча­ет­ся от пья­ни­цы, а испу­ган­ный от трус­ли­во­го. Раз­гне­ван­ный чело­век может быть не гнев­лив; а гнев­ли­вый быва­ет ино­гда не раз­гне­ван­ным.

(2) Гре­ки раз­ли­ча­ют мно­же­ство раз­но­вид­но­стей гне­ва, давая каж­дой свое имя6; я не ста­ну подроб­но оста­нав­ли­вать­ся на этом, посколь­ку в нашем язы­ке для этих видов нет осо­бых назва­ний. Впро­чем, и мы назы­ваем иной харак­тер кру­тым или рез­ким, а так­же гово­рим о людях раз­дра­жи­тель­ных, злоб­ных, беше­ных, крик­ли­вых, тяже­лых, колю­чих, --- все это обо­зна­че­ния раз­ных видов гне­ва; к ним нуж­но отне­сти и чело­ве­ка «нрав­но­го» --- самая мяг­кая раз­но­вид­ность гнев­ли­во­сти. (3) Быва­ет гнев, пол­но­стью выхо­дя­щий в кри­ке; быва­ют при­сту­пы гне­ва столь же упор­ные, сколь частые; быва­ет гнев ску­пой на сло­ва, зато сви­ре­пый на руку; быва­ет такой, что весь выли­ва­ет­ся пото­ка­ми горь­ких слов и про­кля­тий; быва­ет такой, что про­яв­ля­ет­ся вовне лишь холод­но­стью или упре­ком; быва­ет глу­бо­кий тяж­кий гнев, цели­ком обра­щен­ный внутрь. И еще тыся­ча дру­гих видов есть у это­го мно­го­ли­ко­го зла.

5. (1) Итак, мы выяс­ни­ли, что есть гнев, под­вер­же­но ли ему какое-либо живот­ное, кро­ме чело­ве­ка, чем он отли­ча­ет­ся от гнев­ли­во­сти и сколь­ко быва­ет его раз­но­вид­но­стей. Теперь попы­та­ем­ся выяс­нить, сооб­ра­зен ли гнев природе\footnote{«При­ро­да» у сто­и­ков --- цен­траль­ное поня­тие и физи­ки и эти­ки. Фор­му­ли­ров­ка пер­во­го и выс­ше­го мораль­но­го прин­ци­па: «Жить в согла­сии с при­ро­дой» при­пи­сы­ва­ет­ся само­му осно­ва­те­лю шко­лы Зено­ну и разде­ля­ет­ся все­ми без исклю­че­ния его после­до­ва­те­ля­ми (см. SVF I. 179. 552; Stob. Ecl. II. S. 75; Diog. Laert. 7, 87)} и поле­зен ли он, а если да, то не сле­ду­ет ли сохра­нять его в себе хотя бы отча­сти?

(2) Сооб­ра­зен ли гнев при­ро­де, станет нам ясно тот­час, как толь­ко мы посмот­рим, что такое чело­век. Что может быть мяг­че и лас­ко­вее чело­ве­ка, когда дух его настро­ен пра­виль­но? А гнев --- самая жесто­кая вещь на све­те. Какое суще­ство любит дру­гих боль­ше, чем чело­век? А гнев враж­де­бен ко всем на све­те. Чело­век рож­ден для вза­и­мо­по­мо­щи, гнев --- для вза­и­мо­ис­треб­ле­ния; чело­век стре­мит­ся к объ­еди­не­нию, гнев --- к разъ­еди­не­нию; чело­ве­ку свой­ст­вен­но при­но­сить поль­зу, а гне­ву --- вред; чело­век при­хо­дит на помощь даже незна­ко­мо­му, гнев напа­да­ет даже на самых близ­ких; чело­век готов пой­ти на издерж­ки, чтобы сде­лать дру­го­му при­ят­ное, гнев готов сам под­верг­нуть­ся опас­но­сти, лишь бы изве­сти дру­го­го. (3) Сле­до­ва­тель­но, вся­кий, кто станет при­пи­сы­вать луч­ше­му и совер­шен­ней­ше­му созда­нию при­ро­ды этот дикий и губи­тель­ный порок, тот изоб­ли­чит пол­ней­шее незна­ком­ство с при­ро­дой вещей. Мы уже гово­ри­ли о том, что гнев есть жаж­да мести; но тако­му жела­нию нет места в груди чело­ве­ка, от при­ро­ды он самое миро­лю­би­вое суще­ство на све­те. Чело­ве­че­ская жизнь дер­жит­ся бла­го­де­я­ни­я­ми и согла­си­ем, и не страх, а вза­им­ная любовь побуж­да­ет чело­ве­че­ство заклю­чать дого­во­ры об обще­ст­вен­ной вза­и­мо­по­мо­щи.

6. (1) «Так что же, выхо­дит, чело­век нико­гда не нуж­да­ет­ся в нака­за­нии?» --- Отче­го же, нуж­да­ет­ся, толь­ко нака­зы­вать нуж­но без гне­ва, с умом; нака­за­ние долж­но не вредить, а лечить под видом неко­то­ро­го вреда\footnote{Вопрос о вме­ни­мо­сти поступ­ков и тем самым о пра­во­мер­но­сти вся­ко­го вооб­ще поощ­ре­ния или нака­за­ния в систе­ме сто­и­ков труд­но­раз­ре­шим, ибо здесь их физи­ка про­ти­во­ре­чит их соб­ст­вен­ной эти­ке: стои­че­ское пони­ма­ние Бога и мира тре­бу­ет пол­но­го фата­лиз­ма и учит о необ­хо­ди­мой пред­опре­де­лен­но­сти все­го на све­те (как и вся­кий мате­ри­а­лизм), в то вре­мя как эти­ка тре­бу­ет имен­но доб­ро­воль­но­го, разум­но­го и не вынуж­ден­но­го необ­хо­ди­мо­стью само­опре­де­ле­ния и совер­шен­ст­во­ва­ния в доб­ро­де­те­ли. Для оппо­нен­та Сене­ки Ари­сто­те­ля боль­шин­ство чело­ве­че­ских поступ­ков доб­ро­воль­ны и вме­ни­мы, а пото­му вся­кое нака­за­ние есть месть (в част­но­сти, со сто­ро­ны государ­ства) за при­чи­нен­ную неспра­вед­ли­вость; Сене­ка же в этом вопро­се скло­ня­ет­ся к пол­ной невме­ня­е­мо­сти чело­ве­ка, душев­ный склад кото­ро­го и все поступ­ки опре­де­ле­ны изна­чаль­но; отсюда тео­рия нака­за­ния как лече­ния. Харак­тер­ны два част­ных рас­хож­де­ния меж­ду Ари­сто­те­лем и Сене­кой в вопро­се о нака­за­нии: для Ари­сто­те­ля невме­ня­е­мы лишь сума­сшед­шие, для Сене­ки --- прак­ти­че­ски все, ибо все под­вер­же­ны чув­ству, а сле­до­ва­тель­но, не в сво­ем уме. Для Сене­ки само­убий­ство --- послед­няя гаран­тия сво­бо­ды, чести, досто­ин­ства, сча­стья, вели­кий дар богов; для Ари­сто­те­ля --- уго­лов­но нака­зу­е­мое пре­ступ­ле­ние про­тив государ­ства (Ари­сто­тель. Нико­ма­хо­ва эти­ка. 1138a 10—15)}. Искрив­лен­ное древ­ко копья мы обжи­га­ем на огне и заби­ва­ем кли­нья­ми в тис­ки не для того, чтобы сло­мать или сжечь, но чтобы выпря­мить. Точ­но так же мы выправ­ля­ем иско­вер­кан­ные поро­ком нра­вы, при­чи­няя телес­ную или душев­ную боль. (2) Имен­но так посту­па­ет врач: при лег­ком заболе­ва­нии он преж­де все­го пыта­ет­ся немно­го изме­нить повсе­днев­ные при­выч­ки боль­но­го, назна­ча­ет порядок еды, питья, заня­тий, чтобы укре­пить здо­ро­вье таким изме­не­ни­ем жиз­нен­но­го рас­по­ряд­ка. Как пра­ви­ло, помо­га­ет уже одно изме­не­ние обра­за жиз­ни; если же не помо­га­ет, врач дела­ет рас­по­рядок стро­же, кое-что пол­но­стью исклю­чая из него; если и это не при­но­сит поль­зы, вовсе отме­ня­ет пита­ние и изну­ря­ет тело голо­дом; если все эти более мяг­кие меры ока­жут­ся напрас­ны­ми, врач откры­ва­ет вену и пус­ка­ет кровь; нако­нец, если какие-то чле­ны, оста­ва­ясь соеди­нен­ны­ми с телом, нано­сят ему вред, рас­про­стра­няя болезнь, врач нала­га­ет на них руку; ника­кое лече­ние не может счи­тать­ся жесто­ким, если его резуль­тат --- выздо­ров­ле­ние.

(3) Так и тому, кто сто­ит на стра­же зако­нов и управ­ля­ет обще­ст­вом, подо­ба­ет направ­лять под­опеч­ные души лишь сло­вом, да и то выби­рая сло­ва помяг­че: он дол­жен сове­то­вать, что сле­ду­ет делать, вну­шать стрем­ле­ние к чест­но­му и спра­вед­ли­во­му, нена­висть к поро­кам, ува­же­ние к доб­ро­де­те­лям, затем пусть пере­хо­дит к речам более суро­вым, к выго­во­рам и пред­у­преж­де­ни­ям; и толь­ко в послед­нюю оче­редь к нака­за­ни­ям; и то, выби­рая вна­ча­ле лег­кие, не без­воз­врат­но губя­щие; выс­шую меру нака­за­ния пусть он назна­ча­ет лишь за выс­шее пре­ступ­ле­ние, дабы на гибель осуж­да­лись лишь те, чья гибель была бы в инте­ре­сах всех, вклю­чая само­го поги­баю­ще­го. (4) От вра­ча он будет отли­чать­ся лишь тем, что врач по воз­мож­но­сти облег­ча­ет уход из жиз­ни для тех, кого уже не может спа­сти; а управ­ля­ю­щий государ­ст­вом выго­ня­ет осуж­ден­ных из жиз­ни с позо­ром, выстав­ляя их на пуб­лич­ное пору­га­ние: не отто­го, конеч­но, что нака­за­ние достав­ля­ет ему удо­воль­ст­вие --- муд­рый далек от такой бес­че­ло­веч­ной жесто­ко­сти, --- а для того, чтобы они послу­жи­ли пре­до­сте­ре­же­ни­ем для всех про­чих, и раз уж не захо­те­ли при­но­сить поль­зу при жиз­ни, чтобы послу­жи­ли государ­ству хотя бы сво­ей смер­тью. В чело­ве­че­ской при­ро­де нет стрем­ле­ния кого-то нака­зы­вать; поэто­му и гнев, все­гда стре­мя­щий­ся нака­зать кого-то, не сооб­ра­зен чело­ве­че­ской при­ро­де.

(5) При­ве­ду еще один аргу­мент, поза­им­ст­во­ван­ный мной у Пла­то­на (я не вижу ниче­го дур­но­го в том, чтобы при­во­дить чужие мыс­ли; если я их разде­ляю, то они --- и мои тоже): «Доб­рый чело­век не при­чи­ня­ет зла»9. Но нака­за­ние есть при­чи­не­ние зла; сле­до­ва­тель­но, доб­ро­му чело­ве­ку не подо­ба­ет нака­зы­вать, а зна­чит, не подо­ба­ет и гне­вать­ся, посколь­ку гнев свя­зан с нака­за­ни­ем. Если хоро­ший чело­век не раду­ет­ся чужо­му нака­за­нию, он не станет радо­вать­ся и тому чув­ству, для кото­ро­го нака­за­ние --- выс­шее удо­воль­ст­вие; сле­до­ва­тель­но, гнев не есте­стве­нен для чело­ве­ка.

7. (1) Но может быть, хоть гнев и не при­рож­ден, нам сле­до­ва­ло бы взять его на воору­же­ние, посколь­ку он часто быва­ет поле­зен? Он под­ни­ма­ет дух и воз­буж­да­ет его; без гне­ва самый муже­ст­вен­ный чело­век не совер­шил бы на войне подви­гов: это пла­мя необ­хо­ди­мо, чтобы подо­гре­вать муже­ство и посы­лать храб­ре­цов в самую гущу опас­но­сти. Вот отче­го неко­то­рые пола­га­ют, что луч­ше все­го --- вве­сти гнев в извест­ные гра­ни­цы, но не уни­что­жать его совсем; нуж­но избав­лять­ся от излиш­ков гне­ва, пере­ли­ваю­щих через край, оста­вив уме­рен­ное коли­че­ство, ров­но столь­ко, сколь­ко нуж­но для бла­готвор­но­го воздей­ст­вия на чело­ве­ка, чтобы не осла­бе­ла его дея­тель­ная реши­мость, не увя­ли силы и жиз­нен­ная энергия10.

(2) Но, во-пер­вых, от вся­кой пагуб­ной стра­сти лег­че все­го совсем изба­вить­ся, чем научить­ся ею управ­лять; лег­че не допу­стить ее, чем, допу­стив, вве­сти в рам­ки уме­рен­но­сти. Ибо, едва всту­пив в пра­ва вла­де­ния, стра­сти сра­зу ста­но­вят­ся могу­ще­ст­вен­нее, чем их пред­по­ла­гае­мый пра­ви­тель --- разум, и не поз­во­лят ни потес­нить себя, ни ума­лить. (3) Во-вто­рых, сам разум, кото­ро­му мы вру­ча­ем вож­жи, сохра­ня­ет свою власть лишь до тех пор, пока оста­ет­ся вда­ли от чувств; сто­ит ему сопри­кос­нуть­ся с ними и впи­тать в себя часть сквер­ны, и он уже не в состо­я­нии сдер­жи­вать те самые чув­ства, от кото­рых без труда мог дер­жать­ся подаль­ше. Одна­жды взвол­но­ван­ная и потря­сен­ная душа ста­но­вит­ся рабы­ней того, что нару­ши­ло ее покой. (4) Есть вещи, нача­ло кото­рых --- в нашей вла­сти, но затем они захва­ты­ва­ют нас силой и не остав­ля­ют нам пути назад. Так быва­ет с тела­ми, летя­щи­ми вниз, в про­пасть; еди­но­жды сорвав­шись, они уже не власт­ны выби­рать доро­гу, замед­лить свое дви­же­ние или оста­но­вить­ся; стре­ми­тель­ное и необ­ра­ти­мое паде­ние слов­но отру­ба­ет у них вся­кую реши­мость или рас­ка­я­ние, и им уже нель­зя не достичь кон­ца пути, кото­ро­го мож­но было не начи­нать. То же самое про­ис­хо­дит и с нашей душой, когда она бро­сит­ся очер­тя голо­ву в гнев, любовь или дру­гое подоб­ное чув­ство: она уже не в состо­я­нии оста­но­вить­ся; соб­ст­вен­ный вес увле­ка­ет ее вниз, и тянет на дно все­гда устрем­лен­ная кни­зу при­ро­да поро­ка.

8. (1) Самое луч­шее --- тот­час про­гнать толь­ко воз­ни­каю­щее раз­дра­же­ние, пода­вить гнев в самом заро­ды­ше и все­гда вни­ма­тель­но следить за собой, как бы не вспы­лить. Ибо если гне­ву удаст­ся схва­тить нас и пота­щить в дру­гую сто­ро­ну, нам труд­но будет вер­нуть­ся назад, в здра­вое состо­я­ние, посколь­ку там, куда одна­жды полу­чи­ло доступ чув­ство, не оста­ет­ся ни следа разу­ма; если наша воля пре­до­ста­вит чув­ству хоть малей­шие пра­ва, оно сде­ла­ет со всем про­чим, что есть в нас, то, что захо­чет, а не то, что мы ему поз­во­лим. (2) Повто­ряю: вра­га надо отра­жать, как толь­ко он пере­шел гра­ни­цу; ибо когда он уже вошел в город­ские ворота, он едва ли станет выслу­ши­вать усло­вия сво­их плен­ни­ков. Дело в том, что дух наш не может бес­при­страст­но наблюдать за чув­ст­вом со сто­ро­ны, не поз­во­ляя ему захо­дить даль­ше, чем сле­ду­ет; он сам пре­вра­ща­ет­ся в чув­ство и, пре­дан­ный нами, ослаб­лен­ный, уже не в состо­я­нии вер­нуть себе былую силу, столь полез­ную и бла­готвор­ную. (3) Чув­ство и разум, как я уже гово­рил, не поме­ща­ют­ся в нас отдель­но, неза­ви­си­мо друг от дру­га, но явля­ют­ся изме­не­ни­я­ми духа к худ­ше­му или лучшему11. Как может вос­пря­нуть разум, под­дав­ший­ся гне­ву, если он захва­чен и подав­лен поро­ка­ми? Как может он высво­бо­дить­ся из бес­по­рядоч­но­го сме­ше­ния, если в этой сме­си пре­об­ла­да­ет худ­шее?

(4) «Одна­ко быва­ют люди, --- воз­ра­зят мне, --- сдер­жан­ные в гне­ве». Насколь­ко сдер­жан­ные? Настоль­ко, что не дела­ют ниче­го, что дик­ту­ет им гнев, или дела­ют лишь кое-что? Если ниче­го, то, зна­чит, гнев совсем не так необ­хо­дим для реши­тель­ных дей­ст­вий, как вы пола­га­ли, при­зы­вая его на помощь разу­му, у кото­ро­го буд­то бы нет такой мощи, как у гне­ва. (5) Нако­нец, я постав­лю вопрос так: гнев силь­нее разу­ма или сла­бее? Если силь­нее, то как же смог бы разум его обузды­вать, ведь пови­ну­ет­ся обыч­но толь­ко сла­бей­шее? Если сла­бее, то разум и без него спра­вит­ся с любым делом, не нуж­да­ясь в помо­щи более сла­бо­го. (6) «Но есть ведь люди, кото­рые и в гне­ве оста­ют­ся вер­ны себе и не теря­ют само­об­ла­да­ния». Да, но когда? Когда гнев начи­на­ет ути­хать и сам по себе уже про­хо­дит; а не тогда, когда он рас­па­лен до бело­го кале­ния --- тут он силь­нее их. (7) «Так что же, ты не согла­сишь­ся, что ино­гда и в пылу гне­ва отпус­ка­ют нена­вист­ных людей целы­ми и невреди­мы­ми, воз­дер­жи­ва­ясь от мести?» --- Быва­ет. Но в каких слу­ча­ях? Когда одно чув­ство одер­жит верх над дру­гим, и страх или жад­ность одо­ле­ют гнев. Это не покой, уста­нав­ли­ваю­щий­ся под бла­го­де­тель­ным воздей­ст­ви­ем разу­ма, а дур­ное и нена­деж­ное пере­ми­рие меж­ду чув­ства­ми.

9. (1) Гнев не при­но­сит поль­зы и не побуж­да­ет дух к воин­ским подви­гам. Ибо само­до­вле­ю­щая доб­ро­де­тель не нуж­да­ет­ся в помо­щи поро­ка. Когда необ­хо­ди­мо реши­тель­но дей­ст­во­вать, она сама вос­прянет, не рас­па­ля­ясь гне­вом, и будет уси­ли­вать или ослаб­лять свое напря­же­ние ров­но настоль­ко, насколь­ко сочтет нуж­ным, как мета­тель­ные сна­ряды выле­та­ют из ката­пуль­ты с такой силой, какая жела­тель­на выпус­каю­ще­му их. (2) Ари­сто­тель гово­рит: «Гнев необ­хо­дим, ибо без него нель­зя выиг­рать ни одно­го боя; для воен­но­го успе­ха нуж­но, чтобы он напол­нил душу и вос­пла­ме­нил дух. Сле­ду­ет, одна­ко, употреб­лять его не в каче­стве пол­ко­во­д­ца, а как про­сто­го солдата»12. Это невер­но. Если гнев внем­лет разу­му и сле­ду­ет туда, куда его ведут, то это уже не гнев, чье отли­чи­тель­ное свой­ство --- упрям­ство. Если же он сопро­тив­ля­ет­ся разу­му, не ути­хая, когда при­ка­зы­ва­ют, но про­дол­жая буй­ст­во­вать по соб­ст­вен­но­му про­из­во­лу, тогда он такой же негод­ный помощ­ник душе, как сол­дат, не испол­ня­ю­щий при­ка­за к отступ­ле­нию. (3) Итак, если он поз­во­ля­ет заклю­чить себя в извест­ные гра­ни­цы, то это уже не гнев, и надо назы­вать его каким-либо дру­гим име­нем, ибо под гне­вом я пони­маю чув­ство необуздан­ное и неуправ­ля­е­мое. Если же не поз­во­ля­ет, то он пагу­бен для нас и не может счи­тать­ся ни союз­ни­ком, ни помощ­ни­ком. Итак, он либо не гнев, либо бес­по­ле­зен. Ибо если чело­век тре­бу­ет воз­мездия не ради воз­мездия, а пото­му, что так долж­но, его не сле­ду­ет счи­тать раз­гне­ван­ным. Полез­ный воин --- тот, кто уме­ет пови­но­вать­ся при­ка­зу, а любое чув­ство --- столь же пло­хой испол­ни­тель, сколь и рас­по­ряди­тель.

10. (1) Вот поче­му разум нико­гда не возь­мет себе в помощ­ни­ки непред­ска­зу­е­мые и неисто­вые побуж­де­ния, для кото­рых он сам ров­но ниче­го не зна­чит и не может усми­рить их ина­че, как столк­нув с дру­ги­ми таки­ми же: гнев со стра­хом, лень с гне­вом, робость с жад­но­стью. (2) Да мину­ет доб­ро­де­тель такое несча­стье, чтобы разу­му при­шлось при­бе­гать к поро­кам. Такой дух нико­гда не обре­та­ет надеж­но­го покоя, в нем все­гда будет буше­вать буря и вспы­хи­вать мятеж, если он при­вык пола­гать­ся лишь на соб­ст­вен­ные недо­стат­ки, если он не может быть муже­ст­вен без гне­ва, дея­те­лен без жад­но­сти, спо­ко­ен без стра­ха: подоб­ная тира­ния ожи­да­ет вся­ко­го, кто отдаст себя в доб­ро­воль­ное раб­ство како­му-нибудь чув­ству. Неуже­ли вам не стыд­но делать доб­ро­де­те­ли кли­ен­та­ми поро­ков? (3) Если разум ни на что не спо­со­бен без чув­ства, он в кон­це кон­цов утра­тит и свою разум­ность, во всем срав­няв­шись с чув­ства­ми и упо­до­бив­шись им. Какая раз­ни­ца меж­ду чув­ст­вом и разу­мом, если разум без чув­ства без­де­я­те­лен, а чув­ство без разу­ма опро­мет­чи­во? Выхо­дит, они рав­ны, если одно без дру­го­го суще­ст­во­вать не может. Но кто же согла­сит­ся при­рав­нять чув­ство к разу­му? (4) Нам воз­ра­жа­ют, что «чув­ство полез­но, если оно уме­рен­но». Толь­ко в том слу­чае, если оно полез­но по при­ро­де. Но если оно не тер­пит над собой вла­сти разу­ма, то эта его полез­ная уме­рен­ность может выра­жать­ся лишь одним пра­ви­лом: чем мень­ше его будет, тем мень­ше от него будет вреда. Ины­ми сло­ва­ми, уме­рен­ное чув­ство есть не что иное, как уме­рен­ное зло.

11. (1) «Но в схват­ке с вра­гом гнев необ­хо­дим». --- Здесь менее, чем где бы то ни было. Тут осо­бен­но важ­но, чтобы напор и воз­буж­де­ние не выры­ва­лись как попа­ло, а были точ­но рас­счи­та­ны и пови­но­ва­лись нам. Поче­му нам усту­па­ют в бою вар­ва­ры, пре­вос­хо­дя­щие нас и телес­ной силой и вынос­ли­во­стью? Их злей­ший враг --- их соб­ст­вен­ный гнев. Так же и гла­ди­а­то­ров спа­са­ет искус­ство, обез­ору­жи­ва­ет гнев. (2) Кро­ме того, зачем нужен гнев, если разум может сде­лать то же самое? Или ты пола­га­ешь, что охот­ник гне­ва­ет­ся на зве­ря? Но тем не менее он сме­ло встре­ча­ет напа­даю­ще­го и пре­сле­ду­ет убе­гаю­ще­го хищ­ни­ка: все это разум дела­ет без гне­ва. Кто оста­но­вил бес­счет­ные тыся­чи ким­вров и тев­то­нов, пото­ка­ми лив­ших­ся через Аль­пы, да так оста­но­вил, что и вест­ни­ка не оста­лось сооб­щить домой об ужас­ном истреблении\footnote{Ким­вры и тев­то­ны — глав­ные сре­ди вар­вар­ских пле­мен, кото­рые мигри­ро­ва­ли в кон­це II в. до н. э. (113—101 гг.) из Север­ной Гер­ма­нии и нанес­ли мно­го­крат­ные пора­же­ния рим­ля­нам. В 102 г. Марий раз­бил тев­то­нов при Секс­ти­е­вых Водах, а на сле­дую­щий год он вме­сте с Кату­лом уни­что­жил и ким­вров на Раудин­ской рав­нине.}, если не гнев, заме­нив­ший им доб­ро­де­тель? Быва­ет, что гнев кру­шит и низ­вер­га­ет все на сво­ем пути, одна­ко еще чаще он губит сам себя. (3) Кто пре­взой­дет гер­ман­цев в храб­ро­сти? А в стре­ми­тель­но­сти ата­ки? А в люб­ви к ору­жию? Они ведь и рож­да­ют­ся и вскарм­ли­ва­ют­ся ради того, чтобы было кому носить ору­жие, оно у них --- един­ст­вен­ная забота, все про­чее их не вол­ну­ет. А кто пре­взой­дет их в вынос­ли­во­сти и тер­пе­нии? Ведь у боль­шин­ства из них нечем даже при­крыть тело и нет убе­жи­ща, кото­рое мог­ло бы защи­тить от веч­ных моро­зов их стра­ны. (4) И этим-то мужам нано­сят пора­же­ние испан­цы с гал­ла­ми и ази­а­ты с сирий­ца­ми, вои­ны изне­жен­ные и никудыш­ные, еще до того, как рим­ский леги­он пока­зал­ся ввиду сра­же­ния, и виной тому --- исклю­чи­тель­но гнев­ли­вость. А попро­буй-ка толь­ко научи дис­ци­плине эти тела, дай разум этим душам, не ведаю­щим пока ни наслаж­де­ний, ни рос­ко­ши, ни богат­ства, что будет? В луч­шем слу­чае, нам при­дет­ся немед­лен­но воз­вра­щать­ся к древним рим­ским нра­вам, а в худ­шем --- не буду и зага­ды­вать.

(5) Чем иным сумел Фабий\footnote{Квинт Фабий Мак­сим Кунк­та­тор (умер в 203 г. до н. э.) — герой Вто­рой Пуни­че­ской вой­ны, мед­лив­ший начать реши­тель­ные дей­ст­вия и тем обес­си­лив­ший Ган­ни­ба­ла.} вос­ста­но­вить слом­лен­ные силы импе­рии, как не мед­ли­тель­но­стью, бес­ко­неч­ны­ми про­во­лоч­ка­ми и выжида­ни­ем; а ведь имен­но на это реши­тель­но неспо­соб­ны раз­гне­ван­ные. Импе­рия сто­я­ла тогда на краю гибе­ли и погиб­ла бы несо­мнен­но, если бы Фабий решил­ся дей­ст­во­вать так, как под­ска­зы­вал ему гнев. Но ему пред­сто­я­ло решать общую судь­бу государ­ства, и, взве­сив все силы, из кото­рых нель­зя было пожерт­во­вать ни одной, не рискуя поте­рять все, он отло­жил в сто­ро­ну и свою боль, и жаж­ду мще­ния, цели­ком сосре­дото­чив­шись на сооб­ра­же­ни­ях поль­зы и выжидая удоб­но­го слу­чая; таким обра­зом, он сна­ча­ла одер­жал победу над гне­вом, а потом --- над Ган­ни­ба­лом. (6) А Сци­пи­он? Раз­ве не каза­лось, что он вовсе забыл и про Ган­ни­ба­ла, и про пуний­ское вой­ско, на кото­рых ему сле­до­ва­ло бы гне­вать­ся, и настоль­ко не торо­пил­ся пере­но­сить вой­ну в Афри­ку, что дал сво­им недоб­ро­же­ла­те­лям повод объ­явить себя погряз­шим в рос­ко­ши лен­тя­ем?\footnote{Пуб­лий Кор­не­лий Сци­пи­он Афри­кан­ский (235—183 гг. до н. э.) — победи­тель Ган­ни­ба­ла в бит­ве при Заме, мед­лив­ший в Сици­лии зимой 205—204 г., гото­вя экс­пе­ди­цию в Афри­ку.} (7) А вто­рой Сци­пи­он? Раз­ве не сидел он дол­го-дол­го под Нуман­ци­ей, спо­кой­но пере­но­ся ту мучи­тель­но обид­ную и для него, и для наро­да мысль, что Нуман­ция не сда­ет­ся доль­ше, чем Кар­фа­ген?\footnote{Пуб­лий Кор­не­лий Сци­пи­он Эми­ли­ан Афри­кан­ский Нуман­тин­ский (185—129 гг. до н. э.) — раз­ру­ши­тель Кар­фа­ге­на (146 г. до н. э.). Во вре­мя сво­его вто­ро­го кон­суль­ства коман­до­вал вой­ска­ми в Испа­нии, где девять меся­цев оса­ждал город Нуман­цию (134—133 гг. до н. э.).} Он обнес город сплош­ным валом и про­дол­жал оса­ду, пока не довел вра­гов до того, что они ста­ли гиб­нуть от соб­ст­вен­ных мечей. (8) Так что гнев не при­но­сит поль­зы ни в бою, ни вооб­ще на войне: он слиш­ком скло­нен дей­ст­во­вать опро­мет­чи­во и, желая под­верг­нуть опас­но­сти про­тив­ни­ка, не осте­ре­га­ет­ся ее сам. Куда надеж­нее доб­ро­де­тель, кото­рая сна­ча­ла дол­го осмат­ри­ва­ет­ся, выби­ра­ет нуж­ное направ­ле­ние и, все обду­мав, нето­роп­ли­во начи­на­ет дей­ст­во­вать стро­го по пла­ну.

12. (1) «Так что же, --- ска­жут нам, --- доб­рый чело­век не раз­гне­ва­ет­ся, если на его гла­зах ста­нут уби­вать его отца или наси­ло­вать мать?»\footnote{Ср.: Ари­сто­тель. Нико­ма­хо­ва эти­ка. 1126a 5: «Чело­век... недо­ступ­ный гне­ву, не спо­со­бен защи­щать­ся, а меж­ду тем сно­сить уни­же­ния само­му и допус­кать, чтобы уни­жа­ли близ­ких, низ­ко».} --- Нет, не раз­гне­ва­ет­ся, но будет защи­щать их и пока­ра­ет обид­чи­ков. Или вы бои­тесь, что сынов­няя почти­тель­ность ока­жет­ся для него недо­ста­точ­но силь­ным побуж­де­ни­ем, если не будет под­креп­ле­на гне­вом? Вы еще спро­си­те так: «Неуже­ли, видя, как режут его отца или сына, доб­рый чело­век не запла­чет, не упа­дет в обмо­рок?» Мы видим, что так обыч­но ведут себя жен­щи­ны, теря­ю­щие при­сут­ст­вие духа при малей­шем упо­ми­на­нии об опас­но­сти. (2) Доб­рый муж будет испол­нять свой долг без смя­те­ния и стра­ха; не совер­шив ниче­го недо­стой­но­го муж­чи­ны, он тем самым совер­шит достой­ное доб­ро­го мужа. Будут уби­вать отца --- я ста­ну защи­щать его; убьют --- ста­ну мстить, но не пото­му, что мне боль­но, а пото­му, что так долж­но.

(3) «Доб­рые люди гне­ва­ют­ся, когда оби­жа­ют их близ­ких». Эти твои сло­ва, Тео­фраст, отвра­ща­ют людей от более муже­ст­вен­ных настав­ле­ний: ты взы­ва­ешь не к судье, а к зри­те­лям. Ведь в подоб­ном слу­чае вся­кий чело­век ощу­ща­ет гнев за сво­их близ­ких; а все люди охот­но при­зна­ют долж­ным то, что они и так дела­ют. Почти каж­дый счи­та­ет спра­вед­ли­вым то чув­ство, кото­рое сам разде­ля­ет. (4) Но ведь то же чув­ство они испы­ты­ва­ют, когда нет горя­чей воды, когда разо­бьет­ся стек­лян­ный сосуд или баш­мак забрыз­га­ет­ся гря­зью. Этот гнев вызы­ва­ет­ся не пре­дан­ной любо­вью, как у детей, кото­рые оди­на­ко­во горь­ко пла­чут, поте­ряв игруш­ку и лишив­шись роди­те­лей. (5) Гне­вать­ся за сво­их --- свой­ство не пре­дан­ной души, а сла­бой. Пре­крас­но и достой­но высту­пить на защи­ту роди­те­лей, детей, дру­зей, сограж­дан, руко­вод­ст­ву­ясь самим дол­гом, дей­ст­вуя доб­ро­воль­но, рас­суди­тель­но, пред­у­смот­ри­тель­но, а не под воздей­ст­ви­ем вне­зап­но­го беше­но­го побуж­де­ния. Ни одно чув­ство не воз­буж­да­ет в нас такой жаж­ды мести, как гнев; но имен­но поэто­му он мало­при­го­ден для осу­щест­вле­ния мести. Он черес­чур пылок, тороп­лив, безу­мен; и как вся­кое почти вожде­ле­ние, сам меша­ет осу­щест­вле­нию того, к чему стре­мит­ся. Итак, гнев нико­гда не послу­жил ни к чему хоро­ше­му ни в мир­ное вре­мя, ни в войне. Ибо мир он пре­вра­ща­ет в подо­бие вой­ны, а на войне забы­ва­ет, что Марс бес­при­стра­стен, и стре­мит­ся под­чи­нить себе дру­гих, хотя сам собой не вла­де­ет.

(6) Кро­ме того, не сле­ду­ет ожи­дать от поро­ков поль­зы и при­ни­мать их на служ­бу толь­ко пото­му, что когда-то в чем-то они ока­за­лись полез­ны­ми: ведь и лихо­рад­ка при­но­сит облег­че­ние при неко­то­рых болез­нях, но тем не менее луч­ше все­го было бы обой­тись и без болез­ней, и без лихо­рад­ки. Выздо­ро­веть бла­го­да­ря дру­гой болез­ни --- отвра­ти­тель­ный вид исцеления\footnote{Сло­во pathos обо­зна­ча­ет и пас­сив­ность, и состо­я­ние, и страсть, и чув­ство, и болезнь. Сто­и­ки посто­ян­но под­чер­ки­ва­ли, что страст­ное чув­ство --- это имен­но болезнь, так как, во-пер­вых, неесте­ствен­но, а во-вто­рых, есть несо­мнен­ное зло для чело­ве­че­ской души.}. При­мер­но так же обсто­ит дело и с гне­вом: пусть ино­гда он и ока­зы­вал­ся неожи­дан­но спа­си­тель­ным, как яд, кораб­ле­кру­ше­ние или паде­ние с высоты, одна­ко это не осно­ва­ние, чтобы счи­тать его бла­готвор­ным; неред­ко ведь быва­ло, что смер­то­нос­ная вещь слу­жи­ла во бла­го.

13. (1) Далее, если что-то сто­ит того, чтобы им обла­дать, то чем боль­ше его у нас будет, тем лучше\footnote{Пря­мо про­тив ари­сто­телев­ско­го уче­ния о доб­ро­де­те­ли как середине ср.: Ари­сто­тель. Нико­ма­хо­ва эти­ка. 1107a 7.}. Если спра­вед­ли­вость --- бла­го, то никто не ска­жет, что она станет луч­ше отто­го, что немно­го поуба­вит­ся. Если муже­ство --- бла­го, то никто не поже­ла­ет, чтобы его у него ста­ло мень­ше. (2) В таком слу­чае и гне­ва, чем боль­ше, тем луч­ше: кто же станет отка­зы­вать­ся от при­бав­ки како­го-либо бла­га? Одна­ко в боль­шом коли­че­стве он непо­ле­зен; сле­до­ва­тель­но, и в любом непо­ле­зен. Бла­го, кото­рое, уве­ли­чи­ва­ясь, ста­но­вит­ся злом, не бла­го.

(3) Нам гово­рят: «Гнев поле­зен, ибо дела­ет людей воин­ст­вен­нее». В таком слу­чае, полез­но и опья­не­ние, ибо дела­ет людей зади­ри­сты­ми и дерз­ки­ми; нетрез­вый чело­век гораздо ско­рее хва­та­ет­ся за меч. Ска­жи уж тогда, что и умо­по­ме­ша­тель­ство необ­хо­ди­мо для при­да­ния сил: извест­но ведь, что в при­сту­пе бешен­ства безум­цы часто ста­но­вят­ся намно­го силь­нее. (4) А раз­ве не быва­ло тако­го, чтобы страх делал чело­ве­ка храбрее? Часто послед­ние тру­сы под стра­хом смер­ти сами кида­лись в бит­ву. Но гнев, опья­не­ние, страх и про­чие столь же отвра­ти­тель­ные и столь же мимо­лет­ные раз­дра­же­ния не спо­соб­ст­ву­ют укреп­ле­нию доб­ро­де­те­ли, ибо она не нуж­да­ет­ся в поро­ках; они могут лишь нена­дол­го воз­будить обыч­ный кос­ный и трус­ли­вый дух. (5) Гнев дела­ет муже­ст­вен­нее лишь того, кто без гне­ва вооб­ще не знал, что такое муже­ство. Таким обра­зом, он слу­жит не под­креп­ле­ни­ем доб­ро­де­те­ли, а ее заме­ной. Если бы гнев был бла­гом, раз­ве не стал бы он отли­чи­тель­ным свой­ст­вом самых совер­шен­ных сре­ди людей? Но кто же в дей­ст­ви­тель­но­сти лег­че все­го под­да­ет­ся гне­ву? Мла­ден­цы, ста­ри­ки и боль­ные, и вооб­ще все сла­бые и нездо­ро­вые отли­ча­ют­ся раз­дра­жи­тель­но­стью.

14. (1) «Доб­рый чело­век, --- гово­рит Тео­фраст, --- не может не гне­вать­ся на злых». Выхо­дит, чем чело­век луч­ше, тем гнев­ли­вее. Ниче­го подоб­но­го, все наобо­рот: чем он луч­ше, тем миро­лю­би­вее, тем сво­бод­нее от вла­сти любых чувств и ни к кому не пита­ет нена­ви­сти. (2) В самом деле, за что ему нена­видеть тех, кто посту­па­ет дур­но? Ведь на подоб­ные поступ­ки тол­ка­ет их заблуж­де­ние. А бла­го­ра­зум­ный чело­век не станет нена­видеть заблуж­даю­ще­го­ся: в про­тив­ном слу­чае ему при­дет­ся когда-нибудь воз­не­на­видеть само­го себя. Он поду­ма­ет о том, что и сам во мно­гом гре­шит про­тив доб­рых нра­вов; что мно­гие его поступ­ки нуж­да­ют­ся в про­ще­нии, и станет гне­вать­ся на себя. Ведь спра­вед­ли­вый судья не может по-раз­но­му под­хо­дить к реше­нию сво­его и чужо­го дела. (3) А я гово­рю вам, что не най­дет­ся тако­го чело­ве­ка, кото­рый мог бы отпу­стить себе все; а если кто-то заяв­ля­ет, что он неви­нен, зна­чит, он име­ет в виду не соб­ст­вен­ную совесть, а свиде­те­лей, ибо мож­но согре­шить и без свиде­те­лей. Насколь­ко чело­веч­нее отне­стись к греш­ни­кам мяг­ко, с оте­че­ской душой, и не пре­сле­до­вать их, но пытать­ся вер­нуть назад! Ведь если чело­век, не зная доро­ги, заблудит­ся сре­ди вспа­хан­но­го поля, луч­ше выве­сти его на пра­виль­ный путь, чем выго­нять с поля пал­кой.

15. (1) Итак, согре­шаю­ще­го нуж­но исправ­лять: уве­ща­ни­ем и силой, мяг­ко и суро­во; как ради него само­го, так и ради дру­гих делать его луч­ше; тут не обой­тись без нака­за­ния, но гнев недо­пу­стим. Ибо кто же гне­ва­ет­ся на того, кого лечит?

Вы ска­же­те, что они неис­пра­ви­мы, что в них не оста­лось ниче­го мяг­ко­го, подат­ли­во­го, спо­соб­но­го вну­шить доб­рую надеж­ду. Ну что же, от тех, кто и впрямь не может не пор­тить все, с чем сопри­ка­са­ет­ся, сле­ду­ет изба­вить чело­ве­че­ство; пусть пере­ста­нут быть дур­ны­ми един­ст­вен­но доступ­ным им обра­зом, раз на дру­гое они не спо­соб­ны; одна­ко нена­ви­сти в этом быть не долж­но. В самом деле, за что мне его нена­видеть? (2) Напро­тив, избав­ляя его от него само­го, я при­но­шу ему самую боль­шую поль­зу, какую могу. Раз­ве, давая отре­зать себе ногу, чело­век ее нена­видит? Это не нена­висть, а пол­ная жало­сти забота и тяже­лое лече­ние. Мы уби­ва­ем беше­ных собак; зака­лы­ва­ем неукро­ти­мо дико­го быка; пус­ка­ем под нож боль­ных овец, чтобы они не пере­за­ра­зи­ли все ста­до; мы уни­что­жа­ем вся­кий неесте­ствен­ный урод­ли­вый при­плод; даже детей, если они рож­да­ют­ся сла­бы­ми и ненор­маль­ны­ми, мы топим. Но это не гнев, а разум­ный рас­чет: отде­лить вредо­нос­ное от здо­ро­во­го. (3) Тому, кто рас­по­ря­жа­ет­ся нака­за­ни­я­ми, менее все­го на све­те при­ли­чен гнев, ибо нака­за­ние лишь тогда может быть бла­готвор­но и цели­тель­но, когда назна­ча­ет­ся реше­ни­ем суда. Вот поче­му Сократ одна­жды ска­зал сво­е­му рабу: «Я побил бы тебя, если бы не был сер­дит». Он отло­жил нака­за­ние раба до более спо­кой­но­го вре­ме­ни, а в тот момент счел нуж­ным занять­ся соб­ст­вен­ным исправ­ле­ни­ем. Но если даже Сократ не решал­ся под­дать­ся гне­ву, то у кого же он будет дер­жать­ся в рам­ках уме­рен­но­сти?

16. (1) Итак, для обузда­ния заблуд­ших и нака­за­ния пре­ступ­ни­ков нет нуж­ды в гне­ве нака­зу­ю­ще­го; ибо если гнев есть душев­ный порок, то не сле­ду­ет, чтобы греш­ный карал чужие пре­гре­ше­ния. «Выхо­дит, я не дол­жен сер­дить­ся на раз­бой­ни­ка? Выхо­дит, я не впра­ве гне­вать­ся на отра­ви­те­ля?» --- Нет; я ведь не ста­ну гне­вать­ся на себя, вскры­вая себе вену\footnote{О само­убий­стве как акте гне­ва см.: Ари­сто­тель. Нико­ма­хо­ва эти­ка. 1138a 10—13.}. Вся­кое нака­за­ние я при­ме­няю в каче­стве лекар­ства. (2) Если, напри­мер, ты толь­ко еще всту­пил на путь заблуж­де­ний, и тебе еще не слу­чи­лось тяж­ко упасть, хотя споты­ка­ешь­ся ты часто, попы­та­юсь испра­вить тебя пори­ца­ни­ем: спер­ва с гла­зу на глаз, затем --- при­на­род­но. Если же ты зашел уже слиш­ком дале­ко, чтобы тебя мож­но было исце­лить сло­ва­ми, пусть тебя впредь сдер­жи­ва­ет обще­ст­вен­ное бес­че­стье. А если тебя нуж­но про­нять чем-нибудь посиль­нее, чтобы ты почув­ст­во­вал, отпра­вишь­ся в ссыл­ку в незна­ко­мые места. Вот еще один зако­ре­не­лый него­дяй; его испор­чен­ность настоль­ко окреп­ла, что и лече­ние тре­бу­ет­ся кру­тое: про­пи­шем цепи и государ­ст­вен­ную тюрь­му. (3) А ино­му я ска­жу так: «Твоя душа неиз­ле­чи­ма; ты нани­зы­ва­ешь пре­ступ­ле­ние на пре­ступ­ле­ние, уже не дожи­да­ясь, пока появит­ся при­чи­на для их совер­ше­ния, хотя в при­чи­нах для злых дел недо­стат­ка нико­гда не будет; для того, чтобы гре­шить, тебе доволь­но одной вели­кой при­чи­ны --- греха. Ты опил­ся под­ло­стью, и она настоль­ко про­пи­та­ла твои внут­рен­но­сти, что вынуть ее мож­но толь­ко с ними вме­сте; бед­ня­га, ты уже дав­ным-дав­но ищешь смер­ти --- мы помо­жем тебе, выле­чим, изба­вим тебя от безу­мия, из-за кото­ро­го ты муча­ешь­ся и мучишь дру­гих; мы пре­до­ста­вим тебе един­ст­вен­ное еще доступ­ное тебе бла­го --- смерть». Зачем я ста­ну гне­вать­ся на того, кому помо­гаю, чем могу? Ведь убить --- это ино­гда выс­шее мило­сер­дие. (4) Если бы я при­шел в боль­ни­цу или в бога­тый дом как знаю­щий врач, я не стал бы про­пи­сы­вать одно и то же лече­ние раз­ным боль­ным. Теперь же мне пору­че­на забота о здо­ро­вье обще­ства, и я вижу в раз­ных душах самые раз­ные поро­ки; для каж­дой болез­ни нуж­но изыс­кать свое сред­ство: одно­го исце­лит страх перед бес­че­стьем, дру­го­го изгна­ние, третье­го боль, чет­вер­то­го нуж­да, пято­го желе­зо. (5) Так что даже если мне при­дет­ся надеть мою одеж­ду маги­ст­ра­та наизнан­ку и созы­вать народ­ное собра­ние труб­ным сигналом\footnote{Что такое perversa vestis, «одеж­да наизнан­ку», неяс­но. Воз­мож­но, маги­ст­рат, пред­седа­тель­ст­во­вав­ший на судеб­ном заседа­нии по уго­лов­но­му делу, носил тогу каким-то осо­бен­ным обра­зом. Что каса­ет­ся труб­но­го сиг­на­ла, то тру­би­ли перед домом обви­ня­е­мо­го в уго­лов­ном пре­ступ­ле­нии, вызы­вая его в суд, и в дру­гих обще­ст­вен­ных местах, при­гла­шая всех про­чих граж­дан при­сут­ст­во­вать при раз­би­ра­тель­стве}, я взой­ду на три­бу­нал без вся­кой враж­деб­но­сти и яро­сти, и лицо мое будет лицом зако­на, а тор­же­ст­вен­ные сло­ва я про­из­не­су не сры­ваю­щим­ся от бешен­ства, а спо­кой­ным и серь­ез­ным голо­сом, чтобы при­каз испол­нить закон про­зву­чал не гнев­но, а суро­во. Отда­вая пове­ле­ние обез­гла­вить пре­ступ­ни­ка, заши­вая в мешок отце­убий­цу, посы­лая на казнь сол­да­та, ста­вя на вер­ши­ну Тар­пей­ской ска­лы пре­да­те­ля или вра­га государ­ства, я буду дей­ст­во­вать без гне­ва, с тем же выра­же­ни­ем лица и настро­е­ни­ем души, с каким я уби­ваю змей и ядо­ви­тых живот­ных.

(6) «Гнев необ­хо­дим, чтобы нака­зы­вать». --- Неуже­ли? Может быть, тебе кажет­ся, что закон гне­ва­ет­ся на тех, кого не зна­ет, кого не видел, о чьем появ­ле­нии на свет не подо­зре­вал? Мы долж­ны пере­нять дух зако­на, кото­рый не гне­ва­ет­ся, а поста­нов­ля­ет. Ибо если доб­ро­му чело­ве­ку поз­во­ли­тель­но гне­вать­ся на дур­ные поступ­ки, то ему непредо­суди­тель­но и завидо­вать сча­стью дур­ных людей. В самом деле, может ли быть зре­ли­ще отвра­ти­тель­нее, чем про­цве­та­ние иных нече­стив­цев, зло­употреб­ля­ю­щих мило­стью фор­ту­ны, в то вре­мя как для таких, как они, труд­но даже при­ду­мать доста­точ­но злую судь­бу, чтобы она была по заслу­гам? Одна­ко доб­рый чело­век испы­ты­ва­ет не боль­ше зави­сти, наблюдая их бла­го­ден­ст­вие, чем гне­ва, когда видит их пре­ступ­ле­ния: доб­рый судья осуж­да­ет негод­ные поступ­ки, но не испы­ты­ва­ет к ним нена­ви­сти.

(7) «Так что же, неуже­ли, когда муд­ре­цу при­хо­дит­ся зани­мать­ся подоб­ны­ми дела­ми, они нисколь­ко не тро­га­ют его души и не выво­дят ее из обыч­но­го бес­стра­стия?» --- Да, навер­ное, и он чув­ст­ву­ет какое-то лег­кое, едва замет­ное вол­не­ние. По сло­вам Зено­на, и в душе муд­ре­ца оста­ет­ся шрам от зажив­шей раны. Так что и он, види­мо, испы­ты­ва­ет что-то вро­де теней или при­зра­ков чувств, хотя сами они ему совер­шен­но чуж­ды.

17. (1) Ари­сто­тель гово­рит, что неко­то­рые чув­ства, если пра­виль­но ими поль­зо­вать­ся, могут слу­жить хоро­шим оружием\footnote{В точ­но­сти такой цита­ты у Ари­сто­те­ля нет. По свиде­тель­ству Бонит­ца «ору­жие» у него часто озна­ча­ет «орудие», «вспо­мо­га­тель­ное сред­ство». Сене­ка мог иметь в виду место из ари­сто­теле­вой «Поли­ти­ки», где гово­рит­ся об ору­жии доб­ро­де­те­ли, но толь­ко там ору­жи­ем доб­ро­де­те­ли назва­ны не чув­ства, а ум и харак­тер, кото­рые могут слу­жить как доб­ру, так и злу (1253b 33).}. Это было бы вер­но, если бы их мож­но было брать и откла­ды­вать в сто­ро­ну по усмот­ре­нию того, кто ими поль­зу­ет­ся, как воен­ное сна­ря­же­ние. Но это ору­жие, кото­рым Ари­сто­тель снаб­жа­ет доб­ро­де­тель, всту­па­ет в бой само по себе, не дожи­да­ясь руки; вы им не вла­де­е­те --- оно само вла­де­ет вами. (2) Нет, нам не нуж­но ника­кое дру­гое сна­ря­же­ние: при­ро­да доста­точ­но сна­ряди­ла нас разу­мом. В нем она дала нам ору­жие твер­дое, надеж­ное, не знаю­щее изно­са, послуш­ное, не обо­юдо­ост­рое; оно нико­гда не обра­тит­ся про­тив сво­его хозя­и­на. Одно­го разу­ма доста­точ­но не толь­ко для того, чтобы зара­нее обду­мать ход сра­же­ния, но и для того, чтобы про­ве­сти его; и ко все­му про­че­му, что может быть глу­пее, чем разу­му про­сить под­держ­ки у гне­ва, чтобы вещь проч­ная опи­ра­лась на нена­деж­ную, вер­ная пола­га­лась на невер­ную, здра­вая про­си­ла помо­щи у боль­ной?

(3) А если я ска­жу тебе, что разум сам по себе мно­го силь­нее гне­ва даже в тех делах, для кото­рых один толь­ко гнев и счи­та­ет­ся полез­ным? Ведь разум, одна­жды рас­судив, что нечто долж­но быть сде­ла­но, оста­ет­ся тверд в сво­ем реше­нии: он нико­гда не най­дет ниче­го луч­ше­го, чем он сам, к чему сто­и­ло бы пере­ме­нить­ся, и пото­му оста­ет­ся верен еди­но­жды выбран­но­му реше­нию. (4) А гнев часто отсту­па­ет, напри­мер, перед мило­сер­ди­ем; в нем нет насто­я­щей кре­по­сти, он надут пустой спе­сью и вну­ши­те­лен лишь с виду, а раз­ру­ши­те­лен лишь вна­ча­ле, как вет­ры, под­ни­маю­щи­е­ся от зем­ли и беру­щие нача­ло в реках и болотах: они поры­ви­сты, но быст­ро ути­ха­ют. Так и гнев: (5) начи­на­ет­ся поры­вом огром­ной силы, а потом сла­бе­ет и исся­ка­ет рань­ше вре­ме­ни; тот самый гнев, кото­рый весь так и горел кро­во­жад­но­стью, изо­бре­тая мыс­лен­но неслы­хан­ные виды каз­ни, к тому вре­ме­ни, как надо каз­нить, уже, глядишь, утих и смяг­чил­ся. Чув­ство сни­ка­ет быст­ро, разум все­гда оди­на­ков. (6) Там, где чрез­вы­чай­ная суро­вость судьи осно­вы­ва­ет­ся на гне­ве, неред­ко слу­ча­ет­ся, если заслу­жи­ваю­щих смерт­ной каз­ни ока­жет­ся мно­го, что после кро­ви дво­их или тро­их гнев исся­ка­ет и на даль­ней­шие убий­ства его не хва­та­ет. Страш­ны все­гда пер­вые его при­сту­пы; как уку­сы ядо­ви­той змеи смер­тель­ны, когда она, потре­во­жен­ная, толь­ко что выполз­ла из сво­ей норы; но после несколь­ких уку­сов яд исся­ка­ет, и зубы ее ста­но­вят­ся без­вред­ны. (7) Таким обра­зом полу­ча­ет­ся, что совер­шив­шие рав­ные про­ступ­ки пре­тер­пе­ва­ют нерав­ные нака­за­ния, и даже про­ви­нив­ший­ся мень­ше тер­пит боль­ше, если попа­да­ет­ся под более горя­чую руку, да и вооб­ще гнев неурав­но­ве­шен: то он пере­хо­дит вся­кую меру, то исче­за­ет рань­ше вре­ме­ни, ибо он очень снис­хо­ди­те­лен к себе, во всем дает себе волю; суж­де­ния его про­из­воль­ны, и слу­шать он нико­го не жела­ет; не при­зна­ет чужих дово­дов и хода­тайств, твер­до дер­жась одна­жды при­ня­то­го направ­ле­ния, и ни за что не отсту­пит­ся от сво­его суж­де­ния, даже если оно ока­жет­ся непра­виль­ным.

18. (1) Разум тер­пе­ли­во выслу­ши­ва­ет обе сто­ро­ны, а затем и сам про­сит отсроч­ки, чтобы хва­ти­ло вре­ме­ни взве­сить все и най­ти исти­ну; гнев торо­пит­ся. Разум хочет выне­сти такое реше­ние, кото­рое было бы спра­вед­ли­вым; гнев хочет, чтобы счи­та­лось спра­вед­ли­вым то, что он зара­нее решил. (2) Разум не смот­рит ни на что, кро­ме само­го пред­ме­та, о кото­ром идет речь; гнев воз­буж­да­ет­ся пусты­ми и не иду­щи­ми к делу мело­ча­ми. Его может вос­пла­ме­нить слиш­ком уве­рен­ное лицо, черес­чур гром­кий голос, чуть воль­нее обыч­но­го речь, чуть изыс­кан­нее наряд, слиш­ком често­лю­би­вый адво­кат или явная бла­го­склон­ность наро­да; неред­ко гнев про­тив защит­ни­ка застав­ля­ет осудить обви­ня­е­мо­го; даже когда исти­на обна­ру­жи­ва­ет­ся со всей оче­вид­но­стью, гнев леле­ет и береж­но хра­нит свое заблуж­де­ние; он ни за что не поз­во­лит себя пере­убедить, и упор­ст­во­ва­ние в дур­ных начи­на­ни­ях кажет­ся ему достой­нее рас­ка­я­ния.

(3) Я пом­ню Гнея Пизона\footnote{Гней Каль­пур­ний Пизон --- кон­сул 7 г. до н. э., про­кон­сул Афри­ки с 6 г. до н. э. по 12 г. н. э. Друг Авгу­ста и Тибе­рия, изве­стен высо­ко­ме­ри­ем и необуздан­ным нра­вом. С 17 г. --- намест­ник Сирии, где поссо­рил­ся с Гер­ма­ни­ком, имев­шим там неогра­ни­чен­ные пол­но­мо­чия. Свою болезнь в Сирии Гер­ма­ник при­пи­сы­вал яду, кото­рый дал ему Пизон. После смер­ти Гер­ма­ни­ка в 19 г. при­ехал в Рим, где был обви­нен и покон­чил с собой. См.: Тацит. Анна­лы. II. 43—81; III. 8—15.}, мужа, сво­бод­но­го от мно­гих поро­ков, но имев­ше­го пре­врат­ные пред­став­ле­ния: он почи­тал твер­до­стью отсут­ст­вие гиб­ко­сти. Он одна­жды при­ка­зал отве­сти на казнь сол­да­та, вер­нув­ше­го­ся из отпус­ка без сво­его спут­ни­ка, за то, что тот яко­бы убил сво­его това­ри­ща, раз не может его предъ­явить. Сол­дат про­сил отсроч­ки, чтобы отыс­кать про­пав­ше­го, но ему не дали. Осуж­ден­ный был выведен за вал и уже под­ста­вил шею, как вдруг объ­явил­ся его това­рищ, счи­тав­ший­ся уби­тым. (4) Тут цен­ту­ри­он, назна­чен­ный руко­во­дить каз­нью, при­ка­зы­ва­ет охран­ни­ку спря­тать меч и отво­дит осуж­ден­но­го назад к Пизо­ну, чтобы Пизон снял с него обви­не­ние, ибо его уже сня­ла с сол­да­та фор­ту­на. При огром­ном сте­че­нии наро­да обо­их това­ри­щей, обняв­ших друг дру­га, везут назад через лагерь, под гром­кие радост­ные кри­ки их сорат­ни­ков. Тут на три­бу­нал в яро­сти под­ни­ма­ет­ся Пизон и при­ка­зы­ва­ет вести на казнь обо­их, и того сол­да­та, кото­рый не уби­вал, и того, кото­ро­го не уби­ли. (5) Что может быть недо­стой­нее? Двое долж­ны были погиб­нуть отто­го, что один ока­зал­ся неви­нов­ным. Но Пизон доба­вил и третье­го. Он при­ка­зал каз­нить и того цен­ту­ри­о­на, кото­рый вер­нул осуж­ден­но­го. Из-за неви­нов­но­сти одно­го чело­ве­ка обре­че­ны были на гибель трое --- тут же на месте. (6) О, до чего же изо­бре­та­те­лен быва­ет гнев, когда ему нуж­но при­ду­мать при­чи­ны сво­его бешен­ства. «Тебя, --- ска­зал Пизон, --- я при­ка­зы­ваю каз­нить пото­му, что ты уже осуж­ден; тебя --- пото­му, что ты был при­чи­ной осуж­де­ния това­ри­ща по ору­жию; а тебя --- за то, что не испол­нил при­ка­за сво­его вое­на­чаль­ни­ка убить это­го сол­да­та». Так он при­ду­мал три пре­ступ­ле­ния --- отто­го, что не смог най­ти ни одно­го.

19. (1) Повто­ряю: глав­ное зло гнев­ли­во­сти заклю­ча­ет­ся в том, что она неуправ­ля­е­ма. Она гне­ва­ет­ся на саму исти­ну, если ей пока­жет­ся, что исти­на про­ти­во­ре­чит ее воле. С кри­ком, шумом, сотря­са­ясь всем телом, она пре­сле­ду­ет тех, кого одна­жды выбра­ла, осы­пая их руга­нью и про­кля­ти­я­ми. (2) Разум нико­гда не дела­ет это­го; но если надо, мол­ча, спо­кой­но, с кор­нем выры­ва­ет целые дома, уни­что­жая целые семей­ства, опас­ные для государ­ства, вме­сте с жена­ми и детьми, срав­ни­вая с зем­лей жили­ща и наве­ки истреб­ляя самые име­на, враж­деб­ные сво­бо­де. Но при этом он не станет скре­же­тать зуба­ми, дер­гать голо­вой и про­из­во­дить тело­дви­же­ния, непри­лич­ные для судьи, кото­ро­му сле­ду­ет быть тем спо­кой­нее и сохра­нять на лице тем боль­шую важ­ность, чем серь­ез­нее выно­си­мый им при­го­вор.

(3) Как гово­рит Иеро­ним: «Если тебе хочет­ся кого-то заре­зать, какой смысл вна­ча­ле кусать губы?»\footnote{Иеро­ним Родос­ский — фило­соф-пери­па­те­тик (ок. 290—230 гг. до н. э.), пре­по­да­вал в Афи­нах, напи­сал меж­ду про­чим и трак­тат «О без­гне­вии».} Инте­рес­но, что бы он ска­зал, если бы увидел, как про­кон­сул спры­ги­ва­ет со сво­его воз­вы­ше­ния, выры­ва­ет у лик­то­ра фас­ки и разди­ра­ет соб­ст­вен­ную одеж­ду --- и все это толь­ко пото­му, что на ком-то дру­гом одеж­ду разде­рут поз­же, чем ему хоте­лось бы? (4) Какой смысл пере­во­ра­чи­вать стол? Зачем бить посу­ду? Зачем бить­ся о колон­ны? К чему рвать на себе воло­сы и уда­рять себя по бед­ру или в грудь? Какой вели­кий гнев! --- дума­е­те вы. Отто­го, что ему нель­зя сей же момент обру­шить­ся на дру­го­го, как ему хочет­ся, он обра­ща­ет­ся про­тив само­го себя! И вот чело­ве­ка уже креп­ко дер­жат окру­жаю­щие, умо­ляя поми­рить­ся с самим собой.

(5) Ниче­го подоб­но­го не станет делать тот, кто сво­бо­ден от гне­ва. Он назна­чит каж­до­му нака­за­ние, како­го тот заслу­жи­ва­ет, а часто отпу­стит даже и того, кого ули­чил в про­ступ­ке. Если рас­ка­я­ние в соде­ян­ном поз­во­ля­ет наде­ять­ся на луч­шее, если он пой­мет, что под­лость в этом чело­ве­ке еще не пусти­ла глу­бо­кие кор­ни, а запач­ка­ла толь­ко, как гово­рит­ся, кра­е­шек души, он отпу­стит его без­на­ка­зан­но, что не повредит ни про­щен­но­му, ни про­стив­ше­му. (6) Такой судья будет ино­гда карать боль­шие пре­ступ­ле­ния не так суро­во, как мень­шие, если пер­вые совер­ше­ны непред­на­ме­рен­но и без жесто­ко­сти, а послед­ние обна­ру­жи­ва­ют и тща­тель­но скры­вав­ше­е­ся ковар­ство. За один и тот же про­сту­пок он назна­чит раз­ное нака­за­ние, если один чело­век допу­стил его по небреж­но­сти, а дру­гой, напро­тив, ста­рал­ся при­чи­нить как мож­но боль­ше вреда. (7) Он будет ста­рать­ся нико­гда не забы­вать о том, что любое нака­за­ние при­ме­ня­ет­ся с опре­де­лен­ной целью: одно --- для исправ­ле­ния дур­ных людей, дру­гое --- для их уни­что­же­ния. И в том и в дру­гом слу­чае он будет иметь в виду не про­шед­шее, а буду­щее, ибо, как гово­рит Платон\footnote{Пла­тон. Зако­ны. XI. 934 A.}, бла­го­ра­зум­ный чело­век нака­зы­ва­ет не пото­му, что было совер­ше­но пре­ступ­ле­ние, но для того, чтобы оно не было совер­ше­но впредь: про­шед­шее нель­зя изме­нить, но буду­щее мож­но пред­от­вра­тить. Иных он будет уби­вать пуб­лич­но, как образ­цы зако­ре­не­ло­го поро­ка, не под­даю­ще­го­ся исправ­ле­нию, не толь­ко для того, чтобы их самих пре­дать смер­ти, но чтобы их смер­тью устра­шить дру­гих.

(8) Дабы все это взве­сить и пра­виль­но оце­нить, нужен чело­век, сво­бод­ный от вся­ко­го душев­но­го смя­те­ния: толь­ко тако­го сле­ду­ет допус­кать к делу, тре­бу­ю­ще­му вели­чай­шей осмот­ри­тель­но­сти, --- к вла­сти над жиз­нью и смер­тью. Хуже нет, как дове­рить меч раз­гне­ван­но­му.

20. (1) Не сле­ду­ет так­же думать, буд­то гнев в какой-то сте­пе­ни спо­соб­ст­ву­ет вели­чию духа. Это не вели­чие; это наду­тая опу­холь. Когда боль­ное тело разду­ва­ет­ся, нали­ва­ясь зло­ка­че­ст­вен­ной жид­ко­стью, это не рост, а опас­ная отеч­ность. (2) Все, кого без­рас­суд­ный дух побуж­да­ет счи­тать себя сверх­че­ло­ве­ком, дума­ют, что стре­мят­ся к чему-то необык­но­вен­но воз­вы­шен­но­му; но под нога­ми у них нет твер­до­го осно­ва­ния, а все, что вырос­ло без фун­да­мен­та, обре­че­но упасть. У гне­ва нет опо­ры. Он рож­да­ет­ся из того, что не проч­но и не посто­ян­но, и сам поэто­му --- пустое и вет­ре­ное чув­ство; он так же далек от под­лин­но­го вели­чия духа, как наг­лость от храб­ро­сти, занос­чи­вость в себе --- от уве­рен­но­сти, уны­ние от серь­ез­но­сти, жесто­кость от суро­во­сти. (3) Меж­ду воз­вы­шен­ным духом и над­мен­ным, повто­ряю, очень боль­шая раз­ни­ца. Гнев не пред­при­ни­ма­ет ника­ких уси­лий совер­шить что-нибудь пре­крас­ное или вели­кое; напро­тив, гнев­ли­вость кажет­ся мне отли­чи­тель­ной чер­той несчаст­но­го, вяло­го, ничтож­но­го духа, сознаю­ще­го соб­ст­вен­ную сла­бость; подоб­ные души быва­ют так же болез­нен­но чув­ст­ви­тель­ны, как боль­ные гно­я­щи­е­ся тела, лег­чай­шее при­кос­но­ве­ние к кото­рым вызы­ва­ет гром­кие сто­ны. Так что гнев --- самый жен­ст­вен­ный и ребя­че­ский из поро­ков. --- «Одна­ко он встре­ча­ет­ся и у мужей». --- Конеч­но, пото­му что и у мужей быва­ет жен­ский или дет­ский харак­тер.

(4) «Ниче­го подоб­но­го, --- воз­ра­зят нам, --- послу­шай­те, как иные гово­рят в гне­ве, и вы увиди­те, что сами эти речи выда­ют вели­кий дух». --- Да уж, вели­кий --- толь­ко для тех, кто не зна­ет, что такое насто­я­щее вели­чие. А для тех, кто зна­ет, такие речи страш­ны и омер­зи­тель­ны: «Пусть нена­видят, лишь бы боялись»\footnote{Стих из тра­гедии Акция «Атрей» --- Trag. Rom. fragm. 203. (Ribbeck).}. Сра­зу вид­но, что напи­са­но во вре­ме­на Сул­лы. Я даже не знаю, какое из двух поже­ла­ний хуже: быть пред­ме­том нена­ви­сти или стра­ха. --- «Пусть нена­видят». --- Но, ска­зав это, он сооб­ра­зил, что рано или позд­но они ста­нут про­кли­нать вслух, устра­и­вать заго­во­ры и, нако­нец, с ним покон­чат. И чего же он тогда поже­лал в при­да­чу? Да пока­ра­ют его боги, он при­ду­мал сред­ство, достой­ное того, от чего оно долж­но помочь --- нена­ви­сти. «Пусть нена­видят…» --- и что же даль­ше? Лишь бы пови­но­ва­лись? --- Нет. Лишь бы ува­жа­ли? --- Нет. Так что же? --- «Лишь бы боя­лись». На таком усло­вии я не захо­тел бы и люб­ви. (5) Ты дума­ешь, что это ска­за­но от вели­чия духа? Оши­ба­ешь­ся; это не сверх­че­ло­ве­че­ское вели­чие, а вели­кая бес­че­ло­веч­ность.

Не верь сло­вам людей, охва­чен­ных гне­вом: они мно­го шумят, мно­го гро­зят­ся, а внут­ри --- самая трус­ли­вая душа. (6) Не верь тому, что напи­са­но у крас­но­ре­чи­вей­ше­го мужа Тита Ливия: «Муж харак­те­ра ско­рее вели­ко­го, неже­ли доброго». Эти вещи неот­де­ли­мы друг от дру­га: харак­тер либо будет доб­рым, либо не будет вели­ким. Я пони­маю под вели­чи­ем души ее неко­ле­би­мую без­мя­теж­ность и внут­рен­нюю проч­ность; сни­зу довер­ху она рав­но твер­да и неиз­мен­на, чего не может быть в дур­ных харак­те­рах. (7) Они могут быть буй­ны, могут вну­шать страх и нести раз­ру­ше­ние; но вели­чия в них не будет нико­гда, ибо его сила, его проч­ность и его осно­ва­ние --- в доб­ро­те. (8) Впро­чем, манер­ной речью, поведе­ни­ем и всем сво­им внеш­ним видом они часто про­из­во­дят впе­чат­ле­ние вели­че­ст­вен­ное; изре­ка­ют что-нибудь мно­го­зна­чи­тель­ное, что ты при­ни­ма­ешь за про­яв­ле­ние вели­чия духа, вро­де Гая Цезаря\footnote{Мно­го­чис­лен­ные при­ме­ры нече­стия Кали­гу­лы при­во­дят­ся у Све­то­ния. Жиз­не­опи­са­ния XII цеза­рей. Кн. IV.}, кото­рый раз­гне­вал­ся одна­жды на небе­са за то, что сво­им гро­мом они поме­ша­ли ему смот­реть пан­то­ми­му; ему, прав­да, не столь­ко хоте­лось ее смот­реть, сколь­ко потом само­му пред­ста­вить, а тут мол­нии пере­пу­га­ли всю раз­гуль­ную ком­па­нию --- вот уж, поис­ти­не, они попа­ли не туда, куда нуж­но! --- и он вызвал Юпи­те­ра на бой, при­чем не про­сто так, а на смерт­ный, выкрик­нув небу гоме­ров­ский стих: «Или ты меня, или я тебя!»\footnote{Гомер. Или­а­да. XXIII. 724: так Аякс ста­рал­ся вызвать гнев у Одис­сея, чтобы заста­вить его сра­зить­ся.} (9) Какое безу­мие! То ли он был уве­рен, что ему не может при­чи­нить вреда даже Юпи­тер, то ли думал, что сам может повредить Юпи­те­ру. Я пола­гаю, что это его вос­кли­ца­ние при­ба­ви­ло нема­ло реши­мо­сти заго­вор­щи­кам: они реши­ли, что доль­ше ужи­вать­ся с тем, кто не может ужить­ся даже с Юпи­те­ром, пере­хо­дит пре­дел чело­ве­че­ско­го тер­пе­ния.

21. (1) Итак, в гне­ве нет реши­тель­но ниче­го вели­ко­го, ниче­го бла­го­род­но­го, каким бы он ни казал­ся могу­чим, какое бы ни выска­зы­вал пре­зре­ние к богам и людям.
А если кому-то угод­но счи­тать гнев источ­ни­ком душев­но­го вели­чия, то пусть счи­та­ет тако­вым и страсть к рос­ко­ши: ведь она стре­мит­ся вос­сесть на сло­но­вой кости, одеть­ся в пур­пур, кров­лю обить золо­том, пере­не­сти с места на место целые зем­ли, пере­го­ро­дить моря, реки пре­вра­тить в водо­па­ды, а рощи под­ве­сить в возду­хе.
(2) Пусть тогда источ­ни­ком вели­чия счи­та­ет­ся и коры­сто­лю­бие: оно поко­ит­ся на горах золота и серебра, возде­лы­ва­ет паш­ни раз­ме­ром с хоро­шую про­вин­цию, и каж­дый из вили­ков управ­ля­ет у него име­ни­ем куда б\'{о}льшим, чем доста­ва­лось в управ­ле­ние кон­су­лам. (3) Пусть и похоть счи­та­ет­ся вели­чи­ем духа: как-никак, она пере­плы­ва­ет проливы\footnote{Име­ют­ся в виду Геро и Леандр --- одна из излюб­лен­ных пар алек­сан­дрий­ской эро­ти­че­ской поэ­зии. Леандр каж­дую ночь пере­плы­вал Гел­лес­понт, чтобы повидать­ся с Геро.}, каст­ри­ру­ет целые ста­да маль­чи­ков, идет пря­мо на меч раз­гне­ван­но­го супру­га, пре­зи­рая смерть. Вели­чи­ем духа при­дет­ся при­знать и често­лю­бие: не удо­вле­тво­ря­ясь годич­ны­ми поче­стя­ми, оно хочет, если удаст­ся, запол­нить одним-един­ст­вен­ным име­нем весь кален­дарь, назвать в честь одно­го име­ни все посе­ле­ния на зем­ном шаре.

(4) Нет, все эти стра­сти, как бы дале­ко они ни захо­ди­ли и сколь­ко бы про­стран­ства ни захва­ты­ва­ли, все­гда оста­нут­ся узки­ми, жал­ки­ми, низ­мен­ны­ми. Одна доб­ро­де­тель вели­ка и воз­вы­шен­на. Да и не может быть вели­чия без спо­кой­ст­вия.