\chapter{Культура, обычаи и традиции}

\section{Праздники}
\subsection{Рождеств\'{о} Христ\'{о}во}
Рождеств\'{о} Христ\'{о}во --- праздник \explain{правосл\'{а}вного}{правосл\'{а}вный: orthodox} календаря, \explain{устан\'{о}вленный}{established} 7 января (25 декабря ст стиля).

По народному календ\'{а}рю этот день являлся днём з\'{и}мнего \explain{солнцевор\'{о}та}{солнцевор\'{о}т: solistice}, когда начиналось \explain{проб\'{у}ждение}{awakening (пробуждать/пробудить: to waken)} солнца после его дл\'{и}тельного з\'{и}мнего сна. Рождество Христово почиталось по всей России и по своей \explain{значимости}{значимость: significance} в православном календар\'{е} стояло на втором месте после П\'{а}схи. В русской деревне оно \explainDetail{отмеч\'{а}лось}{отмечаться/отметиться}{to celebrate} обычно в течение трёх дней и начиналось с посещения хр\'{а}ма, которое считалось у крестьян делом желательным, но не строго обязательным.

Рождество также отмечалось двумя тр\'{а}пезами: в рожд\'{е}ственский \explainDetail{соч\'{е}льник}{(рождественский) соч\'{е}льник}{Christmas eve; кан\'{у}н: eve} (канун праздника) и \explain{непосредственно}{directly} в Рождество.

Трапеза \explain{накануне}{on the eve of} праздника начиналась с появлением на небе первой вечерней звезды. На стол \explainDetail{подавали}{подавать/подать}{to serve} блины или оладьи с медом, \explain{постные}{пост: fasting} пироги с грибами, картофелем, кашей, пресные пирожки с ягодами, а также кутью из крупных зерен пшеницы с ягодами.

Тр\'{а}пеза, проходившая в день Рождества, предполагала богатый и \explain{разнообразный}{diverse} обед, во время которого подавалось множество мясных и молочных блюд, пирогов.

Рождество б\'{ы}ло первым днём \explain{выполнения}{performance; execution; effectuation} различных \explain{обр\'{я}дов}{обр\'{я}д: ritual}, которые должны б\'{ы}ли \explainDetail{обеспечить}{обеспечивать/обеспечить}{to provide} \explain{благополучие}{well-being} в наступающем солнечном году, \explain{предохранить}{(предохранять): to protect} от \explain{бед}{беда: misfortune; trouble} и несчастий дом, семью, \explain{скот}{cattle}, узнать будущее.

В рожд\'{е}ственский сочельник начинали колядов\'{а}ть. Группы детей, подр\'{о}стков, молодых мужчин и женщин обходили крестьянские дом\'{а} и \explain{исполн\'{я}ли}{исполнять/исполнить: to perform; to carry out} песни (кол\'{я}дки) с величаниями и пожеланиями хоз\'{я}йственного благопол\'{у}чия и семейного \explain{дов\'{о}льства}{довольство: contentment}. Начинались \explain{гад\'{а}ния}{гад\'{а}ние: fortune telling, divination} о судьбе.

С днём Рождества б\'{ы}ли св\'{я}заны разл\'{и}чные \explainDetail{прим\'{е}ты}{примета}{omen}.
Русские крестьяне верили в то, что травы и зернов\'{ы}е культуры будут хорош\'{и}, если на Рождество лежат \explain{глуб\'{о}кие}{глуб\'{о}кий, глуб\'{о}кая, глуб\'{о}кое, глуб\'{о}кие: deep} снег\'{а}; если в Рождество на небе много звёзд --- можно ждать бог\'{а}того \explain{урож\'{а}я}{(noun) harvest} гор\'{о}ха, а если в этот день сильная метель, то пчелы будут хорошо роиться. (По И. Шангиной)


\subsection{Масленица}
М\'{а}сленица --- большой народный праздник \explainDetail{проводов}{проводить}{to see off; пр\'{о}воды: used only in plural} зимы. В традиционном русском \explainDetail{быт\'{у}}{быт}{life; everyday life (в быт\'{у}: locative; in life)} эта неделя ст\'{а}ла самым \explain{\'{я}рким}{яркий: bright; brilliant; outstanding}, наполненным радостью жизни праздником.

Масленица отмечалась по всей России и в деревнях, и в городах. Её празднование считалось для всех русских людей обязательным: «Хоть себя заложи, а масленицу проводи». Неучастие в масленичном веселье могло повлечь за собой, \explain{по поверью}{according to the belief}, «жизнь в г\'{о}рькой беде».
Первые три дня масленой недели шла \explain{подготовка}{preparation} к празднику: привозили дров\'{а} для масленичных костров, убирали \explain{избы}{изба: hut; little house}. Основные \explain{пр\'{а}зднества}{пр\'{а}зднество: festivals} \explain{приходились}{happen} на четверг, пятницу, субботу, воскресенье --- дни широкой масленицы.

Все масленичные развлечения проходили обычно на улице. \explain{Нар\'{я}дно одетые люди}{elegantly dressed people} участвовали в праздничном \explain{гулянье}{celebration}, поздравляли друг друга, шли на ярмарку, удивлялись чудесам, которые показывали в балаганах --- передвижных театрах, радовались кукольным представлениям и «медвежьим потехам» --- выступлениям вожака с медведем. Масленичный комплекс включал в себя такие развлечения, как катание с гор, катание на санях, кулачные бои, шествия ряженых и др. В масленицу звучало много песен, прибауток, приговоров.

Прощ\'{а}лись с масленицей всегда в воскресенье.
В этот день жгли костры, которые символизировали солнце и должны были \explain{способствовать}{ (поспособствовать) to contribute} скорейшему пробуждению природы, и жгли \explain{ч\'{у}чело}{scarecrow} Масленицы. (По И. Шангиной)



\subsection{П\'{а}сха (Воскресение Христово)}
Название П\'{а}сха происходит от древнееврейского слова «песах» (прохождение). В русский язык слово «пасха» пришло из греческого, вместе с принятием православия, но у многих славянских народов праздник Воскресения Христова называется \explain{иначе}{differently; otherwise}: у украинцев --- великдень, у белорусов --- вяликдзень, у болгар --- великден, у поляков ---Wielkanoc, у чехов --- Velikanoc и т.д.

Нет праздника более торжественного, более радостного, чем Светлое
Христово Воскресение. Конец марта --- апрель --- это уже время прихода
настоящей весны, её \explainDetail{победы}{поб\'{е}да}{victory}
над зимними холодами, праздник \explainDetail{возрождения}{возрожд\'{е}ние}{rebirth},
воскрешения природы. Вместе с тем это и начало нового
\explainDetail{сельскохозяйственного}{сельскохоз\'{я}йственный}{agricultural (of the agric. economy)}
года, начало \explainDetail{полев\'{ы}х}{полев\'{о}й}{related to a field} работ
--- \explain{вспашки}{plowing} и \explain{сева}{(сев) sowing}.

В Древней Руси именно с весеннего \explainDetail{равноденствия}{равноденствие}{equinox}
начинался новый год, а чтобы посевы \explain{благополучно}{safely} взошли и вызрели,
скот давал приплод, н\'{у}жно было исполнить различные магические обряды.
Один из таких \explain{сохранившихся}{survived} до настоящего времени обрядов
--- пасхальная тр\'{а}пеза, когда единственный раз в году готовятся блюда,
которые в другие дни к столу не \explainDetail{подаются}{подавать/подать}{to serve}:
куличи и крашеные яйца.

Крашеными яйцами \explainDetail{обменивались}{обмениваться/обменяться}{to exchange},
их дарили родным, соседям, пришедшим поздравить, их брали с собой, отправляясь
в гости, раздавали нищим. (По Л.С. Лаврентьевой и Ю.И. Смирнову)

