\chapter{Погода, Климат, Изменение климата}

\section{Времена года}
\textit{Источник: \url{https://www.memorysecrets.ru/english-texts/vremena-goda-seasons.html}}

Каждое время года имеет свои плюсы и минусы. Давайте детальнее их разберем.

Весна (март, апрель, май) --- самое восхитительное время года. Несмотря на то, что весной \explainDetail{наблюдаются}{(по)наблюдаться}{to be observed} \explainDetail{ливни}{ливень}{shower (rain)}, многие люди \explain{воспринимают}{perceive} весну, как самое лучшее время года. Почему же так \explain{происходит}{happens}? \explain{Дело в том}{the thing is}, весна значит что-то новое для нас. Это шанс начать все заново, изменить жизнь к лучшему. Именно весной рожд\'{а}ется новая жизнь --- \explain{распускаются}{they bloom} почки и начинают цвести цветы.

Лето же (июнь, июль, август) --- это время отдыхать и просто получать удовольствие. Солнце светит с утра до ночи. Взрослые уходят в отпуск, дети --- на каникулы. Большинство людей проводят свои отпуска на море. Там можно не только \explainDetail{загореть}{загорать/загореть}{to tan}, но и встретить много незнакомых людей.

Осень (сентябрь, октябрь, ноябрь) можно разделить на две части --- начало осени и ее конец. В начале осени школьники снова возвращаются к учебе и садятся \explain{за парты}{behind the desks}. Лето постеп\'{е}нно переходит в осень. Но настроение еще \explain{приподнятое}{elevated}. Вскоре желтые листья начнут осыпаться, и этот величественный пейзаж тоже не \explainDetail{позв\'{о}лит}{позвол\'{я}ть/позв\'{о}лить}{to allow} нам грустить. Втор\'{а}я часть осени не так очаров\'{а}тельна, как первая. Дожди, \explain{сл\'{я}коть}{mud} и грязь --- это то, что люди ненавидят больше всего.

Зима (декабрь, январь, февраль) \explain{на нос\'{у}}{фразеологизм: приближаться, близиться (обычно -- по времени)}, приближаются холода. Мы надеваем перчатки, шапки, шубы и вязаные шарфы. Иногда в эту пору школы закрывают на карантин с целью предотвратить распространение гриппа. Хорошая новость для учеников --- вторые зимние каникулы. Рождество тоже вот-вот наступит. На улице порошит снег, дети лепят снеговиков и играют в снежки. Скоро мы будем наряжать елку, вешать на нее различные шары и гирлянды. Пора встречать Новый год!

\section{Что такое изменение климата?}
\textit{Источник: \url{https://www.un.org/ru/climatechange/what-is-climate-change}}

Под изменением климата понимают \ed{долгосрочные}{долгоср\'{о}чный}{long-term (for a long period)} температурные изменения и изменение погодных \ed{условий}{усл\'{о}вие}{condition}. Хотя эти изменения м\'{о}гут быть естественными, как, например, \ed{циклические}{циклический}{cyclical} \ed{колебания}{колеб\'{а}ние}{fluctuation} солнечной активности, с \ed{1800-х}{1800-х}{тысячи восемьсотых} годов антропогенная деятельность является основным \explain{движущим фактором}{driving factor} изменения климата, \explain{главным образом}{mainly} за счёт \ed{сжигания}{сжигание}{burning; combustion} ископаемых видов топлива, таких как уголь, нефть и газ.

В результате сжигания ископаемых видов топлива образуются \explain{выбросы}{emissions} \ed{парниковых газов}{парник\'{о}вые г\'{а}зы}{greenhouse gases}, которые подобно одеялу \explain{окутывают}{they envelop} Землю, удерживая солнечное тело и \explain{повышая}{raising} температуру.

Примерами парниковых газов, выбросы которых вызывают изменение климата, являются \explain{дву\'{о}кись углер\'{о}да}{углекислый газ} и мет\'{а}н. Они образуются, например, при использовании бензина для езды на автомобилях или угля для отопления зданий. \explain{Расчистка земель и лесов}{the ``clearing'' of lang and forest} также может привести к \ed{высвобождению}{высвобожд\'{е}ние}{release} углекислого газа.
Главным источником выбросов метана являются \explain{м\'{у}сорные св\'{а}лки}{garbage dumps}.
К числу основных производителей выбросов относятся энергетика, промышленность, транспорт, здания, сельское хозяйство и землеп\'{о}льзование.

\begin{fancyquotes}
    Концентрация парниковых газов находится на самом высоком уровне за последние 2 миллиона лет
\end{fancyquotes}

И выбросы продолжают расти! В результате этого сейчас Земля на \explain{1,1°C}{одна целая, одна десятая градуса Цельсии} теплее, чем в конце 1800-х годов. Прошедшее десятилетие (2011–2020 г\'{о}ды) было самым тёплым в истории.

Хотя многие думают, что изменение климата означает в основном более высокие температуры, рост температуры -- это только начало истории. Поскольку Земля -- это система, где всё \explain{взаимосв\'{я}зано}{interconnected}, изменения в одной \explain{сфере}{here: area} могут повлиять на изменения во всех остальных.

В настоящее время к последствиям изменения климата относят, \explain{среди прочего}{among other things}, сильные \ed{з\'{а}сухи}{з\'{а}суха}{drought}, \explain{нехватку воды}{water shortage}, сильные пожары, повышение уровня моря, наводнения, \explain{таяние}{melting (n.)} полярных льдов, катастрофические штормы и сокращение \ed{биоразнообр\'{а}зия}{биоразнообр\'{а}зие}{biodiversity}.

\begin{fancyquotes}
    Люди ст\'{а}лкиваются с различными последствиями изменения климата
\end{fancyquotes}

Изменение климата может \ed{сказ\'{а}ться}{ск\'{а}зываться/сказ\'{а}ться + на что}{affect} на нашем здоровье, способности \explain{выращивать}{to grow} \explain{продовольственные культуры}{food crops}, жильё, безопасности и работе.
Некоторые из нас уже сейчас более \ed{уязв\'{и}мы}{уязв\'{и}мый}{vulnerable} к \ed{воздействию}{возд\'{е}йствие}{impact (n)} изменения климата, например, люди, живущие в малых \ed{островн\'{ы}х}{островн\'{о}й}{insular} государствах и других развивающихся странах. Такие последствия, как повышение уровня моря и интрузия соленых вод, дост\'{и}гли такого уровня, что целые общ\'{и}ны были в\'{ы}нуждены \explain{пересел\'{и}ться}{to resettle}, а затяжные засухи \explain{подвергают}{subject; expose} людей риску г\'{о}лода. В будущем ожидается рост числа «климатических беженцев».

\begin{fancyquotes}
    Каждый дополнительный градус \ed{глобального потепления}{глобальное потепление}{global warming} \explain{имеет значение}{matters}
\end{fancyquotes}

В докладе Организации Объединённых Наций за 2018 год тысячи учёных и \ed{правительственных экспертов}{прав\'{и}тельственный эксп\'{е}рт}{gov't expert} согласились с тем, что \explain{ограничение}{limitation} роста глобальной температуры на уровне не более 1,5°C поможет нам избежать самых худших климатических последствий и \explain{сохранить}{save} кл\'{и}мат, \explain{приг\'{о}дный для жизни}{habitable}. Однако, согласно текущим национальным климатическим планам, ожидается, что глобальное потепление дост\'{и}гнет 2,7°C к концу столетия.

Хотя выбросы, вызывающие изменение климата, образуются во всех регионах мира и сказываются на всех, некоторые страны производят их в гораздо больших объёмах, чем другие. В то время как на 100 стран, которые производят меньше всего выбросов, приходится 3 процента от общего объема выбросов, \explain{доля}{share (n)} десяти стран, являющихся самыми крупными производителями выбросов, составляет 68 процентов. Хотя принятие мер по борьбе с изменением климата -- это дело всех и каждого, народы и страны, которые создают больше проблем, должны взять на себя большую ответственность и начать действовать первыми.

\begin{fancyquotes}
    Мы сталкиваемся с огромным \ed{вызовом}{вызов}{challenge}, но нам уже известны многие решения
\end{fancyquotes}

Многие решения в области изменения климата могут быть не только экономически выгодными, но и также улучшить нашу жизнь и защитить окружающую среду. У нас также есть глобальные структуры и соглашения, на основе которых осуществляются \ed{усилия}{усилие}{effort}, направленные на достижение прогресса, такие как цели в области \ed{устойчивого развития}{устойчивый развитие}{sustainable development}, Рамочная конвенция Организации Объединённых Наций об изменении климата и Парижское соглашение. Имеются три широкие категории действий: сокращение выбросов, адаптация к последствиям изменения климата и финансирование необходимых мер по адаптации.

\explain{Перевод}{Translation (meaning transition)} энергетических систем с ископаемого топлива на использование \ed{возобновляемых источников энергии}{возобновляемые источники энергии}{renewable energy sources}, таких как солнце или ветер, позв\'{о}лит сократить выбросы, вызывающие изменение климата. Но начинать нужно прямо сейчас. Несмотря на расширение коалиции стран, \explain{обязавшихся}{who pledged} достичь чистого нулевого уровня выбросов к 2050 году, около половины мер по сокращению выбросов должны быть осуществлен\'{ы} к 2030 году, с тем чтобы удержать потепление на уровне ниже 1,5°C. В период с 2020 по 2030 год производство ископаемого топлива должно сокращаться прим\'{е}рно на 6 процентов в год.

Цель адаптации к последствиям изменения климата состоит в том, чтобы защитить людей, их жилища, предприятия, источники средств к существованию, инфраструктуру и природные экосистемы. Эта деятельность касается не только нынешних последствий, но и тех, с которыми, вероятно, придётся столкнуться в будущем. Хотя меры по адаптации придётся принимать \explain{повсеместно}{everywhere}, в настоящее время приоритетное внимание необходимо уделять наиболее уязв\'{и}мым сло\'{я}м населения, у которых меньше всего ресурсов для того, чтобы \explain{противостоять}{resist} климатическим угр\'{о}зам. Это может \explain{окупиться}{pay off} \explain{сполна}{completely}. Например, системы раннего предупреждения \ed{стихийных бедствий}{стихийное бедствие}{natural disaster} спасают жизни и имущество и могут принести выгоды, в десятки раз превышающие первоначальные \explain{затраты}{costs}.

\begin{fancyquotes}
    Мы можем оплатить счет сейчас или дорого заплатить в будущем
\end{fancyquotes}

Меры по борьбе с изменением климата требуют значительных финансовых \ed{вложений}{вложение}{investment} со стороны правительств и делов\'{ы}х кругов. Однако \explain{бездействие}{inaction} в отношении климата обходится гораздо дороже. Одним из важнейших шагов является выполнение \ed{промышленно развитыми странами}{промышленно развитая стран\'{а}}{industrialised countries} своих обязательств по предоставлению 100 миллиардов долларов в год \ed{развивающимся странам}{развив\'{а}ющаяся стран\'{а}}{developing country}, с тем чтобы они могли адаптироваться и перейти к более «зеленой» экономике.



\section{Причины изменения климата}
\textit{Источник: \url{https://www.un.org/ru/science/causes-effects-climate-change}}



\textbf{Производство электроэнергии}

Значительная доля глобальных выбросов связана с производством электроэнергии и тепла путем сжигания ископаемых видов топлива. Бóльшая часть электроэнергии по-прежнему производится посредством сжигания угля, нефти или газа, в результате чего образуются углекислый газ и закись азота – мощные парниковые газы, которые покрывают Землю и задерживают солнечное тепло. Во всем мире чуть более четверти электроэнергии вырабатывается за счет ветра и солнца и поступает из других возобновляемых источников, которые, в отличие от ископаемых видов топлива, практически не выделяют в атмосферу парниковых газов или загрязняющих веществ.

\textbf{Изготовление товаров}

Предприятия обрабатывающей и других отраслей промышленности производят выбросы, в большинстве случаев являющиеся результатом сжигания ископаемых видов топлива в целях выработки энергии, необходимой для получения цемента, железа, стали, электронных устройств, пластмасс, одежды и других товаров. При добыче полезных ископаемых и других промышленных процессах, равно как и при строительстве, также выделяются газы. Машины, используемые в производственном процессе, зачастую работают на угле, нефти или газе, а некоторые материалы, такие как пластмассы, производятся из химических веществ, получаемых из ископаемых видов топлива. Обрабатывающая промышленность является одним из крупнейших источников выбросов парниковых газов в мире.

\textbf{Вырубка лесов}

В результате вырубки лесов для создания ферм или пастбищ либо по иным причинам образуются выбросы, поскольку вырубаемые деревья высвобождают накопленный углерод. Ежегодно уничтожается около 12 млн гектаров леса. Поскольку леса поглощают углекислый газ, их уничтожение также ограничивает способность природы удерживать выбросы в атмосферу. Обезлесение наряду с сельским хозяйством и другими изменениями в землепользовании является причиной примерно четверти глобальных выбросов парниковых газов.

\textbf{Использование транспорта}

Большинство автомобилей, грузовиков, кораблей и самолетов работают на ископаемых видах топлива. Это делает транспорт одним из главных источников выбросов парниковых газов, особенно выбросов углекислого газа. Наибольшая их часть приходится на дорожные транспортные средства в связи со сжиганием продуктов нефтепереработки, таких как бензин, в двигателях внутреннего сгорания. При этом выбросы морских и воздушных судов продолжают расти. На транспорт приходится почти четверть глобальных выбросов углекислого газа, связанных с энергоснабжением. Существующие тенденции указывают на вероятность значительного увеличения энергопотребления в транспортном секторе в ближайшие годы.

\textbf{Производство продуктов питания}

Производство продуктов питания приводит к выбросам углекислого газа, метана и других парниковых газов разными путями, включая вырубку лесов и расчистку земель для ведения сельского хозяйства и выпаса скота, работу пищеварительных систем коров и овец, производство и применение удобрений и навоза для выращивания сельскохозяйственных культур и использование энергии для эксплуатации сельскохозяйственного оборудования или рыболовецких судов, обычно работающих на ископаемых видах топлива. Все это делает производство продуктов питания одним из основных факторов, способствующих изменению климата. Выбросы парниковых газов также связаны с упаковкой и распространением продуктов питания.

\textbf{Энергоснабжение зданий}

В мировом масштабе жилые и коммерческие здания потребляют более половины всей электроэнергии. В связи с продолжающимся использованием угля, нефти и природного газа для целей отопления и охлаждения они выбрасывают значительные количества парниковых газов. В последние годы повышение спроса на энергию для отопления и охлаждения с ростом численности владельцев кондиционеров и увеличение потребления электричества для освещения и обеспечения работы бытовой техники и подключенных устройств способствовали увеличению выбросов углекислого газа, производимых зданиями и связанных с энергоснабжением.

\textbf{Слишком интенсивное потребление}

Ваш дом и использование электроэнергии, то, как вы передвигаетесь, то, что вы едите, и количество того, что вы выбрасываете, влияют на выбросы парниковых газов. Это же можно сказать о потреблении таких товаров, как одежда, электронные устройства и пластмассы. Значительная часть глобальных выбросов парниковых газов связана с частными домохозяйствами. Наш образ жизни оказывает глубокое воздействие на нашу планету. Самые состоятельные лица несут наибольшую ответственность: на 1 процент самых богатых жителей планеты в совокупности приходится больше выбросов парниковых газов, чем на 50 процентов беднейшего населения.

На основе различных источников ООН


\section{Последствия изменения климата}
\textit{Источник: \url{https://www.un.org/ru/science/causes-effects-climate-change}}


\textbf{Повышение температур}

С увеличением концентрации парниковых газов растет и глобальная температура земной поверхности. Последнее десятилетие – 2011–2020 годы – стало самым теплым за всю историю наблюдений. С 1980-х годов каждое десятилетие было теплее предыдущего. Почти во всех районах суши наблюдается увеличение количества жарких дней и периодов аномальной жары. Повышение температуры увеличивает количество заболеваний, связанных с жарой, и затрудняет работу на открытом воздухе. Природные пожары легче возникают и быстрее распространяются в более жарких условиях. Температура в Арктике повышалась по крайней мере вдвое быстрее, чем в среднем по миру.

\textbf{Усиление штормов}

Многие регионы столкнулись с увеличением интенсивности и частоты разрушительных штормов. При повышении температуры испаряется больше влаги, что усиливает ливневые дожди и наводнения, вызывая более опасные штормы. На частоту и масштабы тропических штормов также влияет потепление океана. Циклоны, ураганы и тайфуны формируются в теплых водах у поверхности океана. Такие ураганы нередко разрушают дома и населенные пункты, становясь причиной гибели людей и огромных экономических потерь.

\textbf{Усиление засухи}

Изменение климата меняет степень доступности воды, делая ее более дефицитным ресурсом в растущем числе регионов. Глобальное потепление усугубляет нехватку воды в регионах, и без того испытывающих ее дефицит, и увеличивает риск сельскохозяйственных засух, влияющих на урожай, и экологических засух, повышающих уязвимость экосистем. Засухи также могут вызывать разрушительные песчаные и пыльные бури, способные перемещать миллиарды тонн песка через континенты. Пустыни расширяются, сокращая площадь земель для выращивания продовольственных культур. Сегодня многие люди постоянно сталкиваются с угрозой нехватки воды.

\textbf{Потепление и повышение уровня океана}

Океан поглощает бóльшую часть тепла, образующегося в процессе глобального потепления. За последние двадцать лет скорость, с которой океан нагревается, сильно возросла на всех его глубинах. По мере потепления океана его объем увеличивается, поскольку при нагревании вода расширяется. Таяние ледовых щитов также приводит к повышению уровня моря, угрожая прибрежным и островным сообществам. Кроме того, океан поглощает из атмосферы углекислый газ. При этом увеличение количества углекислого газа повышает кислотность океана, что ставит под угрозу морскую флору и фауну и коралловые рифы.

\textbf{Исчезновение видов}

Изменение климата создает риски для выживания видов на суше и в океане. Эти риски возрастают по мере повышения температуры. Мир, положение в котором усугубляется изменением климата, теряет виды в тысячу раз быстрее, чем когда-либо в письменной истории человечества. Миллион видов находится под угрозой исчезновения в течение следующих нескольких десятилетий. В число многочисленных угроз, связанных с изменением климата, входят лесные пожары, экстремальные погодные условия и инвазивные вредители и заболевания. Некоторые виды смогут сменить место обитания и выжить, а другие нет.

\textbf{Нехватка продуктов питания}

В группу причин глобального роста распространенности голода и неполноценного питания входят климатические изменения и увеличение количества экстремальных погодных явлений. Рыбные ресурсы, сельскохозяйственные культуры и домашний скот могут быть уничтожены или стать менее продуктивными. В связи с закислением океана морские ресурсы, обеспечивающие питание для миллиардов людей, находятся под угрозой. Изменения снежного и ледяного покрова во многих арктических регионах нарушили систему снабжения продовольствием, обеспечиваемым за счет пастбищного животноводства, охоты и рыболовства. Тепловой стресс может уменьшать количество воды и пастбищ, что приводит к снижению урожайности сельскохозяйственных культур и негативным образом сказывается на поголовье скота.

\textbf{Увеличение рисков для здоровья}

Изменение климата – это самая большая угроза для здоровья людей. Его последствия уже наносят вред здоровью в связи с загрязнением воздуха, распространением заболеваний, возникновением экстремальных погодных явлений, вынужденным перемещением, оказанием давления на психику и обострением проблем голода и неполноценного питания в местах, где люди не могут выращивать продовольственные культуры или обеспечить наличие достаточного количества пищевых продуктов. Экологические факторы ежегодно уносят жизни около 13 млн человек. Изменение погодных условий приводит к распространению заболеваний, а экстремальные погодные явления увеличивают смертность и затрудняют работу систем здравоохранения.

\textbf{Нищета и вынужденное перемещение}

Изменение климата усиливает факторы, ввергающие людей в нищету и не позволяющие им исправить ситуацию, в которой они оказались. Наводнения могут смести городские трущобы, разрушив дома и уничтожив источники средств к существованию. Жара может затруднить работу на открытом воздухе. Нехватка воды может повлиять на урожай. В последние десять лет (2010–2019 годы) связанные с погодой явления приводили к вынужденному перемещению в среднем около 23,1 млн человек в год, повышая риск оказаться в нищете для еще большего числа людей. Большинство беженцев прибывают из самых уязвимых стран, наименее готовых адаптироваться к последствиям изменения климата.

На основе различных источников ООН
