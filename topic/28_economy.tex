% \chapter{Экономика}

\section{Восемь богачей мира владеют половиной богатств Земли}
Восемь \ed{богач\'{е}й}{бог\'{а}ч}{rich person} мира владеют половиной \ed{бог\'{а}тств}{бог\'{а}тство}{wealth} Земли. Всего восемь \explain{толстосумов}{толстосум: moneybag} \explain{владеют}{владеть/завладеть: to possess (влад\'{е}ю, влад\'{е}ешь, влад\'{е}ют)} тем же состоянием, что принадлежит беднейшей половине населения Земли. К такому выводу пришли исследователи благотворительной организацией Oxfam. Более того, исследование показало, что уже в ближайшие 25 лет в мире может появиться первый триллионер. Чтобы потратить такое состояние, н\'{у}жно ежедневно в течение 27 столетий и еще 38 лет расходовать по одному миллиону.

Восемь миллиардеров владеют тем же состоянием, которое находится в руках 3 миллиардов 600 миллионов человек, отмечает РИА Новости.

Великолепную восьмерку возглавляет основатель Microsoft Билл Гейтс. Его состояние оценивается в 75 миллиардов долларов. У Амансио Ортеги 67 миллиардов. Экс-президента Inditex женщины знают по марке магазинов одежды Zara. Третью ступень пьедестала богатства занимает американский \explain{предприниматель}{business-person; entrepreneur} Уоррен Баффетт с состоянием почти 61 миллиард долларов.

Оставшаяся пятёрка толстосумов выглядит так: мексиканский бизнесмен Карлос Слим Элу --- 50 миллиардов долларов, глава компании по электронной торговле Amazon Джефф Безос --- чуть более 45 миллиардов долларов, основатель соцсети Facebook Марк Цукерберг --- чуть менее 45 миллиардов долларов, глава корпорации Oracle Ларри Эллисон с состоянием 43 миллиарда и 600 миллионов долларов и владелец агентства деловой информации Майкл Блумберг с 40 миллиардами.
В 2010 году считалось, что стоимость имущества, которым владеют половина беднейших жителей Земли, равна совокупному состоянию 43 богатейших людей планеты. Так что за последние шесть лет ситуация лишь усугубилась.

``Это возмутительно, что такие средства сосредоточены в руках всего нескольких человек, когда каждый десятый в мире вынужден выживать на менее чем два доллара в день'', --- отметил исполнительный директор Oxfam Уинни Бьянима.

По его мнению, неравенство способствует тому, что сотни миллионов человек находятся в ловушке бедности.

``Это разрушает наши общества и подрывает демократию'', --- настаивает Бьянима.

Сегодня семь из десяти человек живут в странах, где за последние три десятилетия неравенство в доходах возросло. При этом всего один процент населения планеты владеет таким же богатством, как 99 процентов тех, кто не попал в золотой процент.

Кроме того, чтобы труд женщин в мире оплачивался так же, как и мужской, понадобится еще порядка 170 лет, отмечают в Oxfam.

Oxfam (Oxford Committee for Famine Relief) основана в Оксфорде в 1942 году для помощи голодающим. Ее цель --- решение проблем бедности и связанной с ней несправедливости во всем мире. Международное объединение сейчас объединяет 17 организаций в более чем 90 странах мира.

\newpage
\section{НДФЛ без границ}
\textit{Надолго уехавшим из России выпишут 13-процентный налог}

{\it Источник: \url{https://www.kommersant.ru/doc/5988499}}

Минфин доработал поправки о новых правилах уплаты НДФЛ гражданами, более полугода работающими из-за рубежа на российские компании. Согласно обновленной версии, доходы, полученные как по трудовым договорам, так и по договорам гражданско-правового характера, с 2024 года будут облагаться по стандартной ставке 13\%. Таким образом, одинаковый налог будет взиматься как со штатных работников, так и с фрилансеров, как с резидентов, так и с нерезидентов. Для последних это может означать возникновение двойного налогообложения — в РФ и в новой стране пребывания. Попытаться решить проблему можно будет с помощью соглашений об избежании двойного налогообложения, перспективы практического применения которых в нынешних условиях не ясны.

Проект точечных изменений в Налоговом кодексе, предусматривающий в том числе новые правила налогообложения граждан, работающих на российские компании из-за рубежа, доработан для повторного внесения в Госдуму, сообщил в четверг Минфин. Напомним, законопроект был внесен правительством в нижнюю палату 24 апреля, но уже на следующий же день был отозван для «технических уточнений».

Изменения в части НДФЛ для уехавших, однако, оказались весьма существенными. В апрельской версии предлагалось отнести к доходам от источника в России оплату труда, полученную от российского заказчика фрилансерами, работающими по договорам гражданско-правового характера. Пока такие доходы НДФЛ в случае утраты работником резидентства, то есть после полугода проживания за рубежом, вообще не облагаются — и по прежней редакции такие физлица могли подпасть сразу под 30\% налог.

\begin{fancyquotes}
    В новой версии налог для нерезидентов также возникает — но по стандартным ставкам 13\% или 15\% (с доходов, превышающих 5 млн руб. в год).
\end{fancyquotes}

При этом к доходам от источника в России законопроект теперь относит и вознаграждения, полученные при выполнении дистанционным работником трудовой функции по договору с работодателем, являющимся российской организацией или обособленным подразделением иностранной организации, зарегистрированным в РФ,— проще говоря, налог будет взиматься как со штатных работников, так и с фрилансеров, как с резидентов, так и с нерезидентов.

Замминистра финансов Алексей Сазанов такую унификацию объясняет тем, что сейчас российским организациям как налоговым агентам приходится определять ставку налога с дохода работника в зависимости от конкретной ситуации — и «когда сотрудник работает в удаленном форме, конечно, компаниям сложно проверять, является он российским налоговым резидентом или нет». По его словам, новшество должно существенно упростить администрирование для налоговых агентов.

\begin{fancyquotes}
    Компаниям изменения, может, жизнь и облегчат, но самим нерезидентам грозят двойным налогообложением — ведь налоговые органы иностранного государства, как правило, также будут претендовать на обложение доходов своего нового резидента.
\end{fancyquotes}

Как отмечает партнер Kept Донат Подниек, для тех, кто работает в стране, с которой у РФ действует соглашение об избежании двойного налогообложения (это около 80 государств), и платит там налоги с полученной от российского работодателя зарплаты, есть возможность зачесть в РФ иностранный налог. «Правда, в течение какого-то времени, пока не вернут НДФЛ, удержанный в РФ, работник фактически уплатит налоги и в РФ, и за рубежом»,— говорит эксперт.

Как уточнил Алексей Сазанов, в случае приостановки соглашений (такая возможность сейчас обсуждается в отношении недружественных стран — см. “Ъ” от 15 марта), нормы, позволяющие зачесть уплаченный налог, продолжат действовать. Вопрос, отметим, в том, будет ли в нынешних условиях это возможным на практике при нынешнем уровне информационного обмена и сотрудничества с недружественными юрисдикциями. В самой же невыгодной ситуации окажутся те, кто уехал в страны, с которым таких соглашений нет или они денонсированы (например, с Нидерландами),— тогда доход, предупреждает руководитель практики «Структурный и налоговый консалтинг» юркомпании «Лемчик, Крупский и партнеры» Людмила Круглова, будет облагаться налогом и в России, и в стране пребывания без возможности зачета.

\begin{fancyquotes}
    Донат Подниек отмечает, что для тех, у кого должным образом оформлена работа за рубежом и российский работодатель НДФЛ не удерживает, изменения ухудшат положение работников.
\end{fancyquotes}

Но есть случаи, когда допсоглашений о дистанционной работе за рубежом нет и работодатель «для подстраховки» удерживает НДФЛ по ставке 30\% — в таком случае нагрузка снизится. Людмила Круглова отмечает, что в силу отсутствия у налоговых органов возможности автоматически проверять статус резидентства граждан, видимо, власти пришли к выводу, «что ужесточение правил не позволит собрать с уехавших граждан 30\% НДФЛ: добровольно в массовом порядке о потере статуса резидента граждане уведомлять не будут, а проверять этот статус у миллионов работников в ручном режиме невозможно».

По словам ассоциированного партнера МЭФ Legal Анны Зеленской, изменения логичны в связи с цифровой трансформацией рынка труда и распространенностью дистанционной работы. «Тезис о том, что работодатели должны следить за физическим местоположением сотрудника, наверное, действительно несколько устарел», — говорит она.

\newpage
\section{Зарплаты многих трудящихся все еще ниже уровня, который был до начала пандемии}

 {\it Источник: \url{https://news.un.org/ru/story/2022/05/1424372}}


\begin{fancyquotes}
    Ограничения в связи с пандемией привели к серьезным потрясениям на рынке труда, а также росту неравенства внутри стран и между странами. Война в Украине и ряд других факторов еще в большей мере усугубили эти тенденции. Об этом говорится в новом докладе Международной организации труда (МОТ).
\end{fancyquotes}

В первом квартале 2022 года суммарное количество отработанных рабочих часов в мире было на 3,8 процента меньше, чем в четвертом квартале 2019 года – до кризиса, связанного с пандемией. Это эквивалентно дефициту в 112 миллионов рабочих мест с полной
\ed{занятостью}{занятость}{(\textit{official/legal}). Employment. Занятость - это деятельность граждан, связанная с удовлетворением личных и общественных потребностей, не противоречащая законодательству Российской Федерации и приносящая, как правило, им заработок, трудовой доход (далее - заработок). $\sim$ Закон РФ от 19.04.1991 N 1032-1 (ред. от 28.12.2022)}.


Авторы доклада полагают, что причиной такой тенденции являются многие взаимосвязанные глобальные кризисы, в том числе инфляция, особенно цен на энергоносители и продукты питания, финансовые потрясения, долговой кризис и сбои в глобальной цепочке поставок, усугубляемые войной в Украине. Они опасаются дальнейших серьезных потрясений на рынке труда в ближайшие месяцы.

«Российская агрессия против Украины уже влияет на рынки труда в Украине и за ее пределами...», – говорится в докладе.

\textbf{Неравномерное восстановление рынков труда}

Его авторы отмечают также неравномерные темпы восстановления экономики и рынков труда после пандемии. В богатых странах наблюдалось восстановление количества отработанных часов, но страны с низким уровнем дохода и уровнем дохода ниже среднего столкнулись со спадом в первом квартале этого года с разрывом в 3,6 и 5,7 процента соответственно по сравнению с докризисным уровнем. Эксперты опасаются, что этот разрыв будет усугубляться еще больше во втором квартале 2022 года.


В некоторых развивающихся государствах правительства не могут справиться с внешними долгами, а предприятия – с экономической и финансовой неопределенностью. Сами же трудящиеся лишены доступа к социальной защите.

\textbf{Новая реальность – низкие доходы трудящихся}

Таким образом, спустя более двух лет после начала пандемии, все еще ощущается ее воздействие на рынки труда. Трудовые доходы большинства работников все еще ниже того уровня, который был до пандемии. В 2021 году в мире трое из пяти работников проживали в странах, где трудовые доходы не вернулись к уровню, который был в четвертом квартале 2019 года.

Во время пандемии увеличился также гендерный разрыв в количестве отработанных часов. Больше всего пострадали женщины, занятые в неформальном секторе.

Резкий рост числа вакансий в странах с развитой экономикой в конце 2021 и начале 2022 годов привел к улучшению ситуации на рынке труда. «Но в целом, принимая во внимание значительное количество безработных и недоиспользуемую рабочую силу во многих странах, нет убедительных доказательств того, что рынки труда перегреты», – отмечают авторы доклада.

Они призывают власти всерьез взяться за восстановление рынков труда, ориентированных на человека. «Неравномерное и хрупкое восстановление усугубляется усиливающимся сочетанием кризисов. Воздействие на работников и их семьи, особенно в развивающихся странах, будет разрушительным и может привести к социальным и политическим потрясениям», – предупредил

Генеральный директор МОТ Гай Райдер, добавив, что «сегодня как никогда важно сосредоточиться на восстановлении, ориентированном на человека».

В сложившихся условиях эксперты призывают обеспечить справедливую корректировку заработной платы, в том числе минимальной, укреплять системы социальной защиты, а также принять меры по обеспечению продовольственной безопасности.
Они полагают, что сегодня нужна «корректировка макроэкономической политики» таким образом, чтобы она позволяла решать проблемы, связанные с инфляцией и долгом, и в то же время смогла поддержать восстановление, обеспечивающее создание рабочих мест.


\newpage
\section[Сокращение рабочих мест]{Сокращение рабочих мест и социальное неравенство - последствия пандемии}

\textit{Интервью}

\textit{Источник: \url{https://news.un.org/ru/interview/2022/01/1416852}}


\begin{fancyquotes}
    Неравенство между странами увеличивается – полное восстановление экономики, измеряемое ВВП на душу населения, в ближайшей перспективе останется недостижимым для развивающихся стран. В 2022 году объем производства на душу населения в развивающихся странах будет более чем на 2 процента ниже уровня, ожидаемого до начала пандемии. Такой прогноз, сделанный экспертами ООН, в интервью Службе новостей озвучил Григор Агабекян, сотрудник Департамента ООН по экономическим и социальным вопросам.
\end{fancyquotes}

\textbf{Служба новостей ООН: Как пандемия отразилась на показателях неравенства и существует ли опасность дальнейшего расслоения общества? }

\textbf{ГА:} Неравенство между странами увеличивается - полное восстановление экономики, измеряемое ВВП на душу населения, в ближайшей перспективе останется недостижимым для развивающихся стран. Прогнозируется, что в 2022 году объем производства на душу населения в развивающихся странах и странах с переходной экономикой будет более чем на 2 процента ниже уровня, ожидаемого до начала пандемии.  В то же время предполагается, что ВВП на душу населения в развитых странах почти полностью восстановится к 2023 году, достигнув уровня, ожидаемого до начала пандемии. Такая неравномерность темпов восстановления между развитыми и развивающимися странами приведет к увеличению неравенства доходов между странами и сделает почти невозможным сокращение глобального неравенства к 2030 году, как это предусмотрено Целями устойчивого развития.

Пандемия COVID-19 усугубила неравенство не только между странами, но и внутри стран на фоне различий в уровне безработицы и доле трудового дохода в развитых и развивающихся странах. За первые два квартала 2021 года глобальные трудовые доходы сократились на 5,3 процента, и около 108 миллионов рабочих в настоящее время живут в крайней или средней степени бедности, большинство из них в развивающихся странах. Кроме того, пандемия нанесла более тяжелый урон тем, кто находится в нижней части распределения доходов. В то время как средние доходы людей, входящих в нижние 40 процентов мирового распределения доходов в 2021 году были примерно на 6,7 процента ниже по сравнению с 2019 годом, доходы людей в верхних 40 процентах снизились всего на 2,8 процента.

\begin{wrapfigure}{l}{0.5\textwidth}
    \begin{fancyquotes}
        Асимметричное воздействие кризиса на занятость и доход между группами населения также усиливает неравенство внутри стран
    \end{fancyquotes}
\end{wrapfigure}
Асимметричное воздействие кризиса на занятость и доход между группами населения также усиливает неравенство внутри стран. Разрыв между высококвалифицированными и низкоквалифицированными профессиями продолжает расти, не только в связи с разным уровнем заработной платы и разной степенью уязвимости при экономических кризисах, но и, что особенно важно, с возможностью работать удаленно. Многие низкоквалифицированные работники сталкиваются с более опасными условиями труда в секторах, таких как розничная торговля, общественное питание, здравоохранение и услуги по уходу, которые связаны с высоким риском заражения COVID-19. Пандемия также ускорила продолжающееся исчезновение рабочих мест средней квалификации, что способствовало усилению поляризации доходов.

В 2020 году глобальная занятость больше всего сократилась среди работников средней квалификации (-4,7 процента), за ними следуют низкоквалифицированные работники (-3,3 процента) и высококвалифицированные работники (-1,3 процента) по сравнению с 2019 годом.


\newpage
\section[Последствия пандемии для сферы занятости]{На Международной конференции труда обсуждают последствия пандемии для сферы занятости}

\textit{Источник: \url{https://news.un.org/ru/story/2021/06/1404182}}

\begin{fancyquotes}
    Рынок труда во время пандемии: безработица, частичная занятость, работа из дома, необходимость соблюдения трудовых норм и принятия мер социальной защиты населения. Все эти вопросы будут обсуждать участники 109-й Международной конференции труда, которая стартовала в понедельник в Женеве. Впервые в истории – из-за ограничений, cвязанных с COVID-19, – она проходит в виртуальном формате.
\end{fancyquotes}

«После того как в прошлом году Международная конференция труда по объективным причинам была отложена, все участники МОТ – правительства, работодатели и трудящиеся – сошлись во мнении, что на этот раз Конференция должна состояться», – сказал Генеральный директор МОТ Гай Райдер.


«Успешное проведение Конференции… позволит МОТ сделать еще один шаг, причем очень важный, к преодолению последствий пандемии COVID-19, которая в последние полтора года нанесла сокрушительный удар по сфере труда», – добавил он.

По данным МОТ, из-за пандемии под сокращения попали 33 миллиона человек. Еще 81 миллион человек по тем или иным причинам вынуждены были сами уйти с работы – например, чтобы присматривать за детьми после закрытия садов и школ.

В докладе МОТ, опубликованном на прошлой неделе, сообщается, что рынок труда восстанавливается медленными темпами и к допандемийным показателям, скорее всего, удастся вернуться не ранее 2023 года.

По прогнозам, в 2022 году число безработных в мире достигнет 205 миллионов, а глобальный уровень безработицы составит 5,7 процента. В последний раз такой показатель был зарегистрирован в 2013 году, когда на рынке труда все еще ощущались последствия «Великой рецессии», или мирового экономического кризиса.

Самая тяжелая ситуация, отмечают эксперты МОТ, складывается в странах Латинской Америки и Карибского бассейна, а также Европы и Центральной Азии.

Так, например, в Восточной Европе только из-за пандемии в 2020 году работы лишились 2,4 миллиона человек. В Центральной и Западной Азии по причинам, связанным с пандемией, без работы в 2020 году остались 3,2 миллиона человек (при общем уровне безработицы 9,8 процента).