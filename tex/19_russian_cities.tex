\chapter{Города, деревни и достопримечательности}

\section{Герб Междуреченска}
В 1966 году был объявлен конкурс на лучший герб города, в котором было рассмотрено 59 эскизов. После рассмотрения представленных эскизов лучшим был признан проект герба под № 48, который городской комитет ВЛКСМ и предложил для утверждения.

\includegraphics[width=0.3\textwidth]{img/Flag_of_Mezhdurechensk_(Kemerovo_oblast).png}

Автор герба Вадим Гущин, несмотря на нарушение ряда важных законов геральдики, сумел просто и оригинально, отказавшись от традиционных для того времени шестерёнок, колб, отбойных молотков, отразить промышленную специфику, совместив её с природно-географическим положением города.

28 августа 1966 года газета «Знамя шахтёра» представила жителям Междуреченска герб: «Щит разделён на два поля: красное (вверху) — цвет труда и зелёное (нижнее) — цвет тайги. В свете вспыхнувшей искры кусок угля — главного нашего богатства. На зелёном поле — две голубых ленты — Томь и Уса. Таким образом, герб олицетворяет две главнейшие особенности города — направленность труда его жителей и природные условия».

Автор герба совершил небольшую ошибку. Дело в том, что река Уса впадает в реку Томь с её правого берега. В гербе 1966 года сходящиеся реки изображены текущими в левую геральдическую сторону (от зрителя — в правую), что не соответствовало действительности. Ошибка была исправлена, и 18 марта 1993 года был утверждён изменённый герб.
