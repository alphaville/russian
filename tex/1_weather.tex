\chapter{Погода, Климат, изменение климата}

\section{Времена года}
Каждое время года имеет свои плюсы и минусы. Давайте детальнее их разберем.

Весна (март, апрель, май) --- самое восхитительное время года. Несмотря на то, что весной \explainDetail{наблюдаются}{(по)наблюдаться}{to be observed} \explainDetail{ливни}{ливень}{shower (rain)}, многие люди \explain{воспринимают}{perceive} весну, как самое лучшее время года. Почему же так \explain{происходит}{happens}? \explain{Дело в том}{the thing is}, весна значит что-то новое для нас. Это шанс начать все заново, изменить жизнь к лучшему. Именно весной рожд\'{а}ется новая жизнь --- \explain{распускаются}{they bloom} почки и начинают цвести цветы.

Лето же (июнь, июль, август) --- это время отдыхать и просто получать удовольствие. Солнце светит с утра до ночи. Взрослые уходят в отпуск, дети --- на каникулы. Большинство людей проводят свои отпуска на море. Там можно не только \explainDetail{загореть}{загорать/загореть}{to tan}, но и встретить много незнакомых людей.

Осень (сентябрь, октябрь, ноябрь) можно разделить на две части --- начало осени и ее конец. В начале осени школьники снова возвращаются к учебе и садятся \explain{за парты}{behind the desks}. Лето постеп\'{е}нно переходит в осень. Но настроение еще \explain{приподнятое}{elevated}. Вскоре желтые листья начнут осыпаться, и этот величественный пейзаж тоже не \explainDetail{позв\'{о}лит}{позвол\'{я}ть/позв\'{о}лить}{to allow} нам грустить. Втор\'{а}я часть осени не так очаров\'{а}тельна, как первая. Дожди, \explain{сл\'{я}коть}{mud} и грязь --- это то, что люди ненавидят больше всего.

Зима (декабрь, январь, февраль) \explain{на нос\'{у}}{фразеологизм: приближаться, близиться (обычно -- по времени)}, приближаются холода. Мы надеваем перчатки, шапки, шубы и вязаные шарфы. Иногда в эту пору школы закрывают на карантин с целью предотвратить распространение гриппа. Хорошая новость для учеников --- вторые зимние каникулы. Рождество тоже вот-вот наступит. На улице порошит снег, дети лепят снеговиков и играют в снежки. Скоро мы будем наряжать елку, вешать на нее различные шары и гирлянды. Пора встречать Новый год!
