\chapter{Наука и Техника}
\section{Северное сияние}
С \explainDetail{наступлением}{наступление}{adventб beginning} осени тысячи туристов \explain{устремляются}{rush} в \explain{Заполярье}{the region of the Arctic circle}, чтобы увидеть уникальный танец небесных огней -- полярное, или северное сияние, на латыни -- Aurora Borealis.
В теории увидеть это природное явление можно с конца августа до середины апреля: в этот период времени ночи становятся темными, солнечная активность \explainDetail{возрастает}{возрастать/возрасти}{возраст\'{а}ю/-ешь/-ет; возраст\'{у}/-ёшь/-\'{у}т: rise, increase}, а облака \explainDetail{расс\'{е}иваются}{рассеиваться/рассеяться}{to disperse}.
Такое развлечение, как \explain{ох\'{о}та}{hunting} за северным сиянием, с каждым годом становится все популярнее как среди россиян, так и среди иностранных туристов, которые специально ради него готовы ехать на Крайний Север. Главное \explain{доказательство}{proof} удачной охоты -- это, конечно же, снимки северного сияния.


\section{Только дьявол мог выдумать Нобелевскую премию}
% https://www.gazeta.ru/science/2015/11/27_a_7914947.shtml
Екатерина Шутова

\textit{120 лет назад Альфред Нобель подписал \explain{завещ\'{а}ние}{will} по Нобелевской премии.}

120 лет назад Альфред Нобель подписал завещание, согласно которому его \explain{накопл\'{е}ния}{accumulation} поступили в фонд Нобелевской премии -- самой престижной на сегодняшний день \explainDetail{награды}{нагр\'{а}да}{prize}, ежегодно \explainDetail{присуждаемой}{присужд\'{а}емый}{awarded} за выдающиеся научные исследования, революционные изобретения или крупный \explain{вклад}{contribution} в культуру или развитие общества. \explainDetail{Отдел}{отдел}{department} науки «Газеты.Ru» вспоминает \explainDetail{подробности}{подробность}{detail} этого события.

В 1888 году журналисты \explain{оповестили}{notified} мир о смерти Альфреда Нобеля -- химика, инженера и изобретателя динамита. Репортеры ошиблись -- на самом деле погиб Людвиг Нобель, брат Альфреда.

\begin{fancyquotes}
    Удивленный изобретатель прочитал в одной из газет собственный некролог под названием «Торговец смертью мертв».
\end{fancyquotes}

Альфред Нобель не захотел оставаться злодеем в глазах человечества. Поэтому 27 ноября 1895 года в Шведско-Норвежском клубе в Париже ученый составил следующее завещание:

{\it
Я, \explain{нижеподписавшийся}{undersigned}, Альфред Бернхард Нобель, обдумав и решив, настоящим объявляю мое завещание по поводу имущества, нажитого мною... Капитал мои душеприказчики должны перевести в ценные бумаги, создав фонд, проценты с которого будут выдаваться в виде премии тем, кто в течение предшествующего года принес наибольшую пользу человечеству.

Указанные проценты следует разделить на пять равных частей, которые предназначаются: первая часть тому, кто сделал наиболее важное открытие или изобретение в области физики, вторая — в области химии, третья — в области физиологии или медицины, четвертая — создавшему наиболее значительное литературное произведение, отражающее человеческие идеалы, пятая — тому, кто внесет весомый вклад в сплочение народов, уничтожение рабства, снижение численности существующих армий и содействие мирной договоренности.

... Мое особое желание заключается в том, чтобы на \explain{присуждение}{awarding, conferment} премий не \explainDetail{влияла}{влиять/повлиять}{influence} национальность кандидата, чтобы премию получали наиболее \explain{достойные}{worthy}, независимо от того, скандинавы они или нет.}

\subsection{Как огорчить родственников}
Спустя год после написания завещания Альфред Нобель скончался на своей вилле от \explainDetail{кровоизлияния}{кровоизлияние}{hemorrhage} в мозг. За несколько лет до смерти ученый сказал о самом себе следующим образом: «Альфред Нобель -- его \explain{существование}{existence} следовало бы \explain{пресечь}{suppress} при рождении милосердным доктором. Основные добродетели: держит ногти в чистоте и никому не бывает в тягость. Основные недостатки: не имеет семьи, наделен дурным характером и плохим пищеварением.

\begin{fancyquotes}
    Величайший грех: не поклоняется Мамоне. Важнейшие события в его жизни: никаких.
\end{fancyquotes}

\explain{Наследники}{heirs} легендарного изобретателя были крайне \explain{возмущены}{outraged}, что огромные накопления уйдут не им в карман, а на поддержку науки. Они требовали, чтобы завещание было признано недействительным. Интересно, что единственным родственником Нобеля, не пытавшимся присвоить себе деньги, оказался его племянник Эммануил. «Русские называют исполнителя завещания «душеприказчик», то есть «представитель души», — заявил юристам мужчина. — Вот и действуйте соответственно». Позднее Эммануил добавил: «Я не хочу, чтобы достойнейшие ученые в будущем упрекали нашу семью в присвоении средств, которые по праву принадлежат им».

В конечном итоге справедливость восторжествовала — и через несколько лет после смерти ученого были вручены пять первых премий. А с 1969 года по инициативе Шведского банка начала присуждаться Нобелевская премия по экономике.

Лишь однажды деньги из фонда премии пошли на дело, никак не связанное с наукой. Софи фон Капивара, женщина, с которой у талантливого изобретателя были отношения, пообещала раскрыть содержание их переписки и посмертно опозорить Альфреда Нобеля. Душеприказчики в страхе выплатили крупную сумму за 216 писем мецената. Ученые до сих пор шутят, что

\begin{fancyquotes}
    наука была бы богаче, если бы не одна алчная молочница.
\end{fancyquotes}

«Ты славная девушка, но ты действуешь мне на нервы»

Существует миф, согласно которому у Альфреда Нобеля была жена, страстно влюбившаяся в математика, и именно поэтому изобретатель «обделил» всех представителей этой науки. Но на самом деле, как заявляют биографы, меценат никогда не был женат. В молодом возрасте Нобель влюбился в работницу аптеки, но та умерла от чахотки. Потосковав, ученый нашел новую пассию — на этот раз ей стала Сара Бернар, знаменитая актриса. Альфред Нобель написал письмо матери о том, что хочет жениться.

\begin{fancyquotes}
    Недаром актеров в старину не разрешали хоронить на кладбище. У них нет души, сыночек!» — предупредила сына любящая родительница.
\end{fancyquotes}



Послушный Нобель разорвал любовную связь с Бернар.

Следующая женщина появилась в жизни мецената, когда тому уже был 41 год. Альфред Нобель опубликовал в газете объявление о том, что ищет секретаршу. На него откликнулась графиня Берта Кински, с которой у изобретателя начался неторопливый и гармоничный роман.

\begin{fancyquotes}
    Кстати, по одной из версий, именно Кински попросила Нобеля вписать в завещание премию мира. А в 1905 году она стала первой женщиной, удостоенной этой премии.
\end{fancyquotes}

У Кински и Нобеля дело до свадьбы не дошло: однажды ученый обнаружил, что его секретарша исчезла, оставив на столе письмо следующего содержания: «Простите меня, господин Нобель. Я уезжаю в Вену, где меня ждет жених. Пожелайте мне счастья, как я желаю счастья вам. Искренне преданная вам Берта Кински, которая в скором времени станет Бертой фон Зуттер».

Последней женщиной в жизни Нобеля стала вышеупомянутая «алчная молочница», которая изрядно надоедала ученому своей глупостью и необразованностью. «Дорогое дитя. Ты славная девушка, но ты действуешь мне на нервы», — раздраженно писал ей в письмах Альфред Нобель.

Так почему же не существует премии по математике? Возможно, все дело в том, что у Альфреда Нобеля не заладились отношения с великим математиком Миттаг-Леффлером, который должен был стать первым лауреатом, а меценат этого не хотел. Но наиболее вероятная версия заключается в том, что Нобель воспринимал математику как инструмент, как сугубо теоретическую науку.

\subsection{Виагра для хомячков и исследование ругани}

В 1991 году появилась пародия на Нобелевскую премию — Шнобелевская премия. Она вручается «за достижения, которые заставляют сначала засмеяться, а потом — задуматься». Учредитель и идейный вдохновитель «Шнобелевки» — Марк Абрахамс, который, будучи редактором юмористического научного журнала, получал множество писем от читателей с подробным рассказом об их «великих» исследованиях. «Иногда эти люди заслуживали премии — правда, не Нобелевской», — говорил Абрахамс. Так редактор решил награждать ученых за самые нелепые достижения.

В разные годы Шнобелевская премия присуждалась

за разработку протезов яичек для собак, за исследование влияния музыки кантри на частоту самоубийств и за открытие, что «Виагра» помогает хомякам справиться с последствиями резкой смены часовых поясов.

Также пародийную награду получали ученые, доказавшие, что ругань снижает боль, и исследователи, изучавшие оральный секс у летучих мышей.

Первым в мире человеком, удостоенным как Шнобелевской, так и Нобелевской премии, стал Андрей Гейм. Голландский ученый российского происхождения был награжден «Шнобелевкой» за использование магнитов для того, чтобы демонстрировать возможность левитации лягушек. Спустя десять лет Гейм совместно со своим учеником Константином Новоселовым получил Нобелевскую премию за изобретение графена.

\subsection{Война еще не закончена, а премии уже раздают}
«Я готов простить Альфреду Нобелю изобретение динамита, но только дьявол в людском обличье мог выдумать Нобелевскую премию!» — воскликнул ирландский романист и драматург Джордж Бернард Шоу, став лауреатом в области литературы (по ироничному заявлению писателя, произошло это потому, что «в тот год он ничего не опубликовал»). Действительно, самая престижная международная награда — явление весьма резонансное и неоднозначное. В Советском Союзе Нобелевский комитет клеймили за то, что «он ухитрился не заметить Алексея Толстого, Максима Горького, Владимира Маяковского, но зато заметил Ивана Бунина. И только тогда, когда он стал эмигрантом, и только потому, что он стал эмигрантом и врагом советского народа».

В Третьем рейхе ученым было запрещено получать Нобелевскую премию, так как в 1935 году премию мира «За борьбу с милитаризмом в Германии» получил пацифист Карл фон Осецкий — ярый противник нацистского режима. В 1937 году Адольф Гитлер издал указ, согласно которому немцы не имели права принимать премию. Из-за указа награду не получили Герхард Домагк «за открытие антибактериального эффекта пронтозила», Адольф Бутенандт за исследование половых гормонов и Рихард Кун за работу по каротиноидам и витаминам.


\begin{fancyquotes}
    Весьма примечателен тот факт, что Бенито Муссолини и Адольф Гитлер были номинированы на Нобелевскую премию мира в 1935 и 1939 годах соответственно.
\end{fancyquotes}

Нобелевская история знает немало случаев отказа от самой престижной международной награды.

Так, в 1973 году политический деятель Фан Динь Кхай отказался от медали «за работу по разрешению вьетнамского конфликта», аргументируя свое решение тем, что «война еще не закончена, а премии уже раздают». Не захотел быть награжденным и Жан-Поль Сартр — французский писатель и драматург. По мнению Сартра, награда посягнет на его независимость — центральное понятие в философии автора. Вскоре после отказа от Нобелевской премии француз еще раз шокировал общественность, заявив, что уходит из литературы. «Литература — суррогат действенного преобразования мира», — с горечью заметил писатель.

\section{Ученые нашли способ записать данные в пяти измерениях}

\textit{Как 5D-диски изменят представление людей о хранении информации?}

\textbf{Ученые создали 5D-диск высочайшей плотности:} В октябре специалисты Саутгемптонского университета в Великобритании описали способ записи огромного количества данных на компактный диск небольших размеров. Технология, получившая название 5D, позволяет сохранить на специальном накопителе до 500 терабайт информации. Получившиеся диски из кварцевого стекла отличаются высочайшей плотностью, которая в десять тысяч раз превышает плотность оптических дисков Blu-Ray. Новый метод позволит эффективно разместить на небольшой площади облачные сервера для хранения данных пользователей, интернет-компаний, крупных корпораций. По словам ученых, это особенно важно на фоне развития технологий, увеличения количества подключенных к сети устройств и роста количества передаваемых через сеть данных.

\textbf{Облачные сервисы с каждым годом становятся все популярнее:}
За последние пять лет отношение потребителей и бизнеса к облачным сервисам изменилось. Раньше их воспринимали в качестве дополнительного метода резервного копирования данных — информация практически всегда поступала в одну сторону. Причем крупные корпорации в основном использовали дата-центры для хранения некритичной информации. К 2020-м годам организации стали использовать облачные серверы не только для аварийного копирования, но и для постоянного обмена данными внутри конкретного предприятия. Системы облачных хранилищ стали более гибкими, позволяя конкретному потребителю выбрать необходимое количество свободного места и производительность оборудования.

Специалисты Analytics Insight называют основными \explainDetail{преим\'{у}ществами}{преим\'{у}щество}{advantage} \explainDetail{облачных}{облачный}{cloud (adj.); \'{о}блако: cloud} дата-центров \explain{круглос\'{у}точный}{round the clock} доступ к информации, возможность одновременной работы не- скольких пользователей с одним массивом данных, масштабируемость и \explain{г\'{и}бкость}{flexibility}, \explain{снижение}{decline} затрат на хранение данных внутри компании.

\explainDetail{Представители}{представитель}{representative} отрасли отмечают, что в обычное время нагрузка на серверы неравномерна: в одной части дата-центров она может зашкаливать, в другой быть крайне небольшой. По этой причине эксперты предсказывают появление искусственного интеллекта, который мог бы анализировать и распределять нагрузку на оборудование. В том числе по этой причине данные пользователей хранятся в нескольких частях дата-центра.

\textbf{Больше всего в облачных сервисах пользователи ценят скорость передачи данных и безопасность:} По словам основателя облачного провайдера Wasabi Дэйва Френда, от дата-центров будущего потребители ожидают высокого уровня безопасности, производительности оборудования и приемлемой цены за услуги. «Резервные копии должны храниться в разрозненных системах, обеспечивающих максимально возможную изоляцию», — заметил предприниматель. Потенциальный злоумышленник, добравшийся до одного сервера, не должен иметь возможность удалить или зашифровать информацию так, чтобы ее нельзя было восстановить из альтернативных источников. Френд полагает, что на этом должна строиться концепция мультиоблака.

Другими критериями облачного сервиса будущего, по мнению Френда, являются доступная цена и высокая скорость передачи данных. Провайдеры должны будут таким образом скорректировать стоимость услуг и добиться определенного качества оборудования, чтобы оставаться конкурентоспособными и не разочаровывать клиентов.

Представители облачного провайдера CloudSigma рассказали, что дата-центры должны будут отвечать за сохранность данных и скорость передачи информации. Для хранения файлов пользователей и корпоративных клиентов они используют небольшие 2,5-дюймовые диски емкостью 250 гигабайт. В случае, если какой-либо диск выходит из строя, его заменяют, а данные восстанавливают через бэкап. При таком развитии событий клиент не теряет своих данных, хотя и оказывается без доступа к информации на 10-15 минут. Благодаря глубокой интеграции между серверами и оборудованием задержка передачи данных внутри дата-центра очень мала. Для того чтобы разогнать скорость и снизить задержку между серверами и пользователем, в компании полагаются на выделенную гигабитную линию интернета.


\textbf{Диски 5D позволят хранить информацию практически бесконечно:}
По оценке Forbes, к 2025 году к интернету будет подключено около 80 миллиардов устройств, которые будут генерировать около 180 триллионов гигабайт данных. В обозримом будущем хранить данные на классических накопителях будет проблематично — существует риск возникновения дефицита и увеличения стоимости хранения информации. Работающие над технологией 5D специалисты Саутгемптонского университета предлагают записывать информацию на кварцевом стекле с помощью фемтосекундных лазеров и сверхкоротких импульсов. «Запись на кварцевый носитель как бы идет в пяти измерениях — двух оптических и трех пространственных», — отмечают авторы исследования.

Инновация британских инженеров заключается в создании дисков повышенной плотности и размещении на небольшом участке колоссальных объемов данных. Например, на «болванке» размером в один дюйм удалось сохранить шесть гигабайт информации. Накопитель обычного для подобных устройств размера, основанный на кварцевых дисках, может сохранить до 500 терабайт данных. Разработка обещает революцию на рынке хранения информации, так как десятки, если не сотни классических дата-центров можно будет объединить в одну библиотеку.

Преимуществами 5D-дисков также называют долговечность и низкую стоимость обслуживания. По оценке ученых, кварцевые диски не прочнее обычных накопителей, однако могут выдержать температуру до 1800 градусов по Фаренгейту, или около тысячи градусов по Цельсию. В случае пожара в дата-центре информация, скорее всего, сохранится. Кроме того, кварцевое стекло со временем не меняет своих свойств, что позволит держать данные на 5D-накопителях практически вечно.

Единственным узким местом будущей разработки является скорость передачи данных. В настоящий момент ученым удалось разогнать ее до 230 килобайт в секунду — за это время на диск можно записать около ста страниц текста. Однако для того, чтобы полностью заполнить болванку емкостью 500 терабайт, потребуется 60 дней. Либо инженеры найдут способ обойти ограничение, либо 5D-диски так и останутся перспективным оборудованием для записи информации. В крайнем случае на подобных дисках можно сохранять данные для потомков. Так, в 2018 году на кварцевый носитель была записана трилогия романов Айзека Азимова «Основание» — диск отправился в космос вместе с Tesla Roadster Илона Маска.