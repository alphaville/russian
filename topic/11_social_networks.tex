% \chapter{Соцсети}

\section{Мы просто зомби}

\textit{Миллионы людей во всем мире страдают от одиночества и тревоги из-за соцсетей. Как с этим бороться?}

\textit{Источник: \url{https://lenta.ru/articles/2021/11/01/algorithm_4/}}

Как соцсети влияют на молодежь? Какими вырастут зумеры, которые всегда смотрят в экран смартфона? Старшее поколение отвечает на этот вопрос однозначно: интернет делает подростков только хуже. Но сами молодые люди до сих пор не понимают, как именно на них влияют соцсети: 31 процент считает, что положительно, а почти половина не может дать однозначного ответа. Их называют цифровыми аборигенами — людьми, которые не помнят или даже не знают, каким был мир до интернета и соцсетей. Вместе с этим они уже столкнулись с эпидемией психических расстройств и самоубийств, стали меньше заниматься сексом и жениться. Как алгоритмы соцсетей и короткие видео меняют растущее поколение — в материале масштабного спецпроекта «Ленты.ру» «Алгоритм. Кто тобой управляет?»

\textbf{Однорукий смартфон}

У социальных сетей одна цель --- сделать все, чтобы пользователь проводил в них как можно больше времени. Принцип их работы сравним с игровыми автоматами: свайп ленты новостей снизу вверх уж очень похож на дерганье ручки «однорукого бандита». И механика абсолютно та же: люди раз за разом делают одно и то же движение в надежде сорвать куш, увидеть что-то экстраординарное. Но если в казино играют на деньги, в социальных сетях на кону — сам человек, его время и внимание.

Непомерное желание увидеть в ленте новостей нечто удивительное и томительное ожидание лайков часто перерастают в болезненную привязанность. И в этом виноваты не сами пользователи, а соцсети, которые используют любые уязвимости в человеческой психике. Нейробиолог и психиатр Жадсон Брюер в своей книге «Зависимый мозг» прямо говорит о том, что соцсети, как и сигареты, вызывают дофаминовое блаженство.

\begin{fancyquotes}
    Большинство людей, опубликовав фото, обязательно зайдут в соцсеть проверить лайки и комментарии.
\end{fancyquotes}

Это раздражает близких и мешает общению, но мы продолжаем это делать, чувствуя бесконечную вину, — а порой даже тайком, чтобы не вызвать осуждения. Желание посмотреть на экран смартфона появляется вновь и вновь.

В основе этого действия лежит выработанная миллионами лет эволюции простая схема: триггер — поведение — вознаграждение. Когда нас хвалят, мы радуемся, когда нет — расстраиваемся. Для закрепления «хорошего» поведения организм использует нейромедиатор дофамин, который отвечает за обучение и мотивацию. Так люди годами учатся вести себя условно «правильно» и избегать негативных последствий. И чем более явно наше «вознаграждение», тем лучше. В старейшую ловушку эволюции попадают и пользователи соцсетей: желание выложить что-то в социальную сеть (триггер) — публикация поста (поведение) — лайки и реакции (вознаграждение). Поведение прочно закрепляется, и из раза в раз зависимость только растет. Именно так люди реализуют желание быть в обществе и стремление к чему-то большому.

\begin{center}
    \includegraphics[width=0.85\textwidth]{img/kak-formiruetsa-pivichka.png}
\end{center}

По мнению Брюера, привычка завязывать самому шнурки схожа с привычкой писать сообщения за рулем автомобиля, — с той лишь оговоркой, что первая безобидна, а вторая опасна для жизни. Между ними на разных позициях по шкале «опасности» можно расположить привычки жевать жвачку, погружаться в себя и многие другие. Место каждой привычки на шкале чаще всего связывают с уровнем стресса и тем, как люди с ним справляются.

\begin{fancyquotes}
    Социальные сети вырабатывают в пользователях отрицательное подкрепление: не только радуют лайками и одобрением, но и позволяют сбежать от реальности, отвлечься от печалей. И чем чаще туда убегать, тем больше туда тянет: поведение становится автоматическим, формируется аддикция.
\end{fancyquotes}

Что же приводит к зависимости людей от алгоритмов социальных сетей? Согласно исследованию Брюера, существует несколько предпосылок: несинхронная коммуникация, отсутствие барьеров и возможность сплетничать в комфортной среде. Усиливает все это ощущение неопределенности — то есть неизвестно, получит ли человек одобрение, лайкнут ли его пост и прокомментируют ли. Это называется периодическим подкреплением, и именно такую стратегию используют казино и игровые аппараты. Выигрыш случаен, но при этом достаточен для мотивации. И пользователь уже на крючке. Если он вдруг попробует сорваться и не будет открывать соцсеть в течение хотя бы пары часов, приложение обязательно пришлет ему уведомление.

\textbf{Обманчивая близость}

Создатели всех популярных соцсетей прекрасно понимают, что ради заработка используют уязвимости человеческой психики. Но все равно продолжают это делать. В этом, к примеру, признавался бывший президент Facebook Шон Паркер.

Люди пользуются соцсетями, потому что ищут одобрения. Согласно исследованию 2015 года, именно желание быть положительным героем в глазах других людей приводит пользователей в ловушку соцсетей. Каждая публикация — это крик о помощи, знак того, что автору некомфортно, одиноко или попросту скучно. Лайки и комментарии утоляют жажду одобрения, но она появляется снова и снова. «Вы проверяете телефон по утрам, прежде чем пописать, или пока писаете. У вас всего два варианта», — говорит в документальном фильме «Социальная дилемма» Роджер МакНеми, один из ранних инвесторов в Facebook.

\begin{fancyquotes}
    В итоге обилие общения в сети приводит к тому, что человек теряет способность справляться с проблемами и попадает в изоляционный капкан. Так лайки и сообщения не улучшают настроение, а лишь создают его иллюзию. Более того, картинка чужой счастливой жизни не только не радует, но и подавляет других.
\end{fancyquotes}

В 2014 году Facebook решил изучить степень своего влияния на психологическое состояние пользователей. В эксперименте приняли участие 689 тысяч пользователей соцсети. Правда, сами они понятия не имели, что над ними ставят опыты.

Эксперимент состоял в следующем: для части пользователей алгоритм отбирал публикации таким образом, чтобы они были преимущественно позитивными, для другой части в ленту новостей добавили побольше негатива. Исследователи в течение недели наблюдали за тем, как меняются реакции подопытных и что они сами публикуют.

Изменения в поведении людей были поразительными. Оказалось, что человек, встречающий огромное количество негативных публикаций, сам начинает постить такой же контент. Но если резко увеличить в ленте позитив, настроение пользователя улучшается. Манипулировать состоянием людей оказалось слишком просто: достаточно было лишь немного направить алгоритм в нужную сторону.

«Эмоциональные состояния могут передаваться другим через эмоциональное заражение, заставляя людей неосознанно испытывать те же эмоции. Эмоциональное заражение происходит без непосредственного взаимодействия между людьми (достаточно контакта с другом, выражающим эмоцию) и при полном отсутствии невербальных сигналов», — говорилось в исследовании.

Сам факт того, что Facebook тайно, без спроса экспериментировал на живых людях, многих поверг в шок. Оказалось, что соцсеть не только в силах влиять на настроения людей, но и не спрашивает на это разрешения. На самом деле Facebook вправе творить с умами людей что угодно: при регистрации любой пользователь принимает политику использования данных, в которой есть и пункты об экспериментах. Соцсеть это даже не скрывает и любые действия оправдывает «улучшением сервиса».

\textbf{Сила одного}

Но изменение настроения миллионов людей — не единственный видимый эффект от использования соцсетей. Как оказалось, те напрямую влияют на психическое здоровье пользователей. В 2014 году группа исследователей из университетов Пало-Альто и Хьюстона во главе со специалистом из университета Дюкейна Май-Ли Стирс изучила связь активного использования Facebook с симптомами депрессии.

Специалисты выяснили, что пользователи Facebook чувствовали подавленность, сравнивая себя с другими людьми. Сперва ученым удалось выявить связь между временем, проведенным в соцсети, и появившимися у опрошенных депрессивными симптомами. Оказалось, что чем больше времени человек проводит в телефоне, тем хуже он себя чувствует.

\begin{fancyquotes}
    Это не значит, что Facebook вызывает депрессию, но депрессивные настроения и трата времени на сравнение себя с другими в Facebook, как правило, идут рука об руку

    \begin{flushright}
        говорится в исследовании
    \end{flushright}
\end{fancyquotes}

Ситуация \ex{усугубляется}{is aggravated} тем, что в интернете люди не рассказывают о повседневной жизни и рутинных делах. Они делятся радостью, достижениями, красивыми видами. Все это дает понять, что наша жизнь по сравнению с чужой серая и неинтересная.

Люди, у которых и без соцсетей могут быть проблемы, при просмотре ленты часто чувствуют себя одинокими и ненужными. Это лишь подталкивает их к изоляции.

«Исследования показывают, что постоянно сравнивать себя с другими вредно для психики. Слишком частое сравнение явно приводит к ухудшению эмоционального состояния», — объясняла один из авторов исследования.

\begin{center}
    \includegraphics[width=0.75\textwidth]{img/google-facebook-experiments.png}
\end{center}

Однако соцсетям совсем не нужна стабильная психика пользователей: им важно, чтобы те оставались онлайн и приносили прибыль. «Такие компании, как Google и Facebook, постоянно проводят на пользователях эксперименты. Благодаря им они точно знают, как заставить людей поступать так, как они хотят. Это банальная манипуляция», — говорит бывший менеджер Facebook Сэнди Паракилас. По его словам, аудитория современных соцсетей — это подопытные животные. И опыты над ними проводятся не для того, чтобы разработать лекарство от рака или помочь голодающим детям. «Мы просто зомби. Они хотят, чтобы мы смотрели больше рекламы. Так они заработают больше денег», — заключает он.

\textbf{В FOMO верующие}

В 2018 году команда ученых провела еще один эксперимент над 143 студентами Университета Пенсильвании, который доказал прямую причинно-следственную связь между использованием соцсетей и ощущением тревоги и одиночества. Испытуемых разделили на две группы, предварительно замерив их самочувствие и ментальное состояние по семи критериям. Первой группе разрешили пользоваться соцсетями без ограничений, вторые же сидели в Instagram, Facebook и Snapchat лишь 10 минут в день. Спустя три недели у первой группы ничего не изменилось, а вот члены второй группы почувствовали значительные улучшения в настроении.

Ученые уверены: помимо постоянного сравнения себя с другими использование соцсетей вызывает FOMO (fear of missing out), или синдром упущенной выгоды. Страх пропустить что-то важное и ощущение, что жизнь пройдет мимо, стали базовыми для большинства пользователей. Результаты исследования говорят, что без последствий для психики в соцсетях можно проводить не больше 30 минут в день. Но далеко не каждый пользователь способен выдержать это ограничение.

\begin{center}
    \includegraphics[width=0.75\textwidth]{img/30-minuty-v-den.png}
\end{center}

Оксфордский словарь определяет FOMO как «беспокойство о том, что потрясающее или интересное событие может в настоящее время происходить в другом месте; часто вызвано постами, увиденными в социальных сетях». Специалисты утверждают, что этот страх часто идет об руку с непреодолимым, почти маниакальным желанием оставаться в курсе всего, что происходит с знакомыми. Первое масштабное исследование FOMO было опубликовано еще в 2013 году, но тогда специалисты посчитали, что одержимость соцсетями — не причина, а следствие одиночества и замкнутости. Более поздние исследования показывают, что все ровно наоборот.

Еще восемь лет назад больше половины людей в опросе признавались, что боятся пропустить в соцсетях что-то важное, и поэтому периодически их проверяют.

\begin{fancyquotes}
    Каждый третий был готов бросить курить, но только не удалиться из соцсетей. Парадоксально, но многие хотели бы устроить себе «каникулы» или «детокс» от соцсетей, но не рискуют этого делать.
\end{fancyquotes}

Недавние исследования говорят о том, что синдром упущенной выгоды испытывают люди всех возрастов, а не только подростки. Исследователи из года в год настоятельно советуют ограничить использование соцсетей, а в случае с серьезными улучшениями эмоционального состояния — вообще удалить аккаунты. Но, судя по тому, что число пользователей и время пребывания в соцсетях постоянно растут, к этим советам никто не прислушивается.

\textbf{Нездоровое отношение}

Эволюция способствовала тому, что человека волнует мнение о нем ближайшего окружения, его семьи и близких людей. Такое «общественное» мнение — это выработанный тысячелетиями рычаг давления на человека, который позволяет привить нормы поведения и морали. Но соцсети довели эту установку до абсурда: люди физически не готовы знать, что думают о них тысячи и миллионы подписчиков и случайных незнакомцев. Пользователи не могут принимать общественное одобрение или порицание круглые сутки, этого не способна выдерживать никакая психика.

По словам социального психолога Джонатана Хайдта, в наибольшей опасности оказалось поколение Z — дети, рожденные в 1996 году и позднее, которые попали в социальные сети в детстве и подростковом возрасте.

\begin{fancyquotes}
    Молодежь коротает все свободное время в смартфоне, а значит, целое поколение становится более тревожным, депрессивным и изолированным.
\end{fancyquotes}

Последнее подтверждает статистика: среди «зумеров» число тех, кто хотя бы раз ходил на свидание или имел романтические отношения, стремительно падает.

При этом среди американских подростков наблюдается гигантский скачок уровня депрессии и тревожности. Он начался в 2011-2013 годах — ровно тогда, когда социальные сети перенеслись в мобильные телефоны. Уровень селфхарма (умышленное причинение вреда своему телу), ранее стабильный на протяжении многих лет, резко вырос на 62 процента среди девочек-подростков и на 189 процентов среди девочек 10-12 лет. Тот же паттерн наблюдается и в статистике самоубийств: рост почти на 70 процентов среди подростков (15-19 лет) и почти в полтора раза у девочек младшего возраста — хотя исторически они никогда не входили в зону риска.


Не последнюю роль в этом сыграли фильтры для фото, которые есть практически в любом приложении. Самые разнообразные фильтры встречаются в Instagram — соцсети, которая состоит из фотографий. На заре создания приложения с помощью фильтров можно было добавить к фото рамку или эффект выцветшего снимка. Но сейчас их используют для совсем других целей: выпрямить нос, разгладить кожу или увеличить губы.

\begin{center}
    \includegraphics[width=0.75\textwidth]{img/samoubiistva-data.png}
\end{center}

Это крайне негативно влияет на психику пользователей, а особенно — подростков. Так, по данным исследования Бостонского медицинского центра (BMC) в Массачусетсе, сама возможность довести свою внешность до мнимого «совершенства» на картинке вызывает недовольство внешним видом в реальной жизни. А это, в свою очередь, породило гигантский скачок спроса на пластику. В 2013 году среди пациентов пластических клиник в США лишь 13 процентов хотели изменить внешность так, чтобы она была больше похожа на отредактированное селфи. В 2017 году со своей фотографией, обильно покрытой фильтрами, к хирургам приходили уже 55 процентов пациентов.

\begin{center}
    \includegraphics[width=0.75\textwidth]{img/social-networks-more-data.png}
\end{center}

В зоне повышенного риска оказываются молодые люди и подростки, которые и без того часто излишне критичны к собственной внешности. На фотографиях они пытаются подчеркивать скулы, выпрямляют и укорачивают носы, увеличивают глаза, и все это влияет на самооценку и может вызвать телесное дисморфическое расстройство (body dysmorphic disorder, BDD). Телесная дисморфофобия проявляется как чрезмерная озабоченность кажущимися внешними недостатками, и ее жертвы делают все возможное, чтобы их скрыть или исправить, даже с вредом для здоровья. Люди с таким расстройством часто делают не одну, а несколько пластических операций. Заболевание относят к обсессивно-компульсивному спектру. Ученые из Бостонского медицинского центра прямо говорят, что одна из причин BDD — селфи.

\begin{fancyquotes}
    Распространенность этих отфильтрованных изображений может заставить человека чувствовать себя неадекватно из-за того, что в реальном мире он выглядит по-другому, и может даже действовать как спусковой крючок и привести к расстройству (BDD)

    \begin{flushright}
        Ученые из Бостонского медицинского центра
    \end{flushright}
\end{fancyquotes}

Исследования показывают, что активнее всего свои селфи редактируют те подростки, которые недовольны своей внешностью. Более того, пользователи с уже диагностированным расстройством в поисках одобрения чаще других показывают глянцевую версию себя в соцсетях. Телесная дисморфофобия крайне распространена: ранее она поражала только одного из 50 человек, но число таких диагнозов растет с каждым годом. Алгоритмам же выгодно, чтобы пользователи активно пользовались фильтрами или зависали над фотографиями «идеальных», по их мнению, людей, поскольку этот процесс затягивает.

В 2018 году британский врач-косметолог Тиджион Эшо впервые ввел термин Snapchat-дисморфия (Snapchat dismorphia). Так специалист описал запрос клиентов на превращение в «цифровую» отфильтрованную версию себя с помощью хирургии. В основе этих желаний — та же дофаминовая игла, наркотическая зависимость от соцсетей, появляющаяся из-за необходимости одобрения. Гладкая кожа, белые зубы, правильные черты лица — такое «отфильтрованное» изображение наберет больше лайков и комментариев, чем обычное. «Опасность заключается в том, что это не просто ориентир, это становится образом, как пациент видит себя. Он хочет выглядеть точно так же, как этот образ», — говорит Эшо. Он обычно отказывает в любом вмешательстве пациентам, одержимым фильтрами. Однако таких хирургов, как он, мало: большинство берется за переделку лица, даже если это опасно и не принесет пациентам желаемого счастья.

\textbf{Сладкая ложь}

И если Facebook и Instagram — площадки, в которых в основном обитают так называемые миллениалы (люди, рожденные в период с 1981 по 1996 год), то более молодое поколение Z давно освоило принципиально новую площадку TikTok. Соцсеть была запущена в 2016 году, но взрывную популярность начала набирать в конце 2019-го. В январе 2021 года число активных пользователей приложения достигло 689 миллионов человек.


TikTok неслучайно называют соцсетью для молодежи: больше половины (62 процента) ее американских пользователей в возрасте от 10 до 29 лет. Основа соцсети — короткие ролики, которые, как правило, длятся в районе 15 секунд. И в этом ее главная ловушка: видео настолько затягивают, что пользователь TikTok проводит в нем в среднем 52 минуты. 90 процентов людей открывают приложение каждый день.


Феномен TikTok кроется в его алгоритмах. Здесь неважно, сколько у пользователя подписчиков, есть ли у него армия фанатов и сколько сил было потрачено на ролик. Важно только одно — виральность. Лента рекомендаций у каждого пользователя разная и подбирается в соответствии с его вкусами и интересами. Среди пользователей TikTok существует шуточная градация «уровней» соцсети: при первом запуске приложение показывает ролики «для обычных смертных» — тиктоки популярных блогеров, знаменитостей и контент с миллионами лайков. Алгоритм нужно тренировать: активно лайкать и отмечать то, что не понравилось. Со временем он пугающе точно подстраивается под конкретного человека. Существуют отдельные «уровни» для любителей вязания, абстрактных шуток или людей со средним размером одежды. Все они могут набирать миллионы лайков и просмотров и никогда не пересекаться.

\begin{fancyquotes}
    Такой точечный подход к вовлечению пользователей — золотая жила для любого интернет-сервиса.
\end{fancyquotes}

\begin{center}
    \includegraphics[width=0.75\textwidth]{img/tiktok.png}
\end{center}

Секрет рекомендаций еще в 2015 году разгадали в Netflix: 80 процентов зрителей сервиса охотно продолжали смотреть то, что предлагал изучивший их вкусы алгоритм. TikTok же эксплуатирует эти механизмы намного активнее и глубже. «Вы просто пребываете в приятном дофаминовом состоянии. Это происходит почти гипнотически, вы просто смотрите и смотрите в экран», — так объяснила процесс потребления контента в TikTok профессор Университета Южной Калифорнии доктор Джули Олбрайт.

Здесь принцип бесконечного и быстрого стимулирования легкодоступного «счастья» доведен до предела: просто так закрыть приложение и вернуться в реальную жизнь невозможно. «Пять минут в TikTok равны часу в реальной жизни», — горько шутят исследователи феномена соцсети. Доктор Олбрайт подтверждает этот тезис: концентрация внимания снижается, и из-за этого появляется эффект сжатия времени.

По ее словам, молодежь уже сейчас демонстрирует последствия таких деформаций. К примеру, как-то ее студентка рассказала ей, что твердо решила посвятить себя написанию песен. Но с одним условием: если за три месяца у нее не получится прославиться, то придется выбрать другую цель в жизни.

\begin{fancyquotes}
    Такой подход старшему поколению кажется абсурдным, ведь чтобы достичь такой цели, в их понимании, нужно потратить немало сил и времени.
\end{fancyquotes}

Знакомый профессор Олбрайт поделился с ней историей, которую оба сочли очень показательной. Преподаватель спросил у студентов, что они планируют делать ближайшие пять лет после выпуска. «Студенты смотрели на него как на умалишенного. Пятилетний план? О чем ты говоришь, это же целая вечность!» — вспоминает она.

Пандемия коронавируса, развернувшаяся в 2020 году, стала первым мощным историческим событием для зумеров, которое полностью перевернуло их жизнь и взгляд на будущее. «Пандемия определит это поколение. Это первое потрясение, которое они испытали на пути к взрослой жизни или в момент вступления в нее», — считает президент и ведущий исследователь Центра кинетики поколений Джейсон Дорси.

И зумеры действительно испытали на себе последствия мирового локдауна: более половины поколения Z в возрасте от 18 до 23 лет или кто-то из их семьи потеряли работу или столкнулись с сокращением зарплаты. У миллениалов этот показатель тоже высок — среди них 40 процентов почувствовали, как их будущее пошатнулось.

Потеря навыков общения, психологическая зависимость от смартфона, неадекватное восприятие себя, тревожность и депрессия — всем этим зумеров и миллениалов «одарили» алгоритмы соцсетей. Самое образованное поколение в истории человечества оказалось не готово к глобальным потрясениям. Новый мир в ближайшие годы не стабилизируется, и над будущим нависла серьезная угроза. Соцсети им явно не помогут, но зато с радостью заработают на их горе и страхах.