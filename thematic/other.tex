\chapter{Другие статьи}

\section[Пожар в бывшем здании НИИ «Платан»*]{Пожар в бывшем здании НИИ «Платан» во Фрязино полностью потушен*}

\textit{Источник: \url{https://www.kommersant.ru/doc/6790353}}

\textit{Пожарные полностью ликвидировали возгорание в здании бывшего НИИ «Платан» во Фрязино. Об этом сообщили в пресс-службе МЧС России.}

Пожар начался вчера вечером и распространился на площади 5 тыс. кв. м. Открытое горение ликвидировали только ближе к ночи. По последним официальным данным, погибли шесть человек.

При пожаре полностью выгорели семь этажей, обвалились межэтажные конструкции. Замначальника главного управления МЧС по Подмосковью Алексей Логинов говорил, что находившиеся в здании во Фрязино люди не смогли эвакуироваться после серии взрывов и распространения огня. В результате люди были вынуждены прыгать из окон.

Следственный комитет России в качестве возможной причины пожара назвал неисправность проводки. Раньше в здании был головной офис оборонного НИИ «Платан». На данный момент там размещались коммерческие и производственные структуры. Возбуждено уголовное дело по ч. 3 ст. 109 УК РФ — причинение смерти по неосторожности двум и более лицам.

\textit{Источник: \url{https://www.kommersant.ru/doc/6790710}}

\textbf{Спасатели и правоохранители восстанавливают картину пожара во Фрязино.}
Стали известны некоторые подробности пожара, произошедшего в понедельник в подмосковном Фрязино, где горело здание, в котором в свое время располагался оборонный НИИ «Платан». По предварительным данным, причиной ЧП стало возгорание газового баллона в помещении, арендованном одной из коммерческих структур. Напомним, по неокончательным данным, при пожаре погибли восемь человек, двое из которых пытались спастись, выпрыгнув из окон.

По данным источников «Ъ» в правоохранительных структурах, о возгорании в службу 112 сообщил житель дома, соседствующего с восьмиэтажным зданием, в котором до 1990-х находился оборонный НИИ «Платан» (Заводской проезд, д. 2). Заявитель сообщил, что видит дым, выходящий из окна шестого этажа.

Правоохранители установили, что здание общей площадью 19200 кв. м принадлежит предпринимательнице Татьяне Белоусовой, но находится по решению суда в управлении ООО «Исприн» (по данным СПАРК, бизнесвумен в 2017 году была признана банкротом), причем часть помещений сдаются в аренду.

По предварительным данным, возгорание произошло на пятом этаже в помещении ООО «Неофуд ГМБХ РУС», осуществляющего торговлю оборудованием для производства пищевых продуктов, напитков и табачных изделий. Начался пожар в 14:39, был локализован в 19:41, проливка продолжалась до вторника. Площадь пожара составила 5 тыс. кв. м. Во время ЧП взорвался газовый баллон, в результате обрушились перекрытия между этажами.

\begin{fancyquotes}
    По предварительным данным, во время пожара в здании находились девять человек.
\end{fancyquotes}

Спасателям удалось с помощью лестницы эвакуировать 34-летнего Алексея Комягина, которого скорая с диагнозом «токсическое действие продуктов горения» доставила в ГБУЗ МО «ЦГБ им. М. В. Гольца». Двое мужчин пытались спастись, выпрыгнув из окон шестого и седьмого этажей, но погибли.

Пока предварительной причиной ЧП считается воспламенение газового баллона.

Как сообщал областной главк МЧС, в тушении огня были задействованы 140 спасателей, 51 единица техники и 2 вертолета Ка-32 Московского авиационного центра. Порядок вокруг места событий обеспечивали 25 сотрудников полиции.

Следственный отдел по городу Щелково ГСУ СКР по Московской области возбудил дело по ч. 3 ст. 109 УК РФ (причинение смерти по неосторожности двум и более лицам).



\clearpage

\section[Каким знакам зодиака повезет]{Согласно астрологам: каким трем знакам зодиака особенно повезет на этой неделе?}

\textit{Кто из знаков-храбрецов способен противостоять текущим астрологическим обстоятельствам и готов сам «переписать» судьбу, если она его не устраивает? Звезды вам благоволят!}

\textit{Источник: \url{https://www.elle.ru}}

\textbf{Телец}

Даже несмотря на сложное ретроградное время, именно \ed{Тельцы}{телец}{taurus (\textit{horosc.})}, согласно гороскопу немецких астрологов ELLE, на этой неделе смогут «воспарить» над многими тревожными событиями. Причина в том, что планета любви, Венера, в данный момент благоволит представителям этого знака и окутывает их, будто мягким облаком розовой ваты. Одинокие люди могут обрести большое счастье - особенно в повседневной жизни, держите глаза широко открытыми.

\textbf{Рак}

Если в отношении других знаков зодиака ретроградный Меркурий лишь вносит неразбериху во все коммуникации, то Раки - единственные, кто на этой неделе получат отличного партнера для переговоров. Наберитесь смелости и сформулируйте свои цели и требования. У вас все получится! Но будьте осторожны: даже если вы можете многого добиться на этой неделе, не забывайте и о других аспектах. Начните думать о том, что вы готовы дать взамен.

\textbf{Козерог}

Рожденные под знаком зодиака Козерог на этой неделе точно знают, как правильно использовать наработанное контакты. И даже ретроградный Меркурий сейчас выступает на вашей стороне - вам без особых усилий удастся извлечь максимум пользы для себя и близких из любой ситуации. Ищите союзников и доверяйте их профессионализму. Впереди неделя, которая может многое изменить навсегда! «Мечта, о которой вы мечтаете в одиночестве, - это всего лишь мечта. Мечта, о которой вы мечтаете вместе, - это реальность».



\section{Откуда взять энергию?}
Отсутствие энергии --- это первый признак приближающихся несчастий и болезней. В Аюрведе говорится, что если человек продвигается в духовной жизни, то это должно быть видно по двум признакам:

\begin{enumerate}[noitemsep]
    \item Человек с каждым днем становится все счастливей и счастливей.
    \item Его отношения с другими людьми улучшаются. Когда мы получаем тонкую энергию...

          Тонкую энергию мы получаем когда:
          \begin{itemize}[noitemsep, label=+]
              \item голодаем,
              \item \explainDetail{выполняем}{выполнять/выполнить}{to perform} дыхательные упражнения,
              \item \explain{уединяемся}{we stay alone},
              \item даем \explain{обет}{vow} молчания, на какое-то время.
              \item гуляем (или просто находимся) по берегу моря, по гор\'{а}м, \explain{созерцаем}{contemplate} красивые пейзажи природы,
              \item занимаемся \explainDetail{бескорыстно}{бескорыстный}{unselfish} творчеством,
              \item \explain{восхваляем}{praise} \explain{достойную}{worthy} личность, за его возвышенные качества и \explain{поступки}{deeds},
              \item смеемся, \explain{радуемся}{rejoice}, улыбаемся от души,
              \item бескорыстно кому-то помогаем,
              \item проявляем \explain{скромность}{modesty},
              \item молимся \explain{перед}{before + instr.} едой,
              \item едим продукты полные \textit{праной} (жизненной энергией) --- натуральные \explain{злаки}{cereals}, каши, \explain{топлённое масло}{ghee}, мед, фрукты, овощи,
              \item спим с 9-10 вечера, до двух часов ночи (в другое время нервная система не отдыхает, \explain{сколько бы}{as much as we may sleep} мы не спали).
              \item получаем \explain{сеанс}{session} хорошего массажа, от гармоничной личности. Или делаем самомассаж.
              \item обливаемся холодной водой, особенно по утрам и наиболее сильный эффект если мы при этом стоим босиком на земле.
              \item \explain{жертвуем}{sacrifice} своим временем, деньгами...
              \item видим за всем божественную \explainDetail{в\'{о}лю}{воля}{will}.
          \end{itemize}
\end{enumerate}


\textbf{Когда мы теряем энергию...}
К потери энергии приводят:
\begin{itemize}[noitemsep, label=--]
    \item уныние, недовольство судьбой, сожаление о прошлом и страх, неприятие будущего,
    \item  постановка и преследование эгоистичных целей,
    \item бесцельное существование,
    \item обиды (обида)
    \item переедание,
    \item бесконтрольное блуждание ума, неумение сконцентрироваться.
    \item когда мы ед\'{и}м жаренную или старую пищу, пищу приготовленную человеком в гневе или испытывающем другие отрицательные эмоции, при использовании микроволновой печи, продукты, содержащие консерванты, химические добавки, выращенные в искусственных условиях, с использованием химических \explainDetail{удобрений}{удобрение}{fertilizer},
    \item поедание пищи лишенной праны --- кофе, черный чай, белый сахар, белая мука, мясо, алкоголь,
    \item еда в спешке и на ходу,
    \item курение,
    \item \explain{пустые разговоры}{(lit.) void discussions},
    \item неправильное дыхание, например, слишком \explainDetail{ч\'{а}стое}{частый}{frequent} и глубокое,
    \item нахождение под прямыми \explainDetail{луч\'{а}ми}{луч}{ray} Солнца, с 12 до 4 дня, особенно в пустыне,
    \item беспорядочные \explainDetail{половые}{полов\'{о}й}{sexual} связи, секс без любви к партнёру,
    \item \explain{излишний}{unnecessary} сон, сон после 7 \'{у}тра, недостаток сна,
    \item \explain{напряж\'{е}ние}{tension; stress} ум\'{а} и т\'{е}ла,
    \item \explain{\'{а}лчность}{greed} и \explain{жадность}{stinginess}.
\end{itemize}

Восточная психология на 50\% состоит из пранаямы --- теории и практики определенных дыхательных техник, которые позволяют человеку быть всегда наполненным жизненной силой (Праной).

Как утверждают современные просветленные учителя йоги набраться праны мы можем через:
\begin{enumerate}
    \item \textbf{Элемент земли.} питаясь натуральной пищей, жить на природе, созерцать деревья, ходить босиком по земле. Недавно я общался с очень известным аюрведическим доктором, \explainDetail{защитившему}{защитить, защитивший (past act.)}{who defends} диссертацию по медицине, он утверждал, что если человек начинает жить на природе, \explain{вдали от}{away from} больших городов, которые \explain{вынуждают}{necessitate} ездить в метро, ходить по асфальту, то у такого человека быстро восстанавливается иммунитет и он начинает жить здоровой жизнью.

    \item \textbf{Элемент воды.} пить воду из колодцев или \explainDetail{ручьев}{(sing.) ручей, ручья, ручью, ручей, ручьём, ручье, (plur.) ручьи, ручьёв, ручьям, ручьи, ручьям, ручьями, ручьях}{brook; creek}. Плавать в реке или море. \explainDetail{Избегать}{избегать}{to avoid} пить \explain{кофеиносодержащие}{containing caffeine} напитки, алкоголь и соду.

    \item \textbf{Элемент огня.} нахождение на Солнце и употребление пищи содержащей Солнечный свет.

    \item  \textbf{Элемент воздуха.} это самый важный элемент получения Праны, через вдыхание чистого воздуха, особенно в горах, в лесу и на берегу моря. Курение и нахождение в местах большого \explain{скопления}{?} людей, лишает человека праны.

    \item \textbf{Элемент эфира.} культивируя \explain{позитивное}{положительный} мышление, \explainDetail{доброт\'{у}}{доброт\'{а}}{kindness}, хорошее настроение.
\end{enumerate}


И этот \explain{уровень}{level} считается базовым.

Ибо даже, если человек живет на природе и правильно питается, но при этом ходит раздражённый и злой, то наоборот, \explain{изл\'{и}шек}{surplus} Праны еще быстрее разрушит его.
С другой стороны гармоничный человек, то есть добродушный,
\explain{бесстрашный}{fearless}, может довольно долго протянуть в городе,
если он вынужден там жить.
Но даже такому человеку н\'{у}жно следить за питанием
и периодически «вырываться» на природу.

У нас каждую секунду есть выбор --- светить миру, приносить своей жизнью благо и счастье окружающим, улыбаться, \explainDetail{заб\'{о}титься}{(по)заб\'{о}титься (заб\'{о}чусь, заб\'{о}{}тишься, заб\'{о}- тятся)}{to care about} о других, служить бескорыстно, жертвовать, сдерживать низшие \explain{побуждения}{drives (сдерживать низшие побуждения: to control/restrain the lower drives)}, видеть в каждом человеке Учителя, в каждой ситуации видеть Божественное \explain{провидение}{providence}, которое создало эту ситуацию \explain{для того что бы}{so that; in order to} нас чему то научить, благодарить ...

Либо предъявлять претензии, \explain{обижаться}{to take offence}, \explain{жаловаться}{to complain}, \explain{завидовать}{to envy}, ходить с клинообразным выражением лица, погрузиться в свои проблемы, в зарабатывание денег для того что бы потратить их на \explain{удовлетворение}{satisfaction} чувств, \explain{проявлять}{to show; to demonstrate} агрессию.

В этом случае \explain{в независимости от того}{regardless of} сколько у человека денег, он будет несчастный и мрачный. И с каждым днем энергии будет все меньше и меньше. И для того что бы её где-то взять нужны будут искусственные стимуляторы: кофе, сигареты, алкоголь, ночные клубы, выяснение отношений с кем-то. Все это дает вначале подъём, но в итоге приводит к полному разрушению..

Простой регулярный вопрос себе: я свечу миру или поглощаю свет? Может быстро изменить ход наших мыслей и следовательно поступков. И быстро превратить нашу жизнь в красивое яркое \explain{сияние}{shining}, полное любви. И тогда вопросы, где взять энергию уже не \explain{возникают}{emerge}.

\begin{flushright}
    \it Р. Блект
\end{flushright}


Каждый раз, когда кто-то причинил тебе боль, не спеши гневаться. Эта \explain{боль}{(ж) pain; ache (e.g., головн\'{а}я боль: headache)} выпр\'{о}шена тобой у Вселенной. \explainDetail{Неосознанно}{неосознанно}{unconsciously}, ты сам \explain{привлёк}{past tense of привлечь} её.

Причиняющий боль - всего лишь \explain{кукла на верёвочках}{doll on a string (puppet)}, что управляема тобой...

``Будь осторожен с желаниями'' --- слова Вечности. Что несут они в себе?

Когда ты \explain{жаждешь}{to crave for} \explainDetail{исполнения}{исполнение}{fulfilment} одного из своих желаний, ты не задумываешься над тем, что для того, чтобы оно \explainDetail{исполнилось}{исполн\'{я}ть/исп\'{о}лнить}{to carry out}, тебе н\'{у}жно через что-то пройти, чего-то \explain{лишиться}{to lose; to be deprived of; recall that лишать/лишить means to deprive}, что-то приобрести... Как только желание сформировалось и укрепилось в сознании, всё вокруг начинает \explainDetail{перестраиваться}{перестраиваться/перестроиться}{to rebuild; to reorganize; to re-form; to restructure} для того, чтобы оно смогло исполниться. Уходят люди из твоей жизни, что мешают его исполнению, появляются новые, которые пом\'{о}гут, приходят те, что должны научить тебя увидеть дорогу к желанному. Иногда нужн\'{а} сила, без которой не пройти по этому пути, а силу дают боль и трудности. Ты привык к тому, что было ранее и не видишь к чему ведут болезненные изменения вокруг. Но ведь ты хочешь исполнения желаемого? Оно то, чего не было, но то, что должно родиться и это \explainDetail{уберёт}{убирать/убрать}{to clean/tidy; to take away; to remove (убир\'{а}йся отс\'{ю}да!=get lost!)} старое из твоей жизни, что не давало прийти Новому...

Рождение проходит через боль. За Ночью приходит День.

Должно Темноте \explain{сгуститься}{to thicken}, \explain{даб\'{ы}}{so that} Свет силой засиял...

Желая, ты сам, только ты, включаешь механизмы, что начиняют менять жизнь для нового рождения в ней. Ты притягиваешь всё для этого и боль, \explain{в том числе}{including}. Поэтому, помни, тот человек, что причинил тебе боль --- \explain{вызван тобой}{caused by you}. Он - кукла. Не \explain{гневись}{be angry} на него, а благодарностью одари за силу, за помощь в пути к Новому.

\begin{flushright}
    \it Аму Мом
\end{flushright}

Медитация для снятия эмоционального напряжения

Эмоциональное напряжение усиливает существующие в теле мышечные спазмы и повышает тонус нервной системы.

Это усиливает боль физическую и душевную. Нам кажется, что эмоции - это что-то неуправляемое.

Есть методика, выполни ее и забудь про: беспокойство, злость, апатию или другие истощающие эмоции.

\begin{enumerate}
    \item Прими удобное положение или отправься на \explainDetail{прогулку}{прогулка}{walk (n)} в одиночестве

    \item Сидя или гуляя начни \explainDetail{наблюдать}{наблюдать/понаблюдать}{to take care of; to observe; to watch} за своим дыханием, чтобы \explain{переключиться}{to switch} от мыслей к телу

    \item Наблюдая за дыханием \explainDetail{погрузи}{погрузи}{immerse; dip} все свое внимание в тело и почувствуй, где в теле \explainDetail{кроется}{крыться/покрыться}{to be concealed (кроюсь, кроешься, кроются)} напряжение

    \item Не думай, не анализируй и не пытайся понять, просто наблюдай.
          Наблюдай за тем, как начинается вдох и выдох, погружая свое внимание в тело

    \item Когда ты найдешь центр напряжения в теле, собери все внимание в этой части тела

    \item Дыши спокойной и с каждым выдохом \explainDetail{отпускай}{отпускать}{to let go} напряжение,
          если в это время ты \explainDetail{испытаешь}{испытывать/испытать}{to try} какие-то эмоции
          отпускай и их с дыханием

    \item Удели этой практики 10-15 минут и твое внутреннее состояние больше не будет \explain{неуправляемым}{uncontrolled} хаосом.
\end{enumerate}

Улыбаться полезно. Кроме выработки гормона радости для здоровья, это ещё и приятно.
\explainDetail{Искренняя}{искренний/яя/ее/ие}{sincere} улыбка посылает энергию любви,
которая \explainDetail{обладает}{обладать}{to have; to possess; to own} силой, чтобы \explainDetail{согревать}{согревать/согреть}{to warm up} и \explainDetail{исцелять}{исцелять/исцелить}{to heal}. Просто вспомните время, когда вы были \explain{расстроены}{upset} или больны физически, и кто-то, возможно, даже незнакомый, вам искренне улыбнулся --- и внез\'{а}пно вы почувствовали себя лучше.

Не всегда есть желание улыбаться? Упражнение «Улыбка Будды» поможет вам в любое время повысить своё эмоциональное состояние.

Выполняется очень просто. Сначала потренируйтесь перед зеркалом с отрытыми глазами.
Потом закройте глаза и запомните положение \explainDetail{мышц}{мышца}{muscle} лица,
чтобы повторить в любой обстановке.

\begin{enumerate}[noitemsep]
    \item Расположите свои губы так, чтобы \explainDetail{черт\'{а}}{черт\'{a} (ж)}{line},
          разделяющая их, располагалась строго горизонтально (параллельно полу).
    \item Совсем немного приподнимите вверх уголки губ.
\end{enumerate}

Посмотрите для наглядности на фотографию статуи Будды.

А теперь посидите с этой улыбкой хотя бы 5 минут и вы почувствуете ее действие.

«Надевайте» на свое лицо улыбку и вам станет легче жить.

Дополнение.

Очень хорошо с такой улыбкой \explainDetail{осознавать}{осознавать/осознать}{to realize} дыхание.
Следите за дыханием, говоря про себя: «Вдыхаю с улыбкой --- выдыхаю с улыбкой!»



\newpage
\section{Чем женщины отличаются от мужчин?}

\textit{Источник: \url{https://reallanguage.club/}}

\textbf{Note:} C1-level text!

% https://reallanguage.club/russkie-teksty-prodvinutogo-urovnya-s-audio/chem-zhenshhiny-otlichayutsya-ot-muzhchin/

\textit{Автор: Алексей Вайда}

С тех пор как мужчины спустились с деревьев и научились использовать в качестве «весомого» аргумента в споре с сородичами дубину, им удалось многого достигнуть.

Объединяя общины, племена, а затем и народы, мужчины создавали огромные империи и государства. Научные открытия, совершенные сильной половиной человечества, до неузнаваемости преобразили современную жизнь. Отправляясь в путешествия и покоряя все новые и новые земли, мужчины сумели расширить представления о мире до границ целой планеты.

Но на этом устремления мужчин не закончились\footnote{But the aspirations of men did not end there.}. Океанологи исследовали самые глубокие \ex{океанские впадины}{ocean trenches}, космонавты преодолели \ex{притяжение планеты}{gravity of the planet}, археологи заглянули в далекое прошлое, а историки восстановили события «давно ушедших дней». И однажды, осмотревшись вокруг, человек решил, что с него хватит. И стал гордо называть себя \textit{homo sapiens}, т.е. человеком разумным.

Назвав себя «разумным» и даже сумев поверить в это, мужчина в глубине души начал сомневаться, действительно ли это так, ведь до сих пор существует загадка, которую никак не получается разгадать. И имя этой загадки --- женщина.

Та самая мужская логика, которая движет прогрессом и которой мы так гордимся, оказывается беспомощной, как только мы пытаемся с ее помощью \ed{разобраться в}{разобраться в чём-то}{to understand. Ex: умение быстро разобраться в сложных объектах} женщине и понять ее. И более того \ex{здравый смысл}{common sense} становятся нашим слабым и уязвимым местом во взаимоотношениях с прекрасной половиной. Наверное, женщины смогут назвать и еще как минимум одну «ахиллесову пяту» мужчин, но… это уже другая тема.

Тем не менее, несмотря на всю сложность «женского вопроса», мужчинам удалось достичь некоторых успехов.

Но процесс \ed{познания}{познание}{perception, knowledge} женщин оказался очень сложным, ведь известно, что не только мужчины плохо понимают женщин, но и сами женщины часто не понимают самих себя.

Начнем с того, что женщины всегда более объективно оценивают достоинства и недостатки мужчин, чем женщин. Прекрасно объясняет такое \ed{предвзятое}{предвзятый}{biased, prejudiced} отношение высказывание одного ценителя женщин, который заметил, что главной причиною нелюбви женщин друг к другу является мужчина. Можно не согласиться с «причиною \ed{раздора}{раздор}{discord, contention}», но тот факт, что одна подруга никогда не посоветует другой то платье, которое бы ей очень подошло, \ex{наглядно}{by visual demonstration, visually, graphically, clearly} подтверждает такое мнение.

Когда речь заходит об отношении к недостаткам друг друга, мужчины и женщины ведут себя совершенно по-разному. Мужчинам во всем, что касается женских слабостей, свойственна некоторая \ex{снисходительность}{condescension \textit{or} leniency} и \ex{насмешливость}{mockery}.

А вот прекрасная половина склонна не только судить мужчин, но и осуждать со всей строгостью. И это понятно, ведь женщина \ex{наделена}{is endowed} правом выбирать, и ей крайне необходимо умение оценивать будущего претендента «на руку и сердце».

\ed{Нелишним}{нелишний}{not useless, needed, necessary} будет упомянуть и о \ed{злопамятности}{злопамятность}{rancor, vindictiveness}. Мужчины обычно все прощают и забывают. А женщины напротив --- прощают, но никогда не забывают, как кто-то нелестно высказался в их адрес. Поэтому мужская «осторожность» в суждениях о женщинах становится вполне объяснимой.

Не секрет, что внешность имеет для женщин первостепенное значение. И причиной тому служат мужчины, которые как бы высоко ни ценили личные качества женщин, сначала обращают внимание именно на женскую красоту. Хотя представления мужчин об идеальной внешности женщины могут кардинально отличаться, все мы придерживаемся \ed{негласного правила}{негласное правило}{unspoken rule, unwritten rule, oral law}  «здоровая женщина --- красивая женщина». Когда свой выбор делает женщина, критерий «внешней привлекательности» уже не играет для нее такой роли.

Чтобы понять, насколько внешность мужчины для нее «второстепенна», достаточно вспомнить хорошо известную сказку «Красавица и чудовище», а также припомнить несколько наглядных примеров из жизни близких, друзей или знакомых.

Почему так вышло? Все достаточно просто. Женщина --- это, прежде всего, мать, а мужчина --- \ex{добытчик}{(\textit{colloq.}) breadwinner, (good) earner (someone who is able to make good money)}. Поэтому мужчинам и нравятся красивые женщины, ведь их красота подразумевает (хотя мы и не всегда об этом думаем) здоровье будущих детей. Женщинам нравятся мужчины, которые способны и защитить семью, и обеспечить ее всем необходимым.

Таким образом, получается, что красота мужчины служит для женщин лишь приятным дополнением к выше перечисленным качествам. Всем женщинам прекрасно известна склонность мужчин пребывать в двух крайних состояниях, а именно физической активности и полной пассивности. А учитывая, что сами женщины отдают предпочтение более ровным и постоянным нагрузкам, подобное «поведение» мужчин часто встречает непонимание со стороны слабого пола.

Но прежде чем нас судить, давайте заглянем в историю. На протяжении тысячелетий мужчины вели охоту и войны, а женщины были «хранительницами \ed{очага}{очаг}{hearth}» и занимались всеми домашними делами. Так из века в век вырабатывались наши собственные ритмы жизни и складывались наши предпочтения к занятиям разного рода. Поэтому мужчины выбирают профессии, связанные с риском, непредсказуемостью, духом \ed{соперничества}{соперничество}{    rivalry, competitiveness, competition} и борьбы (т.е лишенными однообразия).

Современными исследованиями даже доказано, что у мужчин есть врожденная потребность \ex{колотить}{to knock, to strike, to bang}, бить, забивать, стрелять, то есть использовать «взрывную» энергию. Женщины, обладая терпеливостью и \ed{усидчивостью}{усидчивость}{perseverance}, способны выполнять тонкую и монотонную работу с заданной программой действий (т.е. достаточно предсказуемую).

Существует предположение, что современный язык придуман мужчинами. Возможно это и так, но в совершенстве речью овладели как раз таки женщины. Но общение, призванное сближать людей, по какой-то причине стало чуть ли не главным «камнем преткновения» во взаимоотношениях сильной и слабой половины человечества.

Если смотреть на данную проблему с точки зрения науки, то она объяснит все наши «недопонимания» различием в функционировании мозга. \ed{Определенно}{определённо}{definitely}, что во время общения у мужчин задействована лишь небольшая область одного из полушарий мозга. В то время как у женщин в «работу» включаются более обширные участки, и при этом сразу обоих полушарий.

Использование всего мозга дает женщинам огромное преимущество в общении с мужчинами, но одновременно вносит путаницу в «женские умы», когда требуется отличить правую сторону от левой. Около половины женщин не способны мгновенно сказать, какая рука левая, а какая правая. Так что если женщина показывает поворот не в ту сторону, это вовсе не говорит о ее \ed{рассеянности}{рассеянность}{absent-mindedness}.

Но вернемся к нашему языку. Для женщин важен сам процесс общения, а для мужчин --- конкретный результат. Если мужчина думает молча, то женщина предпочитает \ex{размышлять}{to ponder, to muse, to meditate, to think (over), to reflect} вслух. Мужчины больше любят говорить о своих успехах, женщины --- о своих неудачах.

Даже этих нескольких примеров достаточно, чтобы понять, почему средняя потребность женщины составляет 6000-8000 слов в день, а мужчины ограничиваются 2000-4000 словами.
Да, женщины «болтливы», но только если смотреть с точки зрения мужчин.
И в завершение хотелось бы упомянуть о «мужской» и «женской» дружбе.

Считается, что только мужская дружба бывает настоящей. Что же касается женщин, то дружба между ними редко бывает прочной и долговечной. То же можно сказать и о так называемой «дружбе» мужчины и женщины, которая существует лишь на словах. Ведь в таких отношениях мужчина рассматривается женщиной в лучшем случае как «запасной» вариант.

Причин подобного положения вещей множество, это и недоверие женщин к друг другу, и их постоянное соперничество за внимание, и их естественное стремление быть «единственной». Но главную причину «равнодушия» женщин к дружбе заметил \ex{Ларошфуко}{Франсуа де Ларошфуко (1613-1680) --- французский политический деятель, принадлежавший к аристократическому роду}: Женщины оттого так \ed{безразличны}{безразличный}{indifferent} к дружбе, что она кажется им \ed{пресной}{пресный}{(here) tasteless, insipid} по сравнению с любовью.

Что ж, трудно с ним в этом не согласиться...

\newpage
\section{В ход идет все — гадания, карты Таро и приметы}

\textit{После начала СВО россияне стали больше верить в магию. Зачем им это?}

\textit{Источник: \url{https://lenta.ru/articles/2023/09/28/magia/}}

В начале года россияне стали намного чаще покупать карты Таро для гадания, а участников СВО регулярно замечают с языческими оберегами на форме. Чего ищут жители страны в эзотерических практиках и помогают ли они, «Лента.ру» спросила у психотерапевта и специалиста психологической платформы Alter Елизаветы Крюковой, а еще и у самих россиян.

\textbf{--- «Лента.ру»: Россияне стали все больше покупать карты Таро. С чем это может быть связано?}

--- Елизавета Крюкова: Если рассуждать в рамках магического мышления и психологии, то это связано с увеличением уровня \ed{неопределенности}{неопределённость}{uncertainty}. Когда он повышается, люди ищут способ как-то преодолеть его. И тут идет в ход все — гадания, карты Таро, \ed{приметы}{примета}{omen, sign, mark}...

Жизнь -- это вообще штука неопределенная и непредсказуемая. Сегодня прогнозы во многих областях можно построить с помощью высшей математики и теории вероятностей, но это довольно сложно и доступно далеко не всем. Гораздо проще использовать какие-то \ed{житейские}{житейский}{everyday} способы, которые появились сотни лет назад.

Наука тогда не была развита, и люди обращались к различным способам предсказаний именно с этой целью — снизить неопределенность и обеспечить благоприятное развитие событий. Шаманские знания, \ed{заклинания}{заклинание}{spell, incantation}, ритуалы и другие магические действия --- это все тоже были попытки повлиять на будущее.

\textbf{--- Исследование показало, что на обычные гадальные карты спрос значительно снизился. Как тогда объяснить эту тенденцию?}

Это интересное наблюдение. Возможно, единственная причина тут — это маркетинг. Сейчас в соцсетях появилось очень много гадалок, которые предлагают свои услуги и делают расклады именно на Таро.

\textbf{--- От чего тогда зависит выбор метода предсказания, к которому обращается человек?}

Он происходит таким же образом, как мы покупаем \ex{хлопья}{(corn) flakes} на завтрак, — довольно произвольно. Кому-то \ed{подсказала}{подсказывать/подсказать}{to prompt; to suggest} подруга, кто-то увидел рекламу в социальных сетях, кто-то раньше где-то услышал и подумал, что именно этот способ является рабочим. С точки зрения психологии важен не столько метод, как то, для каких целей он используется, а именно для предсказания будущего и снятия неопределенности.

\textbf{--- Вы упомянули о магическом мышлении. Что это такое?}

Это способ мышления, в котором путаются причинно-следственные связи. С точки зрения психологии это можно назвать когнитивным \ed{искажением}{искажение}{distortion}, то есть ошибкой восприятия реальности.

\begin{fancyquotes}
    Приведу простой пример: допустим, кто-то положил пять рублей под пятку и сдал экзамен. Человек сделал вывод, что этот ритуал приносит удачу, и решил, что будет и дальше на него полагаться. Это типичная ошибка мышления, при которой причина не связана со следствием
\end{fancyquotes}

Без сомнения, при магическом мышлении присутствует и нарушение восприятия реальности. Вещи не соответствуют действительности, люди не могут их доказать, но верят в них.

\textbf{--- Какие факторы влияют на развитие магического мышления?}

Основной — это уровень образования и интеллекта. Чем лучше образован человек, тем сильнее у него развито критическое мышление. А если мы будем подвергать критическому анализу все магические вещи, то быстро убедимся, что им нет доказательств.

Каждый человек в детстве мыслит магическим образом. Ребенок до трех-четырех лет считает себя центром вселенной и уверен, что он является причиной всего.

Допустим, он едет в машине и думает, что солнце бежит за ним, хотя оно находится на месте. Ребенок пока просто не умеет представлять себя на месте других людей. Он считает, что мир вертится вокруг него.

\begin{fancyquotes}
    Это стадия детского эгоцентризма, и она абсолютно нормальна для формирования человеческой личности. Психологи советуют больше разговаривать с детьми и объяснять им все реалистично, рассказывать, как все происходит
\end{fancyquotes}

Если к тому же ребенок растет в среде, где распространены приметы и предсказания, а мама читает гороскопы и
утверждает, что малыш упрямый потому, что родился \ed{Тельцом}{телец}{taurus (\textit{horosc.})}, то до
какого-то времени ребенок будет точно в это верить, пока у него не
\ed{разовьётся}{развиться}{to develop (разовьюсь, разовьёшься, разовьются)} критическое мышление.

Во взрослом возрасте у нас больше шансов самостоятельно сделать выбор, во что нам верить.

\vspace*{1em}
\begin{center}
    \Large «Людям страшно столкнуться с жизнью»
\end{center}
\vspace*{1em}

\textbf{Николай, принимал участие в СВО: } \textit{«До спецоперации я интересовался эзотерикой на \ed{любительском}{любительский}{amateur} уровне. Общался с людьми, которые разбираются в этой теме, пытался \ed{сопоставлять}{сопоставлять/сопоставить}{to compare with} символы. Когда собирался на спецоперацию, то взял с собой кусочек \ed{коры}{кор\'{а} дерева}{tree bark} от дерева неподалёку от дома. Просто решил, что это будет мой личный символ на удачу.
В части, где я оказался, нам еще выдали \ed{нательные крестики}{нательный крест}{pectoral cross}. Это немного придавало сил\footnote{It gave me a little strength}. Потом я узнал, что многие мои знакомые молились за меня. Но не только в церкви, кто-то обращался даже к эзотерикам. Мне было приятно, что близкие обо мне беспокоятся. Сам я в подобные вещи не верю, но кусочек коры я до сих пор не выбросил, хотя уже и не таскаю его везде с собой».}

\textbf{--- На спецоперации довольно много военных, которые используют в качестве нашивок языческие символы и увлекаются эзотерикой. С чем это связано?}

Есть такой известный афоризм: «В \ed{окопах}{окоп}{trench} атеистов не бывает». Он значит, что в случае какой-то серьезной опасности человек будет надеяться на чудо и хвататься за любую ниточку. Иногда все, что остается в таком случае, — это вера в высшие силы, потому что больше не на что полагаться. Это абсолютно нормальный механизм защиты психики.

\textbf{--- В социальных сетях можно найти множество групп по гаданиям. В том числе туда часто пишут родственники \ed{пропавших б\'{е}з вести}{пропавший б\'{е}з вести}{missing (official term)} на СВО солдат, пытаются узнать, что случилось с их мужьями и сыновьями. Что ими движет? }

Здесь разговор снова заходит о снятии неопределенности. Когда мы теряем связь с важным для нас человеком, мы, естественно, испытываем тревогу. Родственнику важно знать, что конкретно произошло, тогда тревоги будет меньше.

Даже если гадалка напишет, что солдат погиб, для близких это все равно легче, чем жить в подвешенном состоянии. Мозг здесь работает таким образом, что человек может отгоревать, пережить свою утрату и двигаться дальше.

Хотя, конечно, большинство людей в этой группе ждут хорошего исхода и до конца в него верят.

\textbf{--- Может ли такой тип мышления быть связан с травмами в прошлом? Допустим, ребенок пережил насилие в детстве, он вырастает и думает, что из-за этого он теперь обладает уникальными способностями и может справляться с трудностями лучше, чем остальные.}

Да, это тоже распространенная история. Тут как раз в тему пример о кармических уроках. Например, у девушки было три парня, все они относились к ней не очень хорошо, и она думает, что это ее кармический урок, который направлен на то, что она что-то должна поменять в себе.

\begin{fancyquotes}
    Тут, опять же, людям страшно столкнуться с жизнью, ее неприятностью и непредсказуемостью. А объяснение о кармическом уроке делает ее легче. Будто тут мы обретаем контроль над ситуацией и можем что-то изменить
\end{fancyquotes}

Кроме того, магическое мышление может стать и способом борьбы с травмой. Если человеку это помогает — почему бы и нет.

\vspace*{1em}
\begin{center}
    \Large «Говорят, что виновата карма»
\end{center}
\vspace*{1em}

\textbf{Анна, увлекается Таро на любительском уровне:} \textit{«Впервые я заинтересовалась раскладами, когда пришла к подруге на день рождения. В тот момент меня игнорировал парень, который мне нравился, я очень переживала и хотела хоть немного понять, ждет ли нас какое-то совместное будущее. В итоге карты показали, что нам не стоит быть вместе. Так оно и вышло, чему я очень рада. Сейчас я часто захожу на сайты с онлайн-гаданиями и смотрю расклады на свои нынешние отношения. Обычно я делаю это, когда меня накрывает волна тревоги за партнера или когда я начинаю переживать, что он меня разлюбил. Если расклад позитивный, то я немного успокаиваюсь, но если выходит что-то негативное, то я начинаю \ex{нервничать}{to be(come) nervous} и продолжаю так называемое гадание на других сайтах до тех пор, \ex{пока не}{until (+\textit{св})} получу желаемый результат». }

\textbf{--- Можно ли говорить, что к магическому мышлению больше склонны люди с высоким уровнем тревожности?}

Не могу привести конкретные статистические данные по этому поводу, но \ex{в общем и целом}{overall} да. Поскольку подобные ритуалы направлены на то, чтобы снять тревогу, то чем ее больше, тем вероятнее, что человек будет пытаться с ней справиться таким образом. Если человек не очень сильно переживает, то он не ищет утешения в подобных методах.

Если говорить о магическом мышлении, то тут важен и эмоциональный интеллект. Если человек поймет, для чего он \ex{прибегает}{resorts} к гаданию, увидит связь между этим процессом и тревогой, то у него появится возможность задать себе вопрос, действительно ли это рабочий метод.

\textbf{--- Может быть, магическое мышление — это способ переложить ответственность за свою жизнь на кого-то другого?}

Да, и такой момент присутствует. Люди подобного склада часто говорят, что в их неудачах виновата карма или какие-то поступки из прошлой жизни. Это снимает необходимость саморефлексии и самоанализа.

Например, девушка долго не может забеременеть, поэтому она решает использовать все возможные способы. Идет к тарологу, а он ей говорит, что это все из-за того, что она в прошлой жизни сделала что-то плохое.

\begin{fancyquotes}
    Нет ничего ужасного в том, чтобы верить в подобные теории, если ты параллельно проходишь лечение и психотерапию. Но люди чаще всего в такие моменты просто говорят, что получили свой кармический урок, а значит, делать им больше ничего не придется и можно сидеть сложа руки
\end{fancyquotes}

Порой с реальностью сталкиваться больно и страшно, потому что в таком случае придется ходить по врачам и, возможно, принять решение не иметь своих детей. А магическое мышление освобождает от ответственности.

\textbf{--- Можно ли к магическому мышлению отнести также веру в теории заговора?}

Конечно. Когда в психологии используется этот термин, мы имеем в виду не только то, что связано именно с магией. Речь скорее идет именно о попытке объяснить для себя неконтролируемое и, как я уже говорила выше, снизить тревожность.

\textbf{--- Вы сказали, что уровень магического мышления зависит от интеллектуального уровня. Но в пандемию очень многие авторитетные и умные люди говорили, что людей пытаются массово чипировать.}

Мне все-таки кажется, что большинство людей, которые являются авторитетами в науке, не поддавались на подобные провокации, а пытались опираться на исследования и рациональные \ex{доводы}{arguments}. Думаю, тут все же есть корреляция с интеллектуальным уровнем. К тому же, конечно, сам факт получения высшего образования не всегда свидетельствует о том, что человек достаточно развит.

\textbf{--- Как тогда можно объяснить тот факт, что не так давно российские чиновники увлеклись оккультизмом? И это не единственный случай, когда политические деятели занимались чем-то подобным. Возможно, кроме интеллекта важную роль играет еще и груз ответственности?}

Интересный вопрос. Думаю, тут на человека очень сильно влияет социальное окружение и в целом принадлежность к какой-то группе.
Представим, что человек с критическим мышлением попал в руководящий кружок, все члены которого, допустим, оказались масонами. И если он хочет сохранить свое место и статус, нормально взаимодействовать с этими людьми, то ему приходится выбирать.

\begin{fancyquotes}
    А тут все зависит уже от желания человека и других разных факторов: отчаянности, критического мышления, предрасположенности к магическому мышлению
\end{fancyquotes}

Не стоит забывать и о социальной принадлежности: людям свойственно повторять за окружающими, чтобы не отличаться от них. Это тоже способ справляться с неопределенностью.

\textbf{--- Представляет ли магическое мышление опасность для самого человека и его близких?}

Думаю, да. Оно предоставляет нам нереалистичные данные, и если мы будем верить, что черная кошка приносит нам неудачу, и будем всячески ее избегать, — это может привести к самым неожиданным последствиям.

Есть еще такая вещь, как \ed{самосбывающиеся пророчества}{самосбывающееся пророчество}{self-fulfilling prophecy}. Вспомним ту же счастливую пятирублевую монету на экзамене. Если я ее забыла, то могу думать, что без нее ничего не сдам, и буду неосознанно вести себя таким образом, что все в итоге приведет к неудаче.

Опасным магическое мышление может быть и для здоровья человека. Допустим, если он прибегает к магической медицине. Это отдельный повод для беспокойства, так как результаты могут быть \ed{плачевными}{плачевный}{deplorable}.

У человека диагностируют серьезную болезнь, он испытывает страх и ищет какие-то более простые способы \ed{исцеления}{исцеление}{healing}, а они чаще всего приводят к ухудшению ситуации вместо улучшения. Снова хочется повторить, что нет ничего плохого в том, чтобы совмещать лечение и ритуалы, но не стоит заменять первое вторым.

\textbf{--- Некоторые люди говорят, что они делают расклады ради развлечения. Действительно ли это так или есть еще какой-то мотив?}

Это довольно широкий вопрос. Возможно, кому-то просто нравятся картинки на картах Таро, поэтому он раскладывает их перед собой. Кто-то, может быть, хочет показать, что любой факт можно \ex{притянуть за уши}{to be far-fetched}, а кто-то втайне надеется, что метод все же сработает.

Тут может быть ситуация как с курильщиками, которые говорят, что они курят ради удовольствия и наслаждаются самим процессом, но не страдают зависимостью. Однако это не так.

Человек может себя убеждать, что он не верит в гадание, а сам гадает и находит успокоение в нужном ему результате.

\vspace*{1em}
\begin{center}
    \Large «Нужен язык саморефлексии»
\end{center}
\vspace*{1em}

\textbf{Евгения, самостоятельно изучает карты Таро: } \textit{«Казалось бы, Таро — это просто иллюстрированная колода карт, но она может нести в себе глубочайший смысл и проясняет те ситуации, которые вы не можете видеть в полной картине. Таро дает возможность узнать мотивы других людей, их намерения и, наконец, заглянуть внутрь себя. Самые популярные расклады — это работа и отношения. Это те сферы, вокруг которых крутится почти вся наша жизнь. Но люди с помощью карт могут узнавать разное. Я люблю делать расклады самой себе, а потом соотносить ту информацию, которую я получила, с реальностью. Однажды мне делала расклад одна женщина, и с ее помощью я прямо открыла себе глаза. Она дала мне понять, в каком направлении мне лучше двигаться. До этого я не понимала, чего хочу от жизни. Мысли о том, что я себя потеряла, были просто ужасны! Она помогла приоткрыть эту завесу». }

\textbf{--- Приходят ли к вам на консультации люди, которые осознают магическое мышление как проблему?}

Довольно редко. Как правило, это всплывает уже в ходе решения каких-то других вопросов. Обычно людям кажется, что то, как они мыслят, — это норма, просто потому, что у нас нет навыка саморефлексии и критического отношения к собственным мыслям.

Чаще всего люди жалуются на тревогу, грусть, другие неприятные эмоции или на внешние события, с которыми не могут \ex{совладать}{to cope}. Способ мышления как одну из возможных проблем мы вскрываем уже в ходе терапии.

\textbf{--- Можно ли сказать, что у россиян так или иначе всегда присутствовало магическое мышление? Или же оно получило большую популярность именно в эпоху коронавируса и смены политической обстановки?}

Я думаю, что оно так или иначе присутствовало всегда. Наши бабушки и дедушки верили в приметы, были распространены различные магические практики. Это так называемая житейская психология. Человеку всегда нужно было снимать неопределенность, поэтому приметы и ритуалы \ex{издревле}{since ancient times} были \ed{неотъемлемой частью}{неотъемлемая часть}{an integral part} человеческой культуры.

В какой-то степени тут работает простейший механизм научения: например, надеть счастливые носки — значит сдать экзамен. Складывается ложная причинно-следственная связь, что именно носки приносят мне пятерку на экзамене.

\begin{fancyquotes}
    Развитие науки, которая во всем всегда сомневается, развеивает веру в такие приметы. Стрессовые и напряженные моменты просто становятся катализатором для активации более простых механизмов работы психики
\end{fancyquotes}

\textbf{--- Во время стрессовых событий повышается риск \ed{внушаемости}{внушаемость}{suggestibility} и веры в теории заговора?}

Безусловно. Как мы уже обсудили выше, до этого человек мог спокойно жить, читать гороскопы, а тут он начинает верить в теории заговора, чтобы успокоить себя и найти событиям разумное объяснение.

\textbf{--- Как можно самостоятельно осознать, что стал \ed{заложником}{заложник}{hostage} магического мышления?}

Мне кажется, что это довольно сложно. Здесь нужен навык саморефлексии. Ведь придется научиться ставить под сомнение собственные привычки, поведение и мысли. Если вы избегаете встреч с черной кошкой, остановитесь и спросите себя: «Что я сейчас делаю? Для чего я это делаю? Откуда у меня это знание?» Порой этого бывает достаточно.

\textbf{--- Может ли магическое мышление стать частью рутины? Допустим, человек воспринимает счастливые носки как ежедневный обычный процесс.}

Конечно, может. Очень часто мы не осознаем, как на автомате выполняем те или иные ритуалы — просто потому, что мы привыкли так делать. Поэтому ключевым фактором в работе с магическим мышлением будет осознанность и сомнение.

\clearpage


\section{Сестрица Алёнушка и братец Иванушка}
% https://deti-online.com/skazki/russkie-narodnye-skazki/sestrica-alyonushka-i-bratec-ivanushka/
% https://www.youtube.com/watch?v=UDaOREoItE8
Жили-были стар\'{и}к да стар\'{у}ха, у них был\'{а} дочка Алёнушка да сын\'{о}к Иванушка. Старик со старухой умерли. Остались Алёнушка да Иванушка одни-одинёшеньки. Пошла Алёнушка на работу и братца с собой взяла. Идут они по д\'{а}льнему пут\'{и}, по шир\'{о}кому п\'{о}лю, и захотелось Иванушке пить.
%
\begin{dialogue}
    \item Сестр\'{и}ца Алёнушка, я пить хочу!
    \item Подожди, братец, дойдем до кол\'{о}дца.
\end{dialogue}
%
Шли-шли, -- солнце высоко, колодец далёко, жар \explainDetail{донимает}{донимать}{(colloq.) to bother, harass}, пот выступ\'{а}ет. Сто\'{и}т коровье коп\'{ы}тце\footnote{hoof (diminutive of коп\'{ы}то)} полн\'{о} водицы.
%
\begin{dialogue}
    \item Сестрица Алёнушка, хлебну\footnote{(colloq.) to drink} я из копытца!
    \item Не пей, братец, телёночком станешь!
\end{dialogue}
%
Братец послушался, пошли дальше. Солнце высоко, колодец далёко, жар донимает, пот выступает. Сто\'{и}т лошадиное копытце полно водицы.
%
\begin{dialogue}
    \item Сестрица Алёнушка, напьюсь я из копытца!
    \item Не пей, братец, жеребёночком станешь!
\end{dialogue}
%
\explainDetail{Вздохнул}{вздых\'{а}ть/вздохн\'{у}ть}{sigh} Иванушка, опять пошли дальше. Идут, идут, -- солнце высоко, колодец далёко, жар донимает, пот выступает. Сто\'{и}т к\'{о}зье копытце полно водицы. Иванушка говорит:
%
\begin{dialogue}
    \item  Сестрица Алёнушка, мочи нет: напьюсь я из копытца!
    \item  Не пей, братец, козлёночком станешь!
\end{dialogue}
%
Не послушался Иванушка и нап\'{и}лся из к\'{о}зьего копытца. Нап\'{и}лся и стал козлёночком\dots Зовёт Алёнушка братца, а вместо Иванушки бежит за ней беленький козлёночек. Залилась\footnote{flooded} Алёнушка слез\'{а}ми, села на стож\'{о}к -- плачет, а козлёночек в\'{о}зле неё скачет. В ту пору ехал мимо купец:
%
\begin{dialogue}
    \item О чём, красная девица, плачешь?
\end{dialogue}
Рассказала ему Алёнушка про свою беду. Купец ей и говорит:
\begin{dialogue}
    \item Под\'{и} за меня замуж. Я тебя наряж\'{у} в златосеребро, и козлёночек будет жить с нами.
\end{dialogue}
Алёнушка под\'{у}мала, под\'{у}мала и пошла за купца замуж. Стали они жить-поживать, и козлёночек с ними живёт, ест-пьёт с Алёнушкой из одной чашки. Один раз купца не было д\'{о}ма. \explainDetail{Откуда не возьм\'{и}сь}{откуда не возьм\'{и}сь}{out of the blue} прих\'{о}дит \explain{в\'{е}дьма}{witch}: стала под Алёнушкино окошко и такто ласково начал\'{а} звать её куп\'{а}ться на реку\footnote{произношение: н\'{а}реку}. Привела ведьма Алёнушку на реку. \explainDetail{Кинулась}{кид\'{а}ться/к\'{и}нуться}{to throw oneself, to fling oneself, to dash, to rush, } на неё, привязала Алёнушке на шею камень и бр\'{о}сила её в в\'{о}ду. А сам\'{а} оборот\'{и}лась Алёнушкой, \explainDetail{нарядилась}{наряж\'{а}ться/наряд\'{и}ться}{to dress as someone, to imitate} в её пл\'{а}тье и пришла в её хор\'{о}мы. Никто ведьму не \explain{распозн\'{а}л}{recognised}. Купец вернулся -- и тот не распознал.

Одному козлёночку всё было ведомо. Повесил он голову, не пьет, не ест. Утром и вечером ходит по бережку около воды и зовёт:
\begin{dialogue}
    \item Алёнушка, сестрица моя! Выплынь, выплынь на бережок\dots
\end{dialogue}

Узнала об этом ведьма и стала просить мужа \explain{зарежь}{slaughter} да зарежь козлёнка.
Купцу жалко было козлёночка, \explain{привык}{got used to + \textit{дат.}} он к нему. А ведьма так пристаёт, так упрашивает, --- делать н\'{е}чего, купец согласился:
%
\begin{dialogue}
    \item Ну, зарежь его\dots
\end{dialogue}
%
Велела ведьма разложить костры высокие, греть котлы чугунные, точить ножи булатные.
Козлёночек проведал, что ему недолго жить, и говорит названому отцу:
%
\begin{dialogue}
    \item Перед смертью пусти меня на речку сходить, водицы испить, кишочки прополоскать.
    \item Ну, сходи.
\end{dialogue}
%
Побежал козлёночек на речку, стал на берегу и жалобнёхонько закричал:
%
\begin{dialogue}
    \item   Алёнушка, сестрица моя! Выплынь, выплынь на бережок.
    Костры горят высокие,
    Котлы кипят чугунные,
    Ножи точат \explain{булатные} {Bulat is a type of steel alloy known in Russia from medieval times; it was regularly mentioned in Russian legends as the material of choice for cold steel. This type of steel was used by the armies of nomadic peoples. Bulat steel was the main type of steel used for swords in the armies of Genghis Khan.},
    Хотят меня зарезати!
\end{dialogue}
%
%
Алёнушка из реки ему отвечает:
%
\begin{dialogue}
    \item Ах, братец мой Иванушка! Тяжёл камень на дно тянет,
    Шёлкова трава ноги спутала,
    Желты пески на груди легли.
\end{dialogue}
%
%
А ведьма ищет козлёночка, не может найти и посылает \explainDetail{слуг\'{у}}{слуг\'{а}}{servant}:
\begin{dialogue}
    \item Пойди найди козлёнка, приведи его ко мне.
\end{dialogue}
% 
%
Пошёл слуга на реку и видит: по берегу бегает козлёночек и жалобнёшенько зовёт:
\begin{dialogue}
    \item Алёнушка, сестрица моя! Выплынь, выплынь на бережок.
    Костры горят высокие,
    Котлы кипят чугунные,
    Ножи точат булатные,
    Хотят меня зарезати!
\end{dialogue}
%
%
А из реки ему отвечают:
\begin{dialogue}
    \item Ах, братец мой Иванушка!
    Тяжёл камень на дно тянет,
    Шелкова трава ноги спутала,
    Желты пески на груди легли.
\end{dialogue}
%
Слуг\'{а} побежал домой и рассказал купцу про то, что слышал на речке. Собрали народ, пошли на реку, закинули сети шелковые и вытащили Алёнушку на берег. Сняли камень с шеи, окунули её в ключевую воду, одели её в нарядное платье. Алёнушка ожила и стала краше, чем была.

А козлёночек от радости три раза перекинулся через голову и обернулся мальчиком Иванушкой.

Ведьму привязали к лошадиному \explainDetail{хвосту}{хвост}{tail}, и пустили в чистое поле.

\section{Маша и медведь}
Жили-были дедушка да бабушка. Была у них внучка Машенька. Собрались раз подружки в лес -- по грибы да по ягоды. Пришли звать с собой и Машеньку.
\begin{dialogue}
    \item Дедушка, бабушка, -- говорит Машенька, -- отпустите меня в лес с подружками!
\end{dialogue}
Дедушка с бабушкой отвечают:
%
\begin{dialogue}
    \item Иди, только смотри от подружек не отставай -- не то заблудишься.
\end{dialogue}
Пришли девушки в лес, стали собирать грибы да ягоды. Вот Машенька -- деревце за деревце, кустик за кустик -- и ушла далеко-далеко от подружек.

Стала она аукаться, стала их звать. А подружки не слышат, не отзываются.
Ходила, ходила Машенька по лесу -- совсем заблудилась.
Пришла она в самую глушь, в самую чащу. Видит-стоит избушка. Постучала Машенька в дверь -- не отвечают. Толкнула она дверь, дверь и открылась.
Вошла Машенька в избушку, села у окна на лавочку.

Села и думает:

\begin{fancyquotes}
    «Кто же здесь живёт? Почему никого не видно?..» А в той избушке жил большущий медведь. Только его тогда дома не было: он по лесу ходил. Вернулся вечером медведь, увидел Машеньку, обрадовался.
\end{fancyquotes}
%
\begin{dialogue}
    \item Ага, -- говорит, -- теперь не отпущу тебя! Будешь у меня жить. Будешь печку топить, будешь кашу варить, меня кашей кормить.
\end{dialogue}

Потужила Маша, погоревала, да ничего не поделаешь. Стала она жить у медведя в избушке.

Медведь на целый день уйдёт в лес, а Машеньке наказывает никуда без него из избушки не выходить.
%
\begin{dialogue}
    \item А если уйдёшь, -- говорит, -- всё равно поймаю и тогда уж съем!
\end{dialogue}
Стала Машенька думать, как ей от медведя убежать. Кругом лес, в какую сторону идти -- не знает, спросить не у кого\dots

Думала она, думала и придумала.

Приходит раз медведь из лесу, а Машенька и говорит ему:
\begin{dialogue}
    \item Медведь, медведь, отпусти меня на денёк в деревню: я бабушке да дедушке гостинцев снесу.
    \item Нет, -- говорит медведь, -- ты в лесу заблудишься. Давай гостинцы, я их сам отнесу!
\end{dialogue}
А Машеньке того и надо!

Напекла она пирожков, достала большой-пребольшой короб и говорит медведю:

\begin{dialogue}
    \item Вот, смотри: я в короб положу пирожки, а ты отнеси их дедушке да бабушке. Да помни: короб по дороге не открывай, пирожки не вынимай. Я на дубок влезу, за тобой следить буду!
    \item Ладно, -- отвечает медведь, -- давай короб! Машенька говорит:
    \item Выйди на крылечко, посмотри, не идёт ли дождик! Только медведь вышел на крылечко, Машенька сейчас же залезла в короб, а на голову себе блюдо с пирожками поставила.
\end{dialogue}

Вернулся медведь, видит -- короб готов. Взвалил его на спину и пошёл в деревню.

Идёт медведь между ёлками, бредёт медведь между берёзками, в овражки спускается, на пригорки поднимается. Шёл-шёл, устал и говорит:

\begin{fancyquotes}
    Сяду на пенёк, Съем пирожок! А Машенька из короба:
    Вижу, вижу! Не садись на пенёк, Не ешь пирожок! Неси бабушке, Неси дедушке!
\end{fancyquotes}

\begin{dialogue}
    \item Ишь какая глазастая, -- говорит медведь, -- всё видит! Поднял он короб и пошёл дальше. Шёл-шёл, шёл-шёл, остановился, сел и говорит:
\end{dialogue}

\begin{fancyquotes}
    Сяду на пенёк, Съем пирожок! А Машенька из короба опять: Вижу, вижу! Не садись на пенёк, Не ешь пирожок! Неси бабушке, Неси дедушке!
\end{fancyquotes}

Удивился медведь:

\begin{dialogue}
    \item Вот какая хитрая! Высоко сидит, далеко глядит! Встал и пошёл скорее.
\end{dialogue}
Пришёл в деревню, нашёл дом, где дедушка с бабушкой жили, и давай изо всех сил стучать в ворота:
\begin{dialogue}
    \item Тук-тук-тук! Отпирайте, открывайте! Я вам от Машеньки гостинцев принёс.
\end{dialogue}
А собаки почуяли медведя и бросились на него. Со всех дворов бегут, лают.

Испугался медведь, поставил короб у ворот и пустился в лес без оглядки.

Вышли тут дедушка да бабушка к воротам. Видят- короб стоит.
\begin{dialogue}
    \item Что это в коробе? -- говорит бабушка.
\end{dialogue}
А дедушка поднял крышку, смотрит и глазам своим не верит: в коробе Машенька сидит -- живёхонька и здоровёхонька.

Обрадовались дедушка да бабушка. Стали Машеньку обнимать, целовать, умницей называть.
