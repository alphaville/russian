\chapter{Музыка}


\section{Пётр Налич}
% https://uznayvse.ru/znamenitosti/biografiya-petr-nalich.html
\subsection{Биография}
Петр Налич – российский певец боснийского происхождения, первый отечественный музыкант, ставший популярным благодаря YouTube. Простенький клип с песней «Guitar» на английском языке с акцентом вкупе с балканской мелодикой и дурашливой атмосферой пришелся по душе подавляющему большинству интернет-аудитории в 2007 году.

Пишет музыку для спектаклей, кинофильмов, мультфильмов. Автор песен на вымышленном языке бабурси, в которых, по признанию самого Петра, нет никакого смысла. Представлял со своим коллективом Россию на Евровидении 2010 года в Осло. Первым среди артистов использовал систему «Pay What You Want» для записи дебютного альбома «Радость простых мелодий».

\subsection{Детство, юност, семья}
Петр родился весной 1981 года в семье москвичей Андрея и Валентины Наличей, где уже подрастал старший сын Павел, впоследствии известный художник-оформитель. Андрей Захидович и Валентина Марковна – архитекторы, отец также занимается скульптурой. Его «Лента Мебиуса» расположена у кинотеатра «Горизонт», спортивный приз «Слава» – также его работа. Позже в творческом союзе с младшим сыном, а также Александром и Сергеем Цигалями создал памятник «Сочувствие», посвященный гуманному обращению с бродячими животными и памяти пса по кличке Мальчик, обитавшего в подземном переходе.

\begin{fancyquotes}
    Образцовым ребенком я, конечно, не был, но и хулиганом тоже. Хотя мне всю жизнь хотелось им быть, потому что хулиганы всегда круче, они нравятся девушкам, а я был домашним мальчиком.
\end{fancyquotes}

Именно отец приобщил сыновей к футболу. Несмотря на то, что Петр в детстве был вполне азартным мальчишкой и любил лазать по крышам, кататься с горки, никаким спортом он особо не увлекался, как и брат. И, когда младшему сыну было примерно 11-12 лет, Андрей Захидович решил всерьез заняться его и Павла физическим развитием. Каждую неделю он сажал их в машину, в другую садились двоюродные братья со своими отцами, и всей компанией ехали на станцию Университет. В то время там была неплохая футбольная площадка, на которой они могли гонять мяч бесплатно. Эту традицию, но уже со своими детьми, Петр продолжает и сейчас.

В одном из интервью Налич рассказывал и о своем дедушке-боснийце, у которого был ангажемент в оперном театре Белграда. Во время войны Захид Омерович попал в нацистский лагерь, где ему сломали гортань. Чудом выжив, он переехал в СССР и стал работать на радио. Больше он не пел, но унаследовал любовь к музыке детям, и отец Петра любил исполнять цыганские и русские романсы. Однажды родители спросили у мальчика, какой подарок он хотел бы получить к 14 дню рождения: гитару или конструктор «Лего-Техникс». Последним он просто грезил, его и попросил. Но папа и мама подарили Пете и гитару тоже.

\begin{fancyquotes}
    Я начал играть романсы, песни Цоя и "Наутилуса" вперемешку с цыганскими и казачьими, и этим репертуаром я набрал приличное количество очков в глазах девушек. Оказалось, что необязательно быть хулиганом, гитара тоже неплохо работает.
\end{fancyquotes}

Это во многом определило дальнейший жизненный путь Налича. Еще в школе он создал хард-рок-группу, в которой с удовольствием пел. Петр постоянно занимался музыкой: сначала в детской музыкальной школе имени Николая Мясковского, затем в музыкальном училище при Московской консерватории. Поступив в архитектурный институт, он стал заниматься вокалом в студии «Орфей» у педагога Ирины Мухиной. Окончив МАРХИ, Петр решил продолжить музыкальное образование и в 2010 году пошел учиться оперному пению:

\begin{fancyquotes}
    Я учился у выдающейся певицы Валентины Левко – к сожалению… ушедшей от нас. И огромной, мощнейшей школой для меня стала Оперная студия при РАМ имени Гнесиных. Сейчас она носит имя Сперанского, а когда я поступил, Юрий Аркадьевич Сперанский был еще жив, и мне посчастливилось с ним работать.
\end{fancyquotes}


\subsection{Первый успех: Guitar}
В 2007 году Петр создал сайт и стал выкладывать сочиненные им композиции, а любительский клип с песней Guitar загрузил на YouTube. Певец признавался, что сам не ожидал столь бурной реакции в интернете. На тот момент у него на сайте было порядка сорока композиций, которые мог скачать кто угодно, но лишь после успеха клипа «Guitar» с цыганскими мотивами и красивым голосом Петра (он пел на английском с сильным русским акцентом, который, впрочем, песне шел лишь на пользу) люди стали интересоваться его творчеством.

Осенью того же года состоялся первый сольный концерт певца в столичном клубе «Апшу». Билеты на выступление были раскуплены моментально, ажиотаж вокруг его персоны стоял небывалый. Счастливчики, побывавшие на концерте, тут же выложили в интернет восторженные отклики. В конце года Рунет признал композицию Guitar лучшей за 2007.

Творческая карьера продолжилась созданием «Музыкального коллектива Петра Налича», который для краткости называли МКПН. 2008 год группа посвятила гастролям по России. Тогда же она стала официальным музыкальным сопроводителем российских спортсменов на пекинской Олимпиаде.

Налич вместе с МКПН выпустил первый альбом «Радость простых мелодий». Затем группа стала участником антверпенского фестиваля Sfincs. В репертуаре коллектива звучали композиции «Медовый, аметистовый», «Взгляд твоих черных очей» и другие.

\subsection{Евровидение и новый стиль}
В 2010 году Налич с коллективом отправился в Осло, представлять Россию на «Евровидении». Лирическая баллада «Lost and Forgotten», спетая им на конкурсе Евровидения, не принесла призового места (он занял 11-е), но была тепло принята публикой.

2010 год также был отмечен выпуском нового альбома «Веселые Бабури», а еще через два года был записан диск «Золотая рыбка». На композицию «Сахарный пакет» был снят клип. Параллельно Петр начал исполнять партии в операх. Спектакли «Богема» и «Евгений Онегин», где Налич соответственно пел партии Рудольфа и Ленского, шли в театре-студии оперы при Академии.

В 2013 году вышел следующий альбом «Песни о любви и Родине», записанный в сопровождении оркестра Юрия Башмета. Следом Петр записал диск «Кухня» на выдуманном языке бабурси, созданном внутри МКПН, спел партию Германа в опере «Пиковая дама», которая прошла в Государственном музее Александра Пушкина, и принял участие в театрализованных онлайн-чтениях «Анна Каренина. Живое издание».

В 2015 году Налич распустил коллектив и объявил для него бессрочный отпуск, а сам занялся написанием музыки для спектаклей «Северная Одиссея» и «Питер Пен». Как композитор Налич получил приглашение от Пермского ТЮЗа. Он создал музыку для постановки «Обыкновенное чудо» по пьесе Евгения Шварца.

Затем Петр удивил поклонников новой программой «Утесов и не только…», которую исполнял в сопровождении эстрадно-симфонического бэнда. Впоследствии артист кардинально поменял состав собственного коллектива, пригласив в него музыкантов, способных на высоком профессиональном уровне исполнять совершенно новые ритмы и мелодии. Также он записал новый альбом «Паровоз», а затем выступил с новым репертуаром в клубе «16 тонн».

Не напрасно Налич назвал себя в одном из интервью «троякодышащей рыбой». В 2018 году разноплановый творческий человек снова стал студентом – Петр поступил в Гнесинку на композиторский факультет, вздыхая в интервью, что надо бы уже как-то определиться, кто он есть, иначе как-то нелепо. И тут же добавлял:

\begin{fancyquotes}
    Но пока естественным образом так складывается, что я занимаюсь и оперой, и сочинительством с бэндом, каким бы пестрым оно ни было по жанрам, и написанием инструментальной музыки для театра и других проектов. Теперь музыки будет еще больше.
\end{fancyquotes}


Параллельно с учебой на композиторском отделении в его оперном репертуаре появилась партия Тамино в «Волшебной флейте». Композиторская копилка самого Петра пополнилась музыкой к спектаклю «Тина», а следом к постановке «Горячее сердце», премьера которой состоялась на большой сцене театра имени Евгения Вахтангова. Расширился и оперный репертуар: Налич появился на разных театральных сценах в образах Неморино («Любовный напиток»), Альфреда Жермона («Травиатта»), Луиджи («Плащ»).

Благодаря собранным на краудфандинговой платформе Planeta средствам, был выпущен очередной альбом «Отражения в лужах». С певицей Женей Любич Петр записал яркую песню «Дежавю», а следом вышел новый диск, посвященный царской семье Romanovs100. Этот проект принес Наличу премию Original Music 2019 New York Festival TV \& Film Awards. Еще одна премия – «Онегин» в номинации «Событие года» досталась опере-променаду «Пиковая дама», где Петр исполнил партию Германа.

Значимым культурным событием 2019 года в Москве стало открытие нового театра – «Московского оперного дома». Его открытие сопровождалось премьерой спектакля «Иоланта». Партия Водемона в исполнении Налича была, как всегда, безупречна. Конец года принес Наличу заслуженную награду: он стал лауреатом Первой премии Всероссийского конкурса молодых композиторов «Партитура времени». Диплом был ему вручен в номинации «Сочинение для голоса и фортепиано».




\section{Виктор Цой}
% https://ru.wikipedia.org/wiki/%D0%A6%D0%BE%D0%B9,_%D0%92%D0%B8%D0%BA%D1%82%D0%BE%D1%80_%D0%A0%D0%BE%D0%B1%D0%B5%D1%80%D1%82%D0%BE%D0%B2%D0%B8%D1%87
\subsection{Биография и творчество}
Виктор Робертович Цой (21 июня 1962 года, Ленинград --- 15 августа 1990 года, близ посёлка Кестерциемс, Латвийская ССР) --- советский рок-музыкант, автор песен и художник. Основатель и лидер рок-группы «Кино», в которой пел, играл на гитаре, писал музыку и стихи. Снялся в нескольких фильмах.

Виктор Цой родился единственным ребёнком в семье инженера корейского происхождения Роберта Максимовича Цоя и преподавательницы физкультуры Валентины Васильевны. Детство музыканта прошло в окрестностях Московского Парка Победы: он родился в роддоме на Кузнецовской улице (располагается внутри парка; сейчас это кардиоцентр), семья до 1990-х гг. жила в примечательном «генеральском доме» на углу Московского проспекта и улицы Бассейной (сейчас это памятник архитектуры). Некоторое время Виктор учился в близлежащей школе на улице Фрунзе, где работала его мама. В 1973 г. родители Цоя развелись, а через год повторно вступили в брак.

С 1974 по 1977 год посещал среднюю художественную школу, где возникла группа «Палата No. 6» во главе с Максимом Пашковым.
После исключения за неуспеваемость из художественного училища имени В. Серова поступил в СГПТУ--61 на специальность резчика по дереву.
В молодости был поклонником Михаила Боярского и Владимира Высоцкого, позднее Брюса Ли, имиджу которого начал подражать.
Увлекался восточными единоборствами и часто дрался «по-китайски» с Юрием Каспаряном.

\subsection{Смерть}
15 августа 1990 года в 12 часов 28 минут Виктор Цой погиб в автокатастрофе. ДТП произошло на 35 километре трассы «Слока --- Талси» под Тукумсом в Латвии, в нескольких десятках километров от Риги. Согласно наиболее правдоподобной официальной версии, Цой заснул за рулём, после чего его «Москвич-2141» тёмно-синего цвета вылетел на встречную полосу и столкнулся с автобусом «Икарус» модели 250 (иногда этот автобус ошибочно идентифицируют как 280 модель.

\begin{fancyquotes}
    Столкновение автомобиля «Москвич-2141» тёмно-синего цвета с рейсовым автобусом «Икарус-280» произошло в 12 час. 28 мин. 15 августа 1990 г. на 35 км трассы Слока --- Талси. Автомобиль двигался по трассе со скоростью не менее 130 км/ч, водитель Цой Виктор Робертович не справился с управлением. Смерть В. Р. Цоя наступила мгновенно, водитель автобуса не пострадал. ...В. Цой был абсолютно трезв накануне гибели. Во всяком случае, он не употреблял алкоголь в течение последних 48 часов до смерти. Анализ клеток мозга свидетельствует о том, что он уснул за рулем, вероятно, от переутомления.\\

    --- из милицейского протокола; по данным сайта kinoman.net
\end{fancyquotes}




19 августа он был похоронен на Богословском кладбище в Ленинграде.

\subsection{Прочие версии гибели}
Создатели документального кино из цикла «Следствие вели...» предположили, что Цой мог попасть в аварию, когда решил переставить другой стороной кассету в своём магнитофоне, тем самым отвлекшись от движения у «слепого поворота» дороги. Речь в передаче шла о кассете с демозаписью последнего альбома. Гитарист Юрий Каспарян ещё в 2002 году опроверг информацию о наличии этой кассеты в автомобиле Цоя: «Пользуясь случаем, хочу развеять миф, что на месте аварии нашли кассету с демо ``Черного альбома''... Все было не так. Я специально приехал в Юрмалу с аппаратурой, с инструментами и мы делали аранжировки для нового альбома. Когда доделали, я забрал кассету и поехал в Петербург. Я приехал утром, вечером узнал о случившемся. И поехал обратно. И кассета все время была у меня в кармане».


\subsection{Творчество}
В конце 1970-х --- начале 1980-х началось тесное общение между Алексеем Рыбиным из хард-роковой группы «Пилигримы» и Виктором Цоем, игравшим на бас-гитаре в группе «Палата № 6», оба они познакомились в гостях у Андрея Панова (Свина), на квартире которого часто собирались компании, а также репетировала его собственная панк-группа «Автоматические удовлетворители».

Виктор Цой и Алексей Рыбин в составе «Автоматических удовлетворителей» ездили в Москву и играли панк-рок-металл на подпольных концертах Артемия Троицкого. Во время аналогичного выступления в Ленинграде по случаю юбилея Андрея Тропилло произошло первое знакомство с Борисом Гребенщиковым

\subsection{Первый альбом}
Летом 1981 года Виктор Цой, Алексей Рыбин и Олег Валинский основали группу «Гарин и Гиперболоиды», которая уже осенью была принята в члены Ленинградского рок-клуба. Вскоре Валинского забирают в армию, а группа, сменив название на «Кино», весной 1982 приступила к записи дебютного альбома. «Кино» под руководством Бориса Гребенщикова записывались на студии Андрея Тропилло в Доме Юного Техника, в записи принимали участие музыканты «Аквариума». Вскоре с ними же «Кино» дали свой первый электрический концерт в рок-клубе, всё выступление шло под драм-машину, а под песню «Когда-то ты был битником» из-за кулис на сцену выскочили БГ, Майк и Панкер. К лету альбом был полностью завершён, продолжительность его звучания составляла 45 минут, откуда и появилось название. Но позже из окончательного варианта была убрана песня «Я --- асфальт», которую можно найти в переиздании «45», где она прилагается в качестве бонус-трека. Запись получила некоторое распространение, о группе заговорили, начались квартирные концерты в Москве и Ленинграде. Вместе с будущим барабанщиком Зоопарка Валерием Кирилловым осенью этого же года «Кино» записывает в студии Андрея Кускова несколько песен, в том числе «Весна» и «Последний герой», вошедшие в сборник «Неизвестные песни Виктора Цоя» (всего четыре издания).

Тогда запись была забракована и распространения не получила, так как Цой забрал ленту себе.

19 февраля 1983 года проходит совместный электрический концерт «Кино» и «Аквариума», музыканты выступали с тёмным макияжем и в костюмах со стразами. При этом они исполняли «Электричку», «Троллейбус» и «Алюминиевые огурцы». В основной состав был приглашён Юрий Каспарян. Весной из-за разногласий с Цоем Алексей Рыбин покидает группу «Кино». Лето уходит на совместные репетиции с новым гитаристом. В результате этого Виктор Цой и Юрий Каспарян записали альбом «46», который вначале задумывался как демозапись «Начальника Камчатки». Алексей Вишня «скинул» запись нескольким друзьям на плёнку. «46» получил широкое распространение и был воспринят как полноценный альбом. Осенью 1983 года Виктор Цой лёг на обследование в психиатрическую больницу на Пряжке, где провёл полтора месяца, избегая призыва в армию. После выписки из психиатрической клиники он пишет песню «Транквилизатор». Весной выступил на втором фестивале рок-клуба, где группа «Кино» получила лауреатское звание, а песня «Я объявляю свой дом безъядерной зоной», открывшая фестиваль, признана лучшей антивоенной песней фестиваля 1984 года.



\subsection{Второй состав «Кино»}
Летом 1984 года в студии «Антроп» Андрея Тропилло начинается запись альбома «Начальник Камчатки», к которому, кроме Виктора, приложили свою руку БГ и Сергей Курёхин.

В феврале 1984 Виктор и Марьяна празднуют свадьбу. На свадьбу были приглашены Гребенщиков, Майк, Титов, Каспарян, Гурьянов и другие.

Весной 1985 «Кино» заработали ещё одно звание лауреата и засели в студию к А. Тропилло писать «Ночь». Работа над записью затянулась из-за желания создать новую музыку с новыми приёмами игры. Альбом никак не получался, Виктор бросил «Ночь» недоделанной и в студии Алексея Вишни занялся записью «Это не любовь», который получился всего за неделю с небольшим. К осени «Это не любовь» была сведена и удачно разошлась по стране, а в январе 1986 вышла «Ночь», среди песен которой были известные «Мама Анархия» и «Видели ночь». Параллельно с выходом пластинки растёт популярность Виктора Цоя, а в феврале на 4-м фестивале рок-клуба «Кино» получает диплом за лучшие тексты. 5 августа 1985 года у Цоя родился сын Саша.


Летом 1986 года Виктор работал в бане на проспекте Ветеранов, он там мыл помещения из брандспойта. Необходимо было приходить на один час в день, но это было время с 22 до 23 часов, что ему мешало, так как Цой проводил это время суток с группой.

Также летом все участники группы уезжают в Киев на съёмки фильма «Конец каникул» (режиссёр Сергей Лысенко), а чуть позже дают совместный концерт с «Аквариумом» и «Алисой» в ДК МИИТ в Москве, с этими же группами в США выходит «Красная волна». Осенью Сергей Фирсов приглашает Виктора работать кочегаром. Цой соглашается, и они оба начинают работать кочегарами в котельной «Камчатка», откуда выросли многие знаменитые рок-музыканты.

В ней Рашид Нугманов организовал съёмки короткометражки «Йя-Хха», там же проходят съёмки фильма «Рок» Алексея Учителя --- оба фильма при участии Цоя. Осень и зима проходят в Ялте на съёмках «Ассы» Сергея Соловьёва.

Весна 1987 богата концертными событиями: премьера «Ассы» в ДК МЭЛЗ, последнее участие на фестивале рок-клуба, где «Кино» получили приз «За творческое совершеннолетие».

На порто-студии «Yamaha MT44» «Кино» начинают записывать альбом «Группа крови». Осенью 1987 года Виктор улетает к Рашиду Нугманову в Алма-Ату на съёмки своего последнего фильма «Игла», в связи с этим «Кино» доработали «Группу крови» и на время прекратили концертную \explain{деятельность}{activity}. В 1988 выходит «Игла» и «Группа крови», которые породили «киноманию».

Начинаются триумфальные гастроли по Советскому Союзу --- «Кино» собирают аншлаги на всех концертах.

16 ноября 1988 на мемориальном концерте памяти Александра Башлачёва публика ведёт себя крайне активно; по плану концерт должна была заканчивать песня Башлачёва «Время колокольчиков» (в записи), памяти которого был посвящён концерт, но по невыясненным причинам во время выступления Цоя (он играл на гитаре) внезапно включили «Время колокольчиков», Цой прекратил играть, не понимая откуда идёт звук, который он не производит и что вообще происходит. Администрация многократно объявляла, что всем н\'{у}жно расходиться, концерт окончен. Цой не уходил, он несколько раз подходил к выключенным микрофонам и проверял, работают ли. Потом разводил руками --- «не работает», и ходил по сцене туда и сюда с цветком, не уходя со сцены, но и не имея возможности петь и что-то сказать публике. Публика не расходилась, люди шумели, кричали, было видно, что что-то идёт не так. Создавалось впечатление, что некая злая воля решила прекратить концерт и включила финальную песню прямо во время выступления Цоя. Через 10 минут этого противостояния администрация включила микрофон. Цой, в очередной раз подойдя проверять микрофон, услышал что он включён, и объявил людям, что по непонятным причинам несвоевременно была включена финальная песня Саши Башлачева, но после этого петь и играть уже не очень удобно. После этого он стал собираться и публика потянулась к выходу.

Весной 1988 записывается черновик, а зимой окончательный вариант альбома «Звезда по имени Солнце», который решили выпустить осенью. Цой знакомится с Юрием Айзеншписом, который с 1989 стал продюсером «Кино», организовывая концертные туры и частые выступления на телевидении, после чего группа обретает всесоюзную популярность. В день 50-летия Цоя Александр Градский в эфире канала «Москва-24» рассказал, что в тот период Артемий Троицкий инспирировал письмо в Московский Горком, которое должно было настроить московских рок-музыкантов против Виктора Цоя.

На телевидении Виктор Цой дебютировал в программе «Взгляд», об этом рассказано в книге «Взгляд» --- битлы перестройки.

В начале 1989 группа «Кино» впервые едет за границу во Францию, где выпускают альбом «Последний герой». Летом Виктор с Юрием Каспаряном едут в США. Тем временем «Игла» выходит на второе место в прокате советских фильмов, а на кинофестивале «Золотой Дюк» в Одессе Виктора Цоя признают лучшим актёром СССР.

24 июня 1990 года прошёл последний концерт «Кино» в Москве на Большой спортивной арене Лужников. На этом концерте, впервые после московской Олимпиады-80 был зажжён огонь в Олимпийской чаше. После этого Цой с Каспаряном уединились на даче под Юрмалой, где на порто-студию начали записывать материал для нового альбома. Этот альбом, дописанный и сведённый музыкантами группы «Кино» уже после смерти Цоя, вышел в январе 1991 и получил символическое название «Чёрный альбом», с соответствующим оформлением обложки.

\section{Елена Ваенга}
Ел\'{е}на В\'{а}енга (настоящее имя --- Ел\'{е}на Влад\'{и}мировна Хрулёва; род. 27 января 1977, Североморск, Мурманская область, РСФСР, СССР) --- российская \explain{эстрадная певица}{pop singer}, автор песен, актриса. Лауреат премий «Шансон года».

В\'{а}енга --- это название родного для Елены Хрулёвой города Североморска до 18 апреля 1951 года, а также реки недалеко от него. В основе названия и псевдонима --- саамское слово «\explain{олен\'{и}ха}{deer}» (кильд. вайонгг). Псевдоним придуман её матерью.

\subsection{Биография}
Родилась 27 января 1977 года в Североморске. \explainDetail{Петь}{петь}{to sing} и \explain{обучаться}{to study (+ dative)} музыке начала с трёх лет.

Мать Елены Ваенги по образованию химик, отец --- инженер, работали в посёлке Вьюжный на \explain{судоремонтном заводе}{shipyard (судно: vessel; ship, plural: суда)} «Нерпа», который обслуживает атомные \explain{подводные лодки}{подводная лодка: submarine}. Про отца и родной Север у Елены Ваенги есть песня:

\begin{fancyquotes}
    {\it У меня глаза северных цветов,\\
        И мне не нужны тропические страны.\\
        Я всегда с тобой рядышком была.\\
        Жаль, что ты уехал слишком рано.\\
        Я вдруг поняла: все эти города\\
        Я должна пройти, как в наказанье.\\
        Но у меня есть дом, а у дома --- я,\\
        А у Севера --- сиянье}
\end{fancyquotes}


Дед Елены со стороны матери --- контр-адмирал Северного флота Василий Семёнович Журавель, он упоминается в книге «Знаменитые люди Санкт-Петербурга». Бабушка Надежда Георгиевна Журавель (её крёстная) (род. 1927). Про неё у Елены Ваенги есть песня: «Моя бабушка любит суши...». Родители отца --- \explain{коренные}{коренн\'{о}й ж\'{и}тель; коренн\'{ы}е ж\'{и}тели: indigenous} петербуржцы, \explain{пережили}{(переживать/пережить) to survive; to experience} блокаду Ленинграда. Дед по линии отца --- зенитчик, во время \explain{Великой Отечественной войны}{second world war} \explainDetail{воевал}{воевать/повоевать}{to fight} под Ораниенбаумом, а бабушка по линии отца была врачом в госпитале в блокадном Ленинграде.

У Елены Ваенги есть младшая сестра Татьяна, она работает в дипломатической сфере, знает несколько языков.

Гражданский муж Елены Ваенги \explain{на протяжении 16 лет}{for 16 years} с 1995 по 2011 год --- Иван Иванович Матвиенко (род. 1957) --- продюсер певицы, по национальности цыган, был женат, его дочь на 2 года старше Елены Ваенги, раньше Иван перегонял машины из Германии.

Племянник, Руслан Сулимовский --- директор её коллектива.

В ночь с пятницы на субботу 10 августа 2012 года Ваенга в родильном доме No. 16 Санкт-Петербурга родила сына Ивана. 30 сентября 2016 года Елена официально вышла замуж за Романа Садырбаева.

\subsection{Творческая деятельность}
Первую песню «Голуби» написала в 9 лет, стала победительницей Всесоюзного конкурса молодых композиторов на Кольском полуострове. После школы приехала в Санкт-Петербург, где закончила музыкальное училище им. Н. А. Римского-Корсакова по классу фортепиано, получив диплом педагога-концертмейстера. Некоторое время преподавала музыку в школе. Факультативно занималась вокалом.

Елена Ваенга с детства мечтала стать актрисой, поэтому после музыкального училища поступила в Театральную академию (ЛГИТМИК) на курс Г. Тростянецкого, но проучилась лишь два месяца, так как её пригласили в Москву записывать первый альбом. Продюсером певицы стал Степан Разин. Под псевдонимом Нина она выпустила клип на песню «Длинные коридоры» (композиция была издана в 2011 году на сборнике «Живая струна»). Альбом был записан, но не вышел. Разочаровавшаяся в шоу-бизнесе певица сбежала от Разина и уехала в Санкт-Петербург. Тем временем её песни взяли в свой репертуар Александр Маршал («Невеста»), Татьяна Тишинская («А ты налей мне белого вина», «Мама, что ты плачешь», «Володенька», «Угостите даму сигаретой»), группы «Стрелки» («Тонкая веточка»), «Божья коровка» («Сердце моё», «Самая любимая моя») и другие известные исполнители. Эти песни распространил её бывший продюсер. Елена Ваенга приняла решение с ним не судиться.

В Санкт-Петербурге Ваенга узнала, что в Балтийском институте экологии, политики и права на кафедре театрального искусства набирает курс П. С. Вельяминов, и в 2000 году пошла учиться к нему. Закончив курс, получила диплом по специальности «драматическое искусство». Выступила в антрепризном спектакле «Свободная пара» в паре с однокурсником Андреем Родимовым (режиссёр Екатерина Шимилёва).

Концертирует с девятнадцати лет. Лауреат петербургского конкурса «Шлягер года 1998» с песней «Цыган», «Достойная песня 2002». Участник концертов-фестивалей «Весна романса» в БКЗ «Октябрьский», «Вольная песня над вольной Невой», «Невский бриз». Дала несколько сольных концертов в ДК имени М.Горького. Гастролирует по России и другим странам и каждый год, в конце января, даёт концерт в БКЗ «Октябрьский» по случаю своего дня рождения.

Настоящая популярность пришла к певице в 2005 году с выходом альбома «Белая птица», в котором было много хитов: «Желаю», «Аэропорт», «Тайга», «Шопен» и заглавная композиция, на которую вышел клип.

28 ноября 2009 года Елена Ваенга получила свой первый приз «Золотой граммофон» за песню «Курю».

4 декабря 2010 года Елена повторила свой успех, получив во второй раз премию «Золотой граммофон» за песню «Аэропорт». В том же году певица впервые стала лауреатом фестиваля «Песня года», исполнив композицию «Абсент». А 12 ноября она дала первый в своей концертной деятельности сольный концерт в Государственном Кремлёвском дворце, трансляция которого прошла на Первом канале 7 января 2011 года. В телеанонсе Елене Ваенга была дана следующая характеристика:
Елена Ваенга --- тонкая, художественная, мечтательная и романтичная натура. Музыкальная одарённость, природный темперамент, трудолюбие, жизнерадостность --- всё это составляющие её жизни и творчества... Несмотря на внешнюю хрупкость и молодость, за спиной у этой очаровательной девушки богатая творческая биография и не такая уж простая человеческая судьба. Жанр, в котором работает певица, с большим трудом определяет даже она сама: «На 50 процентов это фолк-рок, есть старинные баллады, городские романсы, шансон. Но границы между ними провести почти невозможно.»
--- анонс на Первом канале --- «Белая птица». Концерт Елены Ваенги
В 2011 году Елена Ваенга приняла участие в ежегодной церемонии национальной премии Шансон года в Кремле, на которой исполнила песни «Оловянное сердце» и «Девочка». Популярность певицы растёт. В январе этого же года она победила Леонида Агутина в телепередаче «Музыкальный ринг» на канале НТВ, набрав почти в пять раз больше голосов слушателей.

26 ноября 2011 года певица в третий раз получила премию «Золотой граммофон» за песню «Клавиши», но на концерте в Кремле исполнила композицию «Шопен». 21 декабря 2011 года певица в третий раз дала концерт в Кремле. В 2012 году на «Золотой граммофон» претендовали песни «Шопен» и «Где была».

В 2011 году Елена Ваенга дала 150 афишных концертов, гастролировала в США, Германии, Израиле.

Периодически играет в спектакле «Свободная пара», совместно с Андреем Родимовым.

В 2011 году Ваенга впервые попала в список самых успешных деятелей российского шоу-бизнеса, составленный Forbes, и заняла в нём девятую позицию, с годовым доходом более шести миллионов долларов.

В репертуаре певицы --- собственные песни, старинные и современные романсы, баллады и народные песни, а также песни на стихи классиков, таких, как Сергей Есенин («Задымился вечер») и Николай Гумилёв («Жираф», «Шут»). В 2012 году певица провела концертный тур по Украине и Германии. Однако на этом деятельность певицы оборвалась в связи с потерей голоса из-за механического повреждения связок. После выздоровления певица дала последние концерты в городах Средней Волги и ушла в отпуск.

В ноябре 2012 года певица вышла из декрета и возобновила концертную деятельность. По сведениям журнала Forbes, в 2012 году певица в списке самых успешных российских деятелей шоу-бизнеса заняла четырнадцатое место. Сама артистка это отрицает, также как и в прошлом году, утверждая, что её доход гораздо меньше. На данный момент артистка активно гастролирует.

В 2014 году Елена Ваенга стала одним из членов жюри шоу Первого канала «Точь-в-точь».

27 ноября 2015 года состоялся сольный концерт Ваенги в Государственном Кремлёвском дворце, где она выпустила новую программу и представила новый альбом.
