\chapter{Наука и Техника}
\section{Северное сияние}
С \explainDetail{наступлением}{наступление}{adventб beginning} осени тысячи туристов \explain{устремляются}{rush} в \explain{Заполярье}{the region of the Arctic circle}, чтобы увидеть уникальный танец небесных огней -- полярное, или северное сияние, на латыни -- Aurora Borealis.
В теории увидеть это природное явление можно с конца августа до середины апреля: в этот период времени ночи становятся темными, солнечная активность \explainDetail{возрастает}{возрастать/возрасти}{возраст\'{а}ю/-ешь/-ет; возраст\'{у}/-ёшь/-\'{у}т: rise, increase}, а облака \explainDetail{расс\'{е}иваются}{рассеиваться/рассеяться}{to disperse}.
Такое развлечение, как \explain{ох\'{о}та}{hunting} за северным сиянием, с каждым годом становится все популярнее как среди россиян, так и среди иностранных туристов, которые специально ради него готовы ехать на Крайний Север. Главное \explain{доказательство}{proof} удачной охоты -- это, конечно же, снимки северного сияния.


\section{Только дьявол мог выдумать Нобелевскую премию}
% https://www.gazeta.ru/science/2015/11/27_a_7914947.shtml
Екатерина Шутова

\textit{120 лет назад Альфред Нобель подписал \explain{завещ\'{а}ние}{will} по Нобелевской премии.}

120 лет назад Альфред Нобель подписал завещание, согласно которому его \explain{накопл\'{е}ния}{accumulation} поступили в фонд Нобелевской премии -- самой престижной на сегодняшний день \explainDetail{награды}{нагр\'{а}да}{prize}, ежегодно \explainDetail{присуждаемой}{присужд\'{а}емый}{awarded} за выдающиеся научные исследования, революционные изобретения или крупный \explain{вклад}{contribution} в культуру или развитие общества. \explainDetail{Отдел}{отдел}{department} науки «Газеты.Ru» вспоминает \explainDetail{подробности}{подробность}{detail} этого события.

В 1888 году журналисты \explain{оповестили}{notified} мир о смерти Альфреда Нобеля -- химика, инженера и изобретателя динамита. Репортеры ошиблись -- на самом деле погиб Людвиг Нобель, брат Альфреда.

\begin{fancyquotes}
    Удивленный изобретатель прочитал в одной из газет собственный некролог под названием «Торговец смертью мертв».
\end{fancyquotes}

Альфред Нобель не захотел оставаться злодеем в глазах человечества. Поэтому 27 ноября 1895 года в Шведско-Норвежском клубе в Париже ученый составил следующее завещание:

{\it
Я, \explain{нижеподписавшийся}{undersigned}, Альфред Бернхард Нобель, обдумав и решив, настоящим объявляю мое завещание по поводу имущества, нажитого мною... Капитал мои душеприказчики должны перевести в ценные бумаги, создав фонд, проценты с которого будут выдаваться в виде премии тем, кто в течение предшествующего года принес наибольшую пользу человечеству.

Указанные проценты следует разделить на пять равных частей, которые предназначаются: первая часть тому, кто сделал наиболее важное открытие или изобретение в области физики, вторая --- в области химии, третья --- в области физиологии или медицины, четвертая --- создавшему наиболее значительное литературное произведение, отражающее человеческие идеалы, пятая --- тому, кто внесет весомый вклад в сплочение народов, уничтожение рабства, снижение численности существующих армий и содействие мирной договоренности.

... Мое особое желание заключается в том, чтобы на \explain{присуждение}{awarding, conferment} премий не \explainDetail{влияла}{влиять/повлиять}{influence} национальность кандидата, чтобы премию получали наиболее \explain{достойные}{worthy}, независимо от того, скандинавы они или нет.}

\subsection{Как огорчить родственников}
Спустя год после написания завещания Альфред Нобель скончался на своей вилле от \explainDetail{кровоизлияния}{кровоизлияние}{hemorrhage} в мозг. За несколько лет до смерти ученый сказал о самом себе следующим образом: «Альфред Нобель -- его \explain{существование}{existence} следовало бы \explain{пресечь}{suppress} при рождении милосердным доктором. Основные добродетели: держит ногти в чистоте и никому не бывает в тягость. Основные недостатки: не имеет семьи, наделен дурным характером и плохим пищеварением.

\begin{fancyquotes}
    Величайший грех: не поклоняется Мамоне. Важнейшие события в его жизни: никаких.
\end{fancyquotes}

\explain{Наследники}{heirs} легендарного изобретателя были крайне \explain{возмущены}{outraged}, что огромные накопления уйдут не им в карман, а на поддержку науки. Они требовали, чтобы завещание было признано недействительным. Интересно, что единственным родственником Нобеля, не пытавшимся присвоить себе деньги, оказался его племянник Эммануил. «Русские называют исполнителя завещания «душеприказчик», то есть «представитель души», --- заявил юристам мужчина. --- Вот и действуйте соответственно». Позднее Эммануил добавил: «Я не хочу, чтобы достойнейшие ученые в будущем упрекали нашу семью в присвоении средств, которые по праву принадлежат им».

В конечном итоге справедливость восторжествовала --- и через несколько лет после смерти ученого были вручены пять первых премий. А с 1969 года по инициативе Шведского банка начала присуждаться Нобелевская премия по экономике.

Лишь однажды деньги из фонда премии пошли на дело, никак не связанное с наукой. Софи фон Капивара, женщина, с которой у талантливого изобретателя были отношения, пообещала раскрыть содержание их переписки и посмертно опозорить Альфреда Нобеля. Душеприказчики в страхе выплатили крупную сумму за 216 писем мецената. Ученые до сих пор шутят, что

\begin{fancyquotes}
    наука была бы богаче, если бы не одна алчная молочница.
\end{fancyquotes}

«Ты славная девушка, но ты действуешь мне на нервы»

Существует миф, согласно которому у Альфреда Нобеля была жена, страстно влюбившаяся в математика, и именно поэтому изобретатель «обделил» всех представителей этой науки. Но на самом деле, как заявляют биографы, меценат никогда не был женат. В молодом возрасте Нобель влюбился в работницу аптеки, но та умерла от чахотки. Потосковав, ученый нашел новую пассию --- на этот раз ей стала Сара Бернар, знаменитая актриса. Альфред Нобель написал письмо матери о том, что хочет жениться.

\begin{fancyquotes}
    Недаром актеров в старину не разрешали хоронить на кладбище. У них нет души, сыночек!» --- предупредила сына любящая родительница.
\end{fancyquotes}



Послушный Нобель разорвал любовную связь с Бернар.

Следующая женщина появилась в жизни мецената, когда тому уже был 41 год. Альфред Нобель опубликовал в газете объявление о том, что ищет секретаршу. На него откликнулась графиня Берта Кински, с которой у изобретателя начался неторопливый и гармоничный роман.

\begin{fancyquotes}
    Кстати, по одной из версий, именно Кински попросила Нобеля вписать в завещание премию мира. А в 1905 году она стала первой женщиной, удостоенной этой премии.
\end{fancyquotes}

У Кински и Нобеля дело до свадьбы не дошло: однажды ученый обнаружил, что его секретарша исчезла, оставив на столе письмо следующего содержания: «Простите меня, господин Нобель. Я уезжаю в Вену, где меня ждет жених. Пожелайте мне счастья, как я желаю счастья вам. Искренне преданная вам Берта Кински, которая в скором времени станет Бертой фон Зуттер».

Последней женщиной в жизни Нобеля стала вышеупомянутая «алчная молочница», которая изрядно надоедала ученому своей глупостью и необразованностью. «Дорогое дитя. Ты славная девушка, но ты действуешь мне на нервы», --- раздраженно писал ей в письмах Альфред Нобель.

Так почему же не существует премии по математике? Возможно, все дело в том, что у Альфреда Нобеля не заладились отношения с великим математиком Миттаг-Леффлером, который должен был стать первым лауреатом, а меценат этого не хотел. Но наиболее вероятная версия заключается в том, что Нобель воспринимал математику как инструмент, как сугубо теоретическую науку.

\subsection{Виагра для хомячков и исследование ругани}

В 1991 году появилась пародия на Нобелевскую премию --- Шнобелевская премия. Она вручается «за достижения, которые заставляют сначала засмеяться, а потом --- задуматься». Учредитель и идейный вдохновитель «Шнобелевки» --- Марк Абрахамс, который, будучи редактором юмористического научного журнала, получал множество писем от читателей с подробным рассказом об их «великих» исследованиях. «Иногда эти люди заслуживали премии --- правда, не Нобелевской», --- говорил Абрахамс. Так редактор решил награждать ученых за самые нелепые достижения.

В разные годы Шнобелевская премия присуждалась

за разработку протезов яичек для собак, за исследование влияния музыки кантри на частоту самоубийств и за открытие, что «Виагра» помогает хомякам справиться с последствиями резкой смены часовых поясов.

Также пародийную награду получали ученые, доказавшие, что ругань снижает боль, и исследователи, изучавшие оральный секс у летучих мышей.

Первым в мире человеком, удостоенным как Шнобелевской, так и Нобелевской премии, стал Андрей Гейм. Голландский ученый российского происхождения был награжден «Шнобелевкой» за использование магнитов для того, чтобы демонстрировать возможность левитации лягушек. Спустя десять лет Гейм совместно со своим учеником Константином Новоселовым получил Нобелевскую премию за изобретение графена.

\subsection{Война еще не закончена, а премии уже раздают}
«Я готов простить Альфреду Нобелю изобретение динамита, но только дьявол в людском обличье мог выдумать Нобелевскую премию!» --- воскликнул ирландский романист и драматург Джордж Бернард Шоу, став лауреатом в области литературы (по ироничному заявлению писателя, произошло это потому, что «в тот год он ничего не опубликовал»). Действительно, самая престижная международная награда --- явление весьма резонансное и неоднозначное. В Советском Союзе Нобелевский комитет клеймили за то, что «он ухитрился не заметить Алексея Толстого, Максима Горького, Владимира Маяковского, но зато заметил Ивана Бунина. И только тогда, когда он стал эмигрантом, и только потому, что он стал эмигрантом и врагом советского народа».

В Третьем рейхе ученым было запрещено получать Нобелевскую премию, так как в 1935 году премию мира «За борьбу с милитаризмом в Германии» получил пацифист Карл фон Осецкий --- ярый противник нацистского режима. В 1937 году Адольф Гитлер издал указ, согласно которому немцы не имели права принимать премию. Из-за указа награду не получили Герхард Домагк «за открытие антибактериального эффекта пронтозила», Адольф Бутенандт за исследование половых гормонов и Рихард Кун за работу по каротиноидам и витаминам.


\begin{fancyquotes}
    Весьма примечателен тот факт, что Бенито Муссолини и Адольф Гитлер были номинированы на Нобелевскую премию мира в 1935 и 1939 годах соответственно.
\end{fancyquotes}

Нобелевская история знает немало случаев отказа от самой престижной международной награды.

Так, в 1973 году политический деятель Фан Динь Кхай отказался от медали «за работу по разрешению вьетнамского конфликта», аргументируя свое решение тем, что «война еще не закончена, а премии уже раздают». Не захотел быть награжденным и Жан-Поль Сартр --- французский писатель и драматург. По мнению Сартра, награда посягнет на его независимость --- центральное понятие в философии автора. Вскоре после отказа от Нобелевской премии француз еще раз шокировал общественность, заявив, что уходит из литературы. «Литература --- суррогат действенного преобразования мира», --- с горечью заметил писатель.

\section{Ученые нашли способ записать данные в пяти измерениях}

\textit{Как 5D-диски изменят представление людей о хранении информации?}

\textbf{Ученые создали 5D-диск высочайшей плотности:} В октябре специалисты Саутгемптонского университета в Великобритании описали способ записи огромного количества данных на компактный диск небольших размеров. Технология, получившая название 5D, позволяет сохранить на специальном накопителе до 500 терабайт информации. Получившиеся диски из кварцевого стекла отличаются высочайшей плотностью, которая в десять тысяч раз превышает плотность оптических дисков Blu-Ray. Новый метод позволит эффективно разместить на небольшой площади облачные сервера для хранения данных пользователей, интернет-компаний, крупных корпораций. По словам ученых, это особенно важно на фоне развития технологий, увеличения количества подключенных к сети устройств и роста количества передаваемых через сеть данных.

\textbf{Облачные сервисы с каждым годом становятся все популярнее:}
За последние пять лет отношение потребителей и бизнеса к облачным сервисам изменилось. Раньше их воспринимали в качестве дополнительного метода резервного копирования данных --- информация практически всегда поступала в одну сторону. Причем крупные корпорации в основном использовали дата-центры для хранения некритичной информации. К 2020-м годам организации стали использовать облачные серверы не только для аварийного копирования, но и для постоянного обмена данными внутри конкретного предприятия. Системы облачных хранилищ стали более гибкими, позволяя конкретному потребителю выбрать необходимое количество свободного места и производительность оборудования.

Специалисты Analytics Insight называют основными \explainDetail{преим\'{у}ществами}{преим\'{у}щество}{advantage} \explainDetail{облачных}{облачный}{cloud (adj.); \'{о}блако: cloud} дата-центров \explain{круглос\'{у}точный}{round the clock} доступ к информации, возможность одновременной работы не- скольких пользователей с одним массивом данных, масштабируемость и \explain{г\'{и}бкость}{flexibility}, \explain{снижение}{decline} затрат на хранение данных внутри компании.

\explainDetail{Представители}{представитель}{representative} отрасли отмечают, что в обычное время нагрузка на серверы неравномерна: в одной части дата-центров она может зашкаливать, в другой быть крайне небольшой. По этой причине эксперты предсказывают появление искусственного интеллекта, который мог бы анализировать и распределять нагрузку на оборудование. В том числе по этой причине данные пользователей хранятся в нескольких частях дата-центра.

\textbf{Больше всего в облачных сервисах пользователи ценят скорость передачи данных и безопасность:} По словам основателя облачного провайдера Wasabi Дэйва Френда, от дата-центров будущего потребители ожидают высокого уровня безопасности, производительности оборудования и приемлемой цены за услуги. «Резервные копии должны храниться в разрозненных системах, обеспечивающих максимально возможную изоляцию», --- заметил предприниматель. Потенциальный злоумышленник, добравшийся до одного сервера, не должен иметь возможность удалить или зашифровать информацию так, чтобы ее нельзя было восстановить из альтернативных источников. Френд полагает, что на этом должна строиться концепция мультиоблака.

Другими критериями облачного сервиса будущего, по мнению Френда, являются доступная цена и высокая скорость передачи данных. Провайдеры должны будут таким образом скорректировать стоимость услуг и добиться определенного качества оборудования, чтобы оставаться конкурентоспособными и не разочаровывать клиентов.

Представители облачного провайдера CloudSigma рассказали, что дата-центры должны будут отвечать за сохранность данных и скорость передачи информации. Для хранения файлов пользователей и корпоративных клиентов они используют небольшие 2,5-дюймовые диски емкостью 250 гигабайт. В случае, если какой-либо диск выходит из строя, его заменяют, а данные восстанавливают через бэкап. При таком развитии событий клиент не теряет своих данных, хотя и оказывается без доступа к информации на 10-15 минут. Благодаря глубокой интеграции между серверами и оборудованием задержка передачи данных внутри дата-центра очень мала. Для того чтобы разогнать скорость и снизить задержку между серверами и пользователем, в компании полагаются на выделенную гигабитную линию интернета.


\textbf{Диски 5D позволят хранить информацию практически бесконечно:}
По оценке Forbes, к 2025 году к интернету будет подключено около 80 миллиардов устройств, которые будут генерировать около 180 триллионов гигабайт данных. В обозримом будущем хранить данные на классических накопителях будет проблематично --- существует риск возникновения дефицита и увеличения стоимости хранения информации. Работающие над технологией 5D специалисты Саутгемптонского университета предлагают записывать информацию на кварцевом стекле с помощью фемтосекундных лазеров и сверхкоротких импульсов. «Запись на кварцевый носитель как бы идет в пяти измерениях --- двух оптических и трех пространственных», --- отмечают авторы исследования.

Инновация британских инженеров заключается в создании дисков повышенной плотности и размещении на небольшом участке колоссальных объемов данных. Например, на «болванке» размером в один дюйм удалось сохранить шесть гигабайт информации. Накопитель обычного для подобных устройств размера, основанный на кварцевых дисках, может сохранить до 500 терабайт данных. Разработка обещает революцию на рынке хранения информации, так как десятки, если не сотни классических дата-центров можно будет объединить в одну библиотеку.

Преимуществами 5D-дисков также называют долговечность и низкую стоимость обслуживания. По оценке ученых, кварцевые диски не прочнее обычных накопителей, однако могут выдержать температуру до 1800 градусов по Фаренгейту, или около тысячи градусов по Цельсию. В случае пожара в дата-центре информация, скорее всего, сохранится. Кроме того, кварцевое стекло со временем не меняет своих свойств, что позволит держать данные на 5D-накопителях практически вечно.

Единственным узким местом будущей разработки является скорость передачи данных. В настоящий момент ученым удалось разогнать ее до 230 килобайт в секунду --- за это время на диск можно записать около ста страниц текста. Однако для того, чтобы полностью заполнить болванку емкостью 500 терабайт, потребуется 60 дней. Либо инженеры найдут способ обойти ограничение, либо 5D-диски так и останутся перспективным оборудованием для записи информации. В крайнем случае на подобных дисках можно сохранять данные для потомков. Так, в 2018 году на кварцевый носитель была записана трилогия романов Айзека Азимова «Основание» --- диск отправился в космос вместе с Tesla Roadster Илона Маска.



\section{У коронавируса есть механизм самоуничтожения}
% https://russian.rt.com/science/article/937351-pyotr-chumakov-intervyu-virusy
\textit{Пётр Чумаков об эволюции SARS-CoV-2 и лечении вирусами}

\explainDetail{Снижение}{снижение}{decrease} патогенности новых \explainDetail{штаммов}{штамм}{strain (of virus)} SARS-CoV-2 \explain{предопределено}{predetermined} самой логикой эволюции вирусов такого типа. Об этом в интервью RT рассказал член-корреспондент \explain{РАН}{Российская академия наук}, профессор и главный научный сотрудник Института молекулярной биологии РАН Пётр Чумаков. Он также отметил, что многие вирусы можно поставить на службу человеку: например, использовать их непатогенные варианты для защиты людей от опасных \explainDetail{возбудителей}{возбудитель}{pathogen}. За вакцинами, основанными на этом принципе, будущее, считает учёный. По его мнению, вирусы можно использовать и для лечения рака.

{\bf --- С самог\'{о} \explain{начала}{from начало (\textit{суш.}): start, beginning} пандемии многие боялись появления \explain{ос\'{о}бо}{especially, particularly} \explainDetail{смертоносного}{смертоносный}{deadly} штамма вируса. Почему этого пока не случилось? По \explainDetail{предварительной}{предварительный}{preliminary} информации, новый штамм «омикрон» не привёл к \explainDetail{резкому}{резкий}{sharp} росту \explainDetail{летальности}{летальность}{lethality}. Да и в целом опасные мутации \explainDetail{распространённых}{распространённый}{widespread} вирусов случаются не очень часто --- например, даже грипп вызвал смертельную пандемию только один раз, в начале XX века.}

--- Коронавирус SARS-CoV-2 --- новая для человека инфекция. Она перешла в человеческую популяцию только два года назад. До этого коронавирус этого типа циркулировал в основном среди \explainDetail{летучих мышей}{летучая мышь}{bat}. Эти животные \explainDetail{обладают}{обладать}{possess} очень сильной противовирусной защитой, организм летучих мышей заточен для противостояния инфекциям, поскольку они живут в очень \explainDetail{скученных}{скученный}{crowded} колониях. Вирусы, которые способны \explain{пробить}{pierce} эту защиту, должны также быть \explain{вооружены}{armed} очень серьёзными системами преодоления противовирусного иммунитета. И когда такой вирус попадает к человеку, он вызывает тяжёлые \explainDetail{заболевания}{заболевание}{disease}, потому что человеческая иммунная противовирусная система слабее, чем у летучих мышей.

Однако, попав в организм человека, такой вирус тоже должен \explain{приспособиться}{adapt} к новым условиям. Поэтому первые фазы эволюции вируса --- это \explain{накопление}{accumulation} таких мутаций, которые будут приводить к его более эффективному \explainDetail{размножению}{размножение}{reproduction} в организме человека. Это \explain{сопровождается}{is accompanied by} ростом инфекционности и усилением патогенных \explainDetail{свойств}{свойство}{property}. Когда вирус активно размножается, он приспосабливается к организму человека и действует более эффективно.

Вторая фаза эволюции вируса --- аттенуирование. Это приспособление вируса к организму при \explainDetail{ослаблении}{ослабление}{weakening} его патогенности.

При этом инфекционность может расти, потому что вирусу важно быстро \explainDetail{распространяться}{распространяться/распространиться}{spread} на новом хозяине. Однако \explain{излишняя}{superfluous, excessive} патогенность ему не нужна. Дело не в том, что вирус \explain{якобы}{ostensibly} знает, что ему невыгодно убивать человека. Нет, просто это качество --- патогенность --- не востребовано в организме человека и поэтому постепенно ослабевает при накоплении мутаций. В результате вирус вызывает всё меньше летальных исходов и тяжёлых случаев.

\begin{fancyquotes}
    Сейчас штамм коронавируса «омикрон» является примером второй фазы эволюции вируса, \explain{наблюдается}{is observed} его аттенуирование при одновременном увеличении \explain{заразности}{infectiousness}. Итогом должно стать превращение коронавируса в обычное сезонное вирусное заболевание.
\end{fancyquotes}

В случае с «омикроном» примечательна внезапность его появления, он сразу накопил 32 мутации по сравнению с предыдущим штаммом. В Африке очень много иммунодефицитных людей, и, по всей видимости, новый штамм \explainDetail{возник}{возникать/возникнуть}{arise (возн\'{и}к, возн\'{и}кла, возн\'{и}кло, возн\'{и}кли)} в организме именно такого человека. В его организме он прошёл тот путь эволюции, который обычно вирус проходит через большую \explainDetail{цепочку}{цепочка}{chain} \explainDetail{заражений}{заражение}{infection}. В итоге в ускоренном режиме этот вирус превратился в менее патогенный вариант.

Впр\'{о}чем, я бы не хотел никого \explainDetail{расхолаживать}{расхолаживать/расхолодить}{discourage}, говоря о том, что бояться нечего. Нет. Пока что это --- лишь оптимистичный сценарий. Мы пока не имеем достаточного числа случаев заболевания этим штаммом, чтобы заявлять о его низкой опасности. Да, по предварительным данным, пока что от него никто не умер. Но нужно понаблюдать за тем, как будет развиваться ситуация, и продолжать вакцинироваться. Потому что даже если оптимистичный сценарий верный, нельзя исключать, что штамм «омикрон» неожиданно исчезнет, как исчез в Японии штамм «дельта».

\explainDetail{По видимости}{по видимости}{apparently}, у коронавируса есть механизм самоуничтожения. Это может случиться и со штаммом «омикрон», а ему на смену придут более \explain{болезнетворные}{pathogenic} варианты.

{\bf --- Как вакцинация \explainDetail{влияет}{влиять/повлиять}{influence} на мутации вируса --- насколько она снижает их вероятность, \explain{учитывая}{considering, taking into account}, что вакцинированные тоже болеют, пусть и в лёгкой форме?}

--- \explainDetail{Особенность}{особенность}{peculiarity} этого коронавируса в том, что даже очень иммунные люди, которые перенесли заболевание или вакцинированы, могут заразиться \explain{повт\'{о}рно}{repeatedly}, даже не чувствуя при этом симптомов. При этом вирус будет выделяться из носоглотки. Однако вероятность мутаций вируса в такой ситуации всё же не очень велика. Гораздо чаще новые варианты вируса возникают в организмах больных людей, особенно иммунодефицитных.

{\bf --- Вы \explainDetail{упомян\'{у}ли}{упомин\'{а}ть/упомян\'{у}ть}{to mention, to refer} феномен \explainDetail{исчезновения}{исчезнов\'{е}ние}{disappearance} «дельты» в Японии. Не могли бы рассказать об этом \explain{подробней}{in more detail (detail: подробность)}? Случалось ли подобное когда-то раньше?}

--- Нет, это новая гипотеза одного из японских исследователей. Гипотеза, на первый взгляд, экстравагантная. Потому что обычно эволюция идёт таким путём, что \explain{выживает}{survives} сильнейший --- наиболее \explain{жизнеспособный}{viable}. А в этом случае произошло, напротив, эволюционное самозатухание вируса. Согласно гипотезе, это связано с мутацией в гене NSP-14. Это один из неструктурных \explainDetail{белков}{бел\'{о}к}{protein} коронавируса, не входящий в состав вирусной \explainDetail{част\'{и}цы}{част\'{и}ца}{particle}. Он нужен для поддержания репликации вируса, корректирует правильность считывания генома, исправляет ошибки. Если этот белок не функционирует, то вирус начинает с большой \explainDetail{скоростью}{скорость}{speed} накапливать мутации, включая летальные. Они приводят к тому, что вирус уже не может размножаться.

Не знаю, \explain{подтвердится}{confirmed} ли эта гипотеза, однако, когда в Японии секвенировали варианты коронавируса до того, как он исчез, оказалось, что там действительно были мутации белка NSP-14.

Более того, предыдущая \explain{вспышка}{outbreak} SARS-1 в 2003 году тоже \explain{затухла}{faded} сама по себе, её даже не успели подавить вакциной.

Что касается «омикрона», я пока не видел никаких данных о том, что у него \explain{повреждён}{damaged} белок NSP-14. Однако этого нельзя исключать, это объяснило бы скорость накопления в нём мутаций.

{\bf --- Ещё говорят, что штамм «омикрон» очень сильно отличается от других штаммов и что в нём нашли элемент генома человека, который также есть в вирусе \explainDetail{простуды}{простуда}{common cold}. И что это делает его менее заметным для иммунной системы.}

--- Нет, что это элемент генома человека --- это ерунд\'{а}. Это маленькая вставка. И когда мы говорим о \explainDetail{посл\'{е}довательности}{посл\'{е}довательность}{sequence}, \explain{допуст\'{и}м}{let's say}, трёх \explainDetail{аминокислот}{аминокислот\'{а}}{aminoacid}, такая посл\'{е}довательность может встречаться и в человеке, и в растении --- \explain{где угодно}{anywhere}.

{\bf --- Нед\'{а}вно вы сказали, что омикрон-штамм может выступить в качестве естественной вакцины. А были такие прецеденты раньше?}

---  Может быть, были, но они не зафиксированы. Однако что такое «живая вакцина»? Это просто вирус, который не имеет высокой патогенности, но \explain{вызыв\'{а}ет}{causes} иммунный ответ. Обычно такой непатогенный штамм делают искусственно. Например, полиомиелитная вакцина была создана путём селекции, были отобраны вирусы, которые \explainDetail{утратили}{утрачивать/утратить}{to lose} способность \explainDetail{поражать}{пораж\'{а}ть/пораз\'{и}ть}{to hit, to affect} нервные \explainDetail{кл\'{е}тки}{кл\'{е}тка}{cell}. При этом они хорошо размножаются в \explainDetail{кишечнике}{кишечник}{intestine} и вызывают \explain{стойкий}{persistent} иммунитет против полиомиелита. Живая вакцина против \explainDetail{кори}{корь}{measles} тоже была создана на основе патогенного вируса кори, который приобрёл свойства непатогенности.

Природа тоже может создавать такие вирусы. И природная аттенуация вирусного штамма --- это, \explain{по с\'{у}ти}{in fact}, именно такой процесс. Такой вирус \explain{вытесняет}{displaces} патогенные варианты и создаёт иммунную прослойку, которая уже не \explainDetail{позволяет}{позвол\'{я}ть/позв\'{о}лить}{to allow, to permit (also: разреш\'{а}ть/разреш\'{и}ть)} новому патогенному варианту вызвать заболевание.

{\bf --- Ранее в Университете Глазго провели исследование, в результате которого \explain{в\'{ы}яснилось}{it turned out}, что люди очень редко болеют одновременно двумя вирусными заболеваниями. Однако точное объяснение этого тогда найти не удалось. Есть ли сейчас какие-то гипотезы на этот счёт? И можно ли использовать «конкуренцию» вирусов в полезных целях, существуют ли подобные проекты?}

--- Да, есть такое правило, хотя из него тоже бывают исключения. В основе этого явления лежит феномен интерференции. Когда человек или животное \explain{заражается}{gets infected (+\textit{твор.})} вирусом, в ответ в организме \explain{выраб\'{а}тывается}{produced} интерферон --- противовирусный бел\'{о}к. Он циркулирует по всем\'{у} организму, клетки в ответ вырабатывают противовирусное состояние. И другому вирусу будет уже очень трудно внедриться в организм в это время. Однако некоторые вирусы вырабатывают противодействие интерфероновому механизму --- тогда возможна сочетанная инфекция. Возможно также одновременное заражение двумя вирусами, когда первый вирус ещё не вызвал противовирусное состояние.

{\bf --- А учёные думали над тем, чтобы как-то использовать этот интерфероновый механизм для борьбы с болезнями?}

--- Конечно. Такие исследования проводились в 1970-е годы, я принимал в них участие. Есть целый ряд непатогенных вирусов, которые обычно \explain{обитают}{inhabit} в кишечнике здоровых детей в возрасте от двух до пяти лет. Их испытывали как средство профилактики сезонных простудных вирусных инфекций. В начале 1970-х годов было проведено \explain{масштабное}{large-scale} испытание более чем на 300 тыс. человек в шести городах СССР. В итоге заболеваемость \explain{ОРВИ}{острая респираторная вирусная инфекция: acute respiratory viral infection} среди привитых снизилась в 3,5 раза. Это очень хороший результат, такой же, как у специфических противовирусных противогриппозных вакцин.

\begin{fancyquotes}
    Примечательно, что профилактика, основанная на механизме интерференции, защищает не только от одного вируса, а от многих. Поэтому такая неспецифическая профилактика очень важна именно в случае появления новых инфекций, чтобы выиграть время до создания специфической вакцины.
\end{fancyquotes}

Тем более что такие препараты применяются в виде \explain{капель}{drops}, \explain{перорально}{orally}. Это не будет вызывать такого \explainDetail{отторжения}{отторжение}{rejections} у антиваксеров.

{\bf --- А сейчас разрабатываются новые препараты с этим принципом действия?}

--- У нас есть панель таких вирусов. Но проблема в том, что такие вещи с большим трудом пробивают себе дорогу. Например, нам не удалось \explainDetail{убед\'{и}ть}{убеждать/убедить}{to convince [убеждать: убеждаю, убеждаешь, убеждают]} использовать этот метод во время \explain{нынешней}{current} пандемии. Потому что наш подход очень дешёвый, он не очень интересен для бизнеса. Ведь на выпуск вакцин выделяются большие деньги.

Но я уверен, что мы вступаем в такое время, когда вирусы могут использоваться и в качестве оружия. Поэтому нужно иметь стратегический \explain{зап\'{а}с}{stock, reserve} \explain{экстренных}{emergency} средств защиты. И надеюсь, что в итоге собранный нами запас непатогенных вирусов будет использован и признан в качестве такого средства.

{\bf --- Раз вирусы могут \explain{конкурировать}{compete} и вытеснять друг друга, не рассматривается ли научным сообществом, к примеру, идея создать генно-инженерными методами не опасный, но очень контагиозный штамм SARS-CoV-2, чтобы вытеснить те его штаммы, которые приводят к высокой летальности? }

--- Это отличный вопрос. Вообще я считаю, чтобудущее вакцинологии --- это жив\'{ы}е вакцины. Просто сейчас мы идём \explain{проторённым}{well-trodden} путём, создаём традиционные вакцины. Либо это инактивированный вирус, либо какие-то генные инженерные продукты --- на основе других вирусов.

При этом аттенуированные штаммы можно создать из л\'{ю}бого болезнетворного вируса. Зная, какие гены участвуют в патогенезе, можно внести искусственные изменения и сделать из него такой безопасный вирус, который будет тем не менее иммуногенен и будет защищать от инфекции.

Но, к сожалению, пока что этот путь не всеми признаётся, широких работ в этом направлении не ведётся.

{\bf --- В 2020 году в интервью RT вы рассказали о \explainDetail{разработке}{разработка}{development} модифицированных вирусов, которые вызывают \explain{гибель}{death} раковых клеток. Мы хотели бы вернуться к этой теме. Готовятся ли клинические испытания или они уже проводятся?}

--- Да, эта работа продолжается, у нас есть большая панель онколитических вирусов. Почему нужна целая панель, а не один препарат? Потому что \explainDetail{опухоли}{\'{о}пухоль}{tumour} у людей очень индивидуальны. Допустим, у двух людей рак молочной \explainDetail{железы}{желез\'{а}}{gland} одной и той же гистологической категории. Но на молекулярном уровне это совершенно разные заболевания, там разный набор повреждений. \explain{В том числе}{including} повреждаются такие гены, которые могут быть нужны какому-то конкретному вирусу для его репликации. И он уже не может лечить данную \'{о}пухоль. Но может другой --- его нужно подобрать из панели.

Конечно, такой сложный препарат трудно быстро \explain{внедрить}{to implant, to embed, to introduce}, нужно пройти долгие этапы пров\'{е}рок. К сожалению, регуляторика --- \explain{узкое}{narrow} место для биотерапии. На самом деле правила нужно очень сильно менять, чтобы создание и внедрение таких препаратов происход\'{и}ло быстр\'{е}й.

Пока что \explain{проведены}{$<$ провести} только \explain{доклинические}{pre-clinical} испытания, их итоги подводятся. Возможно, что к концу г\'{о}да будет какое-то заключение. После этого встанет вопрос о том, как организовать клинические испытания. Нужно будет найти инвесторов, собрать пациентов-добровольцев\dots{} Всё это \explain{затягивается}{drags on} на годы.

Тем не менее по своему опыту мы знаем, что эта терапия действует и абсолютно безопасна.

{\bf --- Удалось ли собрать какую-то предварительную статистику?}

--- Собрать корректную статистику пока невозможно, потому что все случаи, с которыми мы имели дело, --- это случаи четвёртой стадии заболевания. Просто потому, что только на этой стадии возможны такие эксперименты, когда уже \explain{исчерпаны}{exhausted} все традиционные возможности. Иногда даже не успеваем препарат дать, как человек уже умер.

{\bf --- Люди соглашаются принимать экспериментальный препарат, потому что им нечего уже терять?}

--- Да. Но всё равно такие вещи находятся в «серой» зоне \explainDetail{законодательства}{законодательство}{legislation}. Вообще-то так нельзя делать, но мы всё равно делаем. И какая тут может быть статистика? Мы видим только отдельные сл\'{у}чаи, когда действительно это лечение помогает, когда люди долго живут. И есть много случаев, когда улучшается состояние. Это не значит, что человек \explain{выздоровел}{recovered}. Потому что после, предположим, 16 курсов химиотерапии ресурсы организма всё равно уже сильно \explain{ограничены}{limited (огран\'{и}чен, -а, -о)}. А статистику мы получим, когда проведём клинические испытания.

{\bf --- Букв\'{а}льно в двух словах напомните, пожалуйста, о принципе действия таких препаратов.}

--- Принцип такой: \explain{\'{о}пухолевые}{tumour (\textit{прилагательное})} клетки \explain{высокочувствительны}{highly sensitive} к любым вирусам. Потому что опухолевая клетка --- это не часть организма, это уже новый одноклеточный организм внутри организма. Который начинает конкурировать там со своим хозяином, развиваться в виде \'{о}пухоли. По мере своей эволюции он утрачивает свойства, нужные для поддержания собственно организменных функций, включая механизмы противовирусной защиты клетки. Онколитический вирус --- это вирус, не вызывающий заболевание, поэтому его можно использовать для лечения рака. Однако не каждый вирус может конкретную опухоль уничтожить. Поэтому надо иметь много разных вирусов и подбирать их под конкретного пациента.

Внутри опухоли формируется иммуносупрессивное состояние --- иммунная система не может туда \explain{проникнуть}{penetrate}. Но когда в опухоли начинает размножаться вирус, возникает \explain{воспаление}{inflammation}, которое сопровождается выработкой массы белковых факторов. Они \explain{привлекают}{attract} в опухоль компоненты иммунной системы, которые начинают усиленно атаковать раковые клетки и довершают действие вируса. Это более-менее естественный способ уничтожения раковых клеток, поэтому он практически не даёт \explain{побочных}{collateral} эффектов, \explain{в отличие от}{unlike} химии.

{\bf --- В одной из своих лекций вы рассказывали, что люди начали обращать внимание на позитивное влияние вирусных заболеваний на онкологических больных около ста лет назад. Может быть, с этим явлением могут быть \explain{отч\'{а}сти}{partly} связаны истории о неожиданном и чудесном излечении от рака?}

--- Такие случаи трудно задокументировать, для этого нужно было бы изучить \explain{антител\'{а}}{antibodies} в крови таких пациентов. Хотя, конечно, за такими случаями стоят какие-то механизмы, не исключено, что и какой-то вирус

{\bf --- Где-то в мире уже используются такие препараты для лечения пациентов?}

--- Сейчас во всём мире \explain{наблюдается}{there is observed} бум этого направления исследований. Но, к сожалению, сейчас каждый разработчик делает один препарат на основе одного вируса. И в итоге выясняется, что он эффективен только для 15---20\% пациентов. Мы единственные используем целую панель вирусов, в этом наше преимущество. В США есть препарат на основе рекомбинантного вируса \explainDetail{герпеса}{г\'{е}рпес}{herpes}, который применяется сейчас для лечения развитых форм меланомы. Однако \explain{прорыва}{прор\'{ы}в}{breakthrough} в лечении он не дал --- именно по той причине, что одного вируса мало, нужно в каждом случае перебирать варианты. Либо вводить коктейль. Мы считаем, что эффект\'{и}внее всего использовать коктейль из трёх-пяти разных вирусов.

{\bf --- Когда препараты такого типа пройдут все испытания, как будет проводиться лечение? \explainDetail{Придётся ли}{Придётся ли \textit{кому}\dots{}?}{Is one going to\dots{}} пациентам ездить в какой-то один центр или можно будет внедрить эту терапию по всей стране?}

--- Надеюсь, что будут клиники, отделения, которые будут специализироваться на такой терапии. Есть много способов введения препарата --- нужно выбирать в зависимости от формы рака. Это станет огромным направлением для исследований, для терапии.

{\bf --- А как сейчас настроено сообщество врачей-онкологов, они ждут появления на рынке таких препаратов?}

--- Среди специалистов есть понимание, что онкология \explain{зашла в туп\'{и}к}{reached a dead end} в вопросе лечения развитых форм рака. Поэтому люди с большим энтузиазмом воспринимают все новые возможности, ждут, когда препараты пройдут испытания и регистрацию.


