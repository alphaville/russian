% \chapter{Транспорт}

\section{Дорожной концессии нужен сигнал}

\textit{На какие проекты надо делать ставку при строительстве трасс с привлечением частного капитала}

% https://www.vedomosti.ru/industry/infrastructure_development/articles/2023/03/29/968707-dorozhnoi-kontsessii-nuzhen-signal
\textit{Источник: \url{https://www.vedomosti.ru/industry/infrastructure_development}}

\textit{Олеся Ошанина}

Концессионные соглашения в дорожном строительстве давно уже стали классикой, поскольку реализация столь капиталоемких проектов возможна лишь с государственной поддержкой. В нашей стране строительство дорог регионального значения является наиболее перспективной формой государственно-частного партнерства (ГЧП), однако отсутствие четких сигналов от федеральных властей о том, какие проекты являются приоритетными, является препятствием для активного строительства дорог с привлечением частного капитала.

В условиях современной экономики развитие партнерских отношений государства и частного сектора считается оптимальным способом повышения эффективности использования государственной собственности. Одной из важнейших форм ГЧП являются концессии. Их главным плюсом является возможность привлечения внебюджетных инвестиций и ресурсов в государственный сектор, в то время как концессионер получает возможность эксплуатировать объект и получать доход в свою пользу.

\textbf{Дорогие дороги: объединение капиталов}

ГЧП и концессии в автодорожном строительстве – это мировая классика, указывает директор группы по привлечению финансирования Kept Сергей Игнатущенко. «Инвестор строит дорогу, зная, что 30 лет ему потом эту дорогу содержать и ремонтировать, поэтому построит качественно, – указывает эксперт. – Более того, считается, что дорожные концессии на горизонте жизненного цикла проекта дешевле государственного заказа, поскольку инвестор берет на себя риски уложиться в смету и риски эксплуатации – именно инвестор умеет такими рисками профессионально управлять». Именно в дорогах наиболее просто спрогнозировать и оценить экономический эффект для всех участников проекта, при этом не имеет значения – взимается плата за провоз груза или за проезд автомобиля, соглашается генеральный директор ГЧП.РФ Виталий Нефедов.

\begin{fancyquotes}
    \textbf{ГЧП – цифры и факты.} Согласно аналитическому обзору Национального центра ГЧП, в России по состоянию на конец 2022 г. было 3724 проекта ГЧП, среди которых 2720 были в сфере коммунально-энергетической инфраструктуры, 652 – социальной инфраструктуры. Наибольший объем инвестиций – 2761 млрд руб. – был вложен в проекты регионального уровня, еще 1763 млрд руб. – федерального и 888 млрд руб. были на муниципальном уровне. Общий объем инвестиций по результатам прошлого года достиг 5,4 трлн руб., из которых 3,9 трлн – это частные деньги.
    В 2022 г. объем инвестиций в проекты, прошедшие коммерческое закрытие, достиг 370 млрд руб. Количество проектов, запущенных за год, по разным формам ГЧП достигло 150.
    Наиболее популярной в стране формой ГЧП является концессия – из 3648 проектов ГЧП 2933 имеют форму концессионных соглашений. Больше всего денег привлекается в транспортные проекты. Они наиболее капиталоемкие: на 160 проектов приходится 3,1 трлн руб., из которых 1,83 трлн частные.
\end{fancyquotes}



Автодорожные проекты очень капиталоемки, поэтому без объединения частного и государственного капиталов невозможно обеспечить устойчивое развитие дорожной инфраструктуры и мостов, указывают эксперты. Сроки реализации проекта также могут быть существенными – от двух и до 5–10 лет.

Частная сторона в ГЧП отвечает за проектирование, финансирование, строительство или реконструкцию объекта инфраструктуры, а также участвует в его последующей эксплуатации и техническом обслуживании. «Преимущество концессий для государства и населения – это возможность получить общественно значимый инфраструктурный объект раньше, чем у государства появится возможность самостоятельно его создать и эксплуатировать, – указывает старший менеджер группы юридических услуг группы компаний Б1 Анжелика Бурдейная. – При этом за счет механизмов контроля и штрафов, обычно предусматриваемых в концессионных соглашениях, технические характеристики и качество эксплуатации часто выше, чем могли бы быть без привлечения инвесторов». Для инвесторов концессионное соглашение или соглашение о ГЧП/МЧП интересно сбалансированным распределением рисков между сторонами (например, по сравнению с государственным контрактом), резюмировала она. «Концессия также дает инвестору возможность структурировать проект на более комфортных относительно 44-ФЗ условиях, а также получить финансирование на более выгодных относительно классических инвестиционных проектов условиях, так как при ГЧП риски, с точки зрения банков-кредиторов, существенно ниже», – говорит Нефедов.

\textbf{Выгодно всем}

На сегодняшний день часть региональных проектов может быть реализована за счет привлечения средств частных инвесторов лишь при условии, что проект получает федеральную поддержку. Одной из форм такой поддержки является межбюджетный трансферт, направляемый субъекту РФ на реализацию концессионных проектов в отношении автомобильных дорог. Средства направляются регионом на частичное софинансирование расходов на строительство объектов.

При этом именно концессионные региональные проекты являются оптимальным инструментом для привлечения частных денег, потому что обеспечивают интересы всех участников, так называемый win-win, когда выгодно всем. Например, для инвесторов в них предусмотрены механизмы защиты, субъекты РФ могут за счет частных денег и средств из федерального бюджета построить необходимый для развития региона дорожный объект. Для государства в целом важно, что средства федерального бюджета будут выделены на уже проработанный проект, по которому участники готовы начать строительство, и выполнены все предварительные для этого условия (проработаны условия проекта / заключено соглашение на конкурентных условиях / осуществлено проектирование за счет частных инвестиций / предоставлены земельные участки и т. д.). Такой подход позволяет Федерации решать глобальные задачи без существенных организационных усилий со стороны Федерации на ранних этапах его реализации.

Из наиболее ярких примеров региональных концессионных проектов, реализуемых с привлечением средств федерального бюджета и частных инвесторов, – обход Хабаровска, мост через реку Обь в Новосибирске.

При этом на сегодняшний день есть некоторые проблемы при реализации. По словам Игнатущенко, во многих регионах нет большого опыта ГЧП-проектов, именно поэтому так важна работа по формированию соответствующих компетенций региональных полномочных органов в структурировании, «упаковке» проектов и грамотном распределении рисков. «Качественная подготовка ключевых транспортных проектов, включая обоснование важности для развития экономического потенциала не только отдельного региона, но и прилегающих территорий / субъектов Федерации, повышает привлекательность проекта как для частных инвесторов, так и для федерального центра (при распределении бюджетных инвестиций)», – указал эксперт.


\textbf{На помощь приходят профессионалы}

Чтобы сделать процесс участия бизнеса в ГЧП менее рисковым и снять опасения частных инвесторов, что он не будет реализован, необходимо привлечь надежных партнеров, которые уже имеют соответствующий опыт. Например, инфраструктурный «МЕГАИГРОК» Газпромбанка представляет собой «внутренний консорциум» дочерних и зависимых компаний, которые совместно участвуют в процессе реализации проекта ГЧП.

Банк не просто предоставляет финансирование – он скорее выступает как комплексный игрок рынка ГЧП, который связывает всех участников проектов ГЧП: государство, инвесторов, строительные компании, операторов, банки. «МЕГАИГРОК», как инициатор проекта, за счет комплексной экспертизы и опережающего финансирования на этапе структурирования и проектирования обеспечивает доведение проекта до состояния готовности к выделению внебюджетных средств, федеральной поддержки и началу строительства. На последующих этапах банк обеспечивает контроль финансовых потоков и бесперебойности финансирования, осуществляет банковское сопровождение средств, а также контроль надлежащего строительства объектов (включая контроль качества поставляемых материалов, машин, оборудования и т. д.).

«МЕГАИГРОК» в том числе анализирует возможность продажи инвестиций на фондовом рынке, например с помощью выпуска облигаций. Кроме того, одна из целей «МЕГАИГРОКА» – превращение инфраструктурных сделок в сделки M\&A по мере становления рынка.

Ранее первый вице-президент Газпромбанка Павел Бруссер рассказывал, что благодаря «МЕГАИГРОКУ» на рынке появился некий инфраструктурный конвейер, в котором высвобождающиеся после реализации проектов средства оперативно направляются на создание новых проектов. Потоковое строительство инфраструктурных объектов, особенно транспортных, способствует росту экономики и ее структурной перестройке, указывал он.


\begin{fancyquotes}
    \textbf{Выходим на новый уровень}

    Исполнительный вице-президент – начальник департамента структурирования инфраструктурных проектов и государственно-частного партнерства Газпромбанка Иван Потехин считает, что для выхода на новый уровень применения ГЧП для строительства дорог в регионах требуется изменить роль «МЕГАИГРОКА» с финансового партнера до лидера. Таким образом, обеспечить ускоренное начало строительства инфраструктурных проектов в целях развития дорожной инфраструктуры и мостов позволит комплексная экспертиза «МЕГАИГРОКА» и опережающее финансирование.
    В этой схеме «МЕГАИГРОК» выполняет следующие роли (самостоятельно или привлекая необходимых профильных игроков и партнеров):
    \begin{enumerate}
        \item
        \item инициирование и структурирование обслуживаемых проектов;
        \item обеспечение заключения соглашения и организация проектирования объекта;
        \item доведение проекта до готовности для привлечения банковского финансирования и выделения федеральных средств;
        \item участие в организации и контроле надлежащего строительства объекта, а также последующей эксплуатации объекта.
    \end{enumerate}
\end{fancyquotes}

Уникальная экспертиза и накопленный опыт «МЕГАИГРОКА» позволяют структурировать с выгодой для всех сторон самые сложные и капиталоемкие проекты, например северный обход Перми, дублера Егорьевского шоссе, северный обход Омска, строительство автодороги Солнцево – Лыткарино – Железнодорожный, суммарный объем инвестиций по которым превышает 400 млрд руб. Однако для эффективной реализации все эти проекты требуют федерального софинансирования, отмечают в Газпромбанке.

\textbf{Чтобы средства выделялись быстрее}



По словам экспертов, на практике федеральный бюджет выделяет средства только после того, как все параметры проекта утверждены и завершено проектирование. К этому моменту участники уже потратили не только время, но и деньги на структурирование проекта и проектирование без каких-либо гарантий со стороны федерального центра о последующем выделении средств. «На сегодняшний день наиболее активные регионы привлекают квалифицированных экспертов концессионного рынка для формирования параметров автодорожных проектов с целью участия в отборе (конкурсе) на федеральную поддержку, – указывает Нефедов. – Таким образом, государство благодаря уже имеющимся механизмам получает качественно подготовленный проект для дальнейшего принятия решения о выделении федеральных средств на его реализацию».

Однако такой порядок чреват риском для инвестора, ведь если в федеральном софинансировании будет отказано, то затраты станут невозвратными.



Решением проблемы могла бы быть выработка прозрачного механизма взаимодействия инвесторов, субъектов и федерального центра на всех этапах. Например, государство может утверждать перечень проектов, к реализации которых планируется привлекать частный капитал, предлагают в Газпромбанке. Также важно, чтобы стороны договаривались «на берегу» – т. е. необходим предварительный этап согласования параметров проекта и объема федеральной поддержки перед заключением концессионного соглашения. На этом этапе стороны обсуждают сотрудничество, заключают соглашение, и только тогда инвестор начинает тратить деньги на проектирование. Деньги же из федерального бюджета будут выделяться после окончания этапа проектирования и всех необходимых согласований.

«Составление перечня проектов – важный шаг, нужен четкий сигнал рынку, что планируются такие-то проекты, – отмечает Игнатущенко. – Во многих странах есть примеры инфраструктурных планов, которые в том числе являются маркетинговыми документами для инвесторов». Это больше, чем просто список с источниками финансирования, нужно не просто внебюджетное финансирование, а именно использование опыта и компетенций частных инвесторов в эффективном строительстве и управлении инфраструктурой, отметил эксперт. Также, по словам Игнатущенко, в мире широко распространена практика включения проектирования в обязательства концессионера. «В этом случае инвестор постарается сделать максимально качественное проектирование, понимая, что далее по этому проекту будет строить, также снижается риск необходимости переделывать проект на этапе строительства», – резюмировал эксперт.
Как прошел второй этап Бизнес-регаты «Ведомости»

