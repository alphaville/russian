\chapter{Смертная казнь}

\section{Смертная казнь в России}

 {\it \url{https://ru.wikipedia.org}}
% https://ru.wikipedia.org/wiki/Смертная_казнь_в_России

Смертная казнь в Российской Федерации по действующей конституции 1993 года «носила временный характер и была рассчитана лишь на некоторый переходный период» и больше не может применяться с 16 апреля 1997 года, то есть наказание в виде смертной казни не должно ни назначаться, ни исполняться. Вопрос о её применении окончательно был разъяснён Конституционным судом в 2009 году на основании конституции и международных договоров, но норма о смертной казни осталась в национальном законодательстве, обладающем меньшей правовой силой, чем конституция и международные договоры.

\textbf{Текущее положение.} Согласно Конституции Российской Федерации, казнь установлена уголовным кодексом в качестве исключительной меры наказания за особо тяжкие преступления против жизни при предоставлении обвиняемому права на рассмотрение его дела судом с участием присяжных заседателей. При этом в Конституции оговаривается, что смертная казнь может устанавливаться «впредь до её отмены», де-факто уже произошедшей: в 2009 году сообщалось, что смертная казнь запрещена навсегда[3][4][5], хотя ещё до этого Уполномоченный по ПЧ заявлял, что «смертная казнь в России уже была отменена, в том числе и юридически», и «у нас есть полная отмена смертной казни»[6].

В 1996 году Россию пригласили в Совет Европы только при условии отмены смертной казни[7][8][9]. Президент просто стал игнорировать рассмотрение дел приговорённых к смертной казни (не утверждать и не миловать), что, согласно ст. 184 УИК РФ, заблокировало возможность исполнения всех приговоров.

16 апреля 1997 года Россия подписала Протокол № 6 к Конвенции о защите прав человека и основных свобод относительно отмены смертной казни (в мирное время). Несмотря на то, что 6-й протокол так и не был ратифицирован Россией, с этого момента смертную казнь в России запрещено применять согласно Венской конвенции, которая предписывает государству, подписавшему договор, вести себя в соответствии с договором до его ратификации.

В 1999 году Конституционный суд признал неконституционной возможность вынесения смертных приговоров в отсутствие судов присяжных во всех регионах страны (они отсутствовали в Чечне).

В 2009 году КС признал невозможность назначения смертной казни даже после введения суда присяжных в Чечне, мотивировав это тем, что «в результате длительного моратория на применение смертной казни сформировались устойчивые гарантии права человека не быть подвергнутым смертной казни и сложился конституционно-правовой режим, в рамках которого — с учётом международно-правовой тенденции и обязательств, взятых на себя Российской Федерацией, — происходит необратимый процесс, направленный на отмену смертной казни как исключительной меры наказания, носящей временный характер („впредь до её отмены“) и допускаемой лишь в течение определённого переходного периода, то есть на реализацию цели, закрепленной статьёй 20 (часть 2) Конституции Российской Федерации». Статья 55 Конституции России запрещает отменять или умалять (ущемлять) права человека, которые уже однажды были даны конституцией или международно-правовыми нормами, ставшими частью российской правовой системы[10].

Последний раз казнь была применена в 1996 году[11]. По мнению Тамары Морщаковой, вернуть в России смертную казнь нельзя никакими способами кроме как принятием новой конституции (так как вторая глава Конституции не подлежит изменению) в следующем порядке: принятием Федерального конституционного закона о Конституционном Собрании, внесением инициативы об изменении, одобрением ГД и СФ, созывом Конституционного Собрания, разработкой проекта новой конституции и всенародным референдумом о принятии новой конституции России[12].

\textbf{Смертная казнь в Древней Руси.} Существуют различные мнения по поводу истоков применения смертной казни в Древней Руси: она возникла либо как продолжение обычая кровной мести, либо вследствие византийского влияния[13]. Летописи известны попытки византийских епископов приобщить Русь к канонам Кормчей книги, где говорится о необходимости казни лиц, занимающихся разбоем. «Ты поставлен от Бога на казнь злых людей», — доказывали епископы Владимиру. «На какой-то период карательной практике того времени были известны случаи применения смертной казни за разбой, но смертная казнь не была воспринята русской действительностью, и Владимир отменил её, перейдя к давно известной русскому законодательству системе денежных пеней»[13]. По свидетельству арабского путешественника Ибн Вакшия смертная казнь к разбойникам применялась ещё в 930 году; в 996 году была введена смертная казнь за убийство в разбое[14].

Русская Правда не предусматривает смертной казни, но Краткая редакция Русской Правды законодательно закрепляет право кровной мести, ограничивая круг субъектов этого права: «Убьетъ муж(ъ) мужа, то мстить брату брата, или сынови отца, любо отцю сына, или братучаду, любо сестрину сынови; аще не будетъ кто мьстя, то 40 гривен за голову». Последняя фраза позволяет сделать вывод о том, что в данные отношения уже вмешивается государство — при отсутствии родственников, которые могли отмстить за убитого, с убийцы взыскивался денежный штраф.

При этом, в соответствии со ст. 17, 21, 38 Краткой редакции Русской Правды, допускает убийство без наказания вора, обнаруженного на месте преступления (с ограничениями), и холопа, ударившего свободного мужа.

Окончательная отмена кровной мести была совершена сыновьями Ярослава Владимировича на межкняжеском съезде. Так, по ст. 2 Пространной редакции Русской Правды, «По Ярославе же паки совкупившиеся сынове его: Изяслав, Святослав, Всеволод и мужи их: Коснячько, Перенег, Никифор и отложиша убиение за голову, но кунами ся выкупати». Данная статья свидетельствует о том, что кровная месть юридически была отменена и заменена денежным штрафом.

Отсутствие смертной казни в системе наказаний Русской Правды не означает отсутствия её реального применения. В летописях имеются свидетельства применения смертной казни за мятеж, измену, преступления против христианской веры[15].

\textbf{XVIII век.} Количество составов преступления, за которое назначалась казнь, увеличивалось и достигло пика в царствование Петра Великого, после которого волна пошла на спад, и начались разного рода попытки законодательной отмены или ограничения смертной казни. Воинский Артикул Петра I предполагал применение смертной казни в 123 случаях, однако реально смертная казнь применялась только за мятеж, убийство, измену, а также за казнокрадство и коррупцию; в остальных случаях применялись телесные наказания, ссылка на каторгу (как на определённый срок, так и навечно) и клеймление[15].

В царствование Елизаветы Петровны, как реакция на бессмысленную жестокость наказаний при Анне Иоанновне, были отменены смертная казнь и пытки для лиц младше 17 лет[14]. Указы от 18 (29) июня 1753 года[16], 30 сентября (11 октября) 1754 года[17] заменили «натуральную смертную казнь» на «политическую», которая выражалась в ссылке «на каторжные работы, предварительно подвергнув: наказанию кнутом с вырыванием ноздрей и постановлением клейма» или без такового[18]. Кроме того, все дела, по которым подлежала применению смертная казнь, подлежали передаче в Сенат и рассматривались самой Елизаветой. В теории уголовного права это рассматривается как прообраз введённого в 1990-х годах моратория на смертную казнь с широким применением института помилования к приговорённым к данному наказанию лицам[15]. Отмечается, что замена смертной казни наказанием с применением кнута носила во многом формальный характер, так как по приговорам судов преступникам назначалось большое количество ударов кнутом, что часто приводило к их смерти[19].

Сохранялось такое положение и в царствование Екатерины II, однако отказ от применения смертной казни к общеуголовным преступлениям не исключал её применения в отношении деяний, совершённых против государства[15]. Например, в 1775 году, согласно нормам Уложения 1649 года и Уставов Петра I смертная казнь была применена к руководителям и участникам восстания Пугачёва.

Смертные приговоры в начале XIX века выносились редко: в период царствования Александра I было казнено 84 человека[19].

\textbf{Виды смертной казни.} Нередко практиковался такой вид казни, как посажение на кол, относительно часто — при Иване Грозном. При Петре I, в частности, на кол был посажен любовник опальной царицы Е. Лопухиной бывший майор Степан Глебов.[20] Колесование, применявшееся в России и ранее, при Петре I было закреплено в Воинском Уставе, и применялось регулярно вплоть до XIX века.

В до- и даже послепетровской России нередки были случаи сожжения. По Уложению 1649 года оно полагалось за богохульство. В 1682 году был сожжен в Пустозерске (исчезнувший город вблизи нынешнего Нарьян-Мара) протопоп Аввакум с его тремя сподвижниками. В 1689 г. в Москве, в Немецкой слободе — мистик, автор непонятных стихов в духе Нострадамуса Квирин Кульман, вместе со всеми своими книгами. В 1738 году, в царствование Анны Иоанновны, были сожжены на костре за переход в другую веру: флота капитан-лейтенант Возницын, «вместе с совратителем своим жидом Борохом Лейбовым» — за переход в иудаизм; а татарин Тойгильда Жуляков — за возврат в ислам. К этой последней казни, состоявшейся в Екатеринбурге, приложил руку его основатель В. Н. Татищев. Последний в России приговор к смертной казни через сожжение был вынесен за колдовство Андрею Козицыну в Яренске в декабре 1762 года, однако не был утвержден ввиду действующего моратория на смертную казнь.

Согласно Соборному уложению 1649 года, фальшивомонетчиков казнили, заливая в горло расплавленный металл.

Для женщин, убивших мужей, была принята весьма экзотическая казнь: закапывание в землю заживо по шею. Повешение за ребро ещё в Пугачевщину было делом весьма обыкновенным, так что перед селами, где жили бунтовщики, для острастки выставляли глаголь для повешения за ребро, причем жителям не надо было объяснять, что это такое. Четвертование было в России также казнью нередкой, но для него никогда не употреблялись лошади. В 1775 г. были четвертованы Пугачёв и Перфильев, причём им отсекли сначала голову. Это было последнее четвертование в России. В 1826 году декабристы: Пущин, Кюхельбекер и другие — всего 31 человек, осуждённые по первому разряду — были приговорены к отсечению головы (казнь им заменили каторжными работами), а пятеро, объявленные вне разрядов — к четвертованию (которых в итоге повесили). После этого случаи отсечения головы и четвертования или хотя бы вынесение таких приговоров неизвестны.

[...]

\textbf{Революции 1917 года и Гражданская война.} Смертная казнь была отменена после Февральской революции в 1917 году, но вскоре — снова введена на фронте Временным правительством за воинские преступления, измену, убийство и разбой[15]. Ключевую роль в возобновлении практики смертных приговоров сыграл Л. Г. Корнилов[23].

После установления советской власти смертная казнь была отменена II Всероссийским съездом Советов 28 октября 1917 года[24]. Однако в связи с принятием постановления СНК РСФСР «О красном терроре» от 5 сентября 1918 года смертная казнь была восстановлена: она применялась к лицам, которые имели прикосновенность к белогвардейским организациям, заговорам, мятежам. Исполнялась она путём расстрела, а имена расстрелянных и основания применения к ним смертной казни подлежали опубликованию[22]. Позже данные нормы были включены в Руководящие начала по уголовному праву РСФСР 1919 года[15].

Впервые смертная казнь в данный период была применена 21 июня 1918 года революционным трибуналом ВЦИК к бывшему начальнику морских сил Балтийского флота контр-адмиралу Алексею Щастному[19].

Смертная казнь в этот период также часто применялась в порядке внесудебной расправы[25]. Как в этот период, так и в последующие годы, зачастую дела, приводившие к смертным приговорам, являлись сфабрикованными: известными примерами подобного рода являются дела В. Таганцева (был казнен 61 человек, в том числе 16 женщин), Н. Гумилева и другие[26].

Затем Постановлением ВЦИК и СНК РСФСР от 17 января 1920 г. «Об отмене применения высшей меры наказания (расстрела)» смертная казнь вновь отменяется. Однако уже в приказе Реввоенсовета Республики от 4 мая 1920 г. «О революционных военных трибуналах» революционные военные трибуналы наделяются правом применения смертной казни в виде расстрела[27].

Согласно Декрету ЦИК от 22 мая 1920 г. «О порядке приведения в исполнение губернскими революционными трибуналами приговоров к высшей мере наказания в местностях, объявленных на военном положении, а также в местностях, на кои распространяется власть революционных военных советов фронтов» решением губернского исполкома осуждённый лишался права на обжалование, и помилование, а смертная казнь приводилась в исполнение немедленно[27].

