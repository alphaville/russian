\chapter{Культура, обычаи и традиции}

\section{Праздники}
\subsection{Рождеств\'{о} Христ\'{о}во}

\textit{Источник: \url{https://infourok.ru/}}

Рождеств\'{о} Христ\'{о}во --- праздник \explain{правосл\'{а}вного}{правосл\'{а}вный: orthodox} календаря, \explain{устан\'{о}вленный}{established} 7 января (25 декабря ст стиля).

По народному календ\'{а}рю этот день являлся днём з\'{и}мнего \explain{солнцевор\'{о}та}{солнцевор\'{о}т: solistice}, когда начиналось \explain{проб\'{у}ждение}{awakening (пробуждать/пробудить: to waken)} солнца после его дл\'{и}тельного з\'{и}мнего сна. Рождество Христово почиталось по всей России и по своей \explain{значимости}{значимость: significance} в православном календар\'{е} стояло на втором месте после П\'{а}схи. В русской деревне оно \explainDetail{отмеч\'{а}лось}{отмечаться/отметиться}{to celebrate} обычно в течение трёх дней и начиналось с посещения хр\'{а}ма, которое считалось у крестьян делом желательным, но не строго обязательным.

Рождество также отмечалось двумя тр\'{а}пезами: в рожд\'{е}ственский \explainDetail{соч\'{е}льник}{(рождественский) соч\'{е}льник}{Christmas eve; кан\'{у}н: eve} (канун праздника) и \explain{непосредственно}{directly} в Рождество.

Трапеза \explain{накануне}{on the eve of} праздника начиналась с появлением на небе первой вечерней звезды. На стол \explainDetail{подавали}{подавать/подать}{to serve} блины или оладьи с медом, \explain{постные}{пост: fasting} пироги с грибами, картофелем, кашей, пресные пирожки с ягодами, а также кутью из крупных зерен пшеницы с ягодами.

Тр\'{а}пеза, проходившая в день Рождества, предполагала богатый и \explain{разнообразный}{diverse} обед, во время которого подавалось множество мясных и молочных блюд, пирогов.

Рождество б\'{ы}ло первым днём \explain{выполнения}{performance; execution; effectuation} различных \explain{обр\'{я}дов}{обр\'{я}д: ritual}, которые должны б\'{ы}ли \explainDetail{обеспечить}{обеспечивать/обеспечить}{to provide} \explain{благополучие}{well-being} в наступающем солнечном году, \explain{предохранить}{(предохранять): to protect} от \explain{бед}{беда: misfortune; trouble} и несчастий дом, семью, \explain{скот}{cattle}, узнать будущее.

В рожд\'{е}ственский сочельник начинали колядов\'{а}ть. Группы детей, подр\'{о}стков, молодых мужчин и женщин обходили крестьянские дом\'{а} и \explain{исполн\'{я}ли}{исполнять/исполнить: to perform; to carry out} песни (кол\'{я}дки) с величаниями и пожеланиями хоз\'{я}йственного благопол\'{у}чия и семейного \explain{дов\'{о}льства}{довольство: contentment}. Начинались \explain{гад\'{а}ния}{гад\'{а}ние: fortune telling, divination} о судьбе.

С днём Рождества б\'{ы}ли св\'{я}заны разл\'{и}чные \explainDetail{прим\'{е}ты}{примета}{omen}.
Русские крестьяне верили в то, что травы и зернов\'{ы}е культуры будут хорош\'{и}, если на Рождество лежат \explain{глуб\'{о}кие}{глуб\'{о}кий, глуб\'{о}кая, глуб\'{о}кое, глуб\'{о}кие: deep} снег\'{а}; если в Рождество на небе много звёзд --- можно ждать бог\'{а}того \explain{урож\'{а}я}{(noun) harvest} гор\'{о}ха, а если в этот день сильная метель, то пчелы будут хорошо роиться. (По И. Шангиной)


\subsection{Масленица}
М\'{а}сленица --- большой народный праздник \explainDetail{проводов}{проводить}{to see off; пр\'{о}воды: used only in plural} зимы. В традиционном русском \explainDetail{быт\'{у}}{быт}{life; everyday life (в быт\'{у}: locative; in life)} эта неделя ст\'{а}ла самым \explain{\'{я}рким}{яркий: bright; brilliant; outstanding}, наполненным радостью жизни праздником.

Масленица отмечалась по всей России и в деревнях, и в городах. Её празднование считалось для всех русских людей обязательным: «Хоть себя заложи, а масленицу проводи». Неучастие в масленичном веселье могло повлечь за собой, \explain{по поверью}{according to the belief}, «жизнь в г\'{о}рькой беде».
Первые три дня масленой недели шла \explain{подготовка}{preparation} к празднику: привозили дров\'{а} для масленичных костров, убирали \explain{избы}{изба: hut; little house}. Основные \explain{пр\'{а}зднества}{пр\'{а}зднество: festivals} \explain{приходились}{happen} на четверг, пятницу, субботу, воскресенье --- дни широкой масленицы.

Все масленичные развлечения проходили обычно на улице. \explain{Нар\'{я}дно одетые люди}{elegantly dressed people} участвовали в праздничном \explain{гулянье}{celebration}, поздравляли друг друга, шли на ярмарку, удивлялись чудесам, которые показывали в балаганах --- передвижных театрах, радовались кукольным представлениям и «медвежьим потехам» --- выступлениям вожака с медведем. Масленичный комплекс включал в себя такие развлечения, как катание с гор, катание на санях, кулачные бои, шествия ряженых и др. В масленицу звучало много песен, прибауток, приговоров.

Прощ\'{а}лись с масленицей всегда в воскресенье.
В этот день жгли костры, которые символизировали солнце и должны были \explain{способствовать}{ (поспособствовать) to contribute} скорейшему пробуждению природы, и жгли \explain{ч\'{у}чело}{scarecrow} Масленицы. (По И. Шангиной)



\subsection{П\'{а}сха (Воскресение Христово)}
Название П\'{а}сха происходит от древнееврейского слова «песах» (прохождение). В русский язык слово «пасха» пришло из греческого, вместе с принятием православия, но у многих славянских народов праздник Воскресения Христова называется \explain{иначе}{differently; otherwise}: у украинцев --- великдень, у белорусов --- вяликдзень, у болгар --- великден, у поляков ---Wielkanoc, у чехов --- Velikanoc и т.д.

Нет праздника более торжественного, более радостного, чем Светлое
Христово Воскресение. Конец марта --- апрель --- это уже время прихода
настоящей весны, её \explainDetail{победы}{поб\'{е}да}{victory}
над зимними холодами, праздник \explainDetail{возрождения}{возрожд\'{е}ние}{rebirth},
воскрешения природы. Вместе с тем это и начало нового
\explainDetail{сельскохозяйственного}{сельскохоз\'{я}йственный}{agricultural (of the agric. economy)}
года, начало \explainDetail{полев\'{ы}х}{полев\'{о}й}{related to a field} работ
--- \explain{вспашки}{plowing} и \explain{сева}{(сев) sowing}.

В Древней Руси именно с весеннего \explainDetail{равноденствия}{равноденствие}{equinox}
начинался новый год, а чтобы посевы \explain{благополучно}{safely} взошли и вызрели,
скот давал приплод, н\'{у}жно было исполнить различные магические обряды.
Один из таких \explain{сохранившихся}{survived} до настоящего времени обрядов
--- пасхальная тр\'{а}пеза, когда единственный раз в году готовятся блюда,
которые в другие дни к столу не \explainDetail{подаются}{подавать/подать}{to serve}:
куличи и крашеные яйца.

Крашеными яйцами \explainDetail{обменивались}{обмениваться/обменяться}{to exchange},
их дарили родным, соседям, пришедшим поздравить, их брали с собой, отправляясь
в гости, раздавали нищим. (По Л.С. Лаврентьевой и Ю.И. Смирнову)

\newpage
\section{За здоровье или на здоровье?}

\textit{Источник: \url{http://really-easy-russian.blogspot.com/2012/12/blog-post_30.html}}

Есть хорошая русская традиция: приглашать к себе домой в гости, готовить много разной ед\'{ы} и сидеть долго за столом. За столом гости наливают себе в бокалы – кто наливает вино, кто водку, кто сок. Все вместе \ed{ч\'{о}каются}{чокаться/чокнуться}{to clink glasses (ч\'{о}каюсь, ч\'{о}каешься, ч\'{о}каются / ч\'{о}кнусь, ч\'{о}кнешься, ч\'{о}кнутся)} и потом пьют.
Перед тем как выпить, кто-то обычно говорит тост – пожелание. Например, такой тост «Давайте выпьем за др\'{у}жбу!». Раньше в\'{е}рили, что напиток, которые все пьют вместе, не только объединяет всех, но обладает магической, волшебной силой, поэтому слова-пожелания обязательно \explain{сбудутся}{come true}.
Вот почему России пьют обычно за что-то или кого-то: за здоровье, за любовь, за успехи, за встречу, за хорошее настроение, за радость, за хозяйку дома, за её мужа, за детей и т.д.

В древности в\'{е}рили не только в волшебную силу напитка, которые пили вместе, но и в магическую силу еды. Когда гость \explain{хвал\'{и}л}{praised} еду, которую приготовила хозяйка, она благодарила его и отвечала – и сегодня отвечает: «На здоровье!»\footnote{like ``cheers''}. «На здоровье» выражает, во-п\'{е}рвых, благодарность, это вариант слова «спасибо».
Во-втор\'{ы}х, желание, чтобы еда принесла гостю только хорошее, особенно здоровье. Поэтому когда мы говорим в ресторане, в гостях «спасибо», часто в ответ слышим – «на здоровье!».

Итак, когда поднимаем бокалы и ч\'{о}каемся -- говорим только «за здоровье!», когда гости у вас дома хв\'{а}лят ваше вкусное блюдо – отвечаете «на здоровье!».

\newpage
\section{Как живет Лаос}

\textit{Нищая страна из «опиумного треугольника» с атмосферой, которой не найдешь нигде в мире}

\textit{Источник: \url{https://lenta.ru/articles/2022/10/14/laos/}}

Одна из самых раскрученных достопримечательностей Юго-Восточной Азии — так называемый Золотой треугольник, где сходятся границы трех стран: Таиланда, Лаоса и Бирмы. Раньше в этом регионе были плантации опиумного мака, поэтому неофициально он называется Опиумный треугольник. Еще в прошлом веке здесь происходили войны между мафиозными группировками, а «страшное зелье» распространялось контрабандой по всему миру. Такое положение дел сохранялось до начала 1990-х годов, пока армии Таиланда, Бирмы и Лаоса не разгромили синдикат. В 1996 году главный наркобарон Кхун Са предал своих подельников и бежал. Российский военный репортер Игорь Ротарь отправился в сердце Золотого треугольника и рассказал, чем его впечатлили местные жители. О его приключениях — в материале «Ленты.ру».

\textbf{Путешествие во времени}

В Таиланде былую славу Золотого треугольника превратили в мощный туристический бренд. Толпы туристов на комфортабельных автобусах возят на стык тайской, бирманской и лаосской границ. Здесь, с тайского берега, сидя в хороших ресторанах, туристы любуются видами Лаоса и Мьянмы или катаются на лодках по главной водной артерии Юго-Восточной Азии — реке Меконг.

В этом же месте расположены аж два музея опиума. В одном из них даже есть небольшая маковая плантация. Однако Таиланд российскими туристами изучен уже вдоль и поперек, а вот в лаосской и бирманской части Золотого треугольника гораздо больше самобытности и нехоженых троп.

Лаосскую визу на месяц можно купить за 40 долларов прямо на границе, но россиянам, в отличие от американцев, разрешается пребывать в стране без визы до 30 дней. Оказаться в Лаосе после Таиланда — все равно что переместиться на машине времени лет на сто назад. Почти тот же язык, та же культура и пища, но все в разы беднее и хуже.

Дороги совершенно разбитые. Асфальт ужасного качества проложен только между крупными городами, а между деревнями обычные проселки, превращающиеся в сезон дождей в сплошное грязное месиво. Убогие мостики на реках раскачиваются, и ехать по ним на мотоцикле очень опасно. Живут местные в традиционных домах на сваях, поскольку периодически здесь случаются наводнения. Спят вповалку, готовят на кострах

\begin{center}
    {\Huge
        30 дней
    }

    {\Large
        россияне могут пребывать в Лаосе без визы
    }
\end{center}


По-английски не говорит практически никто. К тому же озвученную транскрипцию местных названий на латинице лаосцы не понимают — латинский алфавит они не знают. Скажем, я остановился в городке Luang Namatha, но когда произносил это словосочетание в разных вариантах, меня не понимали, прочитать же название городка на латинице были не в состоянии.

К слову, цифры у лаосцев тоже другие, поэтому понять, сколько стоит номер в отеле, — серьезная проблема. Кстати, неграмотность среди лаосцев и сегодня составляет 20 процeнтов, а на территориях, относящихся к Золотому треугольнику, наверняка даже больше.

Интересно, что в Лаосе, бывшей французской колонии, многие надписи на государственных учреждениях дублируются по-французски. Для кого это делается — остается загадкой, ведь французского языка местные тоже не знают. Впрочем, какое-то положительное влияние французов все же есть. Так, в отличие от Таиланда, здесь даже в небольших городках можно купить белый хлеб.

\begin{fancyquotes}
    Цены тоже очень странные. Половина жареного цыпленка на рынке стоит меньше доллара, приблизительно столько же — обед в местном кафе, но меню на английском в нем нет, а пища часто выглядит не слишком аппетитно. Если же в ресторане есть меню на английском, что большая редкость, то обед уже потянет на десять долларов
\end{fancyquotes}

Шикарный номер в отеле тоже будет стоить около десяти долларов, то есть дешевле, чем в Таиланде. А вот аренда скутера обойдется в девять долларов, то есть в три раза дороже, чем в соседней стране. Импортные продукты тоже гораздо дороже, чем по ту сторону границы.

\textbf{Национальный характер}

Французские колонизаторы придумали такой афоризм: «Вьетнамцы сажают рис, кхмеры за ним следят, а лаосцы просто смотрят, как он растет». Забавно, что французы так и не смогли справиться с ленью лаосцев и были вынуждены смириться с ней.

Действительно, разница с Таиландом очень ощутима. Лаосцы гораздо медлительнее и заторможеннее соседей. Они искренне хотят выполнить работу хорошо, но делают ее как-то вяло. Создается ощущение, что лаосцы, несмотря на бедность, вообще равнодушны к деньгам.

При этом они очень добродушны и приветливы с туристами. Когда у путешественника случаются какие-то проблемы, они всегда приходят на помощь и, как правило, отказываются брать деньги. Здесь, в отличие от многих бедных стран, не принято обманывать и «разводить» отдыхающих. Преступность в сельском Лаосе тоже практически отсутствует.

Некоторые этнологи объясняют инертность лаосцев тем, что большинство из них — приверженцы буддийского направления Тхеравада, согласно которому материальные блага и карьера несущественны. Однако к той же ветви буддизма принадлежат большинство тайцев, но они больше приспособлены к работе. Самое разумное объяснение — то, что тайцы почти насильно были приучены к капитализму на рубеже XIX-XX веков королем-реформатором Рамой Пятым, а вот лаосцы так и остались жить в раннефеодальном обществе.

\textbf{В краю племен}

Горная долина Луангнамтха — одна из главных достопримечательностей лаоской части Золотого треугольника. Это настоящий этнографический музей под открытом небом. Здесь среди горных джунглей и рисовых полей живет множество племен — как местных, так и переселившихся из Южного Китая и Вьетнама. Часть племен исповедуют буддизм, часть — язычество, или, если быть точным, анимизм.

\begin{fancyquotes}
    Местные язычники сжигают покойников в священном лесу, а над урнами с их прахом водружают шесты и флаги. В деревнях здесь можно попить рисовой водки, а также изучить ручное ткачество
\end{fancyquotes}

И в тайской, и в бирманской, и в лаосской части Золотого треугольника очень много монастырей и монахов. Но в Лаосе есть интересная особенность: среди послушников здесь в основном дети, причем, судя по их поведению, не очень религиозные. Так, например, я видел, как двенадцатилетние послушники курили.

Дело в том, что при монастырях есть школы, где послушники изучают как светские, так и теологические предметы. Скорее всего, в нищем Лаосе отправить мальчика в монастырь — это просто способ прокормить его и дать ему элементарное образование.

\textbf{Homestay по-местному}

Поселился я в homestay в небольшой деревеньке одного из племени анимистов, чьи предки переселились в Лаос из Вьетнама в конце XIX века. Жилье представляло собой традиционную бамбуковую хижину, сразу за которой начинались рисовые поля. Ночлег и еда обошлись всего в 12 долларов. Хозяева были очень радушны, но не без характерных для Лаоса особенностей.

Так, например, зная, что я выпиваю две чашки кофе, мне их подавали сразу — видимо, чтобы не варить кофе дважды. А когда я захотел купить бутылочку вина, хозяева не поняли, как снять упаковку с горлышка бутылки. Впрочем, спустя какое-то время с этой задачей они справились, но под упаковкой оказалась пробка, а штопор им найти так и не удалось.

Но это все мелочи, ведь в Лаосе царит настоящий покой, которым можно наслаждаться, катаясь на велосипеде и глядя на великолепные пейзажи. По вечерам я иногда выбирался в деревню, играл с местными детьми в перетягивание каната и даже пел под мобильник в импровизированном караоке. Забавно, что после дождя местные дети скатывались с горки по грязи так же, как наши ребятишки — по снегу.

Иногда я выбирался на вечернюю службу в ближайший сельский монастырь. Но, увы, задерживаться там долго не получалось. Дело в том, что в буддийских храмах часто есть собаки, их не прогоняют даже во время богослужений. Так вот, псы начинали лаять на меня, хотя на прихожан-лаосцев не реагировали. Наверное, у меня плохая карма.

\textbf{Лас-Вегас по-лаосски}

Впрочем, несколько лет назад в лаосской части Золотого треугольника появилось развлечение совсем другого рода. В свободной экономической зоне, как раз на стыке с тайской и бирманской границей, китайские мафия соорудила поселок с казино, шикарными отелями, ресторанами и борделями.

\begin{fancyquotes}
    Как отметила побывавшая в местных игорных домах корреспондентка BBC, это квинтэссенция китча. Среди клиентов казино лаосцев почти нет, все туристы здесь из Китая и Таиланда
\end{fancyquotes}

Интересно, что нечто подобное уже существовало в бирманском городке Монг Ла, расположенном прямо на китайской границе. В реальности город и его окрестности контролируют сепаратисты, создавшие здесь непризнанное государство Шан.

Я побывал в этом городке около 20 лет назад. Попал я туда почти случайно. Бродя по бирманским горам, неожиданно наткнулся на деревеньку, где остановились на постой мьянманские военные. Бравые солдаты чувствовали себя здесь как дома. Они качались в гамаках на верандах хижин, лениво наблюдая за тем, как крестьянки режут кур, предназначенных на обед этим непрошенным гостям.

Офицеры пригласили меня разделить с ними трапезу. «Об армии в Мьянме говорят всякое, а на самом деле мы просто смотрим за порядком, помогаем людям», — на хорошем английском начал беседу старший из военных.
В диалоге офицер упомянул непризнанное государство Шан и сказал, что туда можно поехать иностранцу.

Добраться до мятежников, как выяснилось, можно даже на обычном такси. Пока мы ехали, подпрыгивая на многочисленных кочках и рытвинах, я вспоминал мьянманские города, где даже электричество подается с перебоями, и с тоской размышлял, какой дырой должно быть это непризнанное государство Шан. И вдруг через три часа пути я неожиданно попал в другой мир.

Проселочная дорога сменилась великолепным шоссе, а вместо привычных для Мьянмы хижин стали появляться аккуратные каменные дома с надписями на китайском. На одном из них даже красовалась реклама Coca-Cola, которая запрещена в Мьянме как чуждый местному образу жизни напиток.

После нищей и неухоженной Мьянмы Монг Ла производил впечатление города с другой планеты: украшенные неоновой рекламой небоскребы, великолепные магазины и изысканные рестораны с самой экзотической пищей (около кафе в качестве приманки для гурманов стояли огромные аквариумы с питонами и другими змеями).

\begin{fancyquotes}
    Электричество и интернет сюда поступают из Поднебесной, а китайские сим-карты работают без роуминга. Бирманские деньги здесь не принимают, расплачиваться можно только юанями. В Китае Монг Ла называли «азиатским Лас-Вегасом» или «Городом греха». И действительно, казино были открыты практически в каждом отеле, а все клиенты были приезжими из Китая
\end{fancyquotes}

Игорный бизнес стал развиваться здесь в начале 1990-х годов, когда местные наркобароны решили покончить с прежним ремеслом и найти ему альтернативу. Как бы подводя итоги своей деятельности, мафия даже открыла музей опиума, посвященный нелегкой деятельности торговцев смертью.

Впрочем, как я убедился, азартными играми список разрешенных здесь пагубных человеческих пристрастий не ограничивался. Однажды мне навстречу попались полупьяные молоденькие европейки, которые шли покачиваясь и отчетливо матерились по-русски. Они оказались родом из Благовещенска, древнейшую профессию начали осваивать еще в Китае, однако затем, во избежание конфликтов с полицией, перебрались в более терпимый Монг Ла, где проституция фактически разрешена.

Китайские власти были недовольны существованием такого криминального анклава, но их терпению пришел конец, когда дочка крупного партийного деятеля проиграла в Монг Ла несколько сотен тысяч долларов. На короткое время Пекин ввел в государство Шан войска, и в 2005-м все казино в Монг Ла были закрыты. Однако, как говорится, свято место пусто не бывает. Китайская мафия попросту перенесла свой бизнес подальше от китайской границы — так и возник лаосский Лас-Вегас.

Лаос — определенно для искушенных туристов. Здесь можно умиротворенно отдохнуть в горах с погружением в местный быт либо покутить в казино и шикарных ресторанах (или даже в борделях). Ну, а на чем именно сделать акцент, каждый выбирает для себя сам.

\newpage
\section{Хэллоуин: что за праздник и что про него рассказать детям}

\textit{Споры относительно пр\'{а}зднования Хэллоу\'{и}на не утихают каждый год. Как же относ\'{и}ться к дате?\footnote{What about the date?} Мы оставляем решение за вами и \ed{д\'{е}лимся}{дел\'{и}ться + \textit{чем}}{to share} историей праздника.}

\textit{Источник: \url{https://gorodskayaferma.ru/blog-halloween}}



\textbf{Откуда он появился?}

Кельтские племена праздновали Новый год в последний день октября и в\'{е}рили, что в ночь с 31 октября на 1 ноября граница между миром жив\'{ы}х и миром мёртвых стир\'{а}лась. А чтобы обмануть злых духов и не \explain{попасться}{to get caught} ими, люди \explain{прикидывались}{pretended} им. Отс\'{ю}да традиция \explain{наряжаться}{to dress up} в \ed{устрашающие}{устрашать}{to frighten} костюмы.

\textbf{Причём здесь тыква?}

Тыква ст\'{а}ла символом праздника по логичной причине. Конец октября --- время сб\'{о}ра урож\'{а}я, а крупный овощ считался символом плодородия. Время это называлось Самайн. Считалось, что огонь внутри тыквы защищал от злых духов и \explain{отп\'{у}гивал}{scared away} \ed{прочую}{прочий}{other} \explain{н\'{е}чисть}{evil spirit}.

\textbf{Хэллоуин праздновали всегда?}

С принятием христианства люди отказались от идеи праздновать Самайн. Но в последствии, древний праздник вновь возродился — 1 ноября католики стали отмечать День всех Святых («All Hallows Even») и люди вернулись к кельтским традициям — стали наряжаться в устрашающие костюмы, зажигать огонь в тыкве и прятаться от духов.

\textbf{Как праздник пришёл к нам?}

Про Хэллоуин мы узнали благодаря массовой культуре и глобализации, а вот общих культурных корней у нас с этой датой действительно нет.

\textbf{А есть ли аналоги в нашей культуре?}

Несколько славянских праздников очень напоминают традиции Хэллоуина. Например, на Святки принято калядовать — стучаться в дверь к незнакомцам и просить угощение. К тому же, в этот пер\'{и}од (между Рождеством и Крещением) считается, что наш мир остаётся без защиты от нечистых сил и именно в эти дни люди могут гадать и переодев\'{а}ться в духов.

Ещё один похожий праздник — Велесова ночь. Кстати, отмечалась она в те же даты, что и Хэллоуин — в ночь с 31 октября на 1 ноября. Так же считалось, что в это время граница между мирами исчезает. Только наши \explain{пр\'{е}дки}{ancestors} ожидали вполне конкретных духов -- ум\'{е}рших р\'{о}дственников, которые могли вернуться на землю, дать совет и поговорить с живыми. Наряжаться в нечисть было не принято, а вот зажигать свечи — да. Огонь был зн\'{а}ком для душ, которые \'{и}щут путь.

\textbf{Какие ещё похожие праздники есть в мире?}

1 и 2 ноября в Мексике тоже \ed{почитают}{почитать}{revere} умерших, праздник называется Día de Muertos. В эти дни умерших родственников пышно встречают: чтобы они могли найти путь, города наполняются миниатюрными \ed{страшилками}{страш\'{и}лка}{short children's horror story}: марципановыми и шоколадными \ed{гробами}{гроб}{coffin} и черепами. На улицы выходят люди в костюмах скел\'{е}тов, а ночью народ уходит на кладбища --- прибираться на могилах и праздновать.

В начале ноября в Перу в городе Пуно празднуют La Diablada. Жители в \ed{красочных}{красочный}{colourful} костюмах посвящают танец духам озера Титикака. Танец рассказывает о добре и зле.

С 30 апреля на 1 мая в Германии отмечают Вальпургиеву ночь. Дома украшают фигурками ведьм. А вот в Скандинавии наоборот защищаются от ведьм — жгут костры и шумят, чтобы отпугнуть их.

В мае в Великобритании город Эдинбург очень гордится своими \ed{привидениями}{привид\'{е}ние}{ghost}, которыми по рассказам жителей, населён каждый дом. Местные даже устраивают в честь них фестиваль, где можно прогуляться по тр\'{о}пам приведений, послушать рассказы м\'{е}диумов или устроить спиритический сеанс.

\textbf{Что будет на Городск\'{о}й ф\'{е}рме на ВДНХ на Хэллоуин?}

30 и 31 октября на Городской ферме на \explain{ВДНХ}{выставка достижений народного хозяйства} будет страшно весело! Мы придумали развлекательную и стилизованную программу с ужасным \ed{квестом}{квест}{quest}, кошмарными мастер-классами, \ed{жуткими}{жуткий}{spooky} угощениями, кинопоказом, сказками и конкурсом на лучший карнавальный костюм. Участвуйте в конкурсе и выигрывайте очень полезный приз! Ждём вас в гости!