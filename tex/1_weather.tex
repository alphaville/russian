\chapter{Погода, Климат, изменение климата}

\section{Времена года}
\textit{Источник: \url{https://www.memorysecrets.ru/english-texts/vremena-goda-seasons.html}}

Каждое время года имеет свои плюсы и минусы. Давайте детальнее их разберем.

Весна (март, апрель, май) --- самое восхитительное время года. Несмотря на то, что весной \explainDetail{наблюдаются}{(по)наблюдаться}{to be observed} \explainDetail{ливни}{ливень}{shower (rain)}, многие люди \explain{воспринимают}{perceive} весну, как самое лучшее время года. Почему же так \explain{происходит}{happens}? \explain{Дело в том}{the thing is}, весна значит что-то новое для нас. Это шанс начать все заново, изменить жизнь к лучшему. Именно весной рожд\'{а}ется новая жизнь --- \explain{распускаются}{they bloom} почки и начинают цвести цветы.

Лето же (июнь, июль, август) --- это время отдыхать и просто получать удовольствие. Солнце светит с утра до ночи. Взрослые уходят в отпуск, дети --- на каникулы. Большинство людей проводят свои отпуска на море. Там можно не только \explainDetail{загореть}{загорать/загореть}{to tan}, но и встретить много незнакомых людей.

Осень (сентябрь, октябрь, ноябрь) можно разделить на две части --- начало осени и ее конец. В начале осени школьники снова возвращаются к учебе и садятся \explain{за парты}{behind the desks}. Лето постеп\'{е}нно переходит в осень. Но настроение еще \explain{приподнятое}{elevated}. Вскоре желтые листья начнут осыпаться, и этот величественный пейзаж тоже не \explainDetail{позв\'{о}лит}{позвол\'{я}ть/позв\'{о}лить}{to allow} нам грустить. Втор\'{а}я часть осени не так очаров\'{а}тельна, как первая. Дожди, \explain{сл\'{я}коть}{mud} и грязь --- это то, что люди ненавидят больше всего.

Зима (декабрь, январь, февраль) \explain{на нос\'{у}}{фразеологизм: приближаться, близиться (обычно -- по времени)}, приближаются холода. Мы надеваем перчатки, шапки, шубы и вязаные шарфы. Иногда в эту пору школы закрывают на карантин с целью предотвратить распространение гриппа. Хорошая новость для учеников --- вторые зимние каникулы. Рождество тоже вот-вот наступит. На улице порошит снег, дети лепят снеговиков и играют в снежки. Скоро мы будем наряжать елку, вешать на нее различные шары и гирлянды. Пора встречать Новый год!

\section{Что такое изменение климата?}
\textit{Источник: \url{https://www.un.org/ru/climatechange/what-is-climate-change}}

Под изменением климата понимают долгосрочные температурные изменения и изменение погодных условий. Хотя эти изменения могут быть естественными, как, например, циклические колебания солнечной активности, с 1800-х годов антропогенная деятельность является основным движущим фактором изменения климата, главным образом за счет сжигания ископаемых видов топлива, таких как уголь, нефть и газ.

В результате сжигания ископаемых видов топлива образуются выбросы парниковых газов, которые подобно одеялу окутывают Землю, удерживая солнечное тело и повышая температуру.

Примерами парниковых газов, выбросы которых вызывают изменение климата, являются двуокись углерода и метан. Они образуются, например, при использовании бензина для езды на автомобилях или угля для отопления зданий. Расчистка земель и лесов также может привести к высвобождению углекислого газа. Главным источником выбросов метана являются мусорные свалки. К числу основных производителей выбросов относятся энергетика, промышленность, транспорт, здания, сельское хозяйство и землепользование.

\begin{fancyquotes}
    Концентрация парниковых газов находится на самом высоком уровне за последние 2 миллиона лет
\end{fancyquotes}

И выбросы продолжают расти! В результате этого сейчас Земля на 1,1°C теплее, чем в конце 1800-х годов. Прошедшее десятилетие (2011–2020 годы) было самым теплым в истории.

Хотя многие думают, что изменение климата означает в основном более высокие температуры, рост температуры — это только начало истории. Поскольку Земля — это система, где все взаимосвязано, изменения в одной сфере могут повлиять на изменения во всех остальных.

В настоящее время к последствиям изменения климата относят, среди прочего, сильные засухи, нехватку воды, сильные пожары, повышение уровня моря, наводнения, таяние полярных льдов, катастрофические штормы и сокращение биоразнообразия.

\begin{fancyquotes}
    Люди сталкиваются с различными последствиями изменения климата
\end{fancyquotes}

Изменение климата может сказаться на нашем здоровье, способности выращивать продовольственные культуры, жилье, безопасности и работе. Некоторые из нас уже сейчас более уязвимы к воздействию изменения климата, например, люди, живущие в малых островных государствах и других развивающихся странах. Такие последствия, как повышение уровня моря и интрузия соленых вод, достигли такого уровня, что целые общины были вынуждены переселиться, а затяжные засухи подвергают людей риску голода. В будущем ожидается рост числа «климатических беженцев».

\begin{fancyquotes}
    Каждый дополнительный градус глобального потепления имеет значение
\end{fancyquotes}

В докладе Организации Объединенных Наций за 2018 год тысячи ученых и правительственных экспертов согласились с тем, что ограничение роста глобальной температуры на уровне не более 1,5°C поможет нам избежать самых худших климатических последствий и сохранить климат, пригодный для жизни. Однако, согласно текущим национальным климатическим планам, ожидается, что глобальное потепление достигнет 2,7°C к концу столетия.

Хотя выбросы, вызывающие изменение климата, образуются во всех регионах мира и сказываются на всех, некоторые страны производят их в гораздо больших объемах, чем другие. В то время как на 100 стран, которые производят меньше всего выбросов, приходится 3 процента от общего объема выбросов, доля десяти стран, являющихся самыми крупными производителями выбросов, составляет 68 процентов. Хотя принятие мер по борьбе с изменением климата — это дело всех и каждого, народы и страны, которые создают больше проблем, должны взять на себя большую ответственность и начать действовать первыми.

\begin{fancyquotes}
    Мы сталкиваемся с огромным вызовом, но нам уже известны многие решения
\end{fancyquotes}

Многие решения в области изменения климата могут быть не только экономически выгодными, но и также улучшить нашу жизнь и защитить окружающую среду. У нас также есть глобальные структуры и соглашения, на основе которых осуществляются усилия, направленные на достижение прогресса, такие как цели в области устойчивого развития, Рамочная конвенция Организации Объединенных Наций об изменении климата и Парижское соглашение. Имеются три широкие категории действий: сокращение выбросов, адаптация к последствиям изменения климата и финансирование необходимых мер по адаптации.

Перевод энергетических систем с ископаемого топлива на использование возобновляемых источников энергии, таких как солнце или ветер, позволит сократить выбросы, вызывающие изменение климата. Но начинать нужно прямо сейчас. Несмотря на расширение коалиции стран, обязавшихся достичь чистого нулевого уровня выбросов к 2050 году, около половины мер по сокращению выбросов должны быть осуществлены к 2030 году, с тем чтобы удержать потепление на уровне ниже 1,5°C. В период с 2020 по 2030 год производство ископаемого топлива должно сокращаться примерно на 6 процентов в год.

Цель адаптации к последствиям изменения климата состоит в том, чтобы защитить людей, их жилища, предприятия, источники средств к существованию, инфраструктуру и природные экосистемы. Эта деятельность касается не только нынешних последствий, но и тех, с которыми, вероятно, придется столкнуться в будущем. Хотя меры по адаптации придется принимать повсеместно, в настоящее время приоритетное внимание необходимо уделять наиболее уязвимым слоям населения, у которых меньше всего ресурсов для того, чтобы противостоять климатическим угрозам. Это может окупиться сполна. Например, системы раннего предупреждения стихийных бедствий спасают жизни и имущество и могут принести выгоды, в десятки раз превышающие первоначальные затраты.

\begin{fancyquotes}
    Мы можем оплатить счет сейчас или дорого заплатить в будущем
\end{fancyquotes}

Меры по борьбе с изменением климата требуют значительных финансовых вложений со стороны правительств и деловых кругов. Однако бездействие в отношении климата обходится гораздо дороже. Одним из важнейших шагов является выполнение промышленно развитыми странами своих обязательств по предоставлению 100 миллиардов долларов в год развивающимся странам, с тем чтобы они могли адаптироваться и перейти к более «зеленой» экономике.


