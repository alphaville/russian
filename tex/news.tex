% !TeX root = ../russian-vocab-2021-22.tex

\chapter{Новосты}

\section{Ее можно считать жертвой}
% https://lenta.ru/articles/2021/12/22/kinder/
\textit{Отец девятилетней студентки МГУ \explainDetail{напал}{нападать/напасть}{attack} на людей в вузе. Что о семье Тепляковых думают педагоги?}

У девочки-вундеркинда Алисы Тепляковой, которая в восемь лет \explain{сдала ЕГЭ}{сдавать/сдать экзамен} и поступила на платное отделение психфака МГУ, началась первая сессия. Во вторник, 21 декабря, появилось видео, где ее отец Евгений Тепляков пытается прорваться через пост охраны факультета с криками, что он «хочет поговорить с преподавателями». Администрация факультета вынуждена была прятать их от агрессивного родителя. Еще в сентябре, когда стало известно, что Алиса станет студенткой МГУ, «Лента.ру» попросила прокомментировать это событие известных педагогов. Все они сомневались, что для психики маленького ребенка учеба в вузе будет \explainDetail{посильной}{посильный/-ая}{Within one's strength, abilities, powers (for a task); посильная задача} задачей. Возможно, \explainDetail{опасения}{опасение}{fear} начинают \explainDetail{сбываться}{сбываться/сбыться}{to come true}. Мы публикуем их мнения об Алисе и методах ее отца.

\subsection{Будет не столько студенткой, сколько подопытным объектом}
\textit{Леонид Кацва, автор учебников и пособий по истории России. Преподаватель московской школы № 1543}

Я смотрел видеоинтервью с Алисой Тепляковой. У нее, видимо, очень тренированная \explainDetail{память}{память (ж)}{memory}. Каких-то других качеств она не показала в выступлении. Разговаривает она как семилетка, уровень ее понимания ситуации --- типичный для маленького ребенка. У нас в школе на \explain{педсовете}{teachers' council} перед началом учебного года говорили об этом, многие считают, что вся эта история очень дурно пахнет. \explainDetail{Имеется в виду}{имеется в виду}{it means} не то что девочка сдала ЕГЭ --- \explain{вызубрить}{memorise} какие-то вещи по нескольким предметам на минимальный балл она могла, если у нее действительно вот такая память. Я видел, как она читает --- быстро, но \explain{судя}{judging} по всему общего \explainDetail{смысла}{смысл}{meaning} текста не понимает. Папа \explain{дрессировал}{trained} детей именно на скорочтение. А скорочтение --- это немного не про чтение в том смысле, как мы его понимаем.

У меня нет вопросов к папе. Он хочет доказать некую идею --- что можно в школе не учиться 11 лет, а \explainDetail{освоить}{осваивать/освоить}{to master} все за три года. К девочке у меня тоже вопросов нет. Потому что в данном случае она --- \explain{орудие}{tool} в руках папы, \explain{в какой-то мере}{to an extent} ее даже можно считать жертвой.

У меня есть вопрос к МГУ: \explainDetail{принять}{принимать/принять}{accept} девятилетнего ребенка на психфак --- это надо все же сильно постараться.  Но гораздо больше у меня вопросов к школе,  которая ее  выпустила. Я ничего про эту школу не знаю. Даже не знаю номера. Видимо, она была на домашней форме   \explainDetail{обучения}{обучение}{learning; education}, на уроки не ходила. Не знаю, как это было оформлено --- экстернат или домашнее обучение, этого не могу сказать. Но если она в восемь лет сдала экзамены за все годы обучения, то, \explain{грубо говоря}{roughly speaking}, девочка должна была с шести лет сдавать экзамены каждые два-три месяца. Это если их принимали.

Папа говорит совершенно открыто, что девочка не прочитала ни одного программного литературного произведения, что она знакомилась с художественными книгами в виде кратких пересказов. Я понимаю, что так некоторые дети и делают, даже в 17-летнем возрасте. Но \explain{все-таки}{even so} это принято скрывать, а не превращать в манифест.

Я 40 лет преподаю историю и, как говорится, зуб даю, что если ребенка начать спрашивать не на уровне тестов, кто командовал теми-то войсками, кто был генеральным секретарем тогда-то, министром, великим князем тогда-то, а начать спрашивать \explain{всерьёз}{seriously}, с причинно-следственными связями, с характеристиками событий, то не о чем будет говорить. Специалисты по естественным наукам и физике также замечают, что даже на уровне физиологии не может ребенок в таком возрасте эти дисциплины качественно осваивать.

У нас были вундеркинды. И я знаю случаи, когда в вуз приходил учиться 14-летний студент. Однако разница между 14 и 17 годами, когда \explain{пол\'{о}жено}{one should, one ought to, one is supposed to} сдавать ЕГЭ, \explain{на порядок}{by an order of magnitude} меньше.
Я уж не говорю о разнице между 17 годами и девятью. Поэтому я в данном случае вижу какую-то \explain{недобросовестность}{dishonesty} с разных сторон. И прежде всего --- школы.
Возможно, она просто решила \explain{подыграть}{play along} папе, не знаю почему. Либо просто отвязаться от этого папы. Потому что папа такой, что проще согласиться на его \explainDetail{условия}{условие}{condition; term}, чем объяснять ему, почему этого делать не стоит.
Но, с другой стороны, есть и контраргумент, почему это может быть не так. За Алисой --- на подходе очередь из ее братьев и сестер. \explainDetail{Причём}{причём}{moreover} если у Алисы имя обычное --- среди девочек школьного возраста Алисы встречаются, то у остальных детей в семье имена скандинавских богов, а не детей из России. И тут у меня \explain{ощущение}{sensation}, что психологическое состояние папы от старшего ребенка к младшим начало \explainDetail{усугубляться}{усугубляться/усугубиться}{get worse}.

На мой взгляд, тут широкое поле деятельности для \explain{Рособрнадзора}{Федеральная служба по надзору в сфере образования и науки (Рособрнадзор): Federal Service for Supervision in Education and Science}. Не думаю, что тут речь о \explainDetail{мошенничестве}{мошенничество}{fraud; cheating} при ЕГЭ --- ребенок с натренированной памятью мог рассчитывать на минимальные баллы, чтобы экзамен считался сданным. Но \explain{полноценное}{of full value} среднее образование она получить не могла. Девочка под папиным \explainDetail{внушением}{внушение}{suggestion} говорит: в школе 11 лет учатся, а в институте --- пять, значит, институт --- проще, я его окончу за два года. Эти слова ребенка \explain{цитируют}{quote} \explain{СМИ}{средства м\'{а}ссовой информации}. Предположим, она окончит институт в 11 лет. Вы пойдете на консультацию к такому специалисту-психологу? Я, честно говоря, \explainDetail{остерегусь}{остерегаться/остеречься}{beware}.

\begin{fancyquotes}
    Это моя гипотеза --- и кроме догадок она ни на чем не основана, --- что на психфак ее приняли не столько для того, чтобы обучать как полноценного студента, сколько для того, чтобы ставить своего рода эксперимент. То есть в этой ситуации она будет не столько студенткой, сколько подопытным объектом, потому что с точки зрения психологии в ее развитии есть какие-то аномалии --- скорее всего положительные, а может быть, и не только
\end{fancyquotes}

Папа говорит, что учител\'{я}, которые заставляют свободно читающего ребенка \explain{по слог\'{а}м}{syllable by syllable} \explainDetail{произносить}{произносить/произнести}{pronounce; (present tense) произнош\'{у}, произн\'{о}сишь, произн\'{о}сят} ма-ма мы-ла ра-му, --- преступники. У него все --- преступники, один он --- молодец. Совершенно понятно, что папа преследует какие-то цели. Не могу сказать, что они материальные, \explain{по-в\'{и}димому}{apparently}, он хочет \explainDetail{прославиться}{прославляться/прославиться}{become famous}, стать великим реформатором образования или кем-то еще \explain{в этом роде}{like that}. Но мне кажется, что эти эксперименты очень опасные.

В моей практике были дети, которые перескакивали через класс --- из шестого в восьмой, из восьмого в десятый. Таких случаев у меня было, если не ошибаюсь, три. Эти ситуации на состояние детей оказали скорее \explain{отрицательное влияние}{negative influence}, чем положительное. Ребята были развитые, скучали в классах по возрасту, но когда их перевели на год вперед, они совершенно потерялись. Мне кажется,  что так делать не надо.

У меня есть дети, которые очень одарены математически и учатся в математическом классе. Они становились призерами Всероссийских олимпиад, но это не повод считать, что во всем остальном дети так же одарены. Знаете, как говорил великий русский поэт Козьма Прутков: «Специалист подобен флюсу, полнота его односторонняя». \explainDetail{Допускаю}{допуск\'{а}ть/допуст\'{и}ть}{admit}, что талантливые дети могут оканчивать школу, \explain{допустим}{let's say}, не за 11 лет, а за девять. Но в то, что ребенок может окончить школу в 9 лет, --- не верю.

\subsection{Слишком умных учеников частенько боятся}
\textit{Леонид Перлов, почетный работник общего образования России, много лет преподавал географию в одной из лучших математических школ страны Лицей «Вторая школа»}

В обычных школах слишком умных учеников частенько просто боятся. Потому что учитель --- живой человек. Он понимает, когда у него не получается, и не понимает --- почему. А не получается просто потому, что он раньше мог не иметь дела с такими детьми. Или школьная администрация от него требует одно, а ребенку нужно совершенно другое. И как найти в этом приемлемую середину --- очень сложный вопрос.

С такими ребятами действительно трудно, \explain{ничуть}{not at all} не легче, чем с детьми с аутическим компонентом, с другими \explainDetail{особенностями}{особенность}{feature, singularity, characteristic, particulatiry} развития.

Просто здесь трудности другого рода. Учитель должен очень много знать не только в области своей математики, географии или литературы, а именно в области педагогики. Эти дети больше требуют, они иначе \explain{воспринимают}{perceive} действительность, спос\'{о}бны быстро анализировать действия того же самого учителя и показать ему, прав он или нет в той или ин\'{о}й ситуации. Им очень много надо от учителя, а учитель далеко не всегда в состоянии им это дать. Нужн\'{а} другая ман\'{е}ра общения с ребенком. И грань между \explainDetail{жесткостью}{жесткость}{harshness} и фамильярностью учителю помогает установить только опыт.

Педагогика --- не наука. Это синтез искусства и \explainDetail{ремесл\'{а}}{ремесл\'{о}}{craft, trade. Plural: ремёсла, (gen) ремёсел}. И в контакте с каждым конкретным учеником педагог работает так, как этому конкретному ученик\'{у} требуется. Естественно, если школа \explainDetail{предоставляет}{предоставлять/предоставить}{provide} педагогу такую \explain{возможность}{opportunity}, если он не \explain{вынужден}{compelled, forced} как большинство учителей трудиться на полторы-две ставки. В моей «Второй школе» у учителей такая возможность есть.

Сейчас \explain{упор}{emphasis} делают на математической \explainDetail{одаренности}{одаренность}{talent, giftedness}, спортивной, музыкальной.
Да и \explain{собственно}{in fact} --- все. Других одаренностей стандарт не \explainDetail{предусматривает}{предусматривать/предусмотреть}{foresee}. А на самом деле этих одаренностей --- миллион. Ребенок вполне может быть талантлив в чем-то, чего пока еще не \explainDetail{проявил}{проявлять/проявить}{to show, to display, to evince, to manifest, to reveal; e.g., проявлять заб\'{о}ту (show concern); интерес (interest); себя (to prove oneself)}. И сам может о своей способности не догадываться.

Одна из задач квалифицированного педагога --- выявить эту одаренность. А вот что у ребенка здорово? Ну вот он \explain{дуб д\'{у}бом}{очень глупый} в математике и совершенно не интересуется химией. Но зато он пальцами чувствует, как из куска пластилина вылепить медведя. Его никто никогда этому не учил, но у него  прекрасно получается. Или, например, он педагогически одарён и обожает возиться с младшими своими товарищами. И у него отлично получается: они его слушают, они его обожают, они на нем виснут. Это одаренность? Думаю --- да.

Но школа сегодня не имеет задачи выявить талант у каждого. Главная задача школы --- выполнение стандарта. Все, наверное, слышали о федеральном государственном стандарте. \explainDetail{Подразумевается}{подразумеваться}{to be implied/meant}, что он --- \explain{некий}{a certain} \explain{эталон}{standard}, на который нужно равняться.
Для работы с детьми высоко мотивированными, грамотными, желающими учиться необходимо \explain{отклониться}{deviate} от этой нормы. Норма не рассчитана на повышенный уровень образования, в первую очередь она не может \explainDetail{удовлетворить}{удовлетворять/удовлетворить}{to satisfy, fulfill, gratify, suffice} требований со стороны ученика. Стандарты «отклонения» не приветствуют. Кроме того, отклонение в любую сторону --- хоть в сторону повышенных \explainDetail{потребностей}{потребность}{need (n.)} со стороны ученика, хоть в сторону работы с детьми с особенностями развития --- все это требует особой, \explain{соответствующей}{corresponding} квалификации учителей. Действующий профессиональный стандарт учителя подразумевает, что педагог обязан работать с любыми детьми в любых условиях. Хотя его никто и никогда не учил этому.

\begin{fancyquotes}
    Для родителей часто ребенок, \explain{скачущий}{galloping, prancing} \explain{со ступеньки на ступеньку}{step by step} в школе, побеждающий в олимпиадах, --- предмет гордости, повод свысока поглядывать на коллегу по работе или на соседей. Но ребенку эти успехи не всегда приносят радость. Рано или поздно родители начинают ему говорить: «Вот ты \explainDetail{з\'{а}нял}{занимать/занять}{occupy, take up, secure; note on stress: з\'{а}нял, занял\'{а}, з\'{а}няло, з\'{а}няли} второе место, а почему не первое? А ну-ка, поработай еще!»
\end{fancyquotes}

Дети, которые перепрыгивали через классы, были и 20 лет назад, и сто лет назад. Но ничего хорошего, как правило, из этого не выходит. Всему свое время, в том числе и детству. Думаю, что и на этот раз исключением эта девочка не станет. Конечно, для таких детей нужен особый подход. Ей нужны знания, соответствующие ее развитию и способностям. Но это вовсе не курсы ЕГЭ по русскому и математике. Подготовительные курсы к ЕГЭ --- это называется \explain{дрессировка}{training}. Медведь вон ездит в цирке на велосипеде. Но, во-первых, он не знает, что это неприятно. А во-вторых, совершенно не понимает, что для него --- медведя --- это нехорошо.

Все же взрослым нужно \explain{поаккуратнее}{more carefully} \explain{подходить к этим вопросам}{approach these questions} и в первую \'{о}чередь выяснить --- это им так кажется, или сам ребенок ощущает, что у него к чему-то талант и он готов в этом направлении развиваться. Очень часто ощущения родителей и детей не совпадают. Например, родители считают, что ребенок математически одаренный, а он мечтает играть на кларнете и в любую свободную минуту летит к инструменту, потому что это ему по-настоящему нравится. При этом он занимается математикой, олимпиадник и так далее, но только потому, что он \explain{послушный}{obedient} ребенок. У меня такие случаи были. Ребенок --- член команды Москвы по шахматам со всеми разрядами, подающий очень большие надежды. В девятом классе мальчик сказал родителям, что на \explain{юниорский}{junior} чемпионат он не поедет и эту страницу своей биографии закрыл. Он намерен поступать на мехмат МГУ, а значит, \explain{оставшиеся}{remaining} до окончания школы два года будет заниматься именно этим. Скандал был нереальный. Но парень, \explain{надо отдать ему должное}{we must give him his due}, \explain{выдержал}{survived}.


\subsection{Ведущая деятельность ребенка --- игра, отец заменяет ее учебой}
\textit{Александр Снегуров, \explain{заслуженный}{distinguished} учитель России, кандидат психологических наук}

Корреспонденты обратились ко мне за комментарием феномена, я высказал свое мнение. Это были корреспонденты телеканала «Россия 24». А потом мне сообщили, что девочку \explainDetail{опрос\'{и}ли}{опрашивать/опросить}{to question, to interview} --- есть ролик с выступлениями ее отца, который я не смог посмотреть. Так вот, ребенку задали ряд вопросов, и выяснилось, что она не знает каких-то тривиальных вещей, после чего выпуск сюжета \explainDetail{отмен\'{и}ли}{отменять/отменить}{cancel, revoke, abolish}.

Да, неудобно говорить о ее достижениях, когда она не знает обычных вещей. А я это допустил еще до ее опроса. Потому что тут \explain{налиц\'{о}}{obvious; present; on hand} диссонанс и \explain{нарушение}{violation}, так скажем, динамики по всем направлениям развития. Обязательно что-то будет отставать. В случае этих вундеркиндов это практически все время имеет место. Знаете ведь, как \explain{тесто}{dough} \explainDetail{замешивают}{замешивать}{knead}? А потом уже от замешанного теста можно взять кусочек и потянуть вверх. Мы можем вытянуть достаточно высок\'{о}. Так же и с детьми. Представим, что мы вытянули один \explain{показатель}{indicator} из общей \explainDetail{палитры}{пал\'{и}тра}{palette} развития ребенка. В данном случае --- сегмент ее интеллектуальной \explain{составляющей}{component}. А социализация и психологическое развитие, оставшиеся в этом \explainDetail{т\'{а}зике}{т\'{а}зик}{basin}, соответствуют возрасту. А мы-то хотим ориентироваться на высоту вытянутого сегмента, чтобы остальное ему соответствовало.

А потом удивляются, почему у таких детей стрессы и \explain{изломанные}{broken} судьбы, а в лучшем случае возвращение к типичному пейзажу в привычный \explain{ландшафт}{landscape} \explainDetail{сверстников}{сверстник}{peer}. Подобное ускоренное обучение \explainDetail{чревато}{чрев\'{а}тый}{fraught with (+\textit{твор.})} этими факторами. Я не буду говорить, что это дурно, просто обозначаю. Ты даёшь обучение, но надо понимать, что миновать вид детства не рекомендуется. Нужно понимать, что \explain{ведущая}{leading} деятельность ребенка --- игра, а в данном случае ее отец \explainDetail{заменяет}{заменять/заменить}{replace} ее учебой, но психика ребенка настроена на эту смену не сейчас.

Да он говорил, что она гуляет и играет, но это все сочетается, может быть, только на его взгляд. А может быть, это всего лишь имитация сочетания. А потом \explain{выявится}{comes to light} большой диссонанс, который обнаружится \explain{внез\'{а}пно}{suddenly}. Допуст\'{и}м, ребенок вдруг чего-то не захочет, например, жить на свете или находиться среди этих людей. Вдруг он может заявить своим родителям --- вы меня \explainDetail{измучили}{измучивать/измучить}{to weary, to exhaust, to tire out} и \explainDetail{достали}{достать}{(colloq.) to annoy sb badly}. Этого не стоит исключать. Я не говорю, что это может \explainDetail{произойти}{происходить/произойти}{happen} в обязательной перспективе, но исключать этого нельзя, и стоит иметь это постоянно в виду.

Я \explain{склонен}{inclined} к позиции, что ее \explain{натаскали}{coached} на сдачу ЕГЭ, при этом \explain{не отрицая}{not denying (отрицать)} все-таки ее интеллектуальных возможностей. Однако в таком случае игнорируется --- хотя не должен --- опыт взросления. Ряд школьных предметов основан именно на \explainDetail{постепенном}{постепенный}{gradual} взрослении, \explain{к примеру}{e.g.,}, литература и история. Ну что она, «Войну и мир» прочитала?

Она поступила на психфак, не \explainDetail{созрев}{созревать/созреть}{to mature [short past adverbial perfective participle of созреть]} и не испытав этапов взросления, которые нужно пройти человеку. Хоть и говорят, что мы, психологи, работаем с опытом другого человека, однако если у тебя есть собственный опыт, это точно не \explain{помешает}{will interfere}. Я не говорю, что каждый должен пройти через большую драму или катастрофу и тогда он сможет работать психологом. Но, \explain{разум\'{е}ется}{of course}, какие-то приблизительные ощущения и \explainDetail{переживания}{переживание}{experience} должны быть, чтобы работа была успешной.

Рост психологический и физиологический обязательно должен отразиться на психике и \explainDetail{сознании}{сознание}{consciousness} ребенка, чтобы была полноценная картина, а этого, по-моему, не произошло, ведь отец мог интеллектуально ее \explain{подгонять}{to fit, to adjust, to accommodate}. Но само созревание\dots{}  Да, может, у нее идут и эти процессы быстрее, но это еще не значит, что они соответствуют ее интеллектуальным \explainDetail{свершениям}{свершение}{accomplishment}. А там, может быть, и свершения интеллектуальные не по всем пунктам. Интересно было бы побеседовать с ребенком не в рамках ЕГЭ, а в рамках общей эрудиции и взглядов на мир. \explain{Сложилось}{to (be) form(ed), to develop, to turn out, to take shape. Example: Оп\'{а}сная ситу\'{а}ция слож\'{и}лась: a dangerous situation took shape} ли у нее \explain{мировоззрение}{world view}, а не \explain{набор}{set} фактов, который опирается на память. Но \explain{складывание}{folding} и формирование мировоззрения --- другая вещь. И жизненный опыт в том числе.

Они и в вузе хотят ускорить обучение, чтобы она окончила его в 11 лет. Задаюсь вопросом: работать она не может, а значит может быть провисание, которое, не исключено, может привести к экзистенциальному кризису: а зачем вы это сделали со мной? А что мне теперь делать, а где мое детство? Это один вариант. Другой --- она сама вернется в свою детскую парадигму, ну и как-то все в общем пейзаже сравняется.

Сам я \explain{сторонник}{supporter} отмены ЕГЭ, но это отдельная беседа. Этот экзамен проверяет память и приспособленность ребенка к определенным процедурам, а это отнюдь не весь спектр потенциала ребенка. У кого-то память не так хороша, у кого-то лучше. Много у меня опций критического характера по этому вопросу. Однако я за различные сценарии итоговой аттестации.

\begin{fancyquotes}
    Возможен и драматичный вариант. Редкий случай, что отец постоянно будет подкидывать ей поленья новых задач, чтобы она не \explain{простаивала}{to idle}. Но \explain{вряд}{hardly} ли это получится в той динамике, которую он задал. Однако до какого-то возраста он может это делать, потом же этот механизм у ребенка-вундеркинда даст \explain{сбой}{glitch, failure}
\end{fancyquotes}

\explainDetail{По поводу}{по п\'{о}воду чег\'{о}-либо}{concerning something} детей-вундеркиндов я считаю, что родители развивают уже имеющуюся \explainDetail{почву}{почва}{soil}. Можно сказать, что это \explain{дар}{gift} или наказание, но в любом случае это данность, которую родители заметили и н\'{а}чали развивать \explainDetail{доступными}{доступный}{accessible, available} способами. Это нетипичный пейзаж, который можно называть \explain{своеобразным}{своеобразный}{peculiar} нарушением или отклонением от нормы. \explainDetail{Обижать}{обижать}{to offend, to hurt sb's feelings} не станем, но примеры многих талантливых людей, \explain{ув\'{ы}}{alas}, подтверждают эту позицию.

