% \chapter{Наркомания и Алкоголизм}

\section{Зависимость от наркотиков}

 {\it Источник: \url{https://20gp.by/informatsiya/}}

Зависимость от наркотиков (наркомания) --- это неспособность человека отказаться от приёма веществ, которые влияют на его психику. Вначале прием наркотических веществ вызывает эйфорический эффект. Появляется желание регулярно принимать наркотики с увеличением доз. Затем у человека исчезают защитные реакции, и постепенно формируется психическая и физическая зависимость от наркотического вещества с выраженным абстинентным синдромом («ломка»).

\textbf{Принчины возникновения}

Основная масса наркоманов употребляет наркотики не по медицинским показаниям. До конца точно неизвестно, почему одни люди становятся наркоманами, а другие нет, даже если они когда-либо употребляли наркотические вещества.

Считается, что роль в развитии наркомании играют определенные особенности личности (незрелость характера, слабый самоконтроль, чрезмерный интерес к новым ощущениям). Есть данные о \ed{предрасположенности}{предрасположенность}{predisposition} к наркомании людей с психопатическими чертами характера, душевно нездоровых. \explain{Повышенный риск}{increased risk} развития наркомании имеют люди, которые выросли в \ed{неблагополучных семьях}{неблагополучная семья}{dysfunctional family}. Не исключено, что склонность к этому заболеванию передается по наследству.

Немаловажную роль в распространении наркомании играют модные тенденции.

\textbf{Симптомы зависимость от наркотиков}

На наркозависимость обычно указывают \explain{косвенные признаки}{indirect signs}:
\begin{enumerate}
    \item неестественный блеск в глазах, расширенные или сильно \explain{суженые зрачки}{constricted pupils} (в зависимости от типа наркотика);
    \item \explain{несвойственное}{unusual} для человека поведение  - гиперактивность или вялость, которые сопровождаются нарушением координации движений;
    \item резкие \explain{перепады настроения}{mood swings};
    \item изменение манеры устной или письменной речи, \ed{почерка}{почерк}{handwritting};
    \item беспричинные бледность или покраснение кожи;
    \item изменение аппетита;
    \item сбой режима сна и бодрствования.
\end{enumerate}

Cимптоматика зависит от вида наркотического вещества:

У тех, кто употребляем марихуаны, расширены зрачки, глаза и губы краснеют. Человек гиперактивен, у него заметно повышается аппетит.

Признаки опийной наркозависимости (опиомания) – внезапная \explain{заторможенность}{lethargy}, \explain{сонливость}{drowsiness; sleepiness}, \explain{замедление речи}{slow speech}. Зрачки сужены, кожа бледная, губы становятся красными.

Под действием психостимуляторов движения человека порывистые и резкие. Зрачки расширены. Он быстро принимает решения, речь связная и беглая.

После употребления галлюциногенов у человека появляются зрительные или звуковые иллюзии. Следствием их длительного применения становятся психотические состояния и депрессивные психозы.

Дополнительные возможные признаки наркомании – следы от уколов, пятна на коже, порезы, ссадины, припухлости непонятного происхождения.

\textbf{Диагностика}

Вначале врач определяет клинические признаки интоксикации (нарушение координации движений, расстройства речи, нарушение мышления, изменение поведения, сознания, вегето-сосудистых реакций). Чтобы убедиться, что клиническая картина вызвана содержанием в организме человека одного или нескольких наркотических веществ, проводится химико-токсикологический анализ.

Для подтверждения зависимости от опиума и его производных используют налорфиновый тест на наркотик.

Не каждый наркотик можно выявить по результатам лабораторных исследований, потому что некоторые из них быстро разрушаются в организме. В таких случаях часто используют методы токсикологической биохимии – мембранную хроматографию и газовую хроматографию.

Установить факт употребления наркотика можно в домашних условиях с помощью экспресс-теста на наркотики (в образце мочи). Этот тест обнаруживает следы опиатов на протяжении 5 суток после однократного приема.


\textbf{Классификация}

К основным видам наркомании относятся: опиоидная наркомания, наркомания от приема стимуляторов, кокаиновая наркомания, каннабиоидная наркомания и другие.  Если наркоман принимает разные наркотики, говорят о полинаркомании.

\textbf{Действия пациента}

Для успешного лечения в специализированном стационаре наркозависимому человеку нужна сильная мотивация. Найти ее помогут родственники и близкие люди. Пациенту необходима госпитализация.

\textbf{Лечение зависимости от наркотиков}

Основные этапы лечения наркозависимости следующие:

дезинтоксикационная, общеукрепляющая, стимулирующая терапия в сочетании с отнятием у пациента наркотического вещества, которым он злоупотребляет;

активное антинаркотическое лечение;

противорецидивная терапия.

Суть лечения сводится к тому, что сначала наркомана избавляют от физической зависимости, а затем проводят курс психотерапевтической реабилитации и поддержки, чтобы в его сознании закрепилась мысль, что ему будет хорошо без наркотиков.

Необходимым условием лечения наркомании является госпитализация и наблюдение за состоянием и поведением пациента.

Для лечения применяют такие методы как гипноз, кодирование, применение успокоительных психотропных препаратов и нейролептиков. Также существуют методики, которые базируются на лояльном отношении к пациенту и не предусматривают принудительного лечения.

\textbf{Осложения}

Без лечения наркозависимость влечет за собой тяжелые последствия: разложение личности, снижение интеллекта вплоть до слабоумия, истощение жизненно важных органов, снижение иммунитета. Многие наркоманы склонны к самоубийству.

Среди наркоманов распространены ВИЧ-инфекция, гепатиты, вызванные нарушением техники внутривенных инъекций. Передозировка наркотика часто заканчивается смертью наркомана.

\textbf{Профилактика зависимости от наркотиков}

Первичная профилактика наркомании лежит в социальной плоскости и направлена на сохранение и развитие условий, которые способствуют здоровью человека, и на предупреждение влияния неблагоприятных факторов.

Вторичная (социально-медицинская) профилактика наркомании направлена на выявление ранних изменений в организме для срочного полного и комплексного лечения, оздоровления среды, в которой находится наркозависимый, и применение воспитательных мер.

Третичная (медицинская) профилактика направлена на предупреждение прогрессирования заболевания, предупреждение обострений и осложнений, на снижение уровня инвалидности и смертности.

\newpage
\section{Почему человек становится зависимым от наркотика}

 {\it \url{https://www.israclinic.com/}}

\begin{fancyquotes}
    Основной причиной наркотической зависимости является желание испытать состояние эйфории, что в дальнейшем приводит к гонке за острыми ощущениями. Практически всегда сначала формируется психологическая тяга, а затем – физическая зависимость. Довольно быстро наступает состояние, когда наркотик принимается не ради удовольствия, а для поддержания нормального функционирования организма.
\end{fancyquotes}

Зависимость от наркотических веществ стала настоящим бичом современной молодежи – из-за легкой доступности все больше подростков пробуют наркотик, и впоследствии становятся зависимыми от него. Дело в том, что наркотические вещества на начальном этапе употребления вызывают состояние эйфории и приподнятости настроения, отодвигая любые имеющиеся проблемы на задний план, что в дальнейшем вызывает у человека потребность еще и еще раз испытать приятные ощущения. Однако эйфория быстро заканчивается, и тогда вступает в дело физическая зависимость, которая заставляет наркомана ежедневно искать средства на дозу, чтобы не чувствовать страшных симптомов наркотической абстиненции.

\textbf{Почему люди начинают принимать наркотики?}

Специалисты выделяют ряд причин, по которым люди начинают принимать наркотические вещества:

\begin{enumerate}
    \item уход от реальности. Наркотики дают возможность на какое-то время забыть о проблемах и неприятностях, погружая зависимого в иную реальность. Однако не стоит забывать, что проблемы сами по себе никуда не денутся, а со временем и с приходом наркотической зависимости их попросту станет еще больше;
    \item за компанию. Это наиболее частая причина, по которой подростки и молодые люди начинают принимать наркотические вещества. Подростка очень легко можно «взять на слабо», что в результате за считанные дни и недели формирует опасную наркотическую зависимость;
    \item жажда острых ощущений. Кто-то прыгает с парашютом в погоне за адреналином, кто-то отправляется путешествовать, а кто-то пробует наркотики. Из простого любопытства;
    \item чтобы понять близкого человека. Наркологи неоднократно фиксируют случаи, когда жены или родные наркомана начинают употреблять наркотические вещества. Отчасти это связано с тем, что такие люди морально надломлены, а также желают лучше понять близкого человека. Понятно, что в итоге это не приводит ни к чему хорошему. Сначала формируется психологическая зависимость от наркотика, а потом – физическая, что в результате уже требует лечения у специалистов.
\end{enumerate}

\textbf{Виды зависимостей от наркотиков}

Зависимость от наркотиков может быть:
\begin{enumerate}
    \item Психологическая;
    \item Физическая.
\end{enumerate}

В начале употребления формируется психологическая зависимость. Человек пробует наркотическое вещество и понимает, что ему нравится чувство эйфории, хорошее настроение и другие эффекты от употребления наркотика. В погоне за острыми ощущениями он начинает пробовать еще и еще. Далее, в зависимости от вида наркотика, спустя некоторое время формируется физическая зависимость, когда наркоман вынужден постоянно принимать психоактивное вещество. Однако он это делает не для того, чтобы вызвать приятные ощущения, а чтобы избежать физических мучений, на сленге именуемых как «ломка».

Последствия действительно страшные: несколько дней и недель может пониматься температура тела до 40°, появляется сильнейшая боль в суставах, расстройства ЖКТ и сбои в работе сердца, отсутствие аппетита, тревога. Также при этом появляются проблемы со сном, наркомана все время мучает бессонница. К этому следует еще прибавить подавленное настроение, депрессию, обильное потоотделение, кошмары, а порой и галлюцинации. Физическая и психическая зависимости от наркотика в итоге превращают некогда здорового человека в пациента наркологических центров и реабилитационных учреждений. Необходимо серьезное лечение, которое не только избавляет человека от физической (химической) зависимости, но и помогает переосмыслить свою жизнь, излечить психологическую тягу к наркотику.


\newpage
\section{Алкоголизм: причины, симптомы, стадии, лечение}

 {\it \url{https://polyclinika.ru/tech/alkogolizm-prichiny-simptomy-lechenie/}}

Основное отличие алкоголика от бытового пьяницы – психическая и физическая зависимости. Пьяница пьет по обстоятельствам, у него практически не бывает запоев. Может воздерживаться после домашнего скандала или когда закончились деньги. Алкоголик на такие «мелочи», как карьера или разрушающаяся семья, не обращает внимания, потому что употребление спиртосодержащих жидкостей находится на первом месте. Алкоголизм всегда, в 100\% случаев, сопровождается деградацией личности и разрушением внутренних органов.

\textbf{Особенности алкоголизма}

Алкоголизм относится к группе токсикоманий. Всемирная организация здравоохранения утверждает, что употребление алкоголя возрастает одновременно с улучшением качества жизни.

Российская статистика несколько иная: в нашей стране количество алкоголиков увеличивалось в тяжелые годы, а за десятилетие с 2008 по 2017 сократилось более чем на 1 миллион человек.

В мире бремя этой тяжелой зависимости несут более 140 миллионов человек. Неоспоримо установлено, что заболевание неизлечимо. Древнегреческий потомственный врач Гиппократ говорил, что пьянство – добровольное безумие. Выражение актуально по сей день, поскольку единственное, что может сохранить разум и тело алкоголика – полный, абсолютный отказ от спиртного.

\textbf{Причины алкоголизма}

Исследователи разделяют причины формирования болезни на 3 равнозначные группы:
\begin{enumerate}
    \item биологическая – генетическая предрасположенность, нервные и психические болезни раннего детского возраста, особенности функционирования нервной системы в виде превалирования процессов возбуждения;
    \item социальные – среда обитания, в которой сформированы питейные традиции, возлияния по любому поводу при низком уровне моральных ценностей;
    \item психологические – изначально низкая самооценка, тревожность, неуверенность, неумение справляться со стрессом, только прием спиртного приносит удовольствие и вызывает эйфорию.
\end{enumerate}

Особое значение медики придают наследственности, когда признаки алкоголизма имеются у родителей пациента. Дети алкоголиков рискуют повторить путь родителей на 30\% больше, чем их здоровые сверстники. С болезненными проявлениями дети знакомятся в слишком раннем возрасте, когда критика еще невозможна, и легко следуют проторенным путем.

Коварство зависимости в том, что организм любого человека вырабатывает эндогенный этанол, необходимый для биохимических процессов. Внешний этанол от того, что выработан внутри, ничем не отличается. К тому же у пациентов недостаточно фермента, расщепляющего этанол, – алкогольдегидрогеназы.

\textbf{Симптомы и стадии алкоголизма}

Зависимость проходит несколько этапов:
\begin{enumerate}
    \item Патологическое влечение – игнорируются все потребности, кроме выпивки. Определить, что человек все-таки станет алкоголиком, просто: пациент всегда находит спрятанное спиртное, в любом месте и в любое время, и делает это по запаху.
    \item Угасание рвотного рефлекса – защитный рефлекс «выключается».
    \item Утрата количественного контроля – нет меры, выпивка «до упора», иногда до потери сознания.
    \item Повышение переносимости с последующим снижением – если на начальных стадиях для достижения опьянения требуются литры, то в финале – рюмки.
\end{enumerate}

Врачи выделяют несколько стадий алкоголизма:
\begin{enumerate}
    \item продромальная – группа риска, человек никогда не отказывается от употребления спиртного, когда есть такая возможность, особенно в компании;
    \item первая – есть психическая и физическая зависимости, тяжелое похмелье, эпизоды провалов памяти (амнезия), колебания настроения, но сохраняется семья и работа;
    \item вторая, часто обозначаемая как «хронический алкоголизм» – сформированы запои (непрерывное употребление), но есть светлые промежутки, присоединяются поражения центральной и периферической нервной систем, количество растет до максимума, многие теряют работу, возникает разлад в семье;
    \item третья – присоединяются осложнения в виде алкогольных психозов и судорожных приступов, разваливается личность, поражаются все органы и системы, утрачиваются социальные связи, для опьянения требуется минимальная доза.
\end{enumerate}

Алкоголизм у мужчин и женщин принципиального отличия не имеет, за исключением того, что женщины проходят все стадии втрое быстрее мужчин, полностью утрачивая привлекательность и скатываясь к неконтролируемой сексуальности.

\textbf{Диагностика алкоголизма}

Диагностика включает полноценное обследование, в котором, помимо нарколога и психиатра, участвуют терапевт и по ситуации – кардиолог, невролог и гастроэнтеролог. Нарколог устанавливает ведущие синдромы алкоголизма – толерантность или переносимость дозы, запои и их длительность. Психиатр совместно с клиническим психологом исследует структуру личности, интеллектуально-мнестическое снижение, обращая внимание на несчастные случаи в состоянии опьянения, оценивает профессиональное и личностное снижение. Также психиатр занимается лечением острых галлюцинозов и хронических бредовых образований.

Невролог лечит судорожный синдром, энцефалопатии (повреждения головного мозга), периферические невропатии, нарушения ходьбы и статики, нарушения чувствительности, инсульты и тромбозы.

Терапевт занимается проблемами сердца (гипертоническая болезнь, нарушения ритма) и дыхательной системы, гастроэнтеролог – болезнями печени, особенно циррозом.

Женский алкоголизм требует участия гинеколога, поскольку этанол разрушает детородные органы, нарушает менструальный цикл, лишая пациентку возможности стать матерью.

Необходимый и важнейший этап диагностики – принятие пациентом своей проблемы, осознание себя больным человеком. На первоначальном этапе усилия медиков направлены именно на то, чтобы пациент признал себя алкоголиком. Без этого невозможно двигаться дальше, в противном случае лечение окажется формальным, пациент «сорвется» при первом удобном случае.

Полезно для диагностики помещать пациента в стационар, где он имеет возможность наглядно и «во всей красе» наблюдать последствия алкоголизма в виде приобретенного слабоумия, серий судорожных приступов или невозможности самостоятельно ходить. Пациент, не утративший способности критически мыслить (на первой или второй стадиях), обычно осознает всю тяжесть проблемы, иногда прекращая употребление спиртного навсегда.

Особого внимания заслуживает пивной алкоголизм (гамбринизм). Эта форма заболевания отсутствует в официальной классификации болезней, травм и причин смерти, поскольку относится к журналистским штампам. Врачи говорят, что болезнь одна, и не имеет значения напиток, который послужил причиной формирования зависимости.

Тем не менее на практике эта форма встречается часто. Пациенты и, к сожалению, родственники относятся к этой форме несерьезно, начиная лечение только на стадии осложнений. Так, «пивное сердце» описано еще врачами Баварии в 1914 году у рабочих пивоваренного завода. Заболевание развивается медленно и как бы незаметно, однако изменения необратимы.

Наиболее точный и простой тест на алкоголизм (MAST) был разработан в 1971 году в Мичиганском университете. Это опросник, описывающий отношение к спиртному, частоту употребления, особенности похмелья, запои, защитные рефлексы, отношение родных, повод выпить и другие особенности. Вариантов опросника существует несколько, они имеют незначительные различия.

Беседа с опытным доктором, тестирование, врачебный осмотр направляют пациента на путь выздоровления. Иногда достаточно содержательной экскурсии в наркологическое отделение с демонстрацией пациентов в терминальной стадии, чтобы получить осознанное согласие на лечение.

Врач-нарколог и тем более психиатр имеют колоссальный опыт общения с пациентами подобного профиля, умеют в первые минуты нащупать болевые точки личности, подобрать слова для убеждения. Возможно, пациент и не согласится лечиться на первом посещении, но однозначно встречу запомнит, сможет обратиться в критической ситуации.

\textbf{Лечение алкоголизма}

Тактика лечения определяется стадией болезни и личностью пациента.

Все начинается с выведения из запоя, которое разумно проводить в стационарных условиях. Состояние пациента может быть настолько тяжелым, что требуется участие реаниматолога.

На детоксикацию отводится несколько дней. Универсального лекарства от алкоголизма не существует, используются медикаменты для коррекции кислотно-щелочного равновесия, антиаритмические и гипотензивные средства, мочегонные, противоотечные, противорвотные, антиконвульсанты, нейролептики и другие.

В легких случаях, когда нет угрозы жизни, этап детоксикации проводится в амбулаторных условиях или на дому. После прерывания запоя с пациентом подробно беседуют, выявляя его истинные установки и устремления.

Отличный результат дает кодировка от алкоголизма, которая бывает медикаментозной или гипнотической. При медикаментозном варианте (подшивка) через небольшой разрез вживляется дисульфирам, дающий резкую вегетативную реакцию при употреблении спиртного – сердцебиение, потливость, дрожь, подъем артериального давления. Пациента предупреждают о последствиях, обычно берут расписку во избежание судебного разбирательства в случае гибели.

При гипнотической кодировке пациента погружают в транс (пограничное между сном и бодрствованием состояние), закрепляя в его психике установку на трезвость. Такой вариант подходит пациентам внушаемым, проходящим лечение впервые или имеющим предыдущий положительный опыт. Этот способ полезно подкреплять каплями и таблетками от алкоголизма, содержащими химические соединения и травяные сборы.

Пациента, живущего в семье и имеющего работу, можно удерживать от рецидива длительное время. Бывших алкоголиков не бывает, разумеется, но за время ремиссии пациент успевает ощутить вкус и преимущества обычной жизни, самостоятельно прилагает усилия для того, чтобы не вернуться к пагубному пристрастию.

Лучшая профилактика алкоголизма – реальные жизненные цели, которые невозможно осуществить с пьянством: семья, дети, карьерный рост, повышение благосостояния. Психологическая коррекция также требуется от жены и других важных для пациента родственников, поскольку избавления от такой разрушительной зависимости можно достичь только общими усилиями.