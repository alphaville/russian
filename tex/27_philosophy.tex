\chapter{Философия}

\section{Слава богу, мы атеисты}

\textit{Источник: \url{https://4brain.ru/blog/slava-bogu-my-ateisty/}}

Чем лучше у человека развито мышление, тем более полно и объемно он воспринимает реальность, тем глубже он проникает в суть вещей с точки зрения науки. В этом плане мышление атеистов очень даже полезно.

Мы не призываем вас становиться атеистами, но рекомендуем изучать свой мозг, развивать мышление и учиться смотреть на все с разных точек восприятия, изучая события через научные концепции.

Предлагаем начать изучение этого вопроса с наиболее актуальной темы для каждого читателя – с точки зрения навыков, которые помогают оставаться на плаву в быстро меняющемся мире.

\begin{fancyquotes}
    \textbf{Некоторые цифры}\\

    {\Huge 55\%}\\
    людей сегодня проживают в городах (в 2000 году проживало менее 50\%).\\[1em]

    {\Huge 89\%}\\
    людей сегодня имеют доступ к Интернету (в 1998 году имели лишь 41\%).\\[1em]

    {\Huge 95\%}\\
    людей сегодня пользуются мобильными телефонами (в 1996 году пользовались лишь 60%).

\end{fancyquotes}

Например, куда нужно идти учиться и какую профессию выбирать, чтобы быть уверенным, что «завтра» она не окажется неактуальной и ненужной? Какое давать образование своим детям и как их воспитывать, чтобы они смогли достичь успеха и стали полноценными членами общества, когда вырастут?

От ответов на эти вопросы зависит качество вашей жизни, и, если вы хотите именно ту жизнь, о которой мечтаете, вы должны быть гибкими, а для этого нужно начать овладевать новыми навыками, которые будут являться современными на протяжении хотя бы нескольких ближайших десятилетий и, вполне возможно, даже дольше.