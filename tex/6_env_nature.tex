\chapter{Окружающая среда и природа}

\section{Городск\'{а}я жизнь}
Вы никогда не думали о последствиях жизни в городе? Казалось бы, \explain{на первый взгляд}{at first sight}, что жизнь в больших экономических и культурных центрах имеет только преимущества, но дальнейшее рассмотр\'{е}ние показывает, что она имеет и \explain{недостатки}{недостаток: disadvantage}.

С \explain{положительной}{полож\'{и}тельный/-ая: positive} стороны, легче найти работу в городе, потому что там обычно много ресторанов, кафе, гостиниц, школ, библиотек, музеев и т.д. Кроме того, жители города имеют прекрасную возможность посетить множество культурных и развлекательных учреждений, таких как музеи, галереи, ночные клубы, дискотеки и многое другое.

С другой стороны, жителям города \explain{приходится}{приходиться/прийтись: have to} жить в загрязненной атмосфере из-за интенсивного автомобильного движения и \explain{промышленных}{industrial} \explain{предприятий}{предпри\'{я}тие: enterprise}. Это может \explain{вызвать}{to cause} \explain{заболевания}{disease} лёгких и проблемы с сердцем. Кроме того, городской образ жизни довольно \explain{напряженный}{intense}, \explain{поскольку}{since/because} приходится много работать, много ездить на автомобиле и, в результате, стоять в пробках...

В заключение, городская жизнь имеет некоторые преимущества. Тем не менее, она также \explain{может нанести ощутимый вред}{can cause significant damage}, так что местные власти должны сделать несколько важных решений, например, они должны \explain{запретить}{[запрещать] to ban} промышленные предприятия в городах и вблизи городов, которые загрязняют воздух и воду токсичными парами.