% \chapter{Безработица}

\section{Безработица}

\textit{Источник: Материал из Википедии — свободной энциклопедии}


Безработица — наличие в стране людей, составляющих часть экономически активного населения, которые способны и желают трудиться, но не могут найти работу.

Согласно методологии Международной организации труда (МОТ) к безработным относят людей трудоспособного возраста, которые не имеют работы в течение некоторого периода времени, способны трудиться и предпринимают усилия по поиску работы, но не могут найти ее. В качестве работы может рассматриваться не только работа по найму, но и самозанятость. Для обеспечения международной сопоставимости данных, МОТ рекомендует относить к трудоспособным всех людей старше 15 лет. Однако подчеркивается, что каждая страна вправе самостоятельно выбирать возрастные критерии трудоспособности. Например, может быть установлен предельный возраст.

В России методику оценки уровня безработицы разрабатывает Росстат. Согласно официальным документам Росстата, трудоспособными считаются граждане в возрасте от 15 до 72 лет. Учащиеся, пенсионеры и инвалиды относятся к категории безработных, если они занимались поиском работы и были готовы приступить к ней.

В 2020 году по данным Международной организации труда в мире насчитывается 400 миллионов безработных (5,26\% населения планеты). Всплеск безработицы произошёл из-за влияния пандемии COVID-19. В Европейском союзе по данным исследовательского центра Европарламента самый высокий уровень безработицы среди молодежи до 25 лет на январь 2018 года наблюдался в Греции (43\%), Испании (36\%) и Италии (31,5\%).

В России в 2018 году по результатам исследования РИА Новости ситуация с безработицей неоднородна, и самый высокий индекс безработицы был зафиксирован в регионах Северного Кавказа, на Алтае и в Тыве.

\textbf{История.} В традиционных обществах заработная плата за работу не выплачивалась, так как деньги вообще не использовались. Люди жили за счёт земли, и земля принадлежала всем, либо никому. Разделение труда было мало ощутимым. Когда были изобретены деньги и началось строительство городов, люди стали зависеть от них, покупая еду у продавцов, вместо того, чтобы выращивать, заниматься собирательством или охотиться самостоятельно. Зависимость от работы как от источника денег для приобретения еды и жилища является основой безработицы.

Количество исторических источников, посвящённых проблеме безработицы ограничено, так как наблюдения велись не всегда и не везде. В определённый исторический период индустриализация привела к отчуждению средств производства от работников и свела к минимуму возможность их самозанятости. Таким образом, работник, который, по каким-то причинам, не имел возможности устроиться на предприятие, не мог самого себя обеспечить работой и становился безработным. Ситуация усугубилась тем, что работники индивидуальных профессий, например врачи, фермеры, ранчеры, прядильщики, мелкие торговцы, стали образовывать крупные закрытые профессиональные объединения и тем, кто не входил в них, приходилось работать в условиях жёсткой конкуренции или становиться безработными.

Безработица как явление стала постепенно входить в экономическую мысль по мере усиления индустриализации и бюрократизации. Процесс формирования этого понятия можно рассматривать на примере Великобритании, так как там он был хорошо задокументирован. В XVI веке в Англии не делалось различий между бродягами и теми, у кого нет работы, все официально именовались постоянные попрошайки (Sturdy beggar[en]) и рассматривались как лица, которых нужно наказать и выслать. Закрытие монастырей в 1530-х годах увеличило нищету, так как монастыри помогали бедным. Кроме того, во времена Тюдоров возросло население и усилился процесс огораживания. У безработных оставалось только два выхода — голодать или нарушать закон.

В 1535 году вышел закон, предусматривающий создание системы общественных работ для борьбы с безработицей, которая финансировалась за счёт налогов на прибыль и капитал. Вышедший в следующем году закон разрешал применять к бродягам телесные наказания.

\textbf{Учёт безработных}

\textit{Международные стандарты учета.}
Единые стандарты учета безработицы разрабатываются Международной организацией труда. Однако отдельные страны могут иметь свои особенности учета. Например, МОТ допускает изменение возрастных критериев отнесения к экономически активному населению. Различия уменьшают международную сопоставимость показателей.

Согласно методике МОТ, безработным считается человек в трудоспособном возрасте (старше 15 лет), который:
\begin{enumerate}
    \item не имеет работы в течение некоторого времени и не является самозанятым;
    \item имеет возможность работать или стать самозанятым;
          предпринимает усилия по поиску работы.
\end{enumerate}


К безработным также относят людей, которые:
\begin{enumerate}
    \item не ищут работу, но готовятся начать поиск в будущем;
          заняты профессиональной подготовкой или переподготовкой и планируют приступить к работе в течение ближайших трех месяцев;
    \item собираются переехать в другую страну ради работы, но ждут переезда.
\end{enumerate}
К категории занятых относят тех, кто имеет оплачиваемую работу по найму или является самозанятым. Занятые и безработные входят в состав рабочей силы (устаревшее название — экономически активное население).

МОТ рекомендует использовать опросы для оценки количества занятых и безработных, так как опросы позволяют получить оценки численности одновременно и по единой методике. Использование данных об официально зарегистрированных безработных трудно использовать для сопоставления с другими данными, полученными из опросов.

\textit{Учёт в России.} В России критерием трудоспособности является возраст от 15 до 72 лет. Учёт безработного населения ведётся двумя методами:
\begin{enumerate}
    \item по данным Министерства труда и социальной защиты Российской Федерации на основании обращений безработных в службу занятости. Поскольку у значительной части населения отсутствует стимул к регистрации своего статуса как безработного, сводные данные являются некорректными. Такие сводные данные публикуются в статистических сборниках справочно[11].
    \item по данным обследования населения по проблемам занятости, которое проводится Росстатом по методике МОТ. До сентября 2009 года оно было ежеквартальным, а начиная с сентября 2009 года оно стало ежемесячным. Объём выборки для обследований определён в размере 0,06\% численности населения в возрасте 15-72 лет на квартал и 0,24\% — на год. В качестве основы выборки используются материалы переписи населения. Размер общероссийской выборки составляет около 260 тыс. чел. (приблизительно 120 тыс. домашних хозяйств), что соответствует 0,24\% численности населения данного возраста. Ежеквартально в целом по России обследуются около 65 тыс. лиц в возрасте 15-72 лет (около 30 тыс. домашних хозяйств), или 0,06\% от численности населения данного возраста.
\end{enumerate}

\textit{Европейский союз (Евростат).}
Евростат ― статистическое управление Европейского союза, ― определяет безработных как лиц в возрасте от 15 до 74 лет, не имеющих трудоустройства, находящихся в поиске работы в течение последних четырёх недель и готовых начать работу в течение двух недель, что соответствует стандартам МОТ. Евростат ведёт учёта как фактического количества безработных, так и уровень безработицы в странах ЕС. Статистические данные доступны по странам-членам Европейского союза в целом

(EU28), а также по Еврозоне (EA19). Евростат также выделяет долгосрочный уровень безработицы, который определяется как количество безработных, которые пребывают в таком состоянии более одного года.

Основным источником информации для Евростста является программа Исследования рабочей силы Европейского Союза (EU-LFS). Она собирает данные обо всех государствах-членах каждый квартал. Для ежемесячных расчётов используются национальные обследования или национальные реестры бюро по трудоустройству в сочетании с ежеквартальными данными EU-LFS. Точный расчёт для отдельных стран, приводящий к согласованным ежемесячным оценкам, зависит от доступности этих данных.

\textbf{Последствия безработицы}

Безработица влечет за собой ряд негативных последствий как для отдельных экономических агентов на микроуровне, так и для экономики на макроуровне.

Последствия на микроуровне.
\begin{enumerate}
    \item Снижение располагаемых доходов и сбережений. Например, снижаются пенсионные накопления, что подрывает благосостояние в будущем.
    \item Потеря квалификации из-за вынужденного простоя.
    \item Возникновение эффекта гистерезиса на рынке труда, когда работники покидают экономически активное население во время кризиса и больше не возвращаются к поиску работы, предпочитая жить на пособие.
\end{enumerate}
На психологическом уровне для конкретного человека потеря работы может обернутся личностным кризисом. «Безработный, пусть даже он обеспечен достойным пособием, опасен. Особенно в России», — отмечает академик РАН Виктор В. Ивантер.

Последствия на макроуровне.
\begin{enumerate}
    \item Недополученный выпуск, возникающий из-за отклонения фактического ВВП от потенциального в результате неполного использования совокупной рабочей силы (см. Закон Оукена).
    \item Сокращение доходной части федерального бюджета в результате уменьшения налоговых поступлений.
    \item Рост общественных затрат на защиту работников от потерь, вызванных безработицей: выплату пособий, реализацию программ по стимулированию роста занятости, профессиональную переподготовку и трудоустройство безработных и т.д.
\end{enumerate}

\newpage
\section{МОТ: в 2022 году число безработных в мире составит 207 млн}

\textit{Источник: \url{https://news.un.org/ru/story/2022/01/1417012}}

\begin{fancyquotes}
    В 2022 году число безработных в мире вырастет до 207 млн, что на 21 млн больше, чем в докризисном 2019 году. Об этом говорится в новом докладе Международной организации труда (МОТ).
\end{fancyquotes}

«Глобальный уровень безработицы, как ожидается, сохранится на уровне, превышающем допандемийные показатели, по меньшей мере до 2023 года», — считают авторы нового доклада. В 2019 году работу не могли найти 186 млн человек, а сегодня — 207 млн.


В МОТ отмечают, что ранние более оптимистичные прогнозы по восстановлению рынка труда не оправдались из-за последствий распространения недавних вариантов COVID-19, таких как «дельта» и «омикрон», а также неопределенности по поводу того, как вирус поведет себя в будущем.

Безработица в мире, как ожидается, сохранится на уровне выше докризисного как минимум до конца 2023 года. «Кризис продолжается уже два года, но перспективы остаются неопределенными, а восстановление остается вялым и непрочным», —заявил Генеральный директор МОТ Гай Райдер.

Он опасается, что ущерб рынкам труда от пандемии может стать необратимым. Сегодня наблюдается тревожный рост масштабов бедности и неравенства. Многим работникам приходится переключаться на новые формы труда — к этому их вынуждает, например, затяжной спад в сфере международного туризма.

Последствия пандемии для рынков труда ощущаются во всех регионах планеты, хотя темпы восстановление везде разные. Наиболее обнадеживающие признаки возрождения наблюдаются в Европе и Северной Америке, а наименее радужные — в Юго-Восточной Азии, Латинской Америке и Карибском бассейне.

Несоразмерно тяжелые последствия пандемии были для занятости женщин, они будут ощущаться в течение ближайших лет. Закрытие учебных заведений и учреждений профессиональной подготовки обернется тяжелыми долгосрочными последствиями для молодежи.

«Без всеобъемлющего восстановления рынка труда последствия пандемии в полной мере не преодолеть, — подчеркнул глава МОТ. По его словам, устойчивого восстановления можно добиться только на основе принципов достойного труда, включая охрану труда, равенство, социальную защиту и социальный диалог.

В докладе содержатся всесторонние прогнозы ситуации на рынке труда на 2022 и 2023 годы. В нем оценивается ход восстановления рынка труда в мире, рассказывается о подходах разных стран к вопросу восстановления после пандемии, анализируются ее последствия для различных категорий работников и отраслей экономики.

Эксперты отмечают, что в некоторых случаях, как и в ходе предыдущих кризисов, в условиях пандемии роль своего рода амортизатора играет временная занятость. В период кризиса многие временные рабочие места закрывались, но в то же время создавались другие, в том числе для тех, кто лишился постоянной работы. В среднем же число временных работников осталось неизменным.

В докладе о перспективах занятости и социальной защиты в мире содержатся рекомендации, призванные обеспечить всеобъемлющее и ориентированное на интересы человека восстановление после кризиса как на национальном, так и на международном уровне.