% \chapter{Мода}

\section{Как мода влияет на нашу жизнь?}

\textit{Источник: \url{https://absurdu.net/fashion/kak-moda-vliyaet-na-nashu-zhizn.html}}

С развитием современных технологий, соцсетей и рекламы большинство из нас осознает, что в той или иной мере мы таки поддаемся определенным трендам и модным тенденциям. И все же, существует немало тех, кто категорически не согласен с этим утверждением. Такие люди утверждают, будто бы мода никоем образом не влияет на их жизнь и, тем более, на поведение. Но так ли это на самом деле? Давайте разбираться!

\textbf{Отражает мировые события и общий настрой людей.}
Ни для кого не секрет, что мода всегда отражает культурные, социальные, политические, экономические и религиозные аспекты общества. Если в мире происходит что-то масштабное, это в любом случае тем или иным образом повлияет на fashion-индустрию. Так формируются определенные тренды. Яркое доказательство этому – тенденции 2021 года. Психологи говорят о том, что при возникновении неприятных событий человек нуждается в том, что могло бы положительным образом повлиять на его настроение. Этот принцип находит свое отражение и в модной индустрии. После пандемии и локдауна многие чувствуют усталость и подавленность. Дизайнеры, ощущая потребности своих клиентов, начали создавать более комфортную, но в то же время креативную одежду: мода понемногу отходит от нейтральных оттенков и непрактичных вещей, наоборот, повседневные предметы гардероба становятся более удобными, а палитра цветов гораздо ярче. Насыщенные цвета радуги положительным образом влияют на наше самоощущение и восприятие мира, а интересные фасоны, асимметрия или архитектурный крой в одежде смотрятся нарядно и элегантно, что способствует ощущению воодушевленности и праздничного настроения.

К интересным выводам пришли также те, кто занимается изучением истории fashion-индустрии. После масштабных войн или экономических кризисов одежда становилась все более откровенной. Фасоны прилегали к телу, подолы юбок и платьев укорачивались. С одной стороны, это было связано с практичностью, а с другой – с банальной нехваткой материалов. Из-за тяжелого положения в такие времена люди просто не могли тратить по 40 метров ткани на одно платье!
То же самое можно сказать и про сферу религии и морали. Все знают, что в мусульманских странах девушки не могут носить одежду, которая бы открывала их шею, запястья или щиколотки. Поэтому мода в таких государствах обычно гораздо более консервативная, но все же поддается изменениям. Тем более, что под хиджабом или паранджей женщина может носить то, что нравится ей, если это соответствует принципам одежды, записанным в Коране. Кроме того, в большинстве случаев они не ограничены в выборе аксессуаров. Так, например, во всемирно известном торговом центре Dubai Mall в ОАЭ, куда съезжаются люди из всех уголков мира, довольно часто можно встретить пару, в которой супруга идет в лодочках от любимого люксового бренда, а руках у нее оригинальная сумка Chanel или Dior.

\textbf{Продвигает инклюзивность, меняет представление о красоте.}
Конечно, представление о «красоте» очень субъективное, но, тем не менее, наше восприятие этого абстрактного понятия в огромной мере зависит именно от «канонов» fashion-индустрии. До 70-80-х прошлого века люди не сильно обращали внимание, на то, как выглядит их тело. Но с популяризацией спорта и развитием некоторых его видов в моду приходит «культ тела», подразумевающий подтянутую спортивную фигуру не только у мужчин, но также и у женщин. Вспомните также, как менялось представление об идеальных модельных параметрах: в 2000-е годы большинство представительниц этой профессии имели явно нездоровый вид, что чаще всего было следствием анорексии. Со временем, когда суровый fashion-мир наконец-то переосмыслил понятие красоты и здоровья, на подиумы стали выходить модели плюс-сайз. То же самое касается и темнокожих девушек: еще в конце прошлого века для большинства из них принять участие в показе или появиться на обложке глянца казалось недостижимой целью, а сегодня вопрос происхождения не является препятствием при построении карьеры. Если говорить об особенностях, как, например, кожа супермодели Винни Харлоу или страбизм (косоглазие) Брунетт Моффи, все больше людей также перестают обращать на это внимание.

Выпускники Лондонского колледжа моды Джудит Ахумба-Велленстайн, Сьюзан Джин и Пак Лунчиу, которые создали онлайн-журнал Hajinsky, посвященный фэшн-психологии, изучив то, какое воздействие мода имеет на жизнь большинства людей, пришли к следующему выводу:

\begin{fancyquotes}
    «Официальная одежда положительно влияет на абстрактное мышление человека, которое связано с такими действиями, как, например, экономия денег. Кроме того, наука показала, как бренды могут улучшить жизнь общества, сделав одежду инклюзивной. Исследования показали, что люди с ограниченными физическими возможностями часто для удобства носят спортивные костюмы и поэтому подвергаются двойной дискриминации — во-первых, из-за своей физических особенностей, во-вторых, из-за своей одежды. Бренды, принимающие это во внимание, могут в корне изменить жизнь человека».
\end{fancyquotes}

\textbf{Исполняет роль идентификатора личности.}
Мода также помогает нам больше понять других и служит неким идентификатором. Например, по национальному костюму мы можем догадаться, из какой страны человек, общий внешний вид иногда указывает на род деятельности, а характерная манера сочетать вещи может служить неким маркером «творчества», «вкуса» и т.д. Те же, кто все еще категорически отказываются принимать факт влияния моды на их жизнь, не могут отрицать того, что в современном мире просто нельзя избежать ситуаций, когда мы должны выглядеть определенным образом. Представьте, что вас пригласили на свадьбу, на торжественный ужин в дорогой ресторан или же вы идете на собеседование в крупное финансовое учреждение с наличием определенного дресс-кода. Для такой ситуации вы вряд ли наденете джинсы. Для подобных случаев существует различные наряды, в том числе классический деловой костюм, который так же пришел в нашу жизнь под влиянием определенных тенденций. Нельзя отрицать тот факт, что внешний вид человека также сказывается на том, как другие воспринимают его. Именно по этой причине мы часто задумываемся о том, какую одежду подбирать не только под определенную ситуацию, но даже под сам круг общения. Наше естественное желание нравиться способствует тому, что время от времени мы обновляем свой гардероб и добавляем туда красивые (которые соответствуют нашему субъективному вкусу) предметы одежды и аксессуаров.

\begin{fancyquotes}
    «Реальность такова, что мода стала частью нашей повседневной жизни. Она связана с политикой и может вызвать социальные перемеФормирует сознаниены. Мода также влияет на наши отношения друг с другом, наше поведение, она может улучшить нашу жизнь, помогая с самоидентификацией»,- заявили вышеупомянутые основатели журнала Hajinsky.
\end{fancyquotes}

\textbf{Формирует сознание.}
И это касается даже тех, кто никогда не следит за ней. Если все начинают носить оверсайз, со временем при покупке новой футболки вы также сделаете выбор в пользу прямого или более свободного фасона. Кроме того, это отражается и в количестве покупаемых предметов. Сейчас большинство людей приобретают намного больше одежды, обуви и аксессуаров, чем это было еще 30 лет назад.

\begin{fancyquotes}
    «При выборе гардероба важно ориентироваться на свои ценности и цели. При этом человек не может существовать в полном отрыве от общества. В этом смысле, если „психология моды“ про гармонизацию запросов общества и внутренних процессов человека, то это может повысить качество его жизни». – Говорит психолог Александра Меньшикова.
\end{fancyquotes}

Также нельзя забывать, что мода – это не только об одежде. Поэтому если вы не следуете трендам при формировании своего гардероба, это не значит, что вы не изучите последние тенденции, когда, например, будете делать ремонт в квартире. То же самое касается приобретения новой техники, автомобиля или даже посуды на кухню.

Хотите вы того или нет, мода в любом случае влияет на ваш выбор повседневных образов, домашней обстановки и даже места для отдыха. И это нельзя определить как «хорошо» или «плохо» — каждый оценивает по-своему. Остается лишь принять тот факт, что мода в большой степени влияет на разные сферы нашей жизни, а вот активно следовать новым тенденциям, принимать их или всеми силами пытаться убежать от них – личное дело каждого. Но, согласитесь, знание — сила. Осознанность поможет вам разумно подходить к вопросу трендов и тенденций, чтобы умело использовать этот инструмент для достижения своих целей. Актуальный и стильный внешний вид еще никому не помешал.