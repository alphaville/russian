\chapter{Общество}

\section{Конформизм в социальных медиа}

\textit{Источник: \url{https://4brain.ru/blog/konformizm-v-socialnyh-media/}}

Интернет стал \ed{находкой}{находка}{find, discovery (anything that is found, especially by good fortune)} для интровертов и других любителей самостоятельности: проверить погоду, смотреть сериалы онлайн, обучиться новому языку, скачать рецепт, выбрать себе новый мотоцикл в тишине и уюте квартиры.

Однако стоит только подписаться на какую-то группу по интересам, вступить в обсуждение, как оказывается, что и в Интернете люди \ex{ходят строем}{walk in formation}, как в армии, и «лайкают» оратора без царя в голове.

Как так получилось, что Интернет, который создавался как площадка для свободного \ed{волеизъявления}{волеизъявление}{expression of will} людей, превратился в \ed{пособника}{пособник}{accomplice} конформизма? Как социальные медиа влияют на людей и почему они проводят в Интернете все больше времени? Осознают ли они сами эти изменения?

Прежде всего, следует остановиться на том, что мы подразумеваем под социальными медиа. Это не только социальные сети Facebook, Twitter, TikTok и «ВКонтакте». К социальным медиа также относят мессенджеры (Telegram, WhatsApp, Viber), «Википедию», YouTube и многое другое. Чтобы проверить, является ли ваш любимый онлайн-ресурс социальным медиа, ответьте себе на вопрос, можете ли вы как пользователь вступить с ним в диалог, опубликовать на нем свое мнение. Если ответ положительный, то это 100\% социальное медиа.

Значительная часть информационных сайтов мутирует в сторону социальных медиа, потому что владельцы заинтересованы в росте числа пользователей и охвата аудитории. Это выгодно для бизнеса. Но выгодно ли это пользователям? В разных странах все чаще задают вопрос, способствуют ли социальные медиа конформизму, и, если да, то как.

\textbf{Суть конформизма.}
Конформизм можно определить как акт согласования установок, убеждений и поведения с групповыми нормами.

Слово «конформизм» --- близкий родственник «конформности». \ex{Обыватели}{everyday people} воспринимают их как синонимы. \href{https://4brain.ru/blog/конформность/}{Конформность} --- это, скорее, человеческое свойство. А \href{https://4brain.ru/blog/конформизм/}{конформизм} --- действия и результат этих действий.
По мнению американского психолога, автора бестселлера «Психология влияния» \href{https://4brain.ru/blog/%D0%BF%D1%80%D0%B8%D0%BD%D1%86%D0%B8%D0%BF%D1%8B-%D0%B2%D0%BB%D0%B8%D1%8F%D0%BD%D0%B8%D1%8F-%D1%87%D0%B0%D0%BB%D0%B4%D0%B8%D0%BD%D0%B8/}{Роберта Чалдини}, конформность возникает, когда человек строит свое поведение, согласовывая его с реакцией других людей [Cialdini \& Goldstein, 2004].

Конформизм является мощным социальным явлением, поскольку люди часто приспосабливаются к поведению других, даже когда действия этих других людей противоречат их собственным убеждениям, как это полвека назад доказал уроженец Российской империи, автор знаменитых экспериментов по исследованию конформности Соломон Аш.

Конформность относится к тенденции людей \ex{соответствовать ожиданиям}{meet the expectations} и нормам социальной группы. Это может включать в себя принятие взглядов, убеждений и поведения группы, чтобы соответствовать другим и быть принятым другими. \ex{Соответствие}{conformity (agreement of situations or objects with an expected outcome)} может возникать из-за множества факторов, включая \href{https://4brain.ru/blog/%D1%81%D0%BF%D1%83%D1%82%D0%BD%D0%B8%D0%BA%D0%B8-%D0%BD%D0%B5%D1%83%D0%B4%D0%B0%D1%87/}{желание одобрения или принятия}, потребность в социальной гармонии или предполагаемое отсутствие личных знаний или уверенности в конкретном вопросе. Хотя соответствие может привести к положительным результатам, таким как расширение сотрудничества и социальная гармония, оно также может привести к негативным последствиям, таким как подавление индивидуальности и потеря критического мышления.

Конформность --- важная область изучения \href{https://4brain.ru/psy/socialnaja-psihologija.php}{социальной психологии}, поскольку она дает представление о том, как люди формируют свои убеждения и поведение в социальных группах.

Конформизм помогает формировать и поддерживать социальные нормы, а также помогает предотвращать действия, опасные для восприятия. На соответствие могут влиять различные факторы, такие как индивидуальный статус, влияние сверстников и групповое давление.

Существуют две ветви конформности: (i) информационное социальное влияние; (ii) нормативное социальное влияние.

Информационное социальное влияние --- это когда мы \ex{полагаемся}{rely (на кого)} на других как на источник информации для управления нашим поведением. Например, вы идете на футбольный матч и не знаете, где вход на стадион; вы можете оглядываться на других и использовать это как сигнал, чтобы следовать за толпой.

Нормативное социальное влияние предполагает, что мы идем вместе с массами, чтобы нравиться им, даже если это идет \ex{вразр\'{е}з}{(с чем-то) contrary to} с нашими собственными убеждениями.

Развитие поведения, направленного на принятие норм группы, и его последующее включение в нашу систему ценностей и моделей поведения играет двоякую роль: с одной стороны, оно способствует нашему чувству принадлежности к группе, а с другой – позволяет группе принять нас как своих членов.

\textbf{Сколько среди нас конформистов?}
После завершения обучения в средней школе многих не оставляет ощущение, что конформисты в ней были все, а «белая ворона», «странная зверушка», как говорят в Латинской Америке, --- только он один. Во взрослой жизни «вороне» зачастую приходится \ex{перекрашиваться}{repaint} или искать общество себе подобных, как «Гадкому утенку» Ханса Кристиана Андерсена.

Российский ученый Георгий Правоторов подтвердил это предположение. Он подсчитал, что в социуме, в группе конформисты преобладают над \href{https://4brain.ru/blog/altruism/}{альтруистами} и \href{https://4brain.ru/blog/egoizm-chto-eto-i-kak-s-nim-zhit/}{эгоистами} в пропорции 3:1:1. Это опровергает устоявшееся в обществе представление, что люди по своей природе в большинстве своем эгоисты, живущие по принципу «Люби себя, чихай на всех! И в жизни ждет тебя успех!» Более того, при разрушении группы структура оси неизбежно восстанавливается, а с нею и доминирование конформистов [Будовская Ю. В., 2011].

В 2013 году китайские ученые исследовали, как именно конформность представлена в современных социальных медиа наподобие Gowalla, Flickr и Weibo. Оказалось, что в Gowalla конформистов почти вдвое больше, чем нонкоформистов. На остальных платформах разница выражена слабее [Jie Tang, 2013].


\textbf{Как устроены социальные медиа?}
Интернет сегодня является крупнейшей социальной группой, к которой мы принадлежим. Это надмножество ассоциаций людей, связанных общими интересами. И все мы, абсолютно все мы хотим чувствовать себя частью одной из этих групп.

Социальное давление, оказываемое социальными медиа на наше понимание мира, на то, как мы идентифицируем себя, возросло за последние десятилетия в геометрической прогрессии. Сегодня наше чувство принадлежности к группе определяется влиянием, которое мы приобретаем в Интернете.

Социальные медиа, в свою очередь, влияют на нас.

Как? Прежде всего, платформы социальных сетей предназначены для поощрения пользователей к соблюдению их алгоритмов. Например, они отдают приоритет контенту, который популярен, привлекает наибольшее внимание и соответствует прошлому поведению пользователя. Это означает, что люди с большей вероятностью увидят сообщения и рекламу, которые соответствуют их существующим убеждениям и ценностям. В результате они могут больше укорениться в своем мнении, нежели столкнуться с различными точками зрения.

Кроме того, социальные сети предоставляют пользователям возможность выражать свое мнение и общаться с единомышленниками. Хотя это может расширять возможности и обеспечивать чувство общности, это также может привести к развитию группового мышления и принуждению к согласию. Пользователи могут чувствовать необходимость соответствовать мнению других, и им может быть отказано в выражении инакомыслия [B. Barmi, 2023].

Давление конформности сказывается на пользователе Интернета особенно сильно, если он:
(i) нуждается в одобрении или поддержке;
(ii) недостаточно верит в себя;
(iii) испытывает тревогу.

Российские психологи считают конформизм фактором риска социосетевой безопасности личности в сетевом коммуникативном пространстве. Конформисты часто оказываются жертвами мошенников и других экспертов по \href{https://4brain.ru/blog/социальная-инженерия/}{социальной инженерии}, которые выманивают деньги и другие ресурсы, используя ключевые человеческие слабости:
жадность,
страх,
любопытство,
\ex{жажду легкой наживы}{thirst for easy money}.

Чтобы противостоять этим угрозам, необходимо укреплять навыки сохранять \href{https://4brain.ru/blog/%D1%81%D0%B8%D1%81%D1%82%D0%B5%D0%BC%D0%B0-%D0%B6%D0%B8%D0%B7%D0%BD%D0%B5%D0%BD%D0%BD%D1%8B%D1%85-%D1%86%D0%B5%D0%BD%D0%BD%D0%BE%D1%81%D1%82%D0%B5%D0%B9/}{ценностные и смысловые ориентации}, адекватные психическое состояние и психологические свойства [«Новое слово в науке и практике», 2013]. Другими словами, \ex{оставаться собой}{remain yourself} в быстро меняющемся мире. Один из способов этого добиться --- развивать мышление и эмоциональный интеллект, распознавать \ex{когнитивные искажения}{cognitive biases} и манипуляции. Не умеете? Приходите на нашу программу \href{https://4brain.ru/lnd/?cb=cog}{«Когнитивистика»}.

Также следует помнить ключевые характеристики социальных медиа:
\begin{itemize}[noitemsep, label=--]
    \item \textbf{Установка на контент: текст, аудио- и видеоматериалы, комментарии.} Это противоречит ценностям человеческого общения, которое ориентировано не столько на то, что сказано, сколько на то, \href{https://4brain.ru/blog/neverbalnoe-obschenie-o-chem-govorit-yazyk-tela/}{как именно это было сказано}. «Интернет лишен сглаживающих конфликт интонаций, улыбок, взглядов, \ed{дружеских похлопываний}{дружеский похлопывание по плечу}{friendly pat on the shoulder} по плечу», --- напоминает российская писательница Елизавета Александрова-Зорина [Газета.ру, 2020].
    \item \textbf{Наличие «волшебной кнопки»}. Американский венчурный инвестор Роджер Макнейми, помогавший Марку Цукербергу создавать Facebook, считает появление в 2008 году кнопки «лайк» («мне нравится») событием, которое изменило поведение пользователей, вызвав зависимость. Как разъясняют\footnote{Объяснение --- это истолкование чего-либо своими словами. Разъяснение --- это то же самое, но более подробное, более детальное} психологи, чужой «лайк» удовлетворяет человеческую потребность в \ed{погл\'{а}живаниях}{погл\'{а}живание}{действие по значению гл. поглаживать (to stroke)}. Они бывают условные («за что-то») и безусловные («просто так, от души»). Социальные медиа с «лайками» навязывают установку, что поглаживание можно только заслужить: чем больше закатов сфотографируешь и выложишь, тем больше поглаживаний получишь. В чем тут подвох?\footnote{В чем тут подвох? --- What is the catch here?} А в том, что, хотя мы все любим ласку, жизнь состоит не только из нее. Иногда приходится изодрать до крови конечности, чтобы добраться до зарослей дикой малины. Комфорт социальных медиа позволяет пользователю получить свою «мисочку лайков», которая приносит удовольствие, сопоставимое от радости поедания чего-то сладкого, без принесения жертв. Вам знакомы зависимые от дикой малины? Вряд ли. Зато людей, бросающих все, чтобы проверить число «лайков», с каждым годом становится все больше. Как и любая зависимость, эта ситуация опасна для людей, которые не находят удовлетворения в семье, реальных отношениях и на рабочем месте. К 2021 году «лайк обесценился и стал причиной неврозов», считает российская журналистка Раксана Бабаева [РБК, 2021].
    \item \textbf{Стадный инстинкт.} Как установили ученые, после ознакомления с комментариями других лиц, критикующих статью с \href{https://4brain.ru/blog/kak-otlichit-fejk-ot-pravdy-instrukciya-na-vse-sluchai-zhizni/}{фейковыми новостями}, склонность писать собственные положительные комментарии и намерение делиться такими новостями ниже, чем после того, как люди получили комментарии в поддержку статьи с фейковыми новостями. Причем влияние «стада» сильнее, чем влияние организации, создавшей сайт или платформу, которую человек использует. Предупреждение об отказе от ответственности не меняет отношение людей и не снижает их склонность «шарить» понравившиеся им статьи, пусть и ложные [Computers in Human Behavior, 2019].
\end{itemize}

Использование алгоритмов и искусственного интеллекта для выявления и манипуляции желаниями пользователей, создание информационного вакуума из склонности большинства к конформности базируется на особенностях человеческого мышления. Каких?

\textbf{Как работает наш мозг?}
Наш мозг симпатизирует конформизму. Именно этим обусловлена пропорция Правоторова.
Почему люди стремятся быть как все? Ответов немногим меньше, чем людей на планете, т.е. 8 млрд. Вот основные:
(i)  Жить по правилам легче, чем без правил,
(ii) Копирование чужого поведения часто приносит социальную и материальную выгоду,
(iii) Конформизм часто способствует выживанию.

Значимость мнения окружающих и конформизм базируются на следующих когнитивных механизмах:
\begin{enumerate}
    \item \textit{Принцип \ed{приверженности}{приверженность}{adherence, devotion} и \ed{последовательности}{последовательность}{consistency, logicality; succession, sequence, series}.} Основан на идее о том, что у людей есть сильное желание вести себя так, чтобы это согласовывалось с их собственными предыдущими заявлениями или действиями. Когда мы наблюдаем, как другие ведут себя определенным образом, у нас возникает мотивация действовать аналогичным образом, чтобы сохранить постоянство в собственной жизни.
    \item \href{https://4brain.ru/blog/%D0%BA%D0%BE%D0%B3%D0%BD%D0%B8%D1%82%D0%B8%D0%B2%D0%BD%D1%8B%D0%B5-%D0%B8%D1%81%D0%BA%D0%B0%D0%B6%D0%B5%D0%BD%D0%B8%D1%8F/}{\textit{Предвзятость социального подтверждения}}. Люди склонны искать одобрения и подтверждения от других, чтобы подтвердить, что мы принимаем правильные решения. Когда мы наблюдаем, что другие совершают определенное действие, мы интерпретируем это как знак того, что это принятый и социально обоснованный выбор.
    \item \textit{Неприятие риска}. Оно также влияет на значимость мнения окружающих и относится к тому факту, что мы стремимся к безопасности и избегаем риска \ex{любой ценой}{at any cost}. Если мы видим, что другие предпринимают действия, не испытывая негативных последствий, мы с большей вероятностью будем чувствовать себя в безопасности, следуя их примеру.
    \item \href{https://4brain.ru/blog/%D0%BF%D1%80%D0%B8%D0%BD%D1%86%D0%B8%D0%BF%D1%8B-%D0%BD%D0%B8%D0%BA%D0%BA%D0%BE%D0%BB%D0%BE-%D0%BC%D0%B0%D0%BA%D0%B8%D0%B0%D0%B2%D0%B5%D0%BB%D0%BB%D0%B8/}{\textit{Предвзятость авторитета}}. Это часть нашей повседневной жизни. В человеческом обществе \ed{прослеживается явная тенденция}{прослеживается тенденция}{there is a tendency (it is possible to trace a tendency)} считать убеждения или мнения определенных людей \ed{действительными}{действительный}{(here) valid} из-за того, кто они есть, \ex{не подвергая сомнению}{without questioning}. Наблюдение за известными деятелями или людьми, имеющими определенную репутацию, совершающими определенные действия, \ex{придает нам уверенность}{gives us confidence}, безопасность и облегчает принятие решений [Woko, 2023].
\end{enumerate}

Устойчивые когнитивные механизмы сформировались под влиянием различных ситуаций извне и внутренних \ed{побудительных факторов}{побудительные факторы}{motivating factors}, самый зн\'{а}чимый из которых для человеческого организма --- гормональный.

\textbf{Дофамин.}
Почему мы что-то делаем? Обычно, потому что это приносит нам удовольствие. Одним из ключевых факторов внутреннего \ed{подкрепления}{подкрепление}{reinforcement} является нейромедиатор дофамин. Ученые настаивают, что называть его «гормоном удовольствия», как это делают СМИ, ошибочно. Скорее, это гормон оценки и мотивации.

Однако сооснователь Napster, Plaxo и Facebook Шон Паркер, скорее всего, об этом не знал, поэтому в 2017 году говорил о дофамине в ошибочном, но понятном обществу ключе: «Нам нужно было время от времени давать вам небольшую дозу дофамина, потому что кто-то лайкнул или прокомментировал вашу фотографию, пост или что-то еще… Это цикл обратной связи социальной проверки … Вы эксплуатируете уязвимость в человеческой психологии … [Изобретатели] понимали это, действовали сознательно, и мы все равно это сделали» [Axios, 2017].

Другими словами, Паркер признал, что создатели ключевых социальных медиа использовали старинный метод «\ed{кнута и пряника}{кнут и пряник}{carrot and stick}», чтобы повлиять на поведение пользователей и привлечь их к своему продукту. Если «лайк» и дофамин --- пряники, то что тогда кнут? Человеческий страх совершить \ex{поведенческую}{поведенческий}{behavioural} ошибку, который мы все чувствуем, когда размышляем, что было бы неплохо пойти \ex{наперекор}{contrary to} групповому конформизму.

\textbf{Самооценка.}
Сильнее всего конформизм человека в социальных медиа влияет на образ тела и самооценку. Постоянное воздействие изображений, \ex{казалось бы}{seemingly; as if}, идеальных тел и образа жизни может привести к чувству неадекватности и принуждению соответствовать этим нереалистичным \href{https://4brain.ru/blog/%D0%BB%D1%83%D0%BA%D0%B8%D0%B7%D0%BC-%D0%B8%D0%BB%D0%B8-face-fascism/}{стандартам красоты}.

В этом смысле социальные сети могут поощрять конформизм во внешности и поведении, что ведет к потерям, в то время как социальные сети могут объединять людей и способствовать обмену идеями, они также могут способствовать развитию конформности.

Ситуация усугубляется, конформность \ed{нарастает}{нарастать/нарасти}{(i) вырастать на поверхности чего-либо; (ii) появляться, произрастать в каком-либо количестве; (iii) копиться, накапливаться в каком-либо количестве; (iv) увеличиваться в размерах, объёме, силе и т. п.} у:
(i) лиц с \ed{заниженной самооценкой}{заниженная самооценка}{low self-esteem};
(ii) лиц, не имеющих союзника;
(iii) в группах из трех и более человек;
(iv) в группах, где присутствуют эксперты или значимые для человека лица;
(v) рядом с коллегами, соотечественниками или другими людьми, принадлежащими к одной социальной среде [«Новое слово в науке и практике», 2013].

В результате \ed{странствия}{странствие}{wandering} по Интернету приводят человека к тому же выводу, что и поражения в личной жизни: осознанию, что неплохо бы \ex{раздобыть}{\textit{разг.} добыть, достать, обычно с трудом} здоровую самооценку, а не вот это все. Где? Здесь --- в нашей статье «\href{https://4brain.ru/blog/10-shagov-k-zdorovoj-samoocenke/}{10 шагов к здоровой самооценке}».

\textbf{Любовь к цифрам.}
Конформизм в социальных медиа напрямую связан с таким психологическим феноменом, как \href{https://4brain.ru/blog/%D0%BF%D1%81%D0%B8%D1%85%D0%BE%D0%BB%D0%BE%D0%B3%D0%B8%D1%8F-%D0%B2%D0%B8%D1%80%D1%83%D1%81%D0%BD%D0%BE%D0%B9-%D1%80%D0%B5%D0%BA%D0%BB%D0%B0%D0%BC%D1%8B/}{социальное доказательство}, при котором люди склонны копировать и следовать\footnote{\textit{кому-чему.} \textit{книжное} руководствоваться чем-нибудь, поступать подобно кому-нибудь} действиям и поведению других людей при определенных обстоятельствах.

Согласно Чалдини, люди считают поведение правильным, когда видят, как это делают другие. Когда человек не знает, как реагировать на конкретную ситуацию, он будет наблюдать за другими и делать то, что они делают. Например, если есть люди, ожидающие в очереди у банкомата, вы встанете в очередь. Именно так знаменитость, \ex{прилюдно}{при людях; гласно, публично} \ed{распивающая}{распивать/расп\'{и}ть}{(i) \textit{разг.} пить, выпивать вместе с кем-либо, сообща; (ii) \textit{разг.} пить долго, не торопясь; проводить время в питье чего-либо} определенный безалкогольный напиток, имеет тенденцию заставлять других людей покупать тот же напиток, особенно своих фанатов.

Социальное доказательство тесно связано с давлением авторитетов и стремлением нашего мозга к \ed{конкретике}{конкр\'{е}тика}{\textit{разг.} конкретные факты, дела, поступки}. Этим обусловлено изобилие цифр и чисел в заголовках последнего десятилетия. Не верите? Проверьте себя, верите ли вы следующим статистическим данным:
\begin{itemize}[noitemsep, label=--]
    \item 57\% клиентов покупают только у компаний, имеющих рейтинг не менее 4 звезд.
    \item 92\% людей доверяют рекомендации коллеги.
    \item 70\% клиентов доверяют рекомендации незнакомого человека.
    \item 97\% клиентов говорят, что \ex{онлайн-обзоры}{online reviews} и \ex{отзывы}{reviews} влияют на их решение о покупке.
    \item Отзывы увеличивают коэффициент конверсии продающих страниц на 34\%.
    \item 40\% людей купили продукт, потому что увидели, как инфлюенсер использует его в социальной сети.
    \item 49\% людей доверяют рекомендациям влиятельных лиц в Twitter.
    \item 14\% людей говорят, что \ex{одобрение}{approval} знаменитостей влияет на их решение о покупке.
    \item 68\% покупателей с большей вероятностью совершат покупку в местном магазине, если у него хорошие отзывы [Trustmary, 2022].
\end{itemize}
Эти данные были собраны в разных государствах разными компаниями, от Nielsen до Brand Rated. Однако спросите себя: когда вы смотрите на столбец из тезисов и цифр, вы действительно хотите проверить первоисточник или подпадаете под очарование конкретики? \ex{То-то и оно}{возглас при подтверждении или подчеркивании сказанного; именно так, конечно, то-то и оно, то-то и есть, то-то вот и есть, то-то вот оно и есть.}.

\textbf{Загадки поведения.}
Век тому назад поведение людей и животных изучали бихевиористы. В 21 веке бихевиоризм уступил место прикладному анализу поведения. Именно такие аналитики исследуют, как меняется поведение людей в социальных медиа под влиянием «лайков», чужих комментариев, трендов, эмодзи, «сторис» и других прозрений создателей социальных медиа.

Это новый антропогенный процесс. Наука лишь приступает к его изучению. Вот лишь некоторые изменения в поведении, которые удалось задокументировать:
\begin{itemize}[noitemsep, label=--]
    \item \textit{Утрата эмоций.} Сфотографировать закат и час рассматривать, как облака меняют цвет и форму, --- это разные действия. Фотография --- это результат, который можно показать обществу. А эмоции никак не упакуешь, их можно только прожить. Даже те, кто владеет талантом писателя, понимают, что «мысль изреченная есть ложь\footnote{a spoken thought is a lie}». Русский поэт Федор Тютчев за век до появления Интернета и социальных медиа сформулировал это в стихотворении «Silentium!» За полвека Интернета люди «\ed{разучились}{разучиться}{утратить навыки, умение делать что-либо} испытывать эмоции без соцсетей», считает Александрова-Зорина: «Разучились наслаждаться моментом, сидя на скамейке в парке, гуляя по улицам старого города, глядя на произведение искусства или на играющего ребенка. Нам обязательно нужно \ex{заснять}{to photograph} этот момент, а потом поделиться с миром» [Газета.ру, 2019].
    \item \textit{Потеря умения \ed{ничегонеделания}{ничегонеделания}{\textit{разг.} праздное времяпрепровождение; безделье}.} Итальянцы считаются изобретателями концепции dolce far niente --- «сладкого ничегонеделания», расслабленного досуга без напряжения и потребления. В 21 веке люди все больше задумываются над тем, что ничего не делать --- это не только роскошь, доступная на Земле не всем и не всегда. Это еще и навык. В среде социальных медиа, которые функционируют 24/7 и производят постоянный поток контента, который человек видит в своей ленте, конформисты следуют за этим потоком. Это приводит к усталости, мешает расслабляться, фантазировать и мечтать. В потоке конформист делает то, что должен. Вне потока --- то, что хочет. Однако, пребывая в потоке, человек все больше забывает, чего именно он хочет, а потому и лишается мотивации что-то делать. Или просто ничего не делать, для чего также нужна мотивация.
    \item \textit{\ed{Подмена}{подмена}{substitution (secretly or unnoticed)} ничегонеделания напряженным бездельем} Социальные медиа подвели человечество к осознанию, что ничегонеделание и безделье --- это разные вещи. Они похожи, потому что в обоих случаях человек ничего не делает, не создает, никуда не идет. Однако ничегонеделание («сиеста»), как и забвение, дарит человеку отдых, силы для новых \ed{свершений}{свершение}{achievement} и место в мозге для новых истин. Безделье ближе к играм. В социальных медиа зачастую это и есть одно и то же. Эта форма досуга приносит радость и усталость, но обычно --- никакого материального результата, что опасно для выживания \ex{в приполярных широтах}{in subpolar latitudes}. Ситуация была определена как проблема за три тысячи лет до возникновения Интернета жившим на территории современной Болгарии европейцем по имени Эзоп: «Ты летом пела, так зимой пляши в \ed{стужу}{стужа}{сильный холод, мороз}» [Министерство культуры РФ, 2023].
    \item \textit{Развитие \ed{непримиримости}{непримиримость}{intransigence}.} «Даже выходя из Интернета в реальный мир, мы остаемся нетерпимы друг к другу, заимствуя манеру общения в соцсетях для разговоров в реальной жизни. И совершенно перестаем слышать своих оппонентов. Они ведь нам больше не собеседники, а болельщики чужой команды, и не о чем с ними говорить», --- зафиксировала Александрова-Зорина. Методом распространения непримиримости являются, в основном, те самые «лайки» и система репостов, считают ученые из Science Advances [How social learning amplifies moral outrage expression in online social networks, 2021].
\end{itemize}
И это лишь немногие изменения поведения, которые происходят в реальной жизни. Уверены, каждому из вас встречалось что-то и пострашнее.

\textbf{Нонконформизм.}
В детстве и юности многим кажется, что, если общество ошибается, нужно бр\'{о}сить ему вызов или покинуть такое общество, как поступил Александр Чацкий в «Горе от ума» Александра Грибоедова.

Однако, взрослея, люди часто с удивлением обнаруживают, что нонконформисты очаровательно конформны внутри своей социальной \ed{ячейки}{ячейка}{cell, nucleus}.

На примере того же школьного общества можно увидеть, что кто-то из учащихся более конформен, а кто-то менее. В социальных медиа эта разница заметна слабее, но также присутствует.

Что отличает конформистов от нонконформистов? Ученые именуют это NFU --- потребностью в уникальности.

«NFU --- универсальная человеческая черта, наряду с тенденцией соответствовать убеждениям и установкам других и социальным нормам», --- пишут авторы научной статьи, опубликованной в 2012 году в журнале NeuroImage.

Индивидуальный NFU обусловлен отличиями в структуре серого и белого вещества коры головного мозга [H. Takeuchi, 2012].

Эти отличия уже настолько хорошо известны ученым, что они научились \ex{глушить}{(here) to suppress} магнитной стимуляцией возникающие в мозгу нонконформиста сигналы ошибки и увеличивать уровень соглашательства химическим путем [НОЖ, 2021].

Является ли нонконформизм таким же результатом генетической мутации, как \ex{невосприимчивость}{insusceptibility} к длительному воздействию алкоголя у Оззи Осборна, еще \ex{предстоит выяснить}{remains to be seen} [New York Post, 2019]. Желающим проверить самих себя на соответствие нормам предлагаем книгу за авторством Билли Салливана, который собрал кейсы Осборна и других видных нонконформистов \ex{воедино}{together} [Pleased to Meet Me, Amazon, 2019].

Если \ex{подытожить}{summarize} вышесказанное, выходит, что продвижению конформизма в социальных медиа способствуют:
\begin{itemize}[noitemsep, label=--]
    \item алгоритмы, отдающие предпочтение популярному контенту;
    \item давление с целью соответствовать групповым нормам;
    \item влияние на самооценку.
\end{itemize}
Чтобы сохранить здоровье и оставаться самими собой, важно осознавать эти влияния и беречь свою индивидуальность и независимость мышления перед лицом такого давления. Один из способов --- развивать критическое мышление: учиться думать, верить интуиции и собственным органам, а не мнению окружающих и агентства ОБС («одна бабка сказала»). Можно учиться самим, а можно записаться на нашу программу «Критическое мышление». Выбор за вами.

В 21 веке уже невозможно отказаться от Интернета и социальных медиа. Но можно научиться ими пользоваться, как электричеством, и обучить этому своих детей. В современном обществе возрастают ценности информационной гигиены, критического мышления, покоя, тишины и здоровья. Можно следовать за трендами, а можно создавать их самим. Выбор, опять же, за вами.

Удачи вам в этом!

\clearpage

\section{Чайлдфри: а почему бы и нет?**}

\textit{Источник: \url{https://4brain.ru/blog/chajldfri-a-pochemu-by-i-net/}}

«Дети --- цветы жизни, но пусть они лучше растут в чужом \ed{цветнике}{цветник}{flower garden}». Знакомо? Подобные мысли посещают \explain{время от времени}{from time to time} любых родителей, потому что дети --- это постоянные \explain{хл\'{о}поты}{chores}, расходы и проблемы. Хлопоты, конечно, могут быть приятными, проблемы --- вполне решаемыми, однако от всего этого, так или иначе, устаёшь.

А отдохнув, шагаешь в новый день, ведя за руку своего капризного и невыспавшегося, но такого милого \ed{карапуза}{карапуз}{little fellow} в детский садик, радуясь первым школьным успехам своего первоклассника, готовясь к выпускному вместе со своим уже выросшим ребёнком, провожая сына в армию и выдавая дочку замуж. «А как иначе?» --- спросит кто-то. Да очень просто!

Наша сегодняшняя тема --- чайлдфри во всех его видах и \ed{проявлениях}{проявление}{manifestation}. И для начала --- небольшой исторический экскурс.

\textbf{Что такое чайлдфри: немного истории}

Термин «чайлдфри» появился по историческим меркам сравнительно недавно, однако точное его происхождение установить все равно не удалось. \ed{Предположительно}{предположительно}{presumably}, этот термин впервые \explain{вошёл в оборот}{entered into circulation} в 70-е годы 20 века в рамках деятельности Национальной организации для не-родителей США, которая \explain{н\'{ы}не}{now} уже не существует [Е. Селивирова, 2010].

Более широкую известность это понятие обрело в 90-е годы, а именно после того, как школьная учительница Лесли Лафэйетт создала интернет-сообщество под названием The Childfree Network, целью которого была борьба с разными видами дискриминации бездетных людей и семей [Е. Алексахина, 2011].

\ed{Уточним}{уточн\'{и}м}{let us clarify}, что 30 лет назад тема была не то чтобы под запретом, а, скажем так, вызывала непонимание в обществе, что приводило к различным эксцессам. Уточн\'{и}м также, что русское слово «чайлдфри» является прямым переводом английского childfree, буквально означающего «без детей» либо «свободный от детей».

Данным термином обозначают людей, сознательно выбравших путь отказа от родительства при том, что их физическое и финансовое состояние вполне позволяет иметь детей. Термин \explain{был внедрён}{was introduced} \explain{в противов\'{е}с}{in opposition} понятию «childless», что означает «бездетный» и обычно употребляется в контексте «не способный иметь детей».

Сегодня термин «childfree» является широко распространённым, \ed{повсеместным}{повсеместный}{ubiquitous} и \ed{общеупотребимым}{общеупотребимый}{commonly used}. Во всяком случае, большинство людей знает, что этот термин обозначает. Если сказать простыми словами, чайлдфри --- это когда м\'{о}гут, но не хот\'{я}т. Однако «внутри» понятия childfree существует множество разных градаций и \ed{оттенков}{оттенок}{shade; tone} смысла, для которых придуманы отдельные термины. Это стоит обсудить подр\'{о}бнее.

\textbf{Виды и типы чайлдфри}

Для родителей со ст\'{а}жем, «обвешанных» заботами о подрастающем поколении и живущих в непрерывном потоке мыслей, чем накормить, во что одеть, куда отправить учиться и на какие гроши жить дальше, после того как детей одели, обули, накормили и сдали деньги в школу на очередные «шторы», типы чайлдфри, возможно, не сл\'{и}шком интересны.

Однако если в вашей жизни существуют ещё какие-то заботы, кроме как о д\'{е}тях и о том, как дожить до зарплаты, разбираться в типах childfree желательно уже для того, чтобы не попадать в неловкие ситуации, пытаясь комментировать чей-то образ жизни и образ мыслей.

В статье «Чайлдфри без паники: социологический взгляд» разобраны основные тренды данного явления на реальных примерах из жизни [Е. Селивирова, 2010]. Мы слишком углубляться не будем, и ограничимся общей характеристикой основных типов.

Основные типы чайлдфри:
\begin{enumerate}
    \item Реджекторы, они же чайлдхейтеры --- люди, которые не любят детей и относятся с некоторой долей брезгливости к детям и самой идее беременности, грудного вскармливания, замены памперсов и прочим неизбежностям, связанным с появлением в семье ребенка.
    \item Откладыватели --- люди, которые откладывают идею завести детей «на потом», когда закончат вуз, найдут работу, решат материальные и жилищные проблемы, купят машину, посмотрят мир, поживут для себя и так до бесконечности.
    \item Аффексьонадо --- предпочитают свободу и отдают себе отчет, что дети свободу изрядно ограничивают. Поэтому делают сознательный выбор в пользу свободы, в том числе свободы от детей.
    \item Отказники --- длительно колеблются, взвешивая «за» и «против» появления детей в семье и чаще всего отказываются от планов по деторождению либо «пропустив» детородный возраст, когда здоровье позволяет завести детей, либо найдя еще 100500 причин, почему им это не нужно.
\end{enumerate}

Это основные типы childfree, но это еще не все. В современном мире существуют различные субкультуры, представители которых не объявляют об отказе от намерения завести детей, однако детей все равно не заводят. Это, если можно так выразиться, «вторичные» чайлдфри, когда отсутствие детей не цель или средство, а прямое следствие их образа жизни и/или исповедуемых ценностей.

Субкультуры и молодежные движения с высоким процентом чайлдфри:

\begin{enumerate}
    \item Кидалт --- дословно «взрослый ребенок». Термин произошел от английского kidult, где kid означает «ребенок» и adult означает «взрослый». Такие «взрослые дети» активно познают мир, имеют множество хобби и пока не готовы покупать игрушки кому-то еще, кроме как себе.
    \item Синглтон --- человек, который предпочитает жить один, потому что ему так удобнее. Один --- это значит, один, без жены, мужа и, соответственно, детей.
    \item Твиксер --- человек, «зависший» между двумя состояниями: статусом тинейджера и статусом взрослого. Как правило, живет с родителями, перебивается случайными заработками, поэтому семью содержать не на что, да и привести потенциального партнера некуда. Термин распространен в США.
    \item Фурита --- примерно то же, что твиксер, но в Японии. Чаще употребляется применительно к молодым людям, решившим не поступать в университет и не получать высшее образование, а потому мало зарабатывающим и не могущим позволить себе семью и детей.
    \item Хикикомори, они же хикки --- в переводе с японского это означает «пребывание в уединении» и предполагает высокую степень добровольной социальной изоляции (не путать с карантином и принудительной самоизоляцией). Уединенный образ жизни мало способствует новым романтическим отношениям, созданию семьи и появлению детей.
    \item Поколение сатори --- в переводе с японского, это поколение, свободное от материальных желаний, которое довольствуется малым и не стремится зарабатывать деньги. Разумеется, такой образ жизни мало пригоден для семьи и вступает в противоречие с законным требованием законного партнера заботиться о материальном достатке семьи и содержании детей.
    \item Поколение NEET --- молодые люди, которые не работают и не учатся, поэтому с высокой степенью вероятности вряд ли дозреют до серьезных взрослых отношений и готовности нести ответственность за семью. Термин появился в Великобритании, обрел определенную популярность в странах Латинской Америки и… правильно --- в Японии!
\end{enumerate}

Как видим, немало молодежных течений, ведущих к чайлдфри, локализуется в Японии. Это не значит, что японцы в меньшей степени любят детей или в меньшей степени стремятся работать, чем другие народы. Причина, скорее, в том, что японцы более педантичны и склонны к детализации, поэтому именно там зародилось множество терминов, описывающих разные оттенки чайлдфри. В прочих странах обычно довольствуются либо общим определением чайлдфри, либо же используют один из уже придуманных кем-то терминов.

На самом деле, можно найти множество других терминов, названий и обозначений для разных вариантов градаций чайлдфри. Не будем приводить их все, потому что все они являются лишь вариациями вышеописанных трендов. А вот откуда взялся общий тренд чайлдфри и субкультуры, сторонники которых, скорее всего, так и не познают радость материнства и отцовства? Давайте разбираться.

\textbf{Истоки и причины движения чайлдфри}

В этой статье мы уже упоминали о Национальной организации для не-родителей США, которая ныне уже не существует [Е. Селивирова, 2010]. Можно ли сегодня сказать, что чайлдфри --- это общественное движение? Многие психологи и социологи полагают, что как раз сегодня есть все основания говорить именно о движении чайлдфри [Д. Клэйн, 2019].

Во-первых, по причине все растущей популярности данной идеи. Во-вторых, по причине растущего общественного интереса к этому явлению. И, наконец, потому что сегодня в процентном отношении реально стало больше людей, не желающих иметь детей и принимающих меры к тому, чтобы дети не появились.

Ученые называют разные статистические данные, что обусловлено разной методикой подсчета. Так, по данным Левада-центра в России порядка 2\% людей не хотят иметь детей, и примерно 9\% ожидают, что детей у них и не будет по разным причинам [Левада-центр, 2019]. Что касается конкретных данных по чайлдфри в России, десяток лет тому назад их насчитали чуть более трех с половиной тысяч [Е. Алексахина, 2011].

В США, по ситуации на 2014 год, порядка 15\% женщин в возрасте 40-44 лет так и не завели ни одного ребенка [Pew Research Center, 2015]. Это общее число без разделения на childfree (когда могут, но не хотят) и childless (когда хотят, но не могут).

В Европе к 2010 году наблюдался рост числа людей, не имеющих детей, относительно данных за 90-е годы 20 столетия. Так, если в 90-е таковых было меньше 10\%, в 2010 году практически во всей Западной Европе показатель уверено превысил отметку в 12\%. Больше всего childfree + childless в Испании (21.6\%), Австрии (21.54\%), Англии (20\%), Финляндии (19.89\%) и Ирландии (19\%) [OECD, 2010].

Более свежие исследования указывают на рост количества чайлдфри в обществе, хотя исследователи по-прежнему сталкиваются с некоторыми трудностями сепарации и разделения childfree и childless в ходе исследовательских программ [J. Neal, 2021].

Связаны эти трудности преимущественно с тем, что общество по-прежнему с трудом понимает людей, добровольно отказавшихся от продолжения рода. А ввиду того, что феномен группового подкрепления, так или иначе, заставляет приспосабливаться к общественному мнению, многие чайлдфри не желают себя идентифицировать именно таким образом. Нередки случаи, когда чайлдфри выдают себя за бесплодных и утверждают, что никакое лечение и никакие репродуктивные технологии им не помогают.

Тем не менее, многие вполне готовы идентифицировать себя именно как чайлдфри, и даже могут внятно сформулировать, почему они выбрали такой жизненный путь. К слову, если воздержаться от путанных методик, а просто на условиях анонимности «в лоб» спросить, хочет ли человек детей, складывается вполне определенная картина. Так, в ходе одного из недавних опросов выяснилось, что порядка 27\% современных молодых людей не желают иметь детей [А. Салькова, 2021]. Во всяком случае, на момент проведения опроса.

Немало интересного в ходе научных исследований выявил наш российский социолог Илья Ломакин из Лаборатории сравнительных социальных исследований НИУ ВШЭ [О. Соболевская, 2020].

Основные причины чайлдфри:
\begin{enumerate}
    \item Нежелание брать на себя ответственность за детей и иметь лишние проблемы в жизни.
    \item Стремление к свободе в распоряжении временем, денежными средствами, и сознательный выбор в пользу свободы, в том числе свободы от детей.
    \item Представление о детях как преграде на пути к самодостаточности, самореализации, карьере и т.д.
    \item Негативное отношение, брезгливость или отвращение к маленьким детям. Обычно имеет физическую биологическую основу --- кто-то не любит мышей, жуков и пауков, а кто-то --- детей.
    \item Боязнь необратимых последствий, потому что в случае, если родительство не принесет удовлетворения или принесет разочарование, «отыграть назад» эту ситуацию уже не получится.
\end{enumerate}

Это основные причины, почему люди становятся чайлдфри. Гораздо реже исследователи указывают на такие причины, как финансовые проблемы или отсутствие официального супруга, постоянного партнера или кого-то, с кем можно было бы разделить радость материнства или отцовства.

Возможно, в этом что-то есть, потому что по-настоящему «заточенные» на материнство женщины даже в отсутствие мужа часто рожают «для себя». А возможными материальными трудностями с учетом нашего менталитета молодежь зачастую «не заморачивается».

Как говорится, «дал Господь зайку --- подаст и лужайку», так что многие молодые люди готовы переложить материальные хлопоты на плечи государства, родителей, муниципальной власти, различных благотворительных организаций. Не все, конечно, но многие.

Совсем небольшой процент объясняет свое нежелание заводить детей глобальными причинами: глобальным потеплением, глобальным изменением климата, необходимостью бороться с перенаселением планеты и т.д. Правда ли они так думают или это отличный лайфхак, чтобы защититься от обвинений в эгоизме, однозначно сказать сложно.

К слову, желающих обвинить, переубедить и перевоспитать представителей чайлдфри по сей день много. Достаточно заглянуть на любой форум чайлдфри в Интернете, чтобы увидеть, сколько там комментаторов, не имеющих никакого отношения к чайлдфри, но желающих доказать этим людям, что они круто неправы. Почему? Давайте обсудим и это.

\textbf{Причины неприятия чайлдфри}

При том, что количество чайлдфри неуклонно растет, и число приверженцев этой идеологии становится больше, их все равно заметно меньше половины населения. А значит, в количественном плане они проигрывают, и в возможности отстоять свою позицию тоже.

Кстати, некоторые ученые предлагают идентифицировать чайлдфри на основе этой идентичности, которую люди принимают и проговаривают. Часть ученых полагает, что чайлдфри предполагает «социально-политическую мобилизацию», публичное отстаивание своих прав и убеждений [О. Соболевская, 2020].

Как бы там ни было, в меньшинстве отстаивать свои взгляды всегда сложнее, чем примкнув к большинству. Выше мы уже упоминали, что некоторые childfree, не желая навлекать на себя гнев общественного мнения, предпочитают идентифицировать себя как childless, неспособных завести детей по медицинским основаниям.

Так или иначе, сегодня масштабы неприятия чайлдфри такие, что этому даже посвящают специальные исследования. Например, «Кто такие чайлдфри и как им живется в Украине, Азербайджане и Центральной Азии» [Т. Ярмощук, Н. Мусави, А. Сафарзода, 2021].

И даже в научном мире, который, казалось бы, должен быть образцом беспристрастности, термин «чайлдфри» употребляется в заведомо негативном контексте. Чего только стоят заголовки наподобие «Коммуникативные стратегии в текстах, репрезентирующих идеологию childfree: на грани экстремизма» [Ю. Антонова, 2013].

Это при том, что, как мы помним, в 140-миллионной России, по ситуации на 2011 год, насчитали аж 3500 чайлдфри [Е. Алексахина, 2011]! Это же как нужно бояться новых веяний и любых перемен, чтобы все не слишком понятное и привычное объявлять экстремизмом?!

Ситуация пока что меняется медленно, о чем свидетельствует интервью главы ВЦИОМа Валерия Федорова под говорящим само за себя названием «К чайлдфри россияне относятся негативно» [Д. Филиппова, 2017].

Демократический Запад традиционно более терпимо относится ко всему новому, а научный мир готов изучать новые тренды без налета предвзятости. Интерес к теме чайлдфри наблюдается едва ли не с момента появления данного феномена.

Так, уже в конце 70-х годов 20 века канадская исследовательница Джейн Виверс исследовала семьи, сознательно отказавшиеся от деторождения, на предмет их мотивов, и подытожила свои наблюдения в монографии Childless By Choice («Бездетный по выбору») [J. Veevers, 1980].

Отдельное исследование посвящено изменениям в обществе, приведшим к росту идей чайлдфри, и исследованию отношения общества к новому на тот момент явлению. Результаты подытожены в книге Continuity and Change in Marriage and Family («Преемственность и перемены в браке и семье») [J. Veevers, 1990].

О причинах чайлдфри как явления мы уже поговорили. Поэтому остановимся на причинах все еще господствующего неприятия данного явления.

Почему общество настороженно относится к чайлдфри:
\begin{enumerate}
    \item Консерватизм --- большинство людей тяжело принимает любые перемены и изменения в социуме, в принципе.
    \item Традиции --- многолетняя пропаганда семейных ценностей принесла определенные плоды, и для многих дети являются самостоятельной ценностью независимо от того, насколько взрослые способны обеспечить детей, сделать их счастливыми, уделить им должное внимание.
    \item Конформизм --- это стремление наличествует на уровне инстинкта, и люди присоединяются к большинству даже тогда, когда это не сулит никаких выгод, а пребывание в меньшинстве не грозит никакой опасностью. Например, в лабораторных условиях, в которых проходили знаменитые эксперименты Аша по исследованию конформности.
    \item Природная агрессивность --- большинство людей агрессивно воспринимает все, что выходит за рамки их восприятия, известных и знакомых им явлений, в том числе людей, способных мыслить и действовать нестандартно.
    \item Теория «стакана воды» --- в социуме превалирует убеждение, что забота о пожилых людях --- это обязанность их детей, поэтому дети являются необходимым элементом нормальной семьи и успешной старости.
    \item Подспудная боязнь нехватки ресурсов --- заботу об одиноких стариках берет на себя государство, поэтому у остальных граждан есть заведомо негативное отношение к «халявщикам» и боязнь того, что лишняя нагрузка на систему социального обеспечения ударит по карману представителей традиционных семейных ценностей, всю жизнь платящих налоги и при этом решающих все свои проблемы самостоятельно.
    \item Зависть --- уже доказано, что большинство чайлдфри живут ничуть не менее, а зачастую и более счастливо, чем семьи с детьми. Получается, что к счастью можно прийти с меньшими хлопотами и энергозатратами, но не все это вовремя поняли.
\end{enumerate}

Да, на сегодняшний день есть немало подтвержденных случаев, когда люди, сознательно отказавшиеся от родительства много лет тому назад, полностью довольны жизнью и никак не сожалеют о своем выборе. Так, психологи Мичиганского государственного университета опросили порядка тысячи взрослых людей с целью выявить корреляцию между наличием или отсутствием детей и степенью удовлетворенности жизнью [C. Brooks, 2021].

Для этого использовалась так называемая «Шкала удовлетворенности жизнью». Выяснилось, что значимой разницы в показателях между чайлдфри и сторонниками традиционных семейных ценностей не наблюдается. Более того, оказалось, что чайлдфри в целом более либеральны и толерантны по отношению к окружающим.

Так или иначе, новые веяния прокладывают себе путь, и чайлдфри становится все больше, в том числе на высшем политическом уровне. В демократическом обществе политикум репрезентует общество, его настроения и тренды, в том числе в части идеологии чайлдфри. Это ни хорошо, ни плохо --- это так.

Как водится, четко разделить childfree и childless не всегда представляется возможным. Однако самого факта бездетности известного политика вполне достаточно, чтобы чайлдфри истолковали это как потенциальную поддержку своей позиции.

Вот только некоторые из известных политиков, которые так и не завели своих детей по разным причинам:
\begin{enumerate}
    \item Эммануэль Макрон, действующий президент Франции.
    \item Марк Рютте, глава правительства Нидерландов.
    \item Ангела Меркель, бывший канцлер Германии.
    \item Тереза Мэй, бывший премьер-министр Великобритании.
    \item Стефан Левен, бывший премьер-министр Швеции.
    \item Паоло Джентилони, бывший глава правительства Италии.
\end{enumerate}

Большинство «бывших» отошли от дел в силу преклонного возраста и прочих обстоятельств, однако достаточно длительное их пребывание на высших государственных должностях однозначно способствовало более толерантному принятию обществом идей чайлдфри.

Тема чайлдфри суверенно занимает свое место в литературе и искусстве. В 2020 году увидела свет книга, которую написала писательница Тала Тоцкая «Чайлдфри» (читать онлайн по ссылке) [Т. Тоцка, 2020].

Еще раньше современные музыкальные исполнители Noize MC и Монеточка порадовали поклонников клипом на свою песню «Чайлдфри»: \url{https://youtu.be/_l0LVFRuHMk}

Возможно, общими усилиями со временем наше общество станет более терпимым и поймет, что частная жизнь --- это личное право и личный выбор каждого. И это касается не только темы рождения детей.

\newpage
\section{Массовое сознание и манипуляция им*}

\textit{Источник: \url{https://4brain.ru/blog/mass-consciousness/}}

Каким бы «массовым» ни было сознание человечества в прошлые столетия, 21 век переплюнул все вместе взятые. Прочно укрепились такие понятия и явления как «продукт массового потребления», «средства массовой информации» и многие другие. Главной задачей обычного человека теперь является если и не попытка разобраться, куда катится этот мир, то хотя бы не поддаваться манипуляциям, которые с каждым годом становятся все изощреннее.

Массовое сознание в широком смысле – сознание больших масс людей, народа. В более узком – особая форма обыденного сознания, которая проявляется под влиянием определенных средств, прежде всего СМИ.

Характеризуется массовое сознание подвижностью, разорванностью, эмоциональностью, противоречивостью, шаблонностью и определенным окостенением (стереотипами). При этом многое зависит от развития науки и средств массовой коммуникации. Можно сказать, что состояние массового сознания определяет также эпоха.

В массовом сознании человека всегда присутствует, по крайней мере, два основных слоя:

\begin{enumerate}
    \item Обыденное сознание: все то, что связано с повседневными потребностями.
    \item Практическое сознание: включает в себя весь жизненный опыт человека.
\end{enumerate}

Также в массовом сознании присутствуют противоречивые элементы, которые делают его подвижным и изменчивым, а значит – готовым к воздействию и манипуляциям. Изменчивость и подвижность характерны для некоторых элементов сознания, например, слухов и мнений.

Помимо этого, стоит различать общественное и политическое сознание:

\begin{enumerate}
    \item Общественное сознание – совокупность коллективных представлений, присущих определенной эпохе, отражающих само общество. Его противопоставляют индивидуальному сознанию. В этой связи общественное сознание обладает некоторыми системными свойствами, которые нельзя свести к свойству индивидуального сознания. Выделяют шесть форм общественного сознания: мораль, наука, искусство, идеология, религия и право.
    \item Политическое сознание – это система знаний, оценок, настроений и чувств, посредством которых происходит осознание политической сферы людьми, группами, нациями. Оно представляет собой систему теорий, идей, представлений, взглядов, верований и убеждений.
\end{enumerate}

Массовое сознание обычно формируется при помощи средств массовой информации. СМИ тиражирует модели поведения, восприятия окружающего мира, знания, образа жизни. Все это в какой-то степени можно было бы назвать нормальным процессом, если бы зачастую не превращалось в манипуляцию массовым сознанием, и именно это страшнее всего.

\textbf{Методы манипуляции массовым сознанием}

Прежде чем мы приступим к стандартным методам манипуляции, давайте посмотрим, какие методы считает наиболее могущественными лингвист и философ Ноам Хомский:

\textit{Отвлечение внимания.} Если в стране и обществе существует большое количество важных и серьезных проблем, информационное пространство насыщается малозначительными проблемами. Суть этого приема не только в том, чтобы фокусировать общество на ненужных вещах, но и в том, чтобы у человека попросту не оставалось времени на размышления.

\textit{Создавать проблемы, а затем предлагать способы их решения.} Этот метод еще называют приемом «Проблема-реакция-решение». Для этого нужно создать некую проблему, которая точно вызовет эмоциональную реакцию среди населения. Теперь сами граждане начинают требовать от правительства принять необходимые меры.

\textit{Отсрочка исполнения.} Используется для того, чтобы продавить непопулярное решение. Оно позиционируется как «необходимое и болезненное» и СМИ сообщают, что его нужно провести в будущем. Общество соглашается, потому что легче думать о светлом будущем, в котором решение так и не будет принято. Таким образом правительство дает своему населению время на то, чтобы свыкнуться с этой мыслью и не слишком радикально протестовать, когда решение или закон будут приняты.

\textit{Способ постепенного применения.} Опасаясь революции, правительство вводит непопулярные меры постепенно, день за днем, месяц за месяцем. Население привыкает к неудобному положению в виду того, что человек способен адаптироваться к любой ситуации.

\textit{Делать больший упор на эмоции, а не на размышления.} Этот классический прием применяется повсеместно: когда правительству нечего сказать по существу, оно, при помощи СМИ, начинает «давить» на эмоции. В особенности на страх. Чтобы не давать возможности народу размышлять на экономические, социальные и политические темы, по всем каналам связи насаживается массовая истерия.

\textit{Знать о людях больше, чем они сами о себе знают.} С каждым годом «система» получает все больше знаний о человеческой психологии при помощи науки. Это позволяет манипулировать людьми намного эффективнее, причем даже в тех случаях, когда они догадываются об этом.

\textit{Усиливать чувство собственной вины.} СМИ налагают на общество или отдельные классы вину за возникновение каких-либо событий. Отныне эти люди вместо того, чтобы бороться с нечестивым правительством, занимаются самоуничижением и рефлексией.

\textit{Культивировать посредственность.} Благодаря самым разным шоу, в которых показываются пошлые, глупые и невоспитанные люди, обществу внушается мысль о том, что это приводит к успеху. Помимо этого, как мы уже говорили, существуют классические методы манипуляции массовым сознанием. Им уже несколько десятков лет, и они уже были опробованы в разных странах.

\textit{Внушение.} Этот метод состоит из нескольких элементов: ввести публику в суггестивное состояние, войти к ней в доверие, стать своим, убедить ее в своей компетентности. Суть внушения состоит в том, что общество не должно о чем-то думать, потому что правительство думает за людей и знает, как лучше.

\textit{Замалчивание фактов.} Из десятков фактов выбирается один-два, которые доказывают манипулистическую точку зрения. Большинство людей хоть и понимает, что были выбраны нужные факты (а не все), но все-таки верит выводам СМИ. Проблема в том, что замалчивание фактов – это разновидность лжи. Возможно, это даже хуже, чем ложь.

\textit{Образ врага.} Вместо того чтобы говорить о внутренних проблемах, СМИ нагнетают атмосферу при помощи появления врага, который угрожает стране и мешает ей развиваться политически, экономически и социально.

\textit{Многократные повторы или «Метод Геббельса».} Многие социологи и психологи считают, что манипуляция тогда будет успешной, когда ее никто не замечает. Но, как показывает практика, это необязательно. «Ложь, повторенная тысячекратно, становится правдой». Удивительно, как быстро то, что сегодня считается жульничеством и обманом, завтра становится истиной.

\textit{Метод «страшилок».} Обществу внушают, что есть только два варианта развития событий: ужасное и плохое. Через какое-то время это «плохое» уже воспринимается как «хорошее» и единственно верное.


\textbf{Фрагментация}

Репортаж, статья или сюжет разбивается на несколько фрагментов, которые не связаны между собой. Зритель или читатель не выносит из этой информации никакого смысла, зато получает заряд мощных негативных эмоций.

Следует сказать, что порой очень сложно выявить манипуляцию, потому что нас каждый день бомбардируют огромным количеством информации. И даже если вы отключите новости, у вас останутся мнения и убеждения – но как узнать, не были ли они сформированы раньше под влиянием манипуляций? Это сложный вопрос, однако несомненно то, что нужно развивать в себе критическое мышление, чтобы воздействие лживой информации на свой мозг и психику свести к минимуму.

Есть некоторые универсальные способы противостояния манипуляциям, которые могут помочь составить более-менее объективную оценку ситуации:
\begin{enumerate}
    \item получить информацию из первых рук;
    \item проанализировать противоречащие друг другу источники (то есть по обе стороны), выявить общие и разнящиеся мнения;
    \item проанализировать отношения различных групп в социальных сетях;
    \item выявить манипуляторов при помощи анализа продвигаемой ими точки зрения.
\end{enumerate}

Также можно попробовать выявить основные признаки манипуляции сознанием:
\begin{enumerate}
    \item СМИ начинают относиться к людям не как к личностям, а как к объектам или вещам, которые лишены свободы выбора.
    \item Каждый раз, когда вы чувствуете, что мишенью СМИ являются ваша психика, ценности, убеждения, представления и установки – это тревожный звоночек, что вы имеете дело с манипуляцией сознания.
    \item Если вы читаете или смотрите новости, цель которых – создавать ощущения страха и неуютности, значит, «это кому-нибудь нужно».
\end{enumerate}

Воспитывайте в себе скептика, когда имеете дело с массовой информацией. Это не значит, что вы должны проверять каждую фразу, которую читаете, но хотя бы держите в голове возможность того, что вами могут манипулировать. Прислушивайтесь к интуиции – как правило, она первой замечает, что кто-то пытается вторгнуться в вашу психику.

Желаем вам удачи!

\clearpage


% \chapter{Соцсети}

\section{Мы просто зомби}

\textit{Миллионы людей во всем мире страдают от одиночества и тревоги из-за соцсетей. Как с этим бороться?}

\textit{Источник: \url{https://lenta.ru/articles/2021/11/01/algorithm_4/}}

Как соцсети влияют на молодежь? Какими вырастут зумеры, которые всегда смотрят в экран смартфона? Старшее поколение отвечает на этот вопрос однозначно: интернет делает подростков только хуже. Но сами молодые люди до сих пор не понимают, как именно на них влияют соцсети: 31 процент считает, что положительно, а почти половина не может дать однозначного ответа. Их называют цифровыми аборигенами — людьми, которые не помнят или даже не знают, каким был мир до интернета и соцсетей. Вместе с этим они уже столкнулись с эпидемией психических расстройств и самоубийств, стали меньше заниматься сексом и жениться. Как алгоритмы соцсетей и короткие видео меняют растущее поколение — в материале масштабного спецпроекта «Ленты.ру» «Алгоритм. Кто тобой управляет?»

\textbf{Однорукий смартфон}

У социальных сетей одна цель --- сделать все, чтобы пользователь проводил в них как можно больше времени. Принцип их работы сравним с игровыми автоматами: свайп ленты новостей снизу вверх уж очень похож на дерганье ручки «однорукого бандита». И механика абсолютно та же: люди раз за разом делают одно и то же движение в надежде сорвать куш, увидеть что-то экстраординарное. Но если в казино играют на деньги, в социальных сетях на кону — сам человек, его время и внимание.

Непомерное желание увидеть в ленте новостей нечто удивительное и томительное ожидание лайков часто перерастают в болезненную привязанность. И в этом виноваты не сами пользователи, а соцсети, которые используют любые уязвимости в человеческой психике. Нейробиолог и психиатр Жадсон Брюер в своей книге «Зависимый мозг» прямо говорит о том, что соцсети, как и сигареты, вызывают дофаминовое блаженство.

\begin{fancyquotes}
    Большинство людей, опубликовав фото, обязательно зайдут в соцсеть проверить лайки и комментарии.
\end{fancyquotes}

Это раздражает близких и мешает общению, но мы продолжаем это делать, чувствуя бесконечную вину, — а порой даже тайком, чтобы не вызвать осуждения. Желание посмотреть на экран смартфона появляется вновь и вновь.

В основе этого действия лежит выработанная миллионами лет эволюции простая схема: триггер — поведение — вознаграждение. Когда нас хвалят, мы радуемся, когда нет — расстраиваемся. Для закрепления «хорошего» поведения организм использует нейромедиатор дофамин, который отвечает за обучение и мотивацию. Так люди годами учатся вести себя условно «правильно» и избегать негативных последствий. И чем более явно наше «вознаграждение», тем лучше. В старейшую ловушку эволюции попадают и пользователи соцсетей: желание выложить что-то в социальную сеть (триггер) — публикация поста (поведение) — лайки и реакции (вознаграждение). Поведение прочно закрепляется, и из раза в раз зависимость только растет. Именно так люди реализуют желание быть в обществе и стремление к чему-то большому.

\begin{center}
    \includegraphics[width=0.85\textwidth]{img/kak-formiruetsa-pivichka.png}
\end{center}

По мнению Брюера, привычка завязывать самому шнурки схожа с привычкой писать сообщения за рулем автомобиля, — с той лишь оговоркой, что первая безобидна, а вторая опасна для жизни. Между ними на разных позициях по шкале «опасности» можно расположить привычки жевать жвачку, погружаться в себя и многие другие. Место каждой привычки на шкале чаще всего связывают с уровнем стресса и тем, как люди с ним справляются.

\begin{fancyquotes}
    Социальные сети вырабатывают в пользователях отрицательное подкрепление: не только радуют лайками и одобрением, но и позволяют сбежать от реальности, отвлечься от печалей. И чем чаще туда убегать, тем больше туда тянет: поведение становится автоматическим, формируется аддикция.
\end{fancyquotes}

Что же приводит к зависимости людей от алгоритмов социальных сетей? Согласно исследованию Брюера, существует несколько предпосылок: несинхронная коммуникация, отсутствие барьеров и возможность сплетничать в комфортной среде. Усиливает все это ощущение неопределенности — то есть неизвестно, получит ли человек одобрение, лайкнут ли его пост и прокомментируют ли. Это называется периодическим подкреплением, и именно такую стратегию используют казино и игровые аппараты. Выигрыш случаен, но при этом достаточен для мотивации. И пользователь уже на крючке. Если он вдруг попробует сорваться и не будет открывать соцсеть в течение хотя бы пары часов, приложение обязательно пришлет ему уведомление.

\textbf{Обманчивая близость}

Создатели всех популярных соцсетей прекрасно понимают, что ради заработка используют уязвимости человеческой психики. Но все равно продолжают это делать. В этом, к примеру, признавался бывший президент Facebook Шон Паркер.

Люди пользуются соцсетями, потому что ищут одобрения. Согласно исследованию 2015 года, именно желание быть положительным героем в глазах других людей приводит пользователей в ловушку соцсетей. Каждая публикация — это крик о помощи, знак того, что автору некомфортно, одиноко или попросту скучно. Лайки и комментарии утоляют жажду одобрения, но она появляется снова и снова. «Вы проверяете телефон по утрам, прежде чем пописать, или пока писаете. У вас всего два варианта», — говорит в документальном фильме «Социальная дилемма» Роджер МакНеми, один из ранних инвесторов в Facebook.

\begin{fancyquotes}
    В итоге обилие общения в сети приводит к тому, что человек теряет способность справляться с проблемами и попадает в изоляционный капкан. Так лайки и сообщения не улучшают настроение, а лишь создают его иллюзию. Более того, картинка чужой счастливой жизни не только не радует, но и подавляет других.
\end{fancyquotes}

В 2014 году Facebook решил изучить степень своего влияния на психологическое состояние пользователей. В эксперименте приняли участие 689 тысяч пользователей соцсети. Правда, сами они понятия не имели, что над ними ставят опыты.

Эксперимент состоял в следующем: для части пользователей алгоритм отбирал публикации таким образом, чтобы они были преимущественно позитивными, для другой части в ленту новостей добавили побольше негатива. Исследователи в течение недели наблюдали за тем, как меняются реакции подопытных и что они сами публикуют.

Изменения в поведении людей были поразительными. Оказалось, что человек, встречающий огромное количество негативных публикаций, сам начинает постить такой же контент. Но если резко увеличить в ленте позитив, настроение пользователя улучшается. Манипулировать состоянием людей оказалось слишком просто: достаточно было лишь немного направить алгоритм в нужную сторону.

«Эмоциональные состояния могут передаваться другим через эмоциональное заражение, заставляя людей неосознанно испытывать те же эмоции. Эмоциональное заражение происходит без непосредственного взаимодействия между людьми (достаточно контакта с другом, выражающим эмоцию) и при полном отсутствии невербальных сигналов», — говорилось в исследовании.

Сам факт того, что Facebook тайно, без спроса экспериментировал на живых людях, многих поверг в шок. Оказалось, что соцсеть не только в силах влиять на настроения людей, но и не спрашивает на это разрешения. На самом деле Facebook вправе творить с умами людей что угодно: при регистрации любой пользователь принимает политику использования данных, в которой есть и пункты об экспериментах. Соцсеть это даже не скрывает и любые действия оправдывает «улучшением сервиса».

\textbf{Сила одного}

Но изменение настроения миллионов людей — не единственный видимый эффект от использования соцсетей. Как оказалось, те напрямую влияют на психическое здоровье пользователей. В 2014 году группа исследователей из университетов Пало-Альто и Хьюстона во главе со специалистом из университета Дюкейна Май-Ли Стирс изучила связь активного использования Facebook с симптомами депрессии.

Специалисты выяснили, что пользователи Facebook чувствовали подавленность, сравнивая себя с другими людьми. Сперва ученым удалось выявить связь между временем, проведенным в соцсети, и появившимися у опрошенных депрессивными симптомами. Оказалось, что чем больше времени человек проводит в телефоне, тем хуже он себя чувствует.

\begin{fancyquotes}
    Это не значит, что Facebook вызывает депрессию, но депрессивные настроения и трата времени на сравнение себя с другими в Facebook, как правило, идут рука об руку

    \begin{flushright}
        говорится в исследовании
    \end{flushright}
\end{fancyquotes}

Ситуация \ex{усугубляется}{is aggravated} тем, что в интернете люди не рассказывают о повседневной жизни и рутинных делах. Они делятся радостью, достижениями, красивыми видами. Все это дает понять, что наша жизнь по сравнению с чужой серая и неинтересная.

Люди, у которых и без соцсетей могут быть проблемы, при просмотре ленты часто чувствуют себя одинокими и ненужными. Это лишь подталкивает их к изоляции.

«Исследования показывают, что постоянно сравнивать себя с другими вредно для психики. Слишком частое сравнение явно приводит к ухудшению эмоционального состояния», — объясняла один из авторов исследования.

\begin{center}
    \includegraphics[width=0.75\textwidth]{img/google-facebook-experiments.png}
\end{center}

Однако соцсетям совсем не нужна стабильная психика пользователей: им важно, чтобы те оставались онлайн и приносили прибыль. «Такие компании, как Google и Facebook, постоянно проводят на пользователях эксперименты. Благодаря им они точно знают, как заставить людей поступать так, как они хотят. Это банальная манипуляция», — говорит бывший менеджер Facebook Сэнди Паракилас. По его словам, аудитория современных соцсетей — это подопытные животные. И опыты над ними проводятся не для того, чтобы разработать лекарство от рака или помочь голодающим детям. «Мы просто зомби. Они хотят, чтобы мы смотрели больше рекламы. Так они заработают больше денег», — заключает он.

\textbf{В FOMO верующие}

В 2018 году команда ученых провела еще один эксперимент над 143 студентами Университета Пенсильвании, который доказал прямую причинно-следственную связь между использованием соцсетей и ощущением тревоги и одиночества. Испытуемых разделили на две группы, предварительно замерив их самочувствие и ментальное состояние по семи критериям. Первой группе разрешили пользоваться соцсетями без ограничений, вторые же сидели в Instagram, Facebook и Snapchat лишь 10 минут в день. Спустя три недели у первой группы ничего не изменилось, а вот члены второй группы почувствовали значительные улучшения в настроении.

Ученые уверены: помимо постоянного сравнения себя с другими использование соцсетей вызывает FOMO (fear of missing out), или синдром упущенной выгоды. Страх пропустить что-то важное и ощущение, что жизнь пройдет мимо, стали базовыми для большинства пользователей. Результаты исследования говорят, что без последствий для психики в соцсетях можно проводить не больше 30 минут в день. Но далеко не каждый пользователь способен выдержать это ограничение.

\begin{center}
    \includegraphics[width=0.75\textwidth]{img/30-minuty-v-den.png}
\end{center}

Оксфордский словарь определяет FOMO как «беспокойство о том, что потрясающее или интересное событие может в настоящее время происходить в другом месте; часто вызвано постами, увиденными в социальных сетях». Специалисты утверждают, что этот страх часто идет об руку с непреодолимым, почти маниакальным желанием оставаться в курсе всего, что происходит с знакомыми. Первое масштабное исследование FOMO было опубликовано еще в 2013 году, но тогда специалисты посчитали, что одержимость соцсетями — не причина, а следствие одиночества и замкнутости. Более поздние исследования показывают, что все ровно наоборот.

Еще восемь лет назад больше половины людей в опросе признавались, что боятся пропустить в соцсетях что-то важное, и поэтому периодически их проверяют.

\begin{fancyquotes}
    Каждый третий был готов бросить курить, но только не удалиться из соцсетей. Парадоксально, но многие хотели бы устроить себе «каникулы» или «детокс» от соцсетей, но не рискуют этого делать.
\end{fancyquotes}

Недавние исследования говорят о том, что синдром упущенной выгоды испытывают люди всех возрастов, а не только подростки. Исследователи из года в год настоятельно советуют ограничить использование соцсетей, а в случае с серьезными улучшениями эмоционального состояния — вообще удалить аккаунты. Но, судя по тому, что число пользователей и время пребывания в соцсетях постоянно растут, к этим советам никто не прислушивается.

\textbf{Нездоровое отношение}

Эволюция способствовала тому, что человека волнует мнение о нем ближайшего окружения, его семьи и близких людей. Такое «общественное» мнение — это выработанный тысячелетиями рычаг давления на человека, который позволяет привить нормы поведения и морали. Но соцсети довели эту установку до абсурда: люди физически не готовы знать, что думают о них тысячи и миллионы подписчиков и случайных незнакомцев. Пользователи не могут принимать общественное одобрение или порицание круглые сутки, этого не способна выдерживать никакая психика.

По словам социального психолога Джонатана Хайдта, в наибольшей опасности оказалось поколение Z — дети, рожденные в 1996 году и позднее, которые попали в социальные сети в детстве и подростковом возрасте.

\begin{fancyquotes}
    Молодежь коротает все свободное время в смартфоне, а значит, целое поколение становится более тревожным, депрессивным и изолированным.
\end{fancyquotes}

Последнее подтверждает статистика: среди «зумеров» число тех, кто хотя бы раз ходил на свидание или имел романтические отношения, стремительно падает.

При этом среди американских подростков наблюдается гигантский скачок уровня депрессии и тревожности. Он начался в 2011-2013 годах — ровно тогда, когда социальные сети перенеслись в мобильные телефоны. Уровень селфхарма (умышленное причинение вреда своему телу), ранее стабильный на протяжении многих лет, резко вырос на 62 процента среди девочек-подростков и на 189 процентов среди девочек 10-12 лет. Тот же паттерн наблюдается и в статистике самоубийств: рост почти на 70 процентов среди подростков (15-19 лет) и почти в полтора раза у девочек младшего возраста — хотя исторически они никогда не входили в зону риска.


Не последнюю роль в этом сыграли фильтры для фото, которые есть практически в любом приложении. Самые разнообразные фильтры встречаются в Instagram — соцсети, которая состоит из фотографий. На заре создания приложения с помощью фильтров можно было добавить к фото рамку или эффект выцветшего снимка. Но сейчас их используют для совсем других целей: выпрямить нос, разгладить кожу или увеличить губы.

\begin{center}
    \includegraphics[width=0.75\textwidth]{img/samoubiistva-data.png}
\end{center}

Это крайне негативно влияет на психику пользователей, а особенно — подростков. Так, по данным исследования Бостонского медицинского центра (BMC) в Массачусетсе, сама возможность довести свою внешность до мнимого «совершенства» на картинке вызывает недовольство внешним видом в реальной жизни. А это, в свою очередь, породило гигантский скачок спроса на пластику. В 2013 году среди пациентов пластических клиник в США лишь 13 процентов хотели изменить внешность так, чтобы она была больше похожа на отредактированное селфи. В 2017 году со своей фотографией, обильно покрытой фильтрами, к хирургам приходили уже 55 процентов пациентов.

\begin{center}
    \includegraphics[width=0.75\textwidth]{img/social-networks-more-data.png}
\end{center}

В зоне повышенного риска оказываются молодые люди и подростки, которые и без того часто излишне критичны к собственной внешности. На фотографиях они пытаются подчеркивать скулы, выпрямляют и укорачивают носы, увеличивают глаза, и все это влияет на самооценку и может вызвать телесное дисморфическое расстройство (body dysmorphic disorder, BDD). Телесная дисморфофобия проявляется как чрезмерная озабоченность кажущимися внешними недостатками, и ее жертвы делают все возможное, чтобы их скрыть или исправить, даже с вредом для здоровья. Люди с таким расстройством часто делают не одну, а несколько пластических операций. Заболевание относят к обсессивно-компульсивному спектру. Ученые из Бостонского медицинского центра прямо говорят, что одна из причин BDD — селфи.

\begin{fancyquotes}
    Распространенность этих отфильтрованных изображений может заставить человека чувствовать себя неадекватно из-за того, что в реальном мире он выглядит по-другому, и может даже действовать как спусковой крючок и привести к расстройству (BDD)

    \begin{flushright}
        Ученые из Бостонского медицинского центра
    \end{flushright}
\end{fancyquotes}

Исследования показывают, что активнее всего свои селфи редактируют те подростки, которые недовольны своей внешностью. Более того, пользователи с уже диагностированным расстройством в поисках одобрения чаще других показывают глянцевую версию себя в соцсетях. Телесная дисморфофобия крайне распространена: ранее она поражала только одного из 50 человек, но число таких диагнозов растет с каждым годом. Алгоритмам же выгодно, чтобы пользователи активно пользовались фильтрами или зависали над фотографиями «идеальных», по их мнению, людей, поскольку этот процесс затягивает.

В 2018 году британский врач-косметолог Тиджион Эшо впервые ввел термин Snapchat-дисморфия (Snapchat dismorphia). Так специалист описал запрос клиентов на превращение в «цифровую» отфильтрованную версию себя с помощью хирургии. В основе этих желаний — та же дофаминовая игла, наркотическая зависимость от соцсетей, появляющаяся из-за необходимости одобрения. Гладкая кожа, белые зубы, правильные черты лица — такое «отфильтрованное» изображение наберет больше лайков и комментариев, чем обычное. «Опасность заключается в том, что это не просто ориентир, это становится образом, как пациент видит себя. Он хочет выглядеть точно так же, как этот образ», — говорит Эшо. Он обычно отказывает в любом вмешательстве пациентам, одержимым фильтрами. Однако таких хирургов, как он, мало: большинство берется за переделку лица, даже если это опасно и не принесет пациентам желаемого счастья.

\textbf{Сладкая ложь}

И если Facebook и Instagram — площадки, в которых в основном обитают так называемые миллениалы (люди, рожденные в период с 1981 по 1996 год), то более молодое поколение Z давно освоило принципиально новую площадку TikTok. Соцсеть была запущена в 2016 году, но взрывную популярность начала набирать в конце 2019-го. В январе 2021 года число активных пользователей приложения достигло 689 миллионов человек.


TikTok неслучайно называют соцсетью для молодежи: больше половины (62 процента) ее американских пользователей в возрасте от 10 до 29 лет. Основа соцсети — короткие ролики, которые, как правило, длятся в районе 15 секунд. И в этом ее главная ловушка: видео настолько затягивают, что пользователь TikTok проводит в нем в среднем 52 минуты. 90 процентов людей открывают приложение каждый день.


Феномен TikTok кроется в его алгоритмах. Здесь неважно, сколько у пользователя подписчиков, есть ли у него армия фанатов и сколько сил было потрачено на ролик. Важно только одно — виральность. Лента рекомендаций у каждого пользователя разная и подбирается в соответствии с его вкусами и интересами. Среди пользователей TikTok существует шуточная градация «уровней» соцсети: при первом запуске приложение показывает ролики «для обычных смертных» — тиктоки популярных блогеров, знаменитостей и контент с миллионами лайков. Алгоритм нужно тренировать: активно лайкать и отмечать то, что не понравилось. Со временем он пугающе точно подстраивается под конкретного человека. Существуют отдельные «уровни» для любителей вязания, абстрактных шуток или людей со средним размером одежды. Все они могут набирать миллионы лайков и просмотров и никогда не пересекаться.

\begin{fancyquotes}
    Такой точечный подход к вовлечению пользователей — золотая жила для любого интернет-сервиса.
\end{fancyquotes}

\begin{center}
    \includegraphics[width=0.75\textwidth]{img/tiktok.png}
\end{center}

Секрет рекомендаций еще в 2015 году разгадали в Netflix: 80 процентов зрителей сервиса охотно продолжали смотреть то, что предлагал изучивший их вкусы алгоритм. TikTok же эксплуатирует эти механизмы намного активнее и глубже. «Вы просто пребываете в приятном дофаминовом состоянии. Это происходит почти гипнотически, вы просто смотрите и смотрите в экран», — так объяснила процесс потребления контента в TikTok профессор Университета Южной Калифорнии доктор Джули Олбрайт.

Здесь принцип бесконечного и быстрого стимулирования легкодоступного «счастья» доведен до предела: просто так закрыть приложение и вернуться в реальную жизнь невозможно. «Пять минут в TikTok равны часу в реальной жизни», — горько шутят исследователи феномена соцсети. Доктор Олбрайт подтверждает этот тезис: концентрация внимания снижается, и из-за этого появляется эффект сжатия времени.

По ее словам, молодежь уже сейчас демонстрирует последствия таких деформаций. К примеру, как-то ее студентка рассказала ей, что твердо решила посвятить себя написанию песен. Но с одним условием: если за три месяца у нее не получится прославиться, то придется выбрать другую цель в жизни.

\begin{fancyquotes}
    Такой подход старшему поколению кажется абсурдным, ведь чтобы достичь такой цели, в их понимании, нужно потратить немало сил и времени.
\end{fancyquotes}

Знакомый профессор Олбрайт поделился с ней историей, которую оба сочли очень показательной. Преподаватель спросил у студентов, что они планируют делать ближайшие пять лет после выпуска. «Студенты смотрели на него как на умалишенного. Пятилетний план? О чем ты говоришь, это же целая вечность!» — вспоминает она.

Пандемия коронавируса, развернувшаяся в 2020 году, стала первым мощным историческим событием для зумеров, которое полностью перевернуло их жизнь и взгляд на будущее. «Пандемия определит это поколение. Это первое потрясение, которое они испытали на пути к взрослой жизни или в момент вступления в нее», — считает президент и ведущий исследователь Центра кинетики поколений Джейсон Дорси.

И зумеры действительно испытали на себе последствия мирового локдауна: более половины поколения Z в возрасте от 18 до 23 лет или кто-то из их семьи потеряли работу или столкнулись с сокращением зарплаты. У миллениалов этот показатель тоже высок — среди них 40 процентов почувствовали, как их будущее пошатнулось.

Потеря навыков общения, психологическая зависимость от смартфона, неадекватное восприятие себя, тревожность и депрессия — всем этим зумеров и миллениалов «одарили» алгоритмы соцсетей. Самое образованное поколение в истории человечества оказалось не готово к глобальным потрясениям. Новый мир в ближайшие годы не стабилизируется, и над будущим нависла серьезная угроза. Соцсети им явно не помогут, но зато с радостью заработают на их горе и страхах.

\clearpage

% \chapter{Транспорт}

\section{Дорожной концессии нужен сигнал*}

\textit{На какие проекты надо делать ставку при строительстве трасс с привлечением частного капитала}

% https://www.vedomosti.ru/industry/infrastructure_development/articles/2023/03/29/968707-dorozhnoi-kontsessii-nuzhen-signal
\textit{Источник: \url{https://www.vedomosti.ru/industry/infrastructure_development}}

\textit{Олеся Ошанина}

Концессионные соглашения в дорожном строительстве давно уже стали классикой, поскольку реализация столь капиталоемких проектов возможна лишь с государственной поддержкой. В нашей стране строительство дорог регионального значения является наиболее перспективной формой государственно-частного партнерства (ГЧП), однако отсутствие четких сигналов от федеральных властей о том, какие проекты являются приоритетными, является препятствием для активного строительства дорог с привлечением частного капитала.

В условиях современной экономики развитие партнерских отношений государства и частного сектора считается оптимальным способом повышения эффективности использования государственной собственности. Одной из важнейших форм ГЧП являются концессии. Их главным плюсом является возможность привлечения внебюджетных инвестиций и ресурсов в государственный сектор, в то время как концессионер получает возможность эксплуатировать объект и получать доход в свою пользу.

\textbf{Дорогие дороги: объединение капиталов}

ГЧП и концессии в автодорожном строительстве – это мировая классика, указывает директор группы по привлечению финансирования Kept Сергей Игнатущенко. «Инвестор строит дорогу, зная, что 30 лет ему потом эту дорогу содержать и ремонтировать, поэтому построит качественно, – указывает эксперт. – Более того, считается, что дорожные концессии на горизонте жизненного цикла проекта дешевле государственного заказа, поскольку инвестор берет на себя риски уложиться в смету и риски эксплуатации – именно инвестор умеет такими рисками профессионально управлять». Именно в дорогах наиболее просто спрогнозировать и оценить экономический эффект для всех участников проекта, при этом не имеет значения – взимается плата за провоз груза или за проезд автомобиля, соглашается генеральный директор ГЧП.РФ Виталий Нефедов.

\begin{fancyquotes}
    \textbf{ГЧП – цифры и факты.} Согласно аналитическому обзору Национального центра ГЧП, в России по состоянию на конец 2022 г. было 3724 проекта ГЧП, среди которых 2720 были в сфере коммунально-энергетической инфраструктуры, 652 – социальной инфраструктуры. Наибольший объем инвестиций – 2761 млрд руб. – был вложен в проекты регионального уровня, еще 1763 млрд руб. – федерального и 888 млрд руб. были на муниципальном уровне. Общий объем инвестиций по результатам прошлого года достиг 5,4 трлн руб., из которых 3,9 трлн – это частные деньги.
    В 2022 г. объем инвестиций в проекты, прошедшие коммерческое закрытие, достиг 370 млрд руб. Количество проектов, запущенных за год, по разным формам ГЧП достигло 150.
    Наиболее популярной в стране формой ГЧП является концессия – из 3648 проектов ГЧП 2933 имеют форму концессионных соглашений. Больше всего денег привлекается в транспортные проекты. Они наиболее капиталоемкие: на 160 проектов приходится 3,1 трлн руб., из которых 1,83 трлн частные.
\end{fancyquotes}



Автодорожные проекты очень капиталоемки, поэтому без объединения частного и государственного капиталов невозможно обеспечить устойчивое развитие дорожной инфраструктуры и мостов, указывают эксперты. Сроки реализации проекта также могут быть существенными – от двух и до 5–10 лет.

Частная сторона в ГЧП отвечает за проектирование, финансирование, строительство или реконструкцию объекта инфраструктуры, а также участвует в его последующей эксплуатации и техническом обслуживании. «Преимущество концессий для государства и населения – это возможность получить общественно значимый инфраструктурный объект раньше, чем у государства появится возможность самостоятельно его создать и эксплуатировать, – указывает старший менеджер группы юридических услуг группы компаний Б1 Анжелика Бурдейная. – При этом за счет механизмов контроля и штрафов, обычно предусматриваемых в концессионных соглашениях, технические характеристики и качество эксплуатации часто выше, чем могли бы быть без привлечения инвесторов». Для инвесторов концессионное соглашение или соглашение о ГЧП/МЧП интересно сбалансированным распределением рисков между сторонами (например, по сравнению с государственным контрактом), резюмировала она. «Концессия также дает инвестору возможность структурировать проект на более комфортных относительно 44-ФЗ условиях, а также получить финансирование на более выгодных относительно классических инвестиционных проектов условиях, так как при ГЧП риски, с точки зрения банков-кредиторов, существенно ниже», – говорит Нефедов.

\textbf{Выгодно всем}

На сегодняшний день часть региональных проектов может быть реализована за счет привлечения средств частных инвесторов лишь при условии, что проект получает федеральную поддержку. Одной из форм такой поддержки является межбюджетный трансферт, направляемый субъекту РФ на реализацию концессионных проектов в отношении автомобильных дорог. Средства направляются регионом на частичное софинансирование расходов на строительство объектов.

При этом именно концессионные региональные проекты являются оптимальным инструментом для привлечения частных денег, потому что обеспечивают интересы всех участников, так называемый win-win, когда выгодно всем. Например, для инвесторов в них предусмотрены механизмы защиты, субъекты РФ могут за счет частных денег и средств из федерального бюджета построить необходимый для развития региона дорожный объект. Для государства в целом важно, что средства федерального бюджета будут выделены на уже проработанный проект, по которому участники готовы начать строительство, и выполнены все предварительные для этого условия (проработаны условия проекта / заключено соглашение на конкурентных условиях / осуществлено проектирование за счет частных инвестиций / предоставлены земельные участки и т. д.). Такой подход позволяет Федерации решать глобальные задачи без существенных организационных усилий со стороны Федерации на ранних этапах его реализации.

Из наиболее ярких примеров региональных концессионных проектов, реализуемых с привлечением средств федерального бюджета и частных инвесторов, – обход Хабаровска, мост через реку Обь в Новосибирске.

При этом на сегодняшний день есть некоторые проблемы при реализации. По словам Игнатущенко, во многих регионах нет большого опыта ГЧП-проектов, именно поэтому так важна работа по формированию соответствующих компетенций региональных полномочных органов в структурировании, «упаковке» проектов и грамотном распределении рисков. «Качественная подготовка ключевых транспортных проектов, включая обоснование важности для развития экономического потенциала не только отдельного региона, но и прилегающих территорий / субъектов Федерации, повышает привлекательность проекта как для частных инвесторов, так и для федерального центра (при распределении бюджетных инвестиций)», – указал эксперт.


\textbf{На помощь приходят профессионалы}

Чтобы сделать процесс участия бизнеса в ГЧП менее рисковым и снять опасения частных инвесторов, что он не будет реализован, необходимо привлечь надежных партнеров, которые уже имеют соответствующий опыт. Например, инфраструктурный «МЕГАИГРОК» Газпромбанка представляет собой «внутренний консорциум» дочерних и зависимых компаний, которые совместно участвуют в процессе реализации проекта ГЧП.

Банк не просто предоставляет финансирование – он скорее выступает как комплексный игрок рынка ГЧП, который связывает всех участников проектов ГЧП: государство, инвесторов, строительные компании, операторов, банки. «МЕГАИГРОК», как инициатор проекта, за счет комплексной экспертизы и опережающего финансирования на этапе структурирования и проектирования обеспечивает доведение проекта до состояния готовности к выделению внебюджетных средств, федеральной поддержки и началу строительства. На последующих этапах банк обеспечивает контроль финансовых потоков и бесперебойности финансирования, осуществляет банковское сопровождение средств, а также контроль надлежащего строительства объектов (включая контроль качества поставляемых материалов, машин, оборудования и т. д.).

«МЕГАИГРОК» в том числе анализирует возможность продажи инвестиций на фондовом рынке, например с помощью выпуска облигаций. Кроме того, одна из целей «МЕГАИГРОКА» – превращение инфраструктурных сделок в сделки M\&A по мере становления рынка.

Ранее первый вице-президент Газпромбанка Павел Бруссер рассказывал, что благодаря «МЕГАИГРОКУ» на рынке появился некий инфраструктурный конвейер, в котором высвобождающиеся после реализации проектов средства оперативно направляются на создание новых проектов. Потоковое строительство инфраструктурных объектов, особенно транспортных, способствует росту экономики и ее структурной перестройке, указывал он.


\begin{fancyquotes}
    \textbf{Выходим на новый уровень}

    Исполнительный вице-президент – начальник департамента структурирования инфраструктурных проектов и государственно-частного партнерства Газпромбанка Иван Потехин считает, что для выхода на новый уровень применения ГЧП для строительства дорог в регионах требуется изменить роль «МЕГАИГРОКА» с финансового партнера до лидера. Таким образом, обеспечить ускоренное начало строительства инфраструктурных проектов в целях развития дорожной инфраструктуры и мостов позволит комплексная экспертиза «МЕГАИГРОКА» и опережающее финансирование.
    В этой схеме «МЕГАИГРОК» выполняет следующие роли (самостоятельно или привлекая необходимых профильных игроков и партнеров):
    \begin{enumerate}
        \item
        \item инициирование и структурирование обслуживаемых проектов;
        \item обеспечение заключения соглашения и организация проектирования объекта;
        \item доведение проекта до готовности для привлечения банковского финансирования и выделения федеральных средств;
        \item участие в организации и контроле надлежащего строительства объекта, а также последующей эксплуатации объекта.
    \end{enumerate}
\end{fancyquotes}

Уникальная экспертиза и накопленный опыт «МЕГАИГРОКА» позволяют структурировать с выгодой для всех сторон самые сложные и капиталоемкие проекты, например северный обход Перми, дублера Егорьевского шоссе, северный обход Омска, строительство автодороги Солнцево – Лыткарино – Железнодорожный, суммарный объем инвестиций по которым превышает 400 млрд руб. Однако для эффективной реализации все эти проекты требуют федерального софинансирования, отмечают в Газпромбанке.

\textbf{Чтобы средства выделялись быстрее}

По словам экспертов, на практике федеральный бюджет выделяет средства только после того, как все параметры проекта утверждены и завершено проектирование. К этому моменту участники уже потратили не только время, но и деньги на структурирование проекта и проектирование без каких-либо гарантий со стороны федерального центра о последующем выделении средств. «На сегодняшний день наиболее активные регионы привлекают квалифицированных экспертов концессионного рынка для формирования параметров автодорожных проектов с целью участия в отборе (конкурсе) на федеральную поддержку, – указывает Нефедов. – Таким образом, государство благодаря уже имеющимся механизмам получает качественно подготовленный проект для дальнейшего принятия решения о выделении федеральных средств на его реализацию».

Однако такой порядок чреват риском для инвестора, ведь если в федеральном софинансировании будет отказано, то затраты станут невозвратными.



Решением проблемы могла бы быть выработка прозрачного механизма взаимодействия инвесторов, субъектов и федерального центра на всех этапах. Например, государство может утверждать перечень проектов, к реализации которых планируется привлекать частный капитал, предлагают в Газпромбанке. Также важно, чтобы стороны договаривались «на берегу» – т. е. необходим предварительный этап согласования параметров проекта и объема федеральной поддержки перед заключением концессионного соглашения. На этом этапе стороны обсуждают сотрудничество, заключают соглашение, и только тогда инвестор начинает тратить деньги на проектирование. Деньги же из федерального бюджета будут выделяться после окончания этапа проектирования и всех необходимых согласований.

«Составление перечня проектов – важный шаг, нужен четкий сигнал рынку, что планируются такие-то проекты, – отмечает Игнатущенко. – Во многих странах есть примеры инфраструктурных планов, которые в том числе являются маркетинговыми документами для инвесторов». Это больше, чем просто список с источниками финансирования, нужно не просто внебюджетное финансирование, а именно использование опыта и компетенций частных инвесторов в эффективном строительстве и управлении инфраструктурой, отметил эксперт. Также, по словам Игнатущенко, в мире широко распространена практика включения проектирования в обязательства концессионера. «В этом случае инвестор постарается сделать максимально качественное проектирование, понимая, что далее по этому проекту будет строить, также снижается риск необходимости переделывать проект на этапе строительства», – резюмировал эксперт.
Как прошел второй этап Бизнес-регаты «Ведомости»

\clearpage

% \chapter{Культ успеха}

\section{Что такое культ успеха*}

\textit{Что такое культ успеха и как отличить свои желания и цели от навязанных?}

\textit{Источник: \url{https://happymonday.ua/ru/chto-takoe-kult-uspeha}}

\textit{Катерина Вольська}

Наш ритм жизни все ускоряется, мы торопимся получить больше денег, выше должность, лучше машину, больше новых гаджетов, книг, тренингов — всего больше. Казалось бы, это вполне естественные желания. Мы же хотим быть успешными и счастливыми, хотим развития — значит, с нами все в порядке. В теории мы все абсолютно правы. Но на практике это выливается в стремление достичь всего и сразу.

\begin{fancyquotes}
    Причем это «все» зачастую нам предлагают извне  — маркетологи, таргетологи, менеджеры по продажам или наше собственное окружение.
\end{fancyquotes}

Это становится похоже на какую-то игру в успех. Со всех сторон нам подбрасывают новые условия этой игры и уверяют, что просто необходимо еще чего-то достичь, еще что-то попробовать и приобрести. Как только мы видим триггер — то, что запускает в нас желание получить еще больше, — то включаемся в погоню за успехом. А в ней нет времени анализировать свои потребности, мотивы, желания — игрок должен быстро бежать к общепринятым «стандартам» успеха, не думая о том, что из этого всего ему действительно интересно и нужно.

Самыми распространенными такими «стандартами» считаются:

\begin{enumerate}
    \item стать СЕО компании к 25-ти годам;
    \item открыть свой бизнес до 30 лет;
    \item жениться/выйти замуж до 25 лет;
    \item родить ребенка/детей до 30 лет.
\end{enumerate}

Знакомо? Согласитесь, что все эти «стандарты» давят — на кого-то в большей степени, на кого-то в меньшей. Давят и заставляют жить в стрессе, если их не проработать.

\textbf{Что такое культ успеха?}

Иногда мы так заняты погоней за успехом, что упускаем что-то действительно важное — то, на чем строится наша жизнь.

Давайте представим, что ваша жизнь — это дом. Его фундамент — это ваш осознанный подход к жизни, решения, цели, желания и выборы. Уделяя этому время, вы закладываете фундамент своего дома. Конечно, дальше вы будете строить стены, крышу, делать ремонт, ухаживать за газонами у дома и создавать уют. Но фундамент — прежде всего.

А теперь представьте, что из-за давления социума вы купите первые попавшиеся стройматериалы, доверите фундамент каким попало строителям и побежите выбирать краску для стен. Или покупать рассаду, потому что прочли пост друга в Facebook о прекрасной клумбе у его дома и срочно хотите себе такую же. В итоге ваш дом рушится, потому что стоит он все-таки на фундаменте, а не на клумбе.

В этом и заключается культ успеха — мы очень быстро хотим получить результат, «как у других», соответствовать планке семьи, друзей и всех тех, кто присутствует в нашем реальном и виртуальном мире. Мы постоянно хаотично мечемся между целями, потому что нам «тоже так надо». Но надо ли действительно?

Речь не о том, чтобы перестать стремиться к росту и развитию. Речь о том, что постоянная гонка и неудовлетворенность собой, своей жизнью, работой, бизнесом, семьей и обесценивание того, что у вас есть, становится эпидемией.

\begin{enumerate}
    \item Одноклассник занимает руководящую должность в крупном холдинге!
    \item Подруга отдыхает 3-4 раза в год!
    \item У знакомых уже семья!
    \item Друга повысили на работе!
    \item У знакомого свой бизнес в 24 года!
    \item Однокурсник зарабатывает \$Х в месяц!
    \item Одноклассница похудела на 10 кг!
    \item Родственница сменила сферу деятельности и теперь работает удаленно и с гибким графиком за те же деньги!
\end{enumerate}

Вся эта информация, доступная нам 24/7, всячески способствует формированию культа успеха. Посты в социальных сетях, рассказы знакомых и бесконечный поток информации в Google и YouTube — все создано, чтобы вы пополняли свое информационное поле и сравнивали себя с другими.

\begin{fancyquotes}
    В свою очередь, другие люди также имеют в своем поле людей, с которыми сравнивают себя. И так по кругу. Мы все строим фейковые дома, которые могут рухнуть в любой момент.
\end{fancyquotes}

Можно обойтись косметическим ремонтом, если повезет. Быстро восстановить силы и ресурсы, подкорректировать образ жизни, привычки и укрепить дом. Сфокусироваться на важном и перестать распыляться.

Можно сделать капитальный ремонт — здесь предстоит серьезная работа над собой, своими ценностями и, возможно, длительная стадия возвращения потерянного и упущенного в гонке за успехом. Для этого необходима пауза и осознание полученного опыта для корректировки своего пути.

А может быть, придется строить новый дом с нуля. Это будет длительный процесс поиска потерянных смыслов и опор. Я использую метафору и сравниваю успех с постройкой дома, но его реконструкция может обернуться настоящим кризисом в реальной жизни. Речь идет о физическом, эмоциональном, духовном истощении и необходимости обратиться к психотерапии.

\textbf{Как понять, что значит успех лично для вас?}

Важно отметить, что наше определение успеха может меняться в разные периоды жизни. Мы можем быть уверены, что успех — это новое место работы, смена профессии, увеличение дохода, отпуск, свадьба, улучшение здоровья, курсы, тренинги, но когда мы этого достигаем, у нас появляется новый образ успешного человека — и мы снова начинаем погоню.

Успех для каждого свой. Зачастую мы ассоциируем его с человеком, который реализовался и достиг своих целей. Скорее всего, он счастлив, ни от кого не зависит и свободно распоряжаться своим временем. Возможно, он достаточно много путешествует. Возможно, у него прекрасная семья. Наверное, его окружают такие же успешные люди. И, вероятно, этот человек никому не подражает — он может быть самим собой, знает, чего хочет, и двигается к своим целям. Что из этого у вас ассоциируется с успешным человеком? А чего совсем нет в вашем списке критериев успешности?

Чтобы понять, где начинается и заканчивается именно ваш образ успешного человека, следует поднять свой уровень осознанности — задать себе неудобные вопросы и честно на них ответить.

\textbf{Упражнение №1}

\begin{enumerate}
    \item Напишите «Для меня успех — это ...».
    \item Напишите «Для других людей успех — это  ...».
    \item Напишите «На самом деле для меня успех — это ...».
    \item Ответьте на следующие вопросы.
\end{enumerate}

На кого из своего окружения я ориентируюсь? (Вы можете быть не знакомы лично). Кого я считаю примером для себя? (Вы можете быть не знакомы лично).
Какая / какие сферы жизни этого человека / этих людей являются для меня примером? Какие достижения этого человека / этих людей являются для меня примером? Чего я хочу из перечисленного выше?
Что за этим стоит, какую свою потрxебность я хочу этим закрыть?

\textbf{Упражнение №2}

А теперь представьте, что у вас уже есть все, что вы описали в упражнении выше, и ответьте на такие вопросы: Что из списка выше для вас действительно важно и ценно? Что из этого вы можете убрать из своей жизни и чувствовать, что у вас все ок? Что из этого списка является самым важным? Что из этого списка является важным для вашего окружения (родители, семья, любимый человек, друзья, виртуальные друзья, подписчики)? Что из этого списка является важным только для вас? Почему именно это важно для вас?

\textbf{Как не дать чужим представлениям об успехе влиять на вашу жизнь?}

Какой бы крутой телефон вы не купили, через год он будет уже не очень, так как появятся новые модели. Какая бы интересная работа у вас не была, вы будете стремиться занять новую должность. Каким бы ярким не был ваш отпуск — вам захочется так, как на фото у знакомых.

{\it
«Я буду успешен, когда …» — это парадокс той самой гонки за успехом.
}

Мы перестаем ценить момент, в котором находимся, перестаем ценить то, что у нас уже есть. Мы обесцениваем наши достижения, потому что нам всегда есть с кем себя сравнить и понять, что мы все еще недостаточно хороши.

\begin{fancyquotes}
    Если речь идет о вдохновении результатами других — прекрасно! Тогда это мотивация двигаться вперед. А если нет? Если вами движет страх быть хуже других или зависть?
\end{fancyquotes}

Можно ведь просто порадоваться за чужие достижения, правда? Но вместо этого мы чаще всего воспринимаем чужой успех как свой провал.

Чтобы минимизировать влияние чужого успеха на свою жизнь и сконцентрироваться на собственном желаемом успехе, выполняйте простые упражнения каждый день. Для этого вам понадобится от 5 до 15 минут.


\begin{enumerate}
    \item Опишите жизнь, о которой вы мечтаете, и перечитывайте описание каждый день. Если вам захочется внести корректировки, когда что-то станет неактуальным, делайте это. Но обязательно фокусируйте себя сами. Иначе это сделают за вас.
    \item Пишите 3-5 благодарностей каждый день — себе, близким, Вселенной, кому угодно. Фокусируйте себя на позитивных моментах — это прибавит вам энергии и напомнит о том, что хорошее уже есть в вашей жизни. Наверняка его очень много, и совсем не обязательно бежать туда, где трава кажется зеленее.
    \item Записывайте 3-5 своих успехов каждый день — что хорошего вы сделали, что получилось, что начали делать, кого порадовали и так далее. Фокусируйте себя на достижениях — это придаст вам уверенность в себе и своих силах, избавит от иллюзии, что вы застряли и не двигаетесь так быстро, как все вокруг.
    \item Следите за своим информационным полем — окружением, семьей, социальными сетями, рекламой и т.д. Но всегда сверяйте свой вектор движения со своими настоящими критериями успешного человека. Уделяйте время своему фундаменту — «подправить» его и сделать косметический ремонт намного легче, чем строить новый дом.
\end{enumerate}

И не забывайте, что всегда будет кто-то лучше, успешнее, счастливее, богаче, стройнее и так далее. Всегда! Но только от вас зависит, как на это реагировать: завидовать и заниматься самобичеванием, впрячься в гонку за успешным успехом (возможно, даже не своим) или вдохновляться и идти собственным путем. И не забывайте, что всегда будет кто-то лучше, успешнее, счастливее, богаче, стройнее и так далее. Всегда! Но только от вас зависит, как на это реагировать: завидовать и заниматься самобичеванием, впрячься в гонку за успешным успехом (возможно, даже не своим) или вдохновляться и идти собственным путем.

\clearpage

% \chapter{Наркомания и Алкоголизм}

\section{Зависимость от наркотиков}

 {\it Источник: \url{https://20gp.by/informatsiya/}}

Зависимость от наркотиков (наркомания) --- это неспособность человека отказаться от приёма веществ, которые влияют на его психику. Вначале прием наркотических веществ вызывает эйфорический эффект. Появляется желание регулярно принимать наркотики с увеличением доз. Затем у человека исчезают защитные реакции, и постепенно формируется психическая и физическая зависимость от наркотического вещества с выраженным абстинентным синдромом («ломка»).

\textbf{Принчины возникновения}

Основная масса наркоманов употребляет наркотики не по медицинским показаниям. До конца точно неизвестно, почему одни люди становятся наркоманами, а другие нет, даже если они когда-либо употребляли наркотические вещества.

Считается, что роль в развитии наркомании играют определенные особенности личности (незрелость характера, слабый самоконтроль, чрезмерный интерес к новым ощущениям). Есть данные о \ed{предрасположенности}{предрасположенность}{predisposition} к наркомании людей с психопатическими чертами характера, душевно нездоровых. \explain{Повышенный риск}{increased risk} развития наркомании имеют люди, которые выросли в \ed{неблагополучных семьях}{неблагополучная семья}{dysfunctional family}. Не исключено, что склонность к этому заболеванию передается по наследству.

Немаловажную роль в распространении наркомании играют модные тенденции.

\textbf{Симптомы зависимость от наркотиков}

На наркозависимость обычно указывают \explain{косвенные признаки}{indirect signs}:
\begin{enumerate}
    \item неестественный блеск в глазах, расширенные или сильно \explain{суженые зрачки}{constricted pupils} (в зависимости от типа наркотика);
    \item \explain{несвойственное}{unusual} для человека поведение  - гиперактивность или вялость, которые сопровождаются нарушением координации движений;
    \item резкие \explain{перепады настроения}{mood swings};
    \item изменение манеры устной или письменной речи, \ed{почерка}{почерк}{handwritting};
    \item беспричинные бледность или покраснение кожи;
    \item изменение аппетита;
    \item сбой режима сна и бодрствования.
\end{enumerate}

Cимптоматика зависит от вида наркотического вещества:

У тех, кто употребляем марихуаны, расширены зрачки, глаза и губы краснеют. Человек гиперактивен, у него заметно повышается аппетит.

Признаки опийной наркозависимости (опиомания) – внезапная \explain{заторможенность}{lethargy}, \explain{сонливость}{drowsiness; sleepiness}, \explain{замедление речи}{slow speech}. Зрачки сужены, кожа бледная, губы становятся красными.

Под действием психостимуляторов движения человека порывистые и резкие. Зрачки расширены. Он быстро принимает решения, речь связная и беглая.

После употребления галлюциногенов у человека появляются зрительные или звуковые иллюзии. Следствием их длительного применения становятся психотические состояния и депрессивные психозы.

Дополнительные возможные признаки наркомании – следы от уколов, пятна на коже, порезы, ссадины, припухлости непонятного происхождения.

\textbf{Диагностика}

Вначале врач определяет клинические признаки интоксикации (нарушение координации движений, расстройства речи, нарушение мышления, изменение поведения, сознания, вегето-сосудистых реакций). Чтобы убедиться, что клиническая картина вызвана содержанием в организме человека одного или нескольких наркотических веществ, проводится химико-токсикологический анализ.

Для подтверждения зависимости от опиума и его производных используют налорфиновый тест на наркотик.

Не каждый наркотик можно выявить по результатам лабораторных исследований, потому что некоторые из них быстро разрушаются в организме. В таких случаях часто используют методы токсикологической биохимии – мембранную хроматографию и газовую хроматографию.

Установить факт употребления наркотика можно в домашних условиях с помощью экспресс-теста на наркотики (в образце мочи). Этот тест обнаруживает следы опиатов на протяжении 5 суток после однократного приема.


\textbf{Классификация}

К основным видам наркомании относятся: опиоидная наркомания, наркомания от приема стимуляторов, кокаиновая наркомания, каннабиоидная наркомания и другие.  Если наркоман принимает разные наркотики, говорят о полинаркомании.

\textbf{Действия пациента}

Для успешного лечения в специализированном стационаре наркозависимому человеку нужна сильная мотивация. Найти ее помогут родственники и близкие люди. Пациенту необходима госпитализация.

\textbf{Лечение зависимости от наркотиков}

Основные этапы лечения наркозависимости следующие:

дезинтоксикационная, общеукрепляющая, стимулирующая терапия в сочетании с отнятием у пациента наркотического вещества, которым он злоупотребляет;

активное антинаркотическое лечение;

противорецидивная терапия.

Суть лечения сводится к тому, что сначала наркомана избавляют от физической зависимости, а затем проводят курс психотерапевтической реабилитации и поддержки, чтобы в его сознании закрепилась мысль, что ему будет хорошо без наркотиков.

Необходимым условием лечения наркомании является госпитализация и наблюдение за состоянием и поведением пациента.

Для лечения применяют такие методы как гипноз, кодирование, применение успокоительных психотропных препаратов и нейролептиков. Также существуют методики, которые базируются на лояльном отношении к пациенту и не предусматривают принудительного лечения.

\textbf{Осложения}

Без лечения наркозависимость влечет за собой тяжелые последствия: разложение личности, снижение интеллекта вплоть до слабоумия, истощение жизненно важных органов, снижение иммунитета. Многие наркоманы склонны к самоубийству.

Среди наркоманов распространены ВИЧ-инфекция, гепатиты, вызванные нарушением техники внутривенных инъекций. Передозировка наркотика часто заканчивается смертью наркомана.

\textbf{Профилактика зависимости от наркотиков}

Первичная профилактика наркомании лежит в социальной плоскости и направлена на сохранение и развитие условий, которые способствуют здоровью человека, и на предупреждение влияния неблагоприятных факторов.

Вторичная (социально-медицинская) профилактика наркомании направлена на выявление ранних изменений в организме для срочного полного и комплексного лечения, оздоровления среды, в которой находится наркозависимый, и применение воспитательных мер.

Третичная (медицинская) профилактика направлена на предупреждение прогрессирования заболевания, предупреждение обострений и осложнений, на снижение уровня инвалидности и смертности.

\newpage
\section{Почему человек становится зависимым от наркотика*}

 {\it \url{https://www.israclinic.com/}}

\begin{fancyquotes}
    Основной причиной наркотической зависимости является желание испытать состояние эйфории, что в дальнейшем приводит к гонке за острыми ощущениями. Практически всегда сначала формируется психологическая тяга, а затем – физическая зависимость. Довольно быстро наступает состояние, когда наркотик принимается не ради удовольствия, а для поддержания нормального функционирования организма.
\end{fancyquotes}

Зависимость от наркотических веществ стала настоящим бичом современной молодежи – из-за легкой доступности все больше подростков пробуют наркотик, и впоследствии становятся зависимыми от него. Дело в том, что наркотические вещества на начальном этапе употребления вызывают состояние эйфории и приподнятости настроения, отодвигая любые имеющиеся проблемы на задний план, что в дальнейшем вызывает у человека потребность еще и еще раз испытать приятные ощущения. Однако эйфория быстро заканчивается, и тогда вступает в дело физическая зависимость, которая заставляет наркомана ежедневно искать средства на дозу, чтобы не чувствовать страшных симптомов наркотической абстиненции.

\textbf{Почему люди начинают принимать наркотики?}

Специалисты выделяют ряд причин, по которым люди начинают принимать наркотические вещества:

\begin{enumerate}
    \item уход от реальности. Наркотики дают возможность на какое-то время забыть о проблемах и неприятностях, погружая зависимого в иную реальность. Однако не стоит забывать, что проблемы сами по себе никуда не денутся, а со временем и с приходом наркотической зависимости их попросту станет еще больше;
    \item за компанию. Это наиболее частая причина, по которой подростки и молодые люди начинают принимать наркотические вещества. Подростка очень легко можно «взять на слабо», что в результате за считанные дни и недели формирует опасную наркотическую зависимость;
    \item жажда острых ощущений. Кто-то прыгает с парашютом в погоне за адреналином, кто-то отправляется путешествовать, а кто-то пробует наркотики. Из простого любопытства;
    \item чтобы понять близкого человека. Наркологи неоднократно фиксируют случаи, когда жены или родные наркомана начинают употреблять наркотические вещества. Отчасти это связано с тем, что такие люди морально надломлены, а также желают лучше понять близкого человека. Понятно, что в итоге это не приводит ни к чему хорошему. Сначала формируется психологическая зависимость от наркотика, а потом – физическая, что в результате уже требует лечения у специалистов.
\end{enumerate}

\textbf{Виды зависимостей от наркотиков}

Зависимость от наркотиков может быть:
\begin{enumerate}
    \item Психологическая;
    \item Физическая.
\end{enumerate}

В начале употребления формируется психологическая зависимость. Человек пробует наркотическое вещество и понимает, что ему нравится чувство эйфории, хорошее настроение и другие эффекты от употребления наркотика. В погоне за острыми ощущениями он начинает пробовать еще и еще. Далее, в зависимости от вида наркотика, спустя некоторое время формируется физическая зависимость, когда наркоман вынужден постоянно принимать психоактивное вещество. Однако он это делает не для того, чтобы вызвать приятные ощущения, а чтобы избежать физических мучений, на сленге именуемых как «ломка».

Последствия действительно страшные: несколько дней и недель может пониматься температура тела до 40°, появляется сильнейшая боль в суставах, расстройства ЖКТ и сбои в работе сердца, отсутствие аппетита, тревога. Также при этом появляются проблемы со сном, наркомана все время мучает бессонница. К этому следует еще прибавить подавленное настроение, депрессию, обильное потоотделение, кошмары, а порой и галлюцинации. Физическая и психическая зависимости от наркотика в итоге превращают некогда здорового человека в пациента наркологических центров и реабилитационных учреждений. Необходимо серьезное лечение, которое не только избавляет человека от физической (химической) зависимости, но и помогает переосмыслить свою жизнь, излечить психологическую тягу к наркотику.


\newpage
\section{Алкоголизм: причины, симптомы, стадии, лечение*}

 {\it \url{https://polyclinika.ru/tech/alkogolizm-prichiny-simptomy-lechenie/}}

Основное отличие алкоголика от бытового пьяницы – психическая и физическая зависимости. Пьяница пьет по обстоятельствам, у него практически не бывает запоев. Может воздерживаться после домашнего скандала или когда закончились деньги. Алкоголик на такие «мелочи», как карьера или разрушающаяся семья, не обращает внимания, потому что употребление спиртосодержащих жидкостей находится на первом месте. Алкоголизм всегда, в 100\% случаев, сопровождается деградацией личности и разрушением внутренних органов.

\textbf{Особенности алкоголизма}

Алкоголизм относится к группе токсикоманий. Всемирная организация здравоохранения утверждает, что употребление алкоголя возрастает одновременно с улучшением качества жизни.

Российская статистика несколько иная: в нашей стране количество алкоголиков увеличивалось в тяжелые годы, а за десятилетие с 2008 по 2017 сократилось более чем на 1 миллион человек.

В мире бремя этой тяжелой зависимости несут более 140 миллионов человек. Неоспоримо установлено, что заболевание неизлечимо. Древнегреческий потомственный врач Гиппократ говорил, что пьянство – добровольное безумие. Выражение актуально по сей день, поскольку единственное, что может сохранить разум и тело алкоголика – полный, абсолютный отказ от спиртного.

\textbf{Причины алкоголизма}

Исследователи разделяют причины формирования болезни на 3 равнозначные группы:
\begin{enumerate}
    \item биологическая – генетическая предрасположенность, нервные и психические болезни раннего детского возраста, особенности функционирования нервной системы в виде превалирования процессов возбуждения;
    \item социальные – среда обитания, в которой сформированы питейные традиции, возлияния по любому поводу при низком уровне моральных ценностей;
    \item психологические – изначально низкая самооценка, тревожность, неуверенность, неумение справляться со стрессом, только прием спиртного приносит удовольствие и вызывает эйфорию.
\end{enumerate}

Особое значение медики придают наследственности, когда признаки алкоголизма имеются у родителей пациента. Дети алкоголиков рискуют повторить путь родителей на 30\% больше, чем их здоровые сверстники. С болезненными проявлениями дети знакомятся в слишком раннем возрасте, когда критика еще невозможна, и легко следуют проторенным путем.

Коварство зависимости в том, что организм любого человека вырабатывает эндогенный этанол, необходимый для биохимических процессов. Внешний этанол от того, что выработан внутри, ничем не отличается. К тому же у пациентов недостаточно фермента, расщепляющего этанол, – алкогольдегидрогеназы.

\textbf{Симптомы и стадии алкоголизма}

Зависимость проходит несколько этапов:
\begin{enumerate}
    \item Патологическое влечение – игнорируются все потребности, кроме выпивки. Определить, что человек все-таки станет алкоголиком, просто: пациент всегда находит спрятанное спиртное, в любом месте и в любое время, и делает это по запаху.
    \item Угасание рвотного рефлекса – защитный рефлекс «выключается».
    \item Утрата количественного контроля – нет меры, выпивка «до упора», иногда до потери сознания.
    \item Повышение переносимости с последующим снижением – если на начальных стадиях для достижения опьянения требуются литры, то в финале – рюмки.
\end{enumerate}

Врачи выделяют несколько стадий алкоголизма:
\begin{enumerate}
    \item продромальная – группа риска, человек никогда не отказывается от употребления спиртного, когда есть такая возможность, особенно в компании;
    \item первая – есть психическая и физическая зависимости, тяжелое похмелье, эпизоды провалов памяти (амнезия), колебания настроения, но сохраняется семья и работа;
    \item вторая, часто обозначаемая как «хронический алкоголизм» – сформированы запои (непрерывное употребление), но есть светлые промежутки, присоединяются поражения центральной и периферической нервной систем, количество растет до максимума, многие теряют работу, возникает разлад в семье;
    \item третья – присоединяются осложнения в виде алкогольных психозов и судорожных приступов, разваливается личность, поражаются все органы и системы, утрачиваются социальные связи, для опьянения требуется минимальная доза.
\end{enumerate}

Алкоголизм у мужчин и женщин принципиального отличия не имеет, за исключением того, что женщины проходят все стадии втрое быстрее мужчин, полностью утрачивая привлекательность и скатываясь к неконтролируемой сексуальности.

\textbf{Диагностика алкоголизма}

Диагностика включает полноценное обследование, в котором, помимо нарколога и психиатра, участвуют терапевт и по ситуации – кардиолог, невролог и гастроэнтеролог. Нарколог устанавливает ведущие синдромы алкоголизма – толерантность или переносимость дозы, запои и их длительность. Психиатр совместно с клиническим психологом исследует структуру личности, интеллектуально-мнестическое снижение, обращая внимание на несчастные случаи в состоянии опьянения, оценивает профессиональное и личностное снижение. Также психиатр занимается лечением острых галлюцинозов и хронических бредовых образований.

Невролог лечит судорожный синдром, энцефалопатии (повреждения головного мозга), периферические невропатии, нарушения ходьбы и статики, нарушения чувствительности, инсульты и тромбозы.

Терапевт занимается проблемами сердца (гипертоническая болезнь, нарушения ритма) и дыхательной системы, гастроэнтеролог – болезнями печени, особенно циррозом.

Женский алкоголизм требует участия гинеколога, поскольку этанол разрушает детородные органы, нарушает менструальный цикл, лишая пациентку возможности стать матерью.

Необходимый и важнейший этап диагностики – принятие пациентом своей проблемы, осознание себя больным человеком. На первоначальном этапе усилия медиков направлены именно на то, чтобы пациент признал себя алкоголиком. Без этого невозможно двигаться дальше, в противном случае лечение окажется формальным, пациент «сорвется» при первом удобном случае.

Полезно для диагностики помещать пациента в стационар, где он имеет возможность наглядно и «во всей красе» наблюдать последствия алкоголизма в виде приобретенного слабоумия, серий судорожных приступов или невозможности самостоятельно ходить. Пациент, не утративший способности критически мыслить (на первой или второй стадиях), обычно осознает всю тяжесть проблемы, иногда прекращая употребление спиртного навсегда.

Особого внимания заслуживает пивной алкоголизм (гамбринизм). Эта форма заболевания отсутствует в официальной классификации болезней, травм и причин смерти, поскольку относится к журналистским штампам. Врачи говорят, что болезнь одна, и не имеет значения напиток, который послужил причиной формирования зависимости.

Тем не менее на практике эта форма встречается часто. Пациенты и, к сожалению, родственники относятся к этой форме несерьезно, начиная лечение только на стадии осложнений. Так, «пивное сердце» описано еще врачами Баварии в 1914 году у рабочих пивоваренного завода. Заболевание развивается медленно и как бы незаметно, однако изменения необратимы.

Наиболее точный и простой тест на алкоголизм (MAST) был разработан в 1971 году в Мичиганском университете. Это опросник, описывающий отношение к спиртному, частоту употребления, особенности похмелья, запои, защитные рефлексы, отношение родных, повод выпить и другие особенности. Вариантов опросника существует несколько, они имеют незначительные различия.

Беседа с опытным доктором, тестирование, врачебный осмотр направляют пациента на путь выздоровления. Иногда достаточно содержательной экскурсии в наркологическое отделение с демонстрацией пациентов в терминальной стадии, чтобы получить осознанное согласие на лечение.

Врач-нарколог и тем более психиатр имеют колоссальный опыт общения с пациентами подобного профиля, умеют в первые минуты нащупать болевые точки личности, подобрать слова для убеждения. Возможно, пациент и не согласится лечиться на первом посещении, но однозначно встречу запомнит, сможет обратиться в критической ситуации.

\textbf{Лечение алкоголизма}

Тактика лечения определяется стадией болезни и личностью пациента.

Все начинается с выведения из запоя, которое разумно проводить в стационарных условиях. Состояние пациента может быть настолько тяжелым, что требуется участие реаниматолога.

На детоксикацию отводится несколько дней. Универсального лекарства от алкоголизма не существует, используются медикаменты для коррекции кислотно-щелочного равновесия, антиаритмические и гипотензивные средства, мочегонные, противоотечные, противорвотные, антиконвульсанты, нейролептики и другие.

В легких случаях, когда нет угрозы жизни, этап детоксикации проводится в амбулаторных условиях или на дому. После прерывания запоя с пациентом подробно беседуют, выявляя его истинные установки и устремления.

Отличный результат дает кодировка от алкоголизма, которая бывает медикаментозной или гипнотической. При медикаментозном варианте (подшивка) через небольшой разрез вживляется дисульфирам, дающий резкую вегетативную реакцию при употреблении спиртного – сердцебиение, потливость, дрожь, подъем артериального давления. Пациента предупреждают о последствиях, обычно берут расписку во избежание судебного разбирательства в случае гибели.

При гипнотической кодировке пациента погружают в транс (пограничное между сном и бодрствованием состояние), закрепляя в его психике установку на трезвость. Такой вариант подходит пациентам внушаемым, проходящим лечение впервые или имеющим предыдущий положительный опыт. Этот способ полезно подкреплять каплями и таблетками от алкоголизма, содержащими химические соединения и травяные сборы.

Пациента, живущего в семье и имеющего работу, можно удерживать от рецидива длительное время. Бывших алкоголиков не бывает, разумеется, но за время ремиссии пациент успевает ощутить вкус и преимущества обычной жизни, самостоятельно прилагает усилия для того, чтобы не вернуться к пагубному пристрастию.

Лучшая профилактика алкоголизма – реальные жизненные цели, которые невозможно осуществить с пьянством: семья, дети, карьерный рост, повышение благосостояния. Психологическая коррекция также требуется от жены и других важных для пациента родственников, поскольку избавления от такой разрушительной зависимости можно достичь только общими усилиями.

\clearpage

\section[Про Бориса Акунина]{На Бориса Акунина завели дело по двум статьям УК}

\textit{На Бориса Акунина завели дело сразу по двум статьям УК --- о «фейках» и «оправдании терроризма»}

\textit{Источник: \url{https://www.bbc.com/russian/articles/cm50vmpje43o}}

\begin{fancyquotes}
    Росфинмониторинг в понедельник внес писателя Бориса Акунина (Григория Чхартишвили) в перечень террористов и экстремистов. Следственный комитет завел на него дело по двум статьям УК. Ранее от продажи его книг стали отказываться магазины.
\end{fancyquotes}

Для \ed{фигурантов}{фигурант}{%
    person involved in; another example is В настоящее время он
    --- единственный фигурант данного дела; }
перечня государство ограничивает возможность
финансовых операций:
у попавших в список почти полностью блокируются
все банковские счета.
Они даже не могут получать зарплату и расходовать более
10 тысяч рублей\footnote{10 000 RUB = 85 GBP} в месяц.
В реестр включают людей не только приговоренных
по экстремистским и террористическим статьям УК РФ,
но и \ed{подследственных}{подследственный}{person under investigation}.

О возбуждении уголовного дела против Чхартишвили (ранее он был объявлен «иноагентом») до внесения его имени в перечень террористов и экстремистов известно не было.

Позже в понедельник Следственный комитет России сообщил, что в отношении писателя возбуждено уголовное дело о публичном оправдании терроризма и о «фейках» об армии (ст. 205.2, ст. 207.3 УК РФ). В сообщении СКР сказано, что его планируют объявить в розыск.

«\ex{Вроде бы мелкое событие}{A seemingly minor event}, запрет книг,
объявление какого-то там писателя террористом, на самом деле важная
\ex{веха}{landmark, milestone}.
Книг в России не запрещали с советских времен.
Писателей не обвиняли в терроризме со времен Большого Террора%
\footnote{«Большой террор» (\textit{разг.} «Еж\'{о}вщина»)
--- термин современной историографии, характеризующий период
наиболее массовых политических (сталинских) репрессий
в СССР 1937—1938 годов.}.
Это не дурной сон, это происходит с Россией
\ex{наяв\'{у}}{in reality}, на самом деле.
Берегите себя, не потеряйте себя, если вы в РФ.
А если вы уехали, но вам очень тяжело
\ex{на чужб\'{и}не}{in a foreign land} и вы подумываете вернуться
--- не возвращайтесь.
Ночь будет становиться черн\'{е}е и черн\'{е}е.
Но потом все-таки рассветет», ---
так прокомментировал новость сам Борис Акунин.

На прошлой неделе издательство АСТ объявило о том,
что приостанавливает распространение книг
Акунина и Дмитрия Быкова (в реестре «иноагентов»).

О прекращении продаж книг Акунина и Быкова
также объявили сеть книжных магазинов «Читай-город — Буквоед»
и сервис цифровых книг «Литрес». В «Читай-городе» решение
о приостановке продаж книг двух писателей объяснили их
«недавними \ed{высказываниями}{высказывание}{statement},
которые получили широкую огласку в СМИ».
О каких именно высказываниях идет речь, неизвестно.

\textbf{Что будет с книгами Акунина в России?}
Хранение книг Акунина пока еще остается законным,
отмечают эксперты.
Как пишет адвокат проекта «Первый отдел» Евгений Смирнов,
согласно закону о противодействии экстремистской деятельности,
запрещается распространение только тех материалов,
которые включены в реестр «экстремистских».

«Никакие книги, написанные Борисом Акуниным, а также фильмы и сериалы,
которые сняты по его романам или сценариям в этот реестр не внесены.
Если художественные произведения начнут пропадать из книжных магазинов
и стримингов --- это будет инициатива этих книжных и этих
стриминговых платформ.
Читать в метро книги о Фандорине тоже не запрещено», --- пояснил Смирнов.

\textbf{«Я помогал, помогаю и буду помогать украинцам».}
Издательство АСТ в объявлении о прекращении продаж книг
Акунина и Быкова ссылалось на «публичные заявления писателей,
которые вызвали широкий общественный резонанс».

\ex{Судя по всему}{as it appears},
имелась в виду история с провластными пранкерами Лексусом
(Алексеем Столяровым) и Вованом (Владимиром Кузнецовым),
которые обманом связались с Акуниным и Быковым и позже
опубликовали отрывки из разговоров с писателями.

Пранкеры говорили с Акуниным якобы от имени президента
Украины Владимира Зеленского и бывшего министра культуры страны
Александра Ткаченко.

В одном из отрывков, которые без контекста опубликовали
прокремлевские пранкеры, Акунин якобы говорит,
что с пониманием относится к позиции
«хороший русский --- мертвый русский».

Позже Акунин в «Фейсбуке» назвал действия пранкеров провокацией.

«Про то, что я якобы согласен с тезисом ``хороший русский
--- мертвый русский'', разумеется, \ex{брехн\'{я}}{nonsense}.
Как принято у этой шушеры, они что-то в моих словах отрезали,
что-то намухлевали, что-то \ex{перекомпоновали}{rearranged}»,
--- отметил Борис Акунин.

В разговоре с пранкерами, если судить по опубликованным отрывкам,
Акунин выразил готовность оказывать поддержку Украине.

«Я думаю, что и все другие деятели российской культуры,
которые выступают против путинской диктатуры,
с удовольствием вам помогут и примут участие»,
--- говорил, в частности, Акунин.
Эта фраза широко разошлась в соцсетях.

Писатель позже прокомментировал это высказывание так:
«Я помогал, помогаю и буду помогать украинцам».

Быков в разговоре с пранкерами произнес фразу,
возмутившую сторонников вторжения в Украину.
Ее в частности цитировало и государственное агентство ТАСС.

«Конечно, когда убивают русских, мне обидно.
Но претензий к вам у меня нет, как у меня нет претензий
к Израилю насчет Газы», — сказал писатель.

Быков и Акунин ранее неоднократно публично выступали
с осуждением российского вторжения в Украину.
Борис Акунин с 2014 года с семьей живет за границей.

Борис Акунин сам сообщал, что несколько театров
--- Российский академический молодежный театр (РАМТ)
в Москве и Александринский театр в Санкт-Петербурге
--- убрали его имя с афиш спектаклей по его произведениям.

Летом пр\'{о}шлого года издательство «Захаров» сообщило,
что московский магазин «Молодая гвардия» снял с продажи
книги писателя Бориса Акунина.
В ответ издательство отозвало из магазина все свои книги.


\newpage
\section{Запрет чайлдфри, налог на бездетность}

\textit{Запрет чайлдфри, налог на бездетность: как в России борются за рождаемость}

\textit{Источник: \url{https://news.ru/}}
% https://news.ru/vlast/zapret-chajldfri-nalog-na-bezdetnost-kak-v-rossii-boryutsya-za-rozhdaemost/

\textit{В Госдуме объявили о подготовке запрета пропаганды чайлдфри}

В Госдуме приступили к работе над законопроектом,
которым будет запрещена пропаганда чайлдфри в информационном
пространстве.
Кто и зачем хочет запретить идеологию бездетности,
запретят ли в России аборты,
каким налогом могут обложить россиян,
как власти б\'{о}рются за демографию?

\textbf{Кто хочет запретить пропаганду чайлдфри?}
О подготовке запрета на пропаганду бездетности рассказала
депутат Госдумы Ирина Филатова.
Над законопроектом работают парламентарии из нескольких
думских фракций.

«Межфракционная рабочая группа по защите традиционных ценностей
Госдумы готовит законопроект о запрете пропаганды чайлдфри.
Инициативу поддержали в ряде федеральных ведомств.
Сегодня в ограниченном формате законопроект обсуждался
на слушаниях в Общественной палате, где тоже нашел поддержку»,
--- рассказала Филатова.

Она подчеркнула, что «речь идет исключительно о деструктивной
пропаганде в информационном пространстве».

Ранее схожий законопроект разрабатывали в Башкирии с целью
защитить детей от пропаганды бездетности.
Депутаты особо подчеркивали, что инициатива не должна коснуться
вопросов бездетности по религиозным соображениям, а также книг
и фильмов о жизни монахов.

\textbf{Запретят ли в России аборты?} В марте 2023 года в Русской
православной церкви призвали ввести в России уголовное наказание
за склонение женщины к аборту в медицинских учреждениях и дополнительно
ужесточить наказания за криминальный аборт.
Депутат Госдумы Руслан Хамзаев развил эту идею,
предложив приравнять склонение к аборту к убийству человека.

Радикальные инициативы не получили немедленной поддержки,
однако к ноябрю в Мордовии, Тверской и Тамбовской областях были
введены штрафы за «склонение беременной женщины к искусственному прерыванию беременности»
--- вне зависимости от того, был произведен аборт или нет.
Патриарх Московский и всея Руси Кирилл поддержал инициативу,
но предложил расширить ее на федеральный уровень.

Член Совета Федерации Ольга Ковитиди призвала не ограничиваться
штрафами за склонение к аборту и ужесточить наказания.
Ей \ed{возразила}{возразить}{to object, to raise an objection}
глава думского комитета по правам семьи, материнства и детства
Нина Останина, призвав «прекратить организованную
\ed{травлю}{тр\'{а}вля}{bullying} врачей».
Останина ранее отмечала, что депутаты,
которые требуют запретить аборты,
ранее сами же голосовали за сокращение материнского капитала.

На региональном уровне наказания за склонение к абортам
получили поддержку: аналогичные меры в ноябре приняли и
в Калининградской области, а в декабре --- в Курской.
В ряде регионов власти сообщили, что частные клиники
в добровольном порядке прекратили проведение абортов.
О запрете процедуры, настаивают депутаты, речи не идет.

«В каждом случае все индивидуально.
У кого-то, например, есть медицинские показания.
[...] Но мы хотим ограничить наших женщин от воздействия
извне в плане пропаганды.
Речь идет, например, о какой-то явной рекламе,
например о баннерах», --- заявил курский парламентарий Юрий Амерев.

\textbf{Кто поддерживает запрет пропаганды чайлдфри?}
Против идеологии чайлдфри уже длительное время выступает депутат
Госдумы РФ Виталий Милонов. Он назвал принципиальную бездетность
«противоестественной системой ложных ценностей,
пропаганда которой не может положительно сказаться на состоянии
внутри страны».

Чайлдфри, по его словам, следует обозначить как экстремистскую
идеологию и запретить в России.

Схожее заявление сделала \ex{зампред}{deputy chairman}
Совета по правам человека Ирина Киркора,
которая усмотрела в чайлдфри «годы идеологической пропаганды»,
навязанной России.

«Безусловно, понятие ``чайлдфри'' не соотносится с теми
базовыми нравственно-духовными ценностями и категориями,
установленными в указе нашим президентом.
Вообще ``чайлдфри'' --- это не наш термин, придуман был не нами,
а, скорее, навязан уже нескольким поколениям, и он,
к сожалению, стал модным.
А к чему это приводит? Это может привести только в демографическую
\ed{яму}{яма}{pit} и в итоге к
\ed{исчезновению}{исчезновение}{disappearance} целой нации»,
--- объявила Киркора в беседе с РИА Новости.

\textbf{Как предложили ответить пропаганде чайлдфри в России?}
Ответом российских законодателей на пропаганду чайлдфри может
стать возвращение налога на бездетность.
Такой существовал в Советском Союзе и может быть введен вновь,
заявил депутат Госдумы Евгений Федоров. По мнению члена комитета
по бюджету и налогам, налог на бездетных поможет обеспечить
материнский капитал.

«Надо ли вводить налог для этого?
Если денег не будет хватать для этих проектов --- надо.
Если денег будет хватать без этого, то не надо.
Это не наказание, это способ решения проблемы»,
--- объявил Федоров.
