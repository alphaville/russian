% \chapter{Культ успеха}

\section{Что такое культ успеха}

\textit{Что такое культ успеха и как отличить свои желания и цели от навязанных?}

\textit{Источник: \url{https://happymonday.ua/ru/chto-takoe-kult-uspeha}}

\textit{Катерина Вольська}

Наш ритм жизни все ускоряется, мы торопимся получить больше денег, выше должность, лучше машину, больше новых гаджетов, книг, тренингов — всего больше. Казалось бы, это вполне естественные желания. Мы же хотим быть успешными и счастливыми, хотим развития — значит, с нами все в порядке. В теории мы все абсолютно правы. Но на практике это выливается в стремление достичь всего и сразу.

\begin{fancyquotes}
    Причем это «все» зачастую нам предлагают извне  — маркетологи, таргетологи, менеджеры по продажам или наше собственное окружение.
\end{fancyquotes}

Это становится похоже на какую-то игру в успех. Со всех сторон нам подбрасывают новые условия этой игры и уверяют, что просто необходимо еще чего-то достичь, еще что-то попробовать и приобрести. Как только мы видим триггер — то, что запускает в нас желание получить еще больше, — то включаемся в погоню за успехом. А в ней нет времени анализировать свои потребности, мотивы, желания — игрок должен быстро бежать к общепринятым «стандартам» успеха, не думая о том, что из этого всего ему действительно интересно и нужно.

Самыми распространенными такими «стандартами» считаются:

\begin{enumerate}
    \item стать СЕО компании к 25-ти годам;
    \item открыть свой бизнес до 30 лет;
    \item жениться/выйти замуж до 25 лет;
    \item родить ребенка/детей до 30 лет.
\end{enumerate}

Знакомо? Согласитесь, что все эти «стандарты» давят — на кого-то в большей степени, на кого-то в меньшей. Давят и заставляют жить в стрессе, если их не проработать.

\textbf{Что такое культ успеха?}

Иногда мы так заняты погоней за успехом, что упускаем что-то действительно важное — то, на чем строится наша жизнь.

Давайте представим, что ваша жизнь — это дом. Его фундамент — это ваш осознанный подход к жизни, решения, цели, желания и выборы. Уделяя этому время, вы закладываете фундамент своего дома. Конечно, дальше вы будете строить стены, крышу, делать ремонт, ухаживать за газонами у дома и создавать уют. Но фундамент — прежде всего.

А теперь представьте, что из-за давления социума вы купите первые попавшиеся стройматериалы, доверите фундамент каким попало строителям и побежите выбирать краску для стен. Или покупать рассаду, потому что прочли пост друга в Facebook о прекрасной клумбе у его дома и срочно хотите себе такую же. В итоге ваш дом рушится, потому что стоит он все-таки на фундаменте, а не на клумбе.

В этом и заключается культ успеха — мы очень быстро хотим получить результат, «как у других», соответствовать планке семьи, друзей и всех тех, кто присутствует в нашем реальном и виртуальном мире. Мы постоянно хаотично мечемся между целями, потому что нам «тоже так надо». Но надо ли действительно?

Речь не о том, чтобы перестать стремиться к росту и развитию. Речь о том, что постоянная гонка и неудовлетворенность собой, своей жизнью, работой, бизнесом, семьей и обесценивание того, что у вас есть, становится эпидемией.

\begin{enumerate}
    \item Одноклассник занимает руководящую должность в крупном холдинге!
    \item Подруга отдыхает 3-4 раза в год!
    \item У знакомых уже семья!
    \item Друга повысили на работе!
    \item У знакомого свой бизнес в 24 года!
    \item Однокурсник зарабатывает \$Х в месяц!
    \item Одноклассница похудела на 10 кг!
    \item Родственница сменила сферу деятельности и теперь работает удаленно и с гибким графиком за те же деньги!
\end{enumerate}

Вся эта информация, доступная нам 24/7, всячески способствует формированию культа успеха. Посты в социальных сетях, рассказы знакомых и бесконечный поток информации в Google и YouTube — все создано, чтобы вы пополняли свое информационное поле и сравнивали себя с другими.

\begin{fancyquotes}
    В свою очередь, другие люди также имеют в своем поле людей, с которыми сравнивают себя. И так по кругу. Мы все строим фейковые дома, которые могут рухнуть в любой момент.
\end{fancyquotes}

Можно обойтись косметическим ремонтом, если повезет. Быстро восстановить силы и ресурсы, подкорректировать образ жизни, привычки и укрепить дом. Сфокусироваться на важном и перестать распыляться.

Можно сделать капитальный ремонт — здесь предстоит серьезная работа над собой, своими ценностями и, возможно, длительная стадия возвращения потерянного и упущенного в гонке за успехом. Для этого необходима пауза и осознание полученного опыта для корректировки своего пути.

А может быть, придется строить новый дом с нуля. Это будет длительный процесс поиска потерянных смыслов и опор. Я использую метафору и сравниваю успех с постройкой дома, но его реконструкция может обернуться настоящим кризисом в реальной жизни. Речь идет о физическом, эмоциональном, духовном истощении и необходимости обратиться к психотерапии.

\textbf{Как понять, что значит успех лично для вас?}

Важно отметить, что наше определение успеха может меняться в разные периоды жизни. Мы можем быть уверены, что успех — это новое место работы, смена профессии, увеличение дохода, отпуск, свадьба, улучшение здоровья, курсы, тренинги, но когда мы этого достигаем, у нас появляется новый образ успешного человека — и мы снова начинаем погоню.

Успех для каждого свой. Зачастую мы ассоциируем его с человеком, который реализовался и достиг своих целей. Скорее всего, он счастлив, ни от кого не зависит и свободно распоряжаться своим временем. Возможно, он достаточно много путешествует. Возможно, у него прекрасная семья. Наверное, его окружают такие же успешные люди. И, вероятно, этот человек никому не подражает — он может быть самим собой, знает, чего хочет, и двигается к своим целям. Что из этого у вас ассоциируется с успешным человеком? А чего совсем нет в вашем списке критериев успешности?

Чтобы понять, где начинается и заканчивается именно ваш образ успешного человека, следует поднять свой уровень осознанности — задать себе неудобные вопросы и честно на них ответить.

\textbf{Упражнение №1}

\begin{enumerate}
    \item Напишите «Для меня успех — это ...».
    \item Напишите «Для других людей успех — это  ...».
    \item Напишите «На самом деле для меня успех — это ...».
    \item Ответьте на следующие вопросы.
\end{enumerate}

На кого из своего окружения я ориентируюсь? (Вы можете быть не знакомы лично). Кого я считаю примером для себя? (Вы можете быть не знакомы лично).
Какая / какие сферы жизни этого человека / этих людей являются для меня примером? Какие достижения этого человека / этих людей являются для меня примером? Чего я хочу из перечисленного выше?
Что за этим стоит, какую свою потрxебность я хочу этим закрыть?

\textbf{Упражнение №2}

А теперь представьте, что у вас уже есть все, что вы описали в упражнении выше, и ответьте на такие вопросы: Что из списка выше для вас действительно важно и ценно? Что из этого вы можете убрать из своей жизни и чувствовать, что у вас все ок? Что из этого списка является самым важным? Что из этого списка является важным для вашего окружения (родители, семья, любимый человек, друзья, виртуальные друзья, подписчики)? Что из этого списка является важным только для вас? Почему именно это важно для вас?

\textbf{Как не дать чужим представлениям об успехе влиять на вашу жизнь?}

Какой бы крутой телефон вы не купили, через год он будет уже не очень, так как появятся новые модели. Какая бы интересная работа у вас не была, вы будете стремиться занять новую должность. Каким бы ярким не был ваш отпуск — вам захочется так, как на фото у знакомых.

{\it
«Я буду успешен, когда …» — это парадокс той самой гонки за успехом.
}

Мы перестаем ценить момент, в котором находимся, перестаем ценить то, что у нас уже есть. Мы обесцениваем наши достижения, потому что нам всегда есть с кем себя сравнить и понять, что мы все еще недостаточно хороши.

\begin{fancyquotes}
    Если речь идет о вдохновении результатами других — прекрасно! Тогда это мотивация двигаться вперед. А если нет? Если вами движет страх быть хуже других или зависть?
\end{fancyquotes}

Можно ведь просто порадоваться за чужие достижения, правда? Но вместо этого мы чаще всего воспринимаем чужой успех как свой провал.

Чтобы минимизировать влияние чужого успеха на свою жизнь и сконцентрироваться на собственном желаемом успехе, выполняйте простые упражнения каждый день. Для этого вам понадобится от 5 до 15 минут.


\begin{enumerate}
    \item Опишите жизнь, о которой вы мечтаете, и перечитывайте описание каждый день. Если вам захочется внести корректировки, когда что-то станет неактуальным, делайте это. Но обязательно фокусируйте себя сами. Иначе это сделают за вас.
    \item Пишите 3-5 благодарностей каждый день — себе, близким, Вселенной, кому угодно. Фокусируйте себя на позитивных моментах — это прибавит вам энергии и напомнит о том, что хорошее уже есть в вашей жизни. Наверняка его очень много, и совсем не обязательно бежать туда, где трава кажется зеленее.
    \item Записывайте 3-5 своих успехов каждый день — что хорошего вы сделали, что получилось, что начали делать, кого порадовали и так далее. Фокусируйте себя на достижениях — это придаст вам уверенность в себе и своих силах, избавит от иллюзии, что вы застряли и не двигаетесь так быстро, как все вокруг.
    \item Следите за своим информационным полем — окружением, семьей, социальными сетями, рекламой и т.д. Но всегда сверяйте свой вектор движения со своими настоящими критериями успешного человека. Уделяйте время своему фундаменту — «подправить» его и сделать косметический ремонт намного легче, чем строить новый дом.
\end{enumerate}

И не забывайте, что всегда будет кто-то лучше, успешнее, счастливее, богаче, стройнее и так далее. Всегда! Но только от вас зависит, как на это реагировать: завидовать и заниматься самобичеванием, впрячься в гонку за успешным успехом (возможно, даже не своим) или вдохновляться и идти собственным путем.И не забывайте, что всегда будет кто-то лучше, успешнее, счастливее, богаче, стройнее и так далее. Всегда! Но только от вас зависит, как на это реагировать: завидовать и заниматься самобичеванием, впрячься в гонку за успешным успехом (возможно, даже не своим) или вдохновляться и идти собственным путем.