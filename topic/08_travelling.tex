\chapter{Путешествия и Туризм}

% --------------------------------------------------
% Туризм
% --------------------------------------------------
\section{Туризм}
Ещё двадцать лет назад немногие люди ездили в отпуск за границу. Большая часть людей проводила отпуск в своей стране. Сегодня ситуация другая, и мир кажется стал намного меньше. Сегодня стало возможным зарезервировать место на морском курорте на другой стороне мира через Интернет.

Не выходя из дома, вы можете \explainDetail{заказать}{заказывать/заказать}{to order (зак\'{а}зываю, -ешь, -ют / закаж\'{у}, зак\'{а}жешь, зак\'{а}жут)} билеты через \explainDetail{сеть}{сеть (ж)}{network} или по телефону. Самолёт \explainDetail{доставит}{доставлять/доставить}{to take sb somewhere} вас прямо туда, куда вы желаете, и через несколько часов после \explain{отбытия} {departure} из своей страны, вы сможете \explain{оказаться}{here: to find yourself} на тропическом \explain{побер\'{е}жье}{coast/bay/shore (побер\'{е}жье, -жья, -жью, -жье, -жьем, -жье)}, \explain{наслаждаясь}{наслаждаться/насладиться: to enjoy (наслаждаюсь, -ешься, -ются)} чистейшим в\'{о}здухом, плавая в кристально чистой, теплой воде тропического моря.

Мы можем путешествовать на автомобиле, поездом или самолётом, если нам \explain{предстоит }{предстоять: to lie ahead} долгая дорога. Некоторые молодые люди предпочитают путешествовать пешком или автостопом, при этом почти ничего не тратя на свое путешествие. Вы встречаете новых друзей, развлекаетесь и \explain{понятия}{понятие: idea} не имеете, где будете завтра. В этом и \explain{заключается}{заключаться/заключиться} большое преимущество для туристов --- тех, кто хочет получить все, что только возможно от исследования мира, \explain{при этом}{while, at the same time} не сильно \explain{утруждая}{утруждать/утрудить: to bother} людей вокруг. Если вы любите горы, вы могли бы подниматься на любые горы по всему \explain{земному шару}{замной шар: earth}. Есть только одно \explain{ограничение}{limitation}. Это деньги. Если вы любите путешествовать, у вас должны быть деньги, потому что это, в действительности, не дешевое хобби.

Экономика некоторых стран существует за счёт туризма. Современный туризм стал высоко развитой индустрией, потому что любой человек любопытен, любознателен и любит \explain{досуг}{leisure}, любит посещать другие места. Именно поэтому туризм процветает.
Люди путешествуют с самого начала своей цивилизации. Тысячи лет назад все люди были \explain{кочевниками}{кочевник: nomad} и собирателями. Всю свою жизнь они \explain{бродили}{бродить/побродить: to wander} \explain{в поисках}{поиск: search} \explain{продовольствия}{продовольствие: food} и лучшей жизни. Таким образом люди \explain{заселили}{заселять/заселить: to inhabit} всю планету Земля.

Так что путешествие и посещение других мест --- это часть нашего сознания. Именно поэтому туризм и путешествие настолько популярны. В настоящее время туризм стал высоко развитым бизнесом. Поезда, автомобили и воздушные реактивные лайнеры, автобусы, суда \explainDetail{предоставляют}{предоставлять/предоставить}{provide} нам комфортное и безопасное путешествие.

Если мы путешествуем \explain{ради}{for the sake of + [gen.]} удовольствия, каждый хотел бы, \explain{во что бы то ни стало}{by all means}, насладиться живописными местами, которые он пролетает, хотел бы увидеть интересные места, насладиться достопримечательностями городов и стран.

В настоящее время люди путешествуют не только ради удовольствия, но также и по делам. Люди должны ехать в другие страны для участия в различных переговорах, для подписания некоторых очень важных документов, для участия в различных выставках, чтобы показать \explain{товары}{товар: merchandise} собственной фирмы или компании. Бизнес-поездки помогают людям получать большее количество информации \explain{относительно}{regarding} \explain{достижений}{достижение: achievement} других компаний, что поможет \explain{создать}{создавать/создать: to create} более успешное дело.

Путешествовать можно по-разному: на корабле, самолёте, автомобиле, пешком. Всё зависит от человека, и его предпочтений.


% --------------------------------------------------
% Салоники
% --------------------------------------------------
\newpage
\section{Что посмотреть в Салониках}
Источник: \url{https://bit.ly/3O5ZooQ}

Для большей части туристов из постсоветского пространства знакомство с Грецией начинается именно отсюда. Международный аэропорт, морск\'{о}й порт, железнодор\'{о}жный вокзал -- Салоники, второй по \ed{величин\'{е}}{величин\'{а}}{size} греческий город, столица области Макед\'{о}ния является крупным транспортным \ed{узл\'{о}м}{\'{у}зел}{node}. Но благодаря богатой истории свои достопримечательности Салоники (Греция) тоже имеет. В городе \explain{сосредот\'{о}чены}{concentrated} памятники трёх эп\'{о}х: эллинист\'{и}ческой, римской и византийской.

Поэтому не стоит, прилетев в Салоники, использовать это место только в качестве транзитного пункта по пути на знамен\'{и}тые греческие курорты, \explain{посвят\'{и}те}{dedicate} и ему самом\'{у} несколько дней. Интересностей в городе Салоники, где древние \ed{раскопки}{раск\'{о}пки}{excavations} можно увидеть во дворе современных жил\'{ы}х кварталов хоть отбавл\'{я}й. \ed{П\'{о}льзуясь}{п\'{о}льзуясь}{taking advantage} советами тех путешественников, кто бывал здесь не однажды, постараемся \explain{провести вас по городу}{to guide you around the city} и \explain{подсказать}{suggest}, что можно посмотреть в Салониках за 3 дня.

\ed{На многих}{на многих}{for many (people)} этот город вначале произв\'{о}дит \explain{противоречивое}{contradictory} впечатление из-за нем\'{ы}сли- мого сочетания эпох и архитектурных стилей. Рядом могут сос\'{е}дствовать красивый парк, цветы, купола старых и новых хр\'{а}мов, древние раск\'{о}пки и тут же -- \explain{ржавые}{rusty} \explain{ограждения}{fences}, неудачные неряшливые граффити на стенах унылых многоэтажек... и вдруг, на стене другого дома -- совершенно оригинальное произведение современного арт искусства! И всё это \explain{чередуется}{alternates} в Салониках, \explain{квартал}{quarter} за кварталом.

Но постеп\'{е}нно нах\'{о}дишь в этой чехард\'{е} и калейдоскопе какую-то особенную гармонию. Некоторые туристы уезжают из Салоников, пон\'{я}в душу этого города и даже немного влюб\'{и}вшись в него.

А наше путешествие только началось. Что посмотреть в Салониках обязательно и никак нельзя пропустить?

\textbf{Прогулка по набережной, Белая Башня и часовой тур по «Культурному маршруту».}
Салоники распол\'{о}жены на берегу \ed{залива}{залив}{gulf} Термаикос. Пройдитесь ранним утром по широкой и красивой набережной, посмотрите на море, порт, рыбаков с \ed{удочками}{удочка}{fishing rod}. Повернувшись лицом к городу, увидите контур самого знакового городского \ed{сооружения}{сооружение}{structures; buildings; constructions}, архитектурного символа и визитной карточки Салоников -- Белую Башню с развивающимся над ней флагом.

И хотя на самом деле она не совсем белая, а «цвета буйволиной к\'{о}жи», история у достопримечательности (XV в) очень занимательная и \explain{заслуживает}{deserves} \ed{отдельного}{отдельный}{individual; separate} рассказа. О ней можно узнать, осмотрев музей, расположенный по кругу на 8 этаж\'{а}х внутри этого 33 метрового (23 м в диаметре) \ed{внушительного}{внушительный}{imposing} сооружения. На самом верху башни смотровая площадка, отсюда открывается красивый вид на набережную, порт и город.

Достопримечательностей в Салониках так много, что только прост\'{о}е \explain{перечисл\'{е}ние}{enumeration} основн\'{ы}х з\'{а}няло бы целую страницу. Но есть замечательная возможность увидеть внушительную их часть даже за 1 час. Конечно, пешком это невозможно, а только сев в автобус №50 тут же, на площади у Белой Башни. Синий экскурсионный автобус отправляется по «Культурному маршруту» каждый час с 8:00 до 21:00. И за 10 евро (5 для детей) в обзорном режиме вы совершите путешествие на машине времени. А остановки (их 8), словно разные эпохи, которые за 25 столетий пережила северная столица Греции. Как нигде, прошлое здесь тесно \explain{соседствует с}{neighbours with} настоящим.

Аудио, видео и гид в автобусе сопровождают экскурсию на английском и греческом. Но для первого визуального знакомства этого достаточно, а в ос\'{о}бо понравившиеся места можно потом вернуться, прихватив с собой карту Салоников с достопримечательностями на русском языке, чтобы не заблудится. Маршрут начинается и заканчивается у башни.

Что же увидят по пути экскурсанты? Недалеко от Белой Башни есть несколько интересных зданий:
\begin{enumerate}
    \item Королевский театр / Национальный Театр Северной Греции
    \item Археологический музей и Музей византийской культуры
    \item Башня телефонной службы
    \item Международная Выставка
    \item Македонский музей современного искусства
\end{enumerate}
Далее по пути чередуются:
\begin{itemize}
    \item памятники эпохи Древнего Рима — раскопки дворца Галерия, Ротонда святого Георгия, \explain{развалины}{ruins} ипподрома и Триумфальная арка, площадь Аристотеля, раскопки римской Агоры
    \item византийские и раннехристианские храмы и памятники – Агия (Святая) София, храм Богоматери Медников (XI в), храм святого Димитрия Солунского; монастыри Салоников – святой Теодоры и Влатадон (XIV в); руины византийских бань.
\end{itemize}

Маршрут проходит мимо площади Аристотеля, Верхнему городу Ано Поли с симпатичными \ed{разноцветными}{разноцветный}{colourful} македонскими домиками.

Об одном из объектов на маршруте по достопримечательностям Салоников расскажем подробнее.

\ed{На заметку}{на заметку}{on a note}! Подборка лучших экскурсий и русскоговорящих гидов в Афинах представлена здесь.

\textbf{Ротонда.} Ротонда святого Георгия постройки конца III века и Триумфальная арка (IV в) – часть дворцового (или погребального) комплекса императора Гая Галерия. В XV веке служила христианской церковью, которая была н\'{а}звана в честь Григория Победоносца. В османские времена турки рядом построили минарет, и почти четыре столетия в храме была \explain{меч\'{е}ть}{mosque}.

В начале XX века здание было возвращено православной церкви, и \explain{с той пор\'{ы}}{since then} здесь Музей христианского искусства. Ротонда находится в \ed{списке}{сп\'{и}сок}{list} памятников Салоников, которые включены ЮНЕСКО в \explain{п\'{е}речень}{list, scroll} объектов Всемирного наследия.

Последнее десятилетие на территории комплекса велись реставрационные работы. Среди описаний достопримечательностей Салоников фото этих работ часто встречаются в рассказах путешественников на форумах. Ведь и во время реставрации в некоторые \explain{помещения}{premises} иногда можно было входить (бесплатно) с камерами, и не запрещалось делать снимки. После официального открытия стоимость входа — 2 евро.

Ротонда находится рядом с Университетским городк\'{о}м, это одно из \explain{мест сбора}{gathering places; venues} и встреч студентов и местной молодёжи.

Утро второго дня и первую его половину можно посвятить каньонингу у Олимпа, а вечером посмотреть спектакль или концерт в Лесном театре.

На заметку! О пляжах \explain{в окрестностях}{in the vicinity of} города Салоники читайте на этой странице, а какой курорт Кассандры (Халкидики) выбрать для пляжного отдыха в этой статье.

\textbf{Лесной Театр.}  Этот театр среди лесных \ed{просторов}{прост\'{о}р}{open space} одно из нескольких \ed{подразделений}{подразделение}{division; unit} NTNG – Национального Театра Северной Греции, в его сост\'{а}ве и театральное училище, прекрасный резерв для \ed{труппы}{труппа}{troupe}. Студенты часто заняты в постановках театра. Посмотреть на спектакль действительно интересно.

Каждый сезон в расписании Театро Дасус премьеры и \ed{прежние}{прежний}{former; old; prior} спектакли собственной труппы, гастрольные выступления других греческих театров.

Кроме собственно театральных постановок здесь весь летний сезон прох\'{о}дят крупные конференции и фестивали, концерты греческих и \ed{заезжих}{заезжий}{visiting} знаменитостей, различные выставки. Такой плотный график и \explain{занятость}{employment; occupation; business} обусловлены отличной акустикой лесной сцены в виде амфитеатра и хорошим техническим оснащением. Количество зрительских мест -- 3894.

Отсюда прекрасные виды на \ed{окрестности}{окр\'{е}стность}{neighbourhood; vicinity} Салоников. И даже если в день вашего посещения нет спектакля или другого мероприятия, всё равно можно прекрасно провести время на свежем воздухе, в кафе, посмотреть окресности, любуясь пейзажами, и привезти домой прекрасные фото достопримечательностей Салоников.

Последний день посвятите шоппингу или просто экскурсии по рынку Модиано, а вечер -- знаменитой Лададике и греческой кухне. И непрем\'{е}нно погуляйте перед отъездом по вечерней набережной.

\textbf{Рынок Модиано.} Modiano Market в Салониках – достопримечательность, часто встречающаяся в \ed{отзывах}{отзыв}{review} и на фото туристов, описания его «\ed{завлекаловок}{завлек\'{а}ловок}{lure/bait}» красочные и вкусные. Если в Салониках у вас есть 3 дня, Модиано – это то место, на которое точно ст\'{о}ит посмотреть.

Хотя рынок и не самый большой, но в остальном по своему колориту и многим признакам напоминает типичный восточный базар. Всё, как там: шум, \explain{гул}{hum; buzz; clatter}, крики торговцев.

Ряды с мясом – полный набор для гурманов-мясоедов. Чуть дальше можно попробовать и купить самый свежий сыр и масло.

Огромнейший выбор оливок: зелёные, чёрные, маринованные, солёные, с приправами и без них, в готовой удобной таре и на развес.

Сезонные фрукты, сладости и \ed{приправы}{приправа}{seasoning} -- каждый найдёт всё, что хотел бы найти.

Но самые интересные -- ряды с \ed{дарами моря}{дар\'{ы} м\'{о}ря}{seafood}. Можно купить свежие морепродукты и тут же, в ближайшей таверне, превратить их во вкуснейший обед. Всё мясо и рыба, продукты, овощи и фрукты на Модиано только местные.

Рынок находится в центре, в начале проспекта Аристотеля (со стороны руин римского форума).

На Модиано много кафе и ресторанчиков, попробуйте вкусно приготовленные греческие блюда и выпейте греческий кофе. Обед обойдётся недорого, даже самый \explain{сытный}{satisfying}. А за трапезой интересно посмотреть на \ed{повседневную жизнь}{повседневная жизнь}{everyday life} жителей Греции и узнавать город и с этой сторон\'{ы}.

\textbf{Лададика.}
Исторический район Лададика -- продолжение набережной и архитектурное наследие Салоников. Злачное место и сосредоточие порочных заведений в прошлом, с начала 2000-х превратилось в один из самых ярких и современных центров ночной жизни в Салониках.

На две части Лададику делит центральная улица Цимиску. Слева остался фрагмент городской стены. Прогуляйтесь по маленьким старинным улочкам. \ed{Изюминка}{из\'{ю}минка}{highlight} этого района Салоников в архитектурном \ed{смешении}{смешение}{mix} стилей середины XIX века и более поздних построек -- именно эта эклектика и привлекает туристов. А с\'{а}ми греки всегда любили проводить здесь время \explain{в праздности}{in idleness; doing nothing}, отдыхая от повседневных \ed{забот}{забота}{worry; concern}.

С 1985 года припортовая Лададика -- охраняемый исторический памятник и строительство новых домов здесь запрещено.

Существующие жил\'{ы}е дом\'{а} отреставрировали предприниматели, открывшие затем свой бизнес на первых этажах отремонтированных стро\'{е}ний. Нижняя часть зданий -- это \explain{ядр\'{о}}{nucleus; core} исторического центра Салоников. Его особый стиль: кованные железные двери \explain{на фоне}{on the background} красного \explain{кирпича}{кирп\'{и}ч}{brick}. Раньше таких домов было много.

Отремонтированы также пол десятка зданий банка Фракия, \ed{склады}{склад}{warehouse} и ангары \explain{переоборудованы}{converted} в магазины и клубы. Открылось немало новых ресторанов, кафе, таверн дискотек и клубов.

Днём местные жители и туристы \explain{ч\'{и}нно}{decorously} попивают здесь кофе, а вечером и ночью открываются двери всех заведений и улочки заливает тёплый жёлтый цвет фонарей. Освещаются открытые террасы в тавернах и кафе. Под бокал вина или стопочку узо, вечер можно провести в уютном ресторане с греческой кухней и живой народной музыкой, здесь их множество. А можно перейти через дорогу и открыть дверь любого клуба, попасть на шумную дискотеку и послушать такой же живой концерт, но в хеви-металл баре.

Почти у каждого \explain{более-менее}{more or less} крупного заведения на Лададике есть свой сайт, и места можно заказать \explain{зар\'{а}нее}{in advance} в сети или по телефону.

Вот и в\'{ы}полнена программа на з дня: «достопримечательности Салоники Греция». И хотя это только малая часть того, что можно посмотреть в этом городе, воспоминания обо всём ув\'{и}денном прочно лягут в \ed{копилку}{коп\'{и}лка}{piggy bank} памяти. И останутся с нами до следующих «каникул в Греции».

Посмотрите видео: \url{https://youtu.be/oObiknNQPUc}



% --------------------------------------------------
% Милос
% --------------------------------------------------
\newpage
\section{Милос – остров Греции с действующим вулканом}

\textit{Источник: \url{https://bit.ly/3OcX6ED}}

Остров Милос обладает уникальными природными красотами и признан греками жемчужиной Эгейского моря. Жители страны и туристы рассказывают об этом курорте с искренним восторгом. Об этом уголке Греции известно многим, ведь именно здесь найдена уникальная статуя богини Венеры Милосской, которая сегодня выставлена в качестве экспоната в Лувре.

\textbf{Общая информация.}
Греческий Милос — один из более 200 островов архипелага Киклады, расположенный в его юго-западной части. Он занимает площадь 16.2 км. кв. Постоянно проживает на острове немного меньше 5000 человек.

Милос имеет вулканическое происхождение и сегодня его характерными географическими особенностями являются причудливой формы скалы с разноцветными горными породами. При этом растительность на острове достаточно скудная, а западная часть острова совсем дикая: здесь не живут люди, из дорог только парочка грунтовых.

\begin{fancyquotes}
    Интересно знать! На Милосе находится один из двух действующих вулканов в Греции.
\end{fancyquotes}

На Милосе очаровательные закаты, естественные пещеры, живописные скалы, чистейшее море с красивыми (хоть и не всегда комфортными) пляжами и, конечно, богатое наследие древнейшей Кикладской архитектуры. Несмотря на перечисленные преимущества, Милос не пользуется большой популярностью у туристов, что привлекает самостоятельных путешественников.


\textbf{Как добраться.}
Остров Милос в Греции расположен на расстоянии 160 км от крупного порта Пирей. Морское сообщение не прекращается даже зимой.

Из Афин добраться на Милос можно на катере или пароме, услуги предоставляют сразу несколько компаний. Дорога занимает около 3,5-6 часов, за это время паром делает несколько остановок, которые позволяют полюбоваться красотами Эгейского моря. В летний сезон количество рейсов паромов увеличивается, поскольку поток туристов растет. Дополнительно предусмотрены рейсы на острова Кикладского архипелага. Расписание нужно узнать заранее, билеты можно забронировать в режиме онлайн одном из сайтов перевозчиков: \url{www.seajets.gr}, \url{www.minoan.gr}, \url{www.zanteferries.gr}, \url{www.bluestarferries.com}, \url{https://aegeanspeedlines.gr}, \url{https://goldenstarferries.gr}.

На Милосе есть аэропорт, который круглый год принимает рейсы из Афин, а в теплое время года сюда прилетают чартерные рейсы.

\textbf{Достопримечательности острова.}
На острове много пляжей, но это не единственная причина, по которой следует посетить Милос в Греции.

В порт Адамантас прибывают все паромы из других точек страны. В городе туристам предлагают экскурсионные туры в разные точки острова, а также морские круизы вокруг Милоса.

\textbf{\ed{Бухта}{б\'{у}хта}{bay} Клефтико.}
Пожалуй, самые яркие впечатления вызывает экскурсия на яхте к бухте Клефтико, расположенной на юго-западе острова. Бухта примечательна белоснежными скалами и пещерой, которая служила прибежищем для пиратов.

Добраться до бухты можно и самостоятельно по суше, но для этого придется пройти небольшой квест – арендовать внедорожник или квадроцикл, проехать часть пути по бездорожью, а затем идти пешком еще 40-60 минут. Больше подробностей узнайте из видео внизу страницы.

\textbf{Город Плака.}
Столица острова – город Плака — находится на высоте более двухсот метров над уровнем моря. С его высоты открывается панорамный вид на залив. Яркой достопримечательностью города считается замок крестоносцев, который находится неподалеку от церкви Богородицы Таласситры.

В южном направлении от Плаки расположены руины древнего поселения Мелоса. Здесь сохранились остатки римского театра и храма. В 1820 году в развалинах города была найдена та самая статуя Венеры, которою сегодня можно увидеть в парижкском Лувре.

\textbf{Природные пещеры.}
Пещеры острова заслуживают отдельного повествования. Сикия – самая необычная пещера, расположенная в западной части Милоса. Сюда регулярно следуют яхты и корабли из Адамантаса, также есть дорога со стороны церкви Святого Иоанна.

Самым посещаемым местом является пещера, образованная четырьмя скалами. Из Адамантаса сюда привозят экскурсии.

В южном направлении от Милоса расположен островок Антимилос, тут разводят ослов редкой породы.

\textbf{Церкви острова Милос.}
Агиоса Николаоса в Адаманте – при церкви работает музей. Святого Харлампия в Адаманте – здесь хранятся древнейшие иконы византийской эпохи. Панагия Корифиатисса в Плаке – построена в 1810 году, отсюда открывается волшебный вид на залив. Панагия тон Родон или Розария – храм украшен во французском стиле. Самый живописный храм на острове – Панагия Фаласситра. Часто на фото острова Милос в Греции часто можно увидеть именно эту церковь. Святого Харлампия в Плакесе славится древними, красивейшими фресками и росписями. Агиос Спиридонас в поселке Триовассалос – на Пасху здесь проводят театральное представление, во время которого сжигают куклу Иуды.
Профити Илиас (Пророка Ильи) в поселке Клима примечательна мраморным фундаментом.
Панагия Портиани в поселке Зефирия – в прошлом храм был митрополитским собором, сегодня находится под охраной Министерства культуры Греции.

\textbf{Музеи острова Милос.}
\begin{enumerate}
    \item Археологический музей. Находится на центральной площади столицы острова. В качестве экспонатов представлены скульптуры, древнее оружие, керамика, украшения. Вход 2 евро.
    \item Церковный музей. Коллекция экспонатов представлена древними византийскими иконами, богатым церковным одеянием и уникальными реликвиями. Вход свободный.
    \item Фольклорный музей. Находится на центральной площади столицы в здании XIX века. Экспонаты – предметы обихода и изделия народного творчества, демонстрирующие культуру и обычаи греческого народа. Вход 4 евро.
\end{enumerate}

\textbf{Поселки на острове.}
Живописное рыбацкое поселение на Милосе в Греции, расположено в тихой бухте, защищенной скалами. Людей здесь очень мало. А немногочисленные отели выглядят, как настоящие рыбацкие домики. Пляж Фиропотамоса чистый, без волн, в цвет воды особенно радует глаз.

\textbf{Клима.} Клима – самая большая рыбацкая деревня. Колоритное место, где дома построены у самой кромки воды, первые этажи построек используются в качестве гаражей для лодок. Двери и балконы домов выкрашены в разные цвета, благодаря чему весь поселок выглядит ярко и привлекательно. Сюда стоит приехать, чтобы сделать колоритные фото.

Деревня Плака, словно, приклеилась к склону горы, ее внешний вид больше всего напоминает традиционную Грецию – белые дома с синими дверями и ставни, украшенные цветами. На вершине городка находится венецианский храм и открывается живописный вид на Милосский залив. Столицу острова Милос лучше всего осматривать, просто прогуливаясь по узким улицам.

\textbf{Трипити.}
Ранее здесь проживали ремесленники, сегодня в поселении туристы посещают древнее христианское кладбище – лабиринт из многочисленных ходов в пещере.

В поселке есть и удобный песчаный пляж, и широким выбором ресторанов, кафе и отелей. Также в Трипити есть, что посмотреть: милосские катакомбы, руины античного театра, церковь Святого Николая и ветряные мельницы на окраинах. При желании все достопримечательности можно обойти пешком.

\textbf{Пляжи.}
Милос славится комфортабельными пляжами, их по всей территории острова более 70. Большинство пляжей появились в результате вулканической активности. Если дует ветер с севера, идеальные для отдыха пляжи – Фириплака, Циградо, Палеохори, Айя Кириаки. При южном ветре лучше отдыхать на пляжах – Саракинико, Митакас и Фиропотамос.

Фиропотамос. Находится в одноименной деревушке, где часто собираются яхтсмены и рыбаки. Пляж удобный для отдыха, здесь развитая инфраструктура и есть деревья, создающие тень.

Саракино. Один из самых живописных пляжей. Находится в бухте, которая ранее использовалась пиратами. Над пляжем нависают белоснежные скалы. Укрыться в тени здесь практически невозможно, это место любят романтические пары.

Палеохори. Один из самых и посещаемых пляжей. Мягкий, мелкий песок окружен разноцветными скалами. Для отдыхающих предусмотрены шезлонги и зонтики, работает Центр виндсерфинга.

Фириплака. На этом пляже любят отдыхать семьи с детьми. Расположен в южной части острова, здесь почти никогда не бывает волн и порывов ветра. Берег образован разноцветными скальными породами.

Айя Кирияки. Живописный пляж с широкой береговой линией и чистейшей водой, окружен скалами. Неподалеку много кафе и ресторанов. Этот пляж создает впечатление уединенного места.

Папафрагас. Пляж находится в крошечной бухте, прибрежная полоса тоже небольшая, уютная. Добраться сюда достаточно сложно, поскольку спуск крутой и узкий. Но, проделав такой путь, вы будете вознаграждены удивительным видом.

\textbf{Климат и погода.}
На острове традиционный для Средиземноморья климат. Летом здесь жарко и сухо, а зимой – мягкая и дождливая погода.

Летом на острове дует освежающий северный ветер Мельтеми. Это сезонное явление, которое начинается во второй половине июля и длится до конца августа. Таким образом, в самый жаркий сезон на Милосе нет изнуряющего зноя.

Оптимальное время заняться изучением вопроса – как добраться до Милоса в Греции – между Пасхой и началом сентября. В мае средняя температура составляет +21 … +23 градус, вода в море прогревается до +18 … +19 градусов. В наиболее жаркие месяцы – июль-август – воздух прогревается до +30 градусов, а вода – до +26 град.

Если вам приходилось смотреть фильм «Пеликан», вы наверняка запомнили сказочные греческие пейзажи. Именно Остров Милос стал местом, где проходили съемки картины. Еще один повод посетить курорт – его форма. Милос похож на подкову, возможно, поездка сюда принесет вам счастье и удачу.

Больше интересной и полезной информации об о. Милос и его пляжах узнайте, посмотрев видео!

Видео: \url{https://youtu.be/U0D-1ijdS3E}




\newpage
\section{Герб Междуреченска}
В 1966 году был объявлен конкурс на лучший герб города, в котором было рассмотрено 59 эскизов. После рассмотрения представленных эскизов лучшим был признан проект герба под № 48, который городской комитет ВЛКСМ и предложил для утверждения.

\includegraphics[width=0.3\textwidth]{img/Flag_of_Mezhdurechensk_(Kemerovo_oblast).png}

Автор герба Вадим Гущин, несмотря на нарушение ряда важных законов геральдики, сумел просто и оригинально, отказавшись от традиционных для того времени шестерёнок, колб, отбойных молотков, отразить промышленную специфику, совместив её с природно-географическим положением города.

28 августа 1966 года газета «Знамя шахтёра» представила жителям Междуреченска герб: «Щит разделён на два поля: красное (вверху) — цвет труда и зелёное (нижнее) — цвет тайги. В свете вспыхнувшей искры кусок угля — главного нашего богатства. На зелёном поле — две голубых ленты — Томь и Уса. Таким образом, герб олицетворяет две главнейшие особенности города — направленность труда его жителей и природные условия».

Автор герба совершил небольшую ошибку. Дело в том, что река Уса впадает в реку Томь с её правого берега. В гербе 1966 года сходящиеся реки изображены текущими в левую геральдическую сторону (от зрителя — в правую), что не соответствовало действительности. Ошибка была исправлена, и 18 марта 1993 года был утверждён изменённый герб.


\section{Кошмар началс\'{я} на второй день}

\textit{Мария Гейн}

\textit{Истории туристов, которые решили бесплатно пожить у местных}

\textit{Истончик: \url{https://lenta.ru/articles/2022/09/27/couchsurfing/}}

За последние 20 лет сервис для поиска бесплатного \ed{ночл\'{е}га}{ночл\'{е}г}{overnight stay} в поездках CouchSurfing \explain{снискал популярность}{gained popularity} у путешественников по всему миру. Для многих такой формат поездок стал не только способом сэкономить \ed{приличные}{приличный}{decent} деньги на отелях, но и возможностью лучше узнать новый город или страну. Тем не менее каучсерфинг может оказаться настоящим испытанием для \ed{неподготовленного}{неподготовленный}{unprepared} туриста — форумы \ed{пестрят}{пестрить + чем?}{are full of} историями о неадекватных хозяевах и ужасных условиях, в которых оказались путешественники. Как решиться на каучсерфинг, насколько опасен такой формат путешествий и чем может обернуться ночлег у незнакомца — в материале «Ленты.ру».

\textbf{От хиппи до диванных \ed{скитальцев}{скиталец}{wanderer}}

Идея \ed{бескорыстного}{бескорыстный}{selfless, disinterested; here бескорыстное гостеприимство: selfless hospitality} гостеприимства родилась \explain{задолго до}{long before} появления интернета и \explain{восходит к}{dates beack to} эпохе хиппи. Не обремененные деньгами и стабильностью, они не имели средств на гостиницы и путешествовали в основном \explain{автостопом}{hitchhiking}. Чтобы не остаться без крыши над головой, хиппи вели так называемые рингушники — записные книжки с адресами и телефонами людей в разных городах, у которых можно было бесплатно погостить некоторое время.

Рингушники были \explain{едва ли}{hardly} не самой важной и постоянной частью жизни хиппи — их тщательно пополняли новыми адресами, а имена \ed{надёжных}{надёжный}{reliable} людей передавали из рук в руки. Позже привычку вести такие записные книжки переняли автостопщики, студенты и все, кто хотел объехать мир за копейки. Все изменилось с появлением интернета.

В основе создания сервиса CouchSurfing лежит красивая история об американце Кейси Фентоне, который отправился в путешествие мечты. В 1999 году 21-летний молодой человек, выросший в семье хиппи, купил дешёвые авиабилеты из Бостона в Исландию. \ed{Осознав}{осознав (осознать)}{realising}, что денег на гостиницы не хватает, молодой человек проявил \explain{изобретательность}{ingenuity, inventiveness}. Он раздобыл базу email-адресов Исландского университета и \explain{разослал}{sent out} сотню писем с просьбой приютить его бесплатно.

К удивлению Фентона, на его письмо \explain{откликнулись}{Откл\'{и}кнуться (откл\'{и}кнусь, откл\'{и}кнешься, откл\'{и}кнутся) / Отклик\'{а}ться (отклик\'{а}юсь, отклик\'{а}ешься, отклик\'{а}ются): to respond (e.g., to email)} десятки студентов. Добрые исландцы даже составили ему готовый план поездки, согласно которому он должен был перемещаться по стране от дома к дому. \explain{Воодушевлённый}{encouraged (by)} исландским \ed{радушием}{радушие}{cordiality}, Фентон решил подарить другим путешественникам такой же опыт.

После возвращения домой он зарегистрировал домен и придумал название проекта. Слово CouchSurfing буквально означает «путешествие по чужим койкам» и состоит двух частей: couch, что в переводе с английского означает диван и surfing — кататься на волне или \explain{странствовать}{wander}. Именно поэтому в народе каучсерферов стали называть диванными скитальцами.

\begin{fancyquotes}
    Словарь каучсерфера\\

    \textit{Хост} — хозяин, принимающий у себя путешественников\\

    \textit{Реквест} — запрос о ночлеге, который путешественник отправляет потенциальному хосту\\

    \textit{Вписка} — непосредственно сам ночлег\\

    \textit{Серфер} — человек, который ищет бесплатное жилье в поездке
\end{fancyquotes}

Пробную версию сайта запустили в 2004 году, в разработке Фентону помогали друзья и единомышленники. За первый год на платформе зарегистрировались 6000 пользователей, на следующий — уже 20 тысяч. CouchSurfing быстро завоевал популярность у бэкпэкеров сначала в Европе и США, а потом стал активно использоваться в Азии, Африке и Латинской Америке.

Спустя почти 20 лет после запуска популярность сервиса продолжает расти. Если верить данным «Википедии», на нем зарегистрировано 12 миллионов пользователей более чем из 200 тысяч населенных пунктов по всему миру — даже на далеком острове Пасхи путешественников готовы принять почти 60 хостов.

Слово «каучсерфинг» плотно вошло в обиход и стало нарицательным — сейчас так называют формат цифрового гостеприимства, когда жители какой-либо страны или города предоставляют безвозмездный ночлег путешественникам.

Помимо непосредственно самого сайта CouchSurfing.com существует несколько аналогичных сервисов. На некоторых при регистрации нужно платить символический членский взнос, где-то — проходить собеседование. Самыми известными альтернативами считаются BeWelcome, Servas International и Trustroots. Есть и более нишевые проекты: например, Warm Showers создан для велопутешественников, а Pasporta Servo — для любителей искусственно созданного языка эсперанто.

Найти хоста также можно на форумах или в социальных сетях — во «ВКонтакте» и на Facebook (запрещена в России; принадлежит компании Meta, которая признана экстремистской организацией и также запрещена в стране) есть множество тематических групп для поиска ночлега в отдельных странах и городах. Однако опытные путешественники предупреждают, что такой способ «вписаться» не всегда безопасен. Если CoushSurfing и подобные сервисы стремятся обеспечить максимальную прозрачность путем верификации учетной записи и системы оценивания хостов, то в соцсетях можно с легкостью нарваться на людей с не самыми чистыми намерениями.

Тем не менее, несмотря на все предосторожности и протоколы безопасности CouchSurfing, ночлег у незнакомых людей может иметь неприятные последствия. Одна из нашумевших историй произошла в 2009 году. Туристка из Гонконга приехала в Великобританию и договорилась о «вписке» в Лидсе — хостом оказался мужчина марокканского происхождения. Он два раза изнасиловал девушку и едва не убил ее. Мужчину приговорили к десяти годам тюрьмы.

Другой случай с CouchSurfing, поставивший под сомнение репутацию сервиса, произошел в Италии в 2014 году. Представившись выдуманным именем, местный житель пригласил в гости 16-летнюю девушку-подростка из Австралии, накачал ее наркотиками и изнасиловал. В ходе судебного разбирательства выяснилось, что мужчина не раз пользовался доверчивостью туристок и заманивал их к себе домой с помощью анкеты на CouchSurfing.

\textbf{Правила хорошего тона}

Страх столкнуться с неадекватным хостом или вовсе нарваться на маньяка часто отпугивает туристов от каучсерфинга. Чтобы максимально себя обезопасить, необходимо придерживаться нескольких принципов. Самое базовое — это внимательно изучить профиль хоста, прочитать отзывы и поделиться его контактами со своими близкими.

\begin{center}
    \Large
    Когда в профиле мужчины напрямую говорится о желании принимать только девушек, а в графе интересов стоит секс — есть повод насторожиться. Также стоит заранее пообщаться с хостом: выяснить, где придется спать (в отдельной комнате или вместе с хозяином), и попросить прислать фото
\end{center}

Как рассказала «Ленте.ру» клинический психолог, эксперт в сфере домашнего насилия, посттравматического стрессового расстройства и зависимостей Олеся Иневская, важно избегать ситуации зависимости от хоста. «У вас должны быть деньги на запасной вариант ночлега не только на случай конфликта или странного поведения хоста, но и на случай отказа. У хоста могут случиться непредвиденные ситуации непосредственно перед заселением», — посоветовала эксперт.

Кроме того, в среде каучсерферов есть негласные правила хорошего тона, соблюдение которых сделает пребывание в чужом доме комфортным. В каучсерфинге все находятся на равных, поэтому важно не злоупотреблять гостеприимством и не нарушать личные границы. Брать продукты без спроса, надолго занимать ванну и приходить в дом в грязной одежде — не лучший способ найти общий язык с хостом.

\begin{fancyquotes}
    Необходимо узнать культурные особенности и обсудить детали проживания: распорядок дня, бытовые условия (как хозяин дома потребляет воду и электричество), количество ночей и время вашего отъезда. Предвидеть, что может стать нарушением границ, практически невозможно

    \begin{flushright}
        Олеся Иневская, клинический психолог
    \end{flushright}
\end{fancyquotes}

Беспроигрышный вариант поладить с хозяином дома — привезти сувениры из родной страны или города, особенно ценятся съедобные подарки. «Предложите хосту совместно приготовить ужин национальной русской кухни, расскажите о традициях. Это станет хорошим выражением благодарности за то, что человек принимает вас у себя дома», — рекомендует практический психолог и член ассоциации когнитивно-поведенческих психотерапевтов Татьяна Сушкова.

Несколько россиян поделились с «Лентой.ру» своим опытом каучсерфинга и рассказали о самых курьезных и жутких случаях.

\textbf{«Буду мыть за вас посуду и травить байки о Путине»}

\textit{Василий, объехал по каучсерфингу две страны}

Началось все с путешествия по Испании. Я был классическим бедным студентом, который решил попутешествовать по Европе. Выбор пал на Испанию, которая была бюджетнее других стран ЕС. За три недели я побывал в Овьедо, Бильбао и Барселоне. Скачав приложение CauchSurfing, я столкнулся с проблемой. Так как я был новым пользователем с нулевым рейтингом, на мои запросы никто не отвечал. Чтобы привлечь внимание хостов, я обещал привезти из России угощения и рекламировал себя в качестве уборщика. К моему удивлению, в Барселоне меня приютили после шуточного предложения рассказывать анекдоты о российском президенте.

Второй страной, где я ни разу не бронировал гостиницу, стала Бельгия. Это маленькое государство, поэтому я решил найти «вписку» где-нибудь в центре и каждый день ездить в новый город. Мой выбор пал на Гент — небольшой средневековый городок с каналами и удобным расположением с точки зрения железных дорог.

Моим хостом стал пожилой фермер-фламандец Ксандр, который почти не говорил по-английски. Он поселил меня в крытый загон для лошадей, который уже не использовался и был оборудован для приема гостей. Хотя к сырости и не совсем приятному запаху пришлось привыкать, там было все необходимое для жизни: кровать, полки для вещей, туалет и даже что-то, напоминающее душ. По утрам Ксандр устраивал завтраки с фермерскими продуктами и в целом оказался очень милым стариком.

Для меня каучсерфинг — это история в первую очередь про авантюризм, ведь ты никогда не знаешь, что будешь делать. Хост может провезти тебя по тем местам, куда не добираются обычные туристы. В том же Генте хост со своим другом устроили мне бесплатную экскурсию по каналам на яхте.

\textbf{«Он напился и лег ко мне в кровать»}

\textit{Инна, убежала от хоста посреди ночи}

Я пользовалась сервисом CauchSurfing.com один раз, когда в 22 года поехала в одиночное путешествие во Францию. Скажу сразу — в этой поездке сбылись мои самые худшие опасения, все не задалось с самого начала. Я договорилась с мужчиной средних лет из Конфлан-Сент-Онорина (пригород в 30 километрах от Парижа) о «вписке» на два дня.

Мы условились встретиться возле его дома в определенное время, но он опоздал на полтора часа. Француз Анри жил в уютной чистенькой студии. Единственное, что меня смущало. — это перспектива спать с ним в одной комнате, пусть и не в одной кровати. Он любезно уступил мне большой раскладной диван, а сам устроился на раскладушке.

\begin{center}
    \Large
    Кошмар начался на второй день, когда он вернулся далеко за полночь с какой-то вечеринки, едва стоя на ногах. Я проснулась от резкого запаха перегара — Анри лежал почти вплотную ко мне и спал. Решение разбудить его оказалось ошибкой
\end{center}

Он не домогался меня и не пытался изнасиловать, но вел себя крайне неадекватно: кричал что-то на французском, ходил по квартире, открывал и закрывал окна. Мне стало банально страшно оставаться с пьяным человеком на 30 квадратных метрах, поэтому я собрала рюкзак и ушла. Было почти пять утра, пришлось ждать открытия кафе, чтобы позавтракать и спокойно умыться. Днем у меня был поезд в Нант, где меня ждала койка в хостеле. С тех пор каучсерфингом я не пользовалась.

\textbf{Не ради денег}

Если с туристами, которые хотят сэкономить на отелях и глубже погрузиться в другую культуру, все понятно, то что движет людьми, готовыми впустить в дом совершенно незнакомых людей?

По словам Вадима из Санкт-Петербурга, который за четыре года принял у себя 13 человек, каучсерфинг — это окно в мир. «До пандемии у меня останавливалось много иностранцев. Тогда я не мог позволить себе отдыхать за границей, поэтому это была возможность близко познакомиться с чужими культурами. Еще один плюс — ты начинаешь по-другому смотреть на свой город. У меня жила пара французов, которая попросила свозить их в Кронштадт. Сам я был там последний раз на экскурсии в седьмом классе и был приятно удивлен, как там все изменилось», — поделился он в беседе с «Лентой.ру».

\begin{fancyquotes}
    Приятный бонус — подарки от путешественников. Как-то два испанца привезли мне три килограмма хамона в знак благодарности
\end{fancyquotes}

Другие опытные хосты рассказывают, что принимают у себя людей ради практики иностранных языков. Например, Виктория из Новосибирска, закончившая институт востоковедения, успела «вписать» к себе 18 туристов из Китая.

«Конечно, неприятные случаи тоже бывали. Однажды, когда меня не было дома, парень из Харбина решил приготовить на моей кухне вок. Видимо, что-то пошло не так, и он испортил дорогую сковородку с антипригарным покрытием», — рассказывает девушка. Чтобы избежать подобных случаев, Виктория рекомендует сразу очертить путешественнику границы дозволенного. Например, брать книги, пользоваться кофемашиной и утюгом — можно, брать средства личной гигиены и продукты из холодильника — нет.

«Сразу видно людей, которые подались в каучсерфинг только ради халявы и воспринимают твой дом как бесплатный отель. Уже на этапе реквеста они много спрашивают про условия проживания и по минимуму говорят о себе. Скорее всего, такой человек даже не помоет за собой посуду, ведь будет считать тебя "персоналом"», — добавила Виктория.

Она подчеркивает, что к поиску «вписки» стоит подходить вдумчиво: не рассылать одинаковые заявки хостам, а делать их индивидуальными — благодаря этому каучсерфинг станет незабываемым опытом для обеих сторон, а не просто бесплатным ночлегом.

\begin{center}
    \Large
    Если хотите оставить после себя грязную посуду и скомканные простыни, идите на Airbnb
\end{center}

Как бы то ни было, каучсерфинг — гораздо больше, чем способ экономить в поездках, это целая философия путешествий и гостеприимства. Сейчас этот сервис может быть особенно полезен российским туристам, поскольку платить за отели за границей без карты иностранного банка невозможно, да и забронировать экскурсии по городу дистанционно не получится. В таких ситуациях хост может не только выручить с жильем, но и показать локальные достопримечательности и заведения.

\newpage
% Достопримечательности Москвы: что посмотреть и где гулять
\section{Достопримечательности Москвы}

\textit{Лана Ржевская\\
    Познаёт Москву каждый день}

\textit{Источник: \url{https://travel.yandex.ru/journal/moscow/}}

В столицу приезжают, чтобы ощутить историю и гордость за страну, побывать в значимых местах и увидеть, как меняется город, почувствовать его креатив и стремление к новому. Даже если вы приезжали в Москву или живёте в столице, всегда можно найти новые интересные точки, события, окружение, возможности.

В подборке 30 знаковых, модных и романтичных мест Москвы для молодёжи и семей с детьми, для традиционных и креативных путешественников. Здесь гуляют и развлекаются, получают новые эмоции, ощущения и знания.


\textbf{Достопримечательности Москвы: что посмотреть и где погулять.} Москва --- это город, в котором \ex{уживаются}{coexist, get along} средневековые \ex{улочки}{streets} XVI века и небоскрёб с самой высокой смотровой площадкой Европы, где в бывшем хрустальном заводе или шоколадной фабрике создают креативные пространства, а светский клуб --- это не лимузин и фрак, а кино в ночи и танцы под винил.

Здесь 500 лет хранит историю Грановитая палата, названная по белому камню на фасаде, теснённому на 4 грани. В Новый год на улицах \ex{наряжают}{(here) decorate} более 500 ёлок. Направляясь в Москву из Нижнего Новгорода, вы проедете то же расстояние, что и путешествуя по всем линиям московского метро, 450 км. А чтобы пройти пешком все улицы Москвы, понадобится не один месяц, ведь это более 6000 км.

В Москве почти 450 музеев и около 10 000 ресторанов, кафе и баров. В десятках парков можно угоститься вкусным кофе и фирменными пончиками. Или посетить международные выставки, соревнования и марафоны. И это очень нравится 12 миллионам москвичей и 20 миллионам туристов, которые ежегодно посещают самый большой российский город.

Знакомство с Москвой традиционно начинают с центра города --- Кремль и всё, что его окружает, неизменно привлекают туристов. Вряд ли найдётся в мире другая столица с главной крепостью на 27 гектарах и окружающим её центром, одновременно историческим и абсолютно современным.

\textbf{Красная площадь.} На этой площади узнавали царские указы и городские новости, провожали и встречали воинов, по указу Петра I показывали образец украшения новогодней ёлки и «огненных праздничных утех», проводили \ed{крестные ходы}{Крестный ход}{в православных и восточнокатолических церквях торжественное церковное шествие с большим крестом (от его несения в начале процессии она и получила своё название), иконами и хоругвями вокруг храма или из одного храма в другой (например, в Пасхальную неделю вокруг церкви, в Крещение на водосвятие), или от одного места к другому (например, от храма к реке для освящения воды, от захоронения мученика к освящению нового храма его имени, вокруг городов и т. д.).} и казни.

Площадь появилась благодаря крупному пожару, уничтожившему постройки с этой стороны кремлёвской стены. Долго в народе так и говорили --- на Пожаре, имея в виду на площади, где был тот самый пожар. Здесь торговали и нанимали попов для службы на дому.

Название «Красная» она получила как статус главной площади Москвы, но такой красивой, как сейчас, стала значительно позже.


Сейчас Красная площадь, знакомая по парадам и новогодним курантам, влечёт всех приезжающих в Москву. Она открыта для посещения в те дни, когда нет мероприятий. Но сюда можно попасть и во время концертов или праздников — можно купить билеты, например на этом сайте\footnote{\url{https://mos-kassir.ru/place/krasnaia-ploshchad.html}}.

Традиционные мероприятия --- августовский фестиваль военных оркестров «Спасская башня», сентябрьский концерт ко Дню города, ноябрьский фестиваль Средневековья и рыцарский турнир, декабрьская ГУМ-ярмарка и празднование Нового года с салютом, фестиваль «Московская Масленица» и празднование Пасхи, репетиции Парада и сам Парад, концерт в День России, июльский цветочный фестиваль.

\textbf{Кремль.} Когда-то на месте Кремля была Ведьмина гора с языческим ритуальным столбом. Обряды на горе пророчили младенцам — здоровье, воинам — силу. При Иване III началось масштабное строительство крепости на холме. Сейчас Кремль --- это 20 башен, 28 гектаров земли и 2235 метров кремлёвской стены высотой до 19 метров и толщиной до 6,5 метров.

Вход в Кремль --- через Кутафью башню. Внутри крепости можно увидеть мощёные площади и уютные скверы, дворцы и храмы, которым не одно столетие. А ещё --- знаменитый Царь-колокол, который не звонил, и Царь-пушку, которая не стреляла.

На Соборной площади в 12 часов по субботам, с мая по сентябрь, можно увидеть почётный караул. Церемония развода пешего и конного караула --- яркое и запоминающееся зрелище: военный оркестр, исторические мундиры времён Николая II, холёные гордые кони, эффектная конная карусель, всё торжественно и патриотично. В середине церемонии будет громко, готовьтесь.

Сидячих мест нет. Чтобы лучше видеть церемонию, заранее займите место поближе к ограждению. Купить комплексный билет на осмотр Соборной площади и билеты в музеи Московского Кремля можно на сайте. Цена от 700 рублей.


\textbf{Манежная площадь, Манеж: Часы мира, шопинг и выставки.} Манежная площадь всегда была шумной и торговой. Названия близлежащих улиц тому подтверждение: Моховая, где торговали мхом, Обжорный ряд, Лоскутный переулок. Сейчас вся торговля перенесена под Манежную площадь в 3-этажный торговый комплекс «Охотный ряд».

Хорошо, что «Манежка», как называют \ex{ТЦ}{торговый центр}, не стала только бутиковой, как ЦУМ с его шикарными витринами и космическими ценами. Здесь много популярных сетевых магазинов с доступными ценами, акциями и скидками. Потому и не пустует, торговля идёт оживлённо.

Интересны стеклянные купола над торговыми рядами. Они изготовлены из высокопрочного технического стекла, сквозь них в ТЦ проникает естественный свет.

На полусфере фонтана «Часы мира» изображена карта. Чтобы узнать, какое сейчас время в выбранном городе, нужно провести воображаемую линию вниз до неподвижного кольца вокруг купола. На нём числа от 1 до 12 — это час. Количество светящихся оконцев над часом, умноженное на 5, это минуты. В итоге получаются часы и минуты в выбранном городе.

Здание Манежа изначально предназначалось для строевых тренировок солдат и офицеров. И сразу стало архитектурным чудом: 45-метровое здание накрыто одним куполом на деревянном каркасе, без каких-либо дополнительных опор. Потом Манеж сменил амплуа: стал местом народных гуляний и концертов, хотя недолго был даже гаражом для машин правительственных кортежей.

Посмотреть расписание мероприятий и купить билеты можно на сайте. График работы вт-вск с 12:00 до 22:00.

\textbf{Площадь Революции.} Её ориентиры — 2 музея и 2 отеля. Музей Отечественной войны 1812 года и Исторический музей с одной стороны, гостиницы «Москва» и «Метрополь» с другой. Пять столетий назад посередине площади протекала река Неглинная, по берегам шла оживлённая торговля в рядах — яблочном и дынном, капустном и ягодном. Позже реку убрали в подземный коллектор, а Воскресенский мост разобрали. А вот Воскресенские ворота остались: около них находится символичный «Нулевой километр», где приезжие кидают монетку, чтобы вернуться.

Бывшая Воскресенская площадь была переименована в марте 1917 года, когда после отречения Николая II в здании Городской Думы создали революционный комитет. На площади стали собираться тысячи демонстрантов.


Несмотря на революционное название, сейчас у площади нет политического подтекста. Напротив, она стала местом проведения городских фестивалей и праздников. Например, летнего фестиваля цветов, осеннего праздника урожая с креативными арт-объектами из даров осени, рыбной или сырной недели. Детская ярмарочная карусель и аниматоры радуют детей, взрослые делают покупки и дегустируют деликатесы.

\textbf{Александровский сад.} После многолюдной Манежной площади приятно отдохнуть в спокойной зелени Александровского сада.

Войдите в Верхний сад через главные ворота с золочёными двуглавыми орлами на столбах.


В Саду несколько точек, связанных с военной историей. Памятные плиты городов-героев, итальянский грот «Руины», который символизирует военную победу и возрождение Москвы после войны 1812 года.

Самая популярная фотолокация Александровского сада — фонтанный ансамбль с каскадами, сказочными героями и «Гейзером».

В Среднем саду, от Кутафьей башни до Боровицкой, расположены кассы музеев Кремля, уютные лавочки для отдыха, сувенирные магазины.


\clearpage

\section{Иммиграция}

\textit{Россияне рассказывают, каково это — жить вдали от родины}

{
    \it В рубрике «Из жизни» каждую неделю публикуются рассказы людей, которые по разным причинам решили переехать в другие страны. Одни отправились за границу в поисках любви, другие уехали на время обучения, а третьи просто решили, что вдали от родины их ждет лучшая жизнь.
}

\textit{Источник: \url{https://lenta.ru/themes/2018/02/12/migration/}}

\subsection{Япония}
% --- Studied 
% --- Added to Anki

\textit{«Русские мерзнуть не могут — кожа особенная» История россиянки, перебравшейся в Японию}

Рина переехала в Японию сразу после университета и влюбилась в страну с первого взгляда. Вопреки стереотипам о не подпускающих к себе японцах она стала своей в небольшом городке, а позже — успешно ассимилировалась в мегаполисе. В рамках цикла материалов о россиянах, перебравшихся за границу, «Лента.ру» публикует ее рассказ о жизни в Токио.

После окончания университета — в тот момент я находилась в Бельгии — мне захотелось продолжить свой жизненный путь за границей. Мне казалось, что этот период времени идеален для получения новых впечатлений и погружения в другую культуру — у меня не было семьи, детей, постоянной работы.

Я подала заявления на стажировки в разные страны, включая США, Германию и Японию. Однако только в Японии стажировка оплачивалась настолько высоко, чтобы можно было поехать туда почти без накоплений. К тому же бесплатно предоставлялось корпоративное жилье.

Шесть месяцев я жила в большом центре для интернов, гостей и партнеров компании, а также командированных из других филиалов. У нас были огромная кухня, сад и даже небольшой горячий источник. По прошествии полугода мне предложили постоянный контракт без срока действия, и я сменила свой статус, получив пятилетнюю рабочую визу.

Сразу оговорюсь, что я ехала в Японию, предварительно сдав международный экзамен на знание японского языка JLPT на высший уровень N1, без этого в нашу компанию иностранцев не брали.

\newpage
\textbf{Наблюдения о Японии}

Япония — настоящий рай для эстета и гурмана. Еда здесь высочайшего качества, получить пищевое отравление практически невозможно, разве что вы сами напортачите и съедите что-то, что слишком долго пролежало в холодильнике.

Японцы помешаны на «сезонных» и «эксклюзивных» продуктах. Первое выражается в том, что регулярно выходят новые версии одного и того же товара — пудинга, мороженого, напитков, снеков — со вкусом сезонных овощей и фруктов или сезона как такового. Летом появляется куча еды со вкусом шоколада и мяты или соленой карамели, осенью — со вкусом сладкого печеного батата, тыквы, ранней весной — клубники.

Суть эксклюзивности в том, что каждая префектура в Японии славится какими-то определенными продуктами: префектура Яманаси — своим виноградом, винами, медом; Хоккайдо — морепродуктами и молочной продукцией; префектура Аомори — яблоками и яблочными пирогами; префектура Ямагата — вишней. Поэтому сладости вроде Kit Kat или косметику вроде тканевых масок для лица иногда можно найти с «эксклюзивными вкусами и ароматами». Они продаются только в конкретных префектуре или городе и основаны на самой известной продукции этого места.

\begin{fancyquotes}
    После России очень удивляет, что нет строгого контроля за продажей алкоголя. В маленьких городах много автоматов, продающих пиво и сливовое вино. В магазинах у меня лишь один раз за три года попросили удостоверение личности, чтобы проверить возраст. В основном покупатель сам жмет на кнопку на специальном экране, чтобы подтвердить, что он совершеннолетний, документы для этого не нужны. Нет запрета пить алкоголь на улице или в транспорте — в пятницу вечером в электричках очень много офисных работников, пьющих пиво
\end{fancyquotes}

Арендная плата в Токио очень высокая, но есть интересный лайфхак, как сэкономить на жилье, если вы несуеверны. Дело в том, что японцы очень не любят селиться в «нехороших квартирах», где кто-то умер или произошел какой-то инцидент. Причем это распространяется и на случаи, когда одинокий пожилой человек мирно отошел в мир иной в своей постели — ничего из ряда вон выходящего в этом нет. Так как никто не хочет заселяться в такую квартиру, риелторы снижают стоимость аренды вдвое, а иногда и втрое на первые год-два контракта аренды. После этого квартира считается «очищенной», ибо в ней жили обычные люди, и арендная плата возвращается к средней по рынку.

\textbf{О любимом времяпровождении в Японии}

Из развлечений я больше всего полюбила ездить на термальные источники. Как правило, они расположены в горах или долинах, на природе, в очень живописных местах. Источники бывают разного состава — сернистые, кислые, сероводородные, солевые — отчего они очень разные по цвету и мутности воды. Бывают и мутные, серого цвета источники, а бывают прозрачные, с водой янтарного оттенка. У всех источников разное влияние на организм, но они одинаково хорошо делают кожу блестящей и красивой, а также уменьшают боль в спине и в мышцах.



В крупных термальных комплексах огромное количество ванн с разными особенностями. Могут быть и лекарственные ванны с добавлением экстрактов растений, меда, цитрусовых, лепестков цветов. У термальных источников всегда есть гостиничные комплексы, где можно попробовать традиционный японский ужин кайсэки с множеством разнообразных миниатюрных блюд и расслабиться в комнате с татами в японском стиле, облачаясь в банный вариант юката — что-то вроде облегченной хлопковой версии кимоно.

\begin{fancyquotes}
    Я очень рекомендую этот вид отдыха всем приезжающим в Японию, но, пожалуйста, имейте в виду, что в банных комнатах и источниках можно находиться только голышом, купальники и плавки запрещены
\end{fancyquotes}

Мужчины и женщины осуществляют банные процедуры раздельно, хотя в отдаленных районах сохраняются места, где нагишом в один источник влезают все, независимо от пола. Также отдельная радость — это горячие источники для ног, неотъемлемый элемент всех термальных деревень. В них можно смыть усталость от долгой прогулки и согреться зимними днями.

\textbf{О медицине и налогообложении }

В Японии очень много типов налогов, шкала налогообложения здесь прогрессивная. Из-за чего ходит шутка, что что бы ты ни делал — работал, владел машиной или недвижимостью, приобретал их, инвестировал деньги, получал имущество или подарки, умирал — за все берут штраф, то есть налоги.

Но есть интересная система, которая называется фурусато но:зей — буквально «налоги — родному городу». Вы в течение года делаете денежные пожертвования какому-то городу, муниципалитету или селу на ваш выбор через специальную систему. В зависимости от суммы пожертвования можно выбрать подарки. Обычно это деликатесы или билеты на концерты, в музеи и парки. Но иногда встречаются интересные варианты вроде «мешка льда из Охотского моря», «дня работы на местной радиостанции в качестве ведущего» или «дня бесплатного катания на яхте по самым красивым местам префектуры».

В конце года вы идете в налоговую, предоставляете документы, что платили фурусато но:зэй. Вам делают налоговый вычет на эту сумму. Но есть ограничения по сумме максимального пожертвования, на которую идет налоговый вычет: 20 процентов резидентского налога. Допустим, если ваш годовой налог 200 тысяч йен, и вы сделали пожертвования на 40 тысяч йен, вам их возместят через вычет на 100 процентов. А если превысить эту сумму и пожертвовать 45 тысяч йен, то никто вам лишние 5 тысяч йен не вернет.

В итоге ваши пожертвования, то есть налоги, пошли любимому городу, где, возможно, мало налогоплательщиков, ваши инвестиции в него были полностью возмещены через налоговый вычет, да вы еще и подарки или продукты получили бесплатно.

Медицина в Японии не бесплатная, но даже если вы не работаете, вы сможете пользоваться государственной системой страхования здоровья. При минимальных взносах в месяц государство будет оплачивать 70 процентов от суммы вашего лечения и 70 процентов от суммы лекарств, купленных по назначению врача. Если вы работаете, взносы будут больше и тип страховки немного изменится, но все будет основано на вашем доходе — страховые отчисления тоже прогрессивны.

Вызов машины скорой помощи в Японии бесплатен, но везде висят плакаты с просьбами не вызывать скорую помощь при недомоганиях, с которыми вы способны сами дойти до больницы. В Японии есть проблемы с психологической поддержкой: консультации у психолога и немедикаментозные способы лечения совсем не развиты. От всех проблем сразу же прописывают антидепрессанты или иные лекарства, иногда с сильными побочными эффектами. Так что в случае каких-то проблем рекомендую искать психологической поддержки на родине.

\newpage
\textbf{Об общении и дружбе в Японии}

Про Японию можно часто услышать, что здесь нельзя стать своим — мол, местные всегда в вас будут видеть туриста или временного жителя, но точно человека другой культуры и других убеждений.

У меня совершенно противоположный опыт: до переезда в Токио я жила в маленьком городе сельского типа, где у людей очень тесные узы с соседями и местным сообществом в целом. Например, не раз случалось, что я приходила в посудную лавку купить чашки для чайной церемонии, а хозяин — его семья управляет этой лавкой уже не первую сотню лет — тут же звал свою жену с поля, чтобы она поприветствовала меня. В благодарность за разговор и покупки она дарила мне овощи со своей грядки — морковку, капусту, брокколи.

Еще меня звали участвовать вместе с жителями района в переноске о-микоси — маленького переносного синтоистского храма — для традиционного летнего фестиваля. Мы часто собирались с жителями района в традиционной японской рюмочной идзакая, где мне дарили праздничный торт на день рождения, утешали в сложные моменты в жизни, помогали советом. Хозяева готовили вкусности вроде сладкого рулета из яйца — конечно, все это в счет не включалось.

Я, кстати, и сейчас туда наведываюсь из Токио, меня всегда очень тепло встречают, ведь все постоянные посетители там — примерно одни и те же люди. Возможно, у тех, кто жил поначалу в больших городах, ситуация была иной, но я чувствовала очень сильную поддержку со стороны как соседей, так и сотрудников моего отдела в компании. На мой первый день рождения в Японии они мне подарили мешок сладостей и большую открытку, где каждый сотрудник от руки написал свое пожелание по-русски. Конечно, перевод был делом авторства Google-переводчика, но важно ведь внимание.

\begin{fancyquotes}
    Почти все мои друзья и приятели тут — японцы и японки. Проблем в общении я никогда не испытывала, контакты тоже легко удавалось устанавливать. Хотя, конечно, пришлось привыкнуть к тому, что подход к общению совсем другой по сравнению с Россией: предполагается, что собеседник умеет «читать между строк»
\end{fancyquotes}

Например, японец может в середине прогулки вдруг спросить: «Ты не проголодалась?» — но таким образом он будет не интересоваться вашим состоянием, а посылать сигнал: «Я проголодался, поэтому, пожалуйста, выбери какой-нибудь ресторанчик, где мы могли бы перекусить». В этом случае правильным ответом будет: «Ой, у меня как раз есть хорошее место на примете со вкусными ланчами!» или «Я не очень голодная, но как насчет перекусить вот в той лапшичной? Я могу взять какой-нибудь легкий перекус». Ответ «Нет, а что?» будет воспринят как неумение «читать» ход беседы, а собеседник в силу ментальности вряд ли скажет: «А я вот голодный, пойдем есть» — и, скорее всего, будет ходить грустный и голодный.

Так же нет культуры прямых отказов — в конце встречи вам обязательно предложат снова выпить и погулять, но исключительно из вежливости, это называется сяко:дзирэй или «социальный комплимент».

Со временем просто привыкаешь, что тот, кто в тебе заинтересован, будет более конкретен в своих приглашениях: спросит, какую кухню или фильмы любишь, поинтересуется, свободны ли следующие выходные. На приглашения с твоей стороны тоже никто напрямую отказываться не будет — несколько раз повторят, что заняты на работе, в надежде, что вы сами догадаетесь, что собеседник в вас не заинтересован. Это стандартный способ ухода от прямого конфликта, конфронтации, а также нежелание причинить собеседнику эмоциональный дискомфорт. Это не плохо и не хорошо — просто другой стиль общения.

\textbf{О работе в традиционной японской компании}

В Японии новых сотрудников без опыта, которые только-только окончили университет, не принято сразу допускать к работе. В крупных компаниях этому предшествуют обучение и практика, во время которых группу нанятых выпускников учат бизнес-этикету, инвестированию своих денег и сбережений, в том числе ради пенсии, использованию программного обеспечения, необходимого для работы, знакомят с товарами и услугами компании, дают необходимые технические или финансовые навыки. Как правило, первый рабочий день у всех начинается в апреле. Практика может длиться от месяца до полугода.

\begin{fancyquotes}
    Я работаю в промышленной отрасли, поэтому, несмотря на то, что я «белый воротничок» и работаю в сфере финансового анализа и корпоративного планирования в главном офисе в Токио, меня, как и всех, в период практики отправили на месяц работать на один из заводов компании в глухую деревню. Там я лично участвовала в создании продукции, сборке и инвентаризации, работая в ночные и дневные смены. Считается, что ты на своей шкуре должен понять, как устроена работа на всех уровнях, и не смотреть свысока из офиса на тех, кто трудится на сборочных линиях и в цехах
\end{fancyquotes}

В период практики бывшие студенты часто живут в одном большом тренировочном центре или общежитии, готовят вместе еду, устраивают вечеринки, что помогает сблизиться и стать друзьями. Все это называется словом до:ки — люди, устроившиеся в компанию в один с тобой срок. Потом, когда всех распределят по отделам финансов, продвинутого инжиниринга, новейших разработок, продаж и так далее, у вас всегда будет кто-то знакомый из до:ки в каждом из отделов, к кому вы можете неофициально обратиться и что-то спросить — это очень удобно.

В традиционных японских компаниях сотрудники делятся на несколько категорий. Сэйсяин — постоянный сотрудник с контрактом на неограниченный срок работы. Его очень сложно уволить по желанию компании — нужно доказать, что сотрудник непригоден, хотя ему дали много возможностей пройти переобучение или проявить себя в другом отделе. Кэйяку сяин — контрактник, у которого ограничен срок работы и которого можно уволить по истечении срока контракта даже без объяснения причин. Хакэн сяин — как контрактные работники, но контракт заключается не напрямую с компанией, а с рекрутинговым агентством.

В нашей компании по цвету бейджика можно понять, кто к какой категории относится. При этом сэйсяин и кэйяку сяин могут делать совершенно одинаковую работу, но премии и социальная программа поддержки у них будут абсолютно разные. Я проработала в одной и той же компании и в том, и в другом качестве, и при подписании «постоянного» контракта мой доход возрос в 1,3 раза. Мне позволили вступить в профсоюз и пользоваться какими-то дополнительными «плюшками» вроде почти бесплатной мини-гостиницы компании в деревне горячих источников. При этом мои рабочие обязанности не поменялись вообще никак.

\textbf{О России}

О России в целом знают очень мало. Меня даже пару раз спрашивали, какой у нас государственный язык, не английский ли. Молодые люди упоминали Эрмитаж, белые ночи в Петербурге, Достоевского, балет, Владивосток — его тут в поездах рекламировали как «самую близкую дверь в Европу рядом с Японией».

\begin{fancyquotes}
    Пожилые люди могут почему-то упомянуть Горбачева, но в основном их знания ограничиваются стереотипами, что все пьют водку, в том числе молодые девушки. Также, по их мнению, в России вечная зима, поэтому русские мерзнуть не могут — мол, кожа особенная
\end{fancyquotes}

В Японии вообще иностранцев — особенно европейцев и американцев — не делят по национальности или культурному бэкграунду, записывая в одну категорию «иностранцев» — гайкокудзин. Когда у нас скажут «у меня есть друг англичанин», в Японии скажут «у меня есть друг иностранец».

Японцы ужасно удивляются, когда рассказываешь им, что в России так же, как в Японии, снимают уличную обувь при входе в дом, удивляются, что мы можем есть икру. Кстати, это слово в японском заимствовано именно из русского. Часто можно услышать вопросы вроде «А вы едите вот это за границей?». Не в России, а за границей, понимаете? Приходится объяснять, что я могу отвечать только за Россию, а едят ли это в других странах мира, не знаю.

Я не хочу загадывать на будущее, потому что никогда не знаешь, что тебе принесет завтрашний день, но в Японии я нашла любимые увлечения, дорогих мне людей, в том числе молодого человека, и работу, поэтому в ближайшее время планов уезжать у меня нет. Но по родному Петербургу и культурному досугу вроде семейных походов в филармонию и на балет я очень скучаю. Надеюсь приезжать в родную страну почаще, когда пандемия пойдет на спад. О жизни в Японии и местных маленьких особенностях я рассказываю в своем Twitter-аккаунте.

\newpage
\subsection{Великобритания}
% --- Studied
% --- Added to Anki

\textit{«Полгода я выла от отчаяния» Рассказ россиянки о непростом переезде и жизни в Великобритании }

Наталья из России никогда не задумывалась о жизни за границей. Она училась на переводчика, была увлечена английским и практиковала его с носителями языка на сайтах языкового обмена. Все изменилось, когда она по воле судьбы познакомилась с британцем и влюбилась. В рамках цикла материалов о россиянах, перебравшихся за границу, «Лента.ру» публикует ее рассказ о непростом переезде и жизни в Великобритании.

\textbf{Роковое знакомство}

Однажды мне написал сообщение молодой парень из Англии, которого звали Райан, и предложил помочь с английским. Мы долго общались как друзья, а потом поняли, что наши отношения перетекают в нечто большее. Все закрутилось, завертелось, и в какой-то момент он купил билеты и приехал ко мне в гости. Да так в России и остался!

Изначально в наши планы не входил переезд в Великобританию. Мне нужно было доучиваться, да и работа уже была неплохая. Райан мечтал преподавать английский в чужой стране и тут же начал воплощать это в реальность. Сначала нас все устраивало.

За полгода мы проверили отношения на прочность и поняли, что идеально сходимся. Мы начали задумываться о планах на будущее. Райан не захотел навсегда оставаться в России — ему тяжело давался русский язык, и он понимал, что на родине может добиться большего. Я не хотела резких перемен, но сдалась.

\textbf{Тяжелый переезд}

Настала его очередь приглашать меня в свою страну. Я сделала гостевую визу и сорвалась к своему джентльмену. Месяцем позже, во время прогулки у Тауэрского моста, он сделал мне предложение руки и сердца. После этого мы задумались о том, как перевезти меня в Великобританию на ПМЖ.

Все оказалось гораздо тяжелее, чем мы думали. Условия визы невесты казались невыполнимыми, поэтому мы поженились в России. После свадьбы для нас наступил сложный период бюрократической волокиты, продолжавшийся целый год, так как условия визы жены были еще замороченнее.

Райану требовалось найти работу с определенным доходом и продержаться на ней как минимум полгода. Для новоиспеченного выпускника вуза это не так уж просто, но он справился. Мне нужно было пройти тест на туберкулез в специальном центре и сдать экзамен по английскому языку. Каждый раз приходилось ездить за тридевять земель.

Понадобились доказательства наших отношений. Переписки, скрины звонков, фотографии, билеты. В общем, тотальный контроль. Но без этого было не обойтись — слишком много в стране фиктивных браков ради гражданства. После всего этого бумажного ада я все-таки получила свою первую визу жены.

\textbf{Три стадии адаптации}

В Англию я влюбилась сразу. Мне нравилась британская вежливость и легкость в общении, старинная архитектура и зеленые парки, даже традиционную еду я оценила по достоинству! Именно так я и представляла Туманный Альбион, о котором столько читала в учебниках по английскому. До сих пор помню то состояние эйфории.

Мне повезло, и в первую же неделю после переезда по визе жены я вышла на работу по специальности — устроилась переводчиком. В маленьком пригороде Лондона это было большой удачей. Я до сих пор работаю в этой компании, оспаривая фразу: «Там вас никто не ждет». На работе меня, как и всех моих коллег из других стран, уважают, а мой труд ценят.

\begin{fancyquotes}
    После стадии эйфории началась вторая стадия — фрустрация. Эти фазы свойственны всем иммигрантам, как я позже узнала
\end{fancyquotes}

Несмотря на это, адаптироваться оказалось непросто. Мне было все не так. Медицина — мрак, к врачам не попасть, да и им на тебя плевать. Люди оказались открытыми только снаружи, а дружбы ни с кем не построить. Закрывается все рано, после пяти вечера город спит.

Около полугода я выла от отчаяния. Меня даже начал раздражать их акцент. Когда слышала за углом русскую речь, хотелось бежать навстречу. Но постепенно я начала ко всему привыкать: подступила к третьей стадии иммиграции — принятие.

\textbf{Минус в плюс}
Всем тяжело выходить из зоны комфорта, сейчас я это понимаю. Сначала я была слепым котенком и не знала, как тут все устроено, куда можно пойти, не понимала негласных правил общения с местными. Только спустя полгода-год я начала более-менее ориентироваться в новой среде.

Постепенно все минусы, которые меня вгоняли в депрессию, стали превращаться в плюсы. Да, многие магазины закрываются довольно рано. Но перестроиться на новый график недолго. Зато сколько разных магазинов здесь можно найти! Да и заказываю я уже все в онлайне — от продуктов и до одежды. Пара дней ожидания — и вуаля! Все, что пожелал, — у тебя на пороге.

Рестораны и кафе — отдельная тема. В Лондоне и крупных городах можно найти рестораны любой кухни. Я распробовала популярные здесь индийскую и китайскую кухни и сейчас не могу без них жить. Британские традиционные блюда в хороших пабах тоже очень даже неплохие.

\begin{fancyquotes}
    Кстати, совет, если окажетесь в Великобритании, пробуйте традиционный фиш-н-чипс именно в прибрежных городах. Там свежая вкуснейшая рыба. Да и другие морепродукты тоже, самые лучшие устрицы — у нас!
\end{fancyquotes}


\textbf{Дружба с британцами — на всю жизнь}

С самого начала я общалась только с британцами: с семьей мужа, его друзьями, девушками в студии танцев и со своими коллегами. Русских в нашем маленьком пригороде я найти не могла, а ведь даже искала первое время. Думаю, тот факт, что в моем окружении были исключительно местные, помог мне адаптироваться быстрее, хоть сначала было и невыносимо.

Со временем я начала понимать нюансы общения с британцами и перенимать их. У них принято быть очень вежливыми, например, не использовать резкие фразы, а увиливать. Наше «нет» они воспримут как грубость. У них считается нормой саркастично шутить друг над другом, и здесь уже мы, русские, можем обидеться на их подколы. Хотя все, что надо сделать, — подколоть в ответ и вместе посмеяться! Сейчас мне эти колкости кажутся безобидными и очень даже веселыми.

Что касается дружбы, то с русскими сдружиться легче. Если в России можно поговорить с незнакомцем по душам после десяти минут знакомства, то к британской душе подступиться тяжелее. Они дружелюбны, улыбчивы, но общение очень долгое время остается поверхностным. Все держится на уровне вежливых фраз: «Как дела? Сегодня такая дождливая погода. Какие планы на выходные?» Только через два года я начала теснее общаться со своими британскими знакомыми.

\begin{fancyquotes}
    Пусть получить статус друга в Англии тяжело, если вы все же сдружились — это на всю жизнь
\end{fancyquotes}

Британцы — люди толерантные и притеснять вас только из-за национальности не будут. Правда, местные не любят тех, кто отказывается адаптироваться и лезет в чужой монастырь со своим уставом. Но таких в любой стране не любят, так ведь? Если вы вежливы, приветливы и миролюбивы, вас примут как своего. Даже с жестким русским акцентом. Конечно, могут по традиции «стереотипно» подколоть про водку или вечную мерзлоту, но обижаться на это не стоит.

Толерантность британцев мне нравится. Им нет дела до жизни чужих людей, если те никому не мешают; им все равно, как ты выглядишь, во что одет, они не докапываются и редко осуждают. По моим ощущениям, здесь меньше резкости, злобы на что-то, что их не касается, да и негатива вообще.

\textbf{Партнерские отношения}

Мужчины-британцы редко оплачивают все полностью. Сразу отмечу, что во время декрета они, конечно, обеспечивают своих жен. А жена помогает мужу, если он вдруг потерял работу. В этом суть партнерских отношений, которые все популярнее в Англии. Обычно люди либо создают общий бюджет, либо делят счета.

Нам ближе вариант с общей копилкой. Мы делим и доход, и домашние обязанности. Все честно, и все счастливы. Меня с самого начала привлекал именно такой подход к совместной жизни.

Про цены, думаю, говорить особого смысла нет. Жилье, коммуналка, транспорт и услуги тут не самые дешевые. Но это восполняется высокими зарплатами. Цены на еду примерно на том же уровне, что и в России, а одежда и другие мелочи стоят дешевле. В целом у меня тут остается намного больше денег от зарплаты, чем оставалось, когда я работала на родине на той же должности.

А вот о местной медицине иммигранты спорят постоянно. Начнем с того, что здесь не залечивают. Если у вас обычная простуда, то вам просто скажут идти домой и пить горячий чай. В таком подходе есть свои плюсы. Если у вас что-то серьезное, то все будет зависеть от доктора.

Мне попадались и те, кто на явную проблему махал рукой, и те, кто из кожи вон лез, чтобы понять, что со мной произошло. Но так было и в России — тоже по-разному. Поэтому тут все довольно неоднозначно.

\textbf{О путешествиях, традициях и самобытности местных жителей}

Один из важнейших плюсов жизни в этой стране для меня — возможность свободно путешествовать. В любую страну Европы можно долететь дешево, быстро и без виз (если уже есть британский паспорт). По самой Великобритании путешествовать тоже удовольствие — местные сохранили огромное количество замков, дворцов, поместий, да и их деревенские домики стоят веками и не меняют внешний вид. Англичане ценят и хранят свои традиции.

Британцы любят собираться на традиционный полуденный чай, говорить о королевской семье, которая тоже стоит во главе страны исключительно ради поддержания традиции. Их любимое место — старинные пабы, где по сей день пьют горький эль, закусывая традиционными свиными шкварками. Многим в Великобритании нравится эта самобытность.

Несколько месяцев назад я получила британское гражданство и паспорт. Это был тяжелый процесс — нужно было пройти через две визы жены, получение ПМЖ и натурализацию. Каждый процесс сопровождался длиннющим списком условий и ценником, от которого волосы встают дыбом.

На весь иммиграционный процесс мы потратили около десяти тысяч фунтов стерлингов (почти миллион рублей). Ставки на иммиграционные сборы в этой стране — одни из самых высоких. Но для нас все это, наконец, позади.

\textbf{Великобритания — новый дом}

После семи лет в Великобритании я могу уверенно сказать, что полностью адаптировалась и люблю это место. Тут я уже четверть жизни и смело называю эту страну своим домом. Конечно, как и везде, здесь есть недостатки. Но я акцентирую внимание только на положительных вещах, которых тоже много. Да и вообще, я всегда стараюсь идти по жизни с поднятой головой!

Скучаю ли по России? Наверное, уже нет. Я слишком привыкла к своему быту здесь. В Англии моя семья, мой дом, мои друзья, моя работа и мое увлечение. В России остались родные, и по ним я скучаю, но стараюсь их навещать.

Мне просто некогда грустить: я работаю в офисе, преподаю английский и русский как иностранный, занимаюсь танцами, учу испанский в разговорных клубах, хожу с подругами на полуденный чай и стараюсь каждые выходные проводить в новом английском городке. Путешествия — наше с Райаном основное хобби. Я прошла все психологические фазы иммиграции и, наконец, счастлива как никогда!

\newpage
\subsection{Арабские Эмираты}
% --- Studied
% --- Added to Anki

\textit{«Это ли не сказка?» История россиянки, которая лишилась работы и переехала в Арабские Эмираты}

\textit{Александра из Москвы не верила, что когда-нибудь побывает в Дубае, и уж тем более не мечтала о переезде туда. Однако пандемия все изменила: они с мужем лишились работы дома, и им пришлось перебраться в Объединенные Арабские Эмираты. В рамках цикла материалов о россиянах за границей «Лента.ру» публикует рассказ Александры о жизни в ОАЭ.}

Еще лет шесть назад я смотрела фотографии одноклассниц в Instagram из Дубая и думала: «Вау, как красиво и богато». Тогда я была уверена, что не смогу тут побывать хотя бы раз. Кто бы мог подумать, что судьба сложится так, что спустя пять лет мы с мужем и ребенком сюда переедем?

\textbf{Быстрый переезд}

Мы с мужем жили и работали в Москве, у нас обоих был бизнес: у него массажные кабинеты, у меня студия красоты. Потом я забеременела, и случилась пандемия. Мне кажется, в те времена паника накрыла всех. Работать мы не могли, все закрыли.

И тут мужа приглашают на работу в Дубай. Он улетел на месяц один. С работой все сложилось, и встал вопрос о переезде. Когда он вернулся в Москву, я родила. Мы быстро сделали загранпаспорт дочке, и муж полетел обратно. За месяц он должен был найти для нас жилье и арендовать машину. А передо мной стояла непростая задача: перелет с двухмесячным ребенком в неизвестную страну. Благо моя мама согласилась лететь со мной и помогала мне.

Прилетев сюда из холодной осенней России, я, конечно, была в шоке. Эти огромные стеклянные высотки, везде пальмы. Больше всего в моей голове не укладывалось, что этой стране всего 49 лет.

Первое жилье мы сняли в районе Дубай Марина. Довольно туристическое и популярное место, там все еще и «на спорте». В каждом доме есть свой бассейн, свой спортзал, до пляжа рукой подать. Это ли не сказка?

\begin{fancyquotes}
    К сожалению, за год жизни здесь большим количеством друзей мне обзавестись не удалось. И так как муж все время на работе, а я дома с ребенком, я чувствую себя немного одиноко

    \begin{flushright}
        Александра
    \end{flushright}
\end{fancyquotes}

\textbf{Быт и отдых местных жителей}

Дубай — город, куда все приезжают работать, зарабатывать и жить для себя. В выходные дни, здесь это пятница и суббота, все гуляют по торговым центрам, барам, ночным клубам. Арабы в выходные любят выезжать на пляж на целый день большими семьями: бабушки, дедушки, мамы, папы, дети. Они возят с собой ковры, столики, складные диванчики, все это раскладывают на песке, накрывают стол и устраивают этакий пикник у воды. Также они любят парки аттракционов, типа тематического парка в стиле Голливуда Motiongate или парка с 90 павильонами, посвященными разным странам мира, Global Village.

Местные жители очень активно пользуются услугами клининга, а еще больше — услугами нянь. Такого понятия как «декретный отпуск» здесь нет. После рождения ребенка мама, если она работает, должна выйти на работу через три месяца. Поэтому здесь нанимают нянь, либо отдают детей в аналог яслей, Nursery, для совсем маленьких деток. На детских площадках 90 процентов детей гуляют с нянями.

Что-то вроде «материнского капитала» выплачивается только местным гражданам и потом им же платится ежемесячное пособие на детей до 18 лет, но там немного, около 150 долларов (11 тысяч рублей). Если иностранная гражданка родила здесь, то гражданство ребенку все равно никто не даст. Более того, если в семье «араб — иностранная гражданка» рождается ребенок, только отец решает, какое гражданство он получит — арабских эмиратов или же по матери.

Роды здесь платные, кесарево сечение можно выбрать не только по медицинским показаниям, но и просто по желанию. 95 процентов местных женщин именно этот способ родоразрешения и выбирают, чтобы «не напрягаться».

\textbf{Медицина, магазины, аренда жилья и городской транспорт}

Мы с дочкой проживаем по туристической визе, каждые три месяца продлеваем ее за деньги. Поэтому и страховка у нас — туристическая. С таким типом страховки больницы связываться особо не хотят, поэтому болеть здесь очень накладно. Однажды нам пришлось обратиться к педиатру, и один прием нам обошелся в 450 дирхам (около девяти тысяч рублей).


Здесь очень любят бюрократию. Например, чтобы арендовать квартиру, приобрести сим-карту, а также открыть счет в банке — у вас должно быть личное удостоверение Emirates ID. Когда арендуешь квартиру, оформляется EJARI — это официальный государственный документ, свидетельствующий о том, что ты теперь живешь в квартире, и все счета будут приходить на твое имя.

Здесь нельзя, как в России — отдал деньги арендодателю и живешь. Когда EJARI готов, заключается отдельный договор с каждой службой, предоставляющей определенную коммунальную услугу: электричество, воду, газ и систему кондиционирования. Что интересно, в России при строительстве домов устанавливаются отопительные системы, а здесь сплит-системы — состоящие из двух блоков кондиционеры, которые служат как для охлаждения воздуха, так и для нагрева. Совсем холодной воды в домах тоже нет.

Еще один момент, который меня после России просто поверг в ступор, — интернет. Точнее, его дороговизна и недоступность.

\begin{fancyquotes}
    Обычный Wi-Fi дома обойдется вам в 400 дирхам (восемь тысяч рублей)! Да-да, именно в восемь тысяч рублей, и это еще по-божески. А за сим-карту с интернетом вы будете платить около 300-500 дирхам (от шести до десяти тысяч рублей). Поэтому, когда вы будете думать, что тысяча рублей за телефон каждый месяц — это дорого, вспомните о моих словах

    \begin{flushright}
        Александра
    \end{flushright}
\end{fancyquotes}

В городе есть общественный транспорт: метро, автобусы, трамваи. Трамваи ходят только на очень маленькие расстояния — буквально пару улиц и есть всего в паре районов. В метро несколько веток, расстояния между станциями довольно большие. Кстати, поезда в метро и трамваи ездят без водителей, поэтому очень интересно прокатиться в первом вагоне. Автобусы ездят по всему городу по расписанию. Но расстояния между остановками тоже довольно большие, поэтому дорога своим ходом может занять немало времени.

Такси здесь дорогое: поездка в одну сторону (около 20 минут езды) будет стоить вам около 60-70 дирхам (от 1,2 до 1,4 тысячи рублей). Но надо отдать должное, все таксисты работают официально, везде есть счетчики, можно оплатить картой и даже есть Wi-Fi. Водители всегда в рубашках и помогают открыть и закрыть двери, вытащить коляску и сумки.

В продуктовых магазинах тоже все очень клиентоориентировано. Овощи и фрукты всегда взвешивают сотрудники. В мясном отделе вам бесплатно разделают выбранный кусок говядины. Интересный факт: здесь вы с трудом найдете свинину, но практически во всех мясных лавках встретите мясо верблюда. На кассе специальный сотрудник соберет вам продукты в пакеты. Кстати, пакетов очень много, раскладывают чуть ли не каждую коробочку в отдельный пакетик. А еще пакеты здесь бесплатные.

\begin{fancyquotes}
    Алкоголь продается только в специальных магазинах, их немного, но они раскиданы по всему городу. В основном они находятся на парковках у торговых центров. Никаких вывесок, реклам или указателей на эти магазины вы не увидите. Поэтому найти их крайне сложно, только по чьей-то наводке. И цены на алкоголь здесь в два раза выше, чем в Москве

    \begin{flushright}
        Александра
    \end{flushright}
\end{fancyquotes}

Местное население алкоголь покупать практически не может: только местные мужчины со специальной лицензией, женщинам же это запрещено совсем. Туристам продают спиртное только по паспорту и лицензии (30-дневной или полученной от местного работодателя), при этом надо быть старше 21 года и следовать определенным правилам: нельзя пить в общественных местах, кроме баров и ресторанов, появляться на публике в нетрезвом виде и водить в таком состоянии автомобиль.

\textbf{Отношение к русским и детям}

Когда говоришь местным жителям, что ты из России, сразу все отвечают: «О, Путин!» Потом некоторые начинают рассказывать, что тоже были в России. Но для них Россия — это и Россия, и Казахстан, и Узбекистан, и Грузия. Вообще, все очень позитивно и дружелюбно относятся к русским, негатива в свой адрес мы не встречали.

А как здесь любят детей! Из-за того что все карьеристы, дети — это уже роскошь. Поэтому детей обожают. Каждый машет рукой, улыбается и даже пытается потрогать мою дочь, когда мы куда-то идем. А некоторые пакистанцы и индийцы просили сфотографироваться с ней или подержать за руку. Кто-то просто самовольно фотографирует ее. К этому мне было тяжело привыкнуть.

Про Эмираты ходит много слухов и существует много разных мнений. К примеру, что здесь при рождении каждому гражданину государство открывает счет, на котором до совершеннолетия копятся деньги. Это не так. Домыслы, что ОАЭ живут только за счет нефти — тоже неправда! Изначально страна разбогатела из-за добычи и продажи жемчуга. Здесь до сих пор почитаются искатели жемчуга (например, в Дубай Молл им посвящен целый фонтан).

Также я была удивлена, что Дубай принадлежит по сути не только арабам, но и индусам. Можно сказать, что число представителей обеих национальностей, проживающих в городе, примерно одинаковое. Очень много индийских продуктов, магазинов, вообще рынок на них очень ориентирован.

\textbf{Мусульманские традиции}

Моя подруга, когда приехала ко мне, боялась жевать жвачку и ходить в шортах, якобы здесь это запрещено. Но нет, Дубай — туристический город и максимально европеизированный. Здесь не все так строго, как нам рассказывают по телевизору.

Да, совсем наглеть не стоит и уважать традиции и веру местных нужно. Но Дубай и Арабские Эмираты в целом идут в ногу со временем, а в чем-то даже опережают другие страны.

\begin{fancyquotes}
    Наиболее устаревшие законы постепенно отменяются: не так давно мусульманским женщинам разрешили водить машину, а еще разрешили совместно жить неженатым парам — раньше сожительство преследовалось по закону

    \begin{flushright}
        Александра
    \end{flushright}
\end{fancyquotes}

Однако запреты все равно есть. Например, в государственные учреждения нельзя входить женщинам с открытыми плечами и в юбке или шортах выше колена. Запрещено открыто проявлять чувства на улице: нельзя много целоваться и слишком открыто трогать друг друга.

Запрещено фотографировать других людей без их согласия. К примеру, однажды я сфотографировала очередь в молле, и охранник сразу подошел, вежливо попросил удалить фотографию и проследил, чтобы я точно это сделала. Нельзя трогать и подходить к людям, которые молятся. Не все успевают оказаться в мечети во время молитвы, некоторые молятся на улицах, площадях и так далее. Еще из-за пандемии сейчас везде запрещено танцевать.

В целом нам здесь нравится. Я даже завела блог в Instagram, в котором рассказываю про жизнь в Дубае. Это чистый и светлый город с огромными возможностями работать и зарабатывать. Скучаю я здесь только по родственникам. Но у меня такие классные друзья и семья, что они уже несколько раз смогли прилететь к нам в гости!

Да, здесь есть свои особенности. Да, нужно время, чтобы обзавестись новыми знакомствами и влиться в эту атмосферу. Но мы молоды, полны жизни и энергии. Когда нам пробовать, если не сейчас?


\newpage
\subsection{США и Канада}
% --- Studied on 23 and 25 Feb 2023
% --- Added to Anki

\textit{«Стыдно за Родину не было» История россиянина, который уехал в Америку за мечтой, вернулся спустя 25 лет и не пожалел}


\textit{Владимир Томшин четверть века прожил в США и Канаде ради близости к легендам тяжелого рока. Сложно ли было на такое решиться, авантюра это или полезный опыт, и почему спустя 25 лет музыкант принял решение вернуться в Россию? В рамках цикла материалов о россиянах за границей «Лента.ру» публикует рассказ Владимира о жизни в Северной Америке.}

\textbf{В погоне за мечтой}
В начале 90-х я был увлечен тяжелым роком. Играл на гитаре в нескольких группах, которые придерживались этого стиля. На тот момент нашими кумирами были Metallica, Megadeth, Pantera, Slayer. А их родина, как известно, Северная Америка. Поэтому у меня возникло стремление поехать туда, чтобы реализовать свои музыкальные способности и умения. Сказано — сделано. Долго ждать не пришлось, визу в США я получил без особого труда.

\begin{fancyquotes}
    На тот момент у меня, казалось бы, было все, что нужно для счастья: группа, в которой я играл, любимая девушка, работа, семья и интересное хобби. Однако желание стать рок-звездой, «играть музыку» на Западе взяли свое
\end{fancyquotes}

Оставив все, я рванул через Атлантику, как мне казалось, на разведку. Думал, что посмотрю, поиграю и вернусь. Однако судьба распорядилась иначе. После шести месяцев жизни в США я узнал, что моя группа распалась, а девушка не дождалась. А вот в Америке мне очень понравилось, и я решил остаться.


\textbf{Неидеальная Америка}

Америку нам всегда рисовали сказочной страной с небоскребами, красивыми мотоциклами и машинами, ровными дорогами и денежными купюрами — всемогущими долларами. Я был готов удивляться и погружаться в незнакомый образ жизни. Шаг за шагом познавал новый для меня континент, людей, их стиль жизни. Отчасти даже пытался подражать им.

Некоторые стереотипы оказались правдой. Улыбчивые люди, полки, полные красивых вещей, супермаркеты, похожие на музеи, добродушные полицейские, машина или две в каждой семье, дома с бассейнами, наличие оружия у населения — вот она красивая жизнь и успех. Для человека, приехавшего из России 90-х, это было похоже на сказку.

\begin{fancyquotes}
    Однако и разочарования не заставили себя долго ждать. Контраст богатства и нищеты в Америке разительный. Пожив немного в этой сказке, ты начинаешь понимать, что попал в трудовой лагерь с усиленным питанием
\end{fancyquotes}

Чтобы держаться на поверхности, надо работать денно и нощно. Большинство моих знакомых имели по две, а то и по три работы. Тебе некогда наслаждаться жизнью, ты постоянно работаешь (не считая 14 дней годового отпуска), потому что надо платить по счетам.

Главным для меня как музыканта было то, что я приехал в Америку играть музыку. Но внезапно понял, что мне некогда этим заниматься, ведь надо беспрестанно работать. Музыка может тебя содержать только в том случае, если ты уже достаточно известен, а это достигается годами. Плюс колоссальная конкуренция. Ведь в каждом доме помимо оружия и машин есть еще и гитара. Музыкантов разного уровня в Америке больше, чем можно себе представить.

\textbf{Новый \explain{поворот}{turn (n.)}}

Недолго пожив в США, я уехал в Канаду. Почему? Да потому что в США считается, что в Канаде жизнь легче. Социальное обеспечение в этой стране действительно лучше, есть почти бесплатная медицина. Но реальность, как обычно, заметно отличалась от мифа.

Первое время, до получения вида на жительство, было нелегко. На высокооплачиваемую работу рассчитывать не стоило. Преодолевать языковой барьер приходилось при каждом разговоре. Спасало то, что Монреаль, где я жил, находится во франкоязычной провинции Квебек. Там английский — второй язык по значимости, и местные жители знают его похуже, чем в других провинциях. Это дало время подтянуть языковые навыки.

\begin{fancyquotes}
    Канада и США — страны иммигрантов. Крупные города разбиты на районы по национальному признаку: французский, итальянский, еврейский, греческий, арабский…
\end{fancyquotes}

Монреаль — не исключение, хотя есть и районы, где совершенно не имеет значения, какого цвета у тебя кожа или какого ты вероисповедания. Большинство граждан ведут себя достаточно мирно. Открытую агрессию я встречал всего пару раз за 16 лет жизни — и то не от местных.


\textbf{Ностальгия все же не миф}

Я много слышал и читал о ностальгии, но не думал, что испытаю это сам. В свободное время я стал больше общаться с нашими иммигрантами. Местные жители — отличные ребята, но друзьями стать не получилось.

Есть мнение, что стать «своим» в другой стране можно, только если попал туда до 20 лет и учился в местной школе и университете. Тогда есть шанс впитать менталитет и завести друзей. А я приехал туда в 23 года и переучиваться не стал. Зачем, если у меня за плечами Уральский государственный технический университет? Я до сих пор считаю, что советское образование было лучшим в мире.

Познакомился с местными музыкантами и вспомнил, что приехал сюда играть музыку. А на нее времени не хватает, потому что жизнь превратилась в рутину.

\begin{fancyquotes}
    Некогда удивительные вещи превратились в обыденность. Деньги оказались просто деньгами, а не манной небесной. Их, как и на Родине, надо зарабатывать
\end{fancyquotes}

А какие-то вещи стали даже раздражать, например, вечный ремонт дорог. Из-за него Монреаль с каждым годом стал все больше походить на полосу препятствий. Невольно начинаешь проводить параллели между нашими мирами и культурами. Со временем стало намного приятнее выражать свои мысли на великом и могучем русском языке, чем пытаться объяснять что-то на английском или французском.

\textbf{Хлеб и зрелища}\footnote{Хлеб и зрелища: Bread and spectacles}

В обычных маркетах еда безвкусная. Я ее называю «пластиковой», настоящий вкус почти отсутствует. Молоко пить невозможно, зелень напоминает обычную траву, мясо хоть и свежее, но неестественно пресное, хлеб похож на вату. Перед возвращением в Россию, несколько лет я старался покупать продукты в русских, китайских и арабских магазинах, а также у местных фермеров на единственном на полуторамиллионный город рынке.

В русских магазинах можно купить такие деликатесы, как соленые огурчики и селедку, пельмени и хлеб домашней выпечки, побаловать себя салатом оливье или селедкой под шубой. А еще приобрести красную икру. Рыбу ловят в местных реках, но канадцы не понимают вкус икры. Сетевые маркеты ее не закупают, ведь она не пользуется широким спросом.

Рестораны не блещут разнообразием меню. Основное блюдо — стейк с овощами или итальянской пастой. Очень редко в заведениях можно встретить супы. Три-четыре салата — уже богатый выбор. Исключения составляют русские рестораны. Там выбор в два раза больше по определению. Естественно, есть и традиционные блюда, такие как борщ, рассольник, пельмени или курица с грибами. Хотя, вернувшись в Москву, я понял, что еда — не такой уж крутой показатель.

Развлечения в Канаде разные — на любой вкус. Одна поездка на Ниагарский водопад дорогого стоит. Это увлекательное и незабываемое путешествие. Кроме посещения смотровой площадки есть возможность подойти на корабле к падающей воде и оказаться в пелене брызг. Еще можно подняться на воздушном шаре или облететь водопад на вертолете, а также посмотреть на это чудо природы изнутри, из-за стены.

Есть яхт-клубы, лыжные спуски, оздоровительные комплексы с саунами и бассейнами и много чего еще. Мы частенько снимали шале в лесу у одного из озер, коих в Стране кленового листа великое множество. Там мы вспоминали свое пионерское прошлое, жгли костры, пели песни под гитару, ходили на каноэ, рыбачили.

\begin{fancyquotes}
    Есть и другого рода развлечения: казино, стрип-клубы, дискотеки. На одной из них я подрабатывал вышибалой, почти как в фильме «Дом у дороги»
\end{fancyquotes}

Летом проводится множество фестивалей. Для представителей ЛГБТ есть целый квартал.

Метро и автобусы ходят исправно, как часы. Правда, порой их работники бастуют. Я был удивлен, узнав о специальных ночных маршрутах. Весьма удобно, если возвращаешься домой поздно и без машины, потому что такси — очень дорогой транспорт. Только посадка в машину стоит 3,5 доллара, а минимальная поездка (два километра) обойдется в 20 баксов.

Что касается жилья, то в Канаде вы либо покупаете недвижимость в ипотеку, либо снимаете. Второе — достаточно удобно, все условия прописываются в договоре и четко соблюдаются арендодателем. Покупать жилье в большом городе нерентабельно. Особенно, если живешь один или не любишь сидеть на одном месте. Так, мне удалось пожить и в пентхаусе, и в доме на берегу озера. За квартиру обычно платят отдельно от коммунальных услуг, а за электричество выставляется специальный счет. Сумма зависит от отопления и подачи горячей воды.

Несоизмеримо дорогая и ненадежная мобильная связь. Только выехал за пределы города — сразу все звонки, в том числе и входящие, оплачиваются по повышенному тарифу. Несмотря на высокие зарплаты, практически все стоит гораздо дороже относительно российских цен. Я пришел к выводу, что у нас цены вполне соответствуют качеству, что не всегда можно сказать о Канаде.

Мне приходилось обращаться к канадской медицине несколько раз, в том числе к дантистам (это отдельная статья расходов). В Канаде медицина существует на налоги граждан. Меня обслужили вполне достойно, я не жалуюсь. Хотя от друзей слышал неприятные истории про долгое ожидание помощи и не лучшее лечение. Фармацевтический бизнес очень развит: людей пичкают всеми возможными и невозможными лекарствами. Сначала выписывают таблетки от болезни (они тоже недешевые), а потом — лекарства от побочных эффектов, вызванных первыми. Многие так и живут.

\textbf{Общение с местными}

Местные жители — вполне нормальные люди, со своими плюсами и минусами. Никто не относился ко мне отрицательно или враждебно. Могу сказать, что в Канаде более мирное население, чем в США. У тех — извечный расовый вопрос всегда стоял ребром. Ко мне тоже относились с подозрением, пока не узнавали, что я русский. Тогда напряжение спадало само собой. В Квебеке франкофоны более терпимы к иммигрантам, чем к англоговорящим канадцам.

Естественно, было много вопросов о нашей стране. Удивление вызывало то, что в России не везде холодно, а погода в Москве очень схожа с монреальской. Встречались и те, кто уже ездил в нашу страну туристом и достаточно лестно о ней высказывался. По крайней мере, стыдно за Родину не было.

\begin{fancyquotes}
    Их главный стереотип о нас, я уверен, создала телепропаганда. То, что мы угрюмые и злые люди, поголовно состоим в русской мафии, а наши девушки самые красивые
\end{fancyquotes}

Все эти мифы были развенчаны после общения с нашими ребятами и со мной лично. Кроме последнего — тут не поспоришь. Наши девушки были и остаются на пьедестале красоты.

Трудно сказать, какие мои стереотипы о канадцах были разрушены. Я прожил там достаточно долго. Часть канадского менталитета укоренилась и во мне. Уже здесь, в России, мои друзья и знакомые отмечали разницу в мышлении и поведении. Некоторые даже акцент улавливают, хотя это уже происходит реже, чем раньше. Иногда бывает трудно подобрать русские слова, в то время как их английские аналоги вертятся на языке.

\textbf{Новые цели и планы}

В ближайших планах — записать четвертый студийный альбом «Томшин Бэнд» и акустический сборник, снять еще один клип и проехать по всей России с гастрольным туром. Творческих идей очень много. Главная цель на новом витке жизненной спирали — донести до слушателя мои песни. Я ведь и в Канаду поехал благодаря любви к музыке. А на Родину вернулся, когда понял, что мой слушатель именно здесь.

Канада — замечательная страна, но я артист, которому нужна своя публика. Большинство моих песен на русском языке. Важно, когда в зале дословно понимают то, что ты хочешь рассказать (в моем случае — спеть). На английском песни тоже есть, но он для меня не так органичен.

\begin{fancyquotes}
    В Канаде у меня осталось немало друзей. Могу сказать, что скучаю по ним и нашему общению. Так что ностальгия у меня теперь двойная
\end{fancyquotes}

Хотелось бы также съездить в Северную Америку с гастролями, показать, насколько изменилось и «выросло» мое творчество. Конечно, видео с выступлений доступны в интернете, но живой концерт — это особая энергетика.

В любом случае я благодарен судьбе за возможность получить бесценный опыт долгосрочного проживания в Канаде. Но рад и тому, что осознание настоящего предназначения вернуло меня на Родину.


\newpage
\subsection{Болгария}
% --- Studied on 25 Feb 2023
% --- Added to Anki

\textit{«Здесь люди уважают себя и свой труд» История россиянки, которая переехала с тремя детьми на родину мужа — в Болгарию}

\textit{Два года назад Наталья перебралась из России в Болгарию — на родину своего мужа. За это время она обжилась в стране, где до того бывала лишь в качестве туристки, и увидела ее с новой стороны. В рамках цикла материалов о россиянах, перебравшихся за границу, «Лента.ру» публикует ее историю.}

Мой муж — наполовину болгарин, и поскольку его родители работали в разных странах, он в разное время жил в Москве, хорватском Загребе и болгарской Софии. Он учился в Софийском университете, затем перевелся в Московский университет, где мы с ним и познакомились. Мы поженились и жили в Москве. У нас родились три ребенка с маленькой разницей в возрасте, мы работали в хороших местах и каждое лето ездили в отпуск на море в Болгарию к родственникам мужа в небольшой город Ямбол.

Муж всегда хотел вернуться обратно в Софию, город своей молодости, где ему было комфортнее, чем в Москве. Он жил в столице во многом ради меня, потому что я считала, что в России мы добьемся большего и лучше обеспечим детей. К тому же мы не знали, сможем ли устроиться в Болгарии. В маленьком приморском Созополе, где мы были летом, работы вообще не было, и родственникам мужа из Ямбола приходилось работать в Германии.

Несколько лет назад мы съездили в отпуск не на море, как всегда, а в горы и в Софию, поговорили со школьными и университетскими друзьями моего мужа, которые там учатся и работают, и поняли, что мы тоже можем попробовать переехать в Болгарию. Тем более, что старший ребенок подрос, пришло время отдавать его в первый класс и выбирать между российской и болгарской системами образования.

Я знала, как мягко и терпеливо болгары относятся к детям, и хотела выбрать болгарское образование. Тем более, что умному ребенку в этой стране легко поступить в хороший университет. Кроме того, мы с мужем решили, что в Болгарии безопаснее, чем в России. В этой стране детей отпускают в школу без взрослых и разрешают самим ездить на общественном транспорте. В итоге мы решили переехать в Болгарию.

Путь получения ПМЖ и потом гражданства в Болгарии довольно долог: сначала нужно пять лет жить с видом на жительство и продлевать его раз в год. Нужно готовить много документов. Муж должен нотариально заверить текст, согласно которому он готов содержать жену, а его брак не фиктивен. Собственник жилья — что готов его сдавать. Еще нужен договор с работы или счет в банке с 12 минимальными окладами и страховка. Подготовить их не так уж сложно, проблема в том, что у сотрудников миграционных служб разные представления о том, что именно нужно. Они могут просить добавить то один, то другой документ, хотя сами видели наших детей и точно знают, что брак не фиктивный.

Нас несколько раз приглашали на собеседование, где мы заполняли большие анкеты о качествах супруга. Это может немного затянуть получение вида на жительство, который в первый раз дается на основе визы D. Получать эту визу надо в московских консульствах, и это совсем не сложно. Главное, чтобы был работающий болгарский телефонный номер и номер собственника жилья, в котором будете жить, потому что сотрудник миграционной службы может на него позвонить.

Хотя Болгария — это совсем не Франция и не Италия, куда можно мечтать поехать в романтический отпуск, я всегда любила эту страну. Это была первая зарубежная страна, в которой я побывала. Тогда мне было 14, мы с сестрой ездили отдыхать в лагерь в Золотых Песках. Мы жили в обшарпанном советском пансионате со сломанными замками на дверях, куда надо было подниматься от пляжа по длинной вертикальной лестнице. Но сам курорт был очень ухоженным, а в соседнем городе Балчике были такие милые болгарские домики с черепичными крышами...

\begin{fancyquotes}
    Мы с сестрой, две маленькие девочки, могли спокойно сами пойти в кафе и поесть там. Все это давало какую-то радость жизни и свободу, которую я до сих пор ощущаю в этой стране
\end{fancyquotes}

Возможно, поэтому, когда мы только переехали, я и ощущала эту эйфорию, хотя объективно нам было непросто. За три дня на машине проехать пять стран с тремя маленькими детьми, при этом вовремя подключаясь к вайфаю в гостинице, чтобы успеть на наши с мужем удаленные работы. Сперва остановились в маленьком горном курорте Сандански, где муж мечтал немного пожить после огромной, по его меркам, Москвы. Затем поехали в провинциальный Ямбол, где жили около цыганского квартала и работали удаленно практически каждый день.

Приходилось постоянно преодолевать какие-то препятствия: разбитую фару Chevrolet заказывали аж из Москвы, миграционная служба требовала новые и новые документы, потом мы переезжали из Ямбола в Софию, и нужно было искать школу и сад для детей. Но это казалось чем-то естественным, необходимым и неизбежным при эмиграции.


Изначально мы приехали, имея две удаленные работы. В Софии муж сразу же нашел хорошую по здешним меркам работу. Но у него очень хороший уровень английского, русского и болгарского, к тому же в Москве он работал руководителем отдела. Я примерно через год устроилась на работу по рекомендации мужа, но через полгода меня уволили из-за не поданных вовремя документов для продления договора. Честно говоря, из-за слабого знания английского мои показатели в работе были настолько низкими, что рано или поздно меня все равно бы выгнали.

Оставшись без работы, я пошла на бесплатные интенсивные курсы болгарского языка, а потом стала учить язык уже за деньги, так как найти бесплатные курсы моего уровня не так просто. В итоге теперь моего знания болгарского языка почти хватает для получения гражданства. А гражданство даст мне массу преимуществ. Например, можно будет ездить в Европу без визы и, если понадобится, работать там. Кроме того, я получу право на бесплатную медицину в Болгарии.

Параллельно я продолжала искать работу и сходила на собеседования в несколько крупных международных кол-центров. Хотя во многих мне удалось пройти несколько ступеней — интервью с эйчаром по телефону и лично, тесты по русскому и английскому языкам, включая технические, — выбирали не меня, а более сильных кандидатов с продвинутым английским и с опытом работы в этой сфере. В результате я по-прежнему брала фрилансы от знакомых из России и через бывшего коллегу нашла подработку в качестве риелтора в крупной болгарской компании «Имотека». Суть моей работы в качестве «внешнего брокера» очень проста: искать новых клиентов, которые хотят купить или продать квартиру, и знакомить их с представителем компании в нужном регионе.

Местные жители относились и относятся ко мне исходя из моей роли в их жизни. Когда я ездила на море с детьми, у меня создавалось впечатление, что они просто хотят содрать с меня побольше денег (и часто это было правдой). В городке Ямболе ко мне относились как к русской жене болгарина — у болгар на эту тему есть анекдоты. В Софии я, скорее, экспат, потому что работаю только с русским и английским. При этом местные больше не считают, что я богаче их, и даже стараются помочь мне как многодетной матери: например, учительница моего сына представила его кандидатуру на обеспечение бесплатным питанием в школе.

\begin{fancyquotes}
    Неожиданно для себя я нашла в Болгарии друзей — людей, которые меня понимают, а я, надеюсь, что понимаю их, — хотя и была готова жить в новой стране без друзей, общаясь только со своей семьей
\end{fancyquotes}

Когда мы покинули закрытый мирок удаленных работ, отдали детей в школы и сады, вышли в офис, стали общаться не только со старыми друзьями и родственниками, а с разными людьми, я увидела в болгарах очень много того, что мне в них нравится.

Например, болгары очень хорошие родители и лояльно относятся к маленьким детям, даже к чужим. Для меня, привыкшей к постоянным нападкам и критике на улице в Москве, это важно. В России незнакомые пожилые женщины вели себя агрессивно, хотя я никак не пыталась привлечь их внимание, а здесь чужие мне болгары находят способы помочь, хотя не обязаны этого делать, и я их не прошу.

Когда мы отдыхали на море с детьми в Созополе, мне казалось, что болгары ленивые. Теперь я вижу, что зачастую туристы и экспаты высокомерно и пренебрежительно относятся к тем, кто работает в сфере услуг, а болгары, напротив, уважают и себя, и свое время, и свой труд. Здесь многие работают в сфере обслуживания, и это достойный труд.

Поскольку Болгария — это туристическая страна, сфера услуг разделена на две части: для туристов и для местных. Заведения для туристов открываются в сезон, привлекательно оформлены и предлагают какую-то специфическую кухню. Например, в приморском Созополе был ресторан родопской кухни (Родопы — это горы на границе Болгарии и Турции). Рестораны для местных, в свою очередь, делятся на простые заведения с болгарской кухней, которые, возможно, не такие нарядные, и стоят не так близко к морю и главным улицам, но еда там такая же вкусная, а стоит дешевле. В престижных районах Софии, таких как Лозенец, Гео Милев, есть и рестораны с оригинальной кухней для обеспеченных местных — сейчас популярны китайские и японские.

Очень важная тема, без знания которой трудно оценить жизнь в Болгарии, — это коммунальные платежи. На эту тему у болгар даже есть анекдот: «Что за паук, который тянет кровь с миллиона человек? Ответ: теплофикация Софии». Мы сняли квартиру с русскоговорящей хозяйкой в доме советской постройки семидесятых годов недалеко от школы, куда ходят дети, и были искренне рады, что в нашей квартире есть центральное отопление, причем с регуляторами на батареях. Всю первую зиму отопление мы держали примерно на втором-третьем делении, но в обеих комнатах и на кухне. В итоге нам не только пришел счет на 1000 левов (40 тысяч рублей), но и каждый месяц отопительного сезона приходилось платить около 300 левов (12 тысяч рублей), потому что платежи начисляются исходя из расходов предыдущего года.

На второй год мы уже подготовились: не включали батареи вообще (тут многие так делают), отапливали электрическим обогревателем. В итоге нам летом, когда происходит перерасчет, вернули больше 500 левов (20 тысяч рублей) из тех, что начислили сверх. За отопление и горячую воду мы каждый месяц платим около 100 левов (четыре тысячи рублей), а за электричество — не больше 70 левов в месяц (2,7 тысячи рублей).

\begin{fancyquotes}
    Бабушка моего мужа из Ямбола зимой топит свою спальню буржуйкой, в ее доме советской постройки есть даже специальный дымовой выход. Уголь на зиму покупает у цыган за 200 левов (восемь тысяч рублей)
\end{fancyquotes}

Я научилась экономить, еще когда мы жили в Москве. Здесь, в Болгарии, считаю очень важным развивать навыки экономии, потому что, во-первых, мы часто не знаем условий и попадаем в такие ситуации, как с оплатой коммунальных расходов. Мой муж, наоборот, радуется здешней жизни, местной кухне, этому городу. С другой стороны, у него и в Софии, и в Москве была нормальная должность, а у меня здесь, как и в Москве, фрилансы. Так что, можно сказать, какие были у нас наработки и сильные стороны, такие мы и привезли сюда.

Медицина — важный вопрос для многих из тех, кто задумывается о переезде в Болгарию. В деревне своевременно получить медицинскую помощь иногда затруднительно, нужно ехать в относительно крупный город и обращаться в приемное отделение больницы. Местные медицинские учреждения выглядят достаточно обшарпанно, лекарств, в том числе необходимых, может и не быть, их придется покупать самим. Мои знакомые, у которых были серьезные проблемы со здоровьем, не пытались даже использовать местную страховку, а ездили лечиться в Россию и даже платили там за лечение.

За два года я освоилась и перестала ощущать себя чужой в этой стране. В Софии, на море, и даже в городках вроде Ямбола живет много русскоговорящих, так что болгары к ним уже привыкли. Они считают, что у русских всегда много денег, и относятся соответственно — пытаются получить с них прибыль. От меня они, судя по всему, никакой прибыли уже не ждут и считают за свою. С другой стороны, я стала одеваться попроще, чем в Москве, больше не хожу с гордым видом — мол, я русская туристка, мне нужны качественные услуги. Кроме того, мой уровень болгарского вырос, а в Софии на русском говорят не только обеспеченные эмигранты из России.

Обычно разговоры о России в формате small-talk сводятся к фразам вроде «жил я в России, хорошо заработал» или «да был я в Москве, все такое огромное!» Болгары гораздо больше любят жаловаться на свою власть, чем обсуждать чужую, так что о политике говорят не так уж много. Обычно это я их утешаю на своем неважном болгарском, что, мол, жить тут можно, соотношение средних зарплат и расходов нормальное, реально купить квартиру, ставки по ипотеке невысокие, маньяков не так много, и так далее. А они мне рассказывают, какая прекрасная жизнь в Германии, где жили какие-то их знакомые. Правда, потом оказывается, что эти знакомые почему-то уже вернулись на родину.

Мой муж хочет жить здесь и растить детей в этой спокойной балканской стране. Я вижу, что для детей это, возможно, и правда хорошо. Нам с детьми помогает бабушка, которая сейчас на пенсии и уже не имеет карьерных амбиций. Она живет в другом городе, и мы можем отдавать ей детей на лето. Я готова адаптироваться здесь, учить язык, изучать правила жизни в этой стране, искать подходящие источники заработка и поддерживать мою семью.

\newpage
\subsection{США и Италия}
% --- NOT studied
% --- NOT added to Anki
\textit{«Здесь считают, что девушки из России сразу выскакивают замуж» История россиянки, которая перебралась в Италию и увидела ее темную сторону}

\textit{Ольга из Омска получила образование в США, а потом уехала в Италию. В этой стране она столкнулась с пренебрежительным отношением к иностранцам, патриархальными порядками и невозможностью пробиться без связей и правильных знакомств. В рамках цикла материалов о соотечественниках за границей «Лента.ру» публикует ее рассказ о жизни в разных странах.}

Когда меня спрашивают о том, как я перебралась в другую страну, хочется ответить встречным вопросом: «В какую именно»? Я родилась в Сибири, но за последние 15 лет мне довелось пожить в США, Италии и Германии.

В школе мне повезло с учительницей английского, поэтому для подростка 90-х я очень неплохо говорила по-английски. Америка тогда ворвалась на экраны, и мне ужасно хотелось туда попасть. В юности не задумываешься о том, насколько твои мечты реалистичны, а просто идешь к ним. И вуаля — в 16 лет мне достается стипендия, по которой подростки из бывшего СССР могли целый год жить в американской семье и ходить в американскую школу.

Меня приютила мормонская семья из штата Юта. Попасть в настолько религиозную среду было странно. Оговорюсь сразу: мормоны младенцев не едят. Они верят в Христа, хотя и считают книгу Мормона важнее Библии, пропагандируют воздержание до брака, но далеко не всегда его соблюдают, и, выступая против абортов, спокойно пользуются контрацепцией и разговаривают о сексе. Кое в чем они оказались даже либеральнее, чем Россия того времени. Самое необычное было другое: оказывается, мормоны не употребляют кофеин. В результате я целый год не видела чая.

Не скажу, что мне повезло с принимающей семьей. Я поладила только с отцом, а с матерью и сестрами отношения так и остались прохладными. И все же этот год стал для меня переломным. Я неплохо адаптировалась и улучшила английский. Вдобавок, американцы часто и искренне делают комплименты, что для постсоветского подростка-интроверта оказалось невероятно важным и расковывающим. Школа была несложной, окружающие по большей части доброжелательными и открытыми. После возвращения домой мне этого очень не хватало.

У меня сложилось впечатление, что в США действует социальный лифт, поэтому я решила во что бы то ни стало вернуться в эту страну. Ждать пришлось почти семь лет. За это время я окончила филфак в родном Омске и стала работать преподавателем английского. Второй раз я поехала в США, когда получила образовательный грант по программе Фулбрайта и смогла вернуться за степенью магистра.

Учеба была полностью оплачена программой, а стипендия составляла минимальный прожиточный минимум. Это был уже взрослый опыт в Атланте. Я сама снимала квартиру и распоряжалась расходами. Денег хватало на еду, одежду и развлечения. Грустно признавать, но я сразу стала позволять себе намного больше, чем когда работала на двух работах в России.

Учеба в университете оставляла немало свободного времени. Я ходила по вечеринкам и клубам по интересам, о которых узнавала из интернета, и с бокалом вина у бассейна вела разговоры о жизни, литературе и разных пустяках. Меня приятно поразило, как легко было познакомиться с людьми самых разных профессий и социальных слоев. Мне встречались журналисты и зубные врачи, студенты и предприниматели. Хотя я приехала из другой страны, нам, как правило, удавалось найти общую тему для разговора. Возможно, этому способствовала жизнь в большом городе, где все немножко чужие, и разговор так же легко начать, как и закончить. Тогда меня это устраивало.

Университетские курсы шли семестрами. Было много проектной работы, связанной с реальной жизнью и задачами. При необходимости можно было обратиться за помощью к преподавателям на факультете, отношения с ними были совершенно партнерскими. Тогда же я узнала, что программа Фулбрайта работает во всем мире, а быть ее стипендиатом в США — действительно престижно. Восхищенные «вау» стали льстить моему эго, которое от этого заметно увеличилось в размерах.

Именно в Атланте я познакомилась с будущим мужем-итальянцем, который работал там по контракту. Когда разговор зашел о браке, я взялась за изучение итальянского. Здесь начинается совсем другая история.

Я приехала в Италию в 2005 году с двумя дипломами о высшем образовании и гордым званием выпускника программы Фулбрайта. Мои итальянские знакомые из Атланты уверяли, что работодатели будут биться за меня не на жизнь, а на смерть. Но я еще не очень хорошо говорила по-итальянски и не могла начать поиски работы.

До переезда любая страна к западу от России представлялась мне немножко Америкой. В Италии с этой иллюзией пришлось расстаться. Оказалось, что здесь все делается по знакомству и повсюду нужны связи. Кто-то должен представить тебя другому, тот третьему — и так далее по списку. Но откуда у иностранки возьмутся знакомства с нужными людьми? После переезда я оказалась в полной социальной изоляции. Моим единственным знакомым был муж.

Мне не хотелось проводить время с соседками-домохозяйками среднего возраста, которых волновали только дети и рецепты. Интернет не выдавал никаких мероприятий, на которых можно заняться нетворкингом и наладить контакты. Даже самого слова «нетворкинг» никто не знал. К тому же 15 лет назад у большинства сегодняшних римских тусовок не было сайтов. Да и теперь мои итальянские знакомые не очень доверяют Google, предпочитая личные рекомендации. В конечном итоге я решила записаться на курсы фотографии.

Мы жили в окрестностях Рима. Сам Рим показался мне, увы, очень неопрятным городом. Изучив итальянский получше, я стала замечать, что слово «эмигрант» в Италии часто означает человека, занимающегося грязной работой, и да, второго сорта. Вдобавок, из-за множества женщин — «бойцов ночного фронта», поехавших на Запад после распада СССР, выражение «с Востока», то есть из Восточной Европы, приобрело недвусмысленную окраску.

Во время общения с незнакомыми людьми я порой сталкивалась с пренебрежительным отношением. Даже позже, в операционной на кесарево, кто-то из персонала не преминул отметить, что женщины вроде меня «едут сюда и быстренько выскакивают замуж». Большинство встречающихся мне итальянцев не говорили по-английски, и слово «Фулбрайт» для них было пустым звуком. Мое обласканное в Америке эго из рук вон плохо справлялось с такой ситуацией.


С первых дней меня не покидало инстинктивное чувство: что-то не так. Причина оказалась в том, что я привыкла говорить без обиняков. В России и США это не вызывало проблем — американцы и сами выражаются резко, когда нужно расставить точки над «i». А в Италии ничего не говорят напрямую. После диалога с итальянцем иногда приходилось гадать, что же хотел сказать собеседник. Поначалу мне даже казалось, что в этом есть скрытое издевательство. Кроме того, итальянцы обожают титулы — адвокат, инженер, учитель. Мне это немного напоминало мексиканские сериалы.

Первая работа в Италии была исключением — я нашла ее по объявлению. Консалтинговая компания, занимающаяся развитием персонала, все же оценила мой американский диплом. Отношения у нас, правда, не сложились. Прежде всего мне физически тяжело было выносить два-три авиаперелета в неделю. К тому же такая работа не сочеталась с планируемым материнством.

Постоянные недоговорки, обсуждения за спиной и жесткая иерархия в маленькой компании вызывали у меня замешательство. При этом начальник не сдерживался, чтобы напомнить о своем статусе. Однажды он повысил на меня голос в присутствии остальных коллег. Этим, впрочем, он не прояснил ситуацию, а просто подчеркнул, «кто кому Вася». Ничего подобного на работе со мной не случалось ни в России, ни в США. Через несколько месяцев я решила уволиться и до сих пор об этом не жалею.

Параллельно мы готовились к свадьбе. Я не притворялась особенно верующей, но против венчания не возражала. Для мужа и его семьи, практикующих католиков, это было невероятно важно. Пришлось пройти подготовительные курсы в церкви. Было удивительно слышать, как другие молодые пары спрашивали священника о том, что думает о контрацепции папа Римский. Такие вопросы плохо сочетались с представлением о либеральности Старого Света.

Как выяснилось, Италия — это все-таки очень патриархальная страна. Еще недавно здесь одобряли браки с насильником, призванные «починить» репутацию жертвы, и закрывали глаза на убийства чести (то есть убийства неверной жены или «непутевой» сестры), которые, судя по криминальной хронике, случаются до сих пор. Правда, теперь есть закон о домашнем насилии, а человек, совершивший убийство в целях самообороны, может быть оправдан. Заметно, что молодые итальянки не собираются больше терпеть того же, что их матери. Да и мужчины стали посговорчивее в плане помощи по дому и воспитания детей.

Вскоре на свет появилась первая дочь, и мы переехали в Пизу. Это совсем небольшой городок, где есть два важных заведения: больница и университет. В Пизе я обнаружила, что, с точки зрения государственных учреждений, у меня просто нет высшего образования. Мои дипломы, которые я получила в России и США, не представляли для них никакой ценности. Меня это возмутило, поэтому в Пизе я получила бакалавра филологии с отличием, а добрая знакомая помогла устроиться корпоративным преподавателем английского и русского.

Через четыре года мы вернулись в Рим, где я продолжила преподавание. Владельцами римской школы были... правильно, отец и брат прежнего работодателя. Тогда же Google наконец-то выдал мне сайт римской ассоциации женщин-профессионалов. Это было теплое международное сообщество, в котором я нашла замечательных подруг — финку и англичанку.

К тому времени стали подрастать мои дети-билингвы. Понятие билингвизма только-только начинает просачиваться в итальянскую реальность. Педиатры, которые мне встречались, ничего об этом не знали, но часто негативно реагировали на разговоры на двух языках, считая, что это будет задерживать развитие ребенка. К счастью, я оказалась упрямой иностранкой, и сейчас мои дети спокойно учат даже не два, а четыре языка.

Год за годом я обрастала новыми интересными знакомствами и перестала драматично воспринимать негативные стороны эмиграции. Я создала свою микрореальность, в которой гармонично уживалось хорошее из разных культур. У меня поменялось ощущение времени и пространства. Я перестала осуждать и начала интересоваться причинами того, почему что-то сложилось в определенной культуре именно так, а не иначе. Для этого в Италии хватает возможностей.

С культурной и туристической точек зрения, эта страна, наверное, никогда не потеряет своей привлекательности. Итальянские регионы сохранили свою самобытность благодаря консервативности и ощущению ревнивого соперничества с соседом, на мой взгляд, совершенно не свойственного русским. Это отражается и в кухне, и в диалектах, и в том, как итальянцы представляются. Обычно вам сначала скажут, что они, например, флорентийцы, только потом — тосканцы и итальянцы.

За последние 10 лет социальный лифт в Италии, кажется, совсем сломался, что вынуждает молодежь уезжать за границу. До середины XX века его, впрочем, не было вовсе. Моя мама и итальянская свекровь — ровесницы, обе — девочки из деревни. Но в Италии крестьянская девочка, выросшая после войны, даже не помышляла о том, чтобы получить образование выше начальных классов.

Сардиния, родина моего мужа, — невероятный по самобытности остров, где время в горных деревнях почти остановилось и не имеет ничего общего с искусственным Изумрудным Берегом, так полюбившимся олигархам. Даже итальянским друзьям мы всегда привозим оттуда что-то такое, чего нет на континенте.

Моя история переездов — это не такая уж редкость в наше время. В мире около 60 миллионов экспатов, и тех, кому приходится адаптироваться к чужой стране, становится все больше. Мне было интересно узнать, какие программы могут помочь этим людям. Поэтому я слетала в Берлин и получила сертификат тренера по культурному интеллекту.

Культурный интеллект — это навык, который позволяет эффективно общаться с представителями других культур. Сейчас много говорят об эмоциональном интеллекте, а культурный подхватывает эстафету там, где эмоциональному сложно понять, что стоит за поведением людей из совсем другой культуры. И его можно развить и проверить. Я успела провести в Италии несколько тренингов по культурному интеллекту для бизнес-студентов и работников государственной сферы, а затем жизнь преподнесла нам сюрприз.

В прошлом году работа закинула мужа в Баварию, и нам пришлось перебраться вслед за ним. Это тоже невероятно интересный и ценный опыт. Только разговор о нем заслуживает отдельной статьи.

\newpage
\subsection{Япония 2}
% --- Studied on 25-26/2/2023
% --- Added to Anki

\textit{«Постоянно осознаешь, что ты здесь чужой» История россиянки, которая уехала в Японию и увидела ее темную сторону}

\textit{Анна Литвинова переехала в Японию в 2017 году, чтобы учиться в музыкальном колледже. Хотя девушка бывала в стране и раньше, ей пришлось непросто: закрытые японцы так и не смогли до конца ее принять. В рамках цикла материалов о перебравшихся за границу россиянах «Лента.ру» публикует ее историю о жизни в Токио.}

У меня было несколько причин для переезда в Японию. Я часто бывала в этой стране с детства — там живет моя родственница, к которой я приезжала в гости из родного Владивостока. К подростковому возрасту я уже привыкла к японскому образу жизни. К тому же оттуда можно быстро добраться до дома, и разница в часовых поясах небольшая — так что я всегда могу оставаться на связи с семьей. А я очень привязанный к дому человек, и для меня это важно.

Поэтому когда я окончила девять классов и музыкальный колледж, выпустилась и решила уехать учиться дальше за границу, выбор пал именно на Японию. Я поступила в Токийский колледж музыки в 19 лет. Училась за деньги родителей сначала на отделении музыкально-гуманитарных наук, а со второго года перевелась на специализацию «Дирижирование».

В марте я окончила четвертый курс бакалавриата. По удачному стечению обстоятельств, а именно хорошему знакомству, я сразу устроилась на работу в японскую компанию. Не совсем по специальности, но в музыкальной сфере. Я занимаюсь международными отношениями в японской компании, специализирующейся на контентном менеджменте, и пока работаю удаленно из Владивостока. В начале следующего года должна вернуться в Японию, если все пойдет по плану.

\textbf{Ожидания и реальность}

Получить японскую визу очень сложно, хотя, говорят, два-три года назад процедуру немного упростили. Тем не менее все равно это достаточно трудоемкий процесс: у меня он занял огромное количество времени — что с учебной визой, что с рабочей. Из-за пандемии сейчас вообще не выдают туристические визы, и нельзя въехать в страну даже в гости к родственникам, разве что в экстренных случаях — например, на похороны. В общем, без веских оснований в Японию точно не пустят.

Но попасть в страну — еще полбеды. Я не ожидала, что возникнут серьезные проблемы с тем, чтобы вписаться в местное общество. К сожалению, это мне не удалось.

\begin{fancyquotes}
    Мне пришлось признать поражение и признаться себе, что я никогда не буду выглядеть как японец и не буду говорить как японец, даже если прекрасно знаю язык. Постоянно осознаешь, что ты здесь чужой, не можешь почувствовать себя частью общества, и это самое большое разочарование
\end{fancyquotes}

Зато, как я и думала, здесь очень безопасно и чисто. А еще это страна действительно вежливых людей. Никаких конфликтов с воплями здесь в принципе не бывает, люди всегда хорошо и приятно себя ведут. Конечно, все представлялось несколько более сказочным, наверное, из-за мультиков Миядзаки. В принципе, эти ожидания тоже иногда оправдывались, но, если честно, в рутине все это волшебство немного померкло.

Что касается каких-то технических условий и уровня жизни, это действительно фантастическая страна. Здесь хорошая система образования, отличная медицина, налаженная госслужба: если нужно получить какую-то бумажку, все действительно работает как часы. Жить в Японии очень комфортно, быт идеально налажен, к тому же здесь хорошо и тепло.

Не нужно вообще ни о чем беспокоиться, что ты что-то где-то забыл, недоплатил, за тебя всегда все сделают, тебе всегда все простят, особенно если ты иностранец. Японцы снисходительно относятся к тому, что ты не знаешь язык. Я как иностранец, наоборот, чаще получала больше преференций, чем каких-то тумаков, за то, что я не сделала что-то. Мне всегда помогали, и даже были какие-то поблажки.

Но, несмотря на все это, Япония все равно остается недружелюбной к иностранцам — такой уж менталитет у местных. Не потому что японцы какие-то агрессивные, нет, просто все время тебя выталкивают, и влиться в общество очень сложно. Я так уверенно говорю, потому что общаюсь с русскими, все здесь жалуются на одно и то же: непонимание окружающих и одиночество.

\textbf{Страна не для всех}

В первое время меня поддерживали надежда и вера в светлое будущее, но при этом почти сразу стало понятно, что будет непросто. С конца первого курса я стала практически перманентно пребывать в депрессивных состояниях и занималась с психологом. Но было много моментов, которые компенсировали это состояние, потому что все было в новинку, да и люди интересные попадались.

Самый ужас пришелся на третий курс, когда сошла волна восхищения и восторга, началась уже просто жизнь. После окончания семестра я вернулась домой, началась пандемия, закрыли границу, я осталась дома. Я сидела и думала, что вообще не хочу туда возвращаться, что мне там плохо. Но в итоге я позанималась с психологом, взяла себя в руки и завершила образование в вузе.

\ed{Однозначно}{однозначно}{definitely} могу сказать, что Япония — это очень клевая и интересная страна, в которой нужно побывать хотя бы раз в жизни. Она комфортная для тех, кто хочет там жить, но это однозначно страна не для всех мигрантов, как, например, Америка. Японию нужно прямо любить, тогда получится что-то перетерпеть, что-то пережить, к чему-то привыкнуть. А если просто приезжаешь как рабочий мигрант или \explain{по приколу}{for fun}, то скорее всего все закончится плохо.

\begin{fancyquotes}
    Япония — особенное место, в которое я не советую соваться просто ради интереса. Лучше иметь какие-то личные причины или всем сердцем любить что-то японское. Вот тебе нравится аниме или какая-то часть японской культуры, и ты хочешь конкретно ее изучать — тогда да, игра стоит свеч
\end{fancyquotes}

\textbf{Холодные люди}

Самой больной темой стал поиск друзей. У меня их в принципе в жизни было немного, и все остались в России. В итоге мне удалось найти лишь огромное количество знакомых. Если у меня случится что-то по-настоящему серьезное, я знаю, что есть люди, которым я могу позвонить, и мне помогут. Но вот таких приятелей, чтобы постоянно видеться, у меня нет. В последнее время я больше общаюсь с русской диаспорой, приятели из колледжа все разбежались по окончании учебы.

И это не потому что японцы плохие, но тут у всех культ работы. Все мои приятели вечно заняты, и выбраться куда-то на чашку кофе очень сложно: возникает бесконечная череда обстоятельств. Японцы действительно трудоголики. Даже не за слишком большие деньги они работают на износ, буквально вусмерть: сверхурочно, без доплаты, без отпусков. Переработки — одна из самых частых причин смертей у местных. У меня отпуск по контракту — 10 дней в году, и так почти у всех.

Настоящих друзей среди японцев у меня все же нет. Им сложно открываться и искренне доверять друг другу. Я не могу сказать за всех, но сама чувствую какую-то всеобщую холодность, на которую жалуются и сами японцы. Они мне говорили, что им очень непросто найти друзей.

\textbf{Японцы и гайдзины}

Тем не менее японцы очень отзывчивые, особенно по отношению к иностранцам. С одной стороны, они немного нас побаиваются, относятся с некой настороженностью, но в то же время с уважением, понимают, что ты нормальный человек. На весь колледж я была одной постоянной студенткой с европейской внешностью. Иногда на полгода приезжали студенты по обмену из Европы, но в основном там учились японцы и немного китайцев и корейцев.

\begin{fancyquotes}
    Мое появление в колледже вызывало фурор, меня знали абсолютно все. Сначала было приятно, потом, естественно, начало напрягать
\end{fancyquotes}

Отношение было доброжелательным. Но иногда чувствовалось, что интерес был только из-за внешности, никто не пытался узнать, что я за человек. Японцы далеки от новой этики, я не могу сказать, что страдала от харассмента, но часто выслушивала скользкие комментарии о себе. Чувствуешь себя какой-то обезьянкой, с тобой постоянно фотографируются. Иногда просто хотелось, чтобы ко мне относились как к человеку, а не отвешивали странные комплименты вроде «ого, ты хорошо ешь палочками».

У меня есть огромное количество знакомых в магазинах возле дома, куда я постоянно хожу. К тому же я общаюсь почти со всеми соседями. Мне кажется, все дело в том, что здесь часто бывают природные катаклизмы, и поэтому все понимают, что лучше общаться с ближними, чтобы они тебе помогли, если что-то пойдет не так. Здесь соседи знают друг друга, общаются, устраивают сборища, вместе проводят время.

Говорят, что в деревнях бывает иначе, потому что там общество более архаичное, люди видят меньше иностранцев и буквально прячутся от них. Но это не потому, что они националисты какие-то. Если и есть что-то такое, то против ближайших соседей типа Китая или Кореи. Я столкнулась с национализмом буквально один раз, и то это был человек, судя по всему, с какими-то ментальными сложностями: женщина мне крикнула вслед что-то типа «белая еврейская шлюха». А так всегда все очень доброжелательно.

\begin{fancyquotes}
    Россия для японцев — это балет, литература, Толстой и Достоевский, это страна исключительно высокой культуры. Многие были в Эрмитаже, ездили по Транссибу. Еще здесь очень любят Чебурашку
\end{fancyquotes}

Если выяснялось, что я русская, это всегда вызывало шквал позитивных эмоций, все вспоминали какие-то русские слова, которые они знают. Я ни разу не сталкивалась с каким-то негативом относительно «северных территорий», как у них называют Курилы, никаких претензий ко мне не было. Интерес к России огромный, в основном, конечно, к культуре, но и к политике тоже.

\textbf{Самое сложное слово}

Главная особенность японцев — неумение сказать «нет». Говорить все напрямую для них очень сложно, начиная от деловых отношений и заканчивая личными. У них в языке есть слово «нет», но им фактически никто никогда не пользуется, заменяя это слово неопределенными фразами. Иногда не можешь понять, человек обдумывает твои слова или просто не может тебе отказать. Я часто сталкиваюсь с этим по работе, мне все время приходится переспрашивать и уточнять.

Еще во время диалога надо постоянно подтверждать, что ты слушаешь собеседника. То есть «ага» во время бизнес-переговоров — это не проявление согласия, так что нечего удивляться, если сразу после этого тебе откажут. Всему этому невозможно научиться, это можно только почувствовать. Особенно это касается поклонов — только со временем понимаешь, как сильно нужно опускать голову.

В сфере обслуживания клиент — это господь бог, он всегда прав. Работник сферы услуг фактически не человек, он немного раб каждого клиента. Сервис должен быть безупречным, несмотря на хамство клиента, больную голову или любое другое обстоятельство.

\textbf{Рай для гурманов}

Япония — это страна для гурманов, тут очень хорошие продукты благодаря высоким стандартам по отношению к импортным товарам. Те же овощи из Китая совершенно другие, чем китайские овощи, которые продаются в России. Огромный выбор местных продуктов: бесконечных лапшичек, соусов, желешек, сладостей. Но если не любишь морепродукты, жить сложновато.

Япония в целом и Токио в частности хороши тем, что здесь можно найти все по любым ценам. Есть магазины, где можно очень дешево по скидке купить после девяти вечера мясо, которое было нарезано часов 10-12 назад. По местным стандартам оно не считается свежим, то же самое и с хлебом. Все всегда очень качественное, я не помню, чтобы я здесь хоть раз отравилась.

В ресторанах всегда вкусно, какого бы уровня они ни были, а еще в Токио представлены все кухни мира, можно прийти в русский ресторан поностальгировать по родине. Тут есть круглосуточные магазины под названием «комбини», где есть все — от минимального набора одежды до линз, лекарств, алкоголя и сигарет.

Общественный транспорт очень комфортный. Много людей, даже состоятельных, не пользуются автомобилями, настолько он удобный и быстрый. Метро прекрасное, станций огромное количество, даже за город можно доехать, не меняя вид транспорта. Везде как дома, куда ни приедешь, — одна и та же карточка метро, одна и та же платежная система. Почти нет проездных, точнее, система такая: идешь в метро, приносишь прописку и адрес места работы или учебы, и тебе делают проездной ровно от твоей станции до работы. А вот такси очень дорогое, 4-5 долларов только за посадку.

\begin{fancyquotes}
    Аренда стоит огромных денег, особенно в Токио. Пока я училась, мама сдавала четырехкомнатную квартиру, чтобы я могла снимать квартиру площадью 16 квадратных метров
\end{fancyquotes}

В основном квартиры у всех очень маленькие, все живут очень скромно. В съемных квартирах нельзя заводить животных, еще жилье нельзя в субаренду сдавать. Вот я сейчас живу в России, но плачу за квартиру, чтобы сохранить ее за собой.

Некоторые вещи в Японии меня очень удивили. Например, здесь до сих пор используют факсы. У них технологическая держава, все летает, и тут вдруг факсы... Документооборот действительно происходит именно так, и в этом есть доля здравого смысла, потому что это правда очень безопасный способ передачи данных. Еще здесь повсеместно используют кеш — огромное количество мест, где нельзя расплатиться картой.

\textbf{Медицина}

Я знаю, что если я чем-то плохим заболею, меня точно спасут. Все, что касается вывода в ремиссию рака, операций — на высшем уровне. Но мне как-то очень не везло с несерьезными заболеваниями. Тут все лечат антибиотиками. Я простудилась — прописали антибиотик на пять дней. Я просто перестала ходить к врачу.

Медицина достаточно дорогая, но обязательно есть страховки. Без страховки жить нельзя. Страховка распространяется на все — не только на муниципальные, но и на частные клиники, даже на некоторые аптеки.

\begin{fancyquotes}
    Здесь очень трудно попасть на любое исследование без назначения, даже на простой анализ крови
\end{fancyquotes}

Японцы, как я понимаю, не фанаты оперативного вмешательства, они всегда попробуют лечить таблеточками, травами, мазями. Здесь мало распространены пластические операции, потому что это считается неестественным. У местных часто деформированы челюсти и зубы, но брекеты почти никто не ставит.

\textbf{Особые развлечения}

Сфера развлечений в Японии развита прекрасно. Японцы много работают, им надо сбрасывать пар. Есть много развлечений, которые могут показаться нам странными. Однако в каком-то смысле японцы гораздо более честны с собой, чем любая другая нация, у них куда меньше табу.

Например, в секс-индустрии у них все кроме вагинального секса можно продавать как услуги. То есть проституция вроде бы как запрещена, но все остальное — нет. У них много развлечений по типу мэйдо-кафе, куда можно прийти пообщаться с девушками в костюмах горничных.

Можно нанять себе семью на час или человека, который тебя выслушает, или девушку, которая просто пообнимается с тобой. Все здесь очень заняты работой, и когда появляется свободное время, им нужно его быстро использовать. Им некогда наращивать социальные связи, куда-то ходить и знакомиться, приглашать на свидания.

\begin{fancyquotes}
    Эротические журналы в супермаркете лежат на соседней полке рядом с другими товарами. Зайдешь в магазин «все по сто иен», аналог Fix Price, и найдешь товары для взрослых даже там. У японцев в культуре нет пиетета перед этим — ну физиология и физиология, это спокойно воспринимается. Нет отношения к сексу как к греху
\end{fancyquotes}

Отдых японцы ассоциируют с пивом. Здесь считается, что человек не может раскрепоститься без алкоголя, он напряжен и зажат, и ему нужно чуть-чуть выпить, чтобы разговор пошел. Большинство людей пьют каждый день и не считают это чем-то предосудительным. Многие снимают стресс с помощью игровых автоматов, за ними часто можно увидеть серьезных мужичков в костюмах.

Но есть и другие формы досуга. Из спорта здесь очень популярен бейсбол и гольф. Многие любят ходить в парки и любоваться природой. Еще японцы обожают искусство. К ним приезжают музыкальные группы, которые давно распались, но для Японии собираются вновь, потому что знают, что здесь можно срубить огромные деньги.

Я знаю людей, которые годами регулярно покупают билеты в филармонии и ходят на концерты. Сюда приезжают абсолютно все выставки, привозят все самые знаменитые картины. Помню, привозили чешского художника Муху, и я пришла туда в праздничную неделю. Покупаю билет, захожу на выставку и понимаю, что там очереди обвиваются вокруг друг друга и пересекают сами себя. В общем, так много людей, что были даже регулировщики очереди.

Здесь рай для любителей культуры и популярной, и традиционной. Понравится и тем, кто считает, что Япония — это страна аниме и пикачу, и тем, кто приедет посмотреть на храмы и насладиться традиционными ремеслами. Мне интереснее второе.

\textbf{Планы на будущее}

Я не планирую оставаться в России насовсем, хотя сейчас из-за пандемии многое переосмыслила. В техническом отношении Япония меня устраивает, но хочется попробовать что-то еще. Семью я тоже хочу перевезти туда, где буду жить.

Пока я жила в Японии, я каждые три-четыре месяца приезжала в Россию, поэтому не успевала особо соскучиться. Скучала в основном по семье и каким-то милым сердцу местам.

\begin{fancyquotes}
    Иногда даже не хватало какой-то неустроенности: в Японии настолько все чисто и хорошо, что ни за что не цепляется глаз
\end{fancyquotes}

Иной раз идешь и думаешь: хоть бы что-то сломалось или испачкалось, настолько все идеальное. А я любитель такой вот эстетики типа как в фильме «Брат», когда все облупленное.

Минимальная зарплата приличная, выпускник университета может рассчитывать минимум на две тысячи долларов. На них, может, и не пошикуешь, но хватит поесть, оплатить квартиру, иногда сходить куда-то.

Посетить Японию надо обязательно, это другая планета, другой мир. Но если соберетесь там жить, нужно уметь заботиться о себе. Токио сожрет тебя и не подавится, поэтому нужно иметь контакты хорошего психолога, иначе можно скатиться непонятно во что.

\newpage
\subsection{Б\'{е}льгия}
% --- Studied on 25 Feb 2023
% --- Added to Anki
\textit{«Стереотипы о России — как и везде: водка, деревня и снег» История сибирячки о жизни в Бельгии}
% https://lenta.ru/articles/2020/12/14/belgium/

\textit{Юлия из Иркутска никогда не планировала жить в Европе, но судьба распорядилась иначе: на учебе в Китае она познакомилась с бельгийцем, вышла за него замуж и переехала к нему на родину. Там сибирячка нашла много достоинств и \ed{недостатков}{недостаток}{disadvantage}. В рамках цикла материалов о россиянах, перебравшихся за границу, «Лента.ру» публикует ее историю о жизни в Генте.}

Я родом из Иркутска и никогда не представляла, что буду жить в другой стране, даже не мечтала об этом. Но первая моя поездка за границу с мамой в 2005 году, в маленький китайский город возле Желтого моря, определила мою судьбу. После этого я поняла, что хочу учить китайский и узнать поближе культуру Китая.

Я поступила в международный институт экономики и лингвистики и оказалась на практике в Шэньяне. Но вместо того, чтобы практиковать китайский язык, улучшила свой английский: все иностранные студенты жили в одном общежитии и общались на английском.

Там была группа бельгийских студентов, которые сторонились нас и особо с нами не взаимодействовали. Как потом оказалось, они боялись русских, думали, что мы агрессивные. Немного погодя они поняли, что русские очень даже дружелюбные и нападать ни на кого не собираются, и мы все сдружились. Один парень из этой группы стал моим мужем, и после нескольких лет учебы и работы в Китае мы решили уехать в Бельгию.

\textbf{Как попасть в Бельгию}

На визу жены я подавать не собиралась, так как мне уже один раз отказывали: русским девушкам до 25 лет не особо охотно дают такую визу. Кроме того, я понимала, что с моим дипломом специалиста и китайским языком работу найти будет очень сложно, поэтому решила получить степень магистра в бизнес-школе в Бельгии. Чтобы комфортно жить во время учебы, мне нужна была стипендия. Для этого я написала мотивационное письмо и готовилась к тестам и собеседованию каждый день.

Меня спас маленький лайфхак: я знала, что у бизнес-школ есть рейтинг, который определяется многими критериями, в том числе количеством иностранных студентов, и что в бизнес-школах учатся преимущественно парни, поэтому у девушек есть небольшое преимущество при поступлении.

Так что я понимала, что не только мне нужно поступить, но и престижной бизнес-школе в Бельгии очень выгодно принять мотивированную студентку из Сибири. В итоге я получила стипендию и легко оформила студенческую визу. Преподавание велось на английском.

\begin{fancyquotes}
    Всем, кто считает, что из маленького города невозможно уехать учиться по стипендии, я скажу так: вы просто еще не пробовали
\end{fancyquotes}

Кстати, в Бельгии одно из самых дешевых и качественных высших образований. Год обучения в государственном университете стоит тысячу евро, почти 90 процентов субсидировано государством. Но учиться тут очень сложно, экзаменационных вопросов не существует, почти 80 процентов студентов отчисляются или просто уходят после первого курса.

\textbf{Страна пива и шоколада}

Поначалу приспособиться к жизни здесь было сложно, так как после Китая и России контраст оказался очень велик. Меня удивляли мелочи, к которым я так привыкла, что не обращала на них внимания раньше. Например, в Бельгии очень много мест, где принимают только наличные. Такси очень дорогое, его не поймать на дороге, даже вызывать по телефону сложно. Транспортная система очень медленная, автобусы и поезда всегда опаздывают, бельгийцы постоянно об этом шутят.

В плане прогресса меня эта страна разочаровала — как я и думала, сюда, как и во многие европейские страны, можно приезжать только ради истории, потому что тут и правда особенно ничего не изменилось со времен средневековых королей. К слову, в Бельгии до сих пор есть король, но власти у него совсем нет, управляет всем парламент.

\begin{fancyquotes}
    До переезда я даже не знала, на каком языке говорят бельгийцы. Мой будущий муж при знакомстве сказал, что он говорит на dutch, и я подумала, что это немецкий. Как потом выяснилось, это был голландский язык
\end{fancyquotes}

В Бельгии три официальных языка и три территории, соответственно. Они отличаются друг от друга как три совершенно разные страны: у них разное правительство и законы, даже ограничения скорости на дорогах разные. Я представляла Бельгию сказочной маленькой страной, где много шоколада и пива, такой она и оказалась.

Из-за того, что страна маленькая и языков много, у них не дублируют фильмы — даже профессии такой нет, поэтому все смотрят кино с субтитрами. Муж был в шоке, когда услышал, что наши фильмы дублируются, а истории о переводе Гоблина он до сих пор рассказывает друзьям как страшилку.

Но у маленькой страны, разумеется, есть и преимущества. Впрочем, как и во всей Европе: расстояния небольшие, поэтому можно оказаться в Париже на выходных за два часа или на праздники уехать в Берлин.

Немного позже я осознала, что Бельгию можно смело называть большой деревней. У них развито сельское хозяйство, они предпочитают жить в домах, а не в квартирах, у людей деревенский менталитет, и ритм жизни очень медленный.

\textbf{Работа и налоги}

Иностранцу, который не является гражданином ЕС, очень сложно найти высокооплачиваемую работу из-за двух причин: язык и рабочая виза. В Бельгии желательно говорить на трех языках (французский, голландский и немецкий), а для визы много особых требований, так как это политический центр Европы.

Бельгийцы настолько строги в этих вопросах, что многие студенты потом уезжают в соседние европейские страны, Нидерланды или Германию, так как там проще найти работу и получить визу. Лично мне не пришлось долго искать работу, потому что я получила диплом магистра в бельгийской школе и уже во время учебы активно занималась поисками, расспрашивая всех профессоров и приглашенных лекторов.

Живя в разных странах, я заметила, что каждая нация всегда на что-то жалуется. Например, кубинцы всегда жалуются на то, что жарко, китайцы — что слишком много людей, а бельгийцы недовольны высокими налогами. Они и вправду высокие!

Подоходный налог достигает 50 процентов. Поэтому задействованы всевозможные обходные пути. Допустим, работодатель часто вместо денежных бонусов включает в контракт чеки на еду или машину с оплачиваемым бензином. В моей компании всем работникам высшего звена выдают Tesla — это тоже способ уменьшить налоги за счет экологичности автомобиля.

\textbf{Цены и медицина}

Снять квартиру в 70 квадратных метров в центре города с гаражом стоит около тысячи евро в месяц, что составляет примерно треть зарплаты. Коммунальные услуги и отопление дорогие, поэтому бельгийцы предпочитают носить свитер и теплую обувь в квартире, чем тратить энергию и деньги.

\begin{fancyquotes}
    18 градусов внутри помещения — это нормально. И когда я мерзну у кого-то в гостях, они удивляются, восклицая: «Ты же сибирячка!» Трудно им объяснить, что в Иркутске в помещении 23 градуса и выше
\end{fancyquotes}

Интернет и мобильная связь тоже очень дорогие, 80 евро в месяц, а главное — нет безлимита. Продукты в магазинах хорошего качества и в принципе цены доступные, к тому же работодатель часто выделяет чеки на покупку продуктов.

Медицинское обслуживание покрывается страховкой, и страховка возвращает до 80 процентов расходов. Больницы очень хорошие, и в целом уровень медицины считается высоким. Еще у бельгийцев развита частная практика, которая тоже покрывается страховкой. В основном врачей посещают в частных медкабинетах, которые расположены на первых этажах многоэтажек или дома у врача.

Местные медики не любят прописывать лекарства, антибиотики почти никогда не прописывают. Часто в ответ на жалобы могут рекомендовать делать массаж или есть фрукты, на крайний случай пить парацетамол. И анализы без показаний тоже трудно получить, нужно слезно выпрашивать.

\textbf{Местные развлечения}

Главное развлечение у бельгийцев — ходить в бары и пить пиво. Еще они любят изысканно поесть и проводить время в ресторанах бельгийской и французской кухни. Бельгийцы, к слову, считают себя изобретателями картошки фри.

Остановлюсь на основном бельгийском напитке — пиве: баров в Бельгии очень много, сортов пива вообще не счесть. Согласно статистике, здесь зарегистрировано больше 1500 сортов оригинального бельгийского пива. Самое дешевое — CaraPils — стоит евро за литр. Самое лучшее пиво в Бельгии — Westvleteren — это сорт траппистского пива, которое производится на территории одноименного аббатства и стоит около 40 евро за бутылку.

Еще одна страсть бельгийцев — велоспорт. В каждой деревне, в каждой компании есть свои велокоманды. На улицах даже стоят специальные скамейки для того, чтобы следить за проезжающими велогонщиками, — это поистине национальный спорт.

В Бельгии, да и Европе в целом очень заботятся о природе: мусор строго сортируется, в каждом офисе или в магазине принимают батарейки, сдавать пустые бутылки — это тоже норма. В центре города сделали зеленую зону пониженной эмиссии, обычные машины там запрещены: только пешком или на велосипеде.

Столь сильная вовлеченность государства и людей сделала и меня более осознанной в плане моего отношения к планете. Когда я приезжаю в Иркутск, я учу родителей: они уже даже жвачку не жуют и не используют одноразовые трубочки для напитков.

\textbf{Разница культур}

Говоря о культурах разных стран, я часто использую метафору о культуре персика и культуре кокоса. Русские — это культура кокоса: твердые и неприступные снаружи, но если прорваться внутрь, там уж душа нараспашку, можно наслаждаться общением и плавать в кокосовом молоке.

Культура персика, наоборот, предполагает очень мягкий барьер снаружи, но твердую косточку, куда почти невозможно прорваться. Бельгийцы такие: они очень вежливые и улыбаются даже незнакомцам. В деревне фермер на тракторе помашет рукой, если будет проезжать мимо. Это очень приятно. С другой стороны, очень сложно построить с ними близкие отношения.

Бельгийцы очень любят слушать про Россию, особенно про Сибирь. Они положительно отзываются мне в лицо о России, и заветная мечта многих — путешествие на поезде по Транссибирской магистрали. Мой муж уже осуществил ее в плацкарте, после чего никогда больше не ел лапшу «Доширак». Он остался под глубоким впечатлением от умения русских играть в шахматы — естественно, он проиграл всему поезду.

\begin{fancyquotes}
    Местные стереотипы о России — как и везде: водка, деревня и снег. Даже один раз спросили, есть ли у нас магазины. Многие бельгийские студенты почему-то рассказывали, что мы ходим строем по улицам и поем песни хором
\end{fancyquotes}

Еще думают, что в России везде холодно, и в принципе Россия для них — это одна Сибирь. Первое, что попросила моя свекровь, — это унты. Ей отправили пару по почте в подарок, но, конечно же, тут она их никогда не надевает, так как снега почти нет.

\textbf{Планы на будущее}

Мы планируем жить в Бельгии, но в идеале — когда выйдем на пенсию. Страна очень дружелюбна к пожилым: скорость жизни очень медленная, отличное медицинское обслуживание, хорошая пенсия, да и дома престарелых просто отличные. Правда, пенсионный возраст — с 65, и его уже планируют поднимать до 67. Но пока мы в самом расцвете сил, я хотела бы пожить в мегаполисе. Я раньше жила в Шанхае и очень скучаю по этому ритму и городской динамике.

Конечно же, я скучаю по России, особенно по своей семье и друзьям, по палаточным лагерям на Байкале, по маминым картофельным оладьям, по огороду на даче. Скучаю по походам в лес по грибы, потому что в Бельгии собирать грибы в лесу строго запрещено. Очень жду, когда пандемия закончится, границы будут снова открыты, тогда я смогу полететь домой и увидеть всех родных.