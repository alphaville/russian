\chapter{Путешествия и транспорт}

\section{Кошмар началс\'{я} на второй день}

\textit{Мария Гейн}

\textit{Истории туристов, которые решили бесплатно пожить у местных}

\textit{Истончик: \url{https://lenta.ru/articles/2022/09/27/couchsurfing/}}

За последние 20 лет сервис для поиска бесплатного \ed{ночл\'{е}га}{ночл\'{е}г}{overnight stay} в поездках CouchSurfing \explain{снискал популярность}{gained popularity} у путешественников по всему миру. Для многих такой формат поездок стал не только способом сэкономить \ed{приличные}{приличный}{decent} деньги на отелях, но и возможностью лучше узнать новый город или страну. Тем не менее каучсерфинг может оказаться настоящим испытанием для \ed{неподготовленного}{неподготовленный}{unprepared} туриста — форумы \ed{пестрят}{пестрить + чем?}{are full of} историями о неадекватных хозяевах и ужасных условиях, в которых оказались путешественники. Как решиться на каучсерфинг, насколько опасен такой формат путешествий и чем может обернуться ночлег у незнакомца — в материале «Ленты.ру».

\textbf{От хиппи до диванных \ed{скитальцев}{скиталец}{wanderer}}

Идея \ed{бескорыстного}{бескорыстный}{selfless, disinterested; here бескорыстное гостеприимство: selfless hospitality} гостеприимства родилась \explain{задолго до}{long before} появления интернета и \explain{восходит к}{dates beack to} эпохе хиппи. Не обремененные деньгами и стабильностью, они не имели средств на гостиницы и путешествовали в основном \explain{автостопом}{hitchhiking}. Чтобы не остаться без крыши над головой, хиппи вели так называемые рингушники — записные книжки с адресами и телефонами людей в разных городах, у которых можно было бесплатно погостить некоторое время.

Рингушники были \explain{едва ли}{hardly} не самой важной и постоянной частью жизни хиппи — их тщательно пополняли новыми адресами, а имена \ed{надёжных}{надёжный}{reliable} людей передавали из рук в руки. Позже привычку вести такие записные книжки переняли автостопщики, студенты и все, кто хотел объехать мир за копейки. Все изменилось с появлением интернета.

В основе создания сервиса CouchSurfing лежит красивая история об американце Кейси Фентоне, который отправился в путешествие мечты. В 1999 году 21-летний молодой человек, выросший в семье хиппи, купил дешёвые авиабилеты из Бостона в Исландию. \ed{Осознав}{осознав (осознать)}{realising}, что денег на гостиницы не хватает, молодой человек проявил \explain{изобретательность}{ingenuity, inventiveness}. Он раздобыл базу email-адресов Исландского университета и \explain{разослал}{sent out} сотню писем с просьбой приютить его бесплатно.

К удивлению Фентона, на его письмо \explain{откликнулись}{Откл\'{и}кнуться (откл\'{и}кнусь, откл\'{и}кнешься, откл\'{и}кнутся) / Отклик\'{а}ться (отклик\'{а}юсь, отклик\'{а}ешься, отклик\'{а}ются): to respond (e.g., to email)} десятки студентов. Добрые исландцы даже составили ему готовый план поездки, согласно которому он должен был перемещаться по стране от дома к дому. \explain{Воодушевлённый}{encouraged (by)} исландским \ed{радушием}{радушие}{cordiality}, Фентон решил подарить другим путешественникам такой же опыт.

После возвращения домой он зарегистрировал домен и придумал название проекта. Слово CouchSurfing буквально означает «путешествие по чужим койкам» и состоит двух частей: couch, что в переводе с английского означает диван и surfing — кататься на волне или \explain{странствовать}{wander}. Именно поэтому в народе каучсерферов стали называть диванными скитальцами.

\begin{fancyquotes}
    Словарь каучсерфера\\

    \textit{Хост} — хозяин, принимающий у себя путешественников\\

    \textit{Реквест} — запрос о ночлеге, который путешественник отправляет потенциальному хосту\\

    \textit{Вписка} — непосредственно сам ночлег\\

    \textit{Серфер} — человек, который ищет бесплатное жилье в поездке
\end{fancyquotes}

Пробную версию сайта запустили в 2004 году, в разработке Фентону помогали друзья и единомышленники. За первый год на платформе зарегистрировались 6000 пользователей, на следующий — уже 20 тысяч. CouchSurfing быстро завоевал популярность у бэкпэкеров сначала в Европе и США, а потом стал активно использоваться в Азии, Африке и Латинской Америке.

Спустя почти 20 лет после запуска популярность сервиса продолжает расти. Если верить данным «Википедии», на нем зарегистрировано 12 миллионов пользователей более чем из 200 тысяч населенных пунктов по всему миру — даже на далеком острове Пасхи путешественников готовы принять почти 60 хостов.

Слово «каучсерфинг» плотно вошло в обиход и стало нарицательным — сейчас так называют формат цифрового гостеприимства, когда жители какой-либо страны или города предоставляют безвозмездный ночлег путешественникам.

Помимо непосредственно самого сайта CouchSurfing.com существует несколько аналогичных сервисов. На некоторых при регистрации нужно платить символический членский взнос, где-то — проходить собеседование. Самыми известными альтернативами считаются BeWelcome, Servas International и Trustroots. Есть и более нишевые проекты: например, Warm Showers создан для велопутешественников, а Pasporta Servo — для любителей искусственно созданного языка эсперанто.

Найти хоста также можно на форумах или в социальных сетях — во «ВКонтакте» и на Facebook (запрещена в России; принадлежит компании Meta, которая признана экстремистской организацией и также запрещена в стране) есть множество тематических групп для поиска ночлега в отдельных странах и городах. Однако опытные путешественники предупреждают, что такой способ «вписаться» не всегда безопасен. Если CoushSurfing и подобные сервисы стремятся обеспечить максимальную прозрачность путем верификации учетной записи и системы оценивания хостов, то в соцсетях можно с легкостью нарваться на людей с не самыми чистыми намерениями.

Тем не менее, несмотря на все предосторожности и протоколы безопасности CouchSurfing, ночлег у незнакомых людей может иметь неприятные последствия. Одна из нашумевших историй произошла в 2009 году. Туристка из Гонконга приехала в Великобританию и договорилась о «вписке» в Лидсе — хостом оказался мужчина марокканского происхождения. Он два раза изнасиловал девушку и едва не убил ее. Мужчину приговорили к десяти годам тюрьмы.

Другой случай с CouchSurfing, поставивший под сомнение репутацию сервиса, произошел в Италии в 2014 году. Представившись выдуманным именем, местный житель пригласил в гости 16-летнюю девушку-подростка из Австралии, накачал ее наркотиками и изнасиловал. В ходе судебного разбирательства выяснилось, что мужчина не раз пользовался доверчивостью туристок и заманивал их к себе домой с помощью анкеты на CouchSurfing.

\textbf{Правила хорошего тона}

Страх столкнуться с неадекватным хостом или вовсе нарваться на маньяка часто отпугивает туристов от каучсерфинга. Чтобы максимально себя обезопасить, необходимо придерживаться нескольких принципов. Самое базовое — это внимательно изучить профиль хоста, прочитать отзывы и поделиться его контактами со своими близкими.

\begin{center}
    \Large
    Когда в профиле мужчины напрямую говорится о желании принимать только девушек, а в графе интересов стоит секс — есть повод насторожиться. Также стоит заранее пообщаться с хостом: выяснить, где придется спать (в отдельной комнате или вместе с хозяином), и попросить прислать фото
\end{center}

Как рассказала «Ленте.ру» клинический психолог, эксперт в сфере домашнего насилия, посттравматического стрессового расстройства и зависимостей Олеся Иневская, важно избегать ситуации зависимости от хоста. «У вас должны быть деньги на запасной вариант ночлега не только на случай конфликта или странного поведения хоста, но и на случай отказа. У хоста могут случиться непредвиденные ситуации непосредственно перед заселением», — посоветовала эксперт.

Кроме того, в среде каучсерферов есть негласные правила хорошего тона, соблюдение которых сделает пребывание в чужом доме комфортным. В каучсерфинге все находятся на равных, поэтому важно не злоупотреблять гостеприимством и не нарушать личные границы. Брать продукты без спроса, надолго занимать ванну и приходить в дом в грязной одежде — не лучший способ найти общий язык с хостом.

\begin{fancyquotes}
    Необходимо узнать культурные особенности и обсудить детали проживания: распорядок дня, бытовые условия (как хозяин дома потребляет воду и электричество), количество ночей и время вашего отъезда. Предвидеть, что может стать нарушением границ, практически невозможно

    \begin{flushright}
        Олеся Иневская, клинический психолог
    \end{flushright}
\end{fancyquotes}

Беспроигрышный вариант поладить с хозяином дома — привезти сувениры из родной страны или города, особенно ценятся съедобные подарки. «Предложите хосту совместно приготовить ужин национальной русской кухни, расскажите о традициях. Это станет хорошим выражением благодарности за то, что человек принимает вас у себя дома», — рекомендует практический психолог и член ассоциации когнитивно-поведенческих психотерапевтов Татьяна Сушкова.

Несколько россиян поделились с «Лентой.ру» своим опытом каучсерфинга и рассказали о самых курьезных и жутких случаях.

\textbf{«Буду мыть за вас посуду и травить байки о Путине»}

\textit{Василий, объехал по каучсерфингу две страны}

Началось все с путешествия по Испании. Я был классическим бедным студентом, который решил попутешествовать по Европе. Выбор пал на Испанию, которая была бюджетнее других стран ЕС. За три недели я побывал в Овьедо, Бильбао и Барселоне. Скачав приложение CauchSurfing, я столкнулся с проблемой. Так как я был новым пользователем с нулевым рейтингом, на мои запросы никто не отвечал. Чтобы привлечь внимание хостов, я обещал привезти из России угощения и рекламировал себя в качестве уборщика. К моему удивлению, в Барселоне меня приютили после шуточного предложения рассказывать анекдоты о российском президенте.

Второй страной, где я ни разу не бронировал гостиницу, стала Бельгия. Это маленькое государство, поэтому я решил найти «вписку» где-нибудь в центре и каждый день ездить в новый город. Мой выбор пал на Гент — небольшой средневековый городок с каналами и удобным расположением с точки зрения железных дорог.

Моим хостом стал пожилой фермер-фламандец Ксандр, который почти не говорил по-английски. Он поселил меня в крытый загон для лошадей, который уже не использовался и был оборудован для приема гостей. Хотя к сырости и не совсем приятному запаху пришлось привыкать, там было все необходимое для жизни: кровать, полки для вещей, туалет и даже что-то, напоминающее душ. По утрам Ксандр устраивал завтраки с фермерскими продуктами и в целом оказался очень милым стариком.

Для меня каучсерфинг — это история в первую очередь про авантюризм, ведь ты никогда не знаешь, что будешь делать. Хост может провезти тебя по тем местам, куда не добираются обычные туристы. В том же Генте хост со своим другом устроили мне бесплатную экскурсию по каналам на яхте.

\textbf{«Он напился и лег ко мне в кровать»}

\textit{Инна, убежала от хоста посреди ночи}

Я пользовалась сервисом CauchSurfing.com один раз, когда в 22 года поехала в одиночное путешествие во Францию. Скажу сразу — в этой поездке сбылись мои самые худшие опасения, все не задалось с самого начала. Я договорилась с мужчиной средних лет из Конфлан-Сент-Онорина (пригород в 30 километрах от Парижа) о «вписке» на два дня.

Мы условились встретиться возле его дома в определенное время, но он опоздал на полтора часа. Француз Анри жил в уютной чистенькой студии. Единственное, что меня смущало. — это перспектива спать с ним в одной комнате, пусть и не в одной кровати. Он любезно уступил мне большой раскладной диван, а сам устроился на раскладушке.

\begin{center}
    \Large
    Кошмар начался на второй день, когда он вернулся далеко за полночь с какой-то вечеринки, едва стоя на ногах. Я проснулась от резкого запаха перегара — Анри лежал почти вплотную ко мне и спал. Решение разбудить его оказалось ошибкой
\end{center}

Он не домогался меня и не пытался изнасиловать, но вел себя крайне неадекватно: кричал что-то на французском, ходил по квартире, открывал и закрывал окна. Мне стало банально страшно оставаться с пьяным человеком на 30 квадратных метрах, поэтому я собрала рюкзак и ушла. Было почти пять утра, пришлось ждать открытия кафе, чтобы позавтракать и спокойно умыться. Днем у меня был поезд в Нант, где меня ждала койка в хостеле. С тех пор каучсерфингом я не пользовалась.

\textbf{Не ради денег}

Если с туристами, которые хотят сэкономить на отелях и глубже погрузиться в другую культуру, все понятно, то что движет людьми, готовыми впустить в дом совершенно незнакомых людей?

По словам Вадима из Санкт-Петербурга, который за четыре года принял у себя 13 человек, каучсерфинг — это окно в мир. «До пандемии у меня останавливалось много иностранцев. Тогда я не мог позволить себе отдыхать за границей, поэтому это была возможность близко познакомиться с чужими культурами. Еще один плюс — ты начинаешь по-другому смотреть на свой город. У меня жила пара французов, которая попросила свозить их в Кронштадт. Сам я был там последний раз на экскурсии в седьмом классе и был приятно удивлен, как там все изменилось», — поделился он в беседе с «Лентой.ру».

\begin{fancyquotes}
    Приятный бонус — подарки от путешественников. Как-то два испанца привезли мне три килограмма хамона в знак благодарности
\end{fancyquotes}

Другие опытные хосты рассказывают, что принимают у себя людей ради практики иностранных языков. Например, Виктория из Новосибирска, закончившая институт востоковедения, успела «вписать» к себе 18 туристов из Китая.

«Конечно, неприятные случаи тоже бывали. Однажды, когда меня не было дома, парень из Харбина решил приготовить на моей кухне вок. Видимо, что-то пошло не так, и он испортил дорогую сковородку с антипригарным покрытием», — рассказывает девушка. Чтобы избежать подобных случаев, Виктория рекомендует сразу очертить путешественнику границы дозволенного. Например, брать книги, пользоваться кофемашиной и утюгом — можно, брать средства личной гигиены и продукты из холодильника — нет.

«Сразу видно людей, которые подались в каучсерфинг только ради халявы и воспринимают твой дом как бесплатный отель. Уже на этапе реквеста они много спрашивают про условия проживания и по минимуму говорят о себе. Скорее всего, такой человек даже не помоет за собой посуду, ведь будет считать тебя "персоналом"», — добавила Виктория.

Она подчеркивает, что к поиску «вписки» стоит подходить вдумчиво: не рассылать одинаковые заявки хостам, а делать их индивидуальными — благодаря этому каучсерфинг станет незабываемым опытом для обеих сторон, а не просто бесплатным ночлегом.

\begin{center}
    \Large
    Если хотите оставить после себя грязную посуду и скомканные простыни, идите на Airbnb
\end{center}

Как бы то ни было, каучсерфинг — гораздо больше, чем способ экономить в поездках, это целая философия путешествий и гостеприимства. Сейчас этот сервис может быть особенно полезен российским туристам, поскольку платить за отели за границей без карты иностранного банка невозможно, да и забронировать экскурсии по городу дистанционно не получится. В таких ситуациях хост может не только выручить с жильем, но и показать локальные достопримечательности и заведения.