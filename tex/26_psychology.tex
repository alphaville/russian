\chapter{Личность, Характер и Психология}
\section{Что такое думскроллинг}

\textit{и почему чтение плохих новостей пагубно влияет на ментальное здоровье?}

\textit{Источник: \url{https://www.elle.ru}}
% https://www.elle.ru/otnosheniya/psikho/chto-takoe-dumskrolling-i-pochemu-chtenie-plokhikh-novostei-pagubno-vliyaet-na-mentalnoe-zdorove/

\begin{fancyquotes}
    Постоянно просматриваете новости? Не выпускаете смартфон из рук? У нас для вас плохие новости — вы зависимы, и это пагубно влияет на ваше здоровье и умственные способности.
\end{fancyquotes}

Думскроллинг (от английского doom — «гибель, судьба, рок, Судный день» и scrolling — «прокрутка») — это склонность к просмотру и чтению плохих новостей, несмотря на то, что они вызывают негативные эмоции, удручают, огорчают и деморализуют. Термин стал относительно широко использоваться в начале пандемии и сейчас снова стал актуален на фоне нестабильной ситуации в мире. Так почему же мы не можем оторваться от плохих новостей и как это влияет на наше психоэмоциональное состояние?

Основная причина бесконтрольного думскроллинга — это боязнь пропустить важные новости. Беспокойный ум стремится понять, что происходит в мире и как это может коснуться лично нас. В таком случае думскроллинг дает ощущение контроля над ситуацией. Интуитивно мы пытаемся подготовиться к потенциальным угрозам. Принцип «предупрежден — значит вооружён» миллионы лет способствует выживанию людей как вида.

\textbf{Почему это вредно?}

Иллюзия контроля над ситуацией, по большому счету, не дает никаких преимуществ. Думскроллинг способствует развитию тревожности и стресса, повышается вероятность панических атак, снижается концентрация. Умные алгоритмы соцсетей предлагают все больше и больше плохих новостей, поиск и чтение пугающих статей превращается в зависимость, человек игнорирует собственные мысли и чувства. Впоследствии чтение негативных новостей может приводить и к ухудшению сна и истощению нервной системы.

\textbf{Как бороться с думскроллингом?}

Не пользуйтесь гаджетами перед сном. Не читайте новостей о коронавирусе, войнах, протестах и других тревожных явлениях на ночь. Если вам сложно контролировать это самостоятельно, то установите будильник и за 2-3 часа до сна переводите телефон в авиарежим.

Читайте только ту информацию, которая вам нужна, не переключайте внимание на другие новости. Перед тем, как читать новости и статьи в интернете, четко определите «цель визита», не обращайте внимание на предложенные статьи или кликбейтные заголовки.

Отвлекитесь от новостного контента. Попробуйте сместить фокус внимания с новостей на интересные статьи, интервью, рецензии.

Займитесь чем-то другим. Кино, музыка, встречи с друзьями — все это поможет провести вам время с куда большей пользой для нервной системы.

\newpage
\section{Синдром спасителя}

\textit{Как добрые намерения скрывают эмоциональные недостатки}

\textit{Что скрывает навязчивое желание помочь ближнему, если вас об этом никто не просит}

\textit{Источник: \url{https://www.elle.ru}}
% https://www.elle.ru/otnosheniya/psikho/sindrom-spasitelya-kak-dobrye-namereniya-skryvayut-emocionalnye-nedostatki/

Проявления синдрома спасителя не всегда очевидны для окружающих и даже тех, кого непрошеные благодетели окружают заботой. Импровизированные супергерои всегда готовы помочь нерасторопным коллегам, спасти близких (и не очень) людей от жизненных невзгод и пагубных пристрастий. Именно такие светлые человечки рады круглосуточно наставлять подшефных по вопросам правильного питания, карьерного роста и токсичных отношений. Ведомые убеждением, что в этом их высшее предназначение, «спасители» возводят альтруизм в культ, в действительности скрывая за самоотдачей эгоизм и невротические изъяны.

Термин «спаситель» используется психологами с 1968 года, с тех пор, как доктор Стивен Карпман, ученик Эрика Берна и знаток транзактного анализа, раскрыл в опубликованной работе модель социального и психологического взаимодействия, названную в его честь «треугольник Карпмана» (он же — «треугольник судьбы» или Karpman drama triangle). В статье Fairy Tales and Script Drama Analysis американский ученый описал три привычные роли, которые мы часто играем в разных ситуациях: жертва, преследователь и спаситель, который вмешивается, как кажется, из желания помочь тому, кого обижают или недооценивают.

Как разъяснил Карпман, ролевая игра, схожая с мелодраматическим сюжетом про «героя, злодея и девицу в беде», раскрывает неочевидный мотив: спаситель заинтересован поддерживать жертву в ее зависимости от себя. Догадываетесь почему?

\textbf{Как из заботы получается созависимость}

«Нужда помогать другим движет личностью, способной реализоваться исключительно в опеке других, — объясняет Анн-Виктуар Русселе, парижский психолог и терапевт. — Такие люди считают своим долгом спасать других в ущерб себе, включая тех, кто в этом абсолютно не нуждается. Они сознательно вступают в созависимые отношения, считая, что не заслуживают любви партнера, но убеждая себя, что эта связь оправдана желанием избавить ее/его от проблем. Настойчивая услужливость имеет в корне нарциссический изъян, скрывающий неуверенность в себе и сопутствующую мотивацию: потребность поднять самооценку. „Спаситель“ становится лучше в собственных глазах, проецируя на ближних позитивные намерения и поступки».

Чего же следует ожидать от непрошеной заботы? «Спасаемые» увиливают от спасения, не предлагая взамен долгожданной компенсации, что, естественно, отзывается в душе супергероя горьким разочарованием. «Я делаю все для всех, но никто никогда ничего не делает для меня», — типичная жалоба отвергнутого спасителя.

«Последствия амбивалентного синдрома непременно дадут о себе знать, — пишут калифорнийские психологи Мэри Ламиа и Мэрилин Кригер в книге „Синдром Белого Рыцаря: спасение себя от потребности спасти других“ (The White Knight Syndrome: Rescuing Yourself from Your Need to Rescue Other). — В начале отношений спаситель кажется удовлетворенным своей самоотверженностью, но со временем становится все более несчастным и бессильным. Она/он буквально выдыхается, теряя смысл, интерес, энергию, ресурсы, что, в свою очередь, отражается на самооценке. Убедившись, что усилия напрасны, «белый рыцарь» выходит из игры эмоционально и психологически истощенным».

\textbf{В чем (опять) виноваты детские травмы}

Потребность спасать указывает на эмоциональный и психологический дисбаланс, ноги которого предсказуемо растут из воспитания, образования и внушенных ценностей. «Она/он спасает всех вокруг, стараясь быть „хорошей девочкой/мальчиком“, чтобы получить одобрение со стороны реального или внутреннего родителя и укрепить самооценку, — описывает природу травмы доктор Русселе. — Возможно, в детстве „спасителю“ приходилось помогать больной матери, заботиться о братьях и сестрах, с раннего возраста посвящая себя нуждам взрослых и опуская свои потребности в нижний ранг приоритетов».

Пресловутый дисбаланс имеет тенденцию преследовать «спасителя» до зрелости, поддерживая в ней/нем потребность окружить себя партнерами, друзьями и коллегами, о которых нужно будет заботиться. И эта иллюзорная стратегия предсказуемо обречена на неудачу, потому что суть всего успешного базируется на гармонии.

\textbf{Как избавиться от синдрома спасителя}

От самоотверженного поведения никто не застрахован, но, к счастью, существуют когнитивные методики, которые помогут скорректировать психологический разлад. «Чтобы избавиться от непреходящего стремления быть кому-то нянькой, бросьте все силы на прокачку самоуважения и любви к себе, — призывают Мэри Ламиа и Мэрилин Кригер. — «Спасителям» надо усилием воли сменить тактику, признав наконец, что их любят не за сервис, который они обеспечивают, а за то, кто они на самом деле».

Увязшим в «спасающем» паттерне непросто пойти на поправку — изменить отношение к себе и окружающему миру мешает страх перед возможным одиночеством. А что, если опекаемые отвернутся и забудут насовсем? А что, если вместе с заботами о других из жизни исчезнет смысл?

Чтобы заблокировать страхи, доктор Русселе советует свести помощь к «контрактному» формату, а попросту — договориться. «Если хотите подстраховать себя от разочарований, вместо того чтобы помогать без спроса, обсудите напрямую, чем вы можете быть полезны для конкретного человека. Так вы заранее поймете, готовы ли оказывать услуги без ожиданий безграничной признательности и вдобавок проявите реальную заботу о себе — в кои-то веки. К тому же это хорошая практика в соблюдении личных границ, на которые каждый из нас имеет право».

\newpage
\section{Мой характер}
\textit{Источник: \url{https://ru4.ilovetranslation.com/yuYUND7JV3L=d/}}

О себе говорить приятно, но немного трудно. Приятно, потому что всем нравится говорить о своих интересах, вкусах и предпочтениях. Но это в то же время трудно, так как изучить человека, особенно себя самого, не так уж просто.

Прежде чем говорить о своем характере, хотелось бы сначала уточнить, что такое характер. Человек отличается от остальных своими качествами. Часто люди говорят, что я не такой как остальные. Но я не считаю, что я какой-то особенный. В темноте все кошки серые. Но если вы подойдете ближе и включите свет, вы увидите, что мне присущи определенные черты.

Но не будем вдаваться в подробности, и немного сократим рассказ. У меня хорошее чувство юмора, я ответственный, трудолюбивый и эмоциональный человек. Мне нравится творчество, и я ценю эту черту в других людях. Я не люблю ложь и чувствую, когда другие лгут.

Я стараюсь никогда не опаздывать и \explain{терпеть}{to brook} не могу, когда другие не приходят \explain{вовремя}{on time}. Я предпочитаю общаться с умными и вежливыми людьми. \explain{Досадно}{it's annoying}, когда тот, кому ты \explain{доверяешь}{доверять/доверить: to trust}, оказывается \explain{ненадежным}{ненадежный: unreliable} человеком.

Я стараюсь \explain{обращаться}{to treat / обратиться} с другими так, как я хотел бы, чтобы они обращались со мной. Я ищу человека со здоровым и сильным ум\'{о}м и телом. Человека, с которым интересно общаться, которому я могу доверять и на кого можно положиться.

Что касается моих интересов, мне нравится психология в плане общения с людьми, а также способа формирования мыслей наилучшим образом. Я очень люблю путешествовать, встречаться с новыми людьми, знакомиться с их традициями и обычаями, их культурой, смотреть достопримечательности. Мне также нравятся разные стили музыки, нравится ритмичная музыка, под которую можно танцевать.

\newpage
\section{Психоанализ Зигмунда Фрейда}

\textit{Предпосылки и базовые идеи за 5 минут}

\textit{Источник: \url{https://bit.ly/3sfBWgd}}

Изучением психики человека уже не один десяток лет занимаются великие умы, но на многие вопросы ответов до сих пор нет. Что скрывается в глубинах человеческого существа? Почему события, произошедшие когда-то в детстве, по сей день оказывают влияние на людей? Что заставляет нас совершать одни и те же ошибки и мёртвой хваткой держаться за опостылевшие отношения? Где берут своё начало сновидения, и какая информация в них заложена? На эти и множество других вопросов, относительно психической реальности человека, может ответить революционный и поправший собой многие основы психологии психоанализ, созданный выдающимся австрийским учёным, неврологом и психиатром Зигмундом Фрейдом.

\textbf{Как появился психоанализ?}

В самом начале своей деятельности Зигмунд Фрейд успел поработать с выдающимися учёными своего времени – физиологом Эрнстом Брюкке, практикующим гипноз врачом Иосифом Брейером, неврологом Жаном-Маре Шарко и другими. Часть мыслей и идей, которые зародились на этом этапе, Фрейд развивал и в своих дальнейших научных трудах.

Если говорить более конкретно, то ещё молодого тогда Фрейда привлекло то, что некоторые симптомы истерии, проявлявшиеся у больных ею, не могли никак быть интерпретированы с физиологической точки зрения. К примеру, человек мог ничего не чувствовать в одной области тела, несмотря на то, что в соседних областях чувствительность сохранялась. Ещё одним доказательством того, что далеко не все психические процессы могут быть объяснены реакцией человеческой нервной системы или актом его сознания, было наблюдение за поведением людей, которые подвергались гипнозу.

Сегодня все понимают, что если находящемуся под гипнозом человеку внушить приказ что-либо выполнить, после своего пробуждения он бессознательно будет стремиться к его выполнению. А если поинтересоваться у него, почему он хочет это сделать, он сможет привести вполне адекватные объяснения своему поведению. Отсюда и получается, что психика человека имеет свойство самостоятельно создавать объяснения каким-то поступкам, даже если в них нет никакой необходимости.

В современность Зигмунда Фрейда само понимание того, что действиями людей могут управлять скрытые от их сознания причины, стало шокирующим откровением. До исследований Фрейда таких терминов как «подсознательное» или «бессознательное» не было вовсе. И его наблюдения стали отправной точкой в развитии психоанализа – анализа человеческой психики с позиции движущих ею сил, а также причин, последствий и воздействия на последующую жизнь человека и состояние его нервно-психического здоровья опыта, полученного им в прошлом.

\textbf{Базовые идеи психоанализа}

Теория психоанализа зиждется на том утверждении Фрейда, что в психической (если удобнее – душевной) природе человека не может быть непоследовательности и перерывов. Любая мысль, любое желание и любой поступок всегда имеет свою причину, обусловленную сознательным или бессознательным намерением. События, имевшие место в прошлом, влияют на будущие. И даже если человек убеждён, что какие-либо его душевные переживания не имеют оснований, всегда присутствуют скрытые связи между одними событиями и другими.

Исходя из этого, Фрейд разделял психику человека на три отдельные области: область сознания, область предсознания и область бессознательного.

\begin{enumerate}
    \item К области бессознательного относятся бессознательные инстинкты, никогда не доступные сознанию. Сюда же можно отнести вытесненные из сознания мысли, чувства и переживания, которые воспринимаются сознанием человека как не имеющие права на существование, грязные или запрещённые. Область бессознательного не подчиняется временным рамкам. Например, какие-то воспоминания из детства, вдруг снова попав в сознание, будут такими же интенсивными, как и в момент своего появления.
    \item К области предсознания относится часть области бессознательного, способная в любой момент стать доступной для сознания.
    \item Область сознания включает в себя всё то, что осознаёт человек в каждый момент своей жизни.
\end{enumerate}

Основными действующими силами человеческой психики, согласно идеям Фрейда, являются именно инстинкты – напряжения, которые направляют человека к какой-либо цели. И эти инстинкты включают в себя два главенствующих:

\begin{enumerate}
    \item Либидо, являющееся энергией жизни
    \item Агрессивная энергия, являющаяся инстинктом смерти
\end{enumerate}

Психоанализ рассматривает, по большей части, либидо, в основе которого лежит сексуальная природа. Оно представляет собой живую энергию, характеристики которой (появление, количество, перемещение, распределение) могут истолковать любые психические расстройства и особенности поведения, мыслей и переживаний индивида.

Личность человека, согласно психоаналитической теории, представлена тремя структурами:
\begin{enumerate}
    \item Оно (Ид)
    \item Я (Эго)
    \item Сверх-Я (Супер-Эго)
\end{enumerate}

Оно (Ид) является всем изначально заложенным в человеке – наследственностью, инстинктами. На Ид никак не влияют законы логики. Его характеристики — это хаотичность и неорганизованность. Но Ид воздействует на Я и Сверх-Я. Причём, его воздействие безгранично.

Я (Эго) является той частью личности человека, которая находится в тесном контакте с окружающими его людьми. Эго берёт своё начало из Ид с того самого момента, когда ребёнок начинает осознавать себя как личность. Ид питает Эго, а Эго защищает его, словно оболочка. То, как взаимосвязаны Эго и Ид, легко отобразить на примере потребности в сексе: Ид могло бы осуществить удовлетворение этой потребности посредством прямого сексуального контакта, но Эго решает, когда, где и при каких условиях этот контакт может быть реализован. Эго способно перенаправлять или сдерживать Ид, тем самым являясь гарантом обеспечения физического и душевного здоровья человека, а также его безопасности.

Сверх-Я (Супер-Эго) произрастает из Эго, являясь хранилищем моральных устоев и законов, ограничений и запретов, которые накладываются на личность. Фрейд утверждал, что Сверх-Я выполняет три функции, коими являются:
\begin{enumerate}
    \item Функция совести
    \item Функция самонаблюдения
    \item Функция, формирующая идеалы
\end{enumerate}

Оно, Я и Сверх-Я необходимы для совместного достижения одной цели – поддержания равновесия между стремлением, ведущим к увеличению удовольствия, и опасностью, возникающей от неудовольствия.

Возникшая в Оно энергия отражается в Я, а Сверх-Я определяет границы Я. Учитывая то, что требования Оно, Сверх-Я и внешней реальности, к которой должен приспособиться человек, нередко являются противоречивыми, это неизбежно приводит к внутриличностным конфликтам. Решение же конфликтов внутри личности происходит посредством нескольких способов:
\begin{enumerate}
    \item Сновидения
    \item Сублимация
    \item Компенсация
    \item Блокировка механизмами защиты
\end{enumerate}

Сновидения могут быть отражением желаний, не реализованных в реальной жизни. Сновидения, которые повторяются, могут быть указателями на определённую потребность, которая не была реализована, и которая может служить помехой на пути свободного самовыражения человека и его психологического роста.

Сублимация является перенаправлением энергии либидо на цели, одобряемые обществом. Нередко такими целями выступает творческая, социальная или интеллектуальная деятельность. Сублимация есть форма успешной защиты, а сублимированная энергия создаёт то, что все мы привыкли называть словом «цивилизация».

Состояние тревожности, которое возникает от неудовлетворённого желания, есть возможность нейтрализовать через прямое обращение к проблеме. Так, энергия, которая не может найти выхода, будет направлена на преодоление препятствий, на уменьшение последствий этих препятствий и на компенсацию того, чего не хватает. В качестве примера можно привести идеальный слух, который развивается у слепых или слабовидящих людей. Человеческая психика способна поступить аналогичным образом: к примеру, у человека, страдающего недостатком способностей, но имеющего сильнейшее желание достичь успеха, может развиться непревзойдённая работоспособность или беспримерная напористость.

Однако бывают и такие ситуации, в которых появившееся напряжение может быть искажено или отвергнуто особыми защитными механизмами, такими как гиперкомпенсация, регрессия, проекция, изоляция, рационализация, отрицание, подавление и другими. Например, неразделённую или потерянную любовь можно подавить («Не помню никакой любви»), отвергнуть («Да любви и не было»), рационализировать («Те отношения были ошибкой»), изолировать («Мне не нужна любовь»), спроецировать, приписав другим свои чувства («Люди не умеют любить по-настоящему»), гиперкомпенсировать («Я предпочитаю свободные отношения») и т.д.

\textbf{Краткое резюме}

Психоанализ Зигмунда Фрейда – это величайшая попытка прийти к пониманию и описанию тех составляющих психической жизни человека, которые до Фрейда были непостижимыми. Самим же термином «психоанализ» в настоящее время называют:

\begin{enumerate}
    \item Научную дисциплину
    \item Комплекс мероприятий по исследованию психических процессов
    \item Методику лечения нарушений невротического характера
\end{enumerate}


Работа Фрейда и его психоанализ даже сегодня нередко критикуются, однако те понятия, которые он ввёл (Ид, Эго, Супер-Эго, механизмы защиты, сублимация, либидо) понимаются и применяются в наше время как учёными, так и просто образованными людьми. Психоанализ нашёл своё отражение во многих науках (социологии, педагогике, этнографии, антропологии и других), а также в искусстве, литературе и даже кинематографе.

\newpage
\section{Номофобия: позвони мне, позвони...}

\textit{Источник: \url{https://4brain.ru/blog/nomofobiya-pozvoni-mne-pozvoni/}}

Когда-то весь мир был театром, а люди в нем – актерами. Сейчас весь мир превратился в Интернет, и Интернет поместился в телефон, а люди стали просто юзерами.

Насколько это хорошо или плохо, что с этим делать и нужно ли что-то с этим делать в принципе, вы сможете ответить, если пройдете наши программы «Когнитивистика» и «Психическая саморегуляция». А наша сегодняшняя тема – номофобия.

\textbf{Что такое номофобия: немного истории}

Слово «номофобия» пришло к нам из английского языка. Термин «nomophobia» происходит от выражения no mobile-phone phobia, что означает страх остаться на какое-то время без мобильного телефона. Впервые термин «номофобия» был введен в статье Nomophobia is the fear of being out of mobile phone contact — and it’s the plague of our 24/7 age («Номофобия это боязнь остаться вне связи с мобильным телефоном – и это чума нашего века 24/7») [YouGov, 2008].

Тогда и была поднята новая на тот момент проблема – номофобия, а статья подводила итоги исследования, проведенного по заказу UK Post Office. В опросе приняли участие более двух тысяч человек, и оказалось, что 48% женщин и 58% мужчин испытывают беспокойство, если остаются без мобильной связи по какой-либо причине (забыли телефон, села батарея, нет сети и прочее) [YouGov, 2008].

Можно сказать, что номофобия – это зависимость от телефона, потому что без телефона номофоб себя ощущает крайне некомфортно. Считается, что номофобия появилась в распоряжении людей одновременно с мобильным телефоном, хотя ее истоки отчетливо прослеживаются в намного более раннем периоде. Будет правильнее сказать, что боязнь отойти далеко от телефона появилась одновременно с телефонами.

С появлением первых стационарных телефонов в серьезных организациях секретарши крупных и средних начальников боялись выскочить на 5 минут в «дамскую комнатку», потому что по закону подлости именно в эти 5 минут звонил их непосредственный начальник.

В «доцифровую» эпоху в разных конторах от ЖЭКа и почты до городской справочной и центральной прачечной сидящая «на телефоне» сотрудница должна была попросить кого-то подежурить «у аппарата», пока у нее перекур, чтобы не оставить без ответа звонки, поступившие в данный отрезок времени.

Если же подменить «человека на телефоне» было некому, решившийся покинуть свой «боевой пост» в страхе молился, чтобы никто не позвонил в тот момент, пока он курит или заваривает себе чай, потому что не взятая после третьего гудка трубка приравнивалась к отсутствию на рабочем месте без уважительных причин.

Такая «ноуфонфобия» поддерживалась жесткой трудовой дисциплиной и санкциями за прогулы, прописанными в Трудовом законодательстве. Сейчас большинство организаций обзаводятся цифровыми каналами связи, а не дозвонившийся по телефону может оставить сообщение в чате и дождаться ответа в асинхронном режиме.

Поэтому сегодня «телефонобоязнь» распространена, разве что, в совсем консервативных организациях, где пока нет системы контроля продуктивности, основанной на конкретных показателях работы, а присутствие на рабочем месте остается единственным мерилом ценности сотрудника.

Зато появилась другая напасть: теперь люди неуютно себя чувствуют, если забыли дома мобильник, когда собирались в магазин за хлебом. Или если долго беседуют с домочадцами, и уже полчаса как не проверяли сообщения «в телефоне». А если человек уехал на работу и обнаружил отсутствие гаджета в кармане уже сидя за рулем, это полностью уважительная причина, чтобы вернуться домой за смартфоном.

На самом деле, если вы ждете важного звонка или сообщения, такое поведение полностью оправдано. И если ваша работа обязывает вас быть всегда на связи, сидите ли вы за компьютером, за рулем или вышли покурить, телефон лучше держать при себе. Равно как нежелательно оставлять свой смартфон без присмотра, если там собраны все ваши приложения и аккаунты с сохраненными паролями для входа. Хотя тут лучше бы продумать дополнительные меры кибербезопасности.

А вот если ничего важнее «котиков» в соцсетях и сплетен от подружек у вас сегодня не предвидится, а вы все равно ни на минуту не можете отвлечься от телефона, тут впору говорить про киберлофинг, фаббинг, а то и про номофобию. Как понять, что у человека номофобия – боязнь остаться без телефона, и как отличить ее от объективной необходимости быть на связи? Как отличить номофобию или зависимость от телефона от простого желания быть в курсе событий? Давайте поговорим об этом.


\textbf{Номофобия: симптомы и последствия}

Итак, каковы же симптомы номофобии? Психоневрологи выделяют несколько четких сигналов, позволяющих считать, что тяга к телефону обрела болезненный характер [Н. Гартман, 2019].

Топ-5 признаков номофобии:

\begin{enumerate}
    \item Волнение, нарастающее по мере снижения уровня заряда аккумулятора.
    \item Постоянная проверка наличия новых писем, сообщений, оповещений.
    \item Постоянное обновление ленты новостей и чтение одних и тех же новостей «по кругу».
    \item Постоянное присутствие в соцсетях, мессенджерах и на прочих онлайн-платформах для общения.
    \item Сильная боязнь испортить гаджет.
\end{enumerate}

Все эти признаки часто сопровождаются явными физическими проявлениями, такими как волнение, тревога и даже нервный тик и тремор в конечностях, если в какой-то момент вдруг не окажется под рукой гаджета, а человек не сможет быстро вспомнить, где он его оставил [Н. Гартман, 2019].

Особо мнительные и психически неустойчивые люди подвержены еще более сильным физиологическим реакциям: усиленное сердцебиение, потоотделение, панические атаки, потеря ориентации в пространстве [Н. Копылова, 2021].

В очень тяжелых случаях возможны и долгосрочные последствия номофобии:
\begin{enumerate}
    \item Усталость и бессонница.
    \item Ослабление коммуникативных навыков.
    \item Склонность к социопатии и нелюдимость.
    \item Грубость и агрессия.
    \item Снижение когнитивных способностей (память, интеллект, концентрация внимания).
    \item Эмоциональная холодность.
    \item Неспособность выразить свои чувства словами.
\end{enumerate}

Почему так происходит? Почему современные люди часто впадают в жесткую зависимость от гаджета? Давайте разберемся и с этим.

\textbf{Номофобия: причины}

На самом деле, тему зависимости от телефона и прочих гаджетов наиболее ярко описывает шутливый диалог, когда юноша сообщает своему приятелю, что купил крутейший смартфон, который умнее человека. В ответ на сомнения и возражения, что такого не может быть, счастливый обладатель умного устройства поясняет, что такое вполне может быть, ведь не зря же он отдал за смартфон 300 тысяч рублей. В этот момент приятель соглашается, что тогда да, смартфон может быть умнее человека, только дело тут не в смартфоне…

Справедливости ради отметим, что в номофобию иногда впадают и вполне образованные люди, а не только малограмотные особи, которых ничего не интересует, кроме пустопорожней болтовни и бессмысленных сообщений с кучей орфографических ошибок. Психотерапевты уже достаточно глубоко исследовали тему и готовы поделиться своим видением причин данного явления [В. Холманских, О. Демьянова, 2020].

Топ-5 причин номофобии:
\begin{enumerate}
    \item Нерешенные личные проблемы, от которых можно отвлечься с помощью телефона.
    \item Сложности с построением отношений и навыками коммуникации в офлайне.
    \item Желание презентовать себя в лучшем свете в виртуальном пространстве, что невозможно без помощи электронных инструментов.
    \item Желание чувствовать себя важным, нужным и осведомленным в любой момент времени.
    \item Страх изоляции – социальной, информационной, прочей.
\end{enumerate}

Заметим, что с началом пандемии страх изоляции обрел реальные очертания. Когда все и везде переходят на «удаленку», выключенный гаджет или телефон вне зоны доступа сродни изоляции от событий внешнего мира. И это касается абсолютно всех, а не только тех, у кого есть проблемы с отношениями и коммуникацией «в реале».

Однако и тут следует знать меру, потому что за 5 минут, которые вам необходимы, чтобы заварить чай, в жизни обычного офисного служащего вряд ли произойдет нечто судьбоносное, что требует сиюминутной реакции и не подождет вашего ответа, пока вы спокойно допьете чашечку горячего чая с бубликом или конфетой.

Для описания причин номофобии иногда используют такой термин, как «эскапизм», он же «эскепизм» или «эскейпизм». Под эскапизмом подразумевается сознательное избегание всего неприятного и рутинного, что есть в этой жизни, в том числе путем «зависания» в телефоне. Тогда каждый раз, когда у человека по какой-либо причине исчезает потенциальная возможность «погрузиться» в телефон, он испытывает беспокойство [В. Холманских, О. Демьянова, 2020].

Такой поход имеет право на жизнь в качестве одной из причин номофобии. Во-первых, причин номофобии гораздо больше, чем просто стремление постоянно отвлекаться от скуки бытия. Во-вторых, термин «эскапизм» гораздо шире, чем просто зависание в телефоне. Сюда же относится бегство от действительности путем погружения в творчество, чтение, размышления, духовные практики, какую-либо иную реальность, отличную от существующей вокруг.

Так ли все плохо с номофобией и может ли быть от нее какая-то польза? Давайте посмотрим.

\textbf{Номофобия: польза и вред}

Думается, вред от избыточной привязанности к телефону вполне очевиден из всего вышеизложенного. Тревожность, беспокойство, неприятные физиологические реакции, а в перспективе снижение памяти, концентрации внимания и прочие проблемы – это вполне достаточные основания говорить о том, что номофобия вредна.

Эти выводы подтверждают и целевые научные исследования среди различных социальных групп. Можем рекомендовать по теме номофобии статьи, опубликованные в ведущих научных изданиях.

Например, статью «Номофобия и опасность для здоровья: использование смартфонов и зависимость среди студентов вузов» («Nomophobia and Health Hazards: Smartphone Use and Addiction Among University Students») [A. Daei et al., 2019].

Или, к примеру, статью, посвященную «Последствиям чрезмерного использования мобильного телефона и психологическим рискам среди штатных медсестер» («Effects of Excessive Use of Mobile Phone and Psychological Hazards among Staff Nurses») [O. Swami et al., 2021]. Скажем прямо, что номофобия в той или иной степени затронула многие социальные группы, а наносимый данной фобией вред стал причиной пристального внимания ученых и медиков.

Однако есть исследователи, которые считают, что «Нет больше фобии о номофобии» («No more phobia about nomophobia») [CityU, 2019]. Так, группа южнокорейских ученых пришла к выводу, что страдающие номофобией люди сильнее вовлечены в работу и в большей степени переживают за результат, нежели те, кто покидает рабочий чат минута в минуту с окончанием рабочего дня. Как следствие, номофобы способны выполнить больший объем работы и добиться лучших результатов.

Но и в этом случае ученые признают, что все хорошо в меру, потому что перманентное пребывание на связи в режиме «25/8» чревато эмоциональным выгоранием и далее постепенным снижением продуктивности. Так или иначе, вреда от номофобии явно больше, чем пользы, поэтому от нее нужно вовремя избавляться, а лучше вовсе не доводить себя до такого состояния.

\textbf{Как победить номофобию?}

Сегодня можно найти массу советов по цифровому детоксу в целом, и борьбе с номофобией в частности. В большинстве случаев рекомендуется сразу переходить к ограничительным мерам: не пользоваться телефоном какое-то время (час, день, неделю), просматривать сообщения строго определенное количество раз в течение суток, удалить наиболее отвлекающие приложения, закладки, аккаунты в соцсетях.

Однако гораздо продуктивнее подход, рекомендующий сначала разобраться в причинах собственной номофобии, а уже потом переходить к каким-то действиям [Н. Копылова, 2021]. Точнее, сначала следует убедиться, что у вас есть признаки номофобии.

Выше мы уже говорили о том, как распознать номофобию, однако можем вам облегчить задачу, предложив ответить «да» или «нет» на следующие утверждения:
\begin{enumerate}
    \item Перед тем, как лечь спать, вы проверяете телефон.
    \item Первым делом после пробуждения вы проверяете телефон.
    \item Вы никогда не выключаете телефон.
    \item Вы паникуете, если осталось менее 30% заряда аккумулятора.
    \item Вы возвращаетесь, если забыли телефон дома, даже если вышли ненадолго.
    \item Вы носите телефон с собой везде, даже дома из комнаты в комнату.
    \item Вы стараетесь отвечать на все письма и сообщения моментально.
    \item Во время живой беседы и любого другого офлайн-занятия вы постоянно прерываетесь, чтобы проверить телефон.
    \item Вы нервничаете, если исчезает мобильная сеть или вай-фай.
\end{enumerate}

Каждый ответ «да» повышает вероятность того, что вас настигла номофобия. Если таких ответов больше трех, пора задуматься, что именно вас пугает в том, чтобы остаться без телефона на какое-то время. Вы ждете какое-то мегаважное сообщение? Однако вряд ли в вашей жизни так уж много ситуаций, когда должны свершиться какие-то суперсобытия.

Вы боитесь, что вас не найдет начальник, когда вы очень нужны? Основной массив офисной работы вполне допускает асинхронный режим общения и решения проблем. Лучше обговорить заранее, что вы будете перезванивать или отвечать на сообщение не позднее, чем в течение часа, чтобы не отвлекаться от переговоров с заказчиками. Заодно выиграете время на все случаи жизни, чтобы обдумать ответ.

Ваша жена устраивает вам сцены ревности, если вы не берете трубку сразу, как только она вам позвонила? Это достаточный повод, чтобы задуматься, насколько здоровые отношения вам удалось построить в семье, и принять меры, чтобы их оздоровить, пока не поздно.

И только после того, как вы справитесь с первопричиной вашего беспокойства, борьба с номофобией как таковой сможет принести результат. Итак, как же избавиться от телефонной зависимости?

Топ-7 способов справиться с номофобией:
\begin{enumerate}
    \item Заранее заряжать телефон, чтобы он не разрядился в неподходящий момент, и брать с собой пауэр-банк, если планируется длительное пребывание вне помещения.
    \item Завершать свои рабочие дела так, чтобы минимизировать вероятность внеплановых звонков в нерабочее время.
    \item Найти интересные увлекательные занятия в повседневной жизни (спорт, танцы, общение с друзьями), от которых не захочется отвлекаться на телефон.
    \item Освоить навыки тайм-менеджмента и выделить время, когда вы будете заниматься исключительно работой, домашними делами или хобби, не отвлекаясь на соцсети и уведомления.
    \item Отключить уведомления о событиях, которые не требуют мгновенной реакции (обновления в соцсетях, сообщения о рассылках и т.д.)
    \item Выделить своему телефону персональное место в квартире, на столе, на тумбочке, которое будет занимать только он, и больше никто и ничто.
    \item Разнообразить способы получения информации.
\end{enumerate}

По поводу последнего пункта уточним, что время можно узнать, посмотрев на часы на руке, книгу можно почитать бумажную, а не электронную, в выходной можно сходить в кино и не искать новый фильм на торрент-трекерах.

Стоит ли вам удалять все подряд приложения из телефона, чтобы они вас не отвлекали, решайте сами. В условиях санкций и ограниченной доступности обновлений это может оказаться не лучшим шагом, если вдруг окажется, что то или другое приложение вам все-таки нужно.

Мы уже говорили о том, что все хорошо в меру, и борьба с номофобией тоже. Избыточная требовательность к себе и беспокойство по поводу того, что вы неидеально распоряжаетесь своим временем, ничуть не полезнее беспокойства по поводу того, чтобы всегда быть «на связи».

Если же вам не удается справиться с проблемой своими силами, есть смысл обратиться к психологу или психотерапевту. На сегодняшний день наработан достаточно объемный опыт психологической помощи людям, впавшим в какую-либо специфическую зависимость. В медицине расстройства такого рода носят название «Специфические (изолированные) фобии» и обозначаются кодом F40.2 [classinform, 2021].

При необходимости может быть назначено медикаментозное лечение. Обращаем ваше внимание, что диагностику и лечение может проводить только врач. Однако вы вполне можете поддержать свой организм в минимальных дозировках такими средствами, как настойка валерианы и витамины группы В [kb, 2019].

И, конечно, оптимальным вариантом будет профилактика и недопущение такого явления, как номофобия в свою жизнь. Отрадно, что этой теме начали уделять внимание в действующей системе образования уже на уровне средней школы.

Например, посвящен теме номофобии проект-исследование «Насколько мы зависимы от телефона» [Ю. Левкина, И. Рубель, 2020]. В ходе реализации данного проекта авторы попытались не только выявить масштабы зависимости от гаджетов среди школьников 7-11 классов, но и выработать меры, позволяющие соблюдать баланс между цифровой активностью и обычным общением, научиться пользоваться телефоном без ущерба для учебного процесса и не мешая окружающим.

Такой подход дает надежду, что со временем тема номофобии станет не столь актуальной, как сейчас, а люди, беря в руки телефон, будут пользоваться цифровыми благами мира и не чувствовать себя беспомощными, если Google вдруг не сможет ответить на их вопрос прямо сейчас.

Впрочем, весь мир театр – как был, так и остался. И пусть этот театр иногда виртуальный, по-настоящему активным людям это не мешает играть главную роль на подмостках своей жизни, не впадая в зависимость от чего бы то ни было.

Мы желаем, чтобы пользование гаджетами было для вас исключительно удобно, комфортно и с пользой для дела. Мы приглашаем вас на наши программы «Когнитивистика» и «Психическая саморегуляция». И просим вас ответить на один вопрос по теме статьи...

\newpage
\section{Профессия психолог: особенности и преимущества специальности}

\textit{Источник: \url{https://4brain.ru/}}

Человек по своей природе склонен даже в спокойное время постоянно испытывать разные внутренние психологические проблемы, такие как трудности в семье, недопонимание с близкими, неуверенность, чувство одиночества, страхи или сезонные депрессии. Что же говорить о периодах каких-то глобальных кризисов, происходящих в последнее время во всем мире…

Людей все больше охватывают страх и паника за свою жизнь, многие оказываются в стрессовых состояниях, которые крайне негативно влияют на здоровье, а со временем могут развиваться в разные фобии.

Имея огромное желание справиться со своими внутренними терзаниями, люди начинают понимать, что обычные беседы «по душам» им уже не помогают, а глубокие внутренние проблемы сможет решить только профессиональный психолог [need4study.com, 2018].

Во многих западных странах уже очень давно введена практика частной психологии. Многие влиятельные лица и знаменитости, а также простые люди давно имеют своих личных психологов, с которыми регулярно консультируются по тем или иным вопросам.

\textbf{Психолог: суть профессии}
«Психология» переводится с греческого языка как «наука о душе». С лингвистической точки зрения термины «психика», «душа» означают одно и тоже. Но со временем, в процессе развития культуры и науки, значения этих понятий разошлись.

Сегодня психология является в первую очередь дисциплиной, которая имеет научное определение, задачи, цели, объекты и все то, что делает ее наукой.

Профессия психолог нужна для изучения личностных черт людей, а также для понимания особенностей мышления и взаимосвязи с окружающими. Профессиональные специалисты помогают людям разными эффективными психологическими приемами как в личных отношениях, так и в профессиональной деятельности.

Роль психолога заключается в том, чтобы помочь человеку пройти через разные сложные ситуации в жизни и найти потенциал для того, чтобы двигаться дальше, используя для этого все необходимые техники. Опытный специалист даст совет как лучше поступить, и при этом он обязательно прислушается к индивидуальным потребностям и ценностям каждого клиента.

Положительные качества консультации в том, что человек, пришедший на нее, не просто пассивно принимает назначенные врачом методики, но и сам активно участвует в том, что происходит. Он исследует и познает собственные трудности и пытается выразить свои ощущения/мысли для дальнейшего тщательного анализа.

Психолог помогает прояснить какие-то события, определить связь между ними и поведением самого человека, старается узнать, как эти связи влияют на будущее и как отражались в прошлом. Другими словами, он помогает избавиться от ненужных переживаний, стать увереннее в себе при достижении поставленных целей и сделать произошедшие изменения максимально устойчивыми.

Хороший психолог всегда старается создать наиболее приемлемые условия для самовыражения человека. Он может большую часть времени просто выслушивать, но при этом всегда подмечает какие-то важные детали, которые обычно ускользают от простого взгляда. Поэтому психолог способен максимально прояснить и выразить свое видение на ту или иную проблему. Он может находить взаимосвязи между мыслями, чувствами и поступками людей, тем самым оказывая огромную помощь в достижении поставленных задач перед человеком.

Чем может помочь профессиональный психолог? Наиболее значимые пункты:
\begin{enumerate}
    \item лучше понимать своих близких;
    \item обрести гармонию со своим внутренним «я»;
    \item быстро находить выход из разных напряженных ситуаций;
    \item справляться с возникающими трудностями;
    \item  решать проблемы в личных отношениях;
    \item справляться с внутренней болью;
    \item избегать конфликтов;
    \item повышать качество жизни путем внутреннего самопознания;
    \item повышать самооценку и уверенность в собственных силах;
    \item бороться с тревогами, фобиями и депрессиями;
    \item преодолевать возрастные кризисы;
    \item решать трудности переходного возраста в подростковом периоде;
    \item избегать разногласий в воспитании детей и в отношениях с родителями;
    \item решать проблемы с коллегами в коллективе или с начальством;
    \item справляться с горем при потере/смерти близкого человека [my-self.ru, 2020].
\end{enumerate}

Многие люди выбирают психологию в качестве своей основной работы.

\textbf{Профессия психолог: специализации}

Итак, психолог. Эта профессия в последние годы стала очень популярной. Существует много разных вариантов деятельности для тех, кто получил психологическое образование, т.к. перед дипломированными специалистами всегда открыто больше дверей и возможностей.

Все больше специалистов этой области находят себя в разных современных направлениях, таких как бизнес-тренерство и коучинг. Часто они могут останавливать свой выбор на HR-службах. Наиболее популярными направлениями в последнее время являются следующие специальности.

\textbf{Клинический психолог}

Большинство людей данного профиля выбирает работу в сфере здравоохранения. Это может быть профессия клинического психолога или специалиста по судебной медицине. Сюда же относятся психоаналитики, различные консультанты служб психологической поддержки по телефону.

Деятельность клинического специалиста-психолога заключается в оказании помощи людям с разного рода зависимостями (алкоголизм, курение, наркотики). Сюда же входят нтернет-зависимость, игромания, психогенное переедание и т.д.

Клинический психолог работает с людьми, имеющими разные психологические отклонения/уязвимости, и может оказывать поддержку в реабилитационных службах, помогая тем, кто запутался в жизненных обстоятельствах, перенес тяжелые психотравмы, а также тяжелобольным и ВИЧ-инфицированным.

\textbf{Школьный психолог}

Очень востребованной является профессия педагог-психолог, особенно в частных учебных заведениях. Школьные специалисты способны помочь детям приспособиться к новым условиям за короткое время с наименьшими потерями для психики.

Школьный специалист на основе определенных тестов видит насколько ребенок готов к обучению. Также он проводит персональные беседы с трудными подростками с целью выявления отклонений.

В работу психолога, работающего с детьми, входит проведение различных тренингов, которые в дальнейшем помогают детям подобрать нужную для себя профессию.

Данный специалист оказывает большую помощь в развитии нормальных отношений в любых детских и подростковых коллективах.

\textbf{Семейный психолог}

Очень часто на консультацию к психологам приходят люди, имеющие проблемы в семье. Целью деятельности этого специалиста является работа с людьми, находящимися в долгих отношениях. Основная задача психолога здесь – помочь партнерам понять друг друга и постараться устранить все возможные непонимания и проблемы.

Консультации у опытного профессионала пригодятся на всю жизнь – люди смогут качественно проработать отношения, преодолеть собственный кризис, а также решить проблемы с другими членами семьи, включая детей и родителей, чтобы вывести отношения на новый уровень.

Хороший специалист по семейным вопросам отлично знаком с общей клинической картиной, понимает психологию развития, учитывает все тонкости отношений и применяет диагностику в совокупности со специальными методами терапии.

\textbf{Спортивный психолог}

Психология очень хорошо развита в спортивной среде. Здесь работа специалистов заключается в том, чтобы настроить спортсмена на победу и вселить уверенность в собственных силах. А также подготовить к соревнованиям без страха и лишних тревог.

Специалист проводит большую психологическую работу и помогает избавиться от неуверенности, поддерживает морально и обучает специальным методикам, способным быстро снимать стресс [www.kp.ru, 2020].

\textbf{Пенитенциарный психолог}

Работа данного специалиста заключается во взаимодействии с заключенными.

Целью деятельности этих психологов является оценка психического состояния осужденных, предупреждение любых попыток суицида, максимальная психологическая поддержка, а также помощь в адаптации к тюремным условиям жизни.

Суть профессии заключается в том, чтобы помогать заключенным изменить взгляды и отношения к жизни, а также подготовить их к выходу на свободу.

Хороший специалист данной сферы должен обладать смелостью, доброжелательностью, умением держать дистанцию, широким кругозором, харизмой, эмпатией и состраданием, а также желанием помогать людям [studika.ru, 2020].

\textbf{Психологи МЧС и МВД}

Специалисты данной области проводят психологическую оценку при приеме сотрудников на работу в силовые структуры, поскольку большинство людей, особенно молодой контингент, нуждаются в моральной поддержке. Ведь работа, так или иначе, связана с насилием.

Работа психологов необходима людям, пришедшим в структуры МВД после того, когда они не сумели найти работу по специальности, а здесь прошли сокращенные курсы за 6 месяцев и получили в руки оружие. Главная задача такого специалиста заключается в том, чтобы убедить сотрудника в правильности своего решения, успокоить его и настроить на дальнейшую работу.

Психологи МВД занимаются изучением социально-психологического климата в коллективе и дают качественные рекомендации по его улучшению. Также они консультируют и помогают адаптироваться военным и людям, вернувшимся недавно из горячих точек [moluch.ru, 2019].

Цель психологов МЧС состоит в том, чтобы помочь людям справиться с сильными переживаниями и горем, возникающими после гибели близких людей или при чрезвычайных ситуациях.

Их задача – выводить людей из острого стрессового состояния, общаться с пострадавшими и успокаивать родственников. Иногда приходится работать и вести переговоры даже с потенциальными самоубийцами.

Другое направление заключается в том, чтобы реабилитировать сотрудников МЧС, проводить с ними большую работу по недопущению профессионального выгорания и развитию моральной устойчивости [httpsvc.ru].

\textbf{Корпоративные психологи}

Здесь специалисты занимаются подбором персонала в коллектив, а также обучением сотрудников разным психологическим методам воздействия на клиентов с целью привлечения большего их количества, развития умения проводить продуктивные переговоры и развивать в себе сильные лидерские качества.

Также одной из задач корпоративного психолога является разрешение конфликтов в компании, улучшение взаимодействия между сотрудниками, мотивация и повышение их стрессоустойчивости.

Как ни странно, но сюда же относятся и вопросы физического оздоровления, которыми также занимаются специалисты. Например, направление психосоматологии сегодня очень хорошо решает эти вопросы, поскольку эмоциональное напряжение в коллективе, а также любые затяжные конфликты могут приводить к росту больничных. В таких случаях на помощь приглашаются квалифицированные корпоративные психологи [trends.rbc.ru, 2019].

Сегодня многие стараются получить профессию психолога. Как стать хорошим специалистом, какие качества для этого нужны и к чему должен быть готов человек при получении профессии?

\textbf{Как стать психологом: особенности обучения}

Чтобы стать психологом, необходимо получить соответствующий диплом о высшем образовании. Для этого после окончания 11 класса следует поступать в университет.

При выборе профессии психолога, чтобы определиться с университетом, нужно отдавать предпочтение престижным учебным заведениям, расположенным в крупных городах.

О том, какие предметы сдавать при поступлении в вуз лучше узнать заранее. Учеба в вузе продлится как минимум 4 года. Если появится желание поступать в аспирантуру или магистратуру, то учиться придется более 6 лет.

На факультете психологии основными предметами являются анатомия, социология, философия, логика, антропология, физиология центральной нервной системы и высшей нервной деятельности, математические действия в психологии и высшая математика. Изучается психология личности и сознания, осваиваются общепрофессиональные дисциплины, история психологии, психология труда, профильная психология и другие направления.

Обучение в высшем заведении включает в себя не только теорию, но и практику – как учебную, так и производственную. Глубина и масштаб полученных знаний будут напрямую зависеть от выбранного вуза, желаний и способностей самого человека, а также технологий обучения, поскольку программы в разных университетах могут кардинально различаться [aif.ru, 2018].

К учебе следует подходить ответственно и серьезно, хорошо подготавливаясь к экзаменам и зачетам, поскольку потом при устройстве на работу некоторые работодатели могут поинтересоваться оценками в дипломе. Окончившие университет студенты-отличники могут претендовать на самые престижные и высокие должности не только в государственных, но и в коммерческих фирмах, где зарплата психолога будет значительно выше.

Начинающий специалист, недавно окончивший высшее учебное заведение, должен постоянно увеличивать объем знаний и качественно улучшать свои профессиональные навыки.

Также обучиться профессии психолога можно на базе 9 классов. Тем, кто решил остановить свой выбор на этом варианте, сложные экзамены типа ЕГЭ сдавать не придется.

В большинство колледжей принимают по среднему баллу в аттестате, но может быть назначено дополнительное собеседование. Такие нюансы всегда зависят от самого учебного заведения.

Предметы, являющиеся обязательными для сдачи экзаменов на факультет Психологии – математика и русский язык, а также на выбор самого учащегося два любых предмета среди биологии, химии и обществознания.

Процесс обучения по специальности Психология в колледже длится от 1 года 10 месяцев до двух лет. Преимущество решивших поступить на психологический факультет после 9 класса в том, что при поступлении в высшее учебное заведение на тот же факультет они могут сразу же перейти на второй либо на третий курс, имея при этом за плечами не очень много опыта [vyuchit.work, 2018].

Чем более качественное будет получено образование, тем выше у человека будет заработная плата.

\textbf{Доходы специалистов}

Доход дипломированного психолога напрямую зависит от его специализации, места работы, уровня квалификации и наличия опыта. Консультирующие психологи, как правило, сначала начинают с невысоких расценок и постепенно повышают их по мере роста своего опыта.

В среднем, практикующий специалист в столице может за один час персональной консультации получать от 2 500 до 4 000 рублей.

В государственных учреждениях средняя зарплата психолога составляет примерно от 45 000 до 90 000 рублей в месяц.

В коммерческих организациях уровень доходов может быть значительно выше [aif.ru, 2020].

\textbf{Профессиональные требования}

Психология – это сфера, где в первую очередь важны именно личные качества специалиста. Далеко не каждый способен стать хорошим практикующим психологом, ведь это не просто прослушал курс и начал искать клиентов. Для того, чтобы успешно работать и помогать людям, необходимо обладать особенными качествами характера, без которых стать профессионалом будет очень трудно.

Психолог должен иметь высокий эмоциональный интеллект, уметь хорошо слушать, а самое главное слышать, быть терпеливым и уметь сопереживать пришедшему на консультацию человеку. В такие моменты клиенты ждут, что им будут сочувствовать в их проблемах и переживаниях.

Специалист этого профиля должен быть очень наблюдательным. Иногда действия и слова могут сильно различаться, а способность улавливать такие изменения всегда будет на руку специалисту.

К примеру, когда человек на консультации говорит, что спокоен, но при этом постоянно теребит свои пальцы, часы, телефон или что-то другое – это напрямую указывает на его нервозность и эмоциональную неустойчивость.

Хороший психолог должен понимать чувства других людей и правильно интерпретировать их действия, подбирая «ключи» к каждому конкретному человеку. Он должен уметь располагать к себе, быть открытым и коммуникабельным, поскольку ему придется много общаться с людьми вне зависимости от своего настроения.

Если при беседе с незнакомыми людьми такой специалист испытывает неловкость, если он не любит длительных разговоров и новых знакомств, ему лучше поискать себя в какой-нибудь другой сфере.

Психолог должен быть тактичным и уметь хранить чужие секреты. Хоть психология и не является медицинской наукой, но правило «врачебной тайны» здесь никто не отменял: хороший психолог никогда и ни при каких обстоятельствах не должен выносить на публику подробности своей работы с людьми.

Также он не имеет права никого осуждать, критиковать, сравнивать с другими или в целом проявлять бестактность по отношению к окружающим. Он должен быть внимательным к переживаниям других людей, и, одновременно с этим, ему нужно уметь абстрагироваться от чужих бед.

Отличный специалист обязан быть непредвзятым по отношению к другим. Ведь за долгие годы работы ему не раз придется встречать людей, чьи взгляды будут кардинально отличаться от его намерений. Поэтому профессиональный психолог не имеет права позволять своим личным убеждениям как-то влиять на работу и на отношение к людям в целом.

Задача настоящего специалиста – помочь человеку самому найти ответы на вопросы и разобраться со своими проблемами, а не навязывать свою точку зрения. Так что он должен быть беспристрастным, тактичным и ответственным.

Психолог не должен работать по каким-то шаблонам, поскольку все люди разные, и то, что помогает одному, может оказаться совершенно бесполезным для другого. А значит, он должен уметь сочетать различные методики и адаптировать их под каждого клиента.

Без этих качеств достойно выполнять обязанности психолога будет слишком сложно, ну а если человек обнаружил у себя подобные качества, пусть даже находящиеся в зачаточном состоянии, ему имеет смысл задуматься о получении этой профессии [aif.ru, 2020].

Работа психолога является концентрированным опытом взаимоотношений с разными людьми. И в этом есть свои плюсы и минусы.

\textbf{Плюсы профессии}

Преимуществом профессии психолога являются умения:
\begin{enumerate}
    \item дающие возможность осознать и многое понять про себя и свою жизнь;
    \item разбираться с жизнью других людей;
    \item помогать другим людям справляться с трудностями;
    \item обрести спокойствие внутри себя;
    \item познать причины поведения людей и их поступков;
    \item формировать философское отношение к внешним событиям;
    \item вести частную практику и быть независимым от работодателей.
\end{enumerate}

Несомненным плюсом профессии психолог является постоянный поиск, стремление росту и духовному развитию. Ведь если у человека внутри ничего этого нет, то как он что-то может дать другому?

Чтобы окончательно разобраться, подходит вам профессия психолога или нет, нужно знать и о минусах этой деятельности.

\textbf{Минусы профессии}

Большим минусом является почти постоянное состояние нервного напряжения. Психологам приходится иметь дело со страхами, отчаянием, депрессией, постоянно разбираться с причинами тяжелых переживаний, а также отыскивать и находить пути избавления от этих проблем.

При этом бывает очень трудно себя контролировать, чтобы не принимать близко к сердцу чужие негативные эмоции.

Минус состоит еще и в большой ответственности, т.к т рекомендаций психолога напрямую будут зависеть поступки и душевное состояние клиента. Чрезмерное чувство ответственности специалиста может являться причиной развития своеобразных страхов совершить что-то непоправимое или неправильное.

Недостатком профессии психолог также является интенсивное общение с незнакомыми людьми. Не каждый человек способен постоянно и помногу быть в контакте с окружающими.

К минусам профессии относится время от времени возникающая усталость, присущая психологам при больших душевных затратах и личных вложениях, которых требует работа. Ведь приходится думать о своих клиентах и после приема, постоянно переживать за них, разговаривать по телефону, чтобы оказать необходимую поддержку и помощь.

Но обычно такая усталость приятна и дорога. Ведь если специалист является востребованным, значит, он может принести пользу и хорошо знает свое дело. Такую усталость нельзя ни с чем сравнить.

Выбирая профессию психолога, человек осознает, что обратного пути у него больше нет. Он становится психологом везде и навсегда, поскольку невозможно не использовать в жизни те знания и опыт, которые он имеет, особенно когда наблюдаешь за собственными детьми, или, например, при общении с любимым человеком.

И порой бывает очень грустно уметь разбираться в каких-то вещах больше, чем другие, к тому же это может отдалять от некоторых близких людей [psynavigator.ru, 2019]

В профессии психолога, несомненно, есть как положительные, так и отрицательные стороны, но все же плюсы этой деятельности реально перевешивают. Т.к. у психолога есть отличная возможность помогать другим людям, спасать их от каких-то неправильных, а порой и глупых решений, и направлять все силы исключительно в правильное русло.

Желаем вам удачи и предлагаем поучаствовать в небольшом опросе: Пользуетесь ли вы услугами психолога?

\newpage
\section{Что делать, если случился нервный срыв?}

\textit{И как не довести себя до крайней степени стресса}

\textit{Источник: \url{https://lenta.ru/articles/2022/11/07/nerves/}}

Многие знают, что для сохранения ментального здоровья нужно читать и смотреть меньше новостей, много спать и сбалансировано питаться. Но не у всех получается следовать\footnote{следовать чему; e.g., он следует моему совету: he is following my advice} этим советам, а рекомендация «\explain{не нервничать}{do not be nervous}» кому-то и вовсе кажется насмешкой. «Лента.ру» вместе с психологами и эндокринологами разобралась в причинах возникновения нервного срыва и в том, как предупредить его появление, чтобы не испортить жизнь себе и близким, а главное — не попасть в больницу.

\textbf{Что такое нервный срыв}

Специалисты расходятся во мнениях о точной формулировке понятия «нервный срыв». Если говорить простым языком, то нервный срыв — это термин, который люди употребляют в ситуациях, когда человек перестает справляться с эмоциональной нагрузкой и «ломается». Это острая фаза стресса, проявляющаяся в виде невротических и депрессивных расстройств.

Психолог, психотерапевт, член Международной профессиональной ассоциации психологов Елизавета Деменштейн обратила внимание, что понятие «нервный срыв» не зафиксировано в Международной статистической классификации болезней именно как болезнь. Однако если обратиться к врачам с жалобами на «нервный срыв», они помогут и назначат лечение.

Нервный срыв может проявляться по-разному. У кого-то возникает состояние ступора, эмоции пропадают, возникает ощущение, что все происходящее нереально, наступает упадок сил и все время хочется лежать. Другие впадают в возбуждение с криками, слезами, истериками, агрессией и желанием крушить все вокруг.

«В большинстве случаев организм человека успевает адекватно реагировать на последствия стресса, который уже давно стал привычным явлением. Но если нагрузка слишком велика, может наступить нервный срыв — состояние, с которым психика не справляется», — пояснил психолог, кандидат психологических наук Валерий Гут. Нервный срыв, по его словам, является лишь эпизодом затяжного стресса.

Психолог калифорнийского университета Риверсайд Мэтью Чанг уверен, что люди не справляются с эмоциональным напряжением, потому что большинству из них нравится предсказуемость и рутина. Их устраивает, что один день похож на другой, так как это дает ощущение уверенности и стабильности. Но зачастую в череду спокойных будней врываются перемены и рушат устоявшийся порядок жизни, заставляют нервничать. Причем нервное потрясение могут вызвать любые яркие события — как положительные, так и отрицательные.

Нервный срыв может быть связан и с переутомлением, хронической усталостью, напряженным периодом на работе или в семейных отношениях, рассказала психолог Деменштейн. В некоторых случаях даже диета может стать спусковым механизмом для срыва, пояснила она.

Психологи, психиатры и психотерапевты должны выяснить, из-за какого длительного стресса случился нервный срыв или на фоне какого психического заболевания мог развиться сам стресс. Таких болезней может быть много: депрессия, посттравматическое расстройство и так далее.

Нервный срыв может произойти у любого человека, в том числе у тех, кто не страдает психическими заболеваниями, отметила в своих исследованиях доцент кафедры психиатрии и психолог Университета Цинциннати Мария Эспинола. Но, по ее словам, люди с психическими расстройствами переживают нервные срывы чаще.

\begin{fancyquotes}
    Хронический стресс и провоцирующий фактор — это составляющие нервного срыва\\

    \begin{flushright}
        Елизавета Деменштейн\\
        психолог, психотерапевт
    \end{flushright}
\end{fancyquotes}

Когнитивно-поведенческий психолог онлайн-школы психологических профессий «Психодемия» Александра Титарева рассказала, что за нервным срывом могут скрываться:

\begin{enumerate}
    \item расстройства, связанные со стрессом (расстройство адаптации, посттравматическое стрессовое расстройство);
    \item тревожные и связанные со стрессом расстройства (генерализованное тревожное расстройство);
    \item аффективные (биполярное или депрессивное расстройство) и многие другие расстройства.
\end{enumerate}

Так что же такое нервный срыв? Сотрудники факультета психологии Государственного университета Уэйна в США дают ему следующее определение: нервный срыв — это состояние повышенного нервного напряжения, длящееся короткий промежуток времени. Ему присущи симптомы, схожие с симптомами тревожного расстройства и депрессии. Но эти заболевания, в отличие от нервного срыва, имеют длительный характер. Нервный срыв часто случается на фоне провоцирующего события — финансовой потери, ссоры в семье, потери близкого и других.

\begin{framed}
    \begin{center}
        {
            \Huge
            26\%
        }

        {
            \Large
            американцев в ходе опроса признались, что хотя бы раз чувствовали себя на грани нервного срыва
        }
    \end{center}
\end{framed}

\textbf{Почему случается нервный срыв}

Главные причины нервного срыва как части долгого стресса.
\begin{enumerate}
    \item Психическое расстройство, провоцирующее срыв (депрессия, биполярное расстройство, посттравматическое расстройство, беспокойство, расстройство адаптации, возникающей из-за реакции на стрессовое событие или череду психотравмирующих ситуаций).
    \item Истощение организма из-за хронических болезней (сбоев в работе сердечно-сосудистой системы, неврологических заболеваний).
    \item Тяжелый график работы без возможности отдохнуть в полноценном отпуске.
    \item Длительная тяжелая обстановка в семье.
    \item Известия о тяжелых событиях (смерть близких, развод, увольнение и другие).
\end{enumerate}

\begin{center}
    \Large
    В ходе исследования, проведенного зарубежными социологами, стало известно, что самыми частыми причинами срыва называют проблемы в отношениях и воспитание ребенка в одиночку (у женщин)
\end{center}

Спровоцировать нервный срыв могут даже такие факторы, как нарушение сна и компульсивное

переедание.

Сохранение режима сна — одно из условий поддержания физического и ментального здоровья. Хроническое недосыпание, а также кошмары, регулярные пробуждения среди ночи, нарушение циркадных ритмов также могут стать причиной нервного срыва, пояснил медицинский психолог, психоаналитик Олег Долгицкий. При этом одни могут бесконечно спать (но просыпаются не отдохнувшими), а другие, наоборот, спят мало, урывками, становятся чересчур бодрыми и активными. Обычно таких людей объединяет то, что они не могут получать удовольствие в своей жизни, добавил он.

Сочетание нескольких или всех перечисленных симптомов с отсутствием удовольствия и наличием психосоматической болезни может приводить к накалу эмоциональных состояний, к нервным срывам, которые говорят о том, что у человека уже закончились силы и ему пора на отдых.


\textbf{Как понять, есть ли нервный срыв у меня}

Невролог центра здоровья Verba Mayr Елена Сопова обратила внимание, что признаков нервного срыва может быть множество в зависимости от причины нервного напряжения и силы его проявления. Но главный из них — это отсутствие способности организма нормально функционировать. Итак, при нервном срыве могут проявляться следующие симптомы.

\begin{enumerate}
    \item Эмоциональные: тревога, тоска, раздражительность, навязчивые мысли, перепады настроения.
    \item Психические: появление страха смерти, ощущение беспомощности.
    \item Сбой в работе иммунной системы: частые простудные заболевания, обострение герпетической инфекции.
    \item Снижение когнитивных функций (памяти, внимания). Это может значительно снизить работоспособность, сказаться на возможности обслуживать себя и отразиться на качестве жизни в целом.
    \item Сбой работы желудочно-кишечного тракта (боли в животе, нарушения стула, метеоризм).
    \item Расстройства вегетативной нервной системы: нарушение работы сердца (аритмия, тахикардия), сухость во рту, повышенная потливость, тремор.
    \item Изменение аппетита и веса.
    \item Проблемы со сном.
\end{enumerate}

Психолог Гут добавил, что при нервном срыве у человека могут появиться мысли о смерти и желание причинить себе вред. Тело реагирует на стресс слабостью, мышечным перенапряжением (стиснутые зубы, «деревянная» спина).

Во время нервного срыва человек понимает, что теряет над собой контроль: не может сдержать слезы, выступающие без веского повода, становится раздражительным, замечает частую смену настроения и ухудшение концентрации, у него повышается уровень тревожности, добавила Деменштейн. В общем, организм всеми возможными способами подает сигналы о том, что ему необходима помощь.

Психиатр Дельвена Томас предупредила, что человек в период срыва может полностью изолироваться от общества. Со стороны может казаться, что больной не интересуется никем и ничем вокруг из-за скуки, но на деле таким образом он борется за свое психическое здоровье и, в общем-то, жизнь.

«В таком состоянии люди не могут быть рядом с другими из-за своей психической неустойчивости, — конкретизировала Томас. — Общим сигналом нервного срыва является уход из привычного окружения. Человек перестает общаться с семьей и друзьями и больше не считается полностью функциональным».

Научные сотрудники Университета Колумбии в США в ходе тестирования 102 человек установили, что у людей с сопутствующими психическими заболеваниями нервный срыв проявлялся по-разному.

Так, участники эксперимента с паническими атаками, усугубленными нервным срывом, часто ощущали удушье и страх смерти. Люди с нервным срывом и аффективным расстройством вели себя очень агрессивно и признавались в постоянном желании кричать. У тех, у кого нервный срыв стал результатом других сопутствующих заболеваний и тревожных расстройств, ярких симптомов было меньше.

Кстати, ученые из Великобритании выяснили, что на женщин и мужчин стресс влияет по-разному. Так, количество женщин, испытывающих стресс, связанный с работой, оказалось на 50 процентов больше, чем у мужчин того же возраста. Это происходит из-за того, что женщины много работают, строят карьеру, но в то же время за ними сохраняются обязанности следить за детьми и домом.

\begin{framed}
    \begin{center}
        {
            \Huge
            у 25процентов
        }

        {
            \Large
            женщин рано или поздно развивается депрессия
        }
    \end{center}
\end{framed}

\textbf{Чем нервный срыв отличается от панической атаки}
Иногда нервный срыв можно спутать с панической атакой из-за некоторых схожих симптомов. Но паническая атака — это совершенно другое состояние, отмечают психологи. Постоянные панические атаки могут привести к тревожному расстройству.

Панической атаке, как и нервному срыву, присуще сильное чувство страха, ощущение того, что непременно должно произойти что-то плохое. Люди при панической атаке боятся потерять контроль над происходящим и даже умереть. При атаке, которая обычно накрывает внезапно, возникает повышенное потоотделение, тремор рук, учащенное сердцебиение. Могут наблюдаться проблемы с ЖКТ и головная боль.

Панические атаки короче, чем нервные срывы, и, когда они проходят, человек ощущает сильный стресс и усталость. Панические атаки очень пугают своей внезапностью и множеством физических симптомов, которых больше, чем при нервном срыве.

Большая разница между панической атакой и нервным срывом заключается в том, что при первой человек порой не может даже подняться с постели, настолько ему плохо, а при втором энергия может бить через край.

\begin{fancyquotes}
    Возможно, кому-то знакомо ощущение помутненного рассудка, когда кружится голова и энергия буквально бьет в мозг. Это как ядерная бомба, которую сбросили с самолета. Если она ударится о землю, будет взрыв

    \begin{flushright}
        Валерий Гут
        \\
        психолог
    \end{flushright}
\end{fancyquotes}

\textbf{Гормоны щитовидной железы и нервный срыв}

Эндокринолог Тодд Ниппольдт в статье «Заболевания щитовидной железы: могут ли они повлиять на настроение человека?» однозначно ответил на заданный в заглавии вопрос. «Да, заболевания щитовидной железы могут влиять на настроение, в первую очередь вызывая тревогу или депрессию. Как правило, чем тяжелее заболевание щитовидной железы, тем сильнее меняется настроение», — пояснил он.

Эндокринолог Екатерина Асташова рассказала «Ленте.ру», что при недостатке гормонов щитовидной железы — гипотиреозе — обменные процессы в организме замедляются, и нервная система угнетается. При гипотиреозе нарушается эмоциональное состояние: у людей отмечается подавленное, тоскливое настроение, приступы тревоги.

\begin{fancyquotes}
    Избыток гормонов щитовидной железы — гипертиреоз или тиреотоксикоз — также вызывает проблемы со здоровьем. Обменные процессы ускоряются, появляется повышенная двигательная и психическая активность. Человек может беспричинно плакать, раздражаться, постоянно находиться на взводе\\

    \begin{flushright}
        Екатерина Асташова
        \\
        эндокринолог
    \end{flushright}
\end{fancyquotes}

Избыток гормонов щитовидной железы встречается у двух процентов женщин и 0,2 процента мужчин, добавила эндокринолог, эксперт по превентивной и anti-age медицине Европейского медицинского центра Камиля Табеева. Недостаток гормонов диагностируют у 4,6 процента женщин.

\begin{fancyquotes}
    У 21 женщины из 1000 и 19 мужчин из 1000 наблюдается недостаток гормонов щитовидной железы\\

    \begin{flushright}
        Камиля Табеева\\
        эндокринолог
    \end{flushright}
\end{fancyquotes}

Табеева отметила, что у таких людей есть риск развития депрессии. Зачастую пациенты наблюдаются с этой болезнью, не зная, что имеют дефицит гормонов, который можно компенсировать с помощью таблеток.

После диагностики заболевания щитовидной железы, которое может протекать в том числе и с переходом из тиреотоксикоза в гипотиреоз, эндокринолог подберет лечение и план наблюдения. И через три-шесть недель эмоциональное состояние пациента придет в норму, резюмировала эндокринолог Асташова.

\textbf{Последствия нервного срыва}

Существует четкая связь между стрессом и ухудшением состояния здоровья. Люди, находящиеся в постоянном стрессе, чаще болеют как психически, так и физически, предупредил психолог Долгицкий.

\begin{fancyquotes}
    Если человек длительное время находится в состоянии стресса, у него на 80 процентов повышается риск возникновения бронхиальной астмы, язвенного колита, эссенциальной гипертензии, нейродермита, ревматоидного артрита, язвенной болезни желудка и язвы двенадцатиперстной кишки\\

    \begin{flushright}
        Олег Долгицкий\\
        медицинский психолог, психоаналитик
    \end{flushright}
\end{fancyquotes}

Тело всегда реагирует на сильный стресс, заявил судебный психиатр Мэттью Чанг. «Мозг посылает сигнал нервной системе в начале стрессовой ситуации о выбросе гормонов стресса — адреналина, норадреналина и кортизола», — отметил он.

Этот каскад химических реакций побуждает тело перенаправлять приток крови к крупным мышцам и сердцу, что в свою очередь учащает дыхание, повышает пульс и кровяное давление и подготавливает тело к психологической и физической реакции. Скелетные мышцы также активируются, напрягаются, чтобы помочь защититься от возможных травм или подготовить тело к выходу из критической ситуации.

Если реакция «бей и беги», возникающая при стрессе, в том числе и нервном срыве, срабатывает слишком часто, это может привести к проблемам:

\begin{enumerate}
    \item болезням сердца;
    \item постоянно высокому кровяному давлению;
    \item повышенному уровню сахара в крови и снижению чувствительности к инсулину;
    \item ослаблению иммунной системы;
    \item повышенному риску ожирения (особенно в области живота);
    \item депрессии и беспокойствам;
    \item головным болям;
    \item бессоннице.
\end{enumerate}

«Стресс, который продолжается в течение длительного времени, может быть опасным для жизни, — добавил Чанг. — Любой, кто испытывает хронический стресс, должен принять меры, чтобы снизить его уровень и позволить телу вернуться в нормальное состояние».

\textbf{Как не допустить нервный срыв}

Психолог Гут, сравнивший нервный срыв с падающей на землю бомбой, отметил, что, пока «бомба» не приземлилась, есть всего несколько секунд, чтобы ее «поймать» и предотвратить удар. Нужно научиться «подставлять сетку» и не давать снаряду разорваться. Это позволит не разрушить жизнь гневом, истериками и скандалами.

Для профилактики стрессов в общем и нервных срывов в частности, нужно соблюдать определенные правила, настаивает нутрициолог Дарья Ермилова. Главным, по ее словам, является соблюдение циркадных ритмов и гигиены сна. «Спать нужно ложиться строго до 23:00, а просыпаться около 7:00, — рассказала она. — Днем нужно проводить максимальное количество времени при дневном освещении, а в вечернее время использовать мягкий свет, избегая синего излучения от техники и гаджетов».

Питание, направленное на снижение воспалительных процессов, поможет поддержать нервную систему, добавила нутрициолог. Из рациона нужно убрать трансжиры, чрезмерное количество углеводов, сахара, молочных продуктов.

\begin{fancyquotes}
    Нужно есть больше мяса нежирных сортов (курица, кролик, телятина, индейка), рыбы и морепродуктов, овощей (белокочанная капуста, брокколи, морковь, тыква), зелени, ягод (особенно черной смородины), а также круп и орехов\\

    \begin{flushright}
        Дарья Ермилова\\
        нутрициолог
    \end{flushright}
\end{fancyquotes}

Адекватная физическая нагрузка повышает чувствительность к инсулину и снижает уровень воспаления в организме, рассказала Ермилова. Это может быть простая ходьба, но не менее 10 тысяч шагов в день, аэробные нагрузки, бег. За физическую активность сойдет даже интенсивная уборка дома.

Большую роль в поддержании психического здоровья играет отказ от вредных привычек: курения, употребления алкоголя и запрещенных веществ.

\begin{framed}
    \begin{center}
        \Large

        Медитация, или дыхательные практики — это простой и приятный способ восстановить баланс парасимпатической и симпатической нервных систем, снизить воспаление в организме и уровень стресса
    \end{center}
\end{framed}

Психолог Титарева добавила, что профилактикой нервного срыва является снижение повседневного стресса. Для этого нужно отказаться от части нагрузки, контролировать режим работы и отдыха, следить за питанием, расслабляться, слушая пение птиц или наблюдая за аквариумными рыбками.

\begin{fancyquotes}
    Подойдут любые способы, помогающие вернуться в спокойное состояние: прогулка на свежем воздухе, общение с любимым питомцем, которого можно потискать, сбор пазлов или картин по номерам\\

    \begin{flushright}
        Александра Титарева\\
        психолог
    \end{flushright}

\end{fancyquotes}

Также необходимо разобраться в причинах, которые привели к нервному срыву, чтобы избежать этого состояния в дальнейшем, отметил психолог Гут. В этом поможет ведение дневника, который станет ценным инструментом проживания и анализа ситуаций и событий.

Нервный срыв — это всегда острое состояние, требующее немедленных действий. Техники самопомощи помогут с ним справиться, но лучше начать заботиться о себе и своем психическом здоровье заранее и не допустить кризиса.

\textbf{Как помочь себе при нервном срыве}

Профилактика срыва — это лучшая помощь себе. Но если ее методы не сработали, то можно прибегнуть к действенным методам самопомощи.

В момент нервного срыва происходит огромный отток энергии. И первое, что нужно сделать, это затормозить отвечающую за стресс систему. Умывание холодной водой, мышечная релаксация и простые физические упражнения (приседания или отжимания) снимут острую фазу, отметил психолог Гут.

После этого нужно создать условия, в которых человек будет чувствовать себя комфортно и безопасно. Добиться этого ощущения помогают еда и сон.

\begin{fancyquotes}
    Также облегчить состояние помогает массаж, с его помощью можно снять зажимы и расслабить мышцы шеи, плеч, которые чаще всего излишне напряжены во время стресса\\

    \begin{flushright}
        Елизавета Деменштейн\\психолог
    \end{flushright}
\end{fancyquotes}

\textbf{Как помочь другому человеку при нервном срыве}

При нервном срыве нужно помочь человеку успокоиться и переключить его на какое-либо занятие или просто отвлечь беседой. Сделать это нужно здесь и сейчас, чтобы он в порыве нервного напряжения не навредил себе и окружающим.

«Важно понимать, что человек при нервном срыве деградирует в реакциях и поведении до уровня ребенка. Поэтому необходимы забота, деликатность и мягкая настойчивость», — рассказывал о срыве в своем канале психолог Артем Толоконин.

Для того чтобы облегчить проявление нервного срыва у другого человека, он посоветовал предложить больному сесть или лечь. При этом нужно быть рядом с ним и дать своим спокойным поведением понять, что все хорошо.

При возможности человека в нервном перевозбуждении нужно обнять. Так, по мнению Толоконина, его внутренний ребенок успокоится, так как закроет потребность в тепле и близости.

\textbf{Лечение нервного срыва}

Нервные срывы часто влекут негативные последствия. Так, психосоматические нарушения могут стать причиной повышения давления, болей в желудке, головокружений. Также по цепочке может ухудшиться основное психическое заболевание, повлекшее нервный срыв. Иногда даже может быть затронута работа некоторых отделов головного мозга.

Постоянное нервное напряжение мешает сохранению отношений в семье, общению с коллегами и друзьями, что грозит проблемами дома и на работе, а также социальной изоляцией и в перспективе — ухудшением финансового положения.

«Не стоит забывать, что часто в нашей стране последствия нервного срыва принято «гасить» психоактивными веществами. Однако в долгосрочной перспективе такое «самолечение» приводит к усилению тревоги и подавленности, ночным кошмарам и даже суициду», — предостерегла Титарева.

\begin{framed}
    \begin{center}
        За психологической поддержкой всегда нужно обращаться к специалисту. В беседе с психологом или психотерапевтом необходимо делать упор на причине, вызвавшей нервный срыв, но при этом не отвергать эмоции. Без выявления причины велик риск повторного срыва, даже после курса поддерживающей терапии
    \end{center}
\end{framed}

«Если вы начинаете чувствовать, что стресс становится слишком сильным, поговорите со своим терапевтом. Он может направить вас к психологу или психиатру. Терапевт также может назначить лечение физических симптомов», — рассказал врач Дэн Бреннан.

По его словам, правильное лечение нервного срыва зависит главным образом от его причины и индивидуальных особенностей пациента. Некоторые методы лечения включают:

\begin{enumerate}
    \item изменение образа жизни;
    \item сокращение количества ежедневных обязательств;
    \item соблюдение здоровой диеты;
    \item отдых при первой необходимости;
    \item практика медитации;
    \item проведение большего времени на природе.
\end{enumerate}

Стрессы и психологические заболевания лечат и с помощью медикаментов. Если обратиться к врачу, он может назначить антидепрессанты или успокаивающие препараты, чтобы облегчить симптомы нервного срыва. Если стресс вызывает бессонницу, могут прописать снотворное, пояснил Бреннан. Нарушения сна могут усугубить стресс и тревогу, которые в свою очередь только усугубят бессонницу — образуется замкнутый круг. Снотворные могут помочь разорвать цикл бессонницы и уменьшить стресс.

Психотерапия поможет справиться с нервным срывом и снизить риск его повторения. Разговор с профессионалом поможет обдумать мысли и найти решения, которые уменьшат стресс и тревогу.

В очень тяжелых случаях может быть уместно даже стационарное лечение, но обычно это возникает тогда, когда нервный срыв может быть сопряжен с риском для жизни человека (суицидальная попытка), прокомментировал психолог Долгицкий. «Тогда в соответствии со ст. 29 закона "О психиатрической помощи и гарантиях прав граждан при ее оказании" пациента госпитализируют, — пояснил психолог. — Это становится возможным в том случае, если человек несет непосредственную опасность для себя или окружающих во время нервного срыва».

Также, по его словам, нужно госпитализировать пациента, если известно, что без профессиональной помощи пострадает его здоровье. Например, когда человек падает в голодные обмороки, теряет сознание или наносит себе повреждения, подвел итог Долгицкий.