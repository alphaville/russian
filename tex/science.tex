\chapter{Наука и Техника}
\section{Северное сияние}
С \explainDetail{наступлением}{наступление}{adventб beginning} осени тысячи туристов \explain{устремляются}{rush} в \explain{Заполярье}{the region of the Arctic circle}, чтобы увидеть уникальный танец небесных огней -- полярное, или северное сияние, на латыни -- Aurora Borealis.
В теории увидеть это природное явление можно с конца августа до середины апреля: в этот период времени ночи становятся темными, солнечная активность \explainDetail{возрастает}{возрастать/возрасти}{возраст\'{а}ю/-ешь/-ет; возраст\'{у}/-ёшь/-\'{у}т: rise, increase}, а облака \explainDetail{расс\'{е}иваются}{рассеиваться/рассеяться}{to disperse}.
Такое развлечение, как \explain{ох\'{о}та}{hunting} за северным сиянием, с каждым годом становится все популярнее как среди россиян, так и среди иностранных туристов, которые специально ради него готовы ехать на Крайний Север. Главное \explain{доказательство}{proof} удачной охоты -- это, конечно же, снимки северного сияния.

\section{Ученые нашли способ записать данные в пяти измерениях}

\textit{Как 5D-диски изменят представление людей о хранении информации?}

\textbf{Ученые создали 5D-диск высочайшей плотности:} В октябре специалисты Саутгемптонского университета в Великобритании описали способ записи огромного количества данных на компактный диск небольших размеров. Технология, получившая название 5D, позволяет сохранить на специальном накопителе до 500 терабайт информации. Получившиеся диски из кварцевого стекла отличаются высочайшей плотностью, которая в десять тысяч раз превышает плотность оптических дисков Blu-Ray. Новый метод позволит эффективно разместить на небольшой площади облачные сервера для хранения данных пользователей, интернет-компаний, крупных корпораций. По словам ученых, это особенно важно на фоне развития технологий, увеличения количества подключенных к сети устройств и роста количества передаваемых через сеть данных.

\textbf{Облачные сервисы с каждым годом становятся все популярнее:}
За последние пять лет отношение потребителей и бизнеса к облачным сервисам изменилось. Раньше их воспринимали в качестве дополнительного метода резервного копирования данных — информация практически всегда поступала в одну сторону. Причем крупные корпорации в основном использовали дата-центры для хранения некритичной информации. К 2020-м годам организации стали использовать облачные серверы не только для аварийного копирования, но и для постоянного обмена данными внутри конкретного предприятия. Системы облачных хранилищ стали более гибкими, позволяя конкретному потребителю выбрать необходимое количество свободного места и производительность оборудования.

Специалисты Analytics Insight называют основными \explainDetail{преим\'{у}ществами}{преим\'{у}щество}{advantage} \explainDetail{облачных}{облачный}{cloud (adj.); \'{о}блако: cloud} дата-центров \explain{круглос\'{у}точный}{round the clock} доступ к информации, возможность одновременной работы не- скольких пользователей с одним массивом данных, масштабируемость и \explain{г\'{и}бкость}{flexibility}, \explain{снижение}{decline} затрат на хранение данных внутри компании.

\explainDetail{Представители}{представитель}{representative} отрасли отмечают, что в обычное время нагрузка на серверы неравномерна: в одной части дата-центров она может зашкаливать, в другой быть крайне небольшой. По этой причине эксперты предсказывают появление искусственного интеллекта, который мог бы анализировать и распределять нагрузку на оборудование. В том числе по этой причине данные пользователей хранятся в нескольких частях дата-центра.

\textbf{Больше всего в облачных сервисах пользователи ценят скорость передачи данных и безопасность:} По словам основателя облачного провайдера Wasabi Дэйва Френда, от дата-центров будущего потребители ожидают высокого уровня безопасности, производительности оборудования и приемлемой цены за услуги. «Резервные копии должны храниться в разрозненных системах, обеспечивающих максимально возможную изоляцию», — заметил предприниматель. Потенциальный злоумышленник, добравшийся до одного сервера, не должен иметь возможность удалить или зашифровать информацию так, чтобы ее нельзя было восстановить из альтернативных источников. Френд полагает, что на этом должна строиться концепция мультиоблака.

Другими критериями облачного сервиса будущего, по мнению Френда, являются доступная цена и высокая скорость передачи данных. Провайдеры должны будут таким образом скорректировать стоимость услуг и добиться определенного качества оборудования, чтобы оставаться конкурентоспособными и не разочаровывать клиентов.

Представители облачного провайдера CloudSigma рассказали, что дата-центры должны будут отвечать за сохранность данных и скорость передачи информации. Для хранения файлов пользователей и корпоративных клиентов они используют небольшие 2,5-дюймовые диски емкостью 250 гигабайт. В случае, если какой-либо диск выходит из строя, его заменяют, а данные восстанавливают через бэкап. При таком развитии событий клиент не теряет своих данных, хотя и оказывается без доступа к информации на 10-15 минут. Благодаря глубокой интеграции между серверами и оборудованием задержка передачи данных внутри дата-центра очень мала. Для того чтобы разогнать скорость и снизить задержку между серверами и пользователем, в компании полагаются на выделенную гигабитную линию интернета.


\textbf{Диски 5D позволят хранить информацию практически бесконечно:}
По оценке Forbes, к 2025 году к интернету будет подключено около 80 миллиардов устройств, которые будут генерировать около 180 триллионов гигабайт данных. В обозримом будущем хранить данные на классических накопителях будет проблематично — существует риск возникновения дефицита и увеличения стоимости хранения информации. Работающие над технологией 5D специалисты Саутгемптонского университета предлагают записывать информацию на кварцевом стекле с помощью фемтосекундных лазеров и сверхкоротких импульсов. «Запись на кварцевый носитель как бы идет в пяти измерениях — двух оптических и трех пространственных», — отмечают авторы исследования.

Инновация британских инженеров заключается в создании дисков повышенной плотности и размещении на небольшом участке колоссальных объемов данных. Например, на «болванке» размером в один дюйм удалось сохранить шесть гигабайт информации. Накопитель обычного для подобных устройств размера, основанный на кварцевых дисках, может сохранить до 500 терабайт данных. Разработка обещает революцию на рынке хранения информации, так как десятки, если не сотни классических дата-центров можно будет объединить в одну библиотеку.

Преимуществами 5D-дисков также называют долговечность и низкую стоимость обслуживания. По оценке ученых, кварцевые диски не прочнее обычных накопителей, однако могут выдержать температуру до 1800 градусов по Фаренгейту, или около тысячи градусов по Цельсию. В случае пожара в дата-центре информация, скорее всего, сохранится. Кроме того, кварцевое стекло со временем не меняет своих свойств, что позволит держать данные на 5D-накопителях практически вечно.

Единственным узким местом будущей разработки является скорость передачи данных. В настоящий момент ученым удалось разогнать ее до 230 килобайт в секунду — за это время на диск можно записать около ста страниц текста. Однако для того, чтобы полностью заполнить болванку емкостью 500 терабайт, потребуется 60 дней. Либо инженеры найдут способ обойти ограничение, либо 5D-диски так и останутся перспективным оборудованием для записи информации. В крайнем случае на подобных дисках можно сохранять данные для потомков. Так, в 2018 году на кварцевый носитель была записана трилогия романов Айзека Азимова «Основание» — диск отправился в космос вместе с Tesla Roadster Илона Маска.