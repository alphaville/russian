\chapter{Искусство}

\section{Художники}
\subsection{В\'{и}ктор Мих\'{а}йлович Васнец\'{о}в}
% https://muzei-mira.com/biografia_hudojnikov/765-viktor-mihaylovich-vasnecov-biografiya.html
Виктор Михайлович Васнецов родился в 1848 году 15 мая в селе со смешным названием Лопьял. Отец Васнецова был священником, также как и его дед и прадед. В 1850 году Михаил Васильевич увёз семью в село Рябово. Это было связано с его службой. У Виктора Васнецова было 5 братьев, один из которых также стал знаменитым художников, звали его Аполлинарий.

Талант Васнецова проявился с детства, но крайне неудачное \explainDetail{денежное}{д\'{е}нежный/-ая/-ое}{monetary} положение в семье не оставило вариантов, как отдать Виктора в Вятское духовное училище в 1858 году. Уже в 14-летнем возрасте Виктор Васнецов учился в Вятской духовной семинарии. Детей священников туда брали бесплатно.

Так и не окончив семинарию, в 1867 году Васнецов отправился в Петербург поступать в Академию художеств. Денег у него было совсем мало, и Виктор выставил на «аукцион» 2 свои картины -- «Молочница» и «Жница». До \explainDetail{отъезда}{отъезд}{departure} он так и не получил за них денег. 60 рублей за эти две картины он получил спустя несколько месяцев уже в Петербурге. \explainDetail{Прибыв}{прибывать/прибыть}{arrive} в столицу, у молодого художника было всего 10 рублей.

Васнецов отлично справился с экзаменом по рисованию и сразу \explain{был зачислен}{was enrolled (\textit{гл. св.} зачислить: to enrol)} в Академию. Около года он занимался в Рисовальной школе, где и познакомился со своим учителем -- И. Крамским.

К занятиям в Академии художеств Васнецов приступил в 1868 году. В это время он \explain{сдружился с}{made friends with} Репиным, и даже одно время они жили на одной квартире.

Хоть Васнецову и нравилось в Академии, но он её не закончил, уехав в Париж в 1876 году, где прожил больше года. В это время там же находился и Репин в \explainDetail{командировке}{командировка}{business trip}. Они также поддерживали дружеские отношения.

После возвращения в Москву Васнецова сразу приняли в Товарищество передвижных художественных выставок. К этому времени стиль рисования художника значительно меняется, да и не только стиль, сам Васнецов перебирается жить в Москву, где сближается с Третьяковым и Мамонтовым. Именно в Москве Васнецов \explainDetail{раскрылся}{раскрыв\'{а}ться/раскр\'{ы}ться}{to open, uncover oneself, to come out}. Ему нравилось находиться в этом городе, он чувствовал себя легко и \explainDetail{выполнял}{выполн\'{я}ть/в\'{ы}полнить}{to perform, execute, carry out} различные творческие работы.

Более 10 лет Васнецов \explainDetail{оформлял}{оформл\'{я}ть/оф\'{о}рмить}{put into shape, form} Владимирский соб\'{о}р в Киеве. В этом ему помогал М. Нестеров. Именно после окончания этой работы, Васнецова можно по праву назвать великим русским иконописцем.

1899 год стал \explainDetail{пиком}{пик}{peak} популярности художника. На своей выставке Васнецов представил публике «Трёх богатырей».

После революции Васнецов стал жить уже не в России, а в СССР, что его серьёзно \explain{угнетало}{угнетать}{opress, depress, despirit}. Люди \explainDetail{уничтожали}{уничтож\'{а}ть/уничт\'{о}жить}{to destroy, obliterate; уничтож\'{е}ние: destruction} его картины, \explainDetail{относились}{относ\'{и}ться + \textit{дат.}}{to treat} неуважительно к художнику. Но до конца своей жизни Виктор Михайлович был в\'{е}рен своему делу -- он рисовал. \explainDetail{\'{У}мер}{умирать/умереть}{умир\'{а}ю, умир\'{а}ешь, умир\'{а}ют; умр\'{у}, умрёшь, умр\'{у}т: to die} он 23 июля 1926 года в Москве, так и не закончив портрет своего друга и ученик\'{а} М. Нестерова.

\subsection{Василий Васильевич Кандинский}
% https://muzei-mira.com/biografia_hudojnikov/2022-vasiliy-vasilevich-kandinskiy.html
Знаменитый создатель легендарного «Синего всадника» обратился к сфере искусств относительно поздно -- в возрасте около 30 лет, что не помешало ему достичь значительных высот, став одним из создателей абстракционизма, основателем многочисленных художественных объединений и педагогом в Высшей школе строительства и художественного конструирования, более известной как Баухаус.

Кандинский происходил из оригинального купеческого сибирского рода, где причудливо смешалась кровь тунгусских князей с древнейшей родословной, не менее старинного княжеского рода манси и каторжников, сосланных в Нерчинск за Бог весть какие провинности.

В детстве будущего художника его семейство много путешествовало по Европе и территории России, а затем поселилось в Одессе, которая тогда была третьим по значимости городом Российской империи. В этом чудесном южном городе Василий закончил гимназию, а также получил музыкальное и художественное образование. Несмотря на несомненное дарование мальчика, родители прочили ему карьеру юриста, что он и воплотил в жизнь, учась с перерывами в Московском университете.

Однако настоящая жизнь Кандинского как художника начинается с выставки импрессионистского искусства в Москве 1895 года, где его в самое сердце поразила работа Клода Моне. В следующем году он уезжает в Мюнхен, где погружается в среду экспрессионизма, но начало Первой Мировой войны прерывает его становление и он возвращается на родину. Но с Советской Россией ему не по пути, и Василий Васильевич в 1921 году навсегда покидает родные пенаты. Он уезжает в Германию, откуда через некоторое время вместе с женой бежит во Францию от нацистов, закрывших Баухаус и признающих только собственное, глубоко формализованное и статичное искусство. В принявшей его стране он получает гражданство, становится известным и живет всю свою оставшуюся жизнь.

За годы своего творчества Кандинский основал объединение «Фаланга», школу, участвовал в «Бубновом валете», затем заложил Новое мюнхенское художественное объединение, а позже -- и знаменитый «Синий всадник».

В начале своей художественной карьеры мастер работал в реалистичной и частично абстрактной манере, экспериментировал с формами и цветом, писал на стекле.

Начав преподавательскую деятельность, Кандинский вскоре стал видным теоретиком абстракционизма и Баухауса. Его самые известные полотна -- это «Москва», «Восток», «Колебание» и «Композиция», однако их очень много и каждое из них имеет полное право называться настоящим шедевром. Его картины исключительно интересно рассматривать, возникает ощущение, что с каждой точки зрения в них открывается что-то новое и необычное.

К сожалению, пришедшие к власти фашисты успели уничтожить множество работ Василия Васильевича, как и других талантливых мастеров, причисленных к категории «дегенеративного искусства». Но и оставшихся полотен нам достаточно, чтобы понимать, каким великим талантом был Кандинский. Скончался мастер в 1944 году в Нейи, пригороде Парижа.


\section{Произведения}
\subsection{Картина «Три Богатыря» ВМ Васнец\'{о}ва}
% https://muzei-mira.com/kartini_russkih_hudojnikov/1321-opisanie-kartiny-bogatyri-tri-bogatyrya-vasnecova-1898.html

Картина В\'{и}ктора Мих\'{а}йловича Васнец\'{о}ва «Богатыр\'{и}» \explain{по пр\'{а}ву}{rightfully } считается настоящим народным \explainDetail{шед\'{е}вром}{шед\'{е}вр}{masterpiece} и с\'{и}мволом отечественного искусства. \explainDetail{Создав\'{а}лась}{создаваться/создаться}{create} картина во второй половине XIX века, когда среди русских художников был\'{а} очень популярна тема народной культуры, русского фольклора. Для многих художников это увлечение оказалось кратковременным, но у Васнецова народная фольклорная тематика \explainDetail{ст\'{а}ла}{стать/становиться}{become (ст\'{а}л/-а/-о)} \explainDetail{осн\'{о}вой}{осн\'{о}ва}{basis} всего \explainDetail{творчества}{творчество}{cretativity}.

На картине «\explainDetail{Богатыр\'{и}}{богат\'{ы}рь}{A bogatyr or витязь is a stereotypical fictional character in medieval Russian legends}» \explainDetail{изображен\'{ы}}{изображён/-\'{а}/-\'{о}}{depicted; изображение: image, depiction; изображ\'{а}ть/изобраз\'{и}ть: to depict} три русских богатыр\'{я}: Илья Муромец, Добрыня Никитич и Алёша Попович -- знаменитые герои народных \explainDetail{был\'{и}н}{был\'{и}на}{epic}.

\explainDetail{Испол\'{и}нские}{испол\'{и}нский}{gigantic} фигуры богатыр\'{е}й и их коней, распол\'{о}женные \explain{на пер\'{е}днем пл\'{а}не}{in the foreground (пер\'{е}дний пл\'{а}н)} картины, символизируют силу и мощь русского народа. Этому \explainDetail{впечатлению}{впечатл\'{е}ние}{impression} \explainDetail{спос\'{о}бствуют}{способствовать/поспособствовать}{contribute to} и \explainDetail{внушительные}{внушительный}{impressive} размеры картины -- 295$\times$446 см.

Над созданием этой картины художник работал почти 30 лет. В 1871 году был с\'{о}здан первый \explain{набр\'{о}сок}{sketch} сюжета в карандаш\'{е}, и с тех пор художник увлёкся идеей создания этой картины. В 1876 году был сделан знаменитый \explain{эск\'{и}з}{sketch} с уже \explainDetail{н\'{а}йденной}{н\'{а}йденный/-ая/-ое}{$<$ найти} основой композиционного решения. Работа над самой картиной длилась с 1881 по 1898 год. Готовая картина была куплена П. Третьяковым, и до сих пор она \explainDetail{украш\'{а}ет}{украш\'{а}ть/укр\'{а}сить}{decorate} Государственную Третьяковскую галерею в Москве.

В центре картины изображён Илья Муромец, народный любимец, герой русских был\'{и}н. Не всем известно, что Илья Муромец не сказочный персонаж, а реальное историческое лицо. История его жизни и \explainDetail{р\'{а}тных}{р\'{а}тный}{military} \explainDetail{п\'{о}двигов}{п\'{о}двиг}{exploit, feat} -- это реальные события. \explain{Впосл\'{е}дствии}{subsequently}, закончив свои труды по охране родины, он стал монахом Киево-Печёрского монастыр\'{я}. Был причислен к лику святых\footnote{was canonised}. Васнецов эти факты знал, создав\'{а}я образ Ильи Муромца. «\explainDetail{Матёр}{матёрый}{mature, fully grown, hardened} человек Илья Муромец» -- говорит былина. А на картине Васнецова мы видим могучего воина и при том \explainDetail{бесх\'{и}тростного}{бесх\'{и}тростный}{ingenuous, silly} открытого человека. В нём \explainDetail{сочетаются}{сочетаться}{combine} исполинская сила и \explain{великод\'{у}шие}{generosity, magnanimity, goodness}. «А конь под Ильёй \explain{лютый}{fierce} зверь» -- продолжает сказание. \explainDetail{Мощная}{мощный/-ая/-ое}{powerful} фигура коня, изображённого на картине с массивной металлической цепью вместо упряжки, \explainDetail{свид\'{е}тельствует}{свид\'{е}тельствовать}{testify; свидетель: witness} об этом.

Добрыня Никитич по народным преданиям был очень образ\'{о}ванным и \explainDetail{м\'{у}жественным}{м\'{у}жественный}{manly} человеком. С его личностью связано много чудес, наприм\'{е}р, заговорённая броня\footnote{charmed armor} на его плечах, \explain{волш\'{е}бный}{magic} меч-кладенец. Добрыня изображён таким как и в былинах -- величавым, с тонкими, благородными чертами лица, подчёркивающими его культурность, образ\'{о}ванность, \explain{решительно}{decisively} вынимающий меч из \explain{н\'{о}жен}{sheath} с готовностью \explainDetail{бр\'{о}ситься}{брос\'{а}ться/бр\'{о}ситься}{rush} в бой, защищая свою родину.

Алёша Попович \explain{по сравнению с}{as compared with} товарищами молод и строен. Он изображён с \explainDetail{л\'{у}ком}{лук}{bow} и стрелами в руках, но \explain{прикреплённые}{attached} к \explainDetail{седлу}{седло}{saddle} гусли свидетельствуют о том, что он не только бесстрашный воин, но и \explain{гусляр}{player of the musical instrument ``gusli''}, песенник, весельчак. В картине много таких деталей, которые характеризуют образы её персонажей.

Упряжки коней, одежда, амуниция не \explain{вымышленные}{fictional}. Такие образцы художник видел в музеях и читал их описания в исторической литературе. Художник \explain{мастерски}{masterfully} передаёт состояние природы, как бы предвещающей о наступлении опасности. Но богатыри представляют собой \explainDetail{надёжную}{надёжный/-ая/-ое}{reliable} и мощную силу защитников родной земли.




\subsection{Картина «Алёнушка» ВМ Васнец\'{о}ва}
% https://muzei-mira.com/kartini_russkih_hudojnikov/1321-opisanie-kartiny-bogatyri-tri-bogatyrya-vasnecova-1898.html

Алёнушка, печальная девочка у \explainDetail{пруда}{пруд}{pond} --- одна из любимых всеми картин В. Васнецова. Художник уд\'{а}чно использует сказочный сюжет, чтобы \explainDetail{раскрыть}{раскрывать/раскрыть}{to open/to discover} сложный и неоднозначный русский характер.

Грусть девочки очень взрослая. Печаль в её глазах граничит с \explainDetail{отчаянием}{отч\'{а}яние}{despair}. Неубранные рыжие волосы, тёмные глаза, нежно-алые губы --- формируют легко читаемый образ ребёнка с тр\'{у}дной судьб\'{о}й.
В Алёнушке совсем нет ничего \explainDetail{сказочного}{сказочный}{fabulous, fairytale-like}, фантастического.
С\'{о}бственно, вся ск\'{а}зочность сюжета подчеркнута лишь одной деталью --- группой \explainDetail{ласточек}{л\'{а}сточка}{swallow}, сидящих над головой \explainDetail{героини}{героиня}{heroine}. Этим с\'{и}мволом (как известно, л\'{а}сточки символиз\'{и}руют над\'{е}жду) художник \explainDetail{уравновешивает}{уравновешивать/уравновесить}{to balance --- уравнов\'{е}шиваю/-ешь/-ют; уравнов\'{е}шу/-ишь/-ят} полный \explainDetail{тоск\'{и}}{тоск\'{а}}{yearning, longing} \'{о}браз героини, даёт надежду на счастливый финал старой русской сказки.

Васнецов нап\'{о}лнил \explain{ф\'{о}новый}{background} пейзаж атмосферой тишины и гр\'{у}сти.
Отлично удал\'{и}сь художнику в\'{о}дная \explain{гладь}{smooth surface} пруда, \explainDetail{камыш\'{и}}{камыш}{reed}, осока, \explainDetail{ели}{ель}{fir tree}.
Всё \explainDetail{неподв\'{и}жно}{неподв\'{и}жный/-ая/-ое}{still, motionless}, тихо, спокойно.
Даже пруд отраж\'{а}ет героиню очень деликатно, \explain{слегк\'{а}}{slightly}.
Чуть трепещут молодые \explainDetail{ос\'{и}ны}{ос\'{и}на}{aspen}. \explainDetail{Едва}{едв\'{а}}{barely, hardly} \explain{хмурится}{turns gloomly} ос\'{е}ннее небо.
Тёмные, зелёные тона пейзажа контрастируют с \explainDetail{румянцем}{румянец}{blush} на лице героини, а ос\'{е}нняя грусть --- с яркими цветами на юбке Алёнушки. Зритель чувствует: ещё мгнов\'{е}ние и сказка прод\'{о}лжится\dots







\subsection{Картина «Витязь на распутье»}

Виктор Михайлович Васнецов с циклом работ, \explain{посвященных}{dedicated (посвященный + \textit{дат.})} сюжетам русских сказок и былин, оказался \explainDetail{новатором}{новатор}{innovator} в этой области \explainDetail{изобразительного искусства}{изобраз\'{и}тельное искусство}{visual art}. За ним закрепилась репутация «художника-сказочника», он настолько проникся духом русской старины и былинного времени, что свой московский дом построил в виде деревянной избы (сейчас там находится мемориальный музей \explainDetail{живоп\'{и}сца}{живоп\'{и}сец}{painter, artist}).

Картина «В\'{и}тязь на расп\'{у}тье» \explain{отч\'{а}сти}{partly} является и отражением судьбы Васнецова.
Будучи \explainDetail{пр\'{и}знанным}{пр\'{и}знанный}{recognised} художником-передвижником, он, как и его товарищи, \explainDetail{исполнял}{исполнять/исполнить}{performed} жанровые композиции в духе остросоциальных тем, волновавших общество в 1870-1890-х.
Но завладевшая им сказочная тематика диктовала \explainDetail{ин\'{о}е}{ин\'{о}й/ин\'{а}я/ин\'{о}е}{(определительное местоимение) другой, отличный от данного; (неопределённое местоимение) некоторый. See \href{https://ru.wiktionary.org/wiki/\%D0\%B8\%D0\%BD\%D0\%BE\%D0\%B9}{wikictionary.org:иной}.} развитие творчества. Живописец уход\'{и}л от проблем современности и \explain{погружался}{plunge, dive} в мир русской старины, рискуя быть \explainDetail{осужденным}{осужденный}{convicted}.

Выбор пути как один из \explain{роков\'{ы}х}{fatal} вопросов человеческой жизни на крупноформатном \explainDetail{холсте}{холст}{canvas} мастера приобрел эпическое звучание.
Перед камнем-предсказателем согнулся под тяжестью фатального \explainDetail{пророчества}{пророчество}{prophecy} опечаленный витязь. \explainDetail{Зловещий}{зловещий}{sinister} в\'{о}рон, садящееся красное солнце нагнетают\footnote{build up the atmosphere} атмосферу. \explainDetail{Нам\'{е}ренный}{нам\'{е}ренный}{intentional} отказ от \explainDetail{изображения}{изображение}{depiction} дороги (как выхода из трудности) художником сделан для того, чтобы показать \explain{неотврат\'{и}мость}{inevitability} судьбы.