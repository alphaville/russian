\chapter{Окружающая среда и изменение климата}

\section{Времена года}
\textit{Источник: \url{https://www.memorysecrets.ru/english-texts/vremena-goda-seasons.html}}

Каждое время года имеет свои плюсы и минусы. Давайте детальнее их разберем.

Весна (март, апрель, май) --- самое восхитительное время года. Несмотря на то, что весной \explainDetail{наблюдаются}{(по)наблюдаться}{to be observed} \explainDetail{ливни}{ливень}{shower (rain)}, многие люди \explain{воспринимают}{perceive} весну, как самое лучшее время года. Почему же так \explain{происходит}{happens}? \explain{Дело в том}{the thing is}, весна значит что-то новое для нас. Это шанс начать все заново, изменить жизнь к лучшему. Именно весной рожд\'{а}ется новая жизнь --- \explain{распускаются}{they bloom} почки и начинают цвести цветы.

Лето же (июнь, июль, август) --- это время отдыхать и просто получать удовольствие. Солнце светит с утра до ночи. Взрослые уходят в отпуск, дети --- на каникулы. Большинство людей проводят свои отпуска на море. Там можно не только \explainDetail{загореть}{загорать/загореть}{to tan}, но и встретить много незнакомых людей.

Осень (сентябрь, октябрь, ноябрь) можно разделить на две части --- начало осени и ее конец. В начале осени школьники снова возвращаются к учебе и садятся \explain{за парты}{behind the desks}. Лето постеп\'{е}нно переходит в осень. Но настроение ещё \explain{приподнятое}{elevated}. Вскоре желтые листья начнут осыпаться, и этот величественный пейзаж тоже не \explainDetail{позв\'{о}лит}{позвол\'{я}ть/позв\'{о}лить}{to allow} нам грустить. Втор\'{а}я часть осени не так очаров\'{а}тельна, как первая. Дожди, \explain{сл\'{я}коть}{mud} и грязь --- это то, что люди ненавидят больше всего.

Зима (декабрь, январь, февраль) \explain{на нос\'{у}}{фразеологизм: приближаться, близиться (обычно -- по времени)}, приближаются холода. Мы надеваем перчатки, шапки, шубы и вязаные шарфы. Иногда в эту пору школы закрывают на карантин с целью предотвратить распространение гриппа. Хорошая новость для учеников --- вторые зимние каникулы. Рождество тоже вот-вот наступит. На улице порошит снег, дети лепят снеговиков и играют в снежки. Скоро мы будем наряжать елку, вешать на нее различные шары и гирлянды. Пора встречать Новый год!

\section{Что такое изменение климата?}
\textit{Источник: \url{https://www.un.org/ru/climatechange/what-is-climate-change}}

Под изменением климата понимают \ed{долгосрочные}{долгоср\'{о}чный}{long-term (for a long period)} температурные изменения и изменение погодных \ed{условий}{усл\'{о}вие}{condition}. Хотя эти изменения м\'{о}гут быть естественными, как, например, \ed{циклические}{циклический}{cyclical} \ed{колебания}{колеб\'{а}ние}{fluctuation} солнечной активности, с \ed{1800-х}{1800-х}{тысячи восемьсотых} годов антропогенная деятельность является основным \explain{движущим фактором}{driving factor} изменения климата, \explain{главным образом}{mainly} за счёт \ed{сжигания}{сжигание}{burning; combustion} ископаемых видов топлива, таких как уголь, нефть и газ.

В результате сжигания ископаемых видов топлива образуются \explain{выбросы}{emissions} \ed{парниковых газов}{парник\'{о}вые г\'{а}зы}{greenhouse gases}, которые подобно одеялу \explain{окутывают}{they envelop} Землю, удерживая солнечное тело и \explain{повышая}{raising} температуру.

Примерами парниковых газов, выбросы которых вызывают изменение климата, являются \explain{дву\'{о}кись углер\'{о}да}{углекислый газ} и мет\'{а}н. Они образуются, например, при использовании бензина для езды на автомобилях или угля для отопления зданий. \explain{Расчистка земель и лесов}{the ``clearing'' of lang and forest} также может привести к \ed{высвобождению}{высвобожд\'{е}ние}{release} углекислого газа.
Главным источником выбросов метана являются \explain{м\'{у}сорные св\'{а}лки}{garbage dumps}.
К числу основных производителей выбросов относятся энергетика, промышленность, транспорт, здания, сельское хозяйство и землеп\'{о}льзование.

\begin{fancyquotes}
    Концентрация парниковых газов находится на самом высоком уровне за последние 2 миллиона лет
\end{fancyquotes}

И выбросы продолжают расти! В результате этого сейчас Земля на \explain{1,1°C}{одна целая, одна десятая градуса Цельсии} теплее, чем в конце 1800-х годов. Прошедшее десятилетие (2011–2020 г\'{о}ды) было самым тёплым в истории.

Хотя многие думают, что изменение климата означает в основном более высокие температуры, рост температуры -- это только начало истории. Поскольку Земля -- это система, где всё \explain{взаимосв\'{я}зано}{interconnected}, изменения в одной \explain{сфере}{here: area} могут повлиять на изменения во всех остальных.

В настоящее время к последствиям изменения климата относят, \explain{среди прочего}{among other things}, сильные \ed{з\'{а}сухи}{з\'{а}суха}{drought}, \explain{нехватку воды}{water shortage}, сильные пожары, повышение уровня моря, наводнения, \explain{т\'{а}яние}{melting (n.)} полярных льдов, катастрофические штормы и сокращение \ed{биоразнообр\'{а}зия}{биоразнообр\'{а}зие}{biodiversity}.

\begin{fancyquotes}
    Люди ст\'{а}лкиваются с различными последствиями изменения климата
\end{fancyquotes}

Изменение климата может \ed{сказ\'{а}ться}{ск\'{а}зываться/сказ\'{а}ться + на что}{affect} на нашем здоровье, способности \explain{выращивать}{to grow} \explain{продовольственные культуры}{food crops}, жильё, безопасности и работе.
Некоторые из нас уже сейчас более \ed{уязв\'{и}мы}{уязв\'{и}мый}{vulnerable} к \ed{воздействию}{возд\'{е}йствие}{impact (n)} изменения климата, например, люди, живущие в малых \ed{островн\'{ы}х}{островн\'{о}й}{insular} государствах и других развивающихся странах. Такие последствия, как повышение уровня моря и интрузия соленых вод, дост\'{и}гли такого уровня, что целые общ\'{и}ны были в\'{ы}нуждены \explain{пересел\'{и}ться}{to resettle}, а затяжные засухи \explain{подвергают}{subject; expose} людей риску г\'{о}лода. В будущем ожидается рост числа «климатических беженцев».

\begin{fancyquotes}
    Каждый дополнительный градус \ed{глобального потепления}{глобальное потепление}{global warming} \explain{имеет значение}{matters}
\end{fancyquotes}

В докладе Организации Объединённых Наций за 2018 год тысячи учёных и \ed{правительственных экспертов}{прав\'{и}тельственный эксп\'{е}рт}{gov't expert} согласились с тем, что \explain{ограничение}{limitation} роста глобальной температуры на уровне не более 1,5°C поможет нам избежать самых худших климатических последствий и \explain{сохранить}{save} кл\'{и}мат, \explain{приг\'{о}дный для жизни}{habitable}. Однако, согласно текущим национальным климатическим планам, ожидается, что глобальное потепление дост\'{и}гнет 2,7°C к концу столетия.

Хотя выбросы, вызывающие изменение климата, образуются во всех регионах мира и сказываются на всех, некоторые страны производят их в гораздо больших объёмах, чем другие. В то время как на 100 стран, которые производят меньше всего выбросов, приходится 3 процента от общего объема выбросов, \explain{доля}{share (n)} десяти стран, являющихся самыми крупными производителями выбросов, составляет 68 процентов. Хотя принятие мер по борьбе с изменением климата -- это дело всех и каждого, народы и страны, которые создают больше проблем, должны взять на себя большую ответственность и начать действовать первыми.

\begin{fancyquotes}
    Мы сталкиваемся с огромным \ed{вызовом}{вызов}{challenge}, но нам уже известны многие решения
\end{fancyquotes}

Многие решения в области изменения климата могут быть не только экономически выгодными, но и также улучшить нашу жизнь и защитить окружающую среду. У нас также есть глобальные структуры и соглашения, на основе которых осуществляются \ed{усилия}{усилие}{effort}, направленные на достижение прогресса, такие как цели в области \ed{устойчивого развития}{устойчивый развитие}{sustainable development}, Рамочная конвенция Организации Объединённых Наций об изменении климата и Парижское соглашение. Имеются три широкие категории действий: сокращение выбросов, адаптация к последствиям изменения климата и финансирование необходимых мер по адаптации.

\explain{Перевод}{Translation (meaning transition)} энергетических систем с ископаемого топлива на использование \ed{возобновляемых источников энергии}{возобновляемые источники энергии}{renewable energy sources}, таких как солнце или ветер, позв\'{о}лит сократить выбросы, вызывающие изменение климата. Но начинать нужно прямо сейчас. Несмотря на расширение коалиции стран, \explain{обязавшихся}{who pledged} достичь чистого нулевого уровня выбросов к 2050 году, около половины мер по сокращению выбросов должны быть осуществлен\'{ы} к 2030 году, с тем чтобы удержать потепление на уровне ниже 1,5°C. В период с 2020 по 2030 год производство ископаемого топлива должно сокращаться прим\'{е}рно на 6 процентов в год.

Цель адаптации к последствиям изменения климата состоит в том, чтобы защитить людей, их жилища, предприятия, источники средств к существованию, инфраструктуру и природные экосистемы. Эта деятельность касается не только нынешних последствий, но и тех, с которыми, вероятно, придётся столкнуться в будущем. Хотя меры по адаптации придётся принимать \explain{повсеместно}{everywhere}, в настоящее время приоритетное внимание необходимо уделять наиболее уязв\'{и}мым сло\'{я}м населения, у которых меньше всего ресурсов для того, чтобы \explain{противостоять}{resist} климатическим угр\'{о}зам. Это может \explain{окупиться}{pay off} \explain{сполна}{completely}. Например, системы раннего предупреждения \ed{стихийных бедствий}{стихийное бедствие}{natural disaster} спасают жизни и имущество и могут принести выгоды, в десятки раз превышающие первоначальные \explain{затраты}{costs}.

\begin{fancyquotes}
    Мы можем оплатить счет сейчас или дорого заплатить в будущем
\end{fancyquotes}

Меры по борьбе с изменением климата требуют значительных финансовых \ed{вложений}{вложение}{investment} со стороны правительств и делов\'{ы}х кругов. Однако \explain{бездействие}{inaction} в отношении климата обходится гораздо дороже. Одним из важнейших шагов является выполнение \ed{промышленно развитыми странами}{промышленно развитая стран\'{а}}{industrialised countries} своих обязательств по предоставлению 100 миллиардов долларов в год \ed{развивающимся странам}{развив\'{а}ющаяся стран\'{а}}{developing country}, с тем чтобы они могли адаптироваться и перейти к более «зеленой» экономике.


\newpage
\section{Причины изменения климата}
\textit{Источник: \url{https://www.un.org/ru/science/causes-effects-climate-change}}



\textbf{Производство электроэнергии}

Значительная доля глобальных выбросов связана с производством электроэнергии и тепла путем сжигания ископаемых видов топлива. Бóльшая часть электроэнергии по-прежнему производится посредством сжигания угля, нефти или газа, в результате чего образуются углекислый газ и закись азота – мощные парниковые газы, которые покрывают Землю и \explain{задерживают}{trap; hold} солнечное тепло. Во всем мире чуть более четверти электроэнергии вырабатывается \explain{за счёт}{here it means ``by means of,'' but it can also mean ``at the expense of.''} ветра и солнца и поступает из других возобновляемых источников, которые, в отличие от ископаемых видов топлива, практически не \explain{выделяют}{emit} в атмосферу парниковых газов или \ed{загрязняющих веществ}{загрязняющее вещество}{pollutant}.

\textbf{Изготовление\footnote{изготовление: manufacturing} товаров}

Предприятия обрабатывающей и других отраслей промышленности производят выбросы, в большинстве случаев являющиеся результатом сжигания ископаемых видов топлива в целях выработки энергии, необходимой для получения цемента, железа, стали, электронных устройств, \ed{пластмасс}{пластм\'{а}сса}{plastic}, одежды и других товаров.
При добыче полезных ископаемых и других промышленных процессах, \explain{равно как}{as well as} и при строительстве, также выделяются газы. Машины, используемые в производственном процессе, \explain{зачаст\'{у}ю}{often} работают на угл\'{е}, нефти или газе, а некоторые материалы, такие как пластмассы, производятся из химических веществ, получаемых из ископаемых видов топлива. Обрабатывающая промышленность является одним из крупнейших источников выбросов парниковых газов в мире.

\textbf{Вырубка лесов\footnote{в\'{ы}рубка лес\'{о}в: deforestation}}

В результате вырубки лесов для создания ферм или пастбищ либо по иным причинам образуются выбросы, поскольку вырубаемые деревья высвобождают накопленный углерод. Ежегодно уничтожается около 12 млн гектаров леса. Поскольку леса \explain{поглощают}{absorb} углекислый газ, их уничтожение также ограничивает способность природы удерживать выбросы в атмосферу. \ed{Обезл\'{е}сение}{обезл\'{е}сение}{deforestation} \explain{наряду с}{along with; alongside} сельским хозяйством и другими изменениями в землепользовании является причиной примерно четверти глобальных выбросов парниковых газов.

\textbf{Использование транспорта}

Большинство автомобилей, грузовиков, кораблей и самолётов работают на ископаемых видах топлива. Это делает транспорт одним из главных источников выбросов парниковых газов, особенно выбросов углекислого газа. Наибольшая их часть приходится на дорожные транспортные средства в связи со сжиганием продуктов нефтепереработки, таких как бензин, в двигателях внутреннего сгорания. При этом выбросы морск\'{и}х и воздушных суд\'{о}в продолжают расти. На транспорт приходится почти четверть глобальных выбросов углекислого газа, связанных с энергоснабжением. \explain{Существующие тенденции}{Existing trends} \explain{указывают на}{indicate} вероятность значительного увеличения энергопотребления в транспортном секторе в ближайшие годы.

\textbf{Производство продуктов питания}

Производство продуктов питания прив\'{о}дит к выбросам углекислого газа, метана и других парниковых газов разными путями, включая вырубку лесов и расчистку земель для ведения сельского хозяйства и выпаса скота, работу пищеварительных систем коров и овец, производство и применение удобрений и навоза для выращивания сельскохозяйственных культур и использование энергии для эксплуатации сельскохозяйственного оборудования или рыболовецких судов, обычно работающих на ископаемых видах топлива. Все это делает производство продуктов питания одним из основных факторов, способствующих изменению климата. Выбросы парниковых газов также связаны с \ed{упаковкой}{упаковка}{packaging} и \ed{распростран\'{е}нием}{распростран\'{е}ние}{here: distribution} продуктов питания.

\textbf{Энергоснабжение зданий\footnote{энергоснабжение зданий: energy supply of buildings}}

В мировом масштабе жил\'{ы}е и коммерческие здания потребляют более половины всей электроэнергии. В связи с продолжающимся использованием угля, нефти и природного газа для целей отопления и \ed{охлаждения}{охлаждение}{cooling} они выбрасывают значительные количества парниковых газов. В последние годы повышение \ed{спроса}{спрос}{demand} на энергию для \ed{отопления и охлаждения}{отопление и охлаждение}{heating and cooling} с р\'{о}стом ч\'{и}сленности влад\'{е}льцев \ed{кондиционеров}{кондицион\'{е}р}{air conditioner} и увеличение потребления электричества для \ed{освещения}{освещение}{lighting} и обеспечения работы бытовой техники и подключенных устройств способствовали увеличению выбросов углекислого газа, производимых зданиями и связанных с энергоснабжением.

\textbf{Слишком интенсивное потребление}

Ваш дом и использование электроэнергии, то, как вы передвигаетесь, то, что вы едите, и количество того, что вы выбрасываете, влияют на выбросы парниковых газов. Это же можно сказать о потреблении таких товаров, как одежда, электронные устройства и пластмассы. Значительная часть глобальных выбросов парниковых газов связана с частными домохозяйствами. Наш образ жизни оказывает глубокое воздействие на нашу планету. Самые \explain{состоятельные}{состоятельный}{wealthy; well-off} л\'{и}ца несут наибольшую ответственность\footnote{Recall: нести ответственность: to have responsibility}: на 1 процент самых богатых жителей планеты в совокупности приходится больше выбросов парниковых газов, чем на 50 процентов беднейшего населения.

На основе различных источников ООН

\newpage
\section{Последствия изменения климата}
\textit{Источник: \url{https://www.un.org/ru/science/causes-effects-climate-change}}


\textbf{Повышение температур}

С увеличением концентрации парниковых газов растёт и глобальная температура земной поверхности. Последнее десятилетие --- 2011–2020 годы --- стало самым тёплым за всю историю наблюдений. С 1980-х годов каждое десятилетие было теплее предыдущего. Почти во всех районах суши наблюдается увеличение количества жарких дней и периодов аномальной жары. Повышение температуры увеличивает количество заболеваний, связанных с жарой, и затрудняет работу на открытом воздухе. Природные пожары легче возникают и быстрее распространяются в более жарких условиях. Температура в Арктике повышалась по крайней мере вдвое быстрее, чем в среднем по миру.

\textbf{Усиление штормов}

Многие регионы \ed{столкн\'{у}лись}{ст\'{а}лкиваться/столкн\'{у}ться + с}{were faced with} с увеличением интенсивности и частоты \ed{разрушительных}{разрушительный}{destructive} штормов. При повышении температуры \explain{испаряется}{evaporates} больше \ed{влаги}{влага}{humidity}, что усиливает ливневые дожди и наводнения, вызывая более опасные штормы. На частоту и масштабы тропических штормов также влияет потепление океана. Циклоны, ураганы и тайфуны формируются в тёплых водах у поверхности океана. Такие ураганы нередко разрушают дом\'{а} и \explain{населенные пункты}{populated areas; settlements}, становясь причиной гибели людей и огромных экономических потерь.

\textbf{Усиление з\'{а}сухи}

Изменение климата меняет степень доступности воды, делая её более дефицитным ресурсом в растущем числе регионов. Глобальное потепление усугубляет нехватку воды в регионах, и без того испытывающих её дефицит, и увеличивает риск сельскохозяйственных засух, влияющих на урожай, и экологических засух, повышающих уязвимость экосистем. Засухи также могут вызывать разрушительные \explain{песчаные}{sandy} и пыльные бури, способные перемещать миллиарды тонн песка через континенты. Пустыни расширяются, сокращая площадь земель для выращивания продовольственных культур. Сегодня многие люди постоянно сталкиваются с угр\'{о}зой нехватки воды.

\textbf{Потепление и повышение уровня океана}

Океан \explain{поглощает}{absorbs} бóлчьшую часть тепла, образующегося в процессе глобального потепления. За последние двадцать лет скорость, с которой океан \explain{нагревается}{heats up}, сильно возросла на всех его глубинах. По мере потепления океана его объём увеличивается, поскольку при нагревании вода расширяется. Таяние ледовых щитов также приводит к повышению уровня моря, \explain{угрожая}{threatening} прибрежным и островным сообществам. Кроме того, океан поглощает из атмосферы углекислый газ. При этом увеличение количества углекислого газа повышает кислотность океана, что ставит под угрозу морскую флору и ф\'{а}уну и коралловые рифы.

\textbf{Исчезновение видов}

Изменение климата создает риски для выживания видов на суше и в океане. Эти риски возрастают по мере повышения температуры. Мир, положение в котором усугубляется изменением климата, теряет виды в тысячу раз быстрее, чем когда-либо в письменной истории человечества. Миллион видов находится под угрозой исчезновения в течение следующих нескольких десятилетий. В число многочисленных угроз, связанных с изменением климата, входят лесные пожары, экстремальные погодные условия и инвазивные вредители и заболевания. Некоторые виды смогут сменить место обитания и выжить, а другие нет.

\textbf{Нехватка продуктов питания}

В группу причин глобального роста распространенности голода и неполноценного питания входят климатические изменения и увеличение количества экстремальных погодных явлений. Рыбные ресурсы, сельскохозяйственные культуры и домашний скот могут быть уничтожены или стать менее продуктивными. В связи с закислением океана морские ресурсы, обеспечивающие питание для миллиардов людей, находятся под угрозой. Изменения снежного и ледяного покрова во многих арктических регионах нарушили систему снабжения продовольствием, обеспечиваемым за счет пастбищного животноводства, охоты и рыболовства. Тепловой стресс может уменьшать количество воды и пастбищ, что приводит к снижению урожайности сельскохозяйственных культур и негативным образом сказывается на поголовье скота.

\textbf{Увеличение рисков для здоровья}

Изменение климата – это самая большая угроза для здоровья людей. Его последствия уже наносят вред здоровью в связи с загрязнением воздуха, распространением заболеваний, возникновением экстремальных погодных явлений, вынужденным перемещением, оказанием давления на психику и обострением проблем голода и неполноценного питания в местах, где люди не могут выращивать продовольственные культуры или обеспечить наличие достаточного количества пищевых продуктов. Экологические факторы ежегодно уносят жизни около 13 млн человек. Изменение погодных условий приводит к распространению заболеваний, а экстремальные погодные явления увеличивают смертность и затрудняют работу систем здравоохранения.

\textbf{Нищета и вынужденное перемещение}

Изменение климата усиливает факторы, ввергающие людей в нищету и не позволяющие им исправить ситуацию, в которой они оказались. Наводнения могут смести городские трущобы, разрушив дома и уничтожив источники средств к существованию. Жара может затруднить работу на открытом воздухе. Нехватка воды может повлиять на урожай. В последние десять лет (2010–2019 годы) связанные с погодой явления приводили к вынужденному перемещению в среднем около 23,1 млн человек в год, повышая риск оказаться в нищете для еще большего числа людей. Большинство беженцев прибывают из самых уязвимых стран, наименее готовых адаптироваться к последствиям изменения климата.

\textit{На основе различных источников ООН}

\newpage
\section{Человечество ждёт серия масштабных природных катастроф}

\textit{В чем причина стихийных бедствий ближайшего будущего?}

\textit{Источник: \url{https://lenta.ru/articles/2022/10/17/trilliondollarbaby/}}

В скором будущем человечество ожидает уникальное для этого столетия явление. Так называемая Ла-Нинья — феномен, с которым связано широкомасштабное охлаждение вод Тихого океана, — грозит оказаться тройной. Последствия Ла-Ниньи, чье название с испанского можно перевести как «девочка», могут стать катастрофическими. Если прогноз сбудется, хаос природных катаклизмов на планете только усилится. В таком сценарии глобальный ущерб от стихийных бедствий достигнет триллиона долларов. Роковая «малышка» — в материале «Ленты.ру».

\textbf{Горячо-холодно.} Нынешняя затяжная Ла-Нинья началась в сентябре 2020 года. Явление уже успело усугубить засуху и наводнения в различных частях мира, а теперь же Всемирная метеорологическая организация (ВМО) предупредила, что явление затянется надолго. С вероятностью в 70 процентов оно сохранится до ноября, с вероятностью в 55 процентов — до февраля следующего года.

Ситуация, как подчеркнул генеральный секретарь ВМО Петтери Таалас, складывается исключительная. В нынешнем столетии природный феномен может охватить сразу три зимы в Северном полушарии и три лета в Южном. Если прогноз ВМО окажется верным, то тройная Ла-Нинья станет третьей по счету за всю историю наблюдений — с 1950 года подобное случалось лишь дважды.

Ла-Нинья (исп. La Niña — малышка, девочка) — это снижение температуры поверхностных вод в центральной и восточной частях Тихого океана. На этом фоне усиливаются стабильные восточные ветра, которые гонят теплую воду от берегов Перу и Чили в сторону Индонезии и Австралии. В результате средние температуры воздуха во всем мире понижаются.

Явление с противоположным эффектом, когда температура воды и воздуха у побережья Южной Америки растет, называют Эль-Ниньо (исп. El Niño — малыш, мальчик), а его чередование с Ла-Ниньей — Южной осцилляцией. Феномен описал британский ученый Гилберт Уокер в 1923 году, однако впервые на него обратили внимание перуанские рыбаки в 1600 годах — Ла-Нинья для них не имела критического значения, но потепление воды при Эль-Ниньо плохо сказывалось на уловах.

На климат нашей планеты Ла-Нинья оказывает охлаждающий эффект, который достигает своего максимума на второй год после ее пика. Характерными проявлениями феномена метеорологи называют прохладную и влажную зиму на севере Европы и в Великобритании, дождливое лето в Индонезии и Австралии, сильные муссоны (ветры, дующие с суши на океан) в Юго-Восточной Азии, холода в Южной части Африки. В этот период зима на Дальнем Востоке в России, в Японии, Корее, Канаде, на севере США обычно бывает более снежной и ветреной. В Техасе, Флориде и других южных штатах Америки, наоборот, случаются сильные засухи.

При этом связанное с деятельностью человека изменение климата, как отмечают ученые, усиливает воздействие Южной осцилляции на окружающую среду. ВМО выяснила, что за последние 50 лет число стихийных бедствий пятикратно увеличилось.

\begin{fancyquotes}
    Это означает более длительные и интенсивные волны тепла, более длительные засухи, более масштабные лесные пожары и более разрушительные наводнения. Усиленный изменением климата эффект от Ла-Ниньи носит глобальный характер

    \begin{flushright}
        Билл Пацертамериканский \\
        океанограф и климатолог
    \end{flushright}
\end{fancyquotes}

По словам ученой из Национального управления океанических и атмосферных исследований при министерстве торговли США Антониетты Капотонди, сила воздействия Южной осцилляции на климат непредсказуема — не наблюдалось двух абсолютно одинаковых по эффекту Эль-Ниньо и Ла-Ниньи. «Мы видели, насколько разнообразными могут быть последствия Южной осцилляции. Эта вариативность усложняет прогнозы по влиянию изменения климата на будущие Эль-Ниньо и Ла-Нинья», — пояснила исследовательница.

Ученые предполагают, что явления Эль-Ниньо и Ла-Нинья в перспективе будут случаться чаще. На фоне активного загрязнения планеты выбросами парниковых газов к концу XXI века они будут возникать не один раз в 20 лет, а раз в десятилетие. При этом осадки сместятся на восток вдоль экваториальной части Тихого океана во время Эль-Ниньо и на запад во время последствий Ла-Ниньи.

При увеличении периодичности и амплитуды Южной осцилляции экстремальные погодные явления будут более выраженными. Некоторые эффекты воздействия глобального потепления на протекание Эль-Ниньо и Ла-Ниньи уже стали очевидными — к примеру, тихоокеанские штормы заметно участились и усилились. В 2020 году зафиксировали рекордные 30 таких ураганов, в 2021-м — 21, а в 2022 — 14.

\textbf{Всем досталось.} Связанные с Ла-Ниньей экстремальные погодные условия затронули многие регионы планеты. Так, засуха накрыла западные части США и Канады, практически опустошив находящиеся на этих территориях водоемы. В результате возник дефицит пресной воды — ее не хватает ни для поливов сельскохозяйственных земель, ни для производства энергии на гидроэлектростанциях. В частности, в Техасе засуха привела к огромным потерям урожая хлопка — цены на него достигли десятилетнего максимума в начале 2022 года. По словам главного исполнительного директора компании Plains Cotton Growers Inc Коди Бессета, долговременная жара вкупе с практическим отсутствием осадков рискуют сделать 2022 год самым сложным для производства сельскохозяйственных культур. Эксперт прогнозирует, что производители недосчитаются более половины привычного объема урожая.

Продолжительная сухая погода также нанесла ущерб урожаю кофе, сахара и апельсинов в Бразилии, которая считается крупнейшим в мире экспортером этих трех культур. Помимо этого, экстремальные погодные условия нарушили работу предприятий по добыче железной руды в стране — в том числе второго по величине в мире производителя Vale SA.

В Аргентине засуха сказалась на посевах сои и кукурузы, которые играют ключевую роль в торговом балансе страны, переживающей острый экономический кризис. Многолетняя нехватка осадков также высушила ключевой водный путь — реку Парану. Поэтому фермеры и продавцы сельхозтоваров столкнулись с дополнительными расходами на логистику — им приходится платить внушительные суммы за отправку товаров через альтернативные порты.

На Австралию, наоборот, обрушились проливные дожди. Ливнями затопило большую часть штатов Новый Уэльс, Квинсленд и Виктория. Наводнения привели к гибели более 20 человек и разрушили более 15 тысяч домов. Страховые выплаты при этом превысили три миллиарда долларов.

Сильные дожди негативно сказались на качестве урожая уже выросших зерновых культур, а также задержали время посева ячменя и пшеницы. Кроме того, от разлива водоемов под угрозой выхода из строя оказались металлургические шахты в Новом Южном Уэльсе и Квинсленде — они являются крупнейшими в мире экспортерами сырья для производства стали. Другим катастрофическим последствием стала гибель животных. Так, связанные с Ла-Ниньей теплые течения в сторону востока погубили пингвинов, которых массово выбросило на пляжи Новой Зеландии.

Пакистан вызванные Ла-Ниньей обильные ливни поставили под угрозу дефолта. Страна подверглась беспрецедентным наводнениям, в результате которых были разрушены десятки миллионов домов, погибло почти 1500 человек. Кроме того, чрезмерные осадки повредили более 70 процентов посевов риса в провинции Синд. Нанесенный стихийными бедствиями ущерб аналитики оценивают в 30 миллиардов долларов. Южноазиатская страна и так страдает от сокращения валютных резервов и самой высокой инфляции за последние десятилетия, а природные факторы лишь усугубят тяжелое экономическое положение.

По данным Международной федерации обществ Красного Креста и Красного Полумесяца, разрушительные наводнения случились также в Бангладеш и Индии — от них пострадали более 7,2 миллиона человек и оказались поврежденными более 300 тысяч домов.

\textbf{Час расплаты.} Эксперты убеждены, что связанные с Ла-Ниньей погодные циклы усугубляют уже существующие глобальные проблемы. «Когда добавляются экстремальные погодные условия, это просто создает сценарий для более высоких цен на энергию, более высоких цен на продукты питания и большей инфляции. Это негативно для мировой экономики», — пояснил президент и основатель Pento Portfolio Strategies Майкл Пенто.

Ущерб от вызванных Ла-Ниньей засух, штормов и наводнений аналитики оценивают в десятки миллиардов долларов. Однако в этот раз природные катаклизмы случались так часто в разных уголках планеты, что реальную сумму потерь сложно вычислить. Получить представление о размере ущерба можно на основании данных страховых компаний. Как сообщают в аналитической и консалтинговой фирме Aon, в 2020 году стихийные бедствия обошлись миру в 268 миллиардов долларов, а в 2021-м — в дополнительные 329 миллиардов долларов.

\begin{framed}
    \begin{center}
        {\Huge
            1 триллион долларов
        }

        {\Large
            составит глобальный ущерб от последствий Ла-Ниньи
        }
    \end{center}
\end{framed}

Если будущий масштаб стихийных бедствий окажется соизмеримым с прошлыми годами, то к концу 2023-го сумма ущерба от Ла-Ниньи может достичь или даже превысить один триллион долларов. При этом речь идет не только о повреждении имущества и потерях урожая. От погодных условий зависят цены на разные товары по всему миру — от чашки кофе до угля, который используют для производства стали. Помимо войн, Ла-Нинья — единственное событие, оказывающее влияние на глобальные рынки и почти все отрасли экономики, считают эксперты.

\textbf{Выход есть?} Сократить риск негативных для планеты последствий явления можно за счет снижения антропогенного воздействия на климат — сокращения уровня эмиссии парниковых газов. Для этого, прежде всего, необходимо отказаться от использования ископаемого топлива — угля, нефти, природного газа и прочих горючих минералов — в пользу возобновляемых источников энергии, таких как солнечный свет, ветер и вода.

На протяжении нескольких десятилетий лидирующие позиции по энергопереходу занимает Евросоюз (ЕС) — в странах блока заметная доля в энергобалансе приходится на зеленые источники. Власти ЕС также поставили амбициозные цели по спасению планеты. Согласно опубликованной в 2021 году стратегии по борьбе с изменением климата, к 2030 году выбросы парниковых газов на территории Европы должны сократиться на 55 процентов относительно уровня 1990 года, а к 2050-му — быть полностью компенсированы.

Однако специальная военная операция России, начатая в феврале на Украине, вынудила ЕС поступиться важной целью. Блок стран резко сократил импорт дешевого российского газа (с 40 процентов в 2021 году до 7 процентов от общего объема) и столкнулся с энергетическим кризисом, так как существующих солнечных, ветряных и гидроэлектростанций недостаточно для обеспечения бесперебойной подачи тепла и света. Чтобы компенсировать нехватку, Европе пришлось вернуться к использованию сильно вредящих планете угля и мазута.

Все это делает недостижимыми цели Парижского соглашения по климату, которое подразумевает сдерживание роста среднемировых температур на отметке 1,5 градуса Цельсия относительно доиндустриального уровня. По прогнозам ученых, даже без учета наихудших сценариев средняя температура на планете повысится от 2,1 до 3,9 градуса Цельсия к 2100 году.

Межправительственная группа экспертов по изменению климата ООН уже предрекла человечеству неминуемую глобальную катастрофу. По прогнозам ученых, повышение температуры на планете в ближайшие десятилетия достигнет критических значений для сельского хозяйства и здравоохранения, а эффект глобального потепления ощутят на себе все страны мира.

Поскольку рост среднемировых температур продолжается, а глобальные выбросы СО2 не сокращаются, вероятно, Эль-Ниньо и Ла-Нинья станут лишь усиливаться, а вместе с ними увеличится и число катастрофических явлений. Среди угроз — более интенсивные дожди, наводнения, сильнейшая засуха во многих регионах, лесные пожары, опасность для прибрежных территорий из-за подъема уровня Мирового океана, усиление таяния вечной мерзлоты. Все это способно привести к затяжным экономическим кризисам, массовым проблемам со здоровьем и гибели людей и животных.

\newpage
\section{Городск\'{а}я жизнь}
Вы никогда не думали о последствиях жизни в городе? Казалось бы, \explain{на первый взгляд}{at first sight}, что жизнь в больших экономических и культурных центрах имеет только преимущества, но дальнейшее рассмотр\'{е}ние показывает, что она имеет и \explain{недостатки}{недостаток: disadvantage}.

С \explain{положительной}{полож\'{и}тельный/-ая: positive} стороны, легче найти работу в городе, потому что там обычно много ресторанов, кафе, гостиниц, школ, библиотек, музеев и т.д. Кроме того, жители города имеют прекрасную возможность посетить множество культурных и развлекательных учреждений, таких как музеи, галереи, ночные клубы, дискотеки и многое другое.

С другой стороны, жителям города \explain{приходится}{приходиться/прийтись: have to} жить в загрязненной атмосфере из-за интенсивного автомобильного движения и \explain{промышленных}{industrial} \explain{предприятий}{предпри\'{я}тие: enterprise}. Это может \explain{вызвать}{to cause} \explain{заболевания}{disease} лёгких и проблемы с сердцем. Кроме того, городской образ жизни довольно \explain{напряженный}{intense}, \explain{поскольку}{since/because} приходится много работать, много ездить на автомобиле и, в результате, стоять в пробках...

В заключение, городская жизнь имеет некоторые преимущества. Тем не менее, она также \explain{может нанести ощутимый вред}{can cause significant damage}, так что местные власти должны сделать несколько важных решений, например, они должны \explain{запретить}{[запрещать] to ban} промышленные предприятия в городах и вблизи городов, которые загрязняют воздух и воду токсичными парами.

\newpage
\section[Загрязнение окружающей среды]{Причины и последствия загрязнения окружающей среды}
\textit{Источник: \url{https://bit.ly/3NV3JeZ}}

Загрязнение окружающей среды в настоящее время является самой большой проблемой, с которой сегодня \explain{ст\'{а}лкивается}{faces, is facing} мир. Наприм\'{е}р, в Соединенных Штатах 40\% рек и 46\% озёр слишком загрязнен\'{ы} для \ed{рыбной л\'{о}вли}{рыбная ловля}{fishing}, \ed{купания}{купание}{bathing} и водных организмов. Это \explain{неудивительно}{not surprising}, когда ежегодно в американские воды \explain{сбрасывается}{dumped} 1,2 триллиона галлонов \explain{неочищенных}{untreated} ливневых вод, промышленных \ed{отходов}{отх\'{о}ды}{waste} и неочищенных \ed{сточных вод}{ст\'{о}чные в\'{о}ды}{sewage}.

Одна треть верхнего \ed{слоя}{спой}{layer} \ed{почвы}{почва}{soil} в мире уже деградирована, и \explain{с учётом}{taking into account} нынешних темпов деградации почвы, вызванной неправильными методами ведения сельского хозяйства и промышленности, а также \ed{обезлесением}{обезлесение}{deforestation}, большая часть верхнего слоя почвы в мире может исчезнуть в течение следующих 60 лет.

Великий смог 1952 года унёс жизни 8000 человек в Лондоне. Это событие \explain{было вызвано}{was caused (by)} периодом холодной погоды \explain{в сочетании с}{in conjunction with} безветренными условиями, которые сформировали \explain{плотный}{dense} слой переносимых по воздуху загрязнителей, в основном от угольных электростанций, над городом.

Существует множество источников загрязнения, каждый из которых по-своему влияет на окружающую среду и живые организмы. В этой статье обсуждаются проблема загрязнения и последствия различных видов загрязнения.

\textbf{Причины.} Причины загрязнения не \explain{ограничиваются}{are limited} только выбросами \ed{ископаемого топлива}{ископ\'{а}емое т\'{о}пливо}{fossil fuel} и \ed{углерода}{углерод}{carbon}. Существует множество других типов загрязнения, включая химическое загрязнение \ed{водоёмов}{водоём}{reservoir} и п\'{о}чвы в результате неправильной утилизации и сельскохозяйственной деятельности, а также шумов\'{о}е и светов\'{о}е загрязнение, создаваемое городами и урбанизацией в результате роста населения.

\textbf{Загрязнение воздуха.}
Существует два типа \ed{загрязнителей}{загрязн\'{и}тель}{pollutant} воздуха: \ed{первичные}{перв\'{и}чный}{primary} и \ed{вторичные}{вторичный}{secondary}. Первичные загрязнители выбрасываются \explain{непосредственно}{directly} из их источника, в то время как вторичные загрязнители образуются, когда первичные загрязнители вступают в реакцию в атмосфере.

\ed{Сжигание}{сжиг\'{а}ние}{burning, combustion} ископаемого топлива для тр\'{а}нспорта и электричества производит как первичные, так и вторичные загрязнители и является одним из крупнейших источников загрязнения воздуха.

\ed{Выхлопные газы}{выхлопные газы}{traffic fumes; exhaust fumes} автомобилей содержат опасные газы и \explain{твёрдые частицы}{solid particles}, включая \ed{углеводороды}{углеводород}{hydrocarbon}, оксиды азота и монооксид углерода. Эти газы \explain{поднимаются}{rise} в атмосферу и \explain{вступают в реакцию}{react} с другими атмосферными газами, создавая ещё б\'{о}лее токсичные газы.

По данным Института Земли, интенсивное использование \ed{удобрений}{удобрение}{fertiliser} в с\'{е}льском хоз\'{я}йстве является основным источником загрязнения воздуха \ed{мелкими частицами}{мелкие частицы}{microparticles}, что \explain{затронуло}{affected} большую часть Европы, России, Китая и США. Считается, что уровень загрязнения, вызванного сельскохозяйственной деятельностью, \explain{превышает}{exceeds} все другие источники загрязнения воздуха мелкими частицами в этих странах.

\ed{Аммиак}{амми\'{а}к}{ammonia} -- это основной загрязнитель воздуха, \explain{образующийся}{emerging} в результате сельскохозяйственной деятельности. Аммиак попадает в воздух в виде газа из концентрированных отходов животноводства и полей, которые \explain{чрезм\'{е}рно}{excessively} уд\'{о}брены.

Затем этот \explain{газообразный}{gaseous} аммиак соединяется с другими загрязнителями, такими как оксиды и сульфаты азота, образующиеся в транспортных средствах и промышленных процессах, с образованием аэрозолей. Аэрозоли -- это \explain{крошечные}{tiny} частицы, которые могут \explain{проникать}{permeate} глубоко в лёгкие и вызывать сердечные и лёгочные заболевания.

Другие сельскохозяйственные загрязнители воздуха включают пестициды, \explain{гербициды}{herbicides} и фунгициды. Все это также способствует загрязнению воды.

\textbf{Загрязнение воды.}
Загрязнение \ed{питательными веществами}{пит\'{а}тельные веществ\'{а}}{nutrients} вызывается сточными водами и удобрениями. Выс\'{о}кие уровни питательных веществ в этих источниках попадают в водоемы и способствуют росту \ed{водорослей}{водоросли}{algae} и \ed{сорняков}{сорняк}{weed}, что может сделать воду \ed{непригодной}{непригодный}{unusable; unfit} для \ed{питья}{питьё}{drinking} и \explain{истощить}{deplete} кислород, что \explain{приведет к гибели}{will lead to death} водных организмов.

Пестициды и гербициды, \explain{применяемые}{used; applied} для сельскохозяйственных культур и жил\'{ы}х районов, концентрируются в почве и перен\'{о}сятся в грунтовые воды с дождевой водой и стоками. По этим причинам каждый раз, когда кто-то пробуривает \ed{скважину}{скважина}{well (water well)} на воду, её необходимо проверять на \explain{наличие}{availability} загрязняющих веществ.

Промышленные отходы являются одной из основных причин загрязнения воды, поскольку они создают первичные и вторичные загрязнители, включая \ed{серу}{сера}{sulfur}, \explain{свинец}{lead} и \explain{ртуть}{mercury}, нитраты и фосфаты, а также разливы нефти.

В \ed{развивающихся странах}{развив\'{а}ющиеся стр\'{а}ны}{developing countries} около 70\% \explain{твёрдых отходов}{solid waste} сбрасывается непосредственно в океан или море. Это вызывает серьёзные проблемы, включая причинение вреда и убийство \ed{морских существ}{морские существа}{sea creatures}, что \explain{в конечном итоге}{eventually} влияет на людей.

\textbf{Загрязнение земли и п\'{о}чвы.}
Загрязнение земель -- это разрушение зем\'{е}ль в результате деятельности человека и неправильного использования земельных ресурсов. Это происходит, когда люди наносят на почву химические вещества, такие как пестициды и гербициды, неправильно утилизируют отходы и \explain{безответственно}{irresponsibly} \explain{эксплуатируют}{exploit} полезные ископаемые при \ed{добыче}{добыча}{mining} полезных ископаемых.

Почва также загрязняется из-за протекающих подземных септиков, канализационных систем, вымывания вредных веществ со \ed{свалок}{свалка}{landfill} и прям\'{о}го сбр\'{о}са сточных вод промышленными \ed{предприятиями}{предприятие}{enterprise} в реки и океаны.

Дождь и наводнение могут переносить загрязнители с других уже загрязнённых зем\'{е}ль в п\'{о}чву в других местах.

Избыточное \explain{земледелие}{agriculture} и чрезм\'{е}рный \explain{в\'{ы}пас}{grazing} в результате сельскохозяйственной деятельности прив\'{о}дят к тому, что п\'{о}чва теряет свою питательную ценность и структуру, вызывая деградацию почвы, ещё один тип загрязнения почвы.

Свалки могут вымывать вредные вещества в почву и водные пути и создавать очень неприятные запахи, а также являются рассадниками \ed{грызун\'{о}в}{грыз\'{у}н}{rodent}, которые являются переносчиками болезней.

\textbf{Шум и световое загрязнение.}
Шум считается загрязнителем окружающей среды, вызываемым бытовыми источниками, общественными мероприятиями, коммерческой и промышленной деятельностью и транспортом.

Световое загрязнение вызвано длительным и чрезмерным использованием искусственного \ed{освещения}{освещение}{lighting} в ночное время, что может вызвать проблемы со здоровьем у людей и нарушить естественные циклы, \explain{в том числе}{including} деятельность \explain{дикой}{wild} природы. Источники светового загрязнения включают электронные рекламные щиты, ночн\'{ы}е спортивные площадки, уличные и автомобильные фонар\'{и}, городск\'{и}е парки, общественные места, аэроп\'{о}рты и жил\'{ы}е районы.

\newpage
\section[Парниковый эффект]{Парниковый эффект: что надо знать о влиянии парниковых газов на Землю}

\textit{Источник: \url{https://trends.rbc.ru/trends/green/603766c39a794772017c8a13}}

О парниковом эффекте обычно говорят в связи с изменением климата. Действительно ли парниковый эффект вреден для нас и что нужно о нем знать?

\textbf{Что такое парниковый эффект?}
Парниковый эффект — это естественное явление, которое повышает температуру на нашей планете для комфортного существования.

Как он возникает? На нашу планету поступает солнечная радиация, которая нагревает поверхность. Излучение от солнца коротковолновое, поэтому парниковые газы, которые находятся вокруг Земли, свободно пропускают его. Какую-то незначительную часть солнечного света могут отразить обратно аэрозоли, которые находятся вместе с парниковыми газами в атмосфере Земли.

В свою очередь, когда планета нагревается, она отдает тепловую радиацию — инфракрасное излучение (длинные волны). Но так как излучение длинноволновое, то парниковые газы не дают полностью ему улететь в космос. Частично тепловому излучению все же удается обойти парниковые газы, но значительная доля отражается обратно, что и повышает температуру на Земле.

Первым, кто описал парниковый эффект, стал французский ученый Жан-Батист Жозеф Фурье в 1824 году, его же называют автором термина.


\textbf{Какие на Земле есть основные парниковые газы}

\textbf{Углекислый газ ($\rm{CO}_2$)}
читается важнейшим парниковым газом антропогенного происхождения. Углекислый газ возникает и естественным путем при круговороте углерода, но именно человек увеличил его концентрацию в атмосфере на 47\% с момента индустриальной революции.

\textbf{Метан ($\rm{CH}_4$)} — по своему парниковому эффекту метан считается даже сильнее, чем углекислый газ, но в атмосфере его заметно меньше. Естественные источники — болота и термитники. Антропогенное происхождение — свалки, сельское хозяйство, добыча угля и природного газа.

Закись азота ($\rm{N}_2\rm{O}$) образуется при сжигании твердых отходов и ископаемого топлива. Значительная часть N2O идет от сельского хозяйства.

Синтетические химические вещества, например, гидрофторуглероды, галогенированные углеводороды, гексафторид серы и другие синтетические газы. Основной источник — это химическая промышленность.

\textbf{Озон ($\rm{O}_3$)} — естественным образом встречается в стратосфере и тропосфере Земли и не вызывает значительного парникового эффекта. [2]

\textbf{Водяной пар} — по объему занимает первое место среди всех парниковых газов, однако прямые выбросы водяного пара влияют на парниковый эффект наименьшим образом. [3]

Сам по себе парниковый эффект — благо для нас, так как без него не было бы жизни на Земле. Если представить, что его не существует, средняя температура на Земле составляла бы $-18^\circ\rm{C}$, то есть реки и океаны всегда были бы замерзшими и нигде не росли растения. С его же помощью на нашей планете средняя температура достигает $+15^\circ\rm{C}$. [4]

Самый сильный парниковый эффект в Солнечной системе существует на Венере. Атмосфера планеты практически полностью состоит из углекислого газа, поэтому температура на поверхности Венеры достигает $475^\circ\rm{C}$.

\textbf{Причины парникового эффекта.}
Земля постоянно получает и отдает энергию. По закону сохранения энергии все это должно пребывать в радиационном балансе. Но человек своими действиями вывел систему из баланса. Когда объем парниковых газов увеличивается, они все чаще и чаще не позволяют теплу покинуть атмосферу Земли. Получается, что даже то инфракрасное излучение, которое когда-то улетало в космос, теперь частично остается с нами — глобальная температура повышается.

Ученые пришли к выводу, что средняя температура на Земле выросла на 1,1℃ с конца XIX века. Разница всего в 4℃ ранее приводила к ледниковым эпохам, поэтому эта цифра не такая уж и маленькая. Сложился научный консенсус, что в резком росте парниковых газов в атмосфере виновата хозяйственная деятельность человека.

Что усиливает парниковый эффект:
\begin{enumerate}
    \item выбросы производств;
    \item добыча полезных ископаемых;
    \item угольные электростанции;
    \item автомобильные выхлопы;
    \item экстенсивное сельское хозяйство;
    \item эксплуатация зданий;
    \item лесные пожары;
    \item вырубки лесов.
\end{enumerate}

Наибольший парниковый эффект вызывает сжигание топлива, его добыча и транспортировка, производство сырья (цемент, сталь и другие металлы), пищевая промышленность, захоронение и сжигание отходов. На них приходится примерно 70\% всех глобальных антропогенных выбросов.

Ученые вывели потенциал глобального потепления, который позволяет сравнить климатические эффекты парниковых газов за различные периоды времени. Например, 1 кг метана поглощает тепловое излучение в 84 раза лучше, чем 1 кг CO$_2$, если брать 20-летний период.

У газов разное время жизни, например, у метана оно составляет около 12 лет, у N$_2$O — 114 лет. Часть антропогенных выбросов углекислого газа удаляются из атмосферы в течении нескольких десятилетий, но значительная часть остается в атмосфере вплоть до нескольких тысячелетий.

\textbf{Последствия парникового эффекта.}
Изменение температуры прямо пропорционально радиационному воздействию. Ученые уже подсчитали, что если количество CO2 удвоится, это вызовет потепление от 1,5°C до 4,5$^{\circ}\rm{C}$ — это так называемая чувствительность климата. Уже сейчас концентрация углекислого газа в 1,5 раза выше доиндустриального уровня.

Некоммерческий исследовательский центр Oxford Economics опубликовал исследование о влиянии глобального потепления на экономику. Ученые взяли за основу показатель оптимальной температуры, при которой люди работают максимально производительно, а сельскохозяйственные культуры дают наибольший урожай. Эксперты определили этот показатель в 15$^{\circ}\rm{C}$. Государства, в которых среднегодовая температура ниже этого значения, могут получить небольшие преимущества от потепления. Страны с более жарким климатом, наоборот, понесут ущерб.

В ходе исследования специалисты из Oxford Economics проанализировали данные о положении в 203 развитых и развивающихся странах и спрогнозировали падение мирового ВВП на 20\% к 2100 году. Такой вывод основан на предположении, что средняя температура продолжит расти с такой же скоростью, что и сейчас (примерно на 0,2$^{\circ}\rm{C}$ в десятилетие). Выводы специалистов из Oxford Economics подтверждают результаты более раннего исследования, которое в 2015 году опубликовали ученые из Стэнфордского университета и Калифорнийского университета в Беркли.

По мнению экспертов из Oxford Economics, больше всего пострадает экономика Индии: ВВП на душу населения в стране упадет на 90\% к 2100 году, если выбросы парниковых газов в атмосферу не снизятся. Специалисты также предположили, каким мог бы быть этот показатель в разных странах, если бы средняя температура была на 1,1$^{\circ}\rm{C}$ ниже. Согласно прогнозу, он был бы значительно выше. Например, ВВП на душу населения в Нигерии мог бы быть на 35\% больше, чем сейчас.


\textbf{Пути решения.}
Существует множество путей решения проблемы, которые можно условно разделить на фантастические и реальные.

К фантастическим относится предложение распылить частички серебра в стратосфере, чтобы те отражали как можно больше солнечного света. Так Солнце не будет нагревать нашу планету, а та в свою очередь меньше будет отдавать тепла. По этой же причине некоторые ученые предлагают искусственно вызывать облака, так как они способны отражать солнечный свет, поступающий на Землю.

Что можно реально делать уже сейчас, чтобы парниковый эффект не навредил нам в будущем:
\begin{enumerate}
    \item сократить использование ископаемого топлива и переходить на возобновляемые источники энергии;
    \item повышать энергоэффективность и модернизировать технологий по сбережению энергии;
    \item заниматься устойчивым лесоуправлением и контролировать лесные пожары;
    \item переходить к экологически бережному сельскому хозяйству;
    \item восстанавливать почвенный покров, так как потеря гумуса напрямую влияет на парниковый эффект;
    \item отказаться от личного транспорта и переходить на велосипеды, общественный транспорт и электромобили.
\end{enumerate}



\newpage
\section{Миру грозит новая эпидемия}

\textit{Из-за потепления климата Землю захватывают опасные заболевания}

\textit{Источник: \url{https://lenta.ru/articles/2023/08/07/denge/}}

\textit{\ex{ВОЗ}{Всемирная организация здравоохранения}: лихорадка денге может перерасти в пандемию из-за глобального потепления}

Изменение климата создает все новые угрозы для человечества --- июнь 2023 года стал самым жарким с начала климатических наблюдений. Экстремальная жара в некоторых регионах Китая, США, Мексики и других стран побила исторические рекорды. При этом температурные аномалии не только приводят к увеличению числа стихийных бедствий, таких как ураганы и природные пожары, но и влияют на разрастание и смещение природных очагов опасных инфекций, что может спровоцировать новую пандемию. Так, по оценкам Всемирной организации здравоохранения (ВОЗ), с 2000 по 2022 год количество зарегистрированных случаев лихорадки денге выросло в восемь раз. Заразиться этой опасной болезнью могут миллиарды человек. О том, как экзотические болезни могут стать угрозой для всего мира, --- в материале «Ленты.ру».

\textbf{Новая угроза}

Денге --- вирусная инфекция, передаваемая человеку комарами рода Aedes. Они \ed{кусают}{кусать/покусать}{to bite; to sting} уже заболевшего человека, обезьяну или летучую мышь, после чего переносят вирус дальше. От человека человеку заражение не передается --- в редких случаях это возможно лишь при \ed{переливании крови}{переливание крови}{blood transfusion} или \ed{донорстве органов}{донорство органов}{organ donation}.

Инкубационный период денге длится от трех дней до двух недель. После этого у зараженного начинают проявляться симптомы: высокая температура, сильная головная боль, тошнота и боли в суставах и мышцах. Иногда болезнь сопровождается сыпью, резью в животе, рвотой, \ed{внутренними кровотечениями}{внутреннее кровотечение}{internal bleeding}, диареей и кровотечениями из носа. Большая часть заболевших выздоравливают, но болезнь может привести к серьезным \ed{осложнениям}{осложнения}{complications} и даже к летальному исходу. Риск тяжелого течения лихорадки выше у уже переболевших ей.

Некогда денге можно было заразиться в \ed{глухих}{глухой}{i. deaf person, ii. (for forests, jungles, etc) wild} джунглях и тропических лесах, однако сейчас стать носителем заболевания можно на излюбленных курортах россиян вроде Египта или Таиланда, и даже в европейских странах. В настоящее время риску денге подвергается примерно половина мирового населения, все из-за изменения климата и расширения ареала обитания переносчиков болезни.

Специфического лечения денге не существует: все, что могут сделать врачи --- купировать болевые симптомы у пациента при помощи парацетамола. Прочие препараты, как ибупрофен и аспирин могут вызвать кровотечения. Вакцины от денге тоже не существует, так что для успешной профилактики и контроля за распространением болезни необходимо бороться с переносчиками инфекции.

\begin{fancyquotes}
    Специфического лечения при лихорадке денге не существует. Заболевшим нужно придерживаться постельного режима, больше спать и пить больше жидкости. Заразившимся оказывают симптоматическую поддержку — назначают обезболивающие препараты и парацетамол. Нестероидные противовоспалительные средства вроде ибупрофена и аспирина при лихорадке использовать нельзя из-за высоких рисков кровотечения. Если болезнь не идет на спад и приобретает тяжелую форму, больных госпитализируют. Обычно выздоровление при легкой форме наступает в течение одной-двух недель.
\end{fancyquotes}

В 2023 году количество заболевших лихорадкой денге приблизилось к историческому максимуму, предупредила ВОЗ. «Около половины населения земного шара подвержено риску заражения денге, эта болезнь поразила примерно 129 стран», — заявил руководитель Глобальной программы ВОЗ по борьбе с тропическими болезнями Раман Велаюдхан.

Хотя денге является эндемичным заболеванием стран Южной и Центральной Америки и Карибского бассейна, в этом сезоне выросло количество заражений за пределами типичных районов распространения. При этом число заболеваний превысило средние показатели за последние пять лет, отметили в организации.

По оценкам ВОЗ, в мире ежегодно регистрируется от 100 до 400 миллионов случаев заболевания. Только в американском регионе зарегистрировано около 2,8 миллиона случаев, из них 101,28 тысячи — летальные. Особенно опасна так называемая вторичная инфекция, поражающая уже переболевших людей, добавил эксперт ВОЗ.

\vspace*{1em}
\begin{center}
    {\Huge 100--400}

    {\Large миллионов}

    случаев заражения лихорадкой денге регистрируется в мире каждый год
\end{center}
\vspace*{1em}

По словам Велаюдхана, особую тревогу у эпидемиологов вызывает тот факт, что заражения тропическими болезнями все чаще происходят в странах Европы. Причина проста — насекомые, которые являются переносчиками опасных инфекций, выходят за пределы своих ареалов обитания и постепенно распространяются по земному шару, чему способствует глобальное потепление и возросшая мобильность людей и товаров.

На сегодняшний день лихорадка денге остается самой быстрораспространяющейся тропической болезнью в мире — с начала 2023 года в Северной и Южной Америке зафиксировали уже около трех миллионов пациентов, что вынуждает специалистов говорить о риске перехода заболеваемости в стадию пандемии.



\textbf{Рекордная заболеваемость}

С начала 2023 года в Малайзии зарегистрировали более 63 тысяч случаев лихорадки денге, 45 из которых завершились летальным исходом, заявила министр здравоохранения страны Залиха Мустафа. За год число заражений выросло на 129,2 процента; количество летальных случаев, выросло на 136,8 процента. По словам министра, чтобы остановить размножение комаров Aedes, каждый должен взять на себя ответственность за поддержание чистоты. Для этого она рекомендовала всем правительственным ведомствам, неправительственным организациям, общинам и добровольцам в штате Теренггану сделать четвертую субботу месяца днем уборки жилых районов. Это необходимо для борьбы с распространением комаров, которые размножаются в том числе в сосудах и бочках, хранящихся на открытом воздухе и даже в выброшенных контейнерах или старых шинах, которые могут накапливать дождевую воду.

В Таиланде тем временем число случаев заражения лихорадкой денге за неделю превысило пять тысяч. Общее число пациентов приблизилось к 40 тысячам, темпы распространения заболевания почти достигли показателей 2019 года, когда в стране была объявлена эпидемия, сообщила местная газета Nation. Если число инфицированных продолжит расти, возможно, потребуются строгие меры для быстрой и эффективной борьбы с инфекцией, предупредил глава таиландского Департамента по контролю за заболеваниями Тарес Крассанайравивонг.

Больше всего случаев заражения денге выявлено в Бангкоке, в граничащих с Мьянмой провинциях Мэхонгсон, Так, Чиангмай, а также в курортных Чонбури, Районг, Чантхабури, Краби и Пхукет. Если число случаев инфицирования продолжит расти, власти будут вправе объявить эпидемию в районах распространения лихорадки.

\begin{fancyquotes}
    Лихорадка денге — инфекционная болезнь, широко распространенная в Юго-Восточной Азии (Таиланд, Индонезия, Китай, Малайзия, Япония, Вьетнам, Мьянма, Сингапур, Филиппины), Индии, Африке (Мозамбик, Судан, Египет), в тропическом и субтропическом поясе Северной, Центральной и Южной Америки (Мексика, Гондурас, Коста-Рика, Пуэрто-Рико, Панама, Бразилия и др.)
\end{fancyquotes}

19 июля ВОЗ сообщила о вспышке лихорадки денге в Северной и Южной Америке. Наибольшее число случаев заражений в 2023 году зарегистрировано в Бразилии, Перу и Боливии. По данным организации, в первом полугодии уже зарегистрировали около трех миллионов подтвержденных случаев заражения, что выше, чем показатель в 2,8 миллиона случаев за весь 2022 год.

С начала года Минздрав Аргентины зафиксировал более 40 тысяч зараженных лихорадкой денге. Большая часть из них встретились с вирусом именно на территории страны. Вирус был выявлен в Буэнос-Айресе и Сантьяго-дель-Эстеро. Точное количество летальных случаев выяснить не удалось.

\begin{fancyquotes}
    Около половины населения земного шара подвержено риску заражения денге, болезнь поражает примерно 129 стран

    \begin{flushright}
        Раман Велаюдхан\\
        руководитель Глобальной программы ВОЗ по борьбе с тропическими болезнями
    \end{flushright}
\end{fancyquotes}

17 июля у жителей египетской деревни Эль-Аликат в провинции Кена к югу от Каира выявили симптомы «неизвестного заболевания». Как сообщили в Минздраве Египта, от 200 до 1 тысячи жителей деревни жаловались на высокую температуру, ломоту в костях, рвоту и головокружение. Для изучения жалоб были направлены бригады по оказанию профилактической и медицинской помощи и эпидемиологического надзора.

\textbf{Заражения в России}

В 2022 году в России выявили 29 случаев среди граждан, прилетевших из южных стран — как правило, заражались туристы, побывавшие в Таиланде, Вьетнаме и Индонезии, реже — в Египте. По данным Роспотребнадзора, до 2019 года число заболеваний росло, а после 2020-го пошло на убыль.

Первые случаи заражения лихорадкой денге в 2023 году зафиксировали в феврале в Красноярском крае. По словам главного санитарного врача региона Дмитрия Горяева, заболели две жительницы Красноярска, которые отдыхали на Пхукете и в Паттайе. К середине июня на территории края зафиксировали уже 18 случаев заболевания среди туристов, вернувшихся из Таиланда и Египта, одним из которых стал годовалый ребенок.

«Все обратившиеся к медикам отмечали укусы насекомых, но при этом репеллентами не пользовались, — заявил главный санитарный врач региона Дмитрий Горяев. — Последний случай был выявлен у беременной женщины, вернувшейся из Египта. На третий день после прилета она пожаловалась на повышение температуры, головные боли и сыпь на теле».

Последние случаи заболевания зафиксировали в середине июля. Восьмилетнего россиянина, вернувшегося из египетского Шарм-эш-Шейха, госпитализировали с недомоганием, после чего медики диагностировали у него лихорадку денге. До болезни ребенок вместе с родителями путешествовал по Нилу и посещал Каир, где ранее объявили об опасности распространении болезни. А в соседней провинции Кена за последнее время было зафиксировано 60 случаев заболевания лихорадкой.

После первых сообщений о вспышке неизвестного заболевания в районе Каира Роспотребнадзор усилил меры санитарно-карантинного контроля, в том числе с помощью автоматизированной информационной системы «Периметр» (проводит анализ эпидемиологической ситуации в режиме реального времени с помощью искусственного интеллекта и составляет прогнозы с точностью до 90 процентов). Ведомство опубликовало методички для граждан, которые планируют отдых в Египте, и рекомендовало по возвращении обследоваться у врача в случае ухудшения состояния.

Региональные отделения Роспотребнадзора выпустили рекомендации туристам воздержаться от поездок в регионы, где выявлены случаи заражения. Так, в Липецке призвали туроператоров «в целях обеспечения санитарно-эпидемиологического благополучия» уведомить туристов о неблагополучной обстановке в Египте и предлагать для поездок более безопасные страны. В ведомстве сообщили о проведении лабораторных исследований клинического материала от заболевших, объектов окружающей среды (сточные воды, источники питьевой воды), а также мерах по уничтожению комаров. Местным жителям рекомендовали не оставлять открытыми ёмкости и контейнеры со стоячей водой, чтобы не допустить размножения насекомых и распространения заболевания.

По словам иммунолога Николая Крючкова, распространение лихорадки денге в России маловероятно: для этого необходимо слишком много условий. Кроме того, переносится заболевание только некоторыми москитами. Однако у болезни есть тревожный потенциал: ареал распространения геморрагических лихорадок увеличивается заметными темпами.

«Конечно, речь пока идет только о вспышках, а не об эпидемиях, но тревожный потенциал есть. В любом случае я бы сказал, что некоторые вспышки в России возможны, но не супервероятны», — заявил эксперт.

Инфекционист телемедицинского сервиса «Доктис» Татьяна Когут также считает, что для России лихорадка не так опасна, так как в местных климатических условиях ее переносчики не обитают, а от человека человеку болезнь не передается. Однако после снятия ковидных ограничений риск завозов вируса в страну повысился, особенно с учетом распространения очагов заболеваемости.


\textbf{Жара им на руку}

Ключевое влияние на распространение лихорадки денге оказывает изменение климата. На вспышки заболеваемости влияют осадки, температура, потепление океана и другие погодные факторы, сообщает международная группа ученых. Именно от климатических показателей зависит активность комаров Aedes, которые помимо денге переносят желтую лихорадку, чикунгунья и вирус Зика, отмечают эксперты.

Изменения климата расширяют ареал обитания насекомых, разносящих болезни — так, чувствительные к изменению температуры и влажности москиты и комары получают новую питательную среду и места для размножения, отмечают в ВОЗ. Так, засуха и жара вынуждают насекомых менять места обитания и собираться вокруг оставшихся источников воды, в результате чего тропические болезни стали появляться на территориях, где ее быть не должно. В свою очередь стихийные бедствия способствуют распространению многих болезней, передающихся через продовольствие и воду, в первую очередь из-за разрушения систем водоснабжения и канализации.

Так, ученые сообщили о переселении еще одного вида тропических комаров в США, в штат Флорида. Сейчас в штате обитает 17 неместных видов комаров, при этом 11 были впервые зарегистрированы за последние два десятилетия, а шесть обнаружены только за последние пять лет.

Глобальное потепление и урбанизация усиливают агрессию комаров по отношению к людям, заявили авторы журнала Current Biology. Они установили, что насекомые, живущие в густонаселенных районах, более охотно выбирают своей целью человека, чем их сородичи из деревни или из дикой природы. При этом комары, живущие в местах с продолжительными и суровыми засухами, более падки на человеческую кровь людей, нежели животную. По мнению ученых, стремительная урбанизация в предстоящие десятилетия будет означать еще большее число кусающих людей комаров в будущем.

По словам советника директора ЦНИИ эпидемиологии Роспотребнадзора Виктор Малеев, из-за глобального потепления, которое на территории России происходит в 2,5 раза быстрее, чем в среднем в мире, у граждан повысится риск заразиться нетипичными для страны лихорадками в несколько раз. Малеев уточнил, что переносчиками тропических болезней могут стать также птицы и лошади, которых кусают зараженные насекомые. По его прогнозам, в ближайшие годы вспышки инфекций могут быть в центральных регионах европейской части России и лесостепной зоне юга Сибири. Для профилактики распространения южных лихорадок необходимо следить за комарами и анализировать случаи смерти птиц, добавил эпидемиолог.

Как утверждают ученые Лондонской школы гигиены и тропической медицины. комаров сделало опаснее увеличение выбросов парниковых газов — изменение средних температур продлевает продолжительность сезона передачи болезней человеку с укусами насекомых. По словам экспертов, к 2070 в группе риска заражения денге и малярией войдут дополнительно 4,7 миллиарда человек по сравнению с 1970-99 годами, особенно в низменностях и городских районах. Эпидемический пояс по тропическим болезням продолжит расширяться в сторону районов с умеренным климатом.

Поскольку эффективной вакцины против денге не существует, борьба с популяциями комаров считается наиболее эффективной стратегией предотвращения распространения вируса, отмечают исследователи.

«Климатические службы и здравоохранение должны работать сообща, чтобы смягчить потенциальное увеличение риска заболеваний, переносимых комарами, которое, по прогнозам, произойдет при различных сценариях изменения климата», — заявила доцент кафедры глобального здравоохранения в Школе глобального общественного здравоохранения Нью-Йоркского университета Есим Тодзан.

\textbf{Безопасность превыше всего}

Лихорадка денге имеет две формы — классическую и геморрагическую, при этом вторая — наиболее опасная, так как при ней летальность составляет от 1,5 до 23 процентов, рассказал доктор медицинских наук, профессор кафедры фундаментальных медицинских дисциплин Государственного университета просвещения Дмитрий Березовский. По словам эксперта, заразиться этой болезнью от человека нельзя, так как ее разносят только кровососущие насекомые. Поэтому для того, чтобы обезопасить себя, необходимо пользоваться репеллентами.

По словам инфекциониста Розы Сивяковой, меры предосторожности при посещении эндемичных стран в Африке, Юго-Восточной Азии, Центральной и Южной Америке нужно соблюдать всегда, ведь комары, помимо лихорадки денге, могут переносить массу других болезней, таких как малярию, энцефалит, лихорадку Зика, лихорадку Западного Нила, а также туляремию.

\begin{fancyquotes}
    Была четко названа провинция: это недалеко от Каира, провинция Кена, где была вспышка заболевания. Вот туда точно не надо ехать. И если вы купили путевку, поинтересуйтесь, как там эпидобстановка — это раз. Второе: вы приехали в отель — поинтересуйтесь, есть ли там комары. Пользуйтесь репеллентами, они помогают. Если появилась температурка <…> — обратитесь к врачу

    \begin{flushright}
        Геннадий Онищенко\\
        эпидемиолог, академик РАН
    \end{flushright}
\end{fancyquotes}

«Необходимо носить одежду, которая максимально закрывает кожные покровы, использовать репелленты и инсектициды, которые отпугивают и уничтожают насекомых. Быть внимательными и не подходить близко к болеющим людям, в помещениях желательно закрывать двери и окна. Открывать только те, на которых есть сетки», — посоветовала эксперт.

Академик РАН, доктор медицинских наук, профессор, заместитель президента Российской академии образования Геннадий Онищенко заявил, что туристам, которые собираются в отпуск в регионы, где обнаружено «неустановленное заболевание», симптомы которого схожи с лихорадкой денге, необходимо соблюдать меры предосторожности.

Стало очевидно, что борьба с изменением климата требует нового взгляда и современных методов, которые повернут ситуацию вспять и остановят процесс глобального потепления. В противном случае человечество рискует столкнуться с очередными вызовами и снова оказаться на пороге пандемии, на этот раз — лихорадки денге. В свое время распространение коронавируса унесло массу человеческих жизней, нанесло колоссальный урон промышленности, туризму и бизнесу. Локдауны и массовые ограничения, которые потребовались для борьбы с заболеванием, заставили экономику перестроиться, а восстановиться от этих изменений удалось не всем. Переживет ли мировое сообщество новую пандемию — неизвестно, а значит, человечеству стоит активнее противостоять изменениям климата, которые создают для него очередные угрозы.

\newpage
\section{Мир поразили гигантские кладбища электромобилей в Китае}

\textit{Почему китайцы массово избавляются от них?}

\textit{Источник: \url{https://lenta.ru/articles/2023/09/23/svalka/}}

Сквозь элементы кузова электрокара, оставленного на окраине города на востоке Китая, прорастает куст. Вокруг \ex{гниют}{are rotting (гнить)} десятки других авто, большей частью тоже покрытых густой растительностью. Речь идет об одном из электромобильных кладбищ --- огромных скоплений старых машин, обнаруживаемых в разных точках КНР. Кадры с брошенными электромобилями опубликовали в августе журналисты Bloomberg: вопросом они занялись по следам расходившихся в интернете похожих снимков и дискуссии о природе возникновения таких стоянок. Гипотезы выдвигались разные — от обвинений в адрес автомобильных компаний, якобы выпускавших никому не нужную продукцию ради господдержки, до идей о бесполезности машин с электродвигателями как таковых. Проблема сохраняется уже не первый год и напоминает о другом подобном прецеденте — огромных свалках велосипедов, образовавшихся после краха сервисов велоаренды. Разобраться с проблемой брошенных электромобилей власти уже обещали, но пока не преуспели, а китайские СМИ между тем уверяют, что ничего плохого в такой практике нет, ведь машины уже не очень новые. Кладбищенская история — в материале «Ленты.ру».

\textbf{Поросло быльём.} Прошедшим летом корреспонденты Bloomberg, вооружившись спутниковыми снимками, отправились в город Ханчжоу --- на поиски площадок с заброшенными электромобилями, широко обсуждавшихся в социальных сетях. Кадры с подобными «кладбищами» массово публиковались еще несколько лет назад, а в 2023-м феномен вновь привлек внимание публики, в том числе вне Китая — из-за англоязычных видео, широко разошедшихся в Сети.

Журналистам агентства в самом деле удалось обнаружить, что массовые «захоронения» в Ханчжоу есть, причем сразу в нескольких районах города. В одном из подобных мест скопилось около тысячи машин, в другом — несколько сотен; все авто поставлены вплотную друг к другу или даже поверх других. Местами через них в буквальном смысле проросли кусты.

\begin{wrapfigure}{r}{0.5\textwidth}
    \begin{fancyquotes}
        Эти машины — поразительное воплощение излишеств и ненужных трат, сопровождающих вливание капитала в растущую отрасль. Их, вероятно, можно назвать и странным памятником колоссальному прогрессу в сфере электротранспорта, наблюдаемому в последние пару лет\\

        \begin{flushright}
            Чуньин Чжан, Дэн Мерто, Линда Лю\\
            журналисты Bloomberg
        \end{flushright}
    \end{fancyquotes}
\end{wrapfigure}
Часть брошенных электрокаров была произведена и зарегистрирована сравнительно давно, как минимум около шести лет назад, часть — более новая, что можно заключить по номерам. В конце 2017-го цвет самих пластин сменили с голубого на зеленый, а на площадках корреспонденты нашли авто с пластинами обоих цветов. Некоторые машины притом явно использовалась до недавнего времени: например, на них наклеены предупреждения о мерах безопасности в период пандемии коронавируса и видны наклейки о техосмотре за прошлый год.

Феномен оставленных авто выглядит побочным эффектом от бурного развития рынка электромобилей в Китае, страна сегодня оказалась уверенным лидером направления на глобальном уровне.


В прошлом году на Китай приходилось около половины от общего числа электрокаров в мире, более 13 миллионов машин и 60 процентов всех глобальных продаж новых машин с электродвигателем. КНР даже успела перевыполнить планы, поставленные на 2025 год.

Предполагалось, что к этому сроку электрическим должен быть каждый пятый приобретаемый в стране автомобиль, а в прошлом году на электрокары, включая гибриды, уже приходилось 29 процентов продаж. Объемы электроавтопарка в стране росли колоссальными темпами: к примеру, в 2016 году было продано около 300 тысяч машин, в 2018-2020 годах — примерно по миллиону, в прошлом году показатель достиг 6 миллионов, а прогноз на этот год составляет 8 миллионов.

\vspace*{1em}
\begin{center}
    {\Huge 8}

    {\Large млн электрокаров}

    будет продано в Китае по итогам 2023 года
\end{center}
\vspace*{1em}

Преуспела КНР и в развитии инфраструктуры зарядок, без которых использовать электрокары невозможно: в этом направлении страна тоже достигла абсолютного лидерства.

\textbf{За деньги --- да.} Успехи объясняются серьезной программой государственной поддержки: она начала действовать еще в конце нулевых годов, когда сектор только формировался, и предполагала стимулирование и спроса, и производства. Так, с 2010 года желающим приобрести электрокар стали предлагать субсидии — до 60 тысяч юаней, почти 9 тысяч долларов по тогдашнему курсу. Программу начинали с нескольких городов, а уже через несколько лет распространили на десятки, причем финансировались покупки и центральными, и региональными властями. Сэкономить получалось около половины стоимости.

С января этого года схема действовать прекратила, однако уже в июне, на фоне общего торможения экономики, был представлен новый четырехлетний план стимулирования рынка через налоговые льготы на сумму в 72 миллиарда долларов.

Годами государство помогало и участникам сектора, в том числе обеспечивая их заказами, что особенно было важно на ранних этапах. К 2019 году в стране было зарегистрировано около 500 производителей электрокаров. Но сегодня их осталось около сотни.

Некоторые из серьезных игроков первых лет развития отрасли уже выбыли из гонки. Аналитики объясняют, что часть из них просто стремились производить машины, соответствующие критериям программы субсидирования, — вопросы качества игнорировались.

\begin{fancyquotes}
    Мы зовем их «машинами под госрегулирование». Единственное важное требование было одно: чтобы автомобиль был электрическим\\

    \begin{flushright}
        Йохен Зиберт\\
        сингапурская консалтинговая компания JSC Automotive
    \end{flushright}
\end{fancyquotes}

К тому же сами электрокары, будучи по сути технологической новинкой, еще не так давно были довольно непрактичными: батареи хватало ненадолго. К примеру, в 2015-2017 годах компания Zhidou Electric Vehicles продала около 100 тысяч машин, способных проехать на одной зарядке всего 100 километров, — но когда ее продукция перестала соответствовать требованиям программ субсидирования, столкнулась с проблемами.

Известно и о случаях мошенничества ради получения субсидий. Некоторые предприятия фальсифицировали данные о производстве электрокаров, выпуская, к примеру, машины без батареи. В некоторых случаях у компаний в принципе не было оборудования для производства: подобный случай обнаружили, к примеру, в конце 2016-го, когда одно предприятие заявило, что произвело около 3500 электромобилей за год, и почти все — в один последний месяц перед снижением субсидий. «Жэньминь жибао» в том же году прямо указывала, что проблема мошенничества имеет довольно серьезные масштабы; за скандалами последовала более жесткая регуляция сектора.

\textbf{Черная магия и ее разоблачение.} Между тем появлялись и предположения по поводу того, что некоторые производители прямо фальсифицируют продажи, а на самом деле электрокары вообще не нужны покупателям. Именно такую гипотезу выдвинул автор недавнего материала еще об одном электромобильном кладбище, который привлек внимание к проблеме у аудитории за пределами Китая.

Рассказывающий о жизни в КНР американский YouTube-блогер с аудиторией в 1,4 миллиона человек опубликовал в середине июня видео с названием «Китай выбрасывает целые поля электрокаров — их оставляют гнить». Видео открывали кадры с воздуха со множеством одинаковых авто — речь идет, утверждал блогер, о более чем 10 тысячах электрокаров Neta V от Hozon, буквально гниющих в поле. Далее следуют съемки других брошенных авто, практически новых, марки BYD, уже снятых вблизи.

Причина появления подобных кладбищ, утверждает автор ролика, — постоянное стремление Китая создавать видимость успеха. Все эти брошенные машины и объясняют рекордные успехи Пекина в электрификации транспорта, объявляет блогер. Он тоже апеллирует к проблеме фальсификаций ради субсидий: машины не были проданы, их просто бросили гнить, отчитавшись о раздутых цифрах продаж.

Ролик за три месяца собрал около четырех миллионов просмотров и даже привел к появлению мемов.


\vspace*{1em}
\begin{center}
    {\Large Кадры оставленных авто из видео пользователи подписывали, например, так: «Электрокары полезны для экологии. Поэтому Китай оставляет их повсюду!»}
\end{center}
\vspace*{1em}

Видео немедленно начали опровергать. Некоторые из «опровержений» были не слишком убедительны: утверждалось, в частности, что на кадрах запечатлены «обычные» автомобили с двигателями внутреннего сгорания, поскольку видно, что они оснащены лючком бензобака. Такая аргументация легко разоблачалась прямо в комментариях: за аналогичной дверцей у части электрокаров спрятан зарядный разъем.

В другом ответном видео первые кадры с «10 тысячами новых Neta V» обсуждались гораздо конструктивнее: еще один англоязычный автор из Китая с каналом Inside China Auto («Китайские машины изнутри») прибыл на то же самое место и запечатлел все сам более детально.

По съемкам ясно, что машин в поле действительно много, хотя количество визуально оценить трудно, — но далеко не все из них Neta V и тем более речь не идет о новых авто. На кадрах видно, что брошенные машины — большей частью каршеринговые, с соответствующими пометками. Ими явно пользовались: внутри продемонстрированных на видео авто лежат забытые вещи.

\textbf{Мировая слава.} Именно каршеринговые сервисы, а также агрегаторы такси, работающие по модели Uber, по-видимому, и стали непосредственными виновниками того, что в Китае годами и в самом деле продолжают существовать электромобильные кладбища — и уже не первый год.

Весной 2019-го китайские СМИ обнаружили сотни электрокаров, припаркованных у берега реки в Ханчжоу. Все автомобили, по данным South China Morning Post, принадлежали каршеринговой компании Microcity, называвшей себя лидером отрасли. Компания утверждала, что машины по-прежнему используются, однако безуспешные попытки арендовать их через приложение прямо на месте наглядно показывали обратное. Один из местных жителей сообщил, что земля принадлежит ему, а за разрешение оставить на ней электрокары он получает от Microcity деньги — примерно по 4,5 тысячи долларов каждый месяц.

К тому времени уже обанкротились некоторые китайские стартапы по краткосрочной аренде авто, с одной стороны, возникавшие на волне «шеринговой» экономики, а с другой — поддержанные бурным субсидированием сектора электрокаров.

Концепция каршеринга в целом оказалась не слишком подходящей для КНР: по словам автора канала Inside China Auto, на деле такие поездки одновременно и довольно дороги, и не очень практичны, поскольку другие виды транспорта в больших китайских городах выгоднее по времени.
К тому же проблемы были с качеством электрокаров в шеринговых сервисах: люди жаловались, что в некоторые машины садиться просто небезопасно.

Электромобили каршеринговой компании Microcity были запечатлены и на сделанных с воздуха кадрах, которые опубликовал в 2020 году китайский блогер У Гоюн. Заброшенные машины он обнаружил еще в нескольких местах Ханчжоу и в Нанкине.

То «кладбище», которое обсуждалось в прессе с 2019 года, никуда не делось и к лету 2021-го, добравшись даже до японского телевидения. Корреспонденты местной All-Nippon News Network показали зрителям несколько тысяч авто на том же поле, которое за прошедшие годы успело зарасти травой. В сюжете коротко сообщалось, что местные власти проинформированы о проблеме и намереваются ее в ближайшее время решить.

Еще одно изображение этого же скопления брошенных машин примерно тогда же опубликовал в Instagram (соцсеть запрещена в России; принадлежит компании Meta, которая признана экстремистской и запрещена в РФ) блогер Greg Abandoned, специализирующийся на съемках различных заброшенных зданий и сооружений. В то время он находился в Китае, однако место на фото подписано не было, а назвать его он отказался. По сети кадр разошелся с ошибочной подписью, что речь идет о брошенных электромобилях во Франции.

\textbf{Сам продал, сам купил.} Связи стоянок брошенных авто с каршеринговыми компаниями и сервисами такси объясняются тем, что на начальных этапах развития рынка в Китае именно они и были основными покупателями электрокаров, объяснил старший аналитик JSC Automotive Хуан Юн. «Очень немногие частные потребители решались их приобрести», — уточнил эксперт, комментируя свежие находки Bloomberg.

Некоторые сервисы поездок притом — на фоне субсидирования покупок электрокаров — были учреждены самими автопроизводителями, которые продавали таким образом собственную продукцию. К примеру, один из самых известных китайских автобрендов Geely стоит за компанией Caocao Chuxing, которая работает и сегодня. С другой стороны, закрылась, к примеру, каршеринговая компания Panda, которая принадлежала Lifan, — та в 2020 году обанкротилась и сама, а потом досталась все той же Geely. Машины бренда Maple от совместного предприятия двух упомянутых производителей Bloomberg обнаружило на одном из посещенных летом кладбищ.

К 2020 году собственные сервисы поездок были почти у всех игроков сектора автопрома в КНР. При этом некоторые из автопроизводителей, как предполагали аналитики, в принципе запускали их именно потому, чтобы с легкостью абсорбировать через дочерние организации те электрокары, которые оставались невостребованными обычными покупателями. Часть сервисов оказалась не готова к урезанию субсидий на электромобили в 2019-м: меры серьезно сказались на их денежном потоке. Именно в том году и появились первые заметные сообщения о брошенных авто.

Многие из более старых оставленных электромобилей притом действительно куда менее совершенны, чем их современные аналоги. Журналисты Bloomberg полагают, что именно появление новых, более функциональных моделей могло привести к тому, что старые просто бросали.

Именно вокруг тезиса о ненужности на рынке старых машин строится реакция на материал Bloomberg в китайской прессе. Местная онлайн-газета The Paper объявила, что публикация Bloomberg вводит читателей в заблуждение.

\vspace*{1em}
\begin{center}
    {\Large Вкратце «опровержение» издания сводится к тому, что речь вовсе не идет о свалке, но аргументируется тезис своеобразно: машины на фото уже старые, а автомобилей последних лет на стоянках нет}
\end{center}
\vspace*{1em}

Отмечает портал и то, что власти на сообщения о проблемах уже реагировали. Авторы репортажа Bloomberg, в свою очередь, тоже говорят, что борьбу со свалками власти пытаются вести: что-то стараются переработать, что-то продать, но дело осложняет то, что из-за банкротства некоторых компаний не всегда ясно, кто имеет право распоряжаться автомобилями.

\textbf{Ложка меда.} Вся история с кладбищами электромобилей напоминает еще об одном китайском скандале со свалками последних лет. В 2018 году бурно обсуждались другие впечатляющие кадры: настоящие горы брошенных велосипедов.

Годом ранее страна переживала настоящий бум сервисов велоаренды. Тогда на рынок вышли десятки компаний с десятками миллионов транспортных средств, но предложение серьезно превосходило спрос, а также реалистичные темпы развития нужной инфраструктуры. В итоге люди оставляли велосипеды в случайных местах или вообще бросали.

Если обобщать, что-то подобное и произошло с электромобилями: не самые эффективные модели в огромном количестве оказались не нужны. Но у актуальной ситуации с авто есть важная особенность: в каком-то смысле сервисы поездок, закупавшие электрокары, сыграли образовательную роль. Как резюмировала одна из авторов материала Bloomberg Линда Лю, потребители в итоге постепенно привыкли к такому типу машин, и сегодня в Китае сформирован большой рынок, появилось доверие к технологии.

\vspace*{1em}
\begin{center}
    {\Huge 13}

    {\Large млн электромобилей}

    насчитывалось в Китае к 2022 году
\end{center}
\vspace*{1em}

Компаниям-производителям же продажи электрокаров помогли развиваться и инвестировать, что обеспечило всей отрасли серьезный и заметный прогресс, согласен ее соавтор Дэн Мерто. Другое дело, что в самом факте наличия свалок никаких плюсов быть не может, в особенности с точки зрения защиты окружающей среды, ради которой в мире и продвигается переход на электродвигатели. Производство сырья и компонентов для электромобилей и самих электромобилей, как любая подобная хозяйственная деятельность, сопряжено с вредом, наносимым природе и климату, в том числе из-за выбросов углекислого газа. Если не использовать машину с электродвигателем хотя бы несколько лет, экологических преимуществ по сравнению с обычными автомобилем с ДВС у нее не будет.

К тому же на батареи для электромобилей идут ценные и с трудом добываемые металлы, за которые сегодня активно конкурируют разные страны, стремясь нарастить их запасы. Потому переработка брошенных авто может быть еще и выгодной, что служит дополнительным аргументом в пользу скорейшей утилизации, резюмирует Мерто: «Ради блага электромобильной индустрии Китая эти старые машины должны были умереть. Они просто заслуживают похорон получше».