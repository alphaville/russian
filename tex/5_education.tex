\chapter{Образование}

\section{Образование}
\textbf{Важность образования.}
Невозможно переоценить важность образования в современном мире. Образование стало ведущей силой технологического прогресса и тем самым всего развития человечества.

\textbf{Виды образования в России.}
В нашей стране есть разные виды образования: дошкольное, начальное, среднее и высшее.
Дошкольное образование охватывает ясли и детские сады. Там за детьми присматривают профессиональные няни и воспитатели. Часто их обучают читать и считать.

В России есть разные виды школ. Все школы начинаются с начального образования. Оно продолжается до 5 класса. Большинство учеников ходят в средние школы, другие идут в лицеи, гимназии, специализированные школы. Если ученики успешно оканчивают среднюю школу, они получают аттестат о среднем образовании. Он дает им возможность поступить в университет или академию, которые являются учреждениями высшего образования.

Учреждения высшего образования сейчас готовят специалистов, магистров и докторов. Диплом о высшем образовании позволяет найти более хорошую работу.

\textbf{Виды образования в других странах.}
В разных странах системы образования различны. В Британии три ступени образования. У Британцев есть начальная школа, средняя школа и высшее образование. Последнее включает профессиональное и собственно высшее образование.
В США система образования очень децентрализована. Это означает, что каждый штат имеет свои законы об образовании. Как правило, там есть начальные школы (6-11 лет), средние школы (11-15 лет) и старшие школы (9-12 классы). Есть несколько способов продолжить образование: университеты, колледжи, местные колледжи, технические и профессиональные школы.