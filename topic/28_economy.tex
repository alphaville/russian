\chapter{Экономика}

\section{Восемь богачей мира владеют половиной богатств Земли}
Восемь \ed{богач\'{е}й}{бог\'{а}ч}{rich person} мира владеют половиной \ed{бог\'{а}тств}{бог\'{а}тство}{wealth} Земли. Всего восемь \explain{толстосумов}{толстосум: moneybag} \explain{владеют}{владеть/завладеть: to possess (влад\'{е}ю, влад\'{е}ешь, влад\'{е}ют)} тем же состоянием, что принадлежит беднейшей половине населения Земли. К такому выводу пришли исследователи благотворительной организацией Oxfam. Более того, исследование показало, что уже в ближайшие 25 лет в мире может появиться первый триллионер. Чтобы потратить такое состояние, н\'{у}жно ежедневно в течение 27 столетий и еще 38 лет расходовать по одному миллиону.

Восемь миллиардеров владеют тем же состоянием, которое находится в руках 3 миллиардов 600 миллионов человек, отмечает РИА Новости.

Великолепную восьмерку возглавляет основатель Microsoft Билл Гейтс. Его состояние оценивается в 75 миллиардов долларов. У Амансио Ортеги 67 миллиардов. Экс-президента Inditex женщины знают по марке магазинов одежды Zara. Третью ступень пьедестала богатства занимает американский \explain{предприниматель}{business-person; entrepreneur} Уоррен Баффетт с состоянием почти 61 миллиард долларов.

Оставшаяся пятёрка толстосумов выглядит так: мексиканский бизнесмен Карлос Слим Элу --- 50 миллиардов долларов, глава компании по электронной торговле Amazon Джефф Безос --- чуть более 45 миллиардов долларов, основатель соцсети Facebook Марк Цукерберг --- чуть менее 45 миллиардов долларов, глава корпорации Oracle Ларри Эллисон с состоянием 43 миллиарда и 600 миллионов долларов и владелец агентства деловой информации Майкл Блумберг с 40 миллиардами.
В 2010 году считалось, что стоимость имущества, которым владеют половина беднейших жителей Земли, равна совокупному состоянию 43 богатейших людей планеты. Так что за последние шесть лет ситуация лишь усугубилась.

``Это возмутительно, что такие средства сосредоточены в руках всего нескольких человек, когда каждый десятый в мире вынужден выживать на менее чем два доллара в день'', --- отметил исполнительный директор Oxfam Уинни Бьянима.

По его мнению, неравенство способствует тому, что сотни миллионов человек находятся в ловушке бедности.

``Это разрушает наши общества и подрывает демократию'', --- настаивает Бьянима.

Сегодня семь из десяти человек живут в странах, где за последние три десятилетия неравенство в доходах возросло. При этом всего один процент населения планеты владеет таким же богатством, как 99 процентов тех, кто не попал в золотой процент.

Кроме того, чтобы труд женщин в мире оплачивался так же, как и мужской, понадобится еще порядка 170 лет, отмечают в Oxfam.

Oxfam (Oxford Committee for Famine Relief) основана в Оксфорде в 1942 году для помощи голодающим. Ее цель --- решение проблем бедности и связанной с ней несправедливости во всем мире. Международное объединение сейчас объединяет 17 организаций в более чем 90 странах мира.

\newpage
\section{НДФЛ без границ}
\textit{Надолго уехавшим из России выпишут 13-процентный налог}

{\it Источник: \url{https://www.kommersant.ru/doc/5988499}}

Минфин доработал поправки о новых правилах уплаты НДФЛ гражданами, более полугода работающими из-за рубежа на российские компании. Согласно обновленной версии, доходы, полученные как по трудовым договорам, так и по договорам гражданско-правового характера, с 2024 года будут облагаться по стандартной ставке 13\%. Таким образом, одинаковый налог будет взиматься как со штатных работников, так и с фрилансеров, как с резидентов, так и с нерезидентов. Для последних это может означать возникновение двойного налогообложения — в РФ и в новой стране пребывания. Попытаться решить проблему можно будет с помощью соглашений об избежании двойного налогообложения, перспективы практического применения которых в нынешних условиях не ясны.

Проект точечных изменений в Налоговом кодексе, предусматривающий в том числе новые правила налогообложения граждан, работающих на российские компании из-за рубежа, доработан для повторного внесения в Госдуму, сообщил в четверг Минфин. Напомним, законопроект был внесен правительством в нижнюю палату 24 апреля, но уже на следующий же день был отозван для «технических уточнений».

Изменения в части НДФЛ для уехавших, однако, оказались весьма существенными. В апрельской версии предлагалось отнести к доходам от источника в России оплату труда, полученную от российского заказчика фрилансерами, работающими по договорам гражданско-правового характера. Пока такие доходы НДФЛ в случае утраты работником резидентства, то есть после полугода проживания за рубежом, вообще не облагаются — и по прежней редакции такие физлица могли подпасть сразу под 30\% налог.

\begin{fancyquotes}
    В новой версии налог для нерезидентов также возникает — но по стандартным ставкам 13\% или 15\% (с доходов, превышающих 5 млн руб. в год).
\end{fancyquotes}

При этом к доходам от источника в России законопроект теперь относит и вознаграждения, полученные при выполнении дистанционным работником трудовой функции по договору с работодателем, являющимся российской организацией или обособленным подразделением иностранной организации, зарегистрированным в РФ,— проще говоря, налог будет взиматься как со штатных работников, так и с фрилансеров, как с резидентов, так и с нерезидентов.

Замминистра финансов Алексей Сазанов такую унификацию объясняет тем, что сейчас российским организациям как налоговым агентам приходится определять ставку налога с дохода работника в зависимости от конкретной ситуации — и «когда сотрудник работает в удаленном форме, конечно, компаниям сложно проверять, является он российским налоговым резидентом или нет». По его словам, новшество должно существенно упростить администрирование для налоговых агентов.

\begin{fancyquotes}
    Компаниям изменения, может, жизнь и облегчат, но самим нерезидентам грозят двойным налогообложением — ведь налоговые органы иностранного государства, как правило, также будут претендовать на обложение доходов своего нового резидента.
\end{fancyquotes}

Как отмечает партнер Kept Донат Подниек, для тех, кто работает в стране, с которой у РФ действует соглашение об избежании двойного налогообложения (это около 80 государств), и платит там налоги с полученной от российского работодателя зарплаты, есть возможность зачесть в РФ иностранный налог. «Правда, в течение какого-то времени, пока не вернут НДФЛ, удержанный в РФ, работник фактически уплатит налоги и в РФ, и за рубежом»,— говорит эксперт.

Как уточнил Алексей Сазанов, в случае приостановки соглашений (такая возможность сейчас обсуждается в отношении недружественных стран — см. “Ъ” от 15 марта), нормы, позволяющие зачесть уплаченный налог, продолжат действовать. Вопрос, отметим, в том, будет ли в нынешних условиях это возможным на практике при нынешнем уровне информационного обмена и сотрудничества с недружественными юрисдикциями. В самой же невыгодной ситуации окажутся те, кто уехал в страны, с которым таких соглашений нет или они денонсированы (например, с Нидерландами),— тогда доход, предупреждает руководитель практики «Структурный и налоговый консалтинг» юркомпании «Лемчик, Крупский и партнеры» Людмила Круглова, будет облагаться налогом и в России, и в стране пребывания без возможности зачета.

\begin{fancyquotes}
    Донат Подниек отмечает, что для тех, у кого должным образом оформлена работа за рубежом и российский работодатель НДФЛ не удерживает, изменения ухудшат положение работников.
\end{fancyquotes}

Но есть случаи, когда допсоглашений о дистанционной работе за рубежом нет и работодатель «для подстраховки» удерживает НДФЛ по ставке 30\% — в таком случае нагрузка снизится. Людмила Круглова отмечает, что в силу отсутствия у налоговых органов возможности автоматически проверять статус резидентства граждан, видимо, власти пришли к выводу, «что ужесточение правил не позволит собрать с уехавших граждан 30\% НДФЛ: добровольно в массовом порядке о потере статуса резидента граждане уведомлять не будут, а проверять этот статус у миллионов работников в ручном режиме невозможно».

По словам ассоциированного партнера МЭФ Legal Анны Зеленской, изменения логичны в связи с цифровой трансформацией рынка труда и распространенностью дистанционной работы. «Тезис о том, что работодатели должны следить за физическим местоположением сотрудника, наверное, действительно несколько устарел»,— говорит она.