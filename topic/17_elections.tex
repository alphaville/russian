% \chapter{Выборы}

\section{Белорусы голосуют за новую конституцию}

\textit{Что изменится в жизни страны и ее граждан?}

\textit{Источник: \url{https://lenta.ru/articles/2022/02/27/bel_ref/}}

27 февраля в Белоруссии пройдет референдум о внесении изменений в конституцию. Согласно новой редакции основного закона, страну ждет перераспределение властных полномочий в пользу Всебелорусского народного собрания (ВНС), закрепление на конституционном уровне ценностей патриотизма и ослабление роли президента. Власти республики во главе с Александром Лукашенко называют событие историческим и говорят о больших переменах в стране в случае принятия поправок. Оппозиция относится к этим заявлениям скептически и обвиняет руководство Белоруссии в имитации политического процесса. «Лента.ру» побеседовала с белорусскими и российскими политологами о перспективах конституционной реформы и узнала, зачем руководство Белоруссии меняет основной закон, как эти перемены воспринимаются в обществе и что они означают для России.

\textbf{Нужен ли референдум белорусам?}

\textit{\textbf{Алексей Дзермант}, политолог, директор Центра изучения и развития континентальной интеграции «Северная Евразия»}

Я участвовал во многих площадках [обсуждения поправок в конституцию]. На некоторых присутствовало по тысяче человек. Скажу, что это абсолютно не протокольные мероприятия, а живые беседы — люди интересуются разными пунктами [поправок в конституцию]. Я бы сказал, что общество разогрели. Оно готово к обсуждению и активному участию. Наибольший интерес вызывает раздел новой конституции, в котором говорится о ВНС. Людей интересует новая конфигурация политической системы. Они спрашивают, не будет ли двоевластия, останется ли Белоруссия президентской республикой. На все вопросы ответа нет, должен еще быть принят отдельный закон о ВНС.

\textit{\textbf{Франтишек Вечерко}, старший советник бывшего кандидата в президенты Белоруссии Светланы Тихановской}

Интерес к референдуму среди белорусов небольшой, потому что он не решает первопричины кризиса — то, что Лукашенко отказался уходить. К референдуму многие относятся как к фарсу.

\begin{fancyquotes}
    Референдум в Белоруссии — это ритуал советского типа, где решение понятно заранее и происходит симуляция народной любви
\end{fancyquotes}

Проект конституции, который они предложили, очень плох. Там перемешаны и Великая Отечественная война, и подвиг народа, и патриотизм, и личное, и гражданское. Все это делает проект неуместным, и белорусы это видят.

\textit{\textbf{Петр Петровский}, политолог, эксперт общественного объединения «Белая Русь»}

Действующая редакция конституции имеет очень большой налет либеральных утопических взглядов. Она принималась в те годы, когда гуманитарная отрасль Белоруссии находилась под влиянием западных нарративов.

А у нас ведь ментальность не протестантская, не такая автономная, как на Западе. Многие нормы, вопросы идентичности, этики и морали у нас принято регулировать на общественных началах. И те нормы, которые на Западе относятся к частной жизни, у нас являются частью жизни общественной. Изменилось и время. На Западе в 1996 году никто не говорил про гей-браки, никто не посягал на традиционные семейные ценности.

\textit{\textbf{Дмитрий Болкунец}, белорусский политолог}

В обсуждении конституции должны принимать участие разные силы — не как сейчас, когда несколько тысяч граждан находятся в тюрьме по политическим соображениям, огромное количество людей находится в эмиграции. В таких условиях это не имеет никакого смысла.

В новой редакции конституции достаточно много популистских лозунгов. Но что бы ни было прописано в основном законе, вопрос в том, как это будет реализовано.

\begin{fancyquotes}
    На самом деле граждане Белоруссии готовы согласиться с любой конституцией — главное, чтобы Лукашенко ушел. Это первостепенная задача
\end{fancyquotes}

\textit{\textbf{Богдан Безпалько}, российский политолог, член Совета по межнациональным отношениям при президенте России}

Настроения в белорусском обществе можно охарактеризовать как апатичные. Лукашенко если и не любят, то пассивно. Протест против него считают бессмысленным — просто потому, что его задавят силой. В республике нет никакой оппозиции. Она либо выехала за границу, либо сидит очень тихо.

Часть населения республики сейчас просто пытается выжить. В Белоруссии сильно выросли цены, инфляция двузначная. Из-за санкций экономика там действительно пошатнулась. Из Минска даже многие врачи и преподаватели уехали. Кто-то уезжает на работу в Россию, кто-то в Литву, в Польшу. Людям не до конституционного референдума и протестов.

\textbf{Что референдум дает Лукашенко и белорусским элитам?}

\textit{Алексей Дзермант:} Президент хочет удержать страну от цветной революции или государственного переворота. Действующая конституция написана под конкретного человека — Александра Лукашенко. Нужно сделать ее более универсальной, чтобы она работала и с другими президентами. Прописать органы и механизмы, которые не позволили бы с государством что-то сделать. Вот это ключевой вопрос удержания страны.

Также изменения вызваны естественным развитием белорусской политической системы. Люди требуют, чтобы какие-то полномочия передавались на иные этажи власти. Общество созрело для этого. Лукашенко понимает, что часть его сверхполномочий можно отдать другим органам. Система должна демократизироваться.

\textit{Франтишек Вечерко:} Первоначально идея изменения конституции была в том, чтобы Лукашенко мог обеспечить себе безопасное место, если вдруг ему придется покинуть должность президента. Чтобы он смог, как бывший президент Казахстана Нурсултан Назарбаев, получить пост «главного уважаемого человека». Но теперь Лукашенко поменял мнение. Он озабочен тем, как бы ему день простоять и ночь продержаться. Сделать фикцию, чтобы ничего не менять.

\begin{fancyquotes}
    Очень маловероятно, что Лукашенко собирается покинуть свой пост. Транзит не начнется с этим референдумом
\end{fancyquotes}

Но новая конституция не решит кризис, она создаст для Лукашенко больше рычагов для влияния на ситуацию. То, что он предлагает в качестве конституционной реформы, может формально менять систему власти, однако по факту все будет решаться так же, как решалось раньше. Он получит должность председателя ВНС — типа генсека ЦК КПСС — и станет совмещать ее с должностью президента. Будет сам контролировать орган, который должен контролировать его.



\textit{Петр Петровский:} Главной задачей конституционной реформы является трансформация «персонального Лукашенко» в «Лукашенко коллективного». Благодаря наделению ВНС полномочиями контролировать решения президента, выбирать ЦИК и судей Верховного и Конституционного судов мы можем жестче придерживаться пути разделения властей.
Всебелорусское народное собрание в Минске, 2021 год

С 1996 года, [когда была принята действующая редакция конституции], многое изменилось. Тогда в стране было чрезвычайное положение: после распада СССР разваливались экономика и государственные институты. Теперь нам надо выстраивать новую систему, в которой будет реальное разделение властей, независимые суды и коллективное руководство.

Референдум однозначно запускает процесс транзита власти в Белоруссии. Но вопрос в том, когда этот транзит произойдет.


\begin{fancyquotes}
    Если во время протестов 2020 года Лукашенко собирался покинуть пост после принятия новой конституции, то теперь, после стабилизации ситуации, его мнение поменялось
\end{fancyquotes}

Он задал себе вопрос: нужно ли ему идти на новые выборы в 2025 году? Мое мнение — да, нужно. За пять лет нового срока ему необходимо подготовить персональную преемственность. Данный вопрос не решен до сих пор.


\textit{Дмитрий Болкунец:} Я думаю, референдум проводится для политического транзита. Но надо понимать, что Лукашенко — человек хитрый, неустойчивый и амбициозный. Он до последнего будет биться за политическое влияние.

На мой взгляд, Лукашенко уйдет [с президентского поста] и в Белоруссии состоятся очередные президентские выборы не позже, чем в России состоятся выборы президента — в марте 2024 года. Как и в какой форме это будет происходить, сказать сложно.

\textit{\textbf{Станислав Бышок}, директор международной мониторинговой организации CIS-EMO}

Некоторые говорят, что новая конституция демократизирует Белоруссию. Но ведь и в старой не написано, что оппозиционная активность запрещена, а лидером страны может быть только человек по имени Александр Лукашенко. Это все лишь попытка создать вид некой демократизации.

Такие страны, как Белоруссия, пытаются изобрести гибридный велосипед: пытаются его улучшить, но никакого содержательного аспекта в этом нет, помимо сохранения действующей власти под другим названием.

Можно сказать, что конституционный референдум запускает транзит власти в Белоруссии, чтобы его не запускать. На пике протестов Лукашенко заявлял, что, конечно же, он скоро уйдет, но сейчас говорит об этом уже с меньшей охотой.

\textit{\textbf{Евгений Минченко}, президент коммуникационного холдинга «Минченко консалтинг» }

Лукашенко проводит конституционную реформу под себя. Он не заинтересован ни в каких демократических реформах. ВНС будет органом-симулякром, который будет формироваться не путем прямых выборов, а путем фактического назначения его участников.

У Лукашенко и его команды есть вполне обоснованные сомнения в электоральной поддержке, поэтому они создают структуры, которые зависят не от воли избирателей, а от начальства. Примеры подобного рода транзитов власти мы видели у [президента Турции Реджепа Тайипа] Эрдогана и ряда других правителей — когда меняется формальная составляющая без изменений фактической системы принятия политических решений. Я думаю, что в Белоруссии будет то же самое.

\textit{Богдан Безпалько:} Задача Всебелорусского народного собрания — это концентрация власти в руках президента Лукашенко и придание ему суперстатуса, которым не обладал даже российский император Николай II в конституционный период — то есть практически статуса монарха. Он будет обладать неподсудностью и иметь сверхполномочия как президент и председатель ВНС.

В белорусской конституционной реформе интересно и то, что она обязывает кандидата в президенты обладать десятилетним стажем работы в государственных органах. И при этом на него не должно быть никакого компромата, даже анонимного. Не говоря уже о том, что кандидат должен родиться на территории независимой Белоруссии, то есть после 1991 года. Это фактическое аннулирование активного избирательного права.

\textit{\textbf{Андрей Суздальцев}, российский политолог}

[Создаваемое после принятия новой конституции] ВНС не будет являться избираемым собранием. Туда будет назначаться высшая номенклатура из разных сфер. То есть это будут представители той группировки, которая уже находится у власти. Никакого отношения к народовластию ВНС не имеет. Это остаток авторитарного режима.

ВНС ни за что не будет отвечать, но получит право влезать во все дела. Перед ним будет отчитываться премьер-министр, представители собрания смогут отменять итоги выборов, а Лукашенко, став главой президиума ВНС, сможет лишать полномочий действующего президента.

\begin{fancyquotes}
    Неограниченные полномочия сделают из этого органа почти что коллективного монарха, который при этом будет несменяемым и совершенно неподотчетным
\end{fancyquotes}

Лукашенко решил создать усиленный вариант того, что сделал [первый президент Казахстана Нурсултан] Назарбаев, когда уходил со своего поста и пожизненно подчинял себе Совет безопасности республики. Однако Назарбаева это не спасло.

Лукашенко не понимает одного: в авторитарных режимах, если ты хоть на шаг сдвинулся со своей позиции, ты уже не вернешь себе былую власть. Это и обнаружил Назарбаев, когда оказалось, что подконтрольный ему [президент Казахстана Касым-Жомарт] Токаев все равно оказался сильнее.

\textbf{Есть ли в белорусском референдуме выгода для России?}

\textit{Алексей Дзермант:} Более стабильная страна, где действует универсальный закон, заточенный не только под одного человека, для России выгодна. Чтобы было стабильное, предсказуемое государство, с руководством, которое крепко держит бразды правления.

Если по результатам референдума люди примут поправки, то они укрепят власть Лукашенко. А это значит — поддержку Кремля. Москва заинтересована прежде всего в том, чтобы в Белоруссии была сильная власть, союзная России. Лукашенко это воплощает, и конституция все это подтверждает и развивает.

\textit{Дмитрий Болкунец:} Я считаю, что Лукашенко является токсичной фигурой для Москвы. Кроме того, он абсолютно бесполезен в плане стратегического сотрудничества. Он ничего делать не будет.

В России это понимают. Косвенный пример: российский Минфин отказывается, судя по заявлениям [главы министерства финансов Антона] Силуанова, выдавать Белоруссии кредиты, которые та запрашивает. Будет выдавать только средства для реструктуризации кредитов прошлых лет. Россия не вкладывает новые деньги в Белоруссию при Лукашенко.

\textit{Петр Петровский:} Единственный внешнеполитический вопрос в новой редакции конституции связан с избавлением от рудимента о стремлении Белоруссии к нейтралитету. Эта норма сохранялась в законе с 1994 года. Но ее логично было бы убрать еще в 1995 году, когда на референдуме было принято решение о строительстве Союзного государства с Россией.

\textit{Станислав Бышок:} Многие подались заблуждению, что принятие в Белоруссии новой конституции выгодно российскому руководству, которое якобы заинтересовано в демократизации республики. Это неправда. Россия в этом не заинтересована.

\begin{fancyquotes}
    В Кремле нет людей, которые продвигали бы альтернативную Лукашенко фигуру. Но я не могу назвать ни одного человека в российской власти, который бы симпатизировал ему
\end{fancyquotes}

То есть они по миллиону причин не любят Лукашенко, но вместе с тем опасаются, что на его место может прийти коллективная Светлана Тихановская, а на границе с Россией появятся базы НАТО. В Кремле свыклись с фигурой действующего белорусского президента.

Принятие новой конституции никак не повлияет на скорость продвижения интеграции в рамках Союзного государства. Это не связанные между собой параллельные процессы.

\textit{Богдан Безпалько:} Новая белорусская конституция невыгодна российским властям. Реформа основного закона только усиливает этот дикий автократический режим, который напоминает наполеоновский. При этом для Москвы он является неподконтрольным. В Кремле не могут быть уверены в том, что будет дальше. Автократ в любой момент может поступить так, как России абсолютно невыгодно. Но другого выхода сейчас нет.

Объективно Лукашенко сейчас толкает в сторону России его дикая токсичность для Запада и то, что он там объявлен нерукопожатным достаточно давно. Но в случае каких-либо изменений Лукашенко точно так же может побежать от России. И мы знаем, что на Западе достаточно циничные люди, чтобы не обращать внимание на то, что он «последний диктатор Европы».

\textbf{Как референдум повлияет на отношения Белоруссии с Западом?}

\textit{Алексей Дзермант:} Вряд ли после референдума будет хуже. Насколько я знаю, уже были заявления о непризнании любых итогов голосования. Новая конституция все равно сделает позицию власти легитимной. Игнорировать итоги голосования не сможет никто, даже Европа. Если народ проголосует за конституцию — это значит, что, хочешь или не хочешь, придется иметь дело с этой властью. Новая конституция поставит определенные точки в кризисе 2020 года.

Я не думаю, что референдум вызовет очередную волну санкций. У оппозиции нет уже той инфраструктуры для информационного давления, чтобы сообщения о якобы фальсификациях сделать массовыми. Эта инфраструктура разгромлена. Конечно, они будут пытаться. Но в целом, я думаю, результаты референдума де-факто будут признаны+ и белорусская власть будет считаться субъектом, с которым нужно вести хоть какие-то переговоры.

\textit{Франтишек Вечерко:} Уже сейчас многие депутаты из Европарламента заявили, что не признают референдум. В ОБСЕ сказали, что даже не будут посылать официальных наблюдателей. Конечно, санкции сейчас рассматриваются.

\begin{fancyquotes}
    ЕС будет вводить санкции, пока Лукашенко будет творить глупости
\end{fancyquotes}

При этом внешняя политика Минска тоже не поменяется после референдума. Думаю, что переход к ВНС ответственности за внешнеполитический курс никак на нее не повлияет. Пока Лукашенко остается наверху, все переговоры будут вестись только с ним.

\textit{Петр Петровский:} С точки зрения Запада проведение референдума тихо и гладко будет легитимировать Александра Лукашенко. Мы знаем, что уже сейчас началась активность западных партнеров по налаживанию дипломатических каналов. Были звонки представителей Госдепа США [министру иностранных дел Белоруссии Владимиру] Макею, контакты руководителей Генштабов США и Белоруссии — это то, что уже было. Я считаю, что Запад после референдума будет дальше инициировать оживление контактов.

Их задача — снова восстановить иностранных агентов в Белоруссии, чтобы не дать развиваться евразийским интеграционным процессам в стране. Они не идиоты, они понимают, что время [оппозиционного политика Светланы] Тихановской проходит. Теперь нужны новые инструменты влияния и лоббирования изменений и реформ в республике.

\textit{Дмитрий Болкунец:} Белорусские власти будут пытаться торговаться с Западом, если референдум пройдет благополучно и спокойно. Будут пытаться продавать Лукашенко в Европе как надежного политика, с которым все в Белоруссии смирились. Я думаю, это не получится, потому что на Западе есть консенсус, что с Лукашенко работать не имеет никого смысла.

\textit{Станислав Бышок:} Принятие новой конституции никак не повлияет на отношения Белоруссии с европейскими странами. На Западе референдум никто не признает. Он никого не впечатлит, за этим не последует нового витка санкций. Это скорее внутренняя история.

\textit{Богдан Безпалько:} Конечно, на Западе не будут признавать этот референдум, однако по большому счету ему большого значения не придают. Особенно после президентских выборов 2020 года. Все понимают, что Лукашенко, подавив протесты, полностью контролирует свою страну и будет делать все, что захочет.

Референдум добавит лишний штрих в картину, но принципиально ничего не изменит. Могут быть какие-то санкции, но вряд ли они будут порождены именно конституционной реформой. И пока Лукашенко может опираться на Россию в экономическом и военном отношении, подобные меры не будут иметь решающего эффекта.

\textbf{Возможны ли новые протесты в Белоруссии по итогам референдума?}

\textit{Петр Петровский:} Белорусская оппозиция уже пытается запугать членов ЦИК, дестабилизировать работу государственных органов через попытки минирования, взлома электронных систем. Они занимаются теми процессами, на которые могут повлиять.

Внутри страны вряд ли имеется какой-то протестный потенциал. Во-первых, общество устало от излишней политизации. Во-вторых, государство задерживало и привлекало к ответственности всех участников незаконных протестов, государство ужесточило наказания. Эти факторы по сарафанному радио разносятся среди населения и его протестная активность резко падает. Я сегодня не вижу готовых даже попытаться подать заявку на массовое мероприятие.

\textit{Станислав Бышок:} В Белоруссии отсутствуют предпосылки для массовых протестов после референдума, как это было в 2020 году. Для революции нужен революционный класс, какие-то активные лидеры и раскол внутри политических элит. Видных представителей оппозиции в республике либо посадили в тюрьму, либо выдавили из страны. Элитный раскол так и не случился.

\begin{fancyquotes}
    В 2020 году никто не протестовал против конституции, никто не требовал проведения референдума по ней. Это был протест против Лукашенко. Теперь народу предлагается новая конституция, которая сохраняет у власти этого человека. Это издевательство
\end{fancyquotes}

\textit{Евгений Минченко:} В Белоруссии на сегодняшний день разгромлена протестная инфраструктура. У оппозиции нет организационных или информационных возможностей, чтобы устроить масштабную агитацию против новой Конституции. Невелика и вероятность массовых протестов. Они уже были, и их достаточно жестко подавили.

Ранее российские власти говорили о том, что власти Белоруссии должны наладить диалог с оппозицией и оппозиционными слоями населения. Однако мы видим, что сегодня этого не происходит, а диалоговые форматы носят имитационный характер.

\textit{Дмитрий Болкунец:} Я не ожидаю, что после протестов состоятся масштабные выступления. Во-первых, референдум мало что решает, во-вторых, выступления явно будут гаситься властями. Часть граждан пойдут на участки, кто-то поддержит призывы оппозиционных политиков портить бюллетени, могут быть отдельные выступления.

Я бы обратил внимание на акции в городах Европы и США. Я убежден, что митинги против референдума будут у посольств и консульств Белоруссии. Это можно считать продвижением оппозиционной повестки.

\textit{Андрей Суздальцев:} Не стоит ждать каких-либо протестов после референдума. Белорусская оппозиция обескровлена. Ее лидеры либо находятся в тюрьмах, либо выехали за рубеж. Конечно, они мечтают, чтобы референдум стал толчком к народным выступлениям. Но подобные мероприятия надо готовить, вкладывать в них деньги. Ничего этого сделано не было.

Рассчитывать на народную инициативу здесь тоже сложно, потому что люди очень разочарованы августом-октябрем 2020 года. Люди поняли, что ходить с шариками и сердечками против авторитарного режима — это все равно что бросаться на амбразуру.

