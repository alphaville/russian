\chapter{Кино}

\section{Кино}
\textbf{Возникновение\footnote{возникновение: emergence}.}
В конце XIX века движение предмета наконец-то \explain{удал\'{о}сь}{удаваться/удаться: to manage} \explain{перенести}{переносить/перенести: to transfer} на экран. Вскоре после этого, кино начало набирать популярность. Первые фильмы были очень короткие, продолжительностью около 1-ого минуты. Они были черно-белыми и без звука. \explain{Спуст\'{я}}{later; after; e.g., спустя три дня} несколько лет продолжительность фильма уже составляла 15-20 минут.

\textbf{Виды фильмов.}
Существует несколько видов фильмов, такие как короткометражное кино, документальные и художественные фильмы.
Короткометражное кино является \explain{отд\'{е}льным}{отд\'{е}льный: separate} жанром. Нужно быть профессионалом, чтобы передать целый ряд чувств за короткий промежуток времени. Продолжительность таких фильмов обычно не \explain{превышает}{превышать: to exceed} 15-20 минут.

В основе документальных фильмов лежат реальные истории и факты. Обычно, это фильмы об исторических \explain{событиях}{событие: event}, знаменитых людях и т.д.
Образовательные фильмы также относятся к этой категории.

Художественные фильмы --- это фильмы, в которых актёры играют определённую роль. Художественные фильмы бывают разного жанра: мелодрамы, комедии, триллеры и другие.

\textbf{Российское кино.}
Стремительное \explain{развитие}{development} российского кино началось в XXI веке. Многие фильмы \explain{напр\'{а}влены}{short form of направленный (adj.): directed} на массового зрителя и, \explain{в большинстве своем}{largely}, развлекательные. Кроме того, \explain{выпускается}{выпускаться/выпуститься: to be released} большое количество фильмов высокого качества. Российский кинематограф известен своими талантливыми режиссёрами, такими как Никита Михалков, Федор Бондарчук, Тимур Бекмамбетов и некоторыми другими.

\textbf{Голливуд.}
Голливуд является самым популярным местом по \explain{производству}{производство: production} фильмов в мире. Ежегодно там \explain{создаются}{создаваться/создаться: to be created} тысячи фильмов. Голливудские фильмы полны спецэффектов, которые привлекают миллионы людей в кинотеатры.

\section{Человечество решает умереть}
\textit{Ярослав Забалуев}\\
\url{https://lenta.ru/articles/2021/12/25/dontlookup/}

{\it Вышла комедия про дураков и апокалипсис с Ди Каприо, Стрип и Бланшетт. Зачем ее смотреть?}

\textit{На Netflix вышла новая комедия известного «Игрой на понижение» Адама Маккея «Не смотрите наверх» --- хвастающая, возможно, самым звездным за последнее время актерским составом. «Лента.ру» рассказывает, почему фильм о конце света с Леонардо Ди Каприо в главной роли --- идеальное кино для конца этого странного года.}

Астрономы Рэндалл Минди (Леонардо Ди Каприо) и Кейт Дибиаски (Дженнифер Лоуренс) обнаруживают, что к Земле мчится комета диаметром в десяток километров. Столкновение с планетой может привести к полному исчезновению не только человеческой, но и вообще всякой жизни. Рэндалл и Кейт грузятся в самолет и отправляются в Вашингтон, чтобы обсудить планы спасения Земли с высочайшими государственными чинами. Однако выясняется, что президент Дженни Орлин (Мэрил Стрип) куда больше увлечена живописным курением и секстингом с каким-то региональным отморозком. Главой администрации работает ее сын Джейсон (Джона Хилл), который в свою очередь в основном хвастается новой татуировкой дракона и упивается властью ходить на работу упоротым.

Минди и Дибиаски пытаются добиться огласки, выступив в популярном телешоу, однако Кейт заслуживает лишь волну мемов в интернете, а Рэндалл --- недвусмысленные знаки внимания ведущей (Кейт Бланшетт), которую возбуждает, что скоро мы все умрем. В какой-то момент правительство США все же снарядит спасательную экспедицию, но сразу после старта развернет ракеты, поскольку на сцену выйдет Питер Ишервелл (Марк Рейланс) --- визионер, объясняющий, что из столкновения с кометой тоже в теории можно извлечь пользу.

Фильмы и сериалы-катастрофы прошлого и нынешнего годов дали повод вновь заговорить о сверхъестественном чутье художников, прозревающих будущее без всяких на то логических объяснений. Вот и разработка «Не смотрите наверх» началась еще во вполне безмятежном ноябре 2019-го. Впрочем, никакой безмятежностью, разумеется, и не пахло --- Дональд Трамп вовсю собирался на второй срок, а Америка все глубже погружалась в депрессию, лекарство от которой так и не придумали до сих пор. Тем не менее за без малого полтора года, которые занял путь картины к зрителю, в мире изменилось слишком многое, создав для «Не смотрите наверх» уникальный и куда более подходящий случаю контекст.

В сверхъестественной проницательности в данном случае стоит обвинять сценариста и режиссера Адама Маккея. Это автор удивительной судьбы. Шесть лет назад он не пожелал сидеть в комедийном жанровом гетто и после дилогии про Рона Бургугди («Телеведущий») бросился покорять новые территории. В итоге его «Игра на понижение» стала одним из самых остроумных фильмов про кризис 2008-го года и принесла Маккею «Оскар» за лучший адаптированный сценарий. Через три года, в 2018-м, Адам решил развить успех и одновременно взвинтить ставки --- его «Власть» имела в своем центре ни много ни мало бывшего вице-президента Дика Чейни. Сатирический байопик абсолютного зла (именно такова трактовка Маккея) был воспринят чуть менее однозначно, несмотря на очередной актерский подвиг Кристиана Бейла. И вот в следующем своем проекте режиссер совместил едкий социальный комментарий с внешне легкомысленным задором своих ранних комедий.

\begin{fancyquotes}
    Кажется, что фильмы про летящие к Земле кометы вышли из моды на рубеже тысячелетий --- нулевые показали, что над нами летают штуки и пострашнее
\end{fancyquotes}

Последней из больших голливудских картин на тему, конечно, был пропагандистский шедевр Майкла Бэя «Армагеддон». Это было кино, где все важные вещи говорили на фоне развевающегося звездно-полосатого флага, а Брюс Уиллис, наконец, смог погибнуть --- но только под песню Aerosmith. Самым ярким послесловием к этому сюжету стала «Меланхолия» Ларса фон Триера, который с явным удовольствием разнес-таки нашу планету в труху. «Не смотрите наверх» отсылает к этим двум фильмам вполне прямолинейно. «Армагеддон» спародирован целыми фрагментами, а Уиллиса заменил Рон Перлман --- и так даже смешнее. С «Меланхолией» у Маккея значительно более нежные отношения. Пародиями тут не пахнет, скорее уж речь идет о трепетном оммаже --- при желании героинь Лоуренс и Бланшетт можно без особых поправок поместить в триеровский контекст.

Однако прямые и не очень аллюзии в данном случае отнюдь не самоцель. Маккей использует энергию предшественников, чтобы напитать ей совершенно авторское высказывание, сделанное, как водится, в сатирическом ключе. Режиссер владеет этим сложнейшим на самом деле жанром виртуозно и в «Не смотрите наверх» явно упивается возможностью не заботиться об исторической достоверности. На орехи тут достается абсолютно всем: озверевшим от самодовольства селебрити политикам, мямлям-ученым, не способным разговаривать человеческим языком, дурящим народ визионерам со своими дурацкими смартфонами… Перечислять мишени Маккея можно долго и с удовольствием, благо режиссер не придерживается более или менее никакой конкретной позиции --- просто стреляет во все, что видит.

За без малого два с половиной часа от такого потока желчи и презрения можно было бы устать, но фокус в том, что «Не смотрите наверх» этих потоков на зрителя, в общем, не льет. Остроумие наблюдений и сарказм авторских комментариев здесь уравновешен удивительно человечной интонацией. У Маккея с его врожденной язвительностью и острым глазом нет ответов на вопрос «что делать?» За происходящим на экране балаганом сквозит растерянность умного человека, вынужденного, как обычно, пытаться хоть как-то достучаться до идиотов. Это, пожалуй, и правда самая адекватная эмоция в мире, где ученый, сообщающий в ток-шоу о гибели человечества, добивается лишь статуса «астронома, которого я бы трахнула». Зато об этом мире можно снять фильм, который под конец очередного безумного года дарит не столько депрессию, сколько радость и умиротворение. Если мы все равно скоро умрем, то нет ни одной причины отказываться от праздничного ужина.

Фильм «Не смотрите наверх» (Don't Look Up) вышел на Netflix 24 декабря

