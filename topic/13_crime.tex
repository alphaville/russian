% \chapter{Закон}

\section[Про Бориса Акунина]{На Бориса Акунина завели дело по двум статьям УК}

\textit{На Бориса Акунина завели дело сразу по двум статьям УК --- о «фейках» и «оправдании терроризма»}

\textit{Источник: \url{https://www.bbc.com/russian/articles/cm50vmpje43o}}

\begin{fancyquotes}
    Росфинмониторинг в понедельник внес писателя Бориса Акунина (Григория Чхартишвили) в перечень террористов и экстремистов. Следственный комитет завел на него дело по двум статьям УК. Ранее от продажи его книг стали отказываться магазины.
\end{fancyquotes}

Для \ed{фигурантов}{фигурант}{%
    person involved in; another example is В настоящее время он
    --- единственный фигурант данного дела; }
перечня государство ограничивает возможность
финансовых операций:
у попавших в список почти полностью блокируются
все банковские счета.
Они даже не могут получать зарплату и расходовать более
10 тысяч рублей\footnote{10 000 RUB = 85 GBP} в месяц.
В реестр включают людей не только приговоренных
по экстремистским и террористическим статьям УК РФ,
но и \ed{подследственных}{подследственный}{person under investigation}.

О возбуждении уголовного дела против Чхартишвили (ранее он был объявлен «иноагентом») до внесения его имени в перечень террористов и экстремистов известно не было.

Позже в понедельник Следственный комитет России сообщил, что в отношении писателя возбуждено уголовное дело о публичном оправдании терроризма и о «фейках» об армии (ст. 205.2, ст. 207.3 УК РФ). В сообщении СКР сказано, что его планируют объявить в розыск.

«\ex{Вроде бы мелкое событие}{A seemingly minor event}, запрет книг,
объявление какого-то там писателя террористом, на самом деле важная
\ex{веха}{landmark, milestone}.
Книг в России не запрещали с советских времен.
Писателей не обвиняли в терроризме со времен Большого Террора%
\footnote{«Большой террор» (\textit{разг.} «Еж\'{о}вщина»)
--- термин современной историографии, характеризующий период
наиболее массовых политических (сталинских) репрессий
в СССР 1937—1938 годов.}.
Это не дурной сон, это происходит с Россией
\ex{наяв\'{у}}{in reality}, на самом деле.
Берегите себя, не потеряйте себя, если вы в РФ.
А если вы уехали, но вам очень тяжело
\ex{на чужб\'{и}не}{in a foreign land} и вы подумываете вернуться
--- не возвращайтесь.
Ночь будет становиться черн\'{е}е и черн\'{е}е.
Но потом все-таки рассветет», ---
так прокомментировал новость сам Борис Акунин.

На прошлой неделе издательство АСТ объявило о том,
что приостанавливает распространение книг
Акунина и Дмитрия Быкова (в реестре «иноагентов»).

О прекращении продаж книг Акунина и Быкова
также объявили сеть книжных магазинов «Читай-город — Буквоед»
и сервис цифровых книг «Литрес». В «Читай-городе» решение
о приостановке продаж книг двух писателей объяснили их
«недавними \ed{высказываниями}{высказывание}{statement},
которые получили широкую огласку в СМИ».
О каких именно высказываниях идет речь, неизвестно.

\textbf{Что будет с книгами Акунина в России?}
Хранение книг Акунина пока еще остается законным,
отмечают эксперты.
Как пишет адвокат проекта «Первый отдел» Евгений Смирнов,
согласно закону о противодействии экстремистской деятельности,
запрещается распространение только тех материалов,
которые включены в реестр «экстремистских».

«Никакие книги, написанные Борисом Акуниным, а также фильмы и сериалы,
которые сняты по его романам или сценариям в этот реестр не внесены.
Если художественные произведения начнут пропадать из книжных магазинов
и стримингов --- это будет инициатива этих книжных и этих
стриминговых платформ.
Читать в метро книги о Фандорине тоже не запрещено», --- пояснил Смирнов.

\textbf{«Я помогал, помогаю и буду помогать украинцам».}
Издательство АСТ в объявлении о прекращении продаж книг
Акунина и Быкова ссылалось на «публичные заявления писателей,
которые вызвали широкий общественный резонанс».

\ex{Судя по всему}{as it appears},
имелась в виду история с провластными пранкерами Лексусом
(Алексеем Столяровым) и Вованом (Владимиром Кузнецовым),
которые обманом связались с Акуниным и Быковым и позже
опубликовали отрывки из разговоров с писателями.

Пранкеры говорили с Акуниным якобы от имени президента
Украины Владимира Зеленского и бывшего министра культуры страны
Александра Ткаченко.

В одном из отрывков, которые без контекста опубликовали
прокремлевские пранкеры, Акунин якобы говорит,
что с пониманием относится к позиции
«хороший русский --- мертвый русский».

Позже Акунин в «Фейсбуке» назвал действия пранкеров провокацией.

«Про то, что я якобы согласен с тезисом ``хороший русский
--- мертвый русский'', разумеется, \ex{брехн\'{я}}{nonsense}.
Как принято у этой шушеры, они что-то в моих словах отрезали,
что-то намухлевали, что-то \ex{перекомпоновали}{rearranged}»,
--- отметил Борис Акунин.

В разговоре с пранкерами, если судить по опубликованным отрывкам,
Акунин выразил готовность оказывать поддержку Украине.

«Я думаю, что и все другие деятели российской культуры,
которые выступают против путинской диктатуры,
с удовольствием вам помогут и примут участие»,
--- говорил, в частности, Акунин.
Эта фраза широко разошлась в соцсетях.

Писатель позже прокомментировал это высказывание так:
«Я помогал, помогаю и буду помогать украинцам».

Быков в разговоре с пранкерами произнес фразу,
возмутившую сторонников вторжения в Украину.
Ее в частности цитировало и государственное агентство ТАСС.

«Конечно, когда убивают русских, мне обидно.
Но претензий к вам у меня нет, как у меня нет претензий
к Израилю насчет Газы», — сказал писатель.

Быков и Акунин ранее неоднократно публично выступали
с осуждением российского вторжения в Украину.
Борис Акунин с 2014 года с семьей живет за границей.

Борис Акунин сам сообщал, что несколько театров
--- Российский академический молодежный театр (РАМТ)
в Москве и Александринский театр в Санкт-Петербурге
--- убрали его имя с афиш спектаклей по его произведениям.

Летом пр\'{о}шлого года издательство «Захаров» сообщило,
что московский магазин «Молодая гвардия» снял с продажи
книги писателя Бориса Акунина.
В ответ издательство отозвало из магазина все свои книги.


\newpage
\section{Запрет чайлдфри, налог на бездетность}

\textit{Запрет чайлдфри, налог на бездетность: как в России борются за рождаемость}

\textit{Источник: \url{https://news.ru/}}
% https://news.ru/vlast/zapret-chajldfri-nalog-na-bezdetnost-kak-v-rossii-boryutsya-za-rozhdaemost/

\textit{В Госдуме объявили о подготовке запрета пропаганды чайлдфри}

В Госдуме приступили к работе над законопроектом,
которым будет запрещена пропаганда чайлдфри в информационном
пространстве.
Кто и зачем хочет запретить идеологию бездетности,
запретят ли в России аборты,
каким налогом могут обложить россиян,
как власти б\'{о}рются за демографию?

\textbf{Кто хочет запретить пропаганду чайлдфри?}
О подготовке запрета на пропаганду бездетности рассказала
депутат Госдумы Ирина Филатова.
Над законопроектом работают парламентарии из нескольких
думских фракций.

«Межфракционная рабочая группа по защите традиционных ценностей
Госдумы готовит законопроект о запрете пропаганды чайлдфри.
Инициативу поддержали в ряде федеральных ведомств.
Сегодня в ограниченном формате законопроект обсуждался
на слушаниях в Общественной палате, где тоже нашел поддержку»,
--- рассказала Филатова.

Она подчеркнула, что «речь идет исключительно о деструктивной
пропаганде в информационном пространстве».

Ранее схожий законопроект разрабатывали в Башкирии с целью
защитить детей от пропаганды бездетности.
Депутаты особо подчеркивали, что инициатива не должна коснуться
вопросов бездетности по религиозным соображениям, а также книг
и фильмов о жизни монахов.

\textbf{Запретят ли в России аборты?} В марте 2023 года в Русской
православной церкви призвали ввести в России уголовное наказание
за склонение женщины к аборту в медицинских учреждениях и дополнительно
ужесточить наказания за криминальный аборт.
Депутат Госдумы Руслан Хамзаев развил эту идею,
предложив приравнять склонение к аборту к убийству человека.

Радикальные инициативы не получили немедленной поддержки,
однако к ноябрю в Мордовии, Тверской и Тамбовской областях были
введены штрафы за «склонение беременной женщины к искусственному прерыванию беременности»
--- вне зависимости от того, был произведен аборт или нет.
Патриарх Московский и всея Руси Кирилл поддержал инициативу,
но предложил расширить ее на федеральный уровень.

Член Совета Федерации Ольга Ковитиди призвала не ограничиваться
штрафами за склонение к аборту и ужесточить наказания.
Ей \ed{возразила}{возразить}{to object, to raise an objection}
глава думского комитета по правам семьи, материнства и детства
Нина Останина, призвав «прекратить организованную
\ed{травлю}{тр\'{а}вля}{bullying} врачей».
Останина ранее отмечала, что депутаты,
которые требуют запретить аборты,
ранее сами же голосовали за сокращение материнского капитала.

На региональном уровне наказания за склонение к абортам
получили поддержку: аналогичные меры в ноябре приняли и
в Калининградской области, а в декабре --- в Курской.
В ряде регионов власти сообщили, что частные клиники
в добровольном порядке прекратили проведение абортов.
О запрете процедуры, настаивают депутаты, речи не идет.

«В каждом случае все индивидуально.
У кого-то, например, есть медицинские показания.
[...] Но мы хотим ограничить наших женщин от воздействия
извне в плане пропаганды.
Речь идет, например, о какой-то явной рекламе,
например о баннерах», --- заявил курский парламентарий Юрий Амерев.

\textbf{Кто поддерживает запрет пропаганды чайлдфри?}
Против идеологии чайлдфри уже длительное время выступает депутат
Госдумы РФ Виталий Милонов. Он назвал принципиальную бездетность
«противоестественной системой ложных ценностей,
пропаганда которой не может положительно сказаться на состоянии
внутри страны».

Чайлдфри, по его словам, следует обозначить как экстремистскую
идеологию и запретить в России.

Схожее заявление сделала \ex{зампред}{deputy chairman}
Совета по правам человека Ирина Киркора,
которая усмотрела в чайлдфри «годы идеологической пропаганды»,
навязанной России.

«Безусловно, понятие ``чайлдфри'' не соотносится с теми
базовыми нравственно-духовными ценностями и категориями,
установленными в указе нашим президентом.
Вообще ``чайлдфри'' --- это не наш термин, придуман был не нами,
а, скорее, навязан уже нескольким поколениям, и он,
к сожалению, стал модным.
А к чему это приводит? Это может привести только в демографическую
\ed{яму}{яма}{pit} и в итоге к
\ed{исчезновению}{исчезновение}{disappearance} целой нации»,
--- объявила Киркора в беседе с РИА Новости.

\textbf{Как предложили ответить пропаганде чайлдфри в России?}
Ответом российских законодателей на пропаганду чайлдфри может
стать возвращение налога на бездетность.
Такой существовал в Советском Союзе и может быть введен вновь,
заявил депутат Госдумы Евгений Федоров. По мнению члена комитета
по бюджету и налогам, налог на бездетных поможет обеспечить
материнский капитал.

«Надо ли вводить налог для этого?
Если денег не будет хватать для этих проектов --- надо.
Если денег будет хватать без этого, то не надо.
Это не наказание, это способ решения проблемы»,
--- объявил Федоров.
