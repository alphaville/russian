\chapter{Война и мир}

\section{Что ждет россиян во время частичной мобилизации?}
\textit{Источник: \url{https://lenta.ru/articles/2022/09/21/ukaz/}}

\textit{Ответы на главные вопросы}

Президент России Владимир Путин 21 сентября 2022 года объявил о введении в России \ed{частичной}{частичный}{partial} мобилизации. По его словам, призыву на военную службу подлежат граждане, состоящие в запасе, и прежде всего те, кто служил в армии. Кто теперь подлежит призыву, кто имеет право на \ed{отсрочку}{отсрочка}{delay; postponement}, каков порядок мобилизации — в этих вопросах разбиралась «Лента.ру».

\ed{Указ}{ук\'{а}з}{decree} «Об объявлении частичной мобилизации в Российской Федерации» опубликован на сайте Кремля в среду, в 9 часов 20 минут.

«Объявить с 21 сентября 2022 года в Российской Федерации частичную мобилизацию», — сообщается в первом пункте документа, подписанного президентом России Владимиром Путиным.

Во втором пункте указано, что мобилизованные граждане получат статус военнослужащих-контрактников с денежным содержанием соответствующего размера. Увольнение из рядов мобилизованных предусмотрено только по возрасту и по состоянию здоровья.

Правительству поручено финансировать мероприятия по проведению частичной мобилизации, главам регионов — обеспечить призыв находящихся в запасе граждан.

\textbf{Сколько человек призовут?}

Согласно заявлению главы Минобороны Сергея Шойгу — 300 тысяч. Это примерно 1,1 процента от общего мобилизационного ресурса России, который, по словам министра, составляет почти 25 миллионов человек. Для каждого региона количество подлежащих мобилизации будет определяться отдельно.

\textbf{Кого призовут в первую очередь?}

По словам Шойгу, на военную службу по мобилизации будут призваны те, кто отслужил в армии, имеет военно-учётную специальность и соответствующий опыт. Перед отправкой в части они пройдут обязательную дополнительную подготовку.

\ed{Полковник}{полковник}{colonel (army)} \explain{в отставке}{retired} Виктор Баранец считает, что первыми под мобилизацию попадут резервисты, которые каждый год по указу президента призывались на военные сборы, стреляли, водили танки и наводили ракеты. Кроме того, будут призваны люди, по разным причинам уволенные с контрактной службы, а также молодые офицеры.

«Резервисты -- это \explain{военнообязанные}{conscripts; liable for military service} граждане государства, которые подлежат мобилизации при необходимости, -- рассказал в свою очередь «Ленте.ру» \explain{судь\'{я}}{judge} в отставке Владимир Комсолев. —- В нашем случае это лица, прошедшие службу и имеющие достаточную подготовку. Они заключили контракт о пребывании в резерве и получают за это деньги. Раз в год резервисты выезжают на военные сборы, раз в месяц участвуют в занятиях».

Юрист Артем Мугунянц в разговоре с «Лентой.ру» отметил, что тех, кто пользовался отсрочкой в период с 18 до 27 лет и не служил, мобилизация, вероятно, не коснётся.

«Мобилизации подлежат только те, кто служил в армии либо офицером, либо проходил срочную службу, либо был контрактником\footnote{Only those who served in the army either as an officer, or did military service, or were a contract soldier \textit{are subject to} mobilization}. — пояснил эксперт. — Можно предполагать следующее. Так как утверждается, что необходимо освобождать территории Украины, им потребуются \explain{наступательные войска}{offensive troops}: танкисты, \explain{морская пехота}{marines}, мотострелковые части. Эти части и войска могут потребоваться в большом количестве. Тех, кто служил в этих войсках, будут мобилизовать в первую очередь».

\textbf{Людей какого возраста могут призывать?}

Согласно статье 53 Федерального закона «О воинской обязанности и военной службе», все военнослужащие запаса делятся на три разряда. В случае мобилизации первым в ВС попадает первый разряд, затем — второй, третий — последним (подробнее о том, что значат категории запаса и категории здоровья — ниже).

По первому разряду призываются солдаты и низшие \ed{чины}{чин}{rank} в возрасте до 35 лет, младшие офицеры в возрасте до 50 лет (в третьем разряде верхние рамки для этих званий повышаются до 50 и 60 лет соответственно) и так далее. Представители генералитета подлежат мобилизации по второму разряду в возрасте до 70 лет.

В Госдуме заявили, что планируют призывать тех, кто попадает под первый разряд.

«Пока человек не снят с военного учёта, он может подлежать мобилизации, — пояснил «Ленте.ру» профессор кафедры уголовного права РГПУ имени А.И. Герцена Сергей Милюков. — Есть специальности в ближнем и дальнем тылу, где возраст не является большой \ed{пом\'{е}хой}{пом\'{е}ха}{hindrance}. Например, обязательно потребуются медики. Призванные врачи будут оказывать медпомощь в госпиталях в тылу. Для лечения и \ed{дол\'{е}чивания}{дол\'{е}чивание}{follow-up treatment} нужно привлечь санатории, как это было в Великую Отечественную войну. Непосредственно же в боестолкновениях должны участвовать более молодые люди».

\textbf{Что означают категории запаса и категории здоровья в военном билете? }

Существует пять категорий годности к военной службе. Их определяют исходя из показателей здоровья и записывают данные об этом в военном билете.
\begin{itemize}
    \item «А» — \explain{годен}{fit} к военной службе;
    \item «Б» — годен к службе с \ed{незначительными}{незначительный}{insignificant} ограничениями;
    \item «В» — ограниченно годен к военной службе;
    \item «Г» — временно не годен;
    \item «Д» — не годен к несению военной службы.
\end{itemize}
Группы здоровья «А», «Б» и «В» по федеральному закону \explain{подлежат}{are subject to} мобилизации.

Всех резервистов \ed{распределяют}{распределять}{distribute} по трем категориям в зависимости от возраста военнослужащего и полученного звания.

Первая категория — граждане, которых призывают в первую очередь в случае мобилизации. \explain{Речь идет о}{This is a very useful expression; it means ``this is about...''} военных, которые не получили офицерский чин (в том числе \ed{прапорщики}{прапорщик}{ensign} и мичманы) и не \explain{перешагнули}{stepped over} возрастной \explain{пор\'{о}г}{threshold} в 35 лет, а также о младшем офицерском составе до 45 лет (лейтенант, капитан). В данной категории числятся старшие офицеры до 50 лет (майор, \explain{подполковник}{lieutenant colonel}); полковники, капитаны 1 ранга до 55 лет; высшее руководство до 60 лет. Первым на службу должен прибыть офицерский состав высшего эшелона.

Вторая категория — военнообязанные старшей возрастной группы или те, кто не прошел в первую волну мобилизации по здоровью. В эту группу попадают солдаты, военнослужащие без офицерского звания в возрасте от 35 до 45 лет, младший руководящий состав от 45 до 50 лет и старшие офицеры от 50 до 55 лет.

Третья категория — военнослужащие, которых призывают только в том случае, если военные действия длятся больше года и необходимы дополнительные силы. Под эту категорию попадают \explain{рядовые}{privates} и \explain{матросы}{sailors}, прапорщики и мичманы самой старшей возрастной категории. Возраст низшего\footnote{низкий, низшего} офицерского состава этой категории до 55 лет, а для капитанов 2 и 3 ранга и подполковников — до 60 лет. В данной категории пребывают военнообязанные женщины в возрасте до 50 лет.

\textbf{Что такое мобилизационное предписание? }

Это документ, который выдается части запасников. Выдача мобилизационного предписания — это \explain{своеобразная}{peculiar, singular, sui generis} перепись военнообязанных лиц. Гражданин, который получил мобилизационное предписание, при объявлении мобилизации должен прибыть в указанное в нем место и срок без дополнительной \ed{повестки}{повестка}{subpoena, writ} и предупреждения.

Как \explain{утверждает}{claims} юрист Павел Чиков, во время мобилизации гражданам приходят повестки, как и в обычное время — лично в руки или под подпись. Повестку также могут вручить по месту работы.

Однако именно запасники, имеющие мобилизационные предписания, по статье 21 ФЗ «О мобилизационной подготовке и мобилизации в Российской Федерации» самостоятельно должны \explain{явиться}{show up} в военный комиссариат.

\textbf{Кто имеет право на отсрочку?}

Как следует из указа, подписанного Путиным, \ed{отсрочку}{отсрочка}{postponement} пол\'{у}чат работники оборонных предприятий. «\explain{Предоставить}{provide} гражданам Российской Федерации, работающим в организациях оборонно-промышленного комплекса, право на отсрочку от призыва на военную службу по мобилизации (на период работы в этих организациях). Категории граждан Российской Федерации, которым предоставляется право на отсрочку, и порядок его предоставления определяются правительством Российской Федерации», — говорится в документе.

Как следует из статьи 18 федерального закона о мобилизации, отсрочка также предоставляется:

\begin{itemize}
    \item временно \ed{негодным}{негодный}{ineligible, not qualified} к военной службе по состоянию здоровья;

    \item ухаживающим за близкими родственниками, которых больше некому содержать;

    \item многодетным родителям (тех, кто имеет на \ed{иждивении}{иждив\'{е}ние}{dependent} не менее четырех детей в возрасте до 16 лет или троих детей при условии беременности супруги сроком не менее 22 недель);

    \item родителям-одиночкам;

    \item членам Совета Федерации и депутатам Госдумы.
\end{itemize}

Другие категории военнообязанных, которым \explain{полагается}{depends, relies} отсрочка (бронь от призыва), еще не известны. Их определяет правительство России, напомнил пресс-секретарь президента Дмитрий Песков.

\textbf{Попадают ли под мобилизацию срочники и те, кто не служил в армии? }

По закону о мобилизации — не попадают. Из него следует, что в ряды вооруженных сил должны быть направлены «граждане, пребывающие в запасе, не имеющие права на отсрочку от призыва на военную службу по мобилизации».

По словам юриста Ольги Лютницкой, \explain{срочники}{conscript} не попадают под эту категорию, так как они еще не переведены в запас.

Если гражданин не служил, например, потому, что был ограниченно годен, но у него есть военный билет, то веротность его мобилизации есть, сказала она «Ленте.ру».

Юрист Мугунянц уточнил, что срочники уже являются военнослужащими по призыву. Задействовать их до объявления полной мобилизации и войны нельзя.

\textbf{Можно ли мужчинам теперь выезжать за границу? }

В Федеральном законе «О мобилизационной подготовке и мобилизации в Российской Федерации» говорится, что «гражданам, состоящим на воинском учете, с момента объявления мобилизации воспрещается выезд с места жительства без разрешения военных комиссариатов, федеральных органов исполнительной власти, имеющих запас».

Юрист Лютницкая утверждает, что пока по вопросу \ed{запрета}{запрет}{ban, prohibition} выезда мужчинам за границу нет никаких законодательных актов и комментариев. Поэтому прямо сейчас никаких ограничений для выезда не существует.

Юрист Мугунянц также отметил, что выезжать никому не запрещено: выезд закрывается только при полной мобилизации.

«Если же человека мобилизовали, то есть признали военнослужащим, дали мобилизационное предписание о том, что он должен явиться, то он действительно не может выехать в другую страну. Потому что он мобилизован», — уточнил эксперт.

При этом скорее всего человека не будут вносить в каике-либо базы, но гражданин должен \explain{учитывать}{take into consideration}, что выезд будет расценен как \explain{уклонение}{evasion} от армии, за которое возникнут соответствующие \explain{уголовные последствия}{criminal implications}.

Член Совета по правам человека при президенте России Кирилл Кабанов заявил, что для людей без повесток на руках юридического запрета на выезд за границу нет.

\textbf{Что ждет тех, кто по достижении 27 лет получил справку вместо военного билета?}

Те, кто не проходил службу в вооруженных силах \explain{без уважительной причины}{without good reason} (например, сознательно \ed{уклонялся}{avoided, dodged}), и по достижении 27-летнего возраста получил справку \explain{взамен}{instead of} военного билета, вероятно, не будут призваны во время частичной мобилизации, так как они тоже — не служившие люди, считает юрист Мугунянц.

«На данном этапе у них никаких проблем нет. Если объявят всеобщую мобилизацию, то [обладателей справок] будут призывать, как и всех. К ним будут применяться те же условия, как к людям, которые не служили, но имеют военный билет. То же касается и тех, кто вообще не имеет никаких документов [подобного рода]. Важно, что они находятся в мобилизационном возрасте и подпадают под [всеобщую] мобилизацию».

По российскому законодательству, обладатель справки не может быть принят на государственную службу, в остальном его права не отличаются от прав владельца военного билета.

\textbf{\explain{Затрагивает}{affects} ли мобилизация женщин?}

По закону, мобилизация граждан, пребывающих в запасе, \explain{предполагает}{suggest} возможность призыва на военную службу медработников женского пола в возрасте до 45 лет. У них на руках есть военные билеты — медработники получают их после учебы.

Адвокат Владимир Шелупахин объяснил «Ленте.ру», что призыв состоящих на воинском учете женщин предполагается только в третью очередь, то есть после мужчин в возрасте 35-45 лет, в том числе граждан, пребывающих в запасе, но годных к военной службе с незначительными ограничениями (категория «Б») или ограниченно годных к военной службе (категория «В»).

«\ed{Вовсе}{вовсе}{at all} \explain{освобождаются}{released} от службы матери-одиночки и многодетные матери», — добавил юрист. Кроме того, на женщин распространяются и другие отсрочки, описанные в федеральном законе.

\textbf{Могут ли повестки приходить через «Госуслуги»?}

По официальной информации — нет. О возможной отправке повесток через сервис \ed{Госуслуг}{Госуслуги}{public services} 21 сентября написал Telegram-канала Baza. По информации издания, всем, кого собираются призвать, придет уведомление в аккаунте Госуслуг дополнением к обычным повесткам через почту.

Однако в Минцифры почти сразу это \explain{опровергли}{refuted}. «В связи с появившимися в соцсетях публикациями о том, что электронные повестки в рамках частичной мобилизации будут \explain{рассылаться}{be sent out} через Госуслуги, сообщаем, что таких планов нет. Необходимая законодательная база для этого отсутствует», — говорится в Telegram-канале министерства.

\newpage
\section{Таких замученных людей я раньше не видел}

\textit{Рассказ россиянина, который несколько суток пытался уехать в Казахстан }

\textit{Источник: \url{https://baza.io/posts/4250c126-5344-458d-af97-21179596c136}}

Сутки на жаре и холоде, драки, голод и бессонница. Со всем этим столкнулся Александр, который после объявления о «частичной мобилизации» решил уехать на машине в Казахстан. Однако там его, как и тысячи других россиян, встретил своеобразный тест на выживание и целый ряд гуманитарных проблем.

Это детальный рассказ Александра о том, что сейчас происходит на границе с Казахстаном в Астраханской области.


\textbf{Добраться до границы}

Мчались к границе под Астраханью в режиме аврала: за ночь собрали чемоданы, бросили ключи от квартиры с котом друзьям и полетели в Волгоград, так как только туда в эти дни были нормальные цены на билеты. Самолёт был полностью забит мужчинами. Когда заходил, услышал, как одна бортпроводница шепнула другой насчёт пассажирки: «О, девушка, ничего себе! Впервые за два дня».

Уже в самолёте стало ясно: ехать, как мы планировали, на поезде до Астрахани очень долго, и потому прямо в аэропорту взяли таксиста. Спросили, за сколько он довезёт до границы под Астраханью, и он, назвав сумму в 15000, сразу получил её в руки. Мы гнали больше ста весь путь. В какой-то момент водитель даже чересчур заспешил и чуть не размазал нас по встречной фуре.

Остановились только один раз — на пустынной заправке у трассы. Там был заправщик, который принялся нам угрожать и затевать драку из-за того, что мы бежим из страны. Мы ничего ему не говорили, но ему было очевидно, что мы москвичи, так как он обронил «из-за таких, как вы, у меня всех братьев забрали».

Обидно было, что он именно нас в этом обвиняет, а не власть. Но вместе с тем жалко парня: многим очевидно, что в сёлах больше пропорции набора.


\textbf{Звёзды и безысходность под Астраханью}

Подобрались к границе Астрахань — Атырау с наступлением темноты, тогда длина пробки до КПП была уже 14,9 километра. Там уже царила анархия: нам с ходу предложили купить велосипед за 50000 рублей, на котором можно было проехать границу. Далее следовал жуткий марш-бросок в кромешной тьме с чемоданами под дождём в дикий холод: вместе с тьмой на пробку опустилась осенняя прохлада и ливень.

Мы прошли вместе с чемоданами около 8 километров, разглядывая, как люди друг с другом ругаются из-за мест, как плачут уставшие дети в машинах, как кричат кошки в переносках. Люди пытались лечь спать в совершенном аду. Над нами, стоит отметить, висело фантастически красивое звёздное небо: в радиусе километров не было огней, и это был фантастический вид на звёзды. Но вместе с тем впервые пришло отчаяние: стало понятно, что выжить тут будет нереально.

Вокруг сновали люди на велосипедах с подвязанными к спине чемоданами, люди на гироскутерах, самокатах, местная банда бородатых наглецов, отжимавших места поближе к границе на продажу, и огромная, невообразимая по длине пробка из легковых и большегрузных машин, у которой не было конца. Спустя два часа такого движения, не найдя конца очереди, мы устали смертельно, и пришло понимание: либо мы сейчас едем в аэропорт и домой, либо должны немедленно устроиться на ночлег. Я был замерзший, мокрый и абсолютно раздавленный безысходностью.

И тут я увидел паренька лет 20, который ехал один в этой пробке на своей машине. Ходили слухи, что пробку на машине можно преодолеть только за сутки-полтора, и я удивился: как он в одиночку, без смены собирался ехать такой период времени без остановки? Мы предложили ему за деньги взять нас до границы, пообещав подменять за рулём для сна. Парень согласился, и мы залезли в машину. Боже, как было уютно и тепло после холодной дождливой улицы! Мы часа три-четыре ехали с ним, стараясь лавировать между дальнобойными машинами, которые пытались блокировать дорогу бомбилам, занимающим места. При этом мы весело болтали за жизнь. Стало понятно, что компания приятная.

Я чувствовал, что дико устал. Время уже было около часа ночи, как вдруг движение встало. Перед нами была вереница дальнобойщиков, уходящая в бесконечную даль, и мы в ожидании, когда вновь начнётся движение, заглушили двигатель и в какой-то момент уснули.


\textbf{«Голодные игры»}

Я проснулся спустя пару часов: было холодно даже внутри машины, хотя спал я в куртке. Парень, который нас взял, курил возле авто, закутанный во всю одежду, что у него была. Мы стояли окружённые со всех сторон дальнобойщиками, которые спали в кабинах. А вокруг был шум двигателей. Казалось, все едут, и только мы стоим в своём маленьком дальнобойном ряду.

Парень сходил на разведку и вернулся с криком «Погнали!». На часах было только 4 утра, и тьма была непроглядная. Мы очень аккуратно вылезали между дальнобойщиками на обочину и, когда выехали, увидели сцену из какого-то плохого спин-оффа фильма «Голодные игры»: сотни автомобилей носятся по полю, пытаясь обогнать друг друга в пробке на дороге и влезть где-то ближе к первому блокпосту ГИБДД.

Мы вместе со всеми выехали на поле и оказались в абсолютной мясорубке: ночь, вой моторов и сотни водителей, выдавливающих друг друга с дороги в кювет. Уже не знаю, как у паренька это вышло, но благодаря своей наглости и терпению он нашёл небольшой перекрёсток с просёлочной дорогой, чтобы вклиниться, и мы заняли позицию до рассвета.


Рассвело к шести, и на перекрёстке появился сотрудник ГИБДД. Он пытался разрулить весь тот кошмар, который произошёл за предрассветные часы. Водителей, ждущих в пробке уже сутки, это злило: кто-то за ночь влез без очереди из-за дальнобойщиков, которые заблокировали поток в знак отместки за потерянное время. Теперь такие водители не давали сотруднику ГИБДД пропускать влезшие с обочины машины.

Инспектор был абсолютно уставший и измождённый и в какой-то момент просто ушёл, оставив всё это дело жить своей жизнью. А жизнь была такая: пока на горке совершали намаз мусульмане, семьи с детьми пытались умыть уставших и невыспавшихся детей, а кто-то гулял с собакой, привезённой с собой в этот кошмар.

«Нам тут не влезть», — хмуро сказал наш водитель, и мы пошли искать сообщников на прорыв. Картина открывалась жуткая. Поговорив с окружающими, мы узнали, что люди, стоявшие в честной пробке, провели здесь уже сутки, тогда как наш маршрут занимал около 10 часов. Многие были с детьми, пожилыми родителями, животными. Кто-то из них набрал попутчиков. Но всех объединяло одно: все они, так же как и мы, ночью были заблокированы колонной дальнобойщиков.

Услышав шум машин и оскорбившись такой наглостью, водители бросились объезжать их по полям, перемешивая все очередности. Этот перекрёсток был всего в 5 километрах от границы.


\textbf{Банка майонеза, драки и ФСБ}

В пробке было много машин с Северного Кавказа: с номерами из Дагестана, Чечни, Ингушетии. Они везли семьи, детей, родителей: их семьям угрожало наказание за то, что они уклоняются от мобилизации, потому всех брали с собой. Здесь же были и напуганные студенты, и взрослые мужчины, и ребята со всей страны: Ростовской, Краснодарской областей, попадался Санкт-Петербург и Карелия. Представьте, люди приехали сюда из Карелии!

Куда, спрашиваю у водителя, здесь ходят в туалет? Он смеется: «А как ты думаешь? Спустись с дороги». И тут я увидел страшное зрелище: отходить от дороги далеко нельзя — вдруг твоё место займут? Тех, кто ушёл дальше 50 метров, разворачивали пограничники. Поэтому люди были вынуждены ходить в туалет прямо здесь, на глазах всей пробки. Десятки тысяч людей, женщин, мужчин, любых вероисповеданий и культур. Все делали это здесь.

Обстановка накалялась: люди не могли поделить дорогу, и вспыхивали драки в разных частях пробки. Солнце окончательно поднялось, и стало жарко. Ещё ночью ты заворачивался во все вещи из чемодана, а сейчас потел в футболке.

Стало понятно, куда уехал сотрудник ГИБДД: он вернулся с группой пограничников и бойцами ФСБ на «Барсе». Они подняли оружие, будто угрожая начать пальбу в воздух, но толпу это мало пугало: несколько ретивых ребят попытались втянуть в драку инспектора. Только с появлением майора ГИБДД — уставшего, с выгоревшей на солнце кожей и грозным голосом — удалось сбить накал и договориться о системе проезда. Разумеется, и она не соблюдалась. Как только инспектор принимался разруливать проблему в одной точке пробки, вспыхивали бои в другой.

Мы смогли въехать на точку, указываемую в многотысячных чатах пробки как ключевую, после которой «ад заканчивается» и «всё становится интеллигентно». Здесь находился узкий мост, и далее авто двигались в образовавшейся колонне одна за другой. Ну как двигались: за 12 часов все продвинулись на 600–700 метров, не более.

Тут появилась гуманитарная проблема: у кого-то заканчивалось топливо, у кого-то еда и вода для детей и взрослых. Некоторые были отправлены в город за едой, но тут же попадали в капкан: их машины не хотели пропускать обратно в задней части пробки, и эти машины выбывали из общей гонки.

К этому времени о пробке уже трубили СМИ. Новости от тех, кто был на границе, расстраивали: они стояли по трое суток. Мы достигли посёлка, в конце которого находился КПП, только к темноте. В посёлке в это время совершенно опустел магазин: только одинокая банка майонеза стояла в пустом холодильнике. Проблема с водой, едой и топливом решена не была: люди в моей части колонны делились этим с соседями, некоторые продолжали толкать машину, чтобы не заводить авто и не тратить топливо. Я отдал две из трёх оставшихся бутылок воды в соседние машины с детьми, а также раздал большую часть своей аптечки. Особенно было популярно обезболивающее.


Обочина в этой части была полна мусора от наших предшественников, а также сбежавшимися на него собаками. Это отпугивало людей от туалета. Еда почти у всех кончилась, наступила ночь, и снова стало холодать.

На помощь пришли жители посёлка. Они за крайне низкие цены стали продавать еду, заряжать телефоны, бесплатно пускать в туалет, а детей — отдохнуть в дома. Но вместе с темнотой пришла и новая проблема: организованные «мафиози» держателей мест вклинивались в ряды. Их целью было удержать позицию в этой части пробки и продать их тем, кто прибыл в конец.

По слухам, место в этой части пробки стоило по 25–30 тысяч рублей. Это приводило к конфликтам и дракам, маханию ножами и угрозам убийством. На самые громкие драки прилетали сотрудники ФСБ на «Барсе», в балаклавах и с автоматами. Это успокаивало конфликтующих — до отъезда офицеров.


\textbf{Дорожная «мафия»}

Здесь, в 4 километрах от границы, уже было много пешеходов с сумками. Они доходили до этой части колонны и просились в машины: переходить границу пешком было нельзя. У обочины были свалены велосипеды: люди, накупившие их в конце пробки за 30–50 тысяч рублей, у КПП узнавали, что на велосипеде было всё-таки нельзя. Пеших кто-то подсаживал бесплатно, а кто-то брал за деньги. И тут начался новый коллапс: каждый, кто слышал о более высокой цене, поднимал цену у себя, и в итоге стоимость места в машине выросла до 40–50 тысяч рублей.

Чем темнее становилось, тем активнее начинались бои во второй части «Голодных игр». Вместе с этим накопилась усталость: за двое суток, из которых на сон пришлось 3 часа, мы не выспались. Первоначальный план, согласно которому мы хотели меняться местами с водителем, разбился о необходимость вылетать из машин по свисту — для обороны мест от влезавших барыг-продавцов. Это длилось всю ночь.

Третьи сутки прошли по абсолютно такому же сценарию, но изменились цены: мы были уже в 2 километрах от КПП, и цены на места для пассажиров достигали 70 тысяч. Еда и вода стали дефицитными, мы совсем отказались от еды в пользу детей, водой нас снабжали местные жители совершенно бесплатно. Для экономии топлива многие стали толкать машины. Чтобы хоть как-то спать, часть не умеющих водить пассажиров освоили основы вождения.

К ночи началась третья серия «Голодных игр» с ещё более активным противостоянием. Здесь уже оставалось менее 500 метров до КПП, и дорога была блокирована десятками мужчин, утверждавших, что это их территория, а машины будут пропускаться по системе «шесть из колонны — одна их».


«Одна их» представляла собой минивэн с вместимостью 6–12 человек, куда сажали пешеходов по цене от 70 до 90 тысяч рублей за место. На дороге появились «служебные машины» с сопровождением ГИБДД, следовавшие в сторону КПП. В числе служащих были замечены пожилые женщины и подростки. Обратно они не возвращались.

Бои длились до утра. Под эффектный рассвет мы пересекли КПП. Там все уже были знакомы: сотрудники ФСБ, помогавшие отвоёвывать места, пограничники, с уставшими лицами проезжавшие мимо пробки на микроавтобусе, работники ГИБДД, нервно курившие после изнуряющей смены в ночном кошмаре.

На КПП были и молодые люди, которых задержали в связи с тем, что они получили повестки. Стало ясно, что слухи о списках на границах вовсе не были слухами.


\textbf{Переход}

Под встающее солнце мы с уже родными соседями пересекали границу и неслись по буферной зоне, в которой, как нам завещали сотрудники российской погранзаставы, не стоит покидать автомобиль. Проехав шесть километров из двенадцати, мы уткнулись в новую пробку.

У буферной зоны была особенность: здесь не было ни сотрудников ГИБДД, ни ФСБ. Зато были свои барыги: они перевозили людей за 30–50 тысяч от одного КПП до другого через встречную полосу. Здесь они были более подготовлены и ездили по двое, с охраной. Но и люди, попавшие в эту зону, были уже ветеранами и прямо между двумя погранзаставами мастерили себе примитивное оружие близкого боя. Как ни странно, боёв не было. Но день прошёл в столкновениях и наездах машин барыг на активистов пробки.

Моторы здесь не заводились, даже для движения в горку, пить воду перестали даже женщины, кто-то набирал воду из местной реки, которая оказалась достаточно чистой. Не было слёз и истерик: к четвёртым суткам погранперехода плакать уже не умели, лишь сурово грустить. Я был знаком с людьми на десятки машин назад и вперёд: отличные ребята со всей европейской части страны, бегущие от правительства и безысходности.

Организовывались сигналы тревоги и группы быстрого реагирования для непропуска барыг. Дальнобойщики, чудом прорвавшиеся в эту зону, грели воду для бытовых нужд: некоторые впервые за эти дни чистили зубы или умывались. На реке появилась лодка: в ней приплыла семья из республики Северного Кавказа, пересекшая таким образом границу: они бесплатно раздавали всем еду, воду и лекарства.

К ночи приехали российские пограничники. Они строго запретили покидать автомобили: таковы были правила проезда. Исключения сделаны были для тех, кто толкает машины. К глубокой темноте мы прошли мост — переход через границу и встали в очередь к казахскому КПП.

Последний этап этого квеста был связан с полным отсутствием сил. Без еды, сна и покоя мы провели по четверо суток. Впереди было ещё шесть часов в пробке. Появился интернет, а вместе с ним в чате пробки появилась информация о местах в машине по 150000 рублей и барыгах, которые перевозили людей в нейтральную буферную зону за одну сумму, а уже в буферной зоне требовали другую. Те, кто не соглашался, отправлялись обратно.

Водители засыпали за рулём: сказывались по 80–96 часов без отдыха. Все были измождены и глубоко несчастны. Казахские пограничники встречали людей сочувствием и крайне лояльным досмотром. Впервые показалось, что мы действительно беженцы: таких несчастных и замученных людей я раньше не видел.

Впереди у нас было ещё шесть часов по бездорожью в степи до ближайшего города.


\newpage
\section{Народ — президент — Бог}


\textit{Зачем «военное духовенство» РПЦ заменяет Евангелие Ветхим Заветом, а \explain{з\'{а}поведь}{commandment} о любви — заповедью об уничтожении врагов }

\textit{Александр Солдатов, обозреватель «Новой газеты»}

\textit{Источник: \url{https://novaya-media.cdn.ampproject.org/}}

Православный федеральный телеканал «Спас» выпустил цикл передач «Война и Библия». Главный редактор телеканала Борис Корчевников в сопровождении настоятеля храма РПЦ при МГИМО протоиерея Игоря Фомина, позируя в полной военной экипировке, продвигают такую трактовку библейских сюжетов, которая, по их мнению, полностью объясняет происходящее. Вышло уже пять серий цикла, основной набор идей повторяется в каждой из них: «СВО» имеет сакрально-мистический характер; на стороне Украины сражаются еретики и сатанисты, а российская армия исполняет заповеди, данные еврейскому народу при исходе из Египта, когда он покорял Палестину — Землю обетованную, истребляя населявшие ее народы.

Мистический крен в российской пропаганде возник в октябре — на фоне кризиса первоначальных целей «спецоперации» («демилитаризация» и «денацификация») и усилился на фоне частичной мобилизации. Хотя религиозность россиян (особенно мужского населения) не очень высока, власть вынуждена обращаться к религиозной риторике — с ее помощью формируется некая синкретическая религия, которая игнорирует различия, например, между православием и исламом, акцентируя внимание на «духовной несовместимости» Запада и России.

Представляя основные постулаты этой религии, помощник Николая Патрушева Алексей Павлов заявил: «С продолжением специальной военной операции становится все более насущным проведение десатанизации Украины, или, как метко выразился глава Чеченской Республики Рамзан Кадыров, ее «полной дешайтанизации».

Развивая этот тренд, Дмитрий Медведев наделил РФ, олицетворяемую президентом, божественными свойствами: «Мы приобрели сакральную силу, — заявил он. — У нас есть возможность отправить всех врагов в геенну огненную». И сформулировал новую цель «СВО»: «Остановить верховного властелина ада, какое бы имя он ни использовал — Сатана, Люцифер или иблис».

\begin{fancyquotes}
    Медведев угрожает украинцам словами из ветхозаветного пророчества Иезекииля: «Не пощадит тебя око Мое, и не помилую. По путям твоим воздам тебе, и мерзости твои с тобою будут; и узнаете, что Я Господь каратель» (Иез. 7:9).
\end{fancyquotes}

В отличие от Евангелия, которое, как утверждает Коран (2:75; 4:46; 5:13, 41), христиане «исказили», Ветхий Завет — священные книги еврейского народа, написанные до пришествия в мир Христа, — равно почитаются христианами и мусульманами. В исламской традиции они известны как «ат-Таурат». В этом контексте неудивительно, что авторы цикла «Война и Библия» апеллируют именно к Ветхому Завету. В 4-й серии они смакуют историю истребления семи народов при завоевании Палестины еврейским воинством под предводительством Иисуса Навина.

Ведущий Борис Корчевников называет заповедь «Пойди и вырежи весь народ» великим испытанием веры. А протоиерей Игорь Фомин ополчается на либералов, которые считают, что «даже правитель не может лишать жизни». «Но Священное Писание, — утверждает воинствующий служитель, — говорит совершенно об обратном».

Подмена тут заключается в том, что христиане имеют другое Писание! Точнее, ключом для понимания Писания у христиан служит Евангелие, а у православных — еще и тысячелетняя святоотеческая традиция его толкования. Иисус Христос в Евангелии часто цитирует заповеди Ветхого Завета, но почти каждый раз предлагает совершенно новое, духовное их понимание. Вспомним Нагорную проповедь: «Сказано древним: не убивай, кто же убьет, подлежит суду. А Я говорю вам, что всякий, гневающийся на брата своего напрасно, подлежит суду… Сказано древним: не прелюбодействуй. А Я говорю вам, что всякий, кто смотрит на женщину с вожделением, уже прелюбодействовал с нею в сердце своем… Сказано: око за око и зуб за зуб. А Я говорю вам: не противься злому. Но кто ударит тебя в правую щеку твою, обрати к нему и другую… Сказано: люби ближнего твоего и ненавидь врага твоего. А Я говорю вам: любите врагов ваших, благословляйте проклинающих вас, благотворите ненавидящим вас и молитесь за обижающих вас и гонящих вас» (Евангелие от Матфея, глава 5).


\begin{fancyquotes}
    Евангелие — самая неудобная книга в современных реалиях.
\end{fancyquotes}

Встречаясь с военным духовенством 1 декабря в храме Христа Спасителя, патриарх Кирилл также не цитировал эту неудобную книгу. Он восхвалял доблесть тех, кто «с оружием в руках защищает родину», и выражал надежду, что поездки духовенства на фронт «закалят» священников. Патриарх намерен «циркулярными письмами» направлять туда все новые и новые партии духовенства. В своей потрясшей христианский мир проповеди 25 сентября он автоматом признал попавшими в рай всех российских воинов, погибших на полях Украины: по его мнению, такая гибель «смывает все грехи, которые человек совершил».

\begin{fancyquotes}
    «Идите смело исполнять свой воинский долг, — напутствовал Кирилл воинов. — Если вы жизнь положили за родину ``\dots'', то вы будете вместе с Богом в Его Царствии».
    Евангелие содержит совсем другие «напутствия воинам».
\end{fancyquotes}

«Возврати меч твой в его место, ибо все, взявшие меч, мечом погибнут» (Мф. 26:52), — говорит Христос апостолу Петру, пытавшемуся защитить Учителя от ареста (об истории толкования этого места Евангелия рассказывала «Новая»). «Наша война не живых делает мертвыми, а мертвых — живыми, изобилуя кротостью и великим смирением, — говорил святитель IV в. Иоанн Златоуст о духовном понимании ветхозаветных сюжетов о войне (\url{https://religion.wikireading.ru/amp190553}). — Мне привычно терпеть преследование, а не преследовать, быть гонимым, а не гнать. Так и Христос побеждал, не распиная, а распятый, не ударяя, но приняв удары».

«Народ — президент — Бог», — торжественно декламирует новую русскую «триаду» протоиерей Игорь Фомин. Она рефреном пронизывает цикл «Война и Библия», апеллируя прямо к подсознанию зрителей, убеждая в том, что все три реальности вечны и бессмысленно возражать или сопротивляться им. Главный посыл этих грубых намеков — президент непогрешим, даже если большинству его подданных непонятен смысл самого важного решения всей его жизни, а может быть, и всей истории России. Народ призывают не искать рациональных объяснений происходящего, а экстатически восклицать вслед за Тертуллианом: «Credo quia absrdum est!» (Верую, ибо абсурдно!).

