\chapter{Карьера и профессии}

\section{Устройство на работу}
Закончив университет и получив диплом юриста, я решил начать искать подходящую по специальности работу.

К сожалению, у меня сразу не получилось занять высокий пост.
Все как в один голос твердили, что у меня недостаточно практики, поэтому сначала н\'{у}жно получить практические знания и поработать несколько месяцев стажером.

Я так и сделал.
Посетив одну фирму в обрабатывающей промышленности, мне очень понравился не только коллектив и месторасположение компании, но и уровень заработной платы и перспектива роста.
Именно поэтому я принял для себя решение получить эту работу, чего бы мне это ни стоило.
Как оказалось, что сделать это было не так-то просто.

Сначала н\'{у}жно было поработать один месяц бесплатно, а потом целых три месяца стажером.
Работая стажером, я получал 30\% от зарплаты.

Эта сумма не была велико, но в то же время мне хватало этих денег на еду, оплату коммунальных услуг и на простые развлечения.

По происшествии 2 месяцев, работодатель заметил мое трудолюбие и талант, поэтому повысил мне зарплату еще на 20\%.

Я был безумно рад этому событию. Прошло 4 месяца и мне предложили полную ставку.
Для этого я заполнил анкету, прошел тест и ознакомился с новыми условиями работы.

К счастью, они были очень выгодными --- премии, надбавки за сверхурочные, оплачиваемый отпуск и оплачиваемые праздничные дни.

Я был на седьмом небе от счастья от такого предложения!

\section{Устройство на работу (2)}
Source: \href{http://irkzan.ru/home/gragd/soiskatel/soiskatelrabota.aspx}{Министерство труда и занятости иркутской области}.

Как начать поиск \explain{подходящего}{suitable} места работы? Как найти интересную работу и одержать победу в конкуренции с другими кандидатами? Для ответа на эти вопросы необходимо помнить:

\begin{itemize}[noitemsep, label=--]
    \item процесс поиска работы Вы должны взять в свои руки, \explain{проявляя}{exhibiting; showing} активность и инициативу;
    \item не ограничивайтесь одной специальностью, составьте список работ, который Вы можете выполнять;
    \item \explain{если Вы определили}{if you have identified} для себя какую работу вы ищете, рассказывайте об этом всем вокруг. Чем больше людей знают об этом, тем лучше;
    \item занимайтесь поиском работы 8 часов в день, считая это свой работой;
    \item работодатели стремятся \explain{нанимать}{to hire} победителей. Будьте уверены в себе, умейте подать себя;
    \item \explain{настройтесь}{tune in (to the fact)} на то, что Вы можете получить десятки \explainDetail{отказов}{отказ}{failure} --- это нормально;
    \item \explain{очередной}{next; following} отказ не должен выбивать Вас из колеи, а наоборот, пробуждать к дальнейшему поиску работы.
\end{itemize}

Современный рынок труда очень специфичен. Каждая сторона --- продавец и покупатель --- старается создать ситуацию выбора для себя: специалист, решивший сменить работу, как правило, \explain{рассматривает}{considers} несколько \explainDetail{предложений}{предложение}{offer} от работодателей, несколько вариантов работы, чтобы выбрать те \explain{условия}{conditions} работы, которые его больше \explain{устраивают}{satisfy} и, одновременно --- продать себя как можно дороже. Работодатель также проводит тщ\'{а}тельный отб\'{о}р кандидатов на рабочее место, чтобы \explain{приобрести}{to acquire} товар как можно лучшего качества и как можно дешевле.

Прежде всего, необходимо установить контакты с рынком труда, поскольку работодатель сам не придет к Вам со своими предложениями. Заставьте информацию работать на себя. Расскажите о том, что Вы ищете работу сослуживцам, родственникам, знакомым, соседям, бывшим одноклассникам и однокурсникам, преподавателям и администраторам учебных заведений, в которых Вы учились и др.

Стоит заглянуть в раздел объявлений в газетах. Прежде всего, обратите внимание на объявления тех фирм, которые указывают свое название, а не выступают обезличенно, скрываясь за абонентным ящиком или ничего не говорящим «организации требуются». Зная, о какой фирме идет речь, Вы можете навести о ней справки и потом решать, стоит ли пытаться попасть в нее. Ни для кого не секрет, что благополучные предприятия не нуждаются в рекламе, и люди, приходят на них сами, чаще всего по рекомендации. Адреса таких предприятий можно взять из специализированных журналов, отраслевой справочной литературы, наконец, просто из телефонных справочников.

Следующим этапом поиска может стать «прозвон» газетных предложений. При общении с работодателем необходимо:

быть вежливым, голос должен быть уверенным;
рядом иметь ручку и листок бумаги для записи необходимой информации;
отвечать на вопросы быстро и кратко, договориться о встрече.

\subsection{Резюме и его роль в трудоустройстве}
Грамотно составленное резюме демонстрирует умение излагать свои мысли на бумаге, умение оценить себя, умение исполнять предстоящие условия и работу. С помощью резюме можно информировать максимальное число работодателей о себе как о претенденте на вакантные рабочие места.

Правила составления резюме:

\begin{itemize}[noitemsep, label=--]
    \item резюме должно быть кратким (не более двух страниц);
    \item резюме должно включать только ту информацию, которая является значимой для работодателя;
    \item старайтесь не использовать сокращений.
\end{itemize}

Следует обязательно указать:

\begin{enumerate}[noitemsep]
    \item \textit{Ф.И.О.}, домашний адрес, контактный телефон. (Резюме, содержащие только адрес электронной почты, обычно не рассматриваются.
    \item \textit{Цель}. Название позиции на которую претендуете. Укажите должность, на которую Вы претендуете. Бывает, что соискатели указывают сразу несколько возможных вариантов трудоустройства, когда варианты имеют близкие функциональные обязанности. Например: «Ищу работу секретаря-референта, офис-менеджера, менеджера по работе с клиентами».
    \item \textit{Образование}. Необходимо указать полностью название учебного заведения, дату поступления и его окончания, специальность. Если высшее образование уже получено, не стоит упоминать о дате окончания средней школы. Если Вы получили дополнительное образование (закончили курсы, прошли тренинги), не перечисляйте все подряд, а только то, что имеет непосредственное отношение к профессии.
    \item \textit{Опыт работы}. Необходимо указать дату поступления и окончания работы, наименование организации, профиль ее деятельности, название должности и краткое описание должностных обязанностей и достижений в хронологическом порядке, начиная с последнего места работы.
    \item \textit{Профессиональные навыки}. В графе указывается знание и степень владения иностранными языками, знание специальных компьютерных программ текстовых редакторов, наличие водительских прав.
    \item \textit{Дополнительная информация}. В графе указывается наличие загранпаспорта, наиболее сильные черты характера и т.д.
    \item Укажите на возможность предоставления рекомендаций.
    \item Вся информация, содержащаяся в резюме, обязательно должна быть достоверной.
\end{enumerate}

\subsection{Пример резюме}
Петров Владимир Петрович

663000 г. Иркутск, ул. Российская,82 кв 16 тел. 33-56-78

\textbf{Цель}: Получение должности коммерческого директора в торговой компании

\textbf{Образование}:

\begin{itemize}[noitemsep, label=--]
    \item 1990-1995 Иркутская государственная экономическая академия. Инженерно-экономический факультет. Диплом инженера-экономиста
    \item 1996-1997 Курсы английского языка при Лингвистическом университете.
    \item 1998 Курсы по маркетингу при учебном центре ИГЭА
\end{itemize}


\textbf{Опыт работы}:

\begin{itemize}[noitemsep, label=--]
    \item 3.1998 --- н/время Фирма «Плюс» (Россия г. Иркутск) начальник отдела продаж.
          Оптовая торговля продовольственными товарами
          (консервы, сухие супы)\\
          Функции: организация продаж, контакты с розничными
          торговыми предприятиями, составление договоров, контроль за
          расчетами. В подчинении 3 человека. За период работы
          расширил сеть торговых точек с 17 до 60.


    \item 8.1995 --- 3.1998 ИЧП «ФОБОС», коммерческий агент.
          Розничная торговля продовольствием и ТНП.\\
          Функции: реализация товара через торговые точки фирмы.
          В 1996 г оборот фирмы достигал 3,5млн. руб. в год
\end{itemize}




\textbf{Дополнительные сведения}:
\begin{itemize}[noitemsep, label=--]
    \item Английский язык (могу изъясняться и работать с профессиональной документацией)
    \item РС --- пользователь (WinWord, Exel). Водительские права кат.В Опыт вождения 4 года. Имеется личный автомобиль.
\end{itemize}




% ----
\subsection{Собеседование с работодателем}
Пришел положительный ответ с предложением явиться на собеседование. Теперь главное --- произвести хорошее впечатление.

\textbf{Подготовка к собеседованию}.

А) Предоставляемые документы

В большой степени решающим для достижения успеха в поисках работы имеют внешний вид предоставляемых документов. Испачканные или порванные документы заставляют читающего их человека предположить, что кандидат в работе также неряшлив и несобран.

Для руководителя отдела кадров, как показывает практика, важнейшим документом, который помогает быстро ознакомиться с личностью кандидата, является автобиография, которая должна быть отпечатана на машинке.

В настоящее время все большее число фирм требует от кандидатов на должность, наряду с другими документами, заполненную личную анкету, которая смогла бы дать ответы на все вопросы, связанные с его первичной оценкой. Специфические условия каждой фирмы заставляют их включать в анкету соответствующие этим условиям вопросы. Фирма придает различное значение тем или иным качествам кандидатов, что и находит отражение в содержании анкеты. Содержание анкеты определяет также постоянно меняющееся положение на рынке труда.

Б) Внешний вид

Оденьтесь так, чтобы Вам было прежде всего удобно и Вы чувствовали бы себя свободно и уверенно, а не как на торжественном приеме у английской королевы. Женщины не должны усердствовать по части косметики и украшений. Не рекомендуется также одевать короткие и узкие юбки и платья и выбирать духи с «навязчивым» ароматом. Тоже самое относится и к мужчинам в отношении лосьона после бритья. Костюм и рубашка должны гармонировать по цвету. Если Ваша работа предполагает наличие у Вас «легкости на подъем», частые поездки и вообще подвижность, можно выбрать для этой встречи спортивный стиль одежды: приличного вида свитер или джемпер с выпущенным воротничком свежей сорочки, спортивного покроя брюки или джинсы, легкая обувь. Не следует путать спортивный стиль со спортивной одеждой. И в Москве, и в Иркутске нередко можно встретить молодых людей, носящих «мастерку» с пиджаком, спортивный костюм в комбинации с рубашкой, галстуком и кожаными туфлями. Нелепо будет выглядеть мужчина, пришедший устраиваться в спортивном костюме даже фирмы «Адидас». Мнение, что цена костюма придает ему солидность, глубоко ошибочно.

До собеседования:
\begin{enumerate}[noitemsep]
    \item проверьте время, дату и путь;
    \item исследуйте компанию;
    \item отрепетируйте вопросы и ответы;
    \item подумайте, что вы оденете и как будете выглядеть
\end{enumerate}



% ----------
\subsection{Пять первых критических минут}
Много кандидатур на разные работы отвергают в течение первых пяти минут собеседования.

Критический момент наступает, когда Вы входите --- в Вашем внешнем виде не должно быть ничего, что может вызвать разочарование.

Лучший первоначальный подход --- это улыбнуться. Это неизменно побуждает дружелюбные чувства в человеке, улыбка дает нам почувствовать себя намного лучше и более уверенно.

Другие полезные подсказки, которые следует использовать в первые пять минут:

не выкладывайте ничего, что принесли с собой до того, как собеседник предложит Вам сделать это;
предоставьте собеседнику возможность первому протянуть Вам руку для рукопожатия;
не садитесь, пока Вам не предложат.


\textbf{Поза.}
Устройтесь удобно, сядьте прямо, но без \explain{напряжения}{напряжение}{stress; tension; voltage}.
Не облокачивайтесь и не кладите руки на стол собеседника.
Не разваливайтесь на стуле.
Вы будете выглядеть куда более представительно, сидя прямо, нога на ногу, ваши руки расслабленно лежат на коленях. Неплохо убедиться, что ваш стул отодвинут от стола собеседника, чтобы дать Вам свободу движений.


\subsection{Получение информации}
Собеседование проводится для того, чтобы обе стороны давали и получали информацию. Одна из главных установок --- получить всю нужную вам информацию о работе и самой организации. Никогда не соглашайтесь на работу, пока не убедитесь, что она Вам подходит.

Не надо:
\begin{enumerate}[noitemsep]
    \item извиняться за свой возраст, здоровье, недостаток опыта;
    \item перебивать собеседника;
    \item критиковать последнего работодателя;
    \item быть слишком фамильярным или самоуверенным;
    \item шутить, ругаться или курить.
    \item Типичные вопросы работодателей при приеме на работу
\end{enumerate}


Почему Вы хотите здесь работать?
\begin{enumerate}[noitemsep]
    \item Выполняли ли вы работу такого рода раньше?
    \item Что Вы делали с тех пор, как стали безработным?
    \item Почему Вы ушли с последнего места работы?
    \item Почему Вы так долго оставались без работы?
    \item Как долго вы намерены работать у нас?
    \item Чем Вы занимались на последнем месте работы?
    \item На каком оборудовании Вы работали?
    \item В чем заключаются Ваши сильные стороны?
    \item Каковы Ваши слабые стороны?
    \item Расскажите нам побольше о себе?
    \item Какую зарплату Вы хотели бы получать?
    \item Были ли у вас конфликтные ситуации на работе?
    \item Когда вы сможете приступить к работе?
    \item Каким образом Вы планируете добираться до работы вовремя?
    \item Есть ли у Вас какие-либо вопросы?
\end{enumerate}

После собеседования:
\begin{enumerate}[noitemsep]
    \item поблагодарите компанию за собеседование в кратком письме;
    \item если от работодателя не будет вестей, то позвоните и спросите, каков результат собеседования.
\end{enumerate}

Помните! В каждом из Вас есть внутренние резервы. Используйте их. Разбудите свою активность --- и успех будет в Ваших руках.

Желаем скорейшего трудоустройства!


\section{Виртуальный ассистент: профессия будущего}
Source: \href{http://www.cyprusmoms.com/virtualnyj-assistent-professiya-budushchego/}{www.cyprusmoms.com}.\\

Представьте ситуацию: вы живёте в стране, в которой не имеете возможности работать. Дети подрастают, хочется реализоваться профессионально, но идей для собственного бизнеса нет, да и времени свободного всего час-два в день. Знакомо?

Думаю, что я не одинока в таком положении. Живу на Кипре уже давн\'{о}, но разрешения на работу нет, да и найти работу в нашей деревне довольно сложно. Поэтому я начала думать об \explainDetail{удалённой}{удалённый/-ая}{remote} работе через интернет --- \explainDetail{вела}{вести/повести (веду, ведёшь, ведут)}{to lead} собственный блог, администрировала несколько групп в Фейсбуке для поддержки местного комьюнити, писала статьи, фотографировала, но не понимала, как перевести это хобби в \explainDetail{опл\'{а}чиваемую}{опл\'{а}чиваемый}{paid (from: оплачивать/оплатить)} деятельность.

Да, я читала рекламные статьи школ, которые онлайн обучают различным профессиям, но мне было непонятно, как потом находить работу, как работать с клиентами, чем \explain{зацепить}{to hook on; to catch (цепь: chain)} клиента, когда на рынке множество таких \explainDetail{новичков}{новичок}{newbie}, как я...

И вот, когда некоторое время назад я прочитала статью о профессии «Виртуальный ассистент», во мне щёлкнуло --- вот оно! То дело, которое я искала.

Кто такой виртуальный ассистент? Это универсальный специалист, который помогает предпринимателю вести бизнес в интернете --- наполняет страницы в социальных сетях, верстает лендинги и презентации, организовывает вебинары и налаживает \explainDetail{почтовую рассылку}{почтовая рассылка}{mailing list}.

В зависимости от предыдущего опыта, ассистент может специализироваться в том или ином направлении. Но в целом, это человек, который хорошо ориентируется в интернете, может найти нужный сервис, написать запрос, проконтролировать подрядчиков и быть тем многоруким многостаночником, который снимет с предпринимателя рутинные обязанности. И не важно, в какой стране живёт предприниматель и какое гражданство имеет Виртуальный ассистент --- они встречаются и сотрудничают в интернете.

{\it К 2020 году 20\% рабочих мест в России будут виртуальными, сказано в исследовании «J’son \& Partners Consulting», сделанном по заказу сервиса «Битрикс24». По данным исследований 2016 года эта цифра в Европе составляет 17\%, а в Японии и США доходит до 40\% от всех работающих.}

Я погуглила и поняла, что в англоязычной среде эта профессия очень распространена, даже существуют ассоциации бизнес-помощников.
На русскоязычном пространстве информации меньше, но есть несколько школ подготовки виртуальных ассистентов.
И все они --- что очень ценно --- обещают помощь со стажировками и трудоустройством. Результаты исследования школ, которые готовят бизнес-помощников, вы можете посмотреть на моей странице в фейсбуке Я остановилась на Международной школе подготовки бизнес-ассистентов и интернет-маркетологов «Helppy» Ольги Шевченко и ни минуты не пожалела. И организация обучения, и полезность информации --- на высоте!

Обучение длится пять недель, и погружение в учебу полное. За 35 дней ты вникаешь в принципы организации интернет-бизнеса, верстаешь презентацию в Пауэр Поинт, составляешь контент-план и график постов в рамках тобой же разработанной стратегии продвижения в соцсетях, верстаешь лендинг --- одностраничный сайт, а так же --- ТА-ДАМ! --- составляешь портфолио для самого себя. Понятно, что это не все темы, а только те, по которым требовалось сдать домашнее задание, и над которым мы все корпели ночами. На все домашки ты получаешь развёрнутые ответы, и очень редко удавалось сдать их с первого раза, спрашивали очень строго --- то типографика хромает, то дизайн подвёл.

Кроме лекций и обучающего материала, в закрытом разделе собрана база данных полезных статей, в секретной группе кипит жизнь --- кураторы и сокурсники обсуждают задания, а по пятницам Ольга Шевченко разговаривает с каждым курсистом отдельно и отвечает на все волнующие вопросы.

Создание портфолио --- это огромный пендаль собственной самооценке. После того, как ты соберешь всё, что ты можешь, применишь правила сильного текста, прикрутишь туда отзывы клиентов (мы делали домашнее задание по заказу интернет-предпринимателей, а они нам писали отзывы), подберёшь шрифты, а потом ещё сверстаешь это в лендинг с красивыми картинками... От этогосамооценка лезет вверх, и ты готова на подвиги и новые свершения. Хотите посмотреть на моё портфолио?Здесь. Я вам его показываю не для того, чтобы похвастаться, а для того, чтобы показать, что у вас может получиться на выходе.

Хорошо, что я начала учиться заранее, поэтому даже успевала спать и вести клиентов, которых нашла тут же, рядом с собой. Дело в том, что, начиная заниматься и вникать в тему, у тебя обостряется зрение и ты видишь, что в этом проекте, например, ты можешь быть полезной, а здесь ты можешь докрутить страницу и получить совсем другие результаты. Ты предлагаешь свои услуги, показываешь, что можешь сделать, как можешь помочь, и люди откликаются. Я и многие сокурсники именно так получили работу.

Я не обещаю лёгкой жизни --- учиться и работать надо будет много, информация в интернете меняется быстро, у Фейсбука, например, нововведения каждую неделю, ежедневно на рынок труда выходит всё больше людей. Надо будет выстраивать свой график работы и думать о тайм-менеджменте. Например, черновик этой статьи я набирала на телефоне в гугл-кипе в то время, пока мастер педикюра работала с моими ногами. А от одного, очень перспективного предложения пришлось отказаться, потому что я понимала --- или работа, или семья, третьего не дано, со всем в данный момент не справлюсь.

Профессия виртуального ассистента --- хорошая ступенька для тех, кто выходит из декрета и живёт в том месте, где устроиться на работу сложно. Это профессия для того, кто умеет работать с большим количеством информации и в состоянии организовать рабочие процессы и самого себя. А дальше можно покорять новые вершины, и истории выпускников школы тому подтверждение.

Удачи!