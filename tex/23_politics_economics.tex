\chapter{Политика и экономика}

\section{Восемь богачей мира владеют половиной богатств Земли}
Всего восемь \explain{толстосумов}{толстосум: moneybag} \explain{владеют}{владеть/завладеть: to possess (влад\'{е}ю, влад\'{е}ешь, влад\'{е}ют)} тем же состоянием, что принадлежит беднейшей половине населения Земли. К такому выводу пришли исследователи благотворительной организацией Oxfam. Более того, исследование показало, что уже в ближайшие 25 лет в мире может появиться первый триллионер. Чтобы потратить такое состояние, н\'{у}жно ежедневно в течение 27 столетий и еще 38 лет расходовать по одному миллиону.

Восемь миллиардеров владеют тем же состоянием, которое находится в руках 3 миллиардов 600 миллионов человек, отмечает РИА Новости.

Великолепную восьмерку возглавляет основатель Microsoft Билл Гейтс. Его состояние оценивается в 75 миллиардов долларов. У Амансио Ортеги 67 миллиардов. Экс-президента Inditex женщины знают по марке магазинов одежды Zara. Третью ступень пьедестала богатства занимает американский \explain{предприниматель}{business-person; entrepreneur} Уоррен Баффетт с состоянием почти 61 миллиард долларов.

Оставшаяся пятёрка толстосумов выглядит так: мексиканский бизнесмен Карлос Слим Элу --- 50 миллиардов долларов, глава компании по электронной торговле Amazon Джефф Безос --- чуть более 45 миллиардов долларов, основатель соцсети Facebook Марк Цукерберг --- чуть менее 45 миллиардов долларов, глава корпорации Oracle Ларри Эллисон с состоянием 43 миллиарда и 600 миллионов долларов и владелец агентства деловой информации Майкл Блумберг с 40 миллиардами.
В 2010 году считалось, что стоимость имущества, которым владеют половина беднейших жителей Земли, равна совокупному состоянию 43 богатейших людей планеты. Так что за последние шесть лет ситуация лишь усугубилась.

``Это возмутительно, что такие средства сосредоточены в руках всего нескольких человек, когда каждый десятый в мире вынужден выживать на менее чем два доллара в день'', --- отметил исполнительный директор Oxfam Уинни Бьянима.

По его мнению, неравенство способствует тому, что сотни миллионов человек находятся в ловушке бедности.

``Это разрушает наши общества и подрывает демократию'', --- настаивает Бьянима.

Сегодня семь из десяти человек живут в странах, где за последние три десятилетия неравенство в доходах возросло. При этом всего один процент населения планеты владеет таким же богатством, как 99 процентов тех, кто не попал в золотой процент.

Кроме того, чтобы труд женщин в мире оплачивался так же, как и мужской, понадобится еще порядка 170 лет, отмечают в Oxfam.

Oxfam (Oxford Committee for Famine Relief) основана в Оксфорде в 1942 году для помощи голодающим. Ее цель --- решение проблем бедности и связанной с ней несправедливости во всем мире. Международное объединение сейчас объединяет 17 организаций в более чем 90 странах мира.


