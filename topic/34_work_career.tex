% \chapter{Работа, Карьера и Профессии}

\section{Устройство на работу}
Закончив университет и получив диплом юриста, я решил начать искать подходящую по специальности работу.

К сожалению, у меня сразу не получилось занять высокий пост.
Все как в один голос твердили, что у меня недостаточно практики, поэтому сначала н\'{у}жно получить практические знания и поработать несколько месяцев стажером.

Я так и сделал.
Посетив одну фирму в обрабатывающей промышленности, мне очень понравился не только коллектив и месторасположение компании, но и уровень заработной платы и перспектива роста.
Именно поэтому я принял для себя решение получить эту работу, чего бы мне это ни стоило.
Как оказалось, что сделать это было не так-то просто.

Сначала н\'{у}жно было поработать один месяц бесплатно, а потом целых три месяца стажером.
Работая стажером, я получал 30\% от зарплаты.

Эта сумма не была велико, но в то же время мне хватало этих денег на еду, оплату коммунальных услуг и на простые развлечения.

По происшествии 2 месяцев, работодатель заметил мое трудолюбие и талант, поэтому повысил мне зарплату еще на 20\%.

Я был безумно рад этому событию. Прошло 4 месяца и мне предложили полную ставку.
Для этого я заполнил анкету, прошел тест и ознакомился с новыми условиями работы.

К счастью, они были очень выгодными --- премии, надбавки за сверхурочные, оплачиваемый отпуск и оплачиваемые праздничные дни.

Я был на седьмом небе от счастья от такого предложения!

\section{Устройство на работу (2)}
Source: \href{http://irkzan.ru/home/gragd/soiskatel/soiskatelrabota.aspx}{Министерство труда и занятости иркутской области}.

Как начать поиск \explain{подходящего}{suitable} места работы? Как найти интересную работу и одержать победу в конкуренции с другими кандидатами? Для ответа на эти вопросы необходимо помнить:

\begin{itemize}[noitemsep, label=--]
    \item процесс поиска работы Вы должны взять в свои руки, \explain{проявляя}{exhibiting; showing} активность и инициативу;
    \item не ограничивайтесь одной специальностью, составьте список работ, который Вы можете выполнять;
    \item \explain{если Вы определили}{if you have identified} для себя какую работу вы ищете, рассказывайте об этом всем вокруг. Чем больше людей знают об этом, тем лучше;
    \item занимайтесь поиском работы 8 часов в день, считая это свой работой;
    \item работодатели стремятся \explain{нанимать}{to hire} победителей. Будьте уверены в себе, умейте подать себя;
    \item \explain{настройтесь}{tune in (to the fact)} на то, что Вы можете получить десятки \explainDetail{отказов}{отказ}{failure} --- это нормально;
    \item \explain{очередной}{next; following} отказ не должен выбивать Вас из колеи, а наоборот, пробуждать к дальнейшему поиску работы.
\end{itemize}

Современный рынок труда очень специфичен. Каждая сторона --- продавец и покупатель --- старается создать ситуацию выбора для себя: специалист, решивший сменить работу, как правило, \explain{рассматривает}{considers} несколько \explainDetail{предложений}{предложение}{offer} от работодателей, несколько вариантов работы, чтобы выбрать те \explain{условия}{conditions} работы, которые его больше \explain{устраивают}{satisfy} и, одновременно --- продать себя как можно дороже. Работодатель также проводит тщ\'{а}тельный отб\'{о}р кандидатов на рабочее место, чтобы \explain{приобрести}{to acquire} товар как можно лучшего качества и как можно дешевле.

Прежде всего, необходимо установить контакты с рынком труда, поскольку работодатель сам не придет к Вам со своими предложениями. Заставьте информацию работать на себя. Расскажите о том, что Вы ищете работу сослуживцам, родственникам, знакомым, соседям, бывшим одноклассникам и однокурсникам, преподавателям и администраторам учебных заведений, в которых Вы учились и др.

Стоит заглянуть в раздел объявлений в газетах. Прежде всего, обратите внимание на объявления тех фирм, которые указывают свое название, а не выступают обезличенно, скрываясь за абонентным ящиком или ничего не говорящим «организации требуются». Зная, о какой фирме идет речь, Вы можете навести о ней справки и потом решать, стоит ли пытаться попасть в нее. Ни для кого не секрет, что благополучные предприятия не нуждаются в рекламе, и люди, приходят на них сами, чаще всего по рекомендации. Адреса таких предприятий можно взять из специализированных журналов, отраслевой справочной литературы, наконец, просто из телефонных справочников.

Следующим этапом поиска может стать «прозвон» газетных предложений. При общении с работодателем необходимо:

быть вежливым, голос должен быть уверенным;
рядом иметь ручку и листок бумаги для записи необходимой информации;
отвечать на вопросы быстро и кратко, договориться о встрече.

\subsection{Резюме и его роль в трудоустройстве}
Грамотно составленное резюме демонстрирует умение излагать свои мысли на бумаге, умение оценить себя, умение исполнять предстоящие условия и работу. С помощью резюме можно информировать максимальное число работодателей о себе как о претенденте на вакантные рабочие места.

Правила составления резюме:

\begin{itemize}[noitemsep, label=--]
    \item резюме должно быть кратким (не более двух страниц);
    \item резюме должно включать только ту информацию, которая является значимой для работодателя;
    \item старайтесь не использовать сокращений.
\end{itemize}

Следует обязательно указать:

\begin{enumerate}[noitemsep]
    \item \textit{Ф.И.О.}, домашний адрес, контактный телефон. (Резюме, содержащие только адрес электронной почты, обычно не рассматриваются.
    \item \textit{Цель}. Название позиции на которую претендуете. Укажите должность, на которую Вы претендуете. Бывает, что соискатели указывают сразу несколько возможных вариантов трудоустройства, когда варианты имеют близкие функциональные обязанности. Например: «Ищу работу секретаря-референта, офис-менеджера, менеджера по работе с клиентами».
    \item \textit{Образование}. Необходимо указать полностью название учебного заведения, дату поступления и его окончания, специальность. Если высшее образование уже получено, не стоит упоминать о дате окончания средней школы. Если Вы получили дополнительное образование (закончили курсы, прошли тренинги), не перечисляйте все подряд, а только то, что имеет непосредственное отношение к профессии.
    \item \textit{Опыт работы}. Необходимо указать дату поступления и окончания работы, наименование организации, профиль ее деятельности, название должности и краткое описание должностных обязанностей и достижений в хронологическом порядке, начиная с последнего места работы.
    \item \textit{Профессиональные навыки}. В графе указывается знание и степень владения иностранными языками, знание специальных компьютерных программ текстовых редакторов, наличие водительских прав.
    \item \textit{Дополнительная информация}. В графе указывается наличие загранпаспорта, наиболее сильные черты характера и т.д.
    \item Укажите на возможность предоставления рекомендаций.
    \item Вся информация, содержащаяся в резюме, обязательно должна быть достоверной.
\end{enumerate}

\subsection{Пример резюме}
Петров Владимир Петрович

663000 г. Иркутск, ул. Российская,82 кв 16 тел. 33-56-78

\textbf{Цель}: Получение должности коммерческого директора в торговой компании

\textbf{Образование}:

\begin{itemize}[noitemsep, label=--]
    \item 1990-1995 Иркутская государственная экономическая академия. Инженерно-экономический факультет. Диплом инженера-экономиста
    \item 1996-1997 Курсы английского языка при Лингвистическом университете.
    \item 1998 Курсы по маркетингу при учебном центре ИГЭА
\end{itemize}


\textbf{Опыт работы}:

\begin{itemize}[noitemsep, label=--]
    \item 3.1998 --- н/время Фирма «Плюс» (Россия г. Иркутск) начальник отдела продаж.
          Оптовая торговля продовольственными товарами
          (консервы, сухие супы)\\
          Функции: организация продаж, контакты с розничными
          торговыми предприятиями, составление договоров, контроль за
          расчетами. В подчинении 3 человека. За период работы
          расширил сеть торговых точек с 17 до 60.


    \item 8.1995 --- 3.1998 ИЧП «ФОБОС», коммерческий агент.
          Розничная торговля продовольствием и ТНП.\\
          Функции: реализация товара через торговые точки фирмы.
          В 1996 г оборот фирмы достигал 3,5млн. руб. в год
\end{itemize}




\textbf{Дополнительные сведения}:
\begin{itemize}[noitemsep, label=--]
    \item Английский язык (могу изъясняться и работать с профессиональной документацией)
    \item РС --- пользователь (WinWord, Exel). Водительские права кат.В Опыт вождения 4 года. Имеется личный автомобиль.
\end{itemize}




% ----
\subsection{Собеседование с работодателем}
Пришел положительный ответ с предложением явиться на собеседование. Теперь главное --- произвести хорошее впечатление.

\textbf{Подготовка к собеседованию}.

А) Предоставляемые документы

В большой степени решающим для достижения успеха в поисках работы имеют внешний вид предоставляемых документов. Испачканные или порванные документы заставляют читающего их человека предположить, что кандидат в работе также неряшлив и несобран.

Для руководителя отдела кадров, как показывает практика, важнейшим документом, который помогает быстро ознакомиться с личностью кандидата, является автобиография, которая должна быть отпечатана на машинке.

В настоящее время все большее число фирм требует от кандидатов на должность, наряду с другими документами, заполненную личную анкету, которая смогла бы дать ответы на все вопросы, связанные с его первичной оценкой. Специфические условия каждой фирмы заставляют их включать в анкету соответствующие этим условиям вопросы. Фирма придает различное значение тем или иным качествам кандидатов, что и находит отражение в содержании анкеты. Содержание анкеты определяет также постоянно меняющееся положение на рынке труда.

Б) Внешний вид

Оденьтесь так, чтобы Вам было прежде всего удобно и Вы чувствовали бы себя свободно и уверенно, а не как на торжественном приеме у английской королевы. Женщины не должны усердствовать по части косметики и украшений. Не рекомендуется также одевать короткие и узкие юбки и платья и выбирать духи с «навязчивым» ароматом. Тоже самое относится и к мужчинам в отношении лосьона после бритья. Костюм и рубашка должны гармонировать по цвету. Если Ваша работа предполагает наличие у Вас «легкости на подъем», частые поездки и вообще подвижность, можно выбрать для этой встречи спортивный стиль одежды: приличного вида свитер или джемпер с выпущенным воротничком свежей сорочки, спортивного покроя брюки или джинсы, легкая обувь. Не следует путать спортивный стиль со спортивной одеждой. И в Москве, и в Иркутске нередко можно встретить молодых людей, носящих «мастерку» с пиджаком, спортивный костюм в комбинации с рубашкой, галстуком и кожаными туфлями. Нелепо будет выглядеть мужчина, пришедший устраиваться в спортивном костюме даже фирмы «Адидас». Мнение, что цена костюма придает ему солидность, глубоко ошибочно.

До собеседования:
\begin{enumerate}[noitemsep]
    \item проверьте время, дату и путь;
    \item исследуйте компанию;
    \item отрепетируйте вопросы и ответы;
    \item подумайте, что вы оденете и как будете выглядеть
\end{enumerate}



% ----------
\subsection{Пять первых критических минут}
Много кандидатур на разные работы отвергают в течение первых пяти минут собеседования.

Критический момент наступает, когда Вы входите --- в Вашем внешнем виде не должно быть ничего, что может вызвать разочарование.

Лучший первоначальный подход --- это улыбнуться. Это неизменно побуждает дружелюбные чувства в человеке, улыбка дает нам почувствовать себя намного лучше и более уверенно.

Другие полезные подсказки, которые следует использовать в первые пять минут:

не выкладывайте ничего, что принесли с собой до того, как собеседник предложит Вам сделать это;
предоставьте собеседнику возможность первому протянуть Вам руку для рукопожатия;
не садитесь, пока Вам не предложат.


\textbf{Поза.}
Устройтесь удобно, сядьте прямо, но без \explain{напряжения}{напряжение}{stress; tension; voltage}.
Не облокачивайтесь и не кладите руки на стол собеседника.
Не разваливайтесь на стуле.
Вы будете выглядеть куда более представительно, сидя прямо, нога на ногу, ваши руки расслабленно лежат на коленях. Неплохо убедиться, что ваш стул отодвинут от стола собеседника, чтобы дать Вам свободу движений.


\subsection{Получение информации}
Собеседование проводится для того, чтобы обе стороны давали и получали информацию. Одна из главных установок --- получить всю нужную вам информацию о работе и самой организации. Никогда не соглашайтесь на работу, пока не убедитесь, что она Вам подходит.

Не надо:
\begin{enumerate}[noitemsep]
    \item извиняться за свой возраст, здоровье, недостаток опыта;
    \item перебивать собеседника;
    \item критиковать последнего работодателя;
    \item быть слишком фамильярным или самоуверенным;
    \item шутить, ругаться или курить.
    \item Типичные вопросы работодателей при приеме на работу
\end{enumerate}


Почему Вы хотите здесь работать?
\begin{enumerate}[noitemsep]
    \item Выполняли ли вы работу такого рода раньше?
    \item Что Вы делали с тех пор, как стали безработным?
    \item Почему Вы ушли с последнего места работы?
    \item Почему Вы так долго оставались без работы?
    \item Как долго вы намерены работать у нас?
    \item Чем Вы занимались на последнем месте работы?
    \item На каком оборудовании Вы работали?
    \item В чем заключаются Ваши сильные стороны?
    \item Каковы Ваши слабые стороны?
    \item Расскажите нам побольше о себе?
    \item Какую зарплату Вы хотели бы получать?
    \item Были ли у вас конфликтные ситуации на работе?
    \item Когда вы сможете приступить к работе?
    \item Каким образом Вы планируете добираться до работы вовремя?
    \item Есть ли у Вас какие-либо вопросы?
\end{enumerate}

После собеседования:
\begin{enumerate}[noitemsep]
    \item поблагодарите компанию за собеседование в кратком письме;
    \item если от работодателя не будет вестей, то позвоните и спросите, каков результат собеседования.
\end{enumerate}

Помните! В каждом из Вас есть внутренние резервы. Используйте их. Разбудите свою активность --- и успех будет в Ваших руках.

Желаем скорейшего трудоустройства!


\section{Виртуальный ассистент: профессия будущего}
Source: \href{http://www.cyprusmoms.com/virtualnyj-assistent-professiya-budushchego/}{www.cyprusmoms.com}.\\

Представьте ситуацию: вы живёте в стране, в которой не имеете возможности работать. Дети подрастают, хочется реализоваться профессионально, но идей для собственного бизнеса нет, да и времени свободного всего час-два в день. Знакомо?

Думаю, что я не одинока в таком положении. Живу на Кипре уже давн\'{о}, но разрешения на работу нет, да и найти работу в нашей деревне довольно сложно. Поэтому я начала думать об \explainDetail{удалённой}{удалённый/-ая}{remote} работе через интернет --- \explainDetail{вела}{вести/повести (веду, ведёшь, ведут)}{to lead} собственный блог, администрировала несколько групп в Фейсбуке для поддержки местного комьюнити, писала статьи, фотографировала, но не понимала, как перевести это хобби в \explainDetail{опл\'{а}чиваемую}{опл\'{а}чиваемый}{paid (from: оплачивать/оплатить)} деятельность.

Да, я читала рекламные статьи школ, которые онлайн обучают различным профессиям, но мне было непонятно, как потом находить работу, как работать с клиентами, чем \explain{зацепить}{to hook on; to catch (цепь: chain)} клиента, когда на рынке множество таких \explainDetail{новичков}{новичок}{newbie}, как я...

И вот, когда некоторое время назад я прочитала статью о профессии «Виртуальный ассистент», во мне щёлкнуло --- вот оно! То дело, которое я искала.

Кто такой виртуальный ассистент? Это универсальный специалист, который помогает предпринимателю вести бизнес в интернете --- наполняет страницы в социальных сетях, верстает лендинги и презентации, организовывает вебинары и налаживает \explainDetail{почтовую рассылку}{почтовая рассылка}{mailing list}.

В зависимости от предыдущего опыта, ассистент может специализироваться в том или ином направлении. Но в целом, это человек, который хорошо ориентируется в интернете, может найти нужный сервис, написать запрос, проконтролировать подрядчиков и быть тем многоруким многостаночником, который снимет с предпринимателя рутинные обязанности. И не важно, в какой стране живёт предприниматель и какое гражданство имеет Виртуальный ассистент --- они встречаются и сотрудничают в интернете.

{\it К 2020 году 20\% рабочих мест в России будут виртуальными, сказано в исследовании «J’son \& Partners Consulting», сделанном по заказу сервиса «Битрикс24». По данным исследований 2016 года эта цифра в Европе составляет 17\%, а в Японии и США доходит до 40\% от всех работающих.}

Я погуглила и поняла, что в англоязычной среде эта профессия очень распространена, даже существуют ассоциации бизнес-помощников.
На русскоязычном пространстве информации меньше, но есть несколько школ подготовки виртуальных ассистентов.
И все они --- что очень ценно --- обещают помощь со стажировками и трудоустройством. Результаты исследования школ, которые готовят бизнес-помощников, вы можете посмотреть на моей странице в фейсбуке Я остановилась на Международной школе подготовки бизнес-ассистентов и интернет-маркетологов «Helppy» Ольги Шевченко и ни минуты не пожалела. И организация обучения, и полезность информации --- на высоте!

Обучение длится пять недель, и погружение в учебу полное. За 35 дней ты вникаешь в принципы организации интернет-бизнеса, верстаешь презентацию в Пауэр Поинт, составляешь контент-план и график постов в рамках тобой же разработанной стратегии продвижения в соцсетях, верстаешь лендинг --- одностраничный сайт, а так же --- ТА-ДАМ! --- составляешь портфолио для самого себя. Понятно, что это не все темы, а только те, по которым требовалось сдать домашнее задание, и над которым мы все корпели ночами. На все домашки ты получаешь развёрнутые ответы, и очень редко удавалось сдать их с первого раза, спрашивали очень строго --- то типографика хромает, то дизайн подвёл.

Кроме лекций и обучающего материала, в закрытом разделе собрана база данных полезных статей, в секретной группе кипит жизнь --- кураторы и сокурсники обсуждают задания, а по пятницам Ольга Шевченко разговаривает с каждым курсистом отдельно и отвечает на все волнующие вопросы.

Создание портфолио --- это огромный пендаль собственной самооценке. После того, как ты соберешь всё, что ты можешь, применишь правила сильного текста, прикрутишь туда отзывы клиентов (мы делали домашнее задание по заказу интернет-предпринимателей, а они нам писали отзывы), подберёшь шрифты, а потом ещё сверстаешь это в лендинг с красивыми картинками... От этогосамооценка лезет вверх, и ты готова на подвиги и новые свершения. Хотите посмотреть на моё портфолио?Здесь. Я вам его показываю не для того, чтобы похвастаться, а для того, чтобы показать, что у вас может получиться на выходе.

Хорошо, что я начала учиться заранее, поэтому даже успевала спать и вести клиентов, которых нашла тут же, рядом с собой. Дело в том, что, начиная заниматься и вникать в тему, у тебя обостряется зрение и ты видишь, что в этом проекте, например, ты можешь быть полезной, а здесь ты можешь докрутить страницу и получить совсем другие результаты. Ты предлагаешь свои услуги, показываешь, что можешь сделать, как можешь помочь, и люди откликаются. Я и многие сокурсники именно так получили работу.

Я не обещаю лёгкой жизни --- учиться и работать надо будет много, информация в интернете меняется быстро, у Фейсбука, например, нововведения каждую неделю, ежедневно на рынок труда выходит всё больше людей. Надо будет выстраивать свой график работы и думать о тайм-менеджменте. Например, черновик этой статьи я набирала на телефоне в гугл-кипе в то время, пока мастер педикюра работала с моими ногами. А от одного, очень перспективного предложения пришлось отказаться, потому что я понимала --- или работа, или семья, третьего не дано, со всем в данный момент не справлюсь.

Профессия виртуального ассистента --- хорошая ступенька для тех, кто выходит из декрета и живёт в том месте, где устроиться на работу сложно. Это профессия для того, кто умеет работать с большим количеством информации и в состоянии организовать рабочие процессы и самого себя. А дальше можно покорять новые вершины, и истории выпускников школы тому подтверждение.

Удачи!

\section{Правила деловой переписки}

\textit{Источник: \url{https://4brain.ru/blog/pravila-delovoj-perepiski/}}

В информационном веке важно обладать умениями и навыками общения в сети Интернет. С одной стороны, письменная речь основывается на тех же правилах, что и устная, с другой, в отдельных направлениях есть необходимость использовать особые приемы, чтобы заочно, не видя собеседника, выстроить с ним правильную коммуникацию и быстрее прийти к нужному результату, используя минимум слов и писем.

Управленцам, менеджерам, редакторам, маркетологам – правила деловой переписки необходимо знать всем. Заинтересовать собеседника, получить результат от взаимодействия с ним можно только при правильном подходе. Стоит учитывать, что при удаленном контакте, когда собеседники не видят друг друга, отсутствует визуальный контакт и возможность использовать множество слов.

В данном случае краткость-сестра таланта – это одно из главных правил продуктивной переписки. Не всегда люди имеют возможность читать длинные письма, поэтому мысли нужно излагать максимально точно и лаконично.

В деловой электронной переписке есть общепринятые правила, грамотно используя которые можно не переживать, что письмо отправится в «Спам». Они подробно изложены в книге создателя сервиса «Главред» Максима Ильяхова «Новые правила деловой переписки», которая считается лучшим пособием в своем роде в России. Соавтор Людмила Сарычева – автор статей о деловом общении.

Опираясь на книгу, мы расскажем о том, как освоить правила деловой переписки, в чем они заключаются и выясним, зачем они вообще нужны, если можно просто излагать свои мысли, чтобы удаленно изъясняться с собеседником.

А чтобы научиться лучше взаимодействовать с людьми и подтянуть или развить навыки общения, рекомендуем пройти нашу программу «Лучшие техники коммуникации». Полученные знания будут полезны в письменной, устной речи, при взаимодействии с коллегами, партнерами и т.д.

\textbf{Типичные ошибки авторов}

Казалось бы, что сложного в том, чтобы написать письмо и, как кажется, договориться с собеседником на расстоянии о чем угодно. В этом как раз кроются типичные ошибки обывателей, чьи письма часто отправляются в «Спам».

На основе материалов, изложенных в книге Максима Ильяхова «Правила деловой переписки», разберем основные проблемы писем, которые буквально лежат на поверхности и видны сразу [М. Ильяхов, 2018]:
\begin{enumerate}
    \item В письме нет темы. Предположим, у адресата большая нагрузка на работе и просматривать почту оперативно он не может. За день на его ящик придет, например, двадцать писем. Каждое нужно прочитать, обработать. Представляете, сколько времени потребуется на чтение? Соответственно заголовок письма обозначит, что внутри, а значит, послужит своеобразным маяком для читающего. Идеальное решение – сделать такой заголовок, который полностью отражает смысл письма. В книге также дана рекомендация помещать в теме такие подробности, которые обозначат, стоит ли читать письмо сразу, какие действия необходимо предпринять для решения вопроса.

    \item «Уважаемые коллеги» --- типовой и очень распространенный шаблон, который, как оказывается, вызывает непонимание и не стимулирует концентрацию силы, а оказывает обратный эффект. Он звучит как «Коллеги, разберитесь сами, кто это сделает», а также не вызывает уважения из-за отсутствия адресата в обращении. Если нужно что-то сделать, решить вопрос, лучше написать конкретному человеку и дать ему подробную информацию, нужные вводные, а не делать коллективную рассылку. Такой подход гораздо эффективнее.

    \item Вторжение в свободное время человека без извинений. «Как проходит выходной / больничный / отпуск?» – эти вопросы точно вызовут раздражение, ведь человек вне рабочего времени занят своими делами. Если возникает острая ситуация, когда участие адресата все же необходимо, важно извиниться за вторжение и обосновать проблему, чтобы человек понял, что задачу решить нужно срочно.

    \item «У меня отличная новость!» По мнению Ильяхова, это дурная фраза, за которой обычно кроется не хорошая, а плохая новость, поэтому лучше сразу переходить к делу, а не создавать вид, что все хорошо.

    \item Абстрактные суждения без постановки конкретной задачи: «Надо бы сделать», «Я тут подумал» и подобные. Людей раздражает неопределенность во всех сферах жизни, в том числе и в постановке задач. Если сотрудник сам будет додумывать детали задачи, он рискует не попасть в мысль того, кто эту самую задачу ставит. В письме должны быть изложены все аспекты вопроса.

\end{enumerate}

Это основные ошибки, которые чаще остальных встречаются в письменной речи обывателей и приводят к медленному решению задач. В книге их перечислено больше: минимум вопросов в одной теме письма, не приложенные файлы, ошибки в обращении, чрезмерное использование шаблонных фраз и стоп-слов. Хочется отослать пишущего к еще одной книге Ильяхова «Пиши, сокращай» – меньше лишних слов, больше четкости и адресности, тогда письмо наверняка будет замечено и прочтено.

\textbf{Что раздражает читателя?}

А теперь поговорим о том, что раздражает читателя делового письма. С отсылкой к «Правилам деловой переписки», книге главреда.

\textbf{Небрежность}

Ошибки пунктуации, неверно составленные предложения, отсутствует подпись. Письма на скорую руку тоже должны быть написаны с умом, иначе получатель может подумать, что задача не первостепенная, раз так небрежно составлена.

\textit{Небрежный текст:} «Здравствуйте Михаил составте пожалуйста список сотрудников для согласования допусков на объект».

\textit{Как надо:} «Добрый день, Михаил! Прошу Вас составить список сотрудников для согласования допусков на объект до 15 января. Директор отдела».

Как видно, в первом предложении есть ошибки орфографии и совсем отсутствуют знаки препинания. Во втором случае просьба оформлена точно, есть обращение, дедлайн, подпись.

\textbf{Панибратство}

Оно выражается в пренебрежительном обращении, например, «народ», «приветик» без соблюдения субординации. Подобное обращение вызывает лишь отторжение, но не самими словами, а нарушением личных границ тех, к кому оно адресовано. Подобное допустимо в неформальной обстановке, но не на работе.

\textbf{Перекладывание ответственности и создание групповых переписок}

В групповых переписках, в которых есть неадресные обращения, никто не понимает, кому полагается решение задачи. Это распространенная проблема в больших коллективах при отсутствии грамотных управленцев.

Как не надо: «Коллеги, нам нужно обсудить вопросы размещения рекламы на сайте партнеров. Какие предложения?»

Как надо: «Сергей, примите решение, стоит ли размещать рекламу у партнеров – обещают 1 млн просмотров. Медиаплан в приложении».

Как идентифицировать сотрудника, который перекладывает ответственность? Он использует такие фразы: «передам коллегам», «это не в моей компетенции», «жду решения руководства». За ними скрывается «сами внесите правки, я в этом не участвую, не трогайте меня».

\textbf{Неуместность фраз и бездумное употребление сокращений}

Случается такое, что отдельные сотрудники используют жаргонизмы и сокращения либо неуместно, т.е. без понимания их смысла, либо настолько часто, что не все коллеги понимают, о чем речь.

\textit{Пример:} «Проведите левкеридж лидеров мнений для повышения узнаваемости бренда и чекните в доке результат».

\textit{Перевод:} «Изучите, что думают эксперты о бренде и как можно повысить его узнаваемость. Результаты занесите в документ».

Какое предложение проще понять? [Top Lead, 2015].

\textbf{Неуважение к чужому времени}

Важное правило общения деловой переписки – не мешай другому делать свою работу. Слова «надо вчера», «это в приоритете», «ASAP» вышибают человека из своего режима. И все бы ничего, но если дело действительно срочное.

Часто бывает так: все бросишь, переключаешься, делаешь эту работу, а когда наступает «завтра», то уже и не так срочно было нужно. В результате время и силы потрачены почти зря. Чужое время – не расходный материал, поэтому относиться к нему следует ответственно и серьезно, а не использовать в качестве развлечения и инструмента повышения собственной крутости.

\textbf{Примеры от Максима Ильяхова:}

Сказали: «Коллеги, мы вас услышали». Как поняли коллеги: «Вы – сборище баранов, которые не знают, чего хотят. Мы презираем вас, но у вас есть деньги».

Сказали: «Это дело на пять минут». Как поняли коллеги: «Это дело на целый день, а то и на два».

Это основные ошибки деловой переписки, с которыми наверняка сталкивался каждый человек. Их главная проблема – это не используемые слова, а смысл, который за ними кроется. Умышленно или неумышленно заложенный – не важно.

\textbf{Зачем учить правила?}

Во-первых, правила деловой переписки по почте предполагают упрощение и улучшение коммуникации – никто (или почти никто) не хочет целый день тратить на написание писем в попытке объяснить адресату, чего от него ждут.

Во-вторых, знание правил ведения переписок помогает наладить отношения с окружающими в целом. Те, кто владеет навыками коммуникации, свободно общаются с разными людьми на любые темы.

Для погружения в тему деловой коммуникации книга «Новые правила деловой переписки» Ильяхова Максима, создателя сервиса «Главред», подходит идеально. Это одно из лучших профессиональных пособий в России. Она написана простым и понятным языком, автор приводит множество наглядных примеров.

Книга – учебник для менеджеров, маркетологов, секретарей, делопроизводителей, администраторов, всех специалистов, ведущих в той или иной мере переговоры по электронной почте.

Знание правил переписки дает несколько конкурентных преимуществ эксперту и его компании:
\begin{enumerate}
    \item Сообщения будут точными, запоминающимися, они не попадут в спам.
    \item Для решения вопроса понадобится минимум писем, а значит, и времени на решение задачи.
    \item Устраняются многословие, «вода», канцеляризмы, избыток в речи которых портит любую коммуникацию.
\end{enumerate}

Деловая переписка – это не реверансы из вымученных шаблонов, а минимум манипуляций и максимум эффективности.

\textbf{Основные правила правильной деловой переписки}

Рассмотрим основные постулаты деловой переписки, ее правила на примерах писем, обратившись к книге Максима Ильяхова. Кстати, многие из них перекликаются с основными рекомендациями Управления государственной службы и кадров Правительства Москвы, поэтому в эффективности их использования сомневаться не приходится [Университет Правительства Москвы, 2015].

\textbf{Уважение времени и внимания адресата}

Итак, первоочередные правила делового письма:
\begin{enumerate}
    \item Одно письмо – одно дело. Нет смысла умещать в одно послание множество вопросов. Во-первых, так их сложнее улавливать и перерабатывать. Во-вторых, это помогает структурировать и саму подачу вопроса – в одном письме точно не смешаются приложения, тезисы и прочие детали. К тому же, человеку не придется переключаться – это тоже работа, которая требует времени и сил.
    \item Назвать письмо в теме, чтобы адресат понял задачу, еще не открыв послание. Так он сможет расставить приоритеты при обработке писем. Заголовки оформляются точно и предельно емко. Например: «Акт сверки “Кристалл” III квартал», «Отчет посещаемости за январь 2022», «Выгрузка контактов, 15 февраля». В компаниях может быть принята своя система обозначения писем, например, «Черновик допсоглашения», «Внутренний план», «Вычитано, отчет», «Заявка на приемку» и т.п.
    \item Обозначить срочность – важный момент, если важно определить дедлайн выполнения задачи. Хорошо, если необходимая дата была читабельной и понятной без лишних телодвижений (например, поиска календаря и отсчета дней). Зашифрованный вариант: «Решить задачу к 17.04.2021», понятный вид: «Решить задачу до этого четверга».
\end{enumerate}

Важное дополнение: в некоторых компаниях электронная почта считается быстрым способом передачи информации, где сотрудники по инструкции по мере поступления проверяют почтовый ящик. Если в вашей компании этого нет, сверхсрочные задачи лучше обсуждать по телефону – это точно быстрый вариант. Формулировка «Сделать как можно скорее» или «Срочно» нерабочая, поскольку не содержит конкретики, а только неуважение к рабочему и личному времени адресата.

И далее переходим к самому интересному – составлению самого тела письма. Если с обозначением заголовка, постановкой даты все более-менее понятно, то с оформлением мысли могут возникнуть проблемы, а то и страх белого листа, когда не знаешь, с чего начать. Для таких случаев инструкция ниже.


\textbf{Структура письма}

Как в любом тексте, в деловом письме должны быть начало, середина и заключение. Пробегая глазами по сообщению, читатель должен понимать, где основная мысль, вопросы, заключения, ссылки, приложения. Располагают их обычно в таком порядке:
\begin{enumerate}
    \item Обращение с именем, приветствие.
    \item Основная смысловая часть, суть письма.
    \item Вопросы к читателю, лучше отдельными выделенными блоками.
    \item Призыв к действию, контакты, ссылки. Оставьте контакты, как с вами можно связаться помимо почты.
\end{enumerate}

Пример плохого оформления письма:

«Сергей, добрый день! Не смог прийти к Вам лично, приношу извинения. Пообщался с ребятами, хочу прояснить несколько моментов. Когда Вы планируете подать заявку на обучение? От этого будет зависеть срок подготовки документов нашим отделом. Кто сможет стать вашими поручителями? Консульство внимательно оценивает этих людей, поэтому их состав нам следует обсудить заранее. Какой график оплат Вам подойдет? Мы предлагаем внести предоплату 50%, но вуз позволяет внести от 15%. По нашему опыту, чем больше предоплата, тем выше шансы получить визу. В случае, если посольство откажет, вуз вернет деньги в полном объеме. Мы можем обсудить этот вопрос по телефону. Иван, +7 901 123-45-67».

Налицо все типичные ошибки: абзацев нет, разделения нет, много лишнего, вся информация в куче и плохо читабельна.

По правилам деловой переписки по электронной почте можно оправить сообщение в таком виде:

«Сергей, добрый день!

Не смог прийти к Вам лично, приношу извинения. Пообщался с ребятами, хочу прояснить несколько моментов.
\begin{enumerate}
    \item Когда Вы планируете подать заявку на обучение? От этого будет зависеть срок подготовки документов нашим отделом.
    \item Кто сможет стать вашими поручителями? Консульство внимательно оценивает этих людей, поэтому их состав нам следует обсудить заранее.
    \item Какой график оплат Вам подойдет? Мы предлагаем внести предоплату 50%, но вуз позволяет внести от 15%. По нашему опыту, чем больше предоплата, тем выше шансы получить визу. В случае, если посольство откажет, вуз вернет деньги в полном объеме.
\end{enumerate}

Мы можем обсудить эти вопросы по телефону.

Иван, +7 901 123—45-67».

Как говорится, найдите несколько отличий! Заметим, что в сообщении содержится несколько вопросов, но на одну тему, поэтому наш Иван сэкономил время адресата, предложил созвониться, проявил заботу о нем.

Если основная задача в большом объеме, старайтесь умесить ее в пару предложений, отбросив лишние слова, используйте списки, абзацы.

\textbf{Например:}

«Виталий!

У нас проблемы с коммутационными комплексами «Броненосец», поэтому Сергею или «Альфе» их не хватит. Тебе надо решить, кому сколько отдать до конца дня.

Почему так получилось: Сергей сделал заказ комплексов для собственного проекта и забирал их с нашего склада постепенно. На данный момент осталось 80 штук, из них для Сергея 40 комплексов, он заберет их завтра.

Для «Альфы» мы должны поставить 60 боксов, с ними мы ведем параллельный проект.

Получается, завтра нам не хватит 20 боксов для «Альфы» или Сергея.

Реши, пожалуйста, вопрос с Александром сегодня до 19:30, скажи мне, я передам складу, чтобы они подготовили к завтрашней выдаче.

Семен, менеджер по логистике.

+7 902 123-45-67».

Суть понятна, структурирована, разложена по блокам.

Если в письмо нужно вложить ссылки, сделайте это в столбик, тогда читателю будет достаточно нажать на нужную, чтобы открыть окно, а не копировать строки, вычленяя запятые и пробелы.

\textbf{Вежливость}

Важные аспекты хорошего письма: не путать имена, не злоупотреблять жаргонизмами и заботиться о читателе. Приветствуется краткость – чем меньше слов, тем менее раздражающим будет письмо.

Важно проявить заботу о читателе – разделить текст, чтобы он был читабельным, предложить варианты действий и возможных решений задач. Придерживайтесь нейтрального и спокойного тона изложения. Не пишите без необходимости, только чтобы напомнить о себе.

Вежливость не в словах, она в отношении.

\textbf{Формировка вопроса}

Правильно задать вопрос – это целая наука. Недостаточно собрать в предложение нужную информацию и поставить в конце вопросительный знак.

Деловое письмо, по правилам деловой переписки, должно содержать правильно поставленный вопрос. Никакой риторики и отвлеченных умозаключений. В некоторых книгах авторы рекомендуют задавать читателю открытые вопросы, например, «Что ты об этом думаешь?» Однако на такой вопрос ответить не просто сложно, не каждый сообразит, как к нему подойти, потому что конкретики в нем нет. Такой вопрос часто игнорируют с надеждой, что «само рассосется».

\textbf{Как не надо: }

«Мы получили 500 откликов, по которым будем составлять картину возражений. С другой стороны, эти люди будут ждать внедрения своих ожиданий в работу.

Мы можем поговорить с потенциальными клиентами, но для этого нам надо выделить сотню менеджеров. Возможно, благодаря этому мы сможем проработать свои скилы.

С другой стороны, наш диалог с текущими клиентами может напомнить «ошибку выжившего», ведь нам надо вести диалог с отказавшимися от сотрудничества клиентами. А как на них выйти – я не знаю, информации в CRM нету.

Что об этом думаешь?»

Гораздо удобнее ответить на конкретные вопросы, если бы их задали в рамках письма:

«Мы можем узнать более глубокие о себе вещи и прокачать скилы. Меня беспокоит, что наш диалог с текущими клиентами может напомнить «ошибку выжившего», ведь нам надо вести диалог с отказавшимися от сотрудничества клиентами. А как на них выйти – я не знаю, информации в CRM нету.

Какие у меня вопросы и сомнения:
\begin{enumerate}
    \item Опрос создаст резонанс в СМИ, а это нам сейчас точно не нужно.
    \item Как работать с ожиданием опрошенных клиентов?
    \item Что, если опросить ключевых клиентов приватно?
    \item Как нам выйти на тех клиентов, кто с нами уже не работает?
\end{enumerate}

Если тебе удобно, давай я тебе позвоню и все обсудим».

Из примеров видно, что по-разному поставленные вопросы предполагают совершенно разной точности ответы. Чем точнее сформулирован запрос, тем быстрее вы получите конструктивное решение от собеседника.

\textbf{Благодарность}

Поблагодарить собеседника иногда важно и нужно, но писать одно «спасибо» – неверное решение. Письмо придется открыть, прочитать, удалить, а особой ценности такое послание не несет. Чтобы исправить это, напишите конкретно, за что благодарите, прикрепите подарок:

«Марина, спасибо! Ты очень выручила нас! В знак благодарности прими курьера с подарком от нас, я попросил его оставить коробку в приемной на твое имя».

Если подарок уместен, продумайте, чтобы адресату он был нужен и удобен для перехвата – вряд ли человеку будет интересно бегать за курьером или караулить его, когда нужно выходить по делам.

Делать подарок необязательно, можно просто выразить благодарность словами и предложить свою помощь:

«Марина, спасибо! Ты нас очень выручила! Если тебе понадобится помощь в организации мероприятий, я буду рад сделать тебе хорошую скидку».

Не каждый протянувший руку помощи человек будет ждать благодарности в ответ, но этот жест, особенно подкрепленный приятным бонусом, определенно прибавит вам значимости.

\textbf{Извинения}

Нередко письмом нужно извиниться за неприятные инциденты. Самый частый повод – срыв сроков сдачи чего-либо. И все зависит от значимости провала.

Пример: «Дмитрий, привет! Я должен был сегодня прислать перевод текста, но не успел сделать его. Извини! Данил».

Человек пишет письмо с извинениями за срыв сдачи перевода в день дедлайна. Заказчик не может подписать договор или сделать публикацию, у него график. Что дадут ему извинения? Ничего, Дмитрию от извинений не станет легче. Данил проявил безответственность, не предложив вариантов выхода из ситуации.

Что он мог сделать:

\begin{enumerate}
    \item Предупредить о проблеме заранее, например, за день-два и договориться об отсрочке либо Дмитрий смог бы найти за это время другого исполнителя.
    \item Предложить решение проблемы, но, опять же. Заранее.
\end{enumerate}

«Правильное» извинение:

«Привет, Дмитрий!

В четверг я должен прислать перевод договора, но не успеваю ко времени. Чтобы не подводить тебя, я нашел переводчика, он готов сделать работу в среду, чтобы остался день на проверку, так я точно успею просмотреть результат. Таким образом, в четверг у нас будет готовый переведенный договор.

Вопрос в бюджете: стоить это будет 10 000 рублей. Мы сможем это оплатить?

Данил».

Не стоит бояться признаться в провале или срыве срока, своевременное извинение и нахождение решения поможет сохранить лицо, репутацию и добрые отношения с партнерами. Но и злоупотреблять этим определенно не нужно.

\textbf{Когда писать не надо}

Есть случаи, когда писать письма, даже очень срочные, не надо – есть все шансы допустить всевозможные ошибки, о которых сказано ранее. Таких ситуаций три:
\begin{enumerate}
    \item На эмоциях. В них поток мысли не структурирован и редко продуман, есть риск наговорить лишнего, нарушить субординацию. На радостях тоже можно натворить ошибок. Сначала успокойтесь, придите в «дзен» и только потом пишите письма.
    \item Вместо планерки. Групповые переписки в попытках что-то решить – это настоящее зло. Десятки людей, читающие одновременно письма и отвечающие на них, создают хаос в диалогах, в ответах теряется суть вопросов, как и сами ответы. Многие такие переписки проходят безрезультатно. Если планерка все же сорвалась, а задачи распланировать необходимо, следует собрать нужных людей в телефонной конференции или на локальном собрании, либо раздать задачи адресно, как уже говорилось ранее.
    \item Когда «горит». У сотрудников разная скорость переработки почты, ведь они обычно занимаются и другой работой тоже. Если задача требует очень быстрого решения, целесообразно не отправлять письма по электронной почте, а позвонить, чтобы не засорять ящик адресата десятком напоминаний «срочно», «пожар-горим».
\end{enumerate}

Еще один случай, когда писать не надо – когда мысли не структурируются, а вопрос очень важный и ставки высоки, для ошибок нет маневра. Возможно, в таком случае будет полезно:
\begin{enumerate}
    \item Проконсультироваться с коллегами, как структурировать материал.
    \item Провести аудиоконференцию.
    \item Организовать личную встречу.
\end{enumerate}

Личный контакт поможет лучше выстроить диалог, проследить за реакциями партнера, обговорить детали, которые в переписке можно упустить, а потом вляпаться в конфузную ситуацию.

\textbf{Холодные письма}

Для диалога с незнакомыми людьми, так называемыми «холодными контактами», выше перечисленные правила остаются актуальными, но есть несколько важных дополнений, которые помогут сделать переписку более продуктивной:
\begin{enumerate}
    \item Сохраняйте спокойный тон. Никаких эмоций, сохраняйте нейтральную тональность общения, как будто разговариваете с незнакомцем вживую. Соблюдайте субординацию, никаких «ты», креатива тоже лучше по минимуму, ведь настроение и предпочтения холодного контакта вы не знаете, а значит, можете спугнуть потенциального клиента. Если очень хочется оригинальности, прикрепите к письму картинку или видео, но сам текст делайте спокойным.
    \item Не делайте выводы заранее. О финансовых возможностях и предпочтениях человека вы не знаете, а значит, и с предложениями надо быть осторожным. «Это выгодное предложение для вас» может оказаться просто неактуальным, замените эту фразу на спокойное предложение «Буду рад рассчитать смету по этому проекту».
    \item Не уговаривайте, на заискивайте, не давите, создайте интригу, чтобы читатель сам захотел продолжить диалог. «Согласитесь, это интересно», «Эта возможность бывает раз в жизни» замените на «Посмотрите наше предложение, возможно, оно Вас заинтересует».
    \item Читатель вам ничего не должен. Возможно, он даже письмо не откроет, поэтому не стоит настойчиво ждать ответа.
    \item Предложите простое действие – рассчитать смету, обсудить детали по телефону или встретиться. Правильно составленное письмо увеличит шансы на то, что человек действительно захочет обратиться к вам за услугой, а понятные действия без грандиозных планов станут первым шажком навстречу к сотрудничеству.
\end{enumerate}

Главное в работе с холодным клиентом – сохранение позиции нейтралитета. Кажется, что яркое письмо привлечет нового клиента, и он буквально должен прийти за услугой или товаром. Но это заблуждение, которое отнимает у пишущего немало энергии на ожидание ответа. Просто сохраняйте субординацию, спокойствие, пишите по основным алгоритмам с перечисленными выше техниками. Если письмо будет составлено верно, заказчик выйдет на контакт сам.

\textbf{Пишите смело}

Владение деловой перепиской – это умение, которое можно нарабатывать. Ему можно обучиться, это будет полезный для любого специалиста навык. Как учиться – все зависит от вас. Кому-то важно пройти офлайнобучение, другим нужен бумажный учебник, и одним из лучших в России считается книга Максима Ильяхова «Новые правила деловой переписки».  Скачать ее на смартфон или компьютер тоже можно, так она гарантированно всегда будет под рукой.

На бытовом и профессиональном уровне научиться взаимодействию с людьми и подтянуть или развить навыки делового и межличностного общения поможет онлайн-программа «Лучшие техники коммуникации». Приобретенные навыки пригодятся во всех сферах жизни, вы научитесь конструктивно вести диалоги, обходить конфликты, отстаивать свои границы экологично и безопасно, поддерживать родных и близких без хода в «спасателя», отвечать на разнообразные вопросы.

Желаем удачи в освоении новых коммуникативных навыков, а также просим принять участие в небольшом опросе: Как вы считаете, нужно ли специально осваивать правила деловой переписки?