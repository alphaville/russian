\documentclass[letterpaper,11pt]{book}

\usepackage[utf8]{inputenc}
\usepackage[english,russian]{babel}
\usepackage{color}
\usepackage[dvipsnames,svgnames]{xcolor}
\usepackage[a4paper, margin=1in]{geometry}
\usepackage[hang,flushmargin]{footmisc}
\usepackage{dblfnote}
\usepackage{enumitem}
\RequirePackage{cmap}
\usepackage[OT2,T1]{fontenc}
\usepackage{wrapfig}

\usepackage{lipsum}
\usepackage{tikz}
\usetikzlibrary{backgrounds}
\makeatletter

\tikzset{%
  fancy quotes/.style={
    text width=\fq@width pt,
    align=justify,
    inner sep=1em,
    anchor=north west,
    minimum width=\linewidth,
  },
  fancy quotes width/.initial={.8\linewidth},
  fancy quotes marks/.style={
    scale=5,
    text=white,
    inner sep=1pt,
  },
  fancy quotes opening/.style={
    fancy quotes marks,
  },
  fancy quotes closing/.style={
    fancy quotes marks,
  },
  fancy quotes background/.style={
    show background rectangle,
    inner frame xsep=0pt,
    background rectangle/.style={
      fill=gray!25,
      rounded corners,
    },
  }
}

\newenvironment{fancyquotes}[1][]{%
\noindent
\tikzpicture[fancy quotes background]
\node[fancy quotes opening,anchor=north west] (fq@ul) at (0,0) {``};
\tikz@scan@one@point\pgfutil@firstofone(fq@ul.east)
\pgfmathsetmacro{\fq@width}{\linewidth - 2*\pgf@x}
\node[fancy quotes,#1] (fq@txt) at (fq@ul.north west) \bgroup}
{\egroup;
\node[overlay,fancy quotes closing,anchor=east] at (fq@txt.south east) {''};
\endtikzpicture}

\makeatother

\DFNalwaysdouble % for this example

\usepackage{hyperref}

\hypersetup{
    bookmarks=true, unicode=false,
    pdftoolbar=true, pdfmenubar=true, pdffitwindow=false,
    pdfstartview={FitH}, pdftitle={Russian Vocabulary},
    pdfauthor={P Sopasakis},
    pdfkeywords={}, pdfsubject={Confidential}, 
    pdfnewwindow=true, colorlinks=true, linkcolor=red,
    citecolor=cyan, filecolor=magenta, urlcolor=RoyalBlue
}

\usepackage{parskip}
\setlength{\parindent}{0pt}

\newcommand{\red}[1]{\textcolor{red}{#1}}
\newcommand{\comment}[1]{\textcolor{gray}{\it #1}}
\newcommand{\explain}[2]{\red{#1}\footnote{#1{}:{} #2}}
\newcommand{\explainDetail}[3]{\red{#1}\footnote{#2{}:{} #3}}


\newenvironment{dialogue}{\list{---}{\it \itemsep=0.05\parskip \topsep=0.75\parskip \parsep=0.75\parskip}}{\endlist}


\title{Сборник текстов\\{\small Газетные и Журнальные Статьи. Тексты об искусстве, истории, музыке, и новостях.}}
\author{Пантелис Сопасакис}

\begin{document}
\maketitle
\tableofcontents

\chapter{Искусство}

\section{Художники}
\subsection{В\'{и}ктор Мих\'{а}йлович Васнец\'{о}в}
% https://muzei-mira.com/biografia_hudojnikov/765-viktor-mihaylovich-vasnecov-biografiya.html
Виктор Михайлович Васнецов родился в 1848 году 15 мая в селе со смешным названием Лопьял. Отец Васнецова был священником, также как и его дед и прадед. В 1850 году Михаил Васильевич увёз семью в село Рябово. Это было связано с его службой. У Виктора Васнецова было 5 братьев, один из которых также стал знаменитым художников, звали его Аполлинарий.

Талант Васнецова проявился с детства, но крайне неудачное \explainDetail{денежное}{д\'{е}нежный/-ая/-ое}{monetary} положение в семье не оставило вариантов, как отдать Виктора в Вятское духовное училище в 1858 году. Уже в 14-летнем возрасте Виктор Васнецов учился в Вятской духовной семинарии. Детей священников туда брали бесплатно.

Так и не окончив семинарию, в 1867 году Васнецов отправился в Петербург поступать в Академию художеств. Денег у него было совсем мало, и Виктор выставил на «аукцион» 2 свои картины -- «Молочница» и «Жница». До \explainDetail{отъезда}{отъезд}{departure} он так и не получил за них денег. 60 рублей за эти две картины он получил спустя несколько месяцев уже в Петербурге. \explainDetail{Прибыв}{прибывать/прибыть}{arrive} в столицу, у молодого художника было всего 10 рублей.

Васнецов отлично справился с экзаменом по рисованию и сразу \explain{был зачислен}{was enrolled (\textit{гл. св.} зачислить: to enrol)} в Академию. Около года он занимался в Рисовальной школе, где и познакомился со своим учителем -- И. Крамским.

К занятиям в Академии художеств Васнецов приступил в 1868 году. В это время он \explain{сдружился с}{made friends with} Репиным, и даже одно время они жили на одной квартире.

Хоть Васнецову и нравилось в Академии, но он её не закончил, уехав в Париж в 1876 году, где прожил больше года. В это время там же находился и Репин в \explainDetail{командировке}{командировка}{business trip}. Они также поддерживали дружеские отношения.

После возвращения в Москву Васнецова сразу приняли в Товарищество передвижных художественных выставок. К этому времени стиль рисования художника значительно меняется, да и не только стиль, сам Васнецов перебирается жить в Москву, где сближается с Третьяковым и Мамонтовым. Именно в Москве Васнецов \explainDetail{раскрылся}{раскрыв\'{а}ться/раскр\'{ы}ться}{to open, uncover oneself, to come out}. Ему нравилось находиться в этом городе, он чувствовал себя легко и \explainDetail{выполнял}{выполн\'{я}ть/в\'{ы}полнить}{to perform, execute, carry out} различные творческие работы.

Более 10 лет Васнецов \explainDetail{оформлял}{оформл\'{я}ть/оф\'{о}рмить}{put into shape, form} Владимирский соб\'{о}р в Киеве. В этом ему помогал М. Нестеров. Именно после окончания этой работы, Васнецова можно по праву назвать великим русским иконописцем.

1899 год стал \explainDetail{пиком}{пик}{peak} популярности художника. На своей выставке Васнецов представил публике «Трёх богатырей».

После революции Васнецов стал жить уже не в России, а в СССР, что его серьёзно \explain{угнетало}{угнетать}{opress, depress, despirit}. Люди \explainDetail{уничтожали}{уничтож\'{а}ть/уничт\'{о}жить}{to destroy, obliterate; уничтож\'{е}ние: destruction} его картины, \explainDetail{относились}{относ\'{и}ться + \textit{дат.}}{to treat} неуважительно к художнику. Но до конца своей жизни Виктор Михайлович был в\'{е}рен своему делу -- он рисовал. \explainDetail{\'{У}мер}{умирать/умереть}{умир\'{а}ю, умир\'{а}ешь, умир\'{а}ют; умр\'{у}, умрёшь, умр\'{у}т: to die} он 23 июля 1926 года в Москве, так и не закончив портрет своего друга и ученик\'{а} М. Нестерова.


\section{Произведения}
\subsection{Картина «Три Богатыря» ВМ Васнец\'{о}ва}
% https://muzei-mira.com/kartini_russkih_hudojnikov/1321-opisanie-kartiny-bogatyri-tri-bogatyrya-vasnecova-1898.html

Картина В\'{и}ктора Мих\'{а}йловича Васнец\'{о}ва «Богатыр\'{и}» \explain{по пр\'{а}ву}{rightfully } считается настоящим народным \explainDetail{шед\'{е}вром}{шед\'{е}вр}{masterpiece} и с\'{и}мволом отечественного искусства. \explainDetail{Создав\'{а}лась}{создаваться/создаться}{create} картина во второй половине XIX века, когда среди русских художников был\'{а} очень популярна тема народной культуры, русского фольклора. Для многих художников это увлечение оказалось кратковременным, но у Васнецова народная фольклорная тематика \explainDetail{ст\'{а}ла}{стать/становиться}{become (ст\'{а}л/-а/-о)} \explainDetail{осн\'{о}вой}{осн\'{о}ва}{basis} всего \explainDetail{творчества}{творчество}{cretativity}.

На картине «\explainDetail{Богатыр\'{и}}{богат\'{ы}рь}{A bogatyr or витязь is a stereotypical fictional character in medieval Russian legends}» \explainDetail{изображен\'{ы}}{изображён/-\'{а}/-\'{о}}{depicted; изображение: image, depiction; изображ\'{а}ть/изобраз\'{и}ть: to depict} три русских богатыр\'{я}: Илья Муромец, Добрыня Никитич и Алёша Попович - знаменитые герои народных \explainDetail{был\'{и}н}{был\'{и}на}{epic}.

\explainDetail{Испол\'{и}нские}{испол\'{и}нский}{gigantic} фигуры богатыр\'{е}й и их коней, распол\'{о}женные \explain{на пер\'{е}днем пл\'{а}не}{in the foreground (пер\'{е}дний пл\'{а}н)} картины, символизируют силу и мощь русского народа. Этому \explainDetail{впечатлению}{впечатл\'{е}ние}{impression} \explainDetail{спос\'{о}бствуют}{способствовать/поспособствовать}{contribute to} и \explainDetail{внушительные}{внушительный}{impressive} размеры картины -- 295$\times$446 см.

Над созданием этой картины художник работал почти 30 лет. В 1871 году был с\'{о}здан первый \explain{набр\'{о}сок}{sketch} сюжета в карандаш\'{е}, и с тех пор художник увлёкся идеей создания этой картины. В 1876 году был сделан знаменитый \explain{эск\'{и}з}{sketch} с уже \explainDetail{н\'{а}йденной}{н\'{а}йденный/-ая/-ое}{$<$ найти} основой композиционного решения. Работа над самой картиной длилась с 1881 по 1898 год. Готовая картина была куплена П. Третьяковым, и до сих пор она \explainDetail{украш\'{а}ет}{украш\'{а}ть/укр\'{а}сить}{decorate} Государственную Третьяковскую галерею в Москве.

В центре картины изображён Илья Муромец, народный любимец, герой русских был\'{и}н. Не всем известно, что Илья Муромец не сказочный персонаж, а реальное историческое лицо. История его жизни и \explainDetail{р\'{а}тных}{р\'{а}тный}{military} \explainDetail{п\'{о}двигов}{п\'{о}двиг}{exploit, feat} -- это реальные события. \explain{Впосл\'{е}дствии}{subsequently}, закончив свои труды по охране родины, он стал монахом Киево-Печёрского монастыр\'{я}. Был причислен к лику святых\footnote{was canonised}. Васнецов эти факты знал, создав\'{а}я образ Ильи Муромца. «\explainDetail{Матёр}{матёрый}{mature, fully grown, hardened} человек Илья Муромец» -- говорит былина. А на картине Васнецова мы видим могучего воина и при том \explainDetail{бесх\'{и}тростного}{бесх\'{и}тростный}{ingenuous, silly} открытого человека. В нём \explainDetail{сочетаются}{сочетаться}{combine} исполинская сила и \explain{великод\'{у}шие}{generosity, magnanimity, goodness}. «А конь под Ильёй \explain{лютый}{fierce} зверь» -- продолжает сказание. \explainDetail{Мощная}{мощный/-ая/-ое}{powerful} фигура коня, изображённого на картине с массивной металлической цепью вместо упряжки, \explainDetail{свид\'{е}тельствует}{свид\'{е}тельствовать}{testify; свидетель: witness} об этом.

Добрыня Никитич по народным преданиям был очень образ\'{о}ванным и \explainDetail{м\'{у}жественным}{м\'{у}жественный}{manly} человеком. С его личностью связано много чудес, наприм\'{е}р, заговорённая броня\footnote{charmed armor} на его плечах, \explain{волш\'{е}бный}{magic} меч-кладенец. Добрыня изображён таким как и в былинах -- величавым, с тонкими, благородными чертами лица, подчёркивающими его культурность, образ\'{о}ванность, \explain{решительно}{decisively} вынимающий меч из \explain{н\'{о}жен}{sheath} с готовностью \explainDetail{бр\'{о}ситься}{брос\'{а}ться/бр\'{о}ситься}{rush} в бой, защищая свою родину.

Алёша Попович \explain{по сравнению с}{as compared with} товарищами молод и строен. Он изображён с \footnote{л\'{у}ком}{bow} и стрелами в руках, но \explain{прикреплённые}{attached} к \explainDetail{седлу}{седло}{saddle} гусли свидетельствуют о том, что он не только бесстрашный воин, но и \explain{гусляр}{player of the musical instrument ``gusli''}, песенник, весельчак. В картине много таких деталей, которые характеризуют образы её персонажей.

Упряжки коней, одежда, амуниция не \explain{вымышленные}{fictional}. Такие образцы художник видел в музеях и читал их описания в исторической литературе. Художник \explain{мастерски}{masterfully} передаёт состояние природы, как бы предвещающей о наступлении опасности. Но богатыри представляют собой \explainDetail{надёжную}{надёжный/-ая/-ое}{reliable} и мощную силу защитников родной земли.




\subsection{Картина «Алёнушка» ВМ Васнец\'{о}ва}
% https://muzei-mira.com/kartini_russkih_hudojnikov/1321-opisanie-kartiny-bogatyri-tri-bogatyrya-vasnecova-1898.html

Алёнушка, печальная девочка у \explainDetail{пруда}{пруд}{pond} --- одна из любимых всеми картин В. Васнецова. Художник уд\'{а}чно использует сказочный сюжет, чтобы \explainDetail{раскрыть}{раскрывать/раскрыть}{to open/to discover} сложный и неоднозначный русский характер.

Грусть девочки очень взрослая. Печаль в её глазах граничит с \explainDetail{отчаянием}{отч\'{а}яние}{despair}. Неубранные рыжие волосы, тёмные глаза, нежно-алые губы --- формируют легко читаемый образ ребёнка с тр\'{у}дной судьб\'{о}й.
В Алёнушке совсем нет ничего \explainDetail{сказочного}{сказочный}{fabulous, fairytale-like}, фантастического.
С\'{о}бственно, вся ск\'{а}зочность сюжета подчеркнута лишь одной деталью --- группой \explainDetail{ласточек}{л\'{а}сточка}{swallow}, сидящих над головой \explainDetail{героини}{героиня}{heroine}. Этим с\'{и}мволом (как известно, л\'{а}сточки символиз\'{и}руют над\'{е}жду) художник \explainDetail{уравновешивает}{уравновешивать/уравновесить}{to balance --- уравнов\'{е}шиваю/-ешь/-ют; уравнов\'{е}шу/-ишь/-ят} полный \explainDetail{тоск\'{и}}{тоск\'{а}}{yearning, longing} \'{о}браз героини, даёт надежду на счастливый финал старой русской сказки.

Васнецов нап\'{о}лнил \explain{ф\'{о}новый}{background} пейзаж атмосферой тишины и гр\'{у}сти.
Отлично удал\'{и}сь художнику в\'{о}дная \explain{гладь}{smooth surface} пруда, \explainDetail{камыш\'{и}}{камыш}{reed}, осока, \explainDetail{ели}{ель}{fir tree}.
Всё \explainDetail{неподв\'{и}жно}{неподв\'{и}жный/-ая/-ое}{still, motionless}, тихо, спокойно.
Даже пруд отраж\'{а}ет героиню очень деликатно, \explain{слегк\'{а}}{slightly}.
Чуть трепещут молодые \explainDetail{ос\'{и}ны}{ос\'{и}на}{aspen}. \explainDetail{Едва}{едв\'{а}}{barely, hardly} \explain{хмурится}{turns gloomly} ос\'{е}ннее небо.
Тёмные, зелёные тона пейзажа контрастируют с \explainDetail{румянцем}{румянец}{blush} на лице героини, а ос\'{е}нняя грусть --- с яркими цветами на юбке Алёнушки. Зритель чувствует: ещё мгнов\'{е}ние и сказка прод\'{о}лжится\dots




\subsection{Сказка «Сестрица Алёнушка и братец Иванушка»}
Жили-были стар\'{и}к да стар\'{у}ха, у них был\'{а} дочка Алёнушка да сын\'{о}к Иванушка. Старик со старухой умерли. Остались Алёнушка да Иванушка одни-одинёшеньки. Пошла Алёнушка на работу и братца с собой взяла. Идут они по д\'{а}льнему пут\'{и}, по шир\'{о}кому п\'{о}лю, и захотелось Иванушке пить.
%
\begin{dialogue}
    \item Сестр\'{и}ца Алёнушка, я пить хочу!
    \item Подожди, братец, дойдем до кол\'{о}дца.
\end{dialogue}
%
Шли-шли, -- солнце высоко, колодец далёко, жар \explainDetail{донимает}{донимать}{(colloq.) to bother, harass}, пот выступ\'{а}ет. Сто\'{и}т коровье коп\'{ы}тце\footnote{hoof (diminutive of коп\'{ы}то)} полн\'{о} водицы.
%
\begin{dialogue}
    \item Сестрица Алёнушка, хлебну\footnote{(colloq.) to drink} я из копытца!
    \item Не пей, братец, телёночком станешь!
\end{dialogue}
%
Братец послушался, пошли дальше. Солнце высоко, колодец далёко, жар донимает, пот выступает. Сто\'{и}т лошадиное копытце полно водицы.
%
\begin{dialogue}
    \item Сестрица Алёнушка, напьюсь я из копытца!
    \item Не пей, братец, жеребёночком станешь!
\end{dialogue}
%
\explainDetail{Вздохнул}{вздых\'{а}ть/вздохн\'{у}ть}{sigh} Иванушка, опять пошли дальше. Идут, идут, - солнце высоко, колодец далёко, жар донимает, пот выступает. Сто\'{и}т к\'{о}зье копытце полно водицы. Иванушка говорит:
%
\begin{dialogue}
    \item  Сестрица Алёнушка, мочи нет: напьюсь я из копытца!
    \item  Не пей, братец, козлёночком станешь!
\end{dialogue}
%
Не послушался Иванушка и нап\'{и}лся из к\'{о}зьего копытца. Нап\'{и}лся и стал козлёночком\dots Зовёт Алёнушка братца, а вместо Иванушки бежит за ней беленький козлёночек. Залилась\footnote{flooded} Алёнушка слез\'{а}ми, села на стож\'{о}к -- плачет, а козлёночек в\'{о}зле неё скачет. В ту пору ехал мимо купец:
%
\begin{dialogue}
    \item О чём, красная девица, плачешь?
\end{dialogue}
Рассказала ему Алёнушка про свою беду. Купец ей и говорит:
\begin{dialogue}
    \item Под\'{и} за меня замуж. Я тебя наряж\'{у} в златосеребро, и козлёночек будет жить с нами.
\end{dialogue}
Алёнушка под\'{у}мала, под\'{у}мала и пошла за купца замуж. Стали они жить-поживать, и козлёночек с ними живёт, ест-пьёт с Алёнушкой из одной чашки. Один раз купца не было д\'{о}ма. \explainDetail{Откуда не возьм\'{и}сь}{откуда не возьм\'{и}сь}{out of the blue} прих\'{о}дит \explain{в\'{е}дьма}{witch}: стала под Алёнушкино окошко и такто ласково начала звать её куп\'{а}ться на реку\footnote{произношение: н\'{а}реку}. Привела ведьма Алёнушку на реку. Кинулась на неё, привязала Алёнушке на шею камень и бр\'{о}сила её в в\'{о}ду. А сам\'{а} оборот\'{и}лась Алёнушкой, \explainDetail{нарядилась}{наряж\'{а}ться/наряд\'{и}ться}{to dress as someone, to imitate} в её пл\'{а}тье и пришла в её хор\'{о}мы. Никто ведьму не \explain{распозн\'{а}л}{recognised}. Купец вернулся -- и тот не распознал.

Одному козлёночку всё было ведомо. Повесил он голову, не пьет, не ест. Утром и вечером ходит по бережку около воды и зовёт:
\begin{dialogue}
    \item Алёнушка, сестрица моя! Выплынь, выплынь на бережок\dots
\end{dialogue}

Узнала об этом ведьма и стала просить мужа \explain{зарежь}{slaughter} да зарежь козлёнка.
Купцу жалко было козлёночка, \explain{привык}{got used to + \textit{дат.}} он к нему. А ведьма так пристаёт, так упрашивает, --- делать н\'{е}чего, купец согласился:
%
\begin{dialogue}
    \item Ну, зарежь его\dots
\end{dialogue}
%
Велела ведьма разложить костры высокие, греть котлы чугунные, точить ножи булатные.
Козлёночек проведал, что ему недолго жить, и говорит названому отцу:
%
\begin{dialogue}
    \item Перед смертью пусти меня на речку сходить, водицы испить, кишочки прополоскать.
    \item Ну, сходи.
\end{dialogue}
%
Побежал козлёночек на речку, стал на берегу и жалобнёхонько закричал:
%
\begin{dialogue}
    \item   Алёнушка, сестрица моя! Выплынь, выплынь на бережок.
    Костры горят высокие,
    Котлы кипят чугунные,
    Ножи точат \explain{булатные} {Bulat is a type of steel alloy known in Russia from medieval times; it was regularly mentioned in Russian legends as the material of choice for cold steel. This type of steel was used by the armies of nomadic peoples. Bulat steel was the main type of steel used for swords in the armies of Genghis Khan.},
    Хотят меня зарезати!
\end{dialogue}
%
%
Алёнушка из реки ему отвечает:
%
\begin{dialogue}
    \item Ах, братец мой Иванушка! Тяжёл камень на дно тянет,
    Шёлкова трава ноги спутала,
    Желты пески на груди легли.
\end{dialogue}
%
%
А ведьма ищет козлёночка, не может найти и посылает \explainDetail{слуг\'{у}}{слуг\'{а}}{servant}:
\begin{dialogue}
    \item Пойди найди козлёнка, приведи его ко мне.
\end{dialogue}
% 
%
Пошёл слуга на реку и видит: по берегу бегает козлёночек и жалобнёшенько зовёт:
\begin{dialogue}
    \item Алёнушка, сестрица моя! Выплынь, выплынь на бережок.
    Костры горят высокие,
    Котлы кипят чугунные,
    Ножи точат булатные,
    Хотят меня зарезати!
\end{dialogue}
%
%
А из реки ему отвечают:
\begin{dialogue}
    \item Ах, братец мой Иванушка!
    Тяжёл камень на дно тянет,
    Шелкова трава ноги спутала,
    Желты пески на груди легли.
\end{dialogue}
%
Слуг\'{а} побежал домой и рассказал купцу про то, что слышал на речке. Собрали народ, пошли на реку, закинули сети шелковые и вытащили Алёнушку на берег. Сняли камень с шеи, окунули её в ключевую воду, одели её в нарядное платье. Алёнушка ожила и стала краше, чем была.

А козлёночек от радости три раза перекинулся через голову и обернулся мальчиком Иванушкой.

Ведьму привязали к лошадиному \explainDetail{хвосту}{хвост}{tail}, и пустили в чистое поле.


\subsection{Картина «Витязь на распутье»}

Виктор Михайлович Васнецов с циклом работ, \explain{посвященных}{dedicated (посвященный + \textit{дат.})} сюжетам русских сказок и былин, оказался \explainDetail{новатором}{новатор}{innovator} в этой области \explainDetail{изобразительного искусства}{изобраз\'{и}тельное искусство}{visual art}. За ним закрепилась репутация «художника-сказочника», он настолько проникся духом русской старины и былинного времени, что свой московский дом построил в виде деревянной избы (сейчас там находится мемориальный музей \explainDetail{живоп\'{и}сца}{живоп\'{и}сец}{painter, artist}).

Картина «В\'{и}тязь на расп\'{у}тье» \explain{отч\'{а}сти}{partly} является и отражением судьбы Васнецова.
Будучи \explain{пр\'{и}знанным}{пр\'{и}знанный}{recognised} художником-передвижником, он, как и его товарищи, \explainDetail{исполнял}{исполнять/исполнить}{performed} жанровые композиции в духе остросоциальных тем, волновавших общество в 1870-1890-х.
Но завладевшая им сказочная тематика диктовала \explainDetail{ин\'{о}е}{ин\'{о}й/ин\'{а}я/ин\'{о}е}{(определительное местоимение) другой, отличный от данного; (неопределённое местоимение) некоторый. See \href{https://ru.wiktionary.org/wiki/\%D0\%B8\%D0\%BD\%D0\%BE\%D0\%B9}{wikictionary.org:иной}.} развитие творчества. Живописец уход\'{и}л от проблем современности и \explain{погружался}{plunge, dive} в мир русской старины, рискуя быть \explainDetail{осужденным}{осужденный}{convicted}.

Выбор пути как один из \explain{роков\'{ы}х}{fatal} вопросов человеческой жизни на крупноформатном \explainDetail{холсте}{холст}{canvas} мастера приобрел эпическое звучание.
Перед камнем-предсказателем согнулся под тяжестью фатального \explainDetail{пророчества}{пророчество}{prophecy} опечаленный витязь. \explainDetail{Зловещий}{зловещий}{sinister} в\'{о}рон, садящееся красное солнце нагнетают\footnote{build up the atmosphere} атмосферу. \explainDetail{Нам\'{е}ренный}{нам\'{е}ренный}{intentional} отказ от \explainDetail{изображения}{изображение}{depiction} дороги (как выхода из трудности) художником сделан для того, чтобы показать \explain{неотврат\'{и}мость}{inevitability} судьбы.
\chapter{Сказки}
\section{Сестрица Алёнушка и братец Иванушка}
% https://deti-online.com/skazki/russkie-narodnye-skazki/sestrica-alyonushka-i-bratec-ivanushka/
% https://www.youtube.com/watch?v=UDaOREoItE8
Жили-были стар\'{и}к да стар\'{у}ха, у них был\'{а} дочка Алёнушка да сын\'{о}к Иванушка. Старик со старухой умерли. Остались Алёнушка да Иванушка одни-одинёшеньки. Пошла Алёнушка на работу и братца с собой взяла. Идут они по д\'{а}льнему пут\'{и}, по шир\'{о}кому п\'{о}лю, и захотелось Иванушке пить.
%
\begin{dialogue}
    \item Сестр\'{и}ца Алёнушка, я пить хочу!
    \item Подожди, братец, дойдем до кол\'{о}дца.
\end{dialogue}
%
Шли-шли, -- солнце высоко, колодец далёко, жар \explainDetail{донимает}{донимать}{(colloq.) to bother, harass}, пот выступ\'{а}ет. Сто\'{и}т коровье коп\'{ы}тце\footnote{hoof (diminutive of коп\'{ы}то)} полн\'{о} водицы.
%
\begin{dialogue}
    \item Сестрица Алёнушка, хлебну\footnote{(colloq.) to drink} я из копытца!
    \item Не пей, братец, телёночком станешь!
\end{dialogue}
%
Братец послушался, пошли дальше. Солнце высоко, колодец далёко, жар донимает, пот выступает. Сто\'{и}т лошадиное копытце полно водицы.
%
\begin{dialogue}
    \item Сестрица Алёнушка, напьюсь я из копытца!
    \item Не пей, братец, жеребёночком станешь!
\end{dialogue}
%
\explainDetail{Вздохнул}{вздых\'{а}ть/вздохн\'{у}ть}{sigh} Иванушка, опять пошли дальше. Идут, идут, -- солнце высоко, колодец далёко, жар донимает, пот выступает. Сто\'{и}т к\'{о}зье копытце полно водицы. Иванушка говорит:
%
\begin{dialogue}
    \item  Сестрица Алёнушка, мочи нет: напьюсь я из копытца!
    \item  Не пей, братец, козлёночком станешь!
\end{dialogue}
%
Не послушался Иванушка и нап\'{и}лся из к\'{о}зьего копытца. Нап\'{и}лся и стал козлёночком\dots Зовёт Алёнушка братца, а вместо Иванушки бежит за ней беленький козлёночек. Залилась\footnote{flooded} Алёнушка слез\'{а}ми, села на стож\'{о}к -- плачет, а козлёночек в\'{о}зле неё скачет. В ту пору ехал мимо купец:
%
\begin{dialogue}
    \item О чём, красная девица, плачешь?
\end{dialogue}
Рассказала ему Алёнушка про свою беду. Купец ей и говорит:
\begin{dialogue}
    \item Под\'{и} за меня замуж. Я тебя наряж\'{у} в златосеребро, и козлёночек будет жить с нами.
\end{dialogue}
Алёнушка под\'{у}мала, под\'{у}мала и пошла за купца замуж. Стали они жить-поживать, и козлёночек с ними живёт, ест-пьёт с Алёнушкой из одной чашки. Один раз купца не было д\'{о}ма. \explainDetail{Откуда не возьм\'{и}сь}{откуда не возьм\'{и}сь}{out of the blue} прих\'{о}дит \explain{в\'{е}дьма}{witch}: стала под Алёнушкино окошко и такто ласково начал\'{а} звать её куп\'{а}ться на реку\footnote{произношение: н\'{а}реку}. Привела ведьма Алёнушку на реку. \explainDetail{Кинулась}{кид\'{а}ться/к\'{и}нуться}{to throw oneself, to fling oneself, to dash, to rush, } на неё, привязала Алёнушке на шею камень и бр\'{о}сила её в в\'{о}ду. А сам\'{а} оборот\'{и}лась Алёнушкой, \explainDetail{нарядилась}{наряж\'{а}ться/наряд\'{и}ться}{to dress as someone, to imitate} в её пл\'{а}тье и пришла в её хор\'{о}мы. Никто ведьму не \explain{распозн\'{а}л}{recognised}. Купец вернулся -- и тот не распознал.

Одному козлёночку всё было ведомо. Повесил он голову, не пьет, не ест. Утром и вечером ходит по бережку около воды и зовёт:
\begin{dialogue}
    \item Алёнушка, сестрица моя! Выплынь, выплынь на бережок\dots
\end{dialogue}

Узнала об этом ведьма и стала просить мужа \explain{зарежь}{slaughter} да зарежь козлёнка.
Купцу жалко было козлёночка, \explain{привык}{got used to + \textit{дат.}} он к нему. А ведьма так пристаёт, так упрашивает, --- делать н\'{е}чего, купец согласился:
%
\begin{dialogue}
    \item Ну, зарежь его\dots
\end{dialogue}
%
Велела ведьма разложить костры высокие, греть котлы чугунные, точить ножи булатные.
Козлёночек проведал, что ему недолго жить, и говорит названому отцу:
%
\begin{dialogue}
    \item Перед смертью пусти меня на речку сходить, водицы испить, кишочки прополоскать.
    \item Ну, сходи.
\end{dialogue}
%
Побежал козлёночек на речку, стал на берегу и жалобнёхонько закричал:
%
\begin{dialogue}
    \item   Алёнушка, сестрица моя! Выплынь, выплынь на бережок.
    Костры горят высокие,
    Котлы кипят чугунные,
    Ножи точат \explain{булатные} {Bulat is a type of steel alloy known in Russia from medieval times; it was regularly mentioned in Russian legends as the material of choice for cold steel. This type of steel was used by the armies of nomadic peoples. Bulat steel was the main type of steel used for swords in the armies of Genghis Khan.},
    Хотят меня зарезати!
\end{dialogue}
%
%
Алёнушка из реки ему отвечает:
%
\begin{dialogue}
    \item Ах, братец мой Иванушка! Тяжёл камень на дно тянет,
    Шёлкова трава ноги спутала,
    Желты пески на груди легли.
\end{dialogue}
%
%
А ведьма ищет козлёночка, не может найти и посылает \explainDetail{слуг\'{у}}{слуг\'{а}}{servant}:
\begin{dialogue}
    \item Пойди найди козлёнка, приведи его ко мне.
\end{dialogue}
% 
%
Пошёл слуга на реку и видит: по берегу бегает козлёночек и жалобнёшенько зовёт:
\begin{dialogue}
    \item Алёнушка, сестрица моя! Выплынь, выплынь на бережок.
    Костры горят высокие,
    Котлы кипят чугунные,
    Ножи точат булатные,
    Хотят меня зарезати!
\end{dialogue}
%
%
А из реки ему отвечают:
\begin{dialogue}
    \item Ах, братец мой Иванушка!
    Тяжёл камень на дно тянет,
    Шелкова трава ноги спутала,
    Желты пески на груди легли.
\end{dialogue}
%
Слуг\'{а} побежал домой и рассказал купцу про то, что слышал на речке. Собрали народ, пошли на реку, закинули сети шелковые и вытащили Алёнушку на берег. Сняли камень с шеи, окунули её в ключевую воду, одели её в нарядное платье. Алёнушка ожила и стала краше, чем была.

А козлёночек от радости три раза перекинулся через голову и обернулся мальчиком Иванушкой.

Ведьму привязали к лошадиному \explainDetail{хвосту}{хвост}{tail}, и пустили в чистое поле.
\chapter{Музыка}


\section{Пётр Налич}
% https://uznayvse.ru/znamenitosti/biografiya-petr-nalich.html
\subsection{Биография}
Петр Налич – российский певец боснийского происхождения, первый отечественный музыкант, ставший популярным благодаря YouTube. \explainDetail{Простенький}{простенький}{diminutive of прост\'{о}й} клип с песней «Guitar» на английском языке с акцентом вкупе с балканской мелодикой и дурашливой атмосферой пришелся по душе подавляющему большинству интернет-аудитории в 2007 году.

Пишет музыку для спектаклей, кинофильмов, мультфильмов. Автор песен на вымышленном языке бабурси, в которых, по признанию самого Петра, нет никакого смысла. Представлял со своим коллективом Россию на Евровидении 2010 года в Осло. Первым среди артистов использовал систему «Pay What You Want» для записи дебютного альбома «Радость простых мелодий».

\subsection{Детство, юност, семья}
Петр родился весной 1981 года в семье москвичей Андрея и Валентины Наличей, где уже подрастал старший сын Павел, впоследствии известный художник-оформитель. Андрей Захидович и Валентина Марковна – архитекторы, отец также занимается скульптурой. Его «Лента Мебиуса» расположена у кинотеатра «Горизонт», спортивный приз «Слава» – также его работа. Позже в творческом союзе с младшим сыном, а также Александром и Сергеем Цигалями создал памятник «Сочувствие», посвященный гуманному обращению с бродячими животными и памяти пса по кличке Мальчик, обитавшего в подземном переходе.

\begin{fancyquotes}
    Образцовым ребенком я, конечно, не был, но и хулиганом тоже. Хотя мне всю жизнь хотелось им быть, потому что хулиганы всегда круче, они нравятся девушкам, а я был домашним мальчиком.
\end{fancyquotes}

Именно отец приобщил сыновей к футболу. Несмотря на то, что Петр в детстве был вполне азартным мальчишкой и любил лазать по крышам, кататься с горки, никаким спортом он особо не увлекался, как и брат. И, когда младшему сыну было примерно 11-12 лет, Андрей Захидович решил всерьез заняться его и Павла физическим развитием. Каждую неделю он сажал их в машину, в другую садились двоюродные братья со своими отцами, и всей компанией ехали на станцию Университет. В то время там была неплохая футбольная площадка, на которой они могли гонять мяч бесплатно. Эту традицию, но уже со своими детьми, Петр продолжает и сейчас.

В одном из интервью Налич рассказывал и о своем дедушке-боснийце, у которого был ангажемент в оперном театре Белграда. Во время войны Захид Омерович попал в нацистский лагерь, где ему сломали гортань. Чудом выжив, он переехал в СССР и стал работать на радио. Больше он не пел, но унаследовал любовь к музыке детям, и отец Петра любил исполнять цыганские и русские романсы. Однажды родители спросили у мальчика, какой подарок он хотел бы получить к 14 дню рождения: гитару или конструктор «Лего-Техникс». Последним он просто грезил, его и попросил. Но папа и мама подарили Пете и гитару тоже.

\begin{fancyquotes}
    Я н\'{а}чал играть романсы, песни Цоя и «Наутилуса» вперемешку с цыганскими и казачьими, и этим репертуаром я набрал приличное количество очков в глазах девушек. Оказалось, что необязательно быть хулиганом, гитара тоже неплохо работает.
\end{fancyquotes}

Это во многом определило дальнейший жизненный путь Налича. Еще в школе он создал хард-рок-группу, в которой с удовольствием пел. Петр постоянно занимался музыкой: сначала в детской музыкальной школе имени Николая Мясковского, затем в музыкальном училище при Московской консерватории. Поступив в архитектурный институт, он стал заниматься вокалом в студии «Орфей» у педагога Ирины Мухиной. Окончив МАРХИ, Петр решил продолжить музыкальное образование и в 2010 году пошел учиться оперному пению:

\begin{fancyquotes}
    Я учился у выдающейся певицы Валентины Левко – к сожалению… ушедшей от нас. И огромной, мощнейшей школой для меня стала Оперная студия при РАМ имени Гнесиных. Сейчас она носит имя Сперанского, а когда я поступил, Юрий Аркадьевич Сперанский был еще жив, и мне посчастливилось с ним работать.
\end{fancyquotes}


\subsection{Первый успех: Guitar}
В 2007 году Петр создал сайт и стал выкладывать сочиненные им композиции, а любительский клип с песней Guitar загрузил на YouTube. Певец признавался, что сам не ожидал столь бурной реакции в интернете. На тот момент у него на сайте было порядка сорока композиций, которые мог скачать кто угодно, но лишь после успеха клипа «Guitar» с цыганскими мотивами и красивым голосом Петра (он пел на английском с сильным русским акцентом, который, впрочем, песне шел лишь на пользу) люди стали интересоваться его творчеством.

Осенью того же года состоялся первый сольный концерт певца в столичном клубе «Апшу». Билеты на выступление были раскуплены моментально, ажиотаж вокруг его персоны стоял небывалый. Счастливчики, побывавшие на концерте, тут же выложили в интернет восторженные отклики. В конце года Рунет признал композицию Guitar лучшей за 2007.

Творческая карьера продолжилась созданием «Музыкального коллектива Петра Налича», который для краткости называли МКПН. 2008 год группа посвятила гастролям по России. Тогда же она стала официальным музыкальным сопроводителем российских спортсменов на пекинской Олимпиаде.

Налич вместе с МКПН выпустил первый альбом «Радость простых мелодий». Затем группа стала участником антверпенского фестиваля Sfincs. В репертуаре коллектива звучали композиции «Медовый, аметистовый», «Взгляд твоих черных очей» и другие.

\subsection{Евровидение и новый стиль}
В 2010 году Налич с коллективом отправился в Осло, представлять Россию на «Евровидении». Лирическая баллада «Lost and Forgotten», спетая им на конкурсе Евровидения, не принесла призового места (он занял 11-е), но была тепло принята публикой.

2010 год также был отмечен выпуском нового альбома «Веселые Бабури», а еще через два года был записан диск «Золотая рыбка». На композицию «Сахарный пакет» был снят клип. Параллельно Петр н\'{а}чал исполнять партии в операх. Спектакли «Богема» и «Евгений Онегин», где Налич соответственно пел партии Рудольфа и Ленского, шли в театре-студии оперы при Академии.

В 2013 году вышел следующий альбом «Песни о любви и Родине», записанный в сопровождении оркестра Юрия Башмета. Следом Петр записал диск «Кухня» на выдуманном языке бабурси, созданном внутри МКПН, спел партию Германа в опере «Пиковая дама», которая прошла в Государственном музее Александра Пушкина, и принял участие в театрализованных онлайн-чтениях «Анна Каренина. Живое издание».

В 2015 году Налич распустил коллектив и объявил для него бессрочный отпуск, а сам занялся написанием музыки для спектаклей «Северная Одиссея» и «Питер Пен». Как композитор Налич получил приглашение от Пермского ТЮЗа. Он создал музыку для постановки «Обыкновенное чудо» по пьесе Евгения Шварца.

Затем Петр удивил поклонников новой программой «Утесов и не только…», которую исполнял в сопровождении эстрадно-симфонического бэнда. Впоследствии артист кардинально поменял состав собственного коллектива, пригласив в него музыкантов, способных на высоком профессиональном уровне исполнять совершенно новые ритмы и мелодии. Также он записал новый альбом «Паровоз», а затем выступил с новым репертуаром в клубе «16 тонн».

Не напрасно Налич назвал себя в одном из интервью «троякодышащей рыбой». В 2018 году разноплановый творческий человек снова стал студентом – Петр поступил в Гнесинку на композиторский факультет, вздыхая в интервью, что надо бы уже как-то определиться, кто он есть, иначе как-то нелепо. И тут же добавлял:

\begin{fancyquotes}
    Но пока естественным образом так складывается, что я занимаюсь и оперой, и сочинительством с бэндом, каким бы пестрым оно ни было по жанрам, и написанием инструментальной музыки для театра и других проектов. Теперь музыки будет еще больше.
\end{fancyquotes}


Параллельно с учебой на композиторском отделении в его оперном репертуаре появилась партия Тамино в «Волшебной флейте». Композиторская копилка самого Петра пополнилась музыкой к спектаклю «Тина», а следом к постановке «Горячее сердце», премьера которой состоялась на большой сцене театра имени Евгения Вахтангова. Расширился и оперный репертуар: Налич появился на разных театральных сценах в образах Неморино («Любовный напиток»), Альфреда Жермона («Травиатта»), Луиджи («Плащ»).

Благодаря собранным на краудфандинговой платформе Planeta средствам, был выпущен очередной альбом «Отражения в лужах». С певицей Женей Любич Петр записал яркую песню «Дежавю», а следом вышел новый диск, посвященный царской семье Romanovs100. Этот проект принес Наличу премию Original Music 2019 New York Festival TV \& Film Awards. Еще одна премия – «Онегин» в номинации «Событие года» досталась опере-променаду «Пиковая дама», где Петр исполнил партию Германа.

Значимым культурным событием 2019 года в Москве стало открытие нового театра – «Московского оперного дома». Его открытие сопровождалось премьерой спектакля «Иоланта». Партия Водемона в исполнении Налича была, как всегда, безупречна. Конец года принес Наличу заслуженную награду: он стал лауреатом Первой премии Всероссийского конкурса молодых композиторов «Партитура времени». Диплом был ему вручен в номинации «Сочинение для голоса и фортепиано».



\newpage
\section{Виктор Цой}
% https://ru.wikipedia.org/wiki/%D0%A6%D0%BE%D0%B9,_%D0%92%D0%B8%D0%BA%D1%82%D0%BE%D1%80_%D0%A0%D0%BE%D0%B1%D0%B5%D1%80%D1%82%D0%BE%D0%B2%D0%B8%D1%87
\subsection{Биография и творчество}
Виктор Робертович Цой (21 июня 1962 года, Ленинград --- 15 августа 1990 года, близ посёлка Кестерциемс, Латвийская ССР) --- советский рок-музыкант, автор песен и художник. Основатель и лидер рок-группы «Кино», в которой пел, играл на гитаре, писал музыку и стихи. Снялся в нескольких фильмах.

Виктор Цой родился единственным ребёнком в семье инженера корейского происхождения Роберта Максимовича Цоя и преподавательницы физкультуры Валентины Васильевны. Детство музыканта прошло в окрестностях Московского Парка Победы: он родился в роддоме на Кузнецовской улице (располагается внутри парка; сейчас это кардиоцентр), семья до 1990-х гг. жила в примечательном «генеральском доме» на углу Московского проспекта и улицы Бассейной (сейчас это памятник архитектуры). Некоторое время Виктор учился в близлежащей школе на улице Фрунзе, где работала его мама. В 1973 г. родители Цоя развелись, а через год повторно вступили в брак.

С 1974 по 1977 год посещал среднюю художественную школу, где возникла группа «Палата No. 6» во главе с Максимом Пашковым.
После исключения за неуспеваемость из художественного училища имени В. Серова поступил в СГПТУ--61 на специальность резчика по дереву.
В молодости был поклонником Михаила Боярского и Владимира Высоцкого, позднее Брюса Ли, имиджу которого н\'{а}чал подражать.
Увлекался восточными единоборствами и часто дрался «по-китайски» с Юрием Каспаряном.

\subsection{Смерть}
15 августа 1990 года в 12 часов 28 минут Виктор Цой погиб в автокатастрофе. ДТП произошло на 35 километре трассы «Слока --- Талси» под Тукумсом в Латвии, в нескольких десятках километров от Риги. Согласно наиболее правдоподобной официальной версии, Цой заснул за рулём, после чего его «Москвич-2141» тёмно-синего цвета вылетел на встречную полосу и столкнулся с автобусом «Икарус» модели 250 (иногда этот автобус ошибочно идентифицируют как 280 модель.

\begin{fancyquotes}
    Столкновение автомобиля «Москвич-2141» тёмно-синего цвета с рейсовым автобусом «Икарус-280» произошло в 12 час. 28 мин. 15 августа 1990 г. на 35 км трассы Слока --- Талси. Автомобиль двигался по трассе со скоростью не менее 130 км/ч, водитель Цой Виктор Робертович не справился с управлением. Смерть В. Р. Цоя наступила мгновенно, водитель автобуса не пострадал. ...В. Цой был абсолютно трезв накануне гибели. Во всяком случае, он не употреблял алкоголь в течение последних 48 часов до смерти. Анализ клеток мозга свидетельствует о том, что он уснул за рулем, вероятно, от переутомления.\\

    --- из милицейского протокола; по данным сайта kinoman.net
\end{fancyquotes}




19 августа он был похоронен на Богословском кладбище в Ленинграде.

\subsection{Прочие версии гибели}
Создатели документального кино из цикла «Следствие вели...» предположили, что Цой мог попасть в аварию, когда решил переставить другой стороной кассету в своём магнитофоне, тем самым отвлекшись от движения у «слепого поворота» дороги. Речь в передаче шла о кассете с демозаписью последнего альбома. Гитарист Юрий Каспарян ещё в 2002 году опроверг информацию о наличии этой кассеты в автомобиле Цоя: «Пользуясь случаем, хочу развеять миф, что на месте аварии нашли кассету с демо «Черного альбома»... Все было не так. Я специально приехал в Юрмалу с аппаратурой, с инструментами и мы делали аранжировки для нового альбома. Когда доделали, я забрал кассету и поехал в Петербург. Я приехал утром, вечером узнал о случившемся. И поехал обратно. И кассета все время была у меня в кармане».


\subsection{Творчество}
В конце 1970-х --- начале 1980-х началось тесное общение между Алексеем Рыбиным из хард-роковой группы «Пилигримы» и Виктором Цоем, игравшим на бас-гитаре в группе «Палата № 6», оба они познакомились в гостях у Андрея Панова (Свина), на квартире которого часто собирались компании, а также репетировала его собственная панк-группа «Автоматические удовлетворители».

Виктор Цой и Алексей Рыбин в составе «Автоматических удовлетворителей» ездили в Москву и играли панк-рок-металл на подпольных концертах Артемия Троицкого. Во время аналогичного выступления в Ленинграде по случаю юбилея Андрея Тропилло произошло первое знакомство с Борисом Гребенщиковым

\subsection{Первый альбом}
Летом 1981 года Виктор Цой, Алексей Рыбин и Олег Валинский основали группу «Гарин и Гиперболоиды», которая уже осенью была принята в члены Ленинградского рок-клуба. Вскоре Валинского забирают в армию, а группа, сменив название на «Кино», весной 1982 приступила к записи дебютного альбома. «Кино» под руководством Бориса Гребенщикова записывались на студии Андрея Тропилло в Доме Юного Техника, в записи принимали участие музыканты «Аквариума». Вскоре с ними же «Кино» дали свой первый электрический концерт в рок-клубе, всё выступление шло под драм-машину, а под песню «Когда-то ты был битником» из-за кулис на сцену выскочили БГ, Майк и Панкер. К лету альбом был полностью завершён, продолжительность его звучания составляла 45 минут, откуда и появилось название. Но позже из окончательного варианта была убрана песня «Я --- асфальт», которую можно найти в переиздании «45», где она прилагается в качестве бонус-трека. Запись получила некоторое распространение, о группе заговорили, начал\'{и}сь квартирные концерты в Москве и Ленинграде. Вместе с будущим барабанщиком Зоопарка Валерием Кирилловым осенью этого же года «Кино» записывает в студии Андрея Кускова несколько песен, в том числе «Весна» и «Последний герой», вошедшие в сборник «Неизвестные песни Виктора Цоя» (всего четыре издания).

Тогда запись была забракована и распространения не получила, так как Цой забрал ленту себе.

19 февраля 1983 года проходит совместный электрический концерт «Кино» и «Аквариума», музыканты выступали с тёмным макияжем и в костюмах со стразами. При этом они исполняли «Электричку», «Троллейбус» и «Алюминиевые огурцы». В основной состав был приглашён Юрий Каспарян. Весной из-за разногласий с Цоем Алексей Рыбин покидает группу «Кино». Лето уходит на совместные репетиции с новым гитаристом. В результате этого Виктор Цой и Юрий Каспарян записали альбом «46», который вначале задумывался как демозапись «Начальника Камчатки». Алексей Вишня «скинул» запись нескольким друзьям на плёнку. «46» получил широкое распространение и был воспринят как полноценный альбом. Осенью 1983 года Виктор Цой лёг на обследование в психиатрическую больницу на Пряжке, где провёл полтора месяца, избегая призыва в армию. После выписки из психиатрической клиники он пишет песню «Транквилизатор». Весной выступил на втором фестивале рок-клуба, где группа «Кино» получила лауреатское звание, а песня «Я объявляю свой дом безъядерной зоной», открывшая фестиваль, признана лучшей антивоенной песней фестиваля 1984 года.



\subsection{Второй состав «Кино»}
Летом 1984 года в студии «Антроп» Андрея Тропилло начинается запись альбома «Начальник Камчатки», к которому, кроме Виктора, приложили свою руку БГ и Сергей Курёхин.

В феврале 1984 Виктор и Марьяна празднуют свадьбу. На свадьбу были приглашены Гребенщиков, Майк, Титов, Каспарян, Гурьянов и другие.

Весной 1985 «Кино» заработали ещё одно звание лауреата и засели в студию к А. Тропилло писать «Ночь». Работа над записью затянулась из-за желания создать новую музыку с новыми приёмами игры. Альбом никак не получался, Виктор бросил «Ночь» недоделанной и в студии Алексея Вишни занялся записью «Это не любовь», который получился всего за неделю с небольшим. К осени «Это не любовь» была сведена и удачно разошлась по стране, а в январе 1986 вышла «Ночь», среди песен которой были известные «Мама Анархия» и «Видели ночь». Параллельно с выходом пластинки растёт популярность Виктора Цоя, а в феврале на 4-м фестивале рок-клуба «Кино» получает диплом за лучшие тексты. 5 августа 1985 года у Цоя родился сын Саша.


Летом 1986 года Виктор работал в бане на проспекте Ветеранов, он там мыл помещения из брандспойта. Необходимо было приходить на один час в день, но это было время с 22 до 23 часов, что ему мешало, так как Цой проводил это время суток с группой.

Также летом все участники группы уезжают в Киев на съёмки фильма «Конец каникул» (режиссёр Сергей Лысенко), а чуть позже дают совместный концерт с «Аквариумом» и «Алисой» в ДК МИИТ в Москве, с этими же группами в США выходит «Красная волна». Осенью Сергей Фирсов приглашает Виктора работать кочегаром. Цой соглашается, и они оба начинают работать кочегарами в котельной «Камчатка», откуда выросли многие знаменитые рок-музыканты.

В ней Рашид Нугманов организовал съёмки короткометражки «Йя-Хха», там же проходят съёмки фильма «Рок» Алексея Учителя --- оба фильма при участии Цоя. Осень и зима проходят в Ялте на съёмках «Ассы» Сергея Соловьёва.

Весна 1987 богата концертными событиями: премьера «Ассы» в ДК МЭЛЗ, последнее участие на фестивале рок-клуба, где «Кино» получили приз «За творческое совершеннолетие».

На порто-студии «Yamaha MT44» «Кино» начинают записывать альбом «Группа крови». Осенью 1987 года Виктор улетает к Рашиду Нугманову в Алма-Ату на съёмки своего последнего фильма «Игла», в связи с этим «Кино» доработали «Группу крови» и на время прекратили концертную \explain{деятельность}{activity}. В 1988 выходит «Игла» и «Группа крови», которые породили «киноманию».

Начинаются триумфальные гастроли по Советскому Союзу --- «Кино» собирают аншлаги на всех концертах.

16 ноября 1988 на мемориальном концерте памяти Александра Башлачёва публика ведёт себя крайне активно; по плану концерт должна была заканчивать песня Башлачёва «Время колокольчиков» (в записи), памяти которого был посвящён концерт, но по невыясненным причинам во время выступления Цоя (он играл на гитаре) внезапно включили «Время колокольчиков», Цой прекратил играть, не понимая откуда идёт звук, который он не производит и что вообще происходит. Администрация многократно объявляла, что всем н\'{у}жно расходиться, концерт окончен. Цой не уходил, он несколько раз подходил к выключенным микрофонам и проверял, работают ли. Потом разводил руками --- «не работает», и ходил по сцене туда и сюда с цветком, не уходя со сцены, но и не имея возможности петь и что-то сказать публике. Публика не расходилась, люди шумели, кричали, было видно, что что-то идёт не так. Создавалось впечатление, что некая злая воля решила прекратить концерт и включила финальную песню прямо во время выступления Цоя. Через 10 минут этого противостояния администрация включила микрофон. Цой, в очередной раз подойдя проверять микрофон, услышал что он включён, и объявил людям, что по непонятным причинам несвоевременно была включена финальная песня Саши Башлачева, но после этого петь и играть уже не очень удобно. После этого он стал собираться и публика потянулась к выходу.

Весной 1988 записывается черновик, а зимой окончательный вариант альбома «Звезда по имени Солнце», который решили выпустить осенью. Цой знакомится с Юрием Айзеншписом, который с 1989 стал продюсером «Кино», организовывая концертные туры и частые выступления на телевидении, после чего группа обретает всесоюзную популярность. В день 50-летия Цоя Александр Градский в эфире канала «Москва-24» рассказал, что в тот период Артемий Троицкий инспирировал письмо в Московский Горком, которое должно было настроить московских рок-музыкантов против Виктора Цоя.

На телевидении Виктор Цой дебютировал в программе «Взгляд», об этом рассказано в книге «Взгляд» --- битлы перестройки.

В начале 1989 группа «Кино» впервые едет за границу во Францию, где выпускают альбом «Последний герой». Летом Виктор с Юрием Каспаряном едут в США. Тем временем «Игла» выходит на второе место в прокате советских фильмов, а на кинофестивале «Золотой Дюк» в Одессе Виктора Цоя признают лучшим актёром СССР.

24 июня 1990 года прошёл последний концерт «Кино» в Москве на Большой спортивной арене Лужников. На этом концерте, впервые после московской Олимпиады-80 был зажжён огонь в Олимпийской чаше. После этого Цой с Каспаряном уединились на даче под Юрмалой, где на порто-студию начали записывать материал для нового альбома. Этот альбом, дописанный и сведённый музыкантами группы «Кино» уже после смерти Цоя, вышел в январе 1991 и получил символическое название «Чёрный альбом», с соответствующим оформлением обложки.

\section{Елена Ваенга}
Ел\'{е}на В\'{а}енга (настоящее имя --- Ел\'{е}на Влад\'{и}мировна Хрулёва; род. 27 января 1977, Североморск, Мурманская область, РСФСР, СССР) --- российская \explain{эстрадная певица}{pop singer}, автор песен, актриса. Лауреат премий «Шансон года».

В\'{а}енга --- это название родного для Елены Хрулёвой города Североморска до 18 апреля 1951 года, а также реки недалеко от него. В основе названия и псевдонима --- саамское слово «\explain{олен\'{и}ха}{deer}» (кильд. вайонгг). Псевдоним придуман её матерью.

\subsection{Биография}
Родилась 27 января 1977 года в Североморске. \explainDetail{Петь}{петь}{to sing} и \explain{обучаться}{to study (+ dative)} музыке начал\'{а} с трёх лет.

Мать Елены Ваенги по образованию химик, отец --- инженер, работали в посёлке Вьюжный на \explain{судоремонтном заводе}{shipyard (судно: vessel; ship, plural: суда)} «Нерпа», который обслуживает атомные \explain{подводные лодки}{подводная лодка: submarine}. Про отца и родной Север у Елены Ваенги есть песня:

\begin{fancyquotes}
    {\it У меня глаза северных цветов,\\
        И мне не нужны тропические страны.\\
        Я всегда с тобой рядышком была.\\
        Жаль, что ты уехал слишком рано.\\
        Я вдруг поняла: все эти города\\
        Я должна пройти, как в наказанье.\\
        Но у меня есть дом, а у дома --- я,\\
        А у Севера --- сиянье}
\end{fancyquotes}


Дед Елены со стороны матери --- контр-адмирал Северного флота Василий Семёнович Журавель, он упоминается в книге «Знаменитые люди Санкт-Петербурга». Бабушка Надежда Георгиевна Журавель (её крёстная) (род. 1927). Про неё у Елены Ваенги есть песня: «Моя бабушка любит суши...». Родители отца --- \explain{коренные}{коренн\'{о}й ж\'{и}тель; коренн\'{ы}е ж\'{и}тели: indigenous} петербуржцы, \explain{пережили}{(переживать/пережить) to survive; to experience} блокаду Ленинграда. Дед по линии отца --- зенитчик, во время \explain{Великой Отечественной войны}{second world war} \explainDetail{воевал}{воевать/повоевать}{to fight} под Ораниенбаумом, а бабушка по линии отца была врачом в госпитале в блокадном Ленинграде.

У Елены Ваенги есть младшая сестра Татьяна, она работает в дипломатической сфере, знает несколько языков.

Гражданский муж Елены Ваенги \explain{на протяжении 16 лет}{for 16 years} с 1995 по 2011 год --- Иван Иванович Матвиенко (род. 1957) --- продюсер певицы, по национальности цыган, был женат, его дочь на 2 года старше Елены Ваенги, раньше Иван перегонял машины из Германии.

Племянник, Руслан Сулимовский --- директор её коллектива.

В ночь с пятницы на субботу 10 августа 2012 года Ваенга в родильном доме No. 16 Санкт-Петербурга родила сына Ивана. 30 сентября 2016 года Елена официально вышла замуж за Романа Садырбаева.

\subsection{Творческая деятельность}
Первую песню «Голуби» написала в 9 лет, стала победительницей Всесоюзного конкурса молодых композиторов на Кольском полуострове. После школы приехала в Санкт-Петербург, где закончила музыкальное училище им. Н. А. Римского-Корсакова по классу фортепиано, получив диплом педагога-концертмейстера. Некоторое время преподавала музыку в школе. Факультативно занималась вокалом.

Елена Ваенга с детства мечтала стать актрисой, поэтому после музыкального училища поступила в Театральную академию (ЛГИТМИК) на курс Г. Тростянецкого, но проучилась лишь два месяца, так как её пригласили в Москву записывать первый альбом. Продюсером певицы стал Степан Разин. Под псевдонимом Нина она выпустила клип на песню «Длинные коридоры» (композиция была издана в 2011 году на сборнике «Живая струна»). Альбом был записан, но не вышел. Разочаровавшаяся в шоу-бизнесе певица сбежала от Разина и уехала в Санкт-Петербург. Тем временем её песни взяли в свой репертуар Александр Маршал («Невеста»), Татьяна Тишинская («А ты налей мне белого вина», «Мама, что ты плачешь», «Володенька», «Угостите даму сигаретой»), группы «Стрелки» («Тонкая веточка»), «Божья коровка» («Сердце моё», «Самая любимая моя») и другие известные исполнители. Эти песни распространил её бывший продюсер. Елена Ваенга приняла решение с ним не судиться.

В Санкт-Петербурге Ваенга узнала, что в Балтийском институте экологии, политики и права на кафедре театрального искусства набирает курс П. С. Вельяминов, и в 2000 году пошла учиться к нему. Закончив курс, получила диплом по специальности «драматическое искусство». Выступила в антрепризном спектакле «Свободная пара» в паре с однокурсником Андреем Родимовым (режиссёр Екатерина Шимилёва).

Концертирует с девятнадцати лет. Лауреат петербургского конкурса «Шлягер года 1998» с песней «Цыган», «Достойная песня 2002». Участник концертов-фестивалей «Весна романса» в БКЗ «Октябрьский», «Вольная песня над вольной Невой», «Невский бриз». Дала несколько сольных концертов в ДК имени М.Горького. Гастролирует по России и другим странам и каждый год, в конце января, даёт концерт в БКЗ «Октябрьский» по случаю своего дня рождения.

Настоящая популярность пришла к певице в 2005 году с выходом альбома «Белая птица», в котором было много хитов: «Желаю», «Аэропорт», «Тайга», «Шопен» и заглавная композиция, на которую вышел клип.

28 ноября 2009 года Елена Ваенга получила свой первый приз «Золотой граммофон» за песню «Курю».

4 декабря 2010 года Елена повторила свой успех, получив во второй раз премию «Золотой граммофон» за песню «Аэропорт». В том же году певица впервые стала лауреатом фестиваля «Песня года», исполнив композицию «Абсент». А 12 ноября она дала первый в своей концертной деятельности сольный концерт в Государственном Кремлёвском дворце, трансляция которого прошла на Первом канале 7 января 2011 года. В телеанонсе Елене Ваенга была дана следующая характеристика:
Елена Ваенга --- тонкая, художественная, мечтательная и романтичная натура. Музыкальная одарённость, природный темперамент, трудолюбие, жизнерадостность --- всё это составляющие её жизни и творчества... Несмотря на внешнюю хрупкость и молодость, за спиной у этой очаровательной девушки богатая творческая биография и не такая уж простая человеческая судьба. Жанр, в котором работает певица, с большим трудом определяет даже она сама: «На 50 процентов это фолк-рок, есть старинные баллады, городские романсы, шансон. Но границы между ними провести почти невозможно.»
--- анонс на Первом канале --- «Белая птица». Концерт Елены Ваенги
В 2011 году Елена Ваенга приняла участие в ежегодной церемонии национальной премии Шансон года в Кремле, на которой исполнила песни «Оловянное сердце» и «Девочка». Популярность певицы растёт. В январе этого же года она победила Леонида Агутина в телепередаче «Музыкальный ринг» на канале НТВ, набрав почти в пять раз больше голосов слушателей.

26 ноября 2011 года певица в третий раз получила премию «Золотой граммофон» за песню «Клавиши», но на концерте в Кремле исполнила композицию «Шопен». 21 декабря 2011 года певица в третий раз дала концерт в Кремле. В 2012 году на «Золотой граммофон» претендовали песни «Шопен» и «Где была».

В 2011 году Елена Ваенга дала 150 афишных концертов, гастролировала в США, Германии, Израиле.

Периодически играет в спектакле «Свободная пара», совместно с Андреем Родимовым.

В 2011 году Ваенга впервые попала в список самых успешных деятелей российского шоу-бизнеса, составленный Forbes, и заняла в нём девятую позицию, с годовым доходом более шести миллионов долларов.

В репертуаре певицы --- собственные песни, старинные и современные романсы, баллады и народные песни, а также песни на стихи классиков, таких, как Сергей Есенин («Задымился вечер») и Николай Гумилёв («Жираф», «Шут»). В 2012 году певица провела концертный тур по Украине и Германии. Однако на этом деятельность певицы оборвалась в связи с потерей голоса из-за механического повреждения связок. После выздоровления певица дала последние концерты в городах Средней Волги и ушла в отпуск.

В ноябре 2012 года певица вышла из декрета и возобновила концертную деятельность. По сведениям журнала Forbes, в 2012 году певица в списке самых успешных российских деятелей шоу-бизнеса заняла четырнадцатое место. Сама артистка это отрицает, также как и в прошлом году, утверждая, что её доход гораздо меньше. На данный момент артистка активно гастролирует.

В 2014 году Елена Ваенга стала одним из членов жюри шоу Первого канала «Точь-в-точь».

27 ноября 2015 года состоялся сольный концерт Ваенги в Государственном Кремлёвском дворце, где она выпустила новую программу и представила новый альбом.

\chapter{Наука и Техника}
\section{Северное сияние}
С \explainDetail{наступлением}{наступление}{adventб beginning} осени тысячи туристов \explain{устремляются}{rush} в \explain{Заполярье}{the region of the Arctic circle}, чтобы увидеть уникальный танец небесных огней -- полярное, или северное сияние, на латыни -- Aurora Borealis.
В теории увидеть это природное явление можно с конца августа до середины апреля: в этот период времени ночи становятся темными, солнечная активность \explainDetail{возрастает}{возрастать/возрасти}{возраст\'{а}ю/-ешь/-ет; возраст\'{у}/-ёшь/-\'{у}т: rise, increase}, а облака \explainDetail{расс\'{е}иваются}{рассеиваться/рассеяться}{to disperse}.
Такое развлечение, как \explain{ох\'{о}та}{hunting} за северным сиянием, с каждым годом становится все популярнее как среди россиян, так и среди иностранных туристов, которые специально ради него готовы ехать на Крайний Север. Главное \explain{доказательство}{proof} удачной охоты -- это, конечно же, снимки северного сияния.

\section{Ученые нашли способ записать данные в пяти измерениях}

\textit{Как 5D-диски изменят представление людей о хранении информации?}

\textbf{Ученые создали 5D-диск высочайшей плотности:} В октябре специалисты Саутгемптонского университета в Великобритании описали способ записи огромного количества данных на компактный диск небольших размеров. Технология, получившая название 5D, позволяет сохранить на специальном накопителе до 500 терабайт информации. Получившиеся диски из кварцевого стекла отличаются высочайшей плотностью, которая в десять тысяч раз превышает плотность оптических дисков Blu-Ray. Новый метод позволит эффективно разместить на небольшой площади облачные сервера для хранения данных пользователей, интернет-компаний, крупных корпораций. По словам ученых, это особенно важно на фоне развития технологий, увеличения количества подключенных к сети устройств и роста количества передаваемых через сеть данных.

\textbf{Облачные сервисы с каждым годом становятся все популярнее:}
За последние пять лет отношение потребителей и бизнеса к облачным сервисам изменилось. Раньше их воспринимали в качестве дополнительного метода резервного копирования данных — информация практически всегда поступала в одну сторону. Причем крупные корпорации в основном использовали дата-центры для хранения некритичной информации. К 2020-м годам организации стали использовать облачные серверы не только для аварийного копирования, но и для постоянного обмена данными внутри конкретного предприятия. Системы облачных хранилищ стали более гибкими, позволяя конкретному потребителю выбрать необходимое количество свободного места и производительность оборудования.

Специалисты Analytics Insight называют основными \explainDetail{преим\'{у}ществами}{преим\'{у}щество}{advantage} \explainDetail{облачных}{облачный}{cloud (adj.); \'{о}блако: cloud} дата-центров \explain{круглос\'{у}точный}{round the clock} доступ к информации, возможность одновременной работы не- скольких пользователей с одним массивом данных, масштабируемость и \explain{г\'{и}бкость}{flexibility}, \explain{снижение}{decline} затрат на хранение данных внутри компании.

\explainDetail{Представители}{представитель}{representative} отрасли отмечают, что в обычное время нагрузка на серверы неравномерна: в одной части дата-центров она может зашкаливать, в другой быть крайне небольшой. По этой причине эксперты предсказывают появление искусственного интеллекта, который мог бы анализировать и распределять нагрузку на оборудование. В том числе по этой причине данные пользователей хранятся в нескольких частях дата-центра.

\textbf{Больше всего в облачных сервисах пользователи ценят скорость передачи данных и безопасность:} По словам основателя облачного провайдера Wasabi Дэйва Френда, от дата-центров будущего потребители ожидают высокого уровня безопасности, производительности оборудования и приемлемой цены за услуги. «Резервные копии должны храниться в разрозненных системах, обеспечивающих максимально возможную изоляцию», — заметил предприниматель. Потенциальный злоумышленник, добравшийся до одного сервера, не должен иметь возможность удалить или зашифровать информацию так, чтобы ее нельзя было восстановить из альтернативных источников. Френд полагает, что на этом должна строиться концепция мультиоблака.

Другими критериями облачного сервиса будущего, по мнению Френда, являются доступная цена и высокая скорость передачи данных. Провайдеры должны будут таким образом скорректировать стоимость услуг и добиться определенного качества оборудования, чтобы оставаться конкурентоспособными и не разочаровывать клиентов.

Представители облачного провайдера CloudSigma рассказали, что дата-центры должны будут отвечать за сохранность данных и скорость передачи информации. Для хранения файлов пользователей и корпоративных клиентов они используют небольшие 2,5-дюймовые диски емкостью 250 гигабайт. В случае, если какой-либо диск выходит из строя, его заменяют, а данные восстанавливают через бэкап. При таком развитии событий клиент не теряет своих данных, хотя и оказывается без доступа к информации на 10-15 минут. Благодаря глубокой интеграции между серверами и оборудованием задержка передачи данных внутри дата-центра очень мала. Для того чтобы разогнать скорость и снизить задержку между серверами и пользователем, в компании полагаются на выделенную гигабитную линию интернета.


\textbf{Диски 5D позволят хранить информацию практически бесконечно:}
По оценке Forbes, к 2025 году к интернету будет подключено около 80 миллиардов устройств, которые будут генерировать около 180 триллионов гигабайт данных. В обозримом будущем хранить данные на классических накопителях будет проблематично — существует риск возникновения дефицита и увеличения стоимости хранения информации. Работающие над технологией 5D специалисты Саутгемптонского университета предлагают записывать информацию на кварцевом стекле с помощью фемтосекундных лазеров и сверхкоротких импульсов. «Запись на кварцевый носитель как бы идет в пяти измерениях — двух оптических и трех пространственных», — отмечают авторы исследования.

Инновация британских инженеров заключается в создании дисков повышенной плотности и размещении на небольшом участке колоссальных объемов данных. Например, на «болванке» размером в один дюйм удалось сохранить шесть гигабайт информации. Накопитель обычного для подобных устройств размера, основанный на кварцевых дисках, может сохранить до 500 терабайт данных. Разработка обещает революцию на рынке хранения информации, так как десятки, если не сотни классических дата-центров можно будет объединить в одну библиотеку.

Преимуществами 5D-дисков также называют долговечность и низкую стоимость обслуживания. По оценке ученых, кварцевые диски не прочнее обычных накопителей, однако могут выдержать температуру до 1800 градусов по Фаренгейту, или около тысячи градусов по Цельсию. В случае пожара в дата-центре информация, скорее всего, сохранится. Кроме того, кварцевое стекло со временем не меняет своих свойств, что позволит держать данные на 5D-накопителях практически вечно.

Единственным узким местом будущей разработки является скорость передачи данных. В настоящий момент ученым удалось разогнать ее до 230 килобайт в секунду — за это время на диск можно записать около ста страниц текста. Однако для того, чтобы полностью заполнить болванку емкостью 500 терабайт, потребуется 60 дней. Либо инженеры найдут способ обойти ограничение, либо 5D-диски так и останутся перспективным оборудованием для записи информации. В крайнем случае на подобных дисках можно сохранять данные для потомков. Так, в 2018 году на кварцевый носитель была записана трилогия романов Айзека Азимова «Основание» — диск отправился в космос вместе с Tesla Roadster Илона Маска.
% !TeX root = ../russian-vocab-2021-22.tex

\chapter{Новосты}

\section{Ее можно считать жертвой}
% https://lenta.ru/articles/2021/12/22/kinder/
Отец девятилетней студентки МГУ \explainDetail{напал}{нападать/напасть}{attack} на людей в вузе. Что о семье Тепляковых думают педагоги?

У девочки-вундеркинда Алисы Тепляковой, которая в восемь лет \explain{сдала ЕГЭ}{сдавать/сдать экзамен} и поступила на платное отделение психфака МГУ, началась первая сессия. Во вторник, 21 декабря, появилось видео, где ее отец Евгений Тепляков пытается прорваться через пост охраны факультета с криками, что он «хочет поговорить с преподавателями». Администрация факультета вынуждена была прятать их от агрессивного родителя. Еще в сентябре, когда стало известно, что Алиса станет студенткой МГУ, «Лента.ру» попросила прокомментировать это событие известных педагогов. Все они сомневались, что для психики маленького ребенка учеба в вузе будет посильной задачей. Возможно, опасения начинают сбываться. Мы публикуем их мнения об Алисе и методах ее отца.

\subsection{Будет не столько студенткой, сколько подопытным объектом}
\textit{Леонид Кацва, автор учебников и пособий по истории России. Преподаватель московской школы № 1543}

Я смотрел видеоинтервью с Алисой Тепляковой. У нее, видимо, очень тренированная память. Каких-то других качеств она не показала в выступлении. Разговаривает она как семилетка, уровень ее понимания ситуации --- типичный для маленького ребенка. У нас в школе на педсовете перед началом учебного года говорили об этом, многие считают, что вся эта история очень дурно пахнет. Имеется в виду не то что девочка сдала ЕГЭ --- вызубрить какие-то вещи по нескольким предметам на минимальный балл она могла, если у нее действительно вот такая память. Я видел, как она читает --- быстро, но судя по всему общего смысла текста не понимает. Папа дрессировал детей именно на скорочтение. А скорочтение --- это немного не про чтение в том смысле, как мы его понимаем.

У меня нет вопросов к папе. Он хочет доказать некую идею --- что можно в школе не учиться 11 лет, а освоить все за три года. К девочке у меня тоже вопросов нет. Потому что в данном случае она --- орудие в руках папы, в какой-то мере ее даже можно считать жертвой.

У меня есть вопрос к МГУ: принять девятилетнего ребенка на психфак --- это надо все же сильно постараться.  Но гораздо больше у меня вопросов к школе,  которая ее  выпустила. Я ничего про эту школу не знаю. Даже не знаю номера. Видимо, она была на домашней форме обучения, на уроки не ходила. Не знаю, как это было оформлено --- экстернат или домашнее обучение, этого не могу сказать. Но если она в восемь лет сдала экзамены за все годы обучения, то, грубо говоря, девочка должна была с шести лет сдавать экзамены каждые два-три месяца. Это если их принимали.

Папа говорит совершенно открыто, что девочка не прочитала ни одного программного литературного произведения, что она знакомилась с художественными книгами в виде кратких пересказов. Я понимаю, что так некоторые дети и делают, даже в 17-летнем возрасте. Но все-таки это принято скрывать, а не превращать в манифест.

Я 40 лет преподаю историю и, как говорится, зуб даю, что если ребенка начать спрашивать не на уровне тестов, кто командовал теми-то войсками, кто был генеральным секретарем тогда-то, министром, великим князем тогда-то, а начать спрашивать всерьез, с причинно-следственными связями, с характеристиками событий, то не о чем будет говорить. Специалисты по естественным наукам и физике также замечают, что даже на уровне физиологии не может ребенок в таком возрасте эти дисциплины качественно осваивать.

У нас были вундеркинды. И я знаю случаи, когда в вуз приходил учиться 14-летний студент. Однако разница между 14 и 17 годами, когда положено сдавать ЕГЭ, на порядок меньше. Я уж не говорю о разнице между 17 годами и девятью. Поэтому я в данном случае вижу какую-то недобросовестность с разных сторон. И прежде всего --- школы. Возможно, она просто решила подыграть папе, не знаю почему. Либо просто отвязаться от этого папы. Потому что папа такой, что проще согласиться на его условия, чем объяснять ему, почему этого делать не стоит. Но, с другой стороны, есть и контраргумент, почему это может быть не так. За Алисой --- на подходе очередь из ее братьев и сестер. Причем если у Алисы имя обычное --- среди девочек школьного возраста Алисы встречаются, то у остальных детей в семье имена скандинавских богов, а не детей из России. И тут у меня ощущение, что психологическое состояние папы от старшего ребенка к младшим начало усугубляться.

На мой взгляд, тут широкое поле деятельности для Рособрнадзора. Не думаю, что тут речь о мошенничестве при ЕГЭ --- ребенок с натренированной памятью мог рассчитывать на минимальные баллы, чтобы экзамен считался сданным. Но полноценное среднее образование она получить не могла. Девочка под папиным внушением говорит: в школе 11 лет учатся, а в институте --- пять, значит, институт --- проще, я его окончу за два года. Эти слова ребенка цитируют СМИ. Предположим, она окончит институт в 11 лет. Вы пойдете на консультацию к такому специалисту-психологу? Я, честно говоря, остерегусь.

\begin{fancyquotes}
    Это моя гипотеза --- и кроме догадок она ни на чем не основана, --- что на психфак ее приняли не столько для того, чтобы обучать как полноценного студента, сколько для того, чтобы ставить своего рода эксперимент. То есть в этой ситуации она будет не столько студенткой, сколько подопытным объектом, потому что с точки зрения психологии в ее развитии есть какие-то аномалии --- скорее всего положительные, а может быть, и не только
\end{fancyquotes}

Папа говорит, что учителя, которые заставляют свободно читающего ребенка по слогам произносить ма-ма мы-ла ра-му, --- преступники. У него все --- преступники, один он --- молодец. Совершенно понятно, что папа преследует какие-то цели. Не могу сказать, что они материальные, по-видимому, он хочет прославиться, стать великим реформатором образования или кем-то еще в этом роде. Но мне кажется, что эти эксперименты очень опасные.

В моей практике были дети, которые перескакивали через класс --- из шестого в восьмой, из восьмого в десятый. Таких случаев у меня было, если не ошибаюсь, три. Эти ситуации на состояние детей оказали скорее отрицательное влияние, чем положительное. Ребята были развитые, скучали в классах по возрасту, но когда их перевели на год вперед, они совершенно потерялись. Мне кажется,  что так делать не надо.

У меня есть дети, которые очень одарены математически и учатся в математическом классе. Они становились призерами Всероссийских олимпиад, но это не повод считать, что во всем остальном дети так же одарены. Знаете, как говорил великий русский поэт Козьма Прутков: «Специалист подобен флюсу, полнота его односторонняя». Допускаю, что талантливые дети могут оканчивать школу, допустим, не за 11 лет, а за девять. Но в то, что ребенок может окончить школу в 9 лет, --- не верю.

\subsection{Слишком умных учеников частенько боятся}
\textit{Леонид Перлов, почетный работник общего образования России, много лет преподавал географию в одной из лучших математических школ страны Лицей «Вторая школа»}

В обычных школах слишком умных учеников частенько просто боятся. Потому что учитель --- живой человек. Он понимает, когда у него не получается, и не понимает --- почему. А не получается просто потому, что он раньше мог не иметь дела с такими детьми. Или школьная администрация от него требует одно, а ребенку нужно совершенно другое. И как найти в этом приемлемую середину --- очень сложный вопрос.

С такими ребятами действительно трудно, ничуть не легче, чем с детьми с аутическим компонентом, с другими особенностями развития.

Просто здесь трудности другого рода. Учитель должен очень много знать не только в области своей математики, географии или литературы, а именно в области педагогики. Эти дети больше требуют, они иначе воспринимают действительность, способны быстро анализировать действия того же самого учителя и показать ему, прав он или нет в той или иной ситуации. Им очень много надо от учителя, а учитель далеко не всегда в состоянии им это дать. Нужна другая манера общения с ребенком. И грань между жесткостью и фамильярностью учителю помогает установить только опыт.

Педагогика --- не наука. Это синтез искусства и ремесла. И в контакте с каждым конкретным учеником педагог работает так, как этому конкретному ученику требуется. Естественно, если школа предоставляет педагогу такую возможность, если он не вынужден как большинство учителей трудиться на полторы-две ставки. В моей «Второй школе» у учителей такая возможность есть.

Сейчас упор делают на математической одаренности, спортивной, музыкальной. Да и собственно --- все. Других одаренностей стандарт не предусматривает. А на самом деле этих одаренностей --- миллион. Ребенок вполне может быть талантлив в чем-то, чего пока еще не проявил. И сам может о своей способности не догадываться.

Одна из задач квалифицированного педагога --- выявить эту одаренность. А вот что у ребенка здорово? Ну вот он дуб дубом в математике и совершенно не интересуется химией. Но зато он пальцами чувствует, как из куска пластилина вылепить медведя. Его никто никогда этому не учил, но у него  прекрасно получается. Или, например, он педагогически одарен и обожает возиться с младшими своими товарищами. И у него отлично получается: они его слушают, они его обожают, они на нем виснут. Это одаренность? Думаю --- да.

Но школа сегодня не имеет задачи выявить талант у каждого. Главная задача школы --- выполнение стандарта. Все, наверное, слышали о федеральном государственном стандарте. Подразумевается, что он --- некий эталон, на который нужно равняться.
Для работы с детьми высоко мотивированными, грамотными, желающими учиться необходимо отклониться от этой нормы. Норма не рассчитана на повышенный уровень образования, в первую очередь она не может удовлетворить требований со стороны ученика. Стандарты «отклонения» не приветствуют. Кроме того, отклонение в любую сторону --- хоть в сторону повышенных потребностей со стороны ученика, хоть в сторону работы с детьми с особенностями развития --- все это требует особой, соответствующей квалификации учителей. Действующий профессиональный стандарт учителя подразумевает, что педагог обязан работать с любыми детьми в любых условиях. Хотя его никто и никогда не учил этому.

\begin{fancyquotes}
    Для родителей часто ребенок, скачущий со ступеньки на ступеньку в школе, побеждающий в олимпиадах, --- предмет гордости, повод свысока поглядывать на коллегу по работе или на соседей. Но ребенку эти успехи не всегда приносят радость. Рано или поздно родители начинают ему говорить: «Вот ты занял второе место, а почему не первое? А ну-ка, поработай еще!»
\end{fancyquotes}

Дети, которые перепрыгивали через классы, были и 20 лет назад, и сто лет назад. Но ничего хорошего, как правило, из этого не выходит. Всему свое время, в том числе и детству. Думаю, что и на этот раз исключением эта девочка не станет. Конечно, для таких детей нужен особый подход. Ей нужны знания, соответствующие ее развитию и способностям. Но это вовсе не курсы ЕГЭ по русскому и математике. Подготовительные курсы к ЕГЭ --- это называется дрессировка. Медведь вон ездит в цирке на велосипеде. Но, во-первых, он не знает, что это неприятно. А во-вторых, совершенно не понимает, что для него --- медведя --- это нехорошо.

Все же взрослым нужно поаккуратнее подходить к этим вопросам и в первую очередь выяснить --- это им так кажется, или сам ребенок ощущает, что у него к чему-то талант и он готов в этом направлении развиваться. Очень часто ощущения родителей и детей не совпадают. Например, родители считают, что ребенок математически одаренный, а он мечтает играть на кларнете и в любую свободную минуту летит к инструменту, потому что это ему по-настоящему нравится. При этом он занимается математикой, олимпиадник и так далее, но только потому, что он послушный ребенок. У меня такие случаи были. Ребенок --- член команды Москвы по шахматам со всеми разрядами, подающий очень большие надежды. В девятом классе мальчик сказал родителям, что на юниорский чемпионат он не поедет и эту страницу своей биографии закрыл. Он намерен поступать на мехмат МГУ, а значит, оставшиеся до окончания школы два года будет заниматься именно этим. Скандал был нереальный. Но парень, надо отдать ему должное, выдержал.


\subsection{Ведущая деятельность ребенка --- игра, отец заменяет ее учебой}
\textit{Александр Снегуров, заслуженный учитель России, кандидат психологических наук}

Корреспонденты обратились ко мне за комментарием феномена, я высказал свое мнение. Это были корреспонденты телеканала «Россия 24». А потом мне сообщили, что девочку опросили --- есть ролик с выступлениями ее отца, который я не смог посмотреть. Так вот, ребенку задали ряд вопросов, и выяснилось, что она не знает каких-то тривиальных вещей, после чего выпуск сюжета отменили.

Да, неудобно говорить о ее достижениях, когда она не знает обычных вещей. А я это допустил еще до ее опроса. Потому что тут налицо диссонанс и нарушение, так скажем, динамики по всем направлениям развития. Обязательно что-то будет отставать. В случае этих вундеркиндов это практически все время имеет место. Знаете ведь, как тесто замешивают? А потом уже от замешанного теста можно взять кусочек и потянуть вверх. Мы можем вытянуть достаточно высоко. Так же и с детьми. Представим, что мы вытянули один показатель из общей палитры развития ребенка. В данном случае --- сегмент ее интеллектуальной составляющей. А социализация и психологическое развитие, оставшиеся в этом тазике, соответствуют возрасту. А мы-то хотим ориентироваться на высоту вытянутого сегмента, чтобы остальное ему соответствовало.

А потом удивляются, почему у таких детей стрессы и изломанные судьбы, а в лучшем случае возвращение к типичному пейзажу в привычный ландшафт сверстников. Подобное ускоренное обучение чревато этими факторами. Я не буду говорить, что это дурно, просто обозначаю. Ты даешь обучение, но надо понимать, что миновать вид детства не рекомендуется. Нужно понимать, что ведущая деятельность ребенка --- игра, а в данном случае ее отец заменяет ее учебой, но психика ребенка настроена на эту смену не сейчас.

Да он говорил, что она гуляет и играет, но это все сочетается, может быть, только на его взгляд. А может быть, это всего лишь имитация сочетания. А потом выявится большой диссонанс, который обнаружится внезапно. Допустим, ребенок вдруг чего-то не захочет, например, жить на свете или находиться среди этих людей. Вдруг он может заявить своим родителям --- вы меня измучили и достали. Этого не стоит исключать. Я не говорю, что это может произойти в обязательной перспективе, но исключать этого нельзя, и стоит иметь это постоянно в виду.

Я склонен к позиции, что ее натаскали на сдачу ЕГЭ, при этом не отрицая все-таки ее интеллектуальных возможностей. Однако в таком случае игнорируется --- хотя не должен --- опыт взросления. Ряд школьных предметов основан именно на постепенном взрослении, к примеру, литература и история. Ну что она, «Войну и мир» прочитала?

Она поступила на психфак, не созрев и не испытав этапов взросления, которые нужно пройти человеку. Хоть и говорят, что мы, психологи, работаем с опытом другого человека, однако если у тебя есть собственный опыт, это точно не помешает. Я не говорю, что каждый должен пройти через большую драму или катастрофу и тогда он сможет работать психологом. Но, разумеется, какие-то приблизительные ощущения и переживания должны быть, чтобы работа была успешной.

Рост психологический и физиологический обязательно должен отразиться на психике и сознании ребенка, чтобы была полноценная картина, а этого, по-моему, не произошло, ведь отец мог интеллектуально ее подгонять. Но само созревание… Да, может, у нее идут и эти процессы быстрее, но это еще не значит, что они соответствуют ее интеллектуальным свершениям. А там, может быть, и свершения интеллектуальные не по всем пунктам. Интересно было бы побеседовать с ребенком не в рамках ЕГЭ, а в рамках общей эрудиции и взглядов на мир. Сложилось ли у нее мировоззрение, а не набор фактов, который опирается на память. Но складывание и формирование мировоззрения --- другая вещь. И жизненный опыт в том числе.

Они и в вузе хотят ускорить обучение, чтобы она окончила его в 11 лет. Задаюсь вопросом: работать она не может, а значит может быть провисание, которое, не исключено, может привести к экзистенциальному кризису: а зачем вы это сделали со мной? А что мне теперь делать, а где мое детство? Это один вариант. Другой --- она сама вернется в свою детскую парадигму, ну и как-то все в общем пейзаже сравняется.

Сам я сторонник отмены ЕГЭ, но это отдельная беседа. Этот экзамен проверяет память и приспособленность ребенка к определенным процедурам, а это отнюдь не весь спектр потенциала ребенка. У кого-то память не так хороша, у кого-то лучше. Много у меня опций критического характера по этому вопросу. Однако я за различные сценарии итоговой аттестации.

\begin{fancyquotes}
    Возможен и драматичный вариант. Редкий случай, что отец постоянно будет подкидывать ей поленья новых задач, чтобы она не простаивала. Но вряд ли это получится в той динамике, которую он задал. Однако до какого-то возраста он может это делать, потом же этот механизм у ребенка-вундеркинда даст сбой
\end{fancyquotes}

По поводу детей-вундеркиндов я считаю, что родители развивают уже имеющуюся почву. Можно сказать, что это дар или наказание, но в любом случае это данность, которую родители заметили и начали развивать доступными способами. Это нетипичный пейзаж, который можно называть своеобразным нарушением или отклонением от нормы. Обижать не станем, но примеры многих талантливых людей, увы, подтверждают эту позицию.







\end{document}
