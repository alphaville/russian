\documentclass[letterpaper,11pt]{book}

\usepackage[utf8]{inputenc}
\usepackage[english,russian]{babel}
\usepackage{color}
\usepackage[dvipsnames,svgnames]{xcolor}
\usepackage[a4paper, margin=1in]{geometry}
\usepackage[hang,flushmargin]{footmisc}
\usepackage{dblfnote}
\usepackage{enumitem}
\RequirePackage{cmap}
\usepackage[OT2,T1]{fontenc}

\usepackage{lipsum}
\usepackage{tikz}
\usetikzlibrary{backgrounds}
\makeatletter

\tikzset{%
  fancy quotes/.style={
    text width=\fq@width pt,
    align=justify,
    inner sep=1em,
    anchor=north west,
    minimum width=\linewidth,
  },
  fancy quotes width/.initial={.8\linewidth},
  fancy quotes marks/.style={
    scale=5,
    text=white,
    inner sep=1pt,
  },
  fancy quotes opening/.style={
    fancy quotes marks,
  },
  fancy quotes closing/.style={
    fancy quotes marks,
  },
  fancy quotes background/.style={
    show background rectangle,
    inner frame xsep=0pt,
    background rectangle/.style={
      fill=gray!25,
      rounded corners,
    },
  }
}

\newenvironment{fancyquotes}[1][]{%
\noindent
\tikzpicture[fancy quotes background]
\node[fancy quotes opening,anchor=north west] (fq@ul) at (0,0) {``};
\tikz@scan@one@point\pgfutil@firstofone(fq@ul.east)
\pgfmathsetmacro{\fq@width}{\linewidth - 2*\pgf@x}
\node[fancy quotes,#1] (fq@txt) at (fq@ul.north west) \bgroup}
{\egroup;
\node[overlay,fancy quotes closing,anchor=east] at (fq@txt.south east) {''};
\endtikzpicture}

\makeatother

\DFNalwaysdouble % for this example

\usepackage{hyperref}

\hypersetup{
    bookmarks=true, unicode=false,
    pdftoolbar=true, pdfmenubar=true, pdffitwindow=false,
    pdfstartview={FitH}, pdftitle={Russian Vocabulary},
    pdfauthor={P Sopasakis},
    pdfkeywords={}, pdfsubject={Confidential}, 
    pdfnewwindow=true, colorlinks=true, linkcolor=red,
    citecolor=cyan, filecolor=magenta, urlcolor=RoyalBlue
}

\usepackage{parskip}
\setlength{\parindent}{0pt}

\newcommand{\red}[1]{\textcolor{red}{#1}}
\newcommand{\comment}[1]{\textcolor{gray}{\it #1}}
\newcommand{\explain}[2]{\red{#1}\footnote{#1{}:{} #2}}
\newcommand{\explainDetail}[3]{\red{#1}\footnote{#2{}:{} #3}}


\newenvironment{dialogue}{\list{---}{\it \itemsep=0.05\parskip \topsep=0.75\parskip \parsep=0.75\parskip}}{\endlist}


\title{Сборник текстов\\{\small Газетные и Журнальные Статьи. Тексты об искусстве, истории, музыке, и новостях.}}
\author{Пантелис Сопасакис}

\begin{document}
\maketitle
\tableofcontents

\chapter{Искусство}

\section{Художники}
\subsection{В\'{и}ктор Мих\'{а}йлович Васнец\'{о}в}
% https://muzei-mira.com/biografia_hudojnikov/765-viktor-mihaylovich-vasnecov-biografiya.html
Виктор Михайлович Васнецов родился в 1848 году 15 мая в селе со смешным названием Лопьял. Отец Васнецова был священником, также как и его дед и прадед. В 1850 году Михаил Васильевич увёз семью в село Рябово. Это было связано с его службой. У Виктора Васнецова было 5 братьев, один из которых также стал знаменитым художников, звали его Аполлинарий.

Талант Васнецова проявился с детства, но крайне неудачное \explainDetail{денежное}{д\'{е}нежный/-ая/-ое}{monetary} положение в семье не оставило вариантов, как отдать Виктора в Вятское духовное училище в 1858 году. Уже в 14-летнем возрасте Виктор Васнецов учился в Вятской духовной семинарии. Детей священников туда брали бесплатно.

Так и не окончив семинарию, в 1867 году Васнецов отправился в Петербург поступать в Академию художеств. Денег у него было совсем мало, и Виктор выставил на «аукцион» 2 свои картины -- «Молочница» и «Жница». До \explainDetail{отъезда}{отъезд}{departure} он так и не получил за них денег. 60 рублей за эти две картины он получил спустя несколько месяцев уже в Петербурге. \explainDetail{Прибыв}{прибывать/прибыть}{arrive} в столицу, у молодого художника было всего 10 рублей.

Васнецов отлично справился с экзаменом по рисованию и сразу \explain{был зачислен}{was enrolled (\textit{гл. св.} зачислить: to enrol)} в Академию. Около года он занимался в Рисовальной школе, где и познакомился со своим учителем -- И. Крамским.

К занятиям в Академии художеств Васнецов приступил в 1868 году. В это время он \explain{сдружился с}{made friends with} Репиным, и даже одно время они жили на одной квартире.

Хоть Васнецову и нравилось в Академии, но он её не закончил, уехав в Париж в 1876 году, где прожил больше года. В это время там же находился и Репин в \explainDetail{командировке}{командировка}{business trip}. Они также поддерживали дружеские отношения.

После возвращения в Москву Васнецова сразу приняли в Товарищество передвижных художественных выставок. К этому времени стиль рисования художника значительно меняется, да и не только стиль, сам Васнецов перебирается жить в Москву, где сближается с Третьяковым и Мамонтовым. Именно в Москве Васнецов \explainDetail{раскрылся}{раскрыв\'{а}ться/раскр\'{ы}ться}{to open, uncover oneself, to come out}. Ему нравилось находиться в этом городе, он чувствовал себя легко и \explainDetail{выполнял}{выполн\'{я}ть/в\'{ы}полнить}{to perform, execute, carry out} различные творческие работы.

Более 10 лет Васнецов \explainDetail{оформлял}{оформл\'{я}ть/оф\'{о}рмить}{put into shape, form} Владимирский соб\'{о}р в Киеве. В этом ему помогал М. Нестеров. Именно после окончания этой работы, Васнецова можно по праву назвать великим русским иконописцем.

1899 год стал \explainDetail{пиком}{пик}{peak} популярности художника. На своей выставке Васнецов представил публике «Трёх богатырей».

После революции Васнецов стал жить уже не в России, а в СССР, что его серьёзно \explain{угнетало}{угнетать}{opress, depress, despirit}. Люди \explainDetail{уничтожали}{уничтож\'{а}ть/уничт\'{о}жить}{to destroy, obliterate; уничтож\'{е}ние: destruction} его картины, \explainDetail{относились}{относ\'{и}ться + \textit{дат.}}{to treat} неуважительно к художнику. Но до конца своей жизни Виктор Михайлович был в\'{е}рен своему делу -- он рисовал. \explainDetail{\'{У}мер}{умирать/умереть}{умир\'{а}ю, умир\'{а}ешь, умир\'{а}ют; умр\'{у}, умрёшь, умр\'{у}т: to die} он 23 июля 1926 года в Москве, так и не закончив портрет своего друга и ученик\'{а} М. Нестерова.


\section{Произведения}
\subsection{Картина «Три Богатыря» ВМ Васнец\'{о}ва}
% https://muzei-mira.com/kartini_russkih_hudojnikov/1321-opisanie-kartiny-bogatyri-tri-bogatyrya-vasnecova-1898.html

Картина В\'{и}ктора Мих\'{а}йловича Васнец\'{о}ва «Богатыр\'{и}» \explain{по пр\'{а}ву}{rightfully } считается настоящим народным \explainDetail{шед\'{е}вром}{шед\'{е}вр}{masterpiece} и с\'{и}мволом отечественного искусства. \explainDetail{Создав\'{а}лась}{создаваться/создаться}{create} картина во второй половине XIX века, когда среди русских художников был\'{а} очень популярна тема народной культуры, русского фольклора. Для многих художников это увлечение оказалось кратковременным, но у Васнецова народная фольклорная тематика \explainDetail{ст\'{а}ла}{стать/становиться}{become (ст\'{а}л/-а/-о)} \explainDetail{осн\'{о}вой}{осн\'{о}ва}{basis} всего \explainDetail{творчества}{творчество}{cretativity}.

На картине «\explainDetail{Богатыр\'{и}}{богат\'{ы}рь}{A bogatyr or витязь is a stereotypical fictional character in medieval Russian legends}» \explainDetail{изображен\'{ы}}{изображён/-\'{а}/-\'{о}}{depicted; изображение: image, depiction; изображ\'{а}ть/изобраз\'{и}ть: to depict} три русских богатыр\'{я}: Илья Муромец, Добрыня Никитич и Алёша Попович - знаменитые герои народных \explainDetail{был\'{и}н}{был\'{и}на}{epic}.

\explainDetail{Испол\'{и}нские}{испол\'{и}нский}{gigantic} фигуры богатыр\'{е}й и их коней, распол\'{о}женные \explain{на пер\'{е}днем пл\'{а}не}{in the foreground (пер\'{е}дний пл\'{а}н)} картины, символизируют силу и мощь русского народа. Этому \explainDetail{впечатлению}{впечатл\'{е}ние}{impression} \explainDetail{спос\'{о}бствуют}{способствовать/поспособствовать}{contribute to} и \explainDetail{внушительные}{внушительный}{impressive} размеры картины -- 295$\times$446 см.

Над созданием этой картины художник работал почти 30 лет. В 1871 году был с\'{о}здан первый \explain{набр\'{о}сок}{sketch} сюжета в карандаш\'{е}, и с тех пор художник увлёкся идеей создания этой картины. В 1876 году был сделан знаменитый \explain{эск\'{и}з}{sketch} с уже \explainDetail{н\'{а}йденной}{н\'{а}йденный/-ая/-ое}{$<$ найти} основой композиционного решения. Работа над самой картиной длилась с 1881 по 1898 год. Готовая картина была куплена П. Третьяковым, и до сих пор она \explainDetail{украш\'{а}ет}{украш\'{а}ть/укр\'{а}сить}{decorate} Государственную Третьяковскую галерею в Москве.

В центре картины изображён Илья Муромец, народный любимец, герой русских был\'{и}н. Не всем известно, что Илья Муромец не сказочный персонаж, а реальное историческое лицо. История его жизни и \explainDetail{р\'{а}тных}{р\'{а}тный}{military} \explainDetail{п\'{о}двигов}{п\'{о}двиг}{exploit, feat} -- это реальные события. \explain{Впосл\'{е}дствии}{subsequently}, закончив свои труды по охране родины, он стал монахом Киево-Печёрского монастыр\'{я}. Был причислен к лику святых\footnote{was canonised}. Васнецов эти факты знал, создав\'{а}я образ Ильи Муромца. «\explainDetail{Матёр}{матёрый}{mature, fully grown, hardened} человек Илья Муромец» -- говорит былина. А на картине Васнецова мы видим могучего воина и при том \explainDetail{бесх\'{и}тростного}{бесх\'{и}тростный}{ingenuous, silly} открытого человека. В нём \explainDetail{сочетаются}{сочетаться}{combine} исполинская сила и \explain{великод\'{у}шие}{generosity, magnanimity, goodness}. «А конь под Ильёй \explain{лютый}{fierce} зверь» -- продолжает сказание. \explainDetail{Мощная}{мощный/-ая/-ое}{powerful} фигура коня, изображённого на картине с массивной металлической цепью вместо упряжки, \explainDetail{свид\'{е}тельствует}{свид\'{е}тельствовать}{testify; свидетель: witness} об этом.

Добрыня Никитич по народным преданиям был очень образ\'{о}ванным и \explainDetail{м\'{у}жественным}{м\'{у}жественный}{manly} человеком. С его личностью связано много чудес, наприм\'{е}р, заговорённая броня\footnote{charmed armor} на его плечах, \explain{волш\'{е}бный}{magic} меч-кладенец. Добрыня изображён таким как и в былинах -- величавым, с тонкими, благородными чертами лица, подчёркивающими его культурность, образ\'{о}ванность, \explain{решительно}{decisively} вынимающий меч из \explain{н\'{о}жен}{sheath} с готовностью \explainDetail{бр\'{о}ситься}{брос\'{а}ться/бр\'{о}ситься}{rush} в бой, защищая свою родину.

Алёша Попович \explain{по сравнению с}{as compared with} товарищами молод и строен. Он изображён с \footnote{л\'{у}ком}{bow} и стрелами в руках, но \explain{прикреплённые}{attached} к \explainDetail{седлу}{седло}{saddle} гусли свидетельствуют о том, что он не только бесстрашный воин, но и \explain{гусляр}{player of the musical instrument ``gusli''}, песенник, весельчак. В картине много таких деталей, которые характеризуют образы её персонажей.

Упряжки коней, одежда, амуниция не \explain{вымышленные}{fictional}. Такие образцы художник видел в музеях и читал их описания в исторической литературе. Художник \explain{мастерски}{masterfully} передаёт состояние природы, как бы предвещающей о наступлении опасности. Но богатыри представляют собой \explainDetail{надёжную}{надёжный/-ая/-ое}{reliable} и мощную силу защитников родной земли.




\subsection{Картина «Алёнушка» ВМ Васнец\'{о}ва}
% https://muzei-mira.com/kartini_russkih_hudojnikov/1321-opisanie-kartiny-bogatyri-tri-bogatyrya-vasnecova-1898.html

Алёнушка, печальная девочка у \explainDetail{пруда}{пруд}{pond} --- одна из любимых всеми картин В. Васнецова. Художник уд\'{а}чно использует сказочный сюжет, чтобы \explainDetail{раскрыть}{раскрывать/раскрыть}{to open/to discover} сложный и неоднозначный русский характер.

Грусть девочки очень взрослая. Печаль в её глазах граничит с \explainDetail{отчаянием}{отч\'{а}яние}{despair}. Неубранные рыжие волосы, тёмные глаза, нежно-алые губы --- формируют легко читаемый образ ребёнка с тр\'{у}дной судьб\'{о}й.
В Алёнушке совсем нет ничего \explainDetail{сказочного}{сказочный}{fabulous, fairytale-like}, фантастического.
С\'{о}бственно, вся ск\'{а}зочность сюжета подчеркнута лишь одной деталью --- группой \explainDetail{ласточек}{л\'{а}сточка}{swallow}, сидящих над головой \explainDetail{героини}{героиня}{heroine}. Этим с\'{и}мволом (как известно, л\'{а}сточки символиз\'{и}руют над\'{е}жду) художник \explainDetail{уравновешивает}{уравновешивать/уравновесить}{to balance --- уравнов\'{е}шиваю/-ешь/-ют; уравнов\'{е}шу/-ишь/-ят} полный \explainDetail{тоск\'{и}}{тоск\'{а}}{yearning, longing} \'{о}браз героини, даёт надежду на счастливый финал старой русской сказки.

Васнецов нап\'{о}лнил \explain{ф\'{о}новый}{background} пейзаж атмосферой тишины и гр\'{у}сти.
Отлично удал\'{и}сь художнику в\'{о}дная \explain{гладь}{smooth surface} пруда, \explainDetail{камыш\'{и}}{камыш}{reed}, осока, \explainDetail{ели}{ель}{fir tree}.
Всё \explainDetail{неподв\'{и}жно}{неподв\'{и}жный/-ая/-ое}{still, motionless}, тихо, спокойно.
Даже пруд отраж\'{а}ет героиню очень деликатно, \explain{слегк\'{а}}{slightly}.
Чуть трепещут молодые \explainDetail{ос\'{и}ны}{ос\'{и}на}{aspen}. \explainDetail{Едва}{едв\'{а}}{barely, hardly} \explain{хмурится}{turns gloomly} ос\'{е}ннее небо.
Тёмные, зелёные тона пейзажа контрастируют с \explainDetail{румянцем}{румянец}{blush} на лице героини, а ос\'{е}нняя грусть --- с яркими цветами на юбке Алёнушки. Зритель чувствует: ещё мгнов\'{е}ние и сказка прод\'{о}лжится\dots




\subsection{Сказка «Сестрица Алёнушка и братец Иванушка»}
Жили-были стар\'{и}к да стар\'{у}ха, у них был\'{а} дочка Алёнушка да сын\'{о}к Иванушка. Старик со старухой умерли. Остались Алёнушка да Иванушка одни-одинёшеньки. Пошла Алёнушка на работу и братца с собой взяла. Идут они по д\'{а}льнему пут\'{и}, по шир\'{о}кому п\'{о}лю, и захотелось Иванушке пить.
%
\begin{dialogue}
    \item Сестр\'{и}ца Алёнушка, я пить хочу!
    \item Подожди, братец, дойдем до кол\'{о}дца.
\end{dialogue}
%
Шли-шли, -- солнце высоко, колодец далёко, жар \explainDetail{донимает}{донимать}{(colloq.) to bother, harass}, пот выступ\'{а}ет. Сто\'{и}т коровье коп\'{ы}тце\footnote{hoof (diminutive of коп\'{ы}то)} полн\'{о} водицы.
%
\begin{dialogue}
    \item Сестрица Алёнушка, хлебну\footnote{(colloq.) to drink} я из копытца!
    \item Не пей, братец, телёночком станешь!
\end{dialogue}
%
Братец послушался, пошли дальше. Солнце высоко, колодец далёко, жар донимает, пот выступает. Сто\'{и}т лошадиное копытце полно водицы.
%
\begin{dialogue}
    \item Сестрица Алёнушка, напьюсь я из копытца!
    \item Не пей, братец, жеребёночком станешь!
\end{dialogue}
%
\explainDetail{Вздохнул}{вздых\'{а}ть/вздохн\'{у}ть}{sigh} Иванушка, опять пошли дальше. Идут, идут, - солнце высоко, колодец далёко, жар донимает, пот выступает. Сто\'{и}т к\'{о}зье копытце полно водицы. Иванушка говорит:
%
\begin{dialogue}
    \item  Сестрица Алёнушка, мочи нет: напьюсь я из копытца!
    \item  Не пей, братец, козлёночком станешь!
\end{dialogue}
%
Не послушался Иванушка и нап\'{и}лся из к\'{о}зьего копытца. Нап\'{и}лся и стал козлёночком\dots Зовёт Алёнушка братца, а вместо Иванушки бежит за ней беленький козлёночек. Залилась\footnote{flooded} Алёнушка слез\'{а}ми, села на стож\'{о}к -- плачет, а козлёночек в\'{о}зле неё скачет. В ту пору ехал мимо купец:
%
\begin{dialogue}
    \item О чём, красная девица, плачешь?
\end{dialogue}
Рассказала ему Алёнушка про свою беду. Купец ей и говорит:
\begin{dialogue}
    \item Под\'{и} за меня замуж. Я тебя наряж\'{у} в златосеребро, и козлёночек будет жить с нами.
\end{dialogue}
Алёнушка под\'{у}мала, под\'{у}мала и пошла за купца замуж. Стали они жить-поживать, и козлёночек с ними живёт, ест-пьёт с Алёнушкой из одной чашки. Один раз купца не было дома. Откуда не возьм\'{и}сь прих\'{о}дит \explain{ведьма}{witch}: стала под Алёнушкино окошко и такто ласково начала звать её купаться на реку\footnote{произношение: н\'{а}реку}. Привела ведьма Алёнушку на реку. Кинулась на неё, привязала Алёнушке на шею камень и бросила её в воду. А сама оборотилась Алёнушкой, нарядилась в её платье и пришла в её хоромы. Никто ведьму не распознал. Купец вернулся - и тот не распознал.

Одному козлёночку все было ведомо. Повесил он голову, не пьет, не ест. Утром и вечером ходит по бережку около воды и зовёт:
\begin{dialogue}
    \item Алёнушка, сестрица моя! Выплынь, выплынь на бережок\dots
\end{dialogue}

Узнала об этом ведьма и стала просить мужа зарежь да зарежь козлёнка.
Купцу жалко было козлёночка, привык он к нему А ведьма так пристает, так упрашивает, --- делать нечего, купец согласился:
%
\begin{dialogue}
    \item Ну, зарежь его\dots
\end{dialogue}
%
Велела ведьма разложить костры высокие, греть котлы чугунные, точить ножи булатные.
Козлёночек проведал, что ему недолго жить, и говорит названому отцу:
%
\begin{dialogue}
    \item Перед смертью пусти меня на речку сходить, водицы испить, кишочки прополоскать.
    \item Ну, сходи.
\end{dialogue}
%
Побежал козлёночек на речку, стал на берегу и жалобнёхонько закричал:
%
\begin{dialogue}
    \item   Алёнушка, сестрица моя! Выплынь, выплынь на бережок.
    Костры горят высокие,
    Котлы кипят чугунные,
    Ножи точат \explain{булатные} {Bulat is a type of steel alloy known in Russia from medieval times; it was regularly mentioned in Russian legends as the material of choice for cold steel. This type of steel was used by the armies of nomadic peoples. Bulat steel was the main type of steel used for swords in the armies of Genghis Khan.},
    Хотят меня зарезати!
\end{dialogue}
%
%
Алёнушка из реки ему отвечает:
%
\begin{dialogue}
    \item Ах, братец мой Иванушка! Тяжёл камень на дно тянет,
    Шёлкова трава ноги спутала,
    Желты пески на груди легли.
\end{dialogue}
%
%
А ведьма ищет козлёночка, не может найти и посылает слугу:
\begin{dialogue}
    \item Пойди найди козлёнка, приведи его ко мне.
\end{dialogue}
% 
%
Пошёл слуга на реку и видит: по берегу бегает козлёночек и жалобнёшенько зовёт:
\begin{dialogue}
    \item Алёнушка, сестрица моя! Выплынь, выплынь на бережок.
    Костры горят высокие,
    Котлы кипят чугунные,
    Ножи точат булатные,
    Хотят меня зарезати!
\end{dialogue}
%
%
А из реки ему отвечают:
\begin{dialogue}
    \item Ах, братец мой Иванушка!
    Тяжёл камень на дно тянет,
    Шелкова трава ноги спутала,
    Желты пески на груди легли.
\end{dialogue}
%
Слуга побежал домой и рассказал купцу про то, что слышал на речке. Собрали народ, пошли на реку, закинули сети шелковые и вытащили Алёнушку на берег. Сняли камень с шеи, окунули её в ключевую воду, одели её в нарядное платье. Алёнушка ожила и стала краше, чем была.

А козлёночек от радости три раза перекинулся через голову и обернулся мальчиком Иванушкой.

Ведьму привязали к лошадиному хвосту, и пустили в чистое поле.


\subsection{Картина «Витязь на распутье»}

Виктор Михайлович Васнецов с циклом работ, \explain{посвященных}{dedicated (посвященный + \textit{дат.})} сюжетам русских сказок и былин, оказался \explainDetail{новатором}{новатор}{innovator} в этой области \explainDetail{изобразительного искусства}{изобраз\'{и}тельное искусство}{visual art}. За ним закрепилась репутация «художника-сказочника», он настолько проникся духом русской старины и былинного времени, что свой московский дом построил в виде деревянной избы (сейчас там находится мемориальный музей \explainDetail{живоп\'{и}сца}{живоп\'{и}сец}{painter, artist}).

Картина «В\'{и}тязь на расп\'{у}тье» \explain{отч\'{а}сти}{partly} является и отражением судьбы Васнецова.
Будучи \explain{пр\'{и}знанным}{пр\'{и}знанный}{recognised} художником-передвижником, он, как и его товарищи, \explainDetail{исполнял}{исполнять/исполнить}{performed} жанровые композиции в духе остросоциальных тем, волновавших общество в 1870-1890-х.
Но завладевшая им сказочная тематика диктовала \explainDetail{ин\'{о}е}{ин\'{о}й/ин\'{а}я/ин\'{о}е}{(определительное местоимение) другой, отличный от данного; (неопределённое местоимение) некоторый. See \href{https://ru.wiktionary.org/wiki/\%D0\%B8\%D0\%BD\%D0\%BE\%D0\%B9}{wikictionary.org:иной}.} развитие творчества. Живописец уход\'{и}л от проблем современности и \explain{погружался}{plunge, dive} в мир русской старины, рискуя быть \explainDetail{осужденным}{осужденный}{convicted}.

Выбор пути как один из \explain{роков\'{ы}х}{fatal} вопросов человеческой жизни на крупноформатном \explainDetail{холсте}{холст}{canvas} мастера приобрел эпическое звучание.
Перед камнем-предсказателем согнулся под тяжестью фатального \explainDetail{пророчества}{пророчество}{prophecy} опечаленный витязь. \explainDetail{Зловещий}{зловещий}{sinister} в\'{о}рон, садящееся красное солнце нагнетают\footnote{build up the atmosphere} атмосферу. \explainDetail{Нам\'{е}ренный}{нам\'{е}ренный}{intentional} отказ от \explainDetail{изображения}{изображение}{depiction} дороги (как выхода из трудности) художником сделан для того, чтобы показать \explain{неотврат\'{и}мость}{inevitability} судьбы.
\chapter{Сказки}
\section{Сестрица Алёнушка и братец Иванушка}
% https://deti-online.com/skazki/russkie-narodnye-skazki/sestrica-alyonushka-i-bratec-ivanushka/
% https://www.youtube.com/watch?v=UDaOREoItE8
Жили-были стар\'{и}к да стар\'{у}ха, у них был\'{а} дочка Алёнушка да сын\'{о}к Иванушка. Старик со старухой умерли. Остались Алёнушка да Иванушка одни-одинёшеньки. Пошла Алёнушка на работу и братца с собой взяла. Идут они по д\'{а}льнему пут\'{и}, по шир\'{о}кому п\'{о}лю, и захотелось Иванушке пить.
%
\begin{dialogue}
    \item Сестр\'{и}ца Алёнушка, я пить хочу!
    \item Подожди, братец, дойдем до кол\'{о}дца.
\end{dialogue}
%
Шли-шли, -- солнце высоко, колодец далёко, жар \explainDetail{донимает}{донимать}{(colloq.) to bother, harass}, пот выступ\'{а}ет. Сто\'{и}т коровье коп\'{ы}тце\footnote{hoof (diminutive of коп\'{ы}то)} полн\'{о} водицы.
%
\begin{dialogue}
    \item Сестрица Алёнушка, хлебну\footnote{(colloq.) to drink} я из копытца!
    \item Не пей, братец, телёночком станешь!
\end{dialogue}
%
Братец послушался, пошли дальше. Солнце высоко, колодец далёко, жар донимает, пот выступает. Сто\'{и}т лошадиное копытце полно водицы.
%
\begin{dialogue}
    \item Сестрица Алёнушка, напьюсь я из копытца!
    \item Не пей, братец, жеребёночком станешь!
\end{dialogue}
%
\explainDetail{Вздохнул}{вздых\'{а}ть/вздохн\'{у}ть}{sigh} Иванушка, опять пошли дальше. Идут, идут, -- солнце высоко, колодец далёко, жар донимает, пот выступает. Сто\'{и}т к\'{о}зье копытце полно водицы. Иванушка говорит:
%
\begin{dialogue}
    \item  Сестрица Алёнушка, мочи нет: напьюсь я из копытца!
    \item  Не пей, братец, козлёночком станешь!
\end{dialogue}
%
Не послушался Иванушка и нап\'{и}лся из к\'{о}зьего копытца. Нап\'{и}лся и стал козлёночком\dots Зовёт Алёнушка братца, а вместо Иванушки бежит за ней беленький козлёночек. Залилась\footnote{flooded} Алёнушка слез\'{а}ми, села на стож\'{о}к -- плачет, а козлёночек в\'{о}зле неё скачет. В ту пору ехал мимо купец:
%
\begin{dialogue}
    \item О чём, красная девица, плачешь?
\end{dialogue}
Рассказала ему Алёнушка про свою беду. Купец ей и говорит:
\begin{dialogue}
    \item Под\'{и} за меня замуж. Я тебя наряж\'{у} в златосеребро, и козлёночек будет жить с нами.
\end{dialogue}
Алёнушка под\'{у}мала, под\'{у}мала и пошла за купца замуж. Стали они жить-поживать, и козлёночек с ними живёт, ест-пьёт с Алёнушкой из одной чашки. Один раз купца не было д\'{о}ма. \explainDetail{Откуда не возьм\'{и}сь}{откуда не возьм\'{и}сь}{out of the blue} прих\'{о}дит \explain{в\'{е}дьма}{witch}: стала под Алёнушкино окошко и такто ласково начал\'{а} звать её куп\'{а}ться на реку\footnote{произношение: н\'{а}реку}. Привела ведьма Алёнушку на реку. \explainDetail{Кинулась}{кид\'{а}ться/к\'{и}нуться}{to throw oneself, to fling oneself, to dash, to rush, } на неё, привязала Алёнушке на шею камень и бр\'{о}сила её в в\'{о}ду. А сам\'{а} оборот\'{и}лась Алёнушкой, \explainDetail{нарядилась}{наряж\'{а}ться/наряд\'{и}ться}{to dress as someone, to imitate} в её пл\'{а}тье и пришла в её хор\'{о}мы. Никто ведьму не \explain{распозн\'{а}л}{recognised}. Купец вернулся -- и тот не распознал.

Одному козлёночку всё было ведомо. Повесил он голову, не пьет, не ест. Утром и вечером ходит по бережку около воды и зовёт:
\begin{dialogue}
    \item Алёнушка, сестрица моя! Выплынь, выплынь на бережок\dots
\end{dialogue}

Узнала об этом ведьма и стала просить мужа \explain{зарежь}{slaughter} да зарежь козлёнка.
Купцу жалко было козлёночка, \explain{привык}{got used to + \textit{дат.}} он к нему. А ведьма так пристаёт, так упрашивает, --- делать н\'{е}чего, купец согласился:
%
\begin{dialogue}
    \item Ну, зарежь его\dots
\end{dialogue}
%
Велела ведьма разложить костры высокие, греть котлы чугунные, точить ножи булатные.
Козлёночек проведал, что ему недолго жить, и говорит названому отцу:
%
\begin{dialogue}
    \item Перед смертью пусти меня на речку сходить, водицы испить, кишочки прополоскать.
    \item Ну, сходи.
\end{dialogue}
%
Побежал козлёночек на речку, стал на берегу и жалобнёхонько закричал:
%
\begin{dialogue}
    \item   Алёнушка, сестрица моя! Выплынь, выплынь на бережок.
    Костры горят высокие,
    Котлы кипят чугунные,
    Ножи точат \explain{булатные} {Bulat is a type of steel alloy known in Russia from medieval times; it was regularly mentioned in Russian legends as the material of choice for cold steel. This type of steel was used by the armies of nomadic peoples. Bulat steel was the main type of steel used for swords in the armies of Genghis Khan.},
    Хотят меня зарезати!
\end{dialogue}
%
%
Алёнушка из реки ему отвечает:
%
\begin{dialogue}
    \item Ах, братец мой Иванушка! Тяжёл камень на дно тянет,
    Шёлкова трава ноги спутала,
    Желты пески на груди легли.
\end{dialogue}
%
%
А ведьма ищет козлёночка, не может найти и посылает \explainDetail{слуг\'{у}}{слуг\'{а}}{servant}:
\begin{dialogue}
    \item Пойди найди козлёнка, приведи его ко мне.
\end{dialogue}
% 
%
Пошёл слуга на реку и видит: по берегу бегает козлёночек и жалобнёшенько зовёт:
\begin{dialogue}
    \item Алёнушка, сестрица моя! Выплынь, выплынь на бережок.
    Костры горят высокие,
    Котлы кипят чугунные,
    Ножи точат булатные,
    Хотят меня зарезати!
\end{dialogue}
%
%
А из реки ему отвечают:
\begin{dialogue}
    \item Ах, братец мой Иванушка!
    Тяжёл камень на дно тянет,
    Шелкова трава ноги спутала,
    Желты пески на груди легли.
\end{dialogue}
%
Слуг\'{а} побежал домой и рассказал купцу про то, что слышал на речке. Собрали народ, пошли на реку, закинули сети шелковые и вытащили Алёнушку на берег. Сняли камень с шеи, окунули её в ключевую воду, одели её в нарядное платье. Алёнушка ожила и стала краше, чем была.

А козлёночек от радости три раза перекинулся через голову и обернулся мальчиком Иванушкой.

Ведьму привязали к лошадиному \explainDetail{хвосту}{хвост}{tail}, и пустили в чистое поле.

\section{Маша и медведь}
Жили-были дедушка да бабушка. Была у них внучка Машенька. Собрались раз подружки в лес -- по грибы да по ягоды. Пришли звать с собой и Машеньку.
\begin{dialogue}
    \item Дедушка, бабушка, -- говорит Машенька, -- отпустите меня в лес с подружками!
\end{dialogue}
Дедушка с бабушкой отвечают:
%
\begin{dialogue}
    \item Иди, только смотри от подружек не отставай -- не то заблудишься.
\end{dialogue}
Пришли девушки в лес, стали собирать грибы да ягоды. Вот Машенька -- деревце за деревце, кустик за кустик -- и ушла далеко-далеко от подружек.

Стала она аукаться, стала их звать. А подружки не слышат, не отзываются.
Ходила, ходила Машенька по лесу -- совсем заблудилась.
Пришла она в самую глушь, в самую чащу. Видит-стоит избушка. Постучала Машенька в дверь -- не отвечают. Толкнула она дверь, дверь и открылась.
Вошла Машенька в избушку, села у окна на лавочку.

Села и думает:

\begin{fancyquotes}
    «Кто же здесь живёт? Почему никого не видно?..» А в той избушке жил большущий медведь. Только его тогда дома не было: он по лесу ходил. Вернулся вечером медведь, увидел Машеньку, обрадовался.
\end{fancyquotes}
%
\begin{dialogue}
    \item Ага, -- говорит, -- теперь не отпущу тебя! Будешь у меня жить. Будешь печку топить, будешь кашу варить, меня кашей кормить.
\end{dialogue}

Потужила Маша, погоревала, да ничего не поделаешь. Стала она жить у медведя в избушке.

Медведь на целый день уйдёт в лес, а Машеньке наказывает никуда без него из избушки не выходить.
%
\begin{dialogue}
    \item А если уйдёшь, -- говорит, -- всё равно поймаю и тогда уж съем!
\end{dialogue}
Стала Машенька думать, как ей от медведя убежать. Кругом лес, в какую сторону идти -- не знает, спросить не у кого\dots

Думала она, думала и придумала.

Приходит раз медведь из лесу, а Машенька и говорит ему:
\begin{dialogue}
    \item Медведь, медведь, отпусти меня на денёк в деревню: я бабушке да дедушке гостинцев снесу.
    \item Нет, -- говорит медведь, -- ты в лесу заблудишься. Давай гостинцы, я их сам отнесу!
\end{dialogue}
А Машеньке того и надо!

Напекла она пирожков, достала большой-пребольшой короб и говорит медведю:

\begin{dialogue}
    \item Вот, смотри: я в короб положу пирожки, а ты отнеси их дедушке да бабушке. Да помни: короб по дороге не открывай, пирожки не вынимай. Я на дубок влезу, за тобой следить буду!
    \item Ладно, -- отвечает медведь, -- давай короб! Машенька говорит:
    \item Выйди на крылечко, посмотри, не идёт ли дождик! Только медведь вышел на крылечко, Машенька сейчас же залезла в короб, а на голову себе блюдо с пирожками поставила.
\end{dialogue}

Вернулся медведь, видит -- короб готов. Взвалил его на спину и пошёл в деревню.

Идёт медведь между ёлками, бредёт медведь между берёзками, в овражки спускается, на пригорки поднимается. Шёл-шёл, устал и говорит:

\begin{fancyquotes}
    Сяду на пенёк, Съем пирожок! А Машенька из короба:
    Вижу, вижу! Не садись на пенёк, Не ешь пирожок! Неси бабушке, Неси дедушке!
\end{fancyquotes}

\begin{dialogue}
    \item Ишь какая глазастая, -- говорит медведь, -- всё видит! Поднял он короб и пошёл дальше. Шёл-шёл, шёл-шёл, остановился, сел и говорит:
\end{dialogue}

\begin{fancyquotes}
    Сяду на пенёк, Съем пирожок! А Машенька из короба опять: Вижу, вижу! Не садись на пенёк, Не ешь пирожок! Неси бабушке, Неси дедушке!
\end{fancyquotes}

Удивился медведь:

\begin{dialogue}
    \item Вот какая хитрая! Высоко сидит, далеко глядит! Встал и пошёл скорее.
\end{dialogue}
Пришёл в деревню, нашёл дом, где дедушка с бабушкой жили, и давай изо всех сил стучать в ворота:
\begin{dialogue}
    \item Тук-тук-тук! Отпирайте, открывайте! Я вам от Машеньки гостинцев принёс.
\end{dialogue}
А собаки почуяли медведя и бросились на него. Со всех дворов бегут, лают.

Испугался медведь, поставил короб у ворот и пустился в лес без оглядки.

Вышли тут дедушка да бабушка к воротам. Видят- короб стоит.
\begin{dialogue}
    \item Что это в коробе? -- говорит бабушка.
\end{dialogue}
А дедушка поднял крышку, смотрит и глазам своим не верит: в коробе Машенька сидит -- живёхонька и здоровёхонька.

Обрадовались дедушка да бабушка. Стали Машеньку обнимать, целовать, умницей называть.

\chapter{Музыка}


\section{Пётр Налич}
% https://uznayvse.ru/znamenitosti/biografiya-petr-nalich.html
\subsection{Биография}
Петр Налич – российский певец боснийского происхождения, первый отечественный музыкант, ставший популярным благодаря YouTube. \explainDetail{Простенький}{простенький}{diminutive of прост\'{о}й} клип с песней «Guitar» на английском языке с акцентом вкупе с балканской мелодикой и дурашливой атмосферой пришелся по душе подавляющему большинству интернет-аудитории в 2007 году.

Пишет музыку для спектаклей, кинофильмов, мультфильмов. Автор песен на вымышленном языке бабурси, в которых, по признанию самого Петра, нет никакого смысла. Представлял со своим коллективом Россию на Евровидении 2010 года в Осло. Первым среди артистов использовал систему «Pay What You Want» для записи дебютного альбома «Радость простых мелодий».

\subsection{Детство, юност, семья}
Петр родился весной 1981 года в семье москвичей Андрея и Валентины Наличей, где уже подрастал старший сын Павел, впоследствии известный художник-оформитель. Андрей Захидович и Валентина Марковна – архитекторы, отец также занимается скульптурой. Его «Лента Мебиуса» расположена у кинотеатра «Горизонт», спортивный приз «Слава» – также его работа. Позже в творческом союзе с младшим сыном, а также Александром и Сергеем Цигалями создал памятник «Сочувствие», посвященный гуманному обращению с бродячими животными и памяти пса по кличке Мальчик, обитавшего в подземном переходе.

\begin{fancyquotes}
    Образцовым ребенком я, конечно, не был, но и хулиганом тоже. Хотя мне всю жизнь хотелось им быть, потому что хулиганы всегда круче, они нравятся девушкам, а я был домашним мальчиком.
\end{fancyquotes}

Именно отец приобщил сыновей к футболу. Несмотря на то, что Петр в детстве был вполне азартным мальчишкой и любил лазать по крышам, кататься с горки, никаким спортом он особо не увлекался, как и брат. И, когда младшему сыну было примерно 11-12 лет, Андрей Захидович решил всерьез заняться его и Павла физическим развитием. Каждую неделю он сажал их в машину, в другую садились двоюродные братья со своими отцами, и всей компанией ехали на станцию Университет. В то время там была неплохая футбольная площадка, на которой они могли гонять мяч бесплатно. Эту традицию, но уже со своими детьми, Петр продолжает и сейчас.

В одном из интервью Налич рассказывал и о своем дедушке-боснийце, у которого был ангажемент в оперном театре Белграда. Во время войны Захид Омерович попал в нацистский лагерь, где ему сломали гортань. Чудом выжив, он переехал в СССР и стал работать на радио. Больше он не пел, но унаследовал любовь к музыке детям, и отец Петра любил исполнять цыганские и русские романсы. Однажды родители спросили у мальчика, какой подарок он хотел бы получить к 14 дню рождения: гитару или конструктор «Лего-Техникс». Последним он просто грезил, его и попросил. Но папа и мама подарили Пете и гитару тоже.

\begin{fancyquotes}
    Я н\'{а}чал играть романсы, песни Цоя и «Наутилуса» вперемешку с цыганскими и казачьими, и этим репертуаром я набрал приличное количество очков в глазах девушек. Оказалось, что необязательно быть хулиганом, гитара тоже неплохо работает.
\end{fancyquotes}

Это во многом определило дальнейший жизненный путь Налича. Еще в школе он создал хард-рок-группу, в которой с удовольствием пел. Петр постоянно занимался музыкой: сначала в детской музыкальной школе имени Николая Мясковского, затем в музыкальном училище при Московской консерватории. Поступив в архитектурный институт, он стал заниматься вокалом в студии «Орфей» у педагога Ирины Мухиной. Окончив МАРХИ, Петр решил продолжить музыкальное образование и в 2010 году пошел учиться оперному пению:

\begin{fancyquotes}
    Я учился у выдающейся певицы Валентины Левко – к сожалению… ушедшей от нас. И огромной, мощнейшей школой для меня стала Оперная студия при РАМ имени Гнесиных. Сейчас она носит имя Сперанского, а когда я поступил, Юрий Аркадьевич Сперанский был еще жив, и мне посчастливилось с ним работать.
\end{fancyquotes}


\subsection{Первый успех: Guitar}
В 2007 году Петр создал сайт и стал выкладывать сочиненные им композиции, а любительский клип с песней Guitar загрузил на YouTube. Певец признавался, что сам не ожидал столь бурной реакции в интернете. На тот момент у него на сайте было порядка сорока композиций, которые мог скачать кто угодно, но лишь после успеха клипа «Guitar» с цыганскими мотивами и красивым голосом Петра (он пел на английском с сильным русским акцентом, который, впрочем, песне шел лишь на пользу) люди стали интересоваться его творчеством.

Осенью того же года состоялся первый сольный концерт певца в столичном клубе «Апшу». Билеты на выступление были раскуплены моментально, ажиотаж вокруг его персоны стоял небывалый. Счастливчики, побывавшие на концерте, тут же выложили в интернет восторженные отклики. В конце года Рунет признал композицию Guitar лучшей за 2007.

Творческая карьера продолжилась созданием «Музыкального коллектива Петра Налича», который для краткости называли МКПН. 2008 год группа посвятила гастролям по России. Тогда же она стала официальным музыкальным сопроводителем российских спортсменов на пекинской Олимпиаде.

Налич вместе с МКПН выпустил первый альбом «Радость простых мелодий». Затем группа стала участником антверпенского фестиваля Sfincs. В репертуаре коллектива звучали композиции «Медовый, аметистовый», «Взгляд твоих черных очей» и другие.

\subsection{Евровидение и новый стиль}
В 2010 году Налич с коллективом отправился в Осло, представлять Россию на «Евровидении». Лирическая баллада «Lost and Forgotten», спетая им на конкурсе Евровидения, не принесла призового места (он занял 11-е), но была тепло принята публикой.

2010 год также был отмечен выпуском нового альбома «Веселые Бабури», а еще через два года был записан диск «Золотая рыбка». На композицию «Сахарный пакет» был снят клип. Параллельно Петр н\'{а}чал исполнять партии в операх. Спектакли «Богема» и «Евгений Онегин», где Налич соответственно пел партии Рудольфа и Ленского, шли в театре-студии оперы при Академии.

В 2013 году вышел следующий альбом «Песни о любви и Родине», записанный в сопровождении оркестра Юрия Башмета. Следом Петр записал диск «Кухня» на выдуманном языке бабурси, созданном внутри МКПН, спел партию Германа в опере «Пиковая дама», которая прошла в Государственном музее Александра Пушкина, и принял участие в театрализованных онлайн-чтениях «Анна Каренина. Живое издание».

В 2015 году Налич распустил коллектив и объявил для него бессрочный отпуск, а сам занялся написанием музыки для спектаклей «Северная Одиссея» и «Питер Пен». Как композитор Налич получил приглашение от Пермского ТЮЗа. Он создал музыку для постановки «Обыкновенное чудо» по пьесе Евгения Шварца.

Затем Петр удивил поклонников новой программой «Утесов и не только…», которую исполнял в сопровождении эстрадно-симфонического бэнда. Впоследствии артист кардинально поменял состав собственного коллектива, пригласив в него музыкантов, способных на высоком профессиональном уровне исполнять совершенно новые ритмы и мелодии. Также он записал новый альбом «Паровоз», а затем выступил с новым репертуаром в клубе «16 тонн».

Не напрасно Налич назвал себя в одном из интервью «троякодышащей рыбой». В 2018 году разноплановый творческий человек снова стал студентом – Петр поступил в Гнесинку на композиторский факультет, вздыхая в интервью, что надо бы уже как-то определиться, кто он есть, иначе как-то нелепо. И тут же добавлял:

\begin{fancyquotes}
    Но пока естественным образом так складывается, что я занимаюсь и оперой, и сочинительством с бэндом, каким бы пестрым оно ни было по жанрам, и написанием инструментальной музыки для театра и других проектов. Теперь музыки будет еще больше.
\end{fancyquotes}


Параллельно с учебой на композиторском отделении в его оперном репертуаре появилась партия Тамино в «Волшебной флейте». Композиторская копилка самого Петра пополнилась музыкой к спектаклю «Тина», а следом к постановке «Горячее сердце», премьера которой состоялась на большой сцене театра имени Евгения Вахтангова. Расширился и оперный репертуар: Налич появился на разных театральных сценах в образах Неморино («Любовный напиток»), Альфреда Жермона («Травиатта»), Луиджи («Плащ»).

Благодаря собранным на краудфандинговой платформе Planeta средствам, был выпущен очередной альбом «Отражения в лужах». С певицей Женей Любич Петр записал яркую песню «Дежавю», а следом вышел новый диск, посвященный царской семье Romanovs100. Этот проект принес Наличу премию Original Music 2019 New York Festival TV \& Film Awards. Еще одна премия – «Онегин» в номинации «Событие года» досталась опере-променаду «Пиковая дама», где Петр исполнил партию Германа.

Значимым культурным событием 2019 года в Москве стало открытие нового театра – «Московского оперного дома». Его открытие сопровождалось премьерой спектакля «Иоланта». Партия Водемона в исполнении Налича была, как всегда, безупречна. Конец года принес Наличу заслуженную награду: он стал лауреатом Первой премии Всероссийского конкурса молодых композиторов «Партитура времени». Диплом был ему вручен в номинации «Сочинение для голоса и фортепиано».



\newpage
\section{Виктор Цой}
% https://ru.wikipedia.org/wiki/%D0%A6%D0%BE%D0%B9,_%D0%92%D0%B8%D0%BA%D1%82%D0%BE%D1%80_%D0%A0%D0%BE%D0%B1%D0%B5%D1%80%D1%82%D0%BE%D0%B2%D0%B8%D1%87
\subsection{Биография и творчество}
Виктор Робертович Цой (21 июня 1962 года, Ленинград --- 15 августа 1990 года, близ посёлка Кестерциемс, Латвийская ССР) --- советский рок-музыкант, автор песен и художник. Основатель и лидер рок-группы «Кино», в которой пел, играл на гитаре, писал музыку и стихи. Снялся в нескольких фильмах.

Виктор Цой родился единственным ребёнком в семье инженера корейского происхождения Роберта Максимовича Цоя и преподавательницы физкультуры Валентины Васильевны. Детство музыканта прошло в окрестностях Московского Парка Победы: он родился в роддоме на Кузнецовской улице (располагается внутри парка; сейчас это кардиоцентр), семья до 1990-х гг. жила в примечательном «генеральском доме» на углу Московского проспекта и улицы Бассейной (сейчас это памятник архитектуры). Некоторое время Виктор учился в близлежащей школе на улице Фрунзе, где работала его мама. В 1973 г. родители Цоя развелись, а через год повторно вступили в брак.

С 1974 по 1977 год посещал среднюю художественную школу, где возникла группа «Палата No. 6» во главе с Максимом Пашковым.
После исключения за неуспеваемость из художественного училища имени В. Серова поступил в СГПТУ--61 на специальность резчика по дереву.
В молодости был поклонником Михаила Боярского и Владимира Высоцкого, позднее Брюса Ли, имиджу которого н\'{а}чал подражать.
Увлекался восточными единоборствами и часто дрался «по-китайски» с Юрием Каспаряном.

\subsection{Смерть}
15 августа 1990 года в 12 часов 28 минут Виктор Цой погиб в автокатастрофе. ДТП произошло на 35 километре трассы «Слока --- Талси» под Тукумсом в Латвии, в нескольких десятках километров от Риги. Согласно наиболее правдоподобной официальной версии, Цой заснул за рулём, после чего его «Москвич-2141» тёмно-синего цвета вылетел на встречную полосу и столкнулся с автобусом «Икарус» модели 250 (иногда этот автобус ошибочно идентифицируют как 280 модель.

\begin{fancyquotes}
    Столкновение автомобиля «Москвич-2141» тёмно-синего цвета с рейсовым автобусом «Икарус-280» произошло в 12 час. 28 мин. 15 августа 1990 г. на 35 км трассы Слока --- Талси. Автомобиль двигался по трассе со скоростью не менее 130 км/ч, водитель Цой Виктор Робертович не справился с управлением. Смерть В. Р. Цоя наступила мгновенно, водитель автобуса не пострадал. ...В. Цой был абсолютно трезв накануне гибели. Во всяком случае, он не употреблял алкоголь в течение последних 48 часов до смерти. Анализ клеток мозга свидетельствует о том, что он уснул за рулем, вероятно, от переутомления.\\

    --- из милицейского протокола; по данным сайта kinoman.net
\end{fancyquotes}




19 августа он был похоронен на Богословском кладбище в Ленинграде.

\subsection{Прочие версии гибели}
Создатели документального кино из цикла «Следствие вели...» предположили, что Цой мог попасть в аварию, когда решил переставить другой стороной кассету в своём магнитофоне, тем самым отвлекшись от движения у «слепого поворота» дороги. Речь в передаче шла о кассете с демозаписью последнего альбома. Гитарист Юрий Каспарян ещё в 2002 году опроверг информацию о наличии этой кассеты в автомобиле Цоя: «Пользуясь случаем, хочу развеять миф, что на месте аварии нашли кассету с демо «Черного альбома»... Все было не так. Я специально приехал в Юрмалу с аппаратурой, с инструментами и мы делали аранжировки для нового альбома. Когда доделали, я забрал кассету и поехал в Петербург. Я приехал утром, вечером узнал о случившемся. И поехал обратно. И кассета все время была у меня в кармане».


\subsection{Творчество}
В конце 1970-х --- начале 1980-х началось тесное общение между Алексеем Рыбиным из хард-роковой группы «Пилигримы» и Виктором Цоем, игравшим на бас-гитаре в группе «Палата № 6», оба они познакомились в гостях у Андрея Панова (Свина), на квартире которого часто собирались компании, а также репетировала его собственная панк-группа «Автоматические удовлетворители».

Виктор Цой и Алексей Рыбин в составе «Автоматических удовлетворителей» ездили в Москву и играли панк-рок-металл на подпольных концертах Артемия Троицкого. Во время аналогичного выступления в Ленинграде по случаю юбилея Андрея Тропилло произошло первое знакомство с Борисом Гребенщиковым

\subsection{Первый альбом}
Летом 1981 года Виктор Цой, Алексей Рыбин и Олег Валинский основали группу «Гарин и Гиперболоиды», которая уже осенью была принята в члены Ленинградского рок-клуба. Вскоре Валинского забирают в армию, а группа, сменив название на «Кино», весной 1982 приступила к записи дебютного альбома. «Кино» под руководством Бориса Гребенщикова записывались на студии Андрея Тропилло в Доме Юного Техника, в записи принимали участие музыканты «Аквариума». Вскоре с ними же «Кино» дали свой первый электрический концерт в рок-клубе, всё выступление шло под драм-машину, а под песню «Когда-то ты был битником» из-за кулис на сцену выскочили БГ, Майк и Панкер. К лету альбом был полностью завершён, продолжительность его звучания составляла 45 минут, откуда и появилось название. Но позже из окончательного варианта была убрана песня «Я --- асфальт», которую можно найти в переиздании «45», где она прилагается в качестве бонус-трека. Запись получила некоторое распространение, о группе заговорили, начал\'{и}сь квартирные концерты в Москве и Ленинграде. Вместе с будущим барабанщиком Зоопарка Валерием Кирилловым осенью этого же года «Кино» записывает в студии Андрея Кускова несколько песен, в том числе «Весна» и «Последний герой», вошедшие в сборник «Неизвестные песни Виктора Цоя» (всего четыре издания).

Тогда запись была забракована и распространения не получила, так как Цой забрал ленту себе.

19 февраля 1983 года проходит совместный электрический концерт «Кино» и «Аквариума», музыканты выступали с тёмным макияжем и в костюмах со стразами. При этом они исполняли «Электричку», «Троллейбус» и «Алюминиевые огурцы». В основной состав был приглашён Юрий Каспарян. Весной из-за разногласий с Цоем Алексей Рыбин покидает группу «Кино». Лето уходит на совместные репетиции с новым гитаристом. В результате этого Виктор Цой и Юрий Каспарян записали альбом «46», который вначале задумывался как демозапись «Начальника Камчатки». Алексей Вишня «скинул» запись нескольким друзьям на плёнку. «46» получил широкое распространение и был воспринят как полноценный альбом. Осенью 1983 года Виктор Цой лёг на обследование в психиатрическую больницу на Пряжке, где провёл полтора месяца, избегая призыва в армию. После выписки из психиатрической клиники он пишет песню «Транквилизатор». Весной выступил на втором фестивале рок-клуба, где группа «Кино» получила лауреатское звание, а песня «Я объявляю свой дом безъядерной зоной», открывшая фестиваль, признана лучшей антивоенной песней фестиваля 1984 года.



\subsection{Второй состав «Кино»}
Летом 1984 года в студии «Антроп» Андрея Тропилло начинается запись альбома «Начальник Камчатки», к которому, кроме Виктора, приложили свою руку БГ и Сергей Курёхин.

В феврале 1984 Виктор и Марьяна празднуют свадьбу. На свадьбу были приглашены Гребенщиков, Майк, Титов, Каспарян, Гурьянов и другие.

Весной 1985 «Кино» заработали ещё одно звание лауреата и засели в студию к А. Тропилло писать «Ночь». Работа над записью затянулась из-за желания создать новую музыку с новыми приёмами игры. Альбом никак не получался, Виктор бросил «Ночь» недоделанной и в студии Алексея Вишни занялся записью «Это не любовь», который получился всего за неделю с небольшим. К осени «Это не любовь» была сведена и удачно разошлась по стране, а в январе 1986 вышла «Ночь», среди песен которой были известные «Мама Анархия» и «Видели ночь». Параллельно с выходом пластинки растёт популярность Виктора Цоя, а в феврале на 4-м фестивале рок-клуба «Кино» получает диплом за лучшие тексты. 5 августа 1985 года у Цоя родился сын Саша.


Летом 1986 года Виктор работал в бане на проспекте Ветеранов, он там мыл помещения из брандспойта. Необходимо было приходить на один час в день, но это было время с 22 до 23 часов, что ему мешало, так как Цой проводил это время суток с группой.

Также летом все участники группы уезжают в Киев на съёмки фильма «Конец каникул» (режиссёр Сергей Лысенко), а чуть позже дают совместный концерт с «Аквариумом» и «Алисой» в ДК МИИТ в Москве, с этими же группами в США выходит «Красная волна». Осенью Сергей Фирсов приглашает Виктора работать кочегаром. Цой соглашается, и они оба начинают работать кочегарами в котельной «Камчатка», откуда выросли многие знаменитые рок-музыканты.

В ней Рашид Нугманов организовал съёмки короткометражки «Йя-Хха», там же проходят съёмки фильма «Рок» Алексея Учителя --- оба фильма при участии Цоя. Осень и зима проходят в Ялте на съёмках «Ассы» Сергея Соловьёва.

Весна 1987 богата концертными событиями: премьера «Ассы» в ДК МЭЛЗ, последнее участие на фестивале рок-клуба, где «Кино» получили приз «За творческое совершеннолетие».

На порто-студии «Yamaha MT44» «Кино» начинают записывать альбом «Группа крови». Осенью 1987 года Виктор улетает к Рашиду Нугманову в Алма-Ату на съёмки своего последнего фильма «Игла», в связи с этим «Кино» доработали «Группу крови» и на время прекратили концертную \explain{деятельность}{activity}. В 1988 выходит «Игла» и «Группа крови», которые породили «киноманию».

Начинаются триумфальные гастроли по Советскому Союзу --- «Кино» собирают аншлаги на всех концертах.

16 ноября 1988 на мемориальном концерте памяти Александра Башлачёва публика ведёт себя крайне активно; по плану концерт должна была заканчивать песня Башлачёва «Время колокольчиков» (в записи), памяти которого был посвящён концерт, но по невыясненным причинам во время выступления Цоя (он играл на гитаре) внезапно включили «Время колокольчиков», Цой прекратил играть, не понимая откуда идёт звук, который он не производит и что вообще происходит. Администрация многократно объявляла, что всем н\'{у}жно расходиться, концерт окончен. Цой не уходил, он несколько раз подходил к выключенным микрофонам и проверял, работают ли. Потом разводил руками --- «не работает», и ходил по сцене туда и сюда с цветком, не уходя со сцены, но и не имея возможности петь и что-то сказать публике. Публика не расходилась, люди шумели, кричали, было видно, что что-то идёт не так. Создавалось впечатление, что некая злая воля решила прекратить концерт и включила финальную песню прямо во время выступления Цоя. Через 10 минут этого противостояния администрация включила микрофон. Цой, в очередной раз подойдя проверять микрофон, услышал что он включён, и объявил людям, что по непонятным причинам несвоевременно была включена финальная песня Саши Башлачева, но после этого петь и играть уже не очень удобно. После этого он стал собираться и публика потянулась к выходу.

Весной 1988 записывается черновик, а зимой окончательный вариант альбома «Звезда по имени Солнце», который решили выпустить осенью. Цой знакомится с Юрием Айзеншписом, который с 1989 стал продюсером «Кино», организовывая концертные туры и частые выступления на телевидении, после чего группа обретает всесоюзную популярность. В день 50-летия Цоя Александр Градский в эфире канала «Москва-24» рассказал, что в тот период Артемий Троицкий инспирировал письмо в Московский Горком, которое должно было настроить московских рок-музыкантов против Виктора Цоя.

На телевидении Виктор Цой дебютировал в программе «Взгляд», об этом рассказано в книге «Взгляд» --- битлы перестройки.

В начале 1989 группа «Кино» впервые едет за границу во Францию, где выпускают альбом «Последний герой». Летом Виктор с Юрием Каспаряном едут в США. Тем временем «Игла» выходит на второе место в прокате советских фильмов, а на кинофестивале «Золотой Дюк» в Одессе Виктора Цоя признают лучшим актёром СССР.

24 июня 1990 года прошёл последний концерт «Кино» в Москве на Большой спортивной арене Лужников. На этом концерте, впервые после московской Олимпиады-80 был зажжён огонь в Олимпийской чаше. После этого Цой с Каспаряном уединились на даче под Юрмалой, где на порто-студию начали записывать материал для нового альбома. Этот альбом, дописанный и сведённый музыкантами группы «Кино» уже после смерти Цоя, вышел в январе 1991 и получил символическое название «Чёрный альбом», с соответствующим оформлением обложки.

\section{Елена Ваенга}
Ел\'{е}на В\'{а}енга (настоящее имя --- Ел\'{е}на Влад\'{и}мировна Хрулёва; род. 27 января 1977, Североморск, Мурманская область, РСФСР, СССР) --- российская \explain{эстрадная певица}{pop singer}, автор песен, актриса. Лауреат премий «Шансон года».

В\'{а}енга --- это название родного для Елены Хрулёвой города Североморска до 18 апреля 1951 года, а также реки недалеко от него. В основе названия и псевдонима --- саамское слово «\explain{олен\'{и}ха}{deer}» (кильд. вайонгг). Псевдоним придуман её матерью.

\subsection{Биография}
Родилась 27 января 1977 года в Североморске. \explainDetail{Петь}{петь}{to sing} и \explain{обучаться}{to study (+ dative)} музыке начал\'{а} с трёх лет.

Мать Елены Ваенги по образованию химик, отец --- инженер, работали в посёлке Вьюжный на \explain{судоремонтном заводе}{shipyard (судно: vessel; ship, plural: суда)} «Нерпа», который обслуживает атомные \explain{подводные лодки}{подводная лодка: submarine}. Про отца и родной Север у Елены Ваенги есть песня:

\begin{fancyquotes}
    {\it У меня глаза северных цветов,\\
        И мне не нужны тропические страны.\\
        Я всегда с тобой рядышком была.\\
        Жаль, что ты уехал слишком рано.\\
        Я вдруг поняла: все эти города\\
        Я должна пройти, как в наказанье.\\
        Но у меня есть дом, а у дома --- я,\\
        А у Севера --- сиянье}
\end{fancyquotes}


Дед Елены со стороны матери --- контр-адмирал Северного флота Василий Семёнович Журавель, он упоминается в книге «Знаменитые люди Санкт-Петербурга». Бабушка Надежда Георгиевна Журавель (её крёстная) (род. 1927). Про неё у Елены Ваенги есть песня: «Моя бабушка любит суши...». Родители отца --- \explain{коренные}{коренн\'{о}й ж\'{и}тель; коренн\'{ы}е ж\'{и}тели: indigenous} петербуржцы, \explain{пережили}{(переживать/пережить) to survive; to experience} блокаду Ленинграда. Дед по линии отца --- зенитчик, во время \explain{Великой Отечественной войны}{second world war} \explainDetail{воевал}{воевать/повоевать}{to fight} под Ораниенбаумом, а бабушка по линии отца была врачом в госпитале в блокадном Ленинграде.

У Елены Ваенги есть младшая сестра Татьяна, она работает в дипломатической сфере, знает несколько языков.

Гражданский муж Елены Ваенги \explain{на протяжении 16 лет}{for 16 years} с 1995 по 2011 год --- Иван Иванович Матвиенко (род. 1957) --- продюсер певицы, по национальности цыган, был женат, его дочь на 2 года старше Елены Ваенги, раньше Иван перегонял машины из Германии.

Племянник, Руслан Сулимовский --- директор её коллектива.

В ночь с пятницы на субботу 10 августа 2012 года Ваенга в родильном доме No. 16 Санкт-Петербурга родила сына Ивана. 30 сентября 2016 года Елена официально вышла замуж за Романа Садырбаева.

\subsection{Творческая деятельность}
Первую песню «Голуби» написала в 9 лет, стала победительницей Всесоюзного конкурса молодых композиторов на Кольском полуострове. После школы приехала в Санкт-Петербург, где закончила музыкальное училище им. Н. А. Римского-Корсакова по классу фортепиано, получив диплом педагога-концертмейстера. Некоторое время преподавала музыку в школе. Факультативно занималась вокалом.

Елена Ваенга с детства мечтала стать актрисой, поэтому после музыкального училища поступила в Театральную академию (ЛГИТМИК) на курс Г. Тростянецкого, но проучилась лишь два месяца, так как её пригласили в Москву записывать первый альбом. Продюсером певицы стал Степан Разин. Под псевдонимом Нина она выпустила клип на песню «Длинные коридоры» (композиция была издана в 2011 году на сборнике «Живая струна»). Альбом был записан, но не вышел. Разочаровавшаяся в шоу-бизнесе певица сбежала от Разина и уехала в Санкт-Петербург. Тем временем её песни взяли в свой репертуар Александр Маршал («Невеста»), Татьяна Тишинская («А ты налей мне белого вина», «Мама, что ты плачешь», «Володенька», «Угостите даму сигаретой»), группы «Стрелки» («Тонкая веточка»), «Божья коровка» («Сердце моё», «Самая любимая моя») и другие известные исполнители. Эти песни распространил её бывший продюсер. Елена Ваенга приняла решение с ним не судиться.

В Санкт-Петербурге Ваенга узнала, что в Балтийском институте экологии, политики и права на кафедре театрального искусства набирает курс П. С. Вельяминов, и в 2000 году пошла учиться к нему. Закончив курс, получила диплом по специальности «драматическое искусство». Выступила в антрепризном спектакле «Свободная пара» в паре с однокурсником Андреем Родимовым (режиссёр Екатерина Шимилёва).

Концертирует с девятнадцати лет. Лауреат петербургского конкурса «Шлягер года 1998» с песней «Цыган», «Достойная песня 2002». Участник концертов-фестивалей «Весна романса» в БКЗ «Октябрьский», «Вольная песня над вольной Невой», «Невский бриз». Дала несколько сольных концертов в ДК имени М.Горького. Гастролирует по России и другим странам и каждый год, в конце января, даёт концерт в БКЗ «Октябрьский» по случаю своего дня рождения.

Настоящая популярность пришла к певице в 2005 году с выходом альбома «Белая птица», в котором было много хитов: «Желаю», «Аэропорт», «Тайга», «Шопен» и заглавная композиция, на которую вышел клип.

28 ноября 2009 года Елена Ваенга получила свой первый приз «Золотой граммофон» за песню «Курю».

4 декабря 2010 года Елена повторила свой успех, получив во второй раз премию «Золотой граммофон» за песню «Аэропорт». В том же году певица впервые стала лауреатом фестиваля «Песня года», исполнив композицию «Абсент». А 12 ноября она дала первый в своей концертной деятельности сольный концерт в Государственном Кремлёвском дворце, трансляция которого прошла на Первом канале 7 января 2011 года. В телеанонсе Елене Ваенга была дана следующая характеристика:
Елена Ваенга --- тонкая, художественная, мечтательная и романтичная натура. Музыкальная одарённость, природный темперамент, трудолюбие, жизнерадостность --- всё это составляющие её жизни и творчества... Несмотря на внешнюю хрупкость и молодость, за спиной у этой очаровательной девушки богатая творческая биография и не такая уж простая человеческая судьба. Жанр, в котором работает певица, с большим трудом определяет даже она сама: «На 50 процентов это фолк-рок, есть старинные баллады, городские романсы, шансон. Но границы между ними провести почти невозможно.»
--- анонс на Первом канале --- «Белая птица». Концерт Елены Ваенги
В 2011 году Елена Ваенга приняла участие в ежегодной церемонии национальной премии Шансон года в Кремле, на которой исполнила песни «Оловянное сердце» и «Девочка». Популярность певицы растёт. В январе этого же года она победила Леонида Агутина в телепередаче «Музыкальный ринг» на канале НТВ, набрав почти в пять раз больше голосов слушателей.

26 ноября 2011 года певица в третий раз получила премию «Золотой граммофон» за песню «Клавиши», но на концерте в Кремле исполнила композицию «Шопен». 21 декабря 2011 года певица в третий раз дала концерт в Кремле. В 2012 году на «Золотой граммофон» претендовали песни «Шопен» и «Где была».

В 2011 году Елена Ваенга дала 150 афишных концертов, гастролировала в США, Германии, Израиле.

Периодически играет в спектакле «Свободная пара», совместно с Андреем Родимовым.

В 2011 году Ваенга впервые попала в список самых успешных деятелей российского шоу-бизнеса, составленный Forbes, и заняла в нём девятую позицию, с годовым доходом более шести миллионов долларов.

В репертуаре певицы --- собственные песни, старинные и современные романсы, баллады и народные песни, а также песни на стихи классиков, таких, как Сергей Есенин («Задымился вечер») и Николай Гумилёв («Жираф», «Шут»). В 2012 году певица провела концертный тур по Украине и Германии. Однако на этом деятельность певицы оборвалась в связи с потерей голоса из-за механического повреждения связок. После выздоровления певица дала последние концерты в городах Средней Волги и ушла в отпуск.

В ноябре 2012 года певица вышла из декрета и возобновила концертную деятельность. По сведениям журнала Forbes, в 2012 году певица в списке самых успешных российских деятелей шоу-бизнеса заняла четырнадцатое место. Сама артистка это отрицает, также как и в прошлом году, утверждая, что её доход гораздо меньше. На данный момент артистка активно гастролирует.

В 2014 году Елена Ваенга стала одним из членов жюри шоу Первого канала «Точь-в-точь».

27 ноября 2015 года состоялся сольный концерт Ваенги в Государственном Кремлёвском дворце, где она выпустила новую программу и представила новый альбом.

\chapter{Наука и Техника}
\section{Северное сияние}
С \explainDetail{наступлением}{наступление}{adventб beginning} осени тысячи туристов \explain{устремляются}{rush} в \explain{Заполярье}{the region of the Arctic circle}, чтобы увидеть уникальный танец небесных огней -- полярное, или северное сияние, на латыни -- Aurora Borealis.
В теории увидеть это природное явление можно с конца августа до середины апреля: в этот период времени ночи становятся темными, солнечная активность \explainDetail{возрастает}{возрастать/возрасти}{возраст\'{а}ю/-ешь/-ет; возраст\'{у}/-ёшь/-\'{у}т: rise, increase}, а облака \explainDetail{расс\'{е}иваются}{рассеиваться/рассеяться}{to disperse}.
Такое развлечение, как \explain{ох\'{о}та}{hunting} за северным сиянием, с каждым годом становится все популярнее как среди россиян, так и среди иностранных туристов, которые специально ради него готовы ехать на Крайний Север. Главное \explain{доказательство}{proof} удачной охоты -- это, конечно же, снимки северного сияния.


\section{Только дьявол мог выдумать Нобелевскую премию}
% https://www.gazeta.ru/science/2015/11/27_a_7914947.shtml
Екатерина Шутова

\textit{120 лет назад Альфред Нобель подписал \explain{завещ\'{а}ние}{will} по Нобелевской премии.}

120 лет назад Альфред Нобель подписал завещание, согласно которому его \explain{накопл\'{е}ния}{accumulation} поступили в фонд Нобелевской премии -- самой престижной на сегодняшний день \explainDetail{награды}{нагр\'{а}да}{prize}, ежегодно \explainDetail{присуждаемой}{присужд\'{а}емый}{awarded} за выдающиеся научные исследования, революционные изобретения или крупный \explain{вклад}{contribution} в культуру или развитие общества. \explainDetail{Отдел}{отдел}{department} науки «Газеты.Ru» вспоминает \explainDetail{подробности}{подробность}{detail} этого события.

В 1888 году журналисты \explain{оповестили}{notified} мир о смерти Альфреда Нобеля -- химика, инженера и изобретателя динамита. Репортеры ошиблись -- на самом деле погиб Людвиг Нобель, брат Альфреда.

\begin{fancyquotes}
    Удивленный изобретатель прочитал в одной из газет собственный некролог под названием «Торговец смертью мертв».
\end{fancyquotes}

Альфред Нобель не захотел оставаться злодеем в глазах человечества. Поэтому 27 ноября 1895 года в Шведско-Норвежском клубе в Париже ученый составил следующее завещание:

{\it
Я, \explain{нижеподписавшийся}{undersigned}, Альфред Бернхард Нобель, обдумав и решив, настоящим объявляю мое завещание по поводу имущества, нажитого мною... Капитал мои душеприказчики должны перевести в ценные бумаги, создав фонд, проценты с которого будут выдаваться в виде премии тем, кто в течение предшествующего года принес наибольшую пользу человечеству.

Указанные проценты следует разделить на пять равных частей, которые предназначаются: первая часть тому, кто сделал наиболее важное открытие или изобретение в области физики, вторая — в области химии, третья — в области физиологии или медицины, четвертая — создавшему наиболее значительное литературное произведение, отражающее человеческие идеалы, пятая — тому, кто внесет весомый вклад в сплочение народов, уничтожение рабства, снижение численности существующих армий и содействие мирной договоренности.

... Мое особое желание заключается в том, чтобы на \explain{присуждение}{awarding, conferment} премий не \explainDetail{влияла}{влиять/повлиять}{influence} национальность кандидата, чтобы премию получали наиболее \explain{достойные}{worthy}, независимо от того, скандинавы они или нет.}

\subsection{Как огорчить родственников}
Спустя год после написания завещания Альфред Нобель скончался на своей вилле от \explainDetail{кровоизлияния}{кровоизлияние}{hemorrhage} в мозг. За несколько лет до смерти ученый сказал о самом себе следующим образом: «Альфред Нобель -- его \explain{существование}{existence} следовало бы \explain{пресечь}{suppress} при рождении милосердным доктором. Основные добродетели: держит ногти в чистоте и никому не бывает в тягость. Основные недостатки: не имеет семьи, наделен дурным характером и плохим пищеварением.

\begin{fancyquotes}
    Величайший грех: не поклоняется Мамоне. Важнейшие события в его жизни: никаких.
\end{fancyquotes}

\explain{Наследники}{heirs} легендарного изобретателя были крайне \explain{возмущены}{outraged}, что огромные накопления уйдут не им в карман, а на поддержку науки. Они требовали, чтобы завещание было признано недействительным. Интересно, что единственным родственником Нобеля, не пытавшимся присвоить себе деньги, оказался его племянник Эммануил. «Русские называют исполнителя завещания «душеприказчик», то есть «представитель души», — заявил юристам мужчина. — Вот и действуйте соответственно». Позднее Эммануил добавил: «Я не хочу, чтобы достойнейшие ученые в будущем упрекали нашу семью в присвоении средств, которые по праву принадлежат им».

В конечном итоге справедливость восторжествовала — и через несколько лет после смерти ученого были вручены пять первых премий. А с 1969 года по инициативе Шведского банка начала присуждаться Нобелевская премия по экономике.

Лишь однажды деньги из фонда премии пошли на дело, никак не связанное с наукой. Софи фон Капивара, женщина, с которой у талантливого изобретателя были отношения, пообещала раскрыть содержание их переписки и посмертно опозорить Альфреда Нобеля. Душеприказчики в страхе выплатили крупную сумму за 216 писем мецената. Ученые до сих пор шутят, что

\begin{fancyquotes}
    наука была бы богаче, если бы не одна алчная молочница.
\end{fancyquotes}

«Ты славная девушка, но ты действуешь мне на нервы»

Существует миф, согласно которому у Альфреда Нобеля была жена, страстно влюбившаяся в математика, и именно поэтому изобретатель «обделил» всех представителей этой науки. Но на самом деле, как заявляют биографы, меценат никогда не был женат. В молодом возрасте Нобель влюбился в работницу аптеки, но та умерла от чахотки. Потосковав, ученый нашел новую пассию — на этот раз ей стала Сара Бернар, знаменитая актриса. Альфред Нобель написал письмо матери о том, что хочет жениться.

\begin{fancyquotes}
    Недаром актеров в старину не разрешали хоронить на кладбище. У них нет души, сыночек!» — предупредила сына любящая родительница.
\end{fancyquotes}



Послушный Нобель разорвал любовную связь с Бернар.

Следующая женщина появилась в жизни мецената, когда тому уже был 41 год. Альфред Нобель опубликовал в газете объявление о том, что ищет секретаршу. На него откликнулась графиня Берта Кински, с которой у изобретателя начался неторопливый и гармоничный роман.

\begin{fancyquotes}
    Кстати, по одной из версий, именно Кински попросила Нобеля вписать в завещание премию мира. А в 1905 году она стала первой женщиной, удостоенной этой премии.
\end{fancyquotes}

У Кински и Нобеля дело до свадьбы не дошло: однажды ученый обнаружил, что его секретарша исчезла, оставив на столе письмо следующего содержания: «Простите меня, господин Нобель. Я уезжаю в Вену, где меня ждет жених. Пожелайте мне счастья, как я желаю счастья вам. Искренне преданная вам Берта Кински, которая в скором времени станет Бертой фон Зуттер».

Последней женщиной в жизни Нобеля стала вышеупомянутая «алчная молочница», которая изрядно надоедала ученому своей глупостью и необразованностью. «Дорогое дитя. Ты славная девушка, но ты действуешь мне на нервы», — раздраженно писал ей в письмах Альфред Нобель.

Так почему же не существует премии по математике? Возможно, все дело в том, что у Альфреда Нобеля не заладились отношения с великим математиком Миттаг-Леффлером, который должен был стать первым лауреатом, а меценат этого не хотел. Но наиболее вероятная версия заключается в том, что Нобель воспринимал математику как инструмент, как сугубо теоретическую науку.

\subsection{Виагра для хомячков и исследование ругани}

В 1991 году появилась пародия на Нобелевскую премию — Шнобелевская премия. Она вручается «за достижения, которые заставляют сначала засмеяться, а потом — задуматься». Учредитель и идейный вдохновитель «Шнобелевки» — Марк Абрахамс, который, будучи редактором юмористического научного журнала, получал множество писем от читателей с подробным рассказом об их «великих» исследованиях. «Иногда эти люди заслуживали премии — правда, не Нобелевской», — говорил Абрахамс. Так редактор решил награждать ученых за самые нелепые достижения.

В разные годы Шнобелевская премия присуждалась

за разработку протезов яичек для собак, за исследование влияния музыки кантри на частоту самоубийств и за открытие, что «Виагра» помогает хомякам справиться с последствиями резкой смены часовых поясов.

Также пародийную награду получали ученые, доказавшие, что ругань снижает боль, и исследователи, изучавшие оральный секс у летучих мышей.

Первым в мире человеком, удостоенным как Шнобелевской, так и Нобелевской премии, стал Андрей Гейм. Голландский ученый российского происхождения был награжден «Шнобелевкой» за использование магнитов для того, чтобы демонстрировать возможность левитации лягушек. Спустя десять лет Гейм совместно со своим учеником Константином Новоселовым получил Нобелевскую премию за изобретение графена.

\subsection{Война еще не закончена, а премии уже раздают}
«Я готов простить Альфреду Нобелю изобретение динамита, но только дьявол в людском обличье мог выдумать Нобелевскую премию!» — воскликнул ирландский романист и драматург Джордж Бернард Шоу, став лауреатом в области литературы (по ироничному заявлению писателя, произошло это потому, что «в тот год он ничего не опубликовал»). Действительно, самая престижная международная награда — явление весьма резонансное и неоднозначное. В Советском Союзе Нобелевский комитет клеймили за то, что «он ухитрился не заметить Алексея Толстого, Максима Горького, Владимира Маяковского, но зато заметил Ивана Бунина. И только тогда, когда он стал эмигрантом, и только потому, что он стал эмигрантом и врагом советского народа».

В Третьем рейхе ученым было запрещено получать Нобелевскую премию, так как в 1935 году премию мира «За борьбу с милитаризмом в Германии» получил пацифист Карл фон Осецкий — ярый противник нацистского режима. В 1937 году Адольф Гитлер издал указ, согласно которому немцы не имели права принимать премию. Из-за указа награду не получили Герхард Домагк «за открытие антибактериального эффекта пронтозила», Адольф Бутенандт за исследование половых гормонов и Рихард Кун за работу по каротиноидам и витаминам.


\begin{fancyquotes}
    Весьма примечателен тот факт, что Бенито Муссолини и Адольф Гитлер были номинированы на Нобелевскую премию мира в 1935 и 1939 годах соответственно.
\end{fancyquotes}

Нобелевская история знает немало случаев отказа от самой престижной международной награды.

Так, в 1973 году политический деятель Фан Динь Кхай отказался от медали «за работу по разрешению вьетнамского конфликта», аргументируя свое решение тем, что «война еще не закончена, а премии уже раздают». Не захотел быть награжденным и Жан-Поль Сартр — французский писатель и драматург. По мнению Сартра, награда посягнет на его независимость — центральное понятие в философии автора. Вскоре после отказа от Нобелевской премии француз еще раз шокировал общественность, заявив, что уходит из литературы. «Литература — суррогат действенного преобразования мира», — с горечью заметил писатель.

\section{Ученые нашли способ записать данные в пяти измерениях}

\textit{Как 5D-диски изменят представление людей о хранении информации?}

\textbf{Ученые создали 5D-диск высочайшей плотности:} В октябре специалисты Саутгемптонского университета в Великобритании описали способ записи огромного количества данных на компактный диск небольших размеров. Технология, получившая название 5D, позволяет сохранить на специальном накопителе до 500 терабайт информации. Получившиеся диски из кварцевого стекла отличаются высочайшей плотностью, которая в десять тысяч раз превышает плотность оптических дисков Blu-Ray. Новый метод позволит эффективно разместить на небольшой площади облачные сервера для хранения данных пользователей, интернет-компаний, крупных корпораций. По словам ученых, это особенно важно на фоне развития технологий, увеличения количества подключенных к сети устройств и роста количества передаваемых через сеть данных.

\textbf{Облачные сервисы с каждым годом становятся все популярнее:}
За последние пять лет отношение потребителей и бизнеса к облачным сервисам изменилось. Раньше их воспринимали в качестве дополнительного метода резервного копирования данных — информация практически всегда поступала в одну сторону. Причем крупные корпорации в основном использовали дата-центры для хранения некритичной информации. К 2020-м годам организации стали использовать облачные серверы не только для аварийного копирования, но и для постоянного обмена данными внутри конкретного предприятия. Системы облачных хранилищ стали более гибкими, позволяя конкретному потребителю выбрать необходимое количество свободного места и производительность оборудования.

Специалисты Analytics Insight называют основными \explainDetail{преим\'{у}ществами}{преим\'{у}щество}{advantage} \explainDetail{облачных}{облачный}{cloud (adj.); \'{о}блако: cloud} дата-центров \explain{круглос\'{у}точный}{round the clock} доступ к информации, возможность одновременной работы не- скольких пользователей с одним массивом данных, масштабируемость и \explain{г\'{и}бкость}{flexibility}, \explain{снижение}{decline} затрат на хранение данных внутри компании.

\explainDetail{Представители}{представитель}{representative} отрасли отмечают, что в обычное время нагрузка на серверы неравномерна: в одной части дата-центров она может зашкаливать, в другой быть крайне небольшой. По этой причине эксперты предсказывают появление искусственного интеллекта, который мог бы анализировать и распределять нагрузку на оборудование. В том числе по этой причине данные пользователей хранятся в нескольких частях дата-центра.

\textbf{Больше всего в облачных сервисах пользователи ценят скорость передачи данных и безопасность:} По словам основателя облачного провайдера Wasabi Дэйва Френда, от дата-центров будущего потребители ожидают высокого уровня безопасности, производительности оборудования и приемлемой цены за услуги. «Резервные копии должны храниться в разрозненных системах, обеспечивающих максимально возможную изоляцию», — заметил предприниматель. Потенциальный злоумышленник, добравшийся до одного сервера, не должен иметь возможность удалить или зашифровать информацию так, чтобы ее нельзя было восстановить из альтернативных источников. Френд полагает, что на этом должна строиться концепция мультиоблака.

Другими критериями облачного сервиса будущего, по мнению Френда, являются доступная цена и высокая скорость передачи данных. Провайдеры должны будут таким образом скорректировать стоимость услуг и добиться определенного качества оборудования, чтобы оставаться конкурентоспособными и не разочаровывать клиентов.

Представители облачного провайдера CloudSigma рассказали, что дата-центры должны будут отвечать за сохранность данных и скорость передачи информации. Для хранения файлов пользователей и корпоративных клиентов они используют небольшие 2,5-дюймовые диски емкостью 250 гигабайт. В случае, если какой-либо диск выходит из строя, его заменяют, а данные восстанавливают через бэкап. При таком развитии событий клиент не теряет своих данных, хотя и оказывается без доступа к информации на 10-15 минут. Благодаря глубокой интеграции между серверами и оборудованием задержка передачи данных внутри дата-центра очень мала. Для того чтобы разогнать скорость и снизить задержку между серверами и пользователем, в компании полагаются на выделенную гигабитную линию интернета.


\textbf{Диски 5D позволят хранить информацию практически бесконечно:}
По оценке Forbes, к 2025 году к интернету будет подключено около 80 миллиардов устройств, которые будут генерировать около 180 триллионов гигабайт данных. В обозримом будущем хранить данные на классических накопителях будет проблематично — существует риск возникновения дефицита и увеличения стоимости хранения информации. Работающие над технологией 5D специалисты Саутгемптонского университета предлагают записывать информацию на кварцевом стекле с помощью фемтосекундных лазеров и сверхкоротких импульсов. «Запись на кварцевый носитель как бы идет в пяти измерениях — двух оптических и трех пространственных», — отмечают авторы исследования.

Инновация британских инженеров заключается в создании дисков повышенной плотности и размещении на небольшом участке колоссальных объемов данных. Например, на «болванке» размером в один дюйм удалось сохранить шесть гигабайт информации. Накопитель обычного для подобных устройств размера, основанный на кварцевых дисках, может сохранить до 500 терабайт данных. Разработка обещает революцию на рынке хранения информации, так как десятки, если не сотни классических дата-центров можно будет объединить в одну библиотеку.

Преимуществами 5D-дисков также называют долговечность и низкую стоимость обслуживания. По оценке ученых, кварцевые диски не прочнее обычных накопителей, однако могут выдержать температуру до 1800 градусов по Фаренгейту, или около тысячи градусов по Цельсию. В случае пожара в дата-центре информация, скорее всего, сохранится. Кроме того, кварцевое стекло со временем не меняет своих свойств, что позволит держать данные на 5D-накопителях практически вечно.

Единственным узким местом будущей разработки является скорость передачи данных. В настоящий момент ученым удалось разогнать ее до 230 килобайт в секунду — за это время на диск можно записать около ста страниц текста. Однако для того, чтобы полностью заполнить болванку емкостью 500 терабайт, потребуется 60 дней. Либо инженеры найдут способ обойти ограничение, либо 5D-диски так и останутся перспективным оборудованием для записи информации. В крайнем случае на подобных дисках можно сохранять данные для потомков. Так, в 2018 году на кварцевый носитель была записана трилогия романов Айзека Азимова «Основание» — диск отправился в космос вместе с Tesla Roadster Илона Маска.



\section{У коронавируса есть механизм самоуничтожения}
% https://russian.rt.com/science/article/937351-pyotr-chumakov-intervyu-virusy
\textit{Пётр Чумаков об эволюции SARS-CoV-2 и лечении вирусами}

Снижение патогенности новых штаммов SARS-CoV-2 предопределено самой логикой эволюции вирусов такого типа. Об этом в интервью RT рассказал член-корреспондент РАН, профессор и главный научный сотрудник Института молекулярной биологии РАН Пётр Чумаков. Он также отметил, что многие вирусы можно поставить на службу человеку: например, использовать их непатогенные варианты для защиты людей от опасных возбудителей. За вакцинами, основанными на этом принципе, будущее, считает учёный. По его мнению, вирусы можно использовать и для лечения рака.

{\bf --- С самого начала пандемии многие боялись появления особо смертоносного штамма вируса. Почему этого пока не случилось? По предварительной информации, новый штамм «омикрон» не привёл к резкому росту летальности. Да и в целом опасные мутации распространённых вирусов случаются не очень часто — например, даже грипп вызвал смертельную пандемию только один раз, в начале XX века. }

— Коронавирус SARS-CoV-2 — новая для человека инфекция. Она перешла в человеческую популяцию только два года назад. До этого коронавирус этого типа циркулировал в основном среди летучих мышей. Эти животные обладают очень сильной противовирусной защитой, организм летучих мышей заточен для противостояния инфекциям, поскольку они живут в очень скученных колониях. Вирусы, которые способны пробить эту защиту, должны также быть вооружены очень серьёзными системами преодоления противовирусного иммунитета. И когда такой вирус попадает к человеку, он вызывает тяжёлые заболевания, потому что человеческая иммунная противовирусная система слабее, чем у летучих мышей.

Однако, попав в организм человека, такой вирус тоже должен приспособиться к новым условиям. Поэтому первые фазы эволюции вируса — это накопление таких мутаций, которые будут приводить к его более эффективному размножению в организме человека. Это сопровождается ростом инфекционности и усилением патогенных свойств. Когда вирус активно размножается, он приспосабливается к организму человека и действует более эффективно.

Вторая фаза эволюции вируса — аттенуирование. Это приспособление вируса к организму при ослаблении его патогенности.

При этом инфекционность может расти, потому что вирусу важно быстро распространяться на новом хозяине. Однако излишняя патогенность ему не нужна. Дело не в том, что вирус якобы знает, что ему невыгодно убивать человека. Нет, просто это качество — патогенность — не востребовано в организме человека и поэтому постепенно ослабевает при накоплении мутаций. В результате вирус вызывает всё меньше летальных исходов и тяжёлых случаев.

\begin{fancyquotes}
    Сейчас штамм коронавируса «омикрон» является примером второй фазы эволюции вируса, наблюдается его аттенуирование при одновременном увеличении заразности. Итогом должно стать превращение коронавируса в обычное сезонное вирусное заболевание.
\end{fancyquotes}

В случае с «омикроном» примечательна внезапность его появления, он сразу накопил 32 мутации по сравнению с предыдущим штаммом. В Африке очень много иммунодефицитных людей, и, по всей видимости, новый штамм возник в организме именно такого человека. В его организме он прошёл тот путь эволюции, который обычно вирус проходит через большую цепочку заражений. В итоге в ускоренном режиме этот вирус превратился в менее патогенный вариант.

Впрочем, я бы не хотел никого расхолаживать, говоря о том, что бояться нечего. Нет. Пока что это — лишь оптимистичный сценарий. Мы пока не имеем достаточного числа случаев заболевания этим штаммом, чтобы заявлять о его низкой опасности. Да, по предварительным данным, пока что от него никто не умер. Но нужно понаблюдать за тем, как будет развиваться ситуация, и продолжать вакцинироваться. Потому что даже если оптимистичный сценарий верный, нельзя исключать, что штамм «омикрон» неожиданно исчезнет, как исчез в Японии штамм «дельта».

По видимости, у коронавируса есть механизм самоуничтожения. Это может случиться и со штаммом «омикрон», а ему на смену придут более болезнетворные варианты.

{\bf — Как вакцинация влияет на мутации вируса — насколько она снижает их вероятность, учитывая, что вакцинированные тоже болеют, пусть и в лёгкой форме?}

— Особенность этого коронавируса в том, что даже очень иммунные люди, которые перенесли заболевание или вакцинированы, могут заразиться повторно, даже не чувствуя при этом симптомов. При этом вирус будет выделяться из носоглотки. Однако вероятность мутаций вируса в такой ситуации всё же не очень велика. Гораздо чаще новые варианты вируса возникают в организмах больных людей, особенно иммунодефицитных.

{\bf — Вы упомянули феномен исчезновения «дельты» в Японии. Не могли бы рассказать об этом подробней? Случалось ли подобное когда-то раньше? }

— Нет, это новая гипотеза одного из японских исследователей. Гипотеза, на первый взгляд, экстравагантная. Потому что обычно эволюция идёт таким путём, что выживает сильнейший — наиболее жизнеспособный. А в этом случае произошло, напротив, эволюционное самозатухание вируса. Согласно гипотезе, это связано с мутацией в гене NSP-14. Это один из неструктурных белков коронавируса, не входящий в состав вирусной частицы. Он нужен для поддержания репликации вируса, корректирует правильность считывания генома, исправляет ошибки. Если этот белок не функционирует, то вирус начинает с большой скоростью накапливать мутации, включая летальные. Они приводят к тому, что вирус уже не может размножаться.

Не знаю, подтвердится ли эта гипотеза, однако, когда в Японии секвенировали варианты коронавируса до того, как он исчез, оказалось, что там действительно были мутации белка NSP-14.

Более того, предыдущая вспышка SARS-1 в 2003 году тоже затухла сама по себе, её даже не успели подавить вакциной.

Что касается «омикрона», я пока не видел никаких данных о том, что у него повреждён белок NSP-14. Однако этого нельзя исключать, это объяснило бы скорость накопления в нём мутаций.

{\bf — Ещё говорят, что штамм «омикрон» очень сильно отличается от других штаммов и что в нём нашли элемент генома человека, который также есть в вирусе простуды. И что это делает его менее заметным для иммунной системы. }

— Нет, что это элемент генома человека — это ерунда. Это маленькая вставка. И когда мы говорим о последовательности, допустим, трёх аминокислот, такая последовательность может встречаться и в человеке, и в растении — \explain{где угодно}{anywhere}.

{\bf — Недавно вы сказали, что омикрон-штамм может выступить в качестве естественной вакцины. А были такие прецеденты раньше?}

—  Может быть, были, но они не зафиксированы. Однако что такое «живая вакцина»? Это просто вирус, который не имеет высокой патогенности, но вызывает иммунный ответ. Обычно такой непатогенный штамм делают искусственно. Например, полиомиелитная вакцина была создана путём селекции, были отобраны вирусы, которые утратили способность поражать нервные клетки. При этом они хорошо размножаются в кишечнике и вызывают стойкий иммунитет против полиомиелита. Живая вакцина против кори тоже была создана на основе патогенного вируса кори, который приобрёл свойства непатогенности.

Природа тоже может создавать такие вирусы. И природная аттенуация вирусного штамма — это, по сути, именно такой процесс. Такой вирус вытесняет патогенные варианты и создаёт иммунную прослойку, которая уже не позволяет новому патогенному варианту вызвать заболевание.

{\bf — Ранее в Университете Глазго провели исследование, в результате которого выяснилось, что люди очень редко болеют одновременно двумя вирусными заболеваниями. Однако точное объяснение этого тогда найти не удалось. Есть ли сейчас какие-то гипотезы на этот счёт? И можно ли использовать «конкуренцию» вирусов в полезных целях, существуют ли подобные проекты? }

— Да, есть такое правило, хотя из него тоже бывают исключения. В основе этого явления лежит феномен интерференции. Когда человек или животное заражается вирусом, в ответ в организме вырабатывается интерферон — противовирусный белок. Он циркулирует по всему организму, клетки в ответ вырабатывают противовирусное состояние. И другому вирусу будет уже очень трудно внедриться в организм в это время. Однако некоторые вирусы вырабатывают противодействие интерфероновому механизму — тогда возможна сочетанная инфекция. Возможно также одновременное заражение двумя вирусами, когда первый вирус ещё не вызвал противовирусное состояние.

{\bf — А учёные думали над тем, чтобы как-то использовать этот интерфероновый механизм для борьбы с болезнями? }

— Конечно. Такие исследования проводились в 1970-е годы, я принимал в них участие. Есть целый ряд непатогенных вирусов, которые обычно обитают в кишечнике здоровых детей в возрасте от двух до пяти лет. Их испытывали как средство профилактики сезонных простудных вирусных инфекций. В начале 1970-х годов было проведено масштабное испытание более чем на 300 тыс. человек в шести городах СССР. В итоге заболеваемость ОРВИ среди привитых снизилась в 3,5 раза. Это очень хороший результат, такой же, как у специфических противовирусных противогриппозных вакцин.

\begin{fancyquotes}
    Примечательно, что профилактика, основанная на механизме интерференции, защищает не только от одного вируса, а от многих. Поэтому такая неспецифическая профилактика очень важна именно в случае появления новых инфекций, чтобы выиграть время до создания специфической вакцины.
\end{fancyquotes}

Тем более что такие препараты применяются в виде капель, перорально. Это не будет вызывать такого отторжения у антиваксеров.

{\bf — А сейчас разрабатываются новые препараты с этим принципом действия?}

— У нас есть панель таких вирусов. Но проблема в том, что такие вещи с большим трудом пробивают себе дорогу. Например, нам не удалось убедить использовать этот метод во время нынешней пандемии. Потому что наш подход очень дешёвый, он не очень интересен для бизнеса. Ведь на выпуск вакцин выделяются большие деньги.

Но я уверен, что мы вступаем в такое время, когда вирусы могут использоваться и в качестве оружия. Поэтому нужно иметь стратегический запас экстренных средств защиты. И надеюсь, что в итоге собранный нами запас непатогенных вирусов будет использован и признан в качестве такого средства.

{\bf — Раз вирусы могут конкурировать и вытеснять друг друга, не рассматривается ли научным сообществом, к примеру, идея создать генно-инженерными методами не опасный, но очень контагиозный штамм SARS-CoV-2, чтобы вытеснить те его штаммы, которые приводят к высокой летальности? }

— Это отличный вопрос. Вообще я считаю, чтобудущее вакцинологии — это живые вакцины. Просто сейчас мы идём проторённым путём, создаём традиционные вакцины. Либо это инактивированный вирус, либо какие-то генные инженерные продукты — на основе других вирусов.

При этом аттенуированные штаммы можно создать из любого болезнетворного вируса. Зная, какие гены участвуют в патогенезе, можно внести искусственные изменения и сделать из него такой безопасный вирус, который будет тем не менее иммуногенен и будет защищать от инфекции.

Но, к сожалению, пока что этот путь не всеми признаётся, широких работ в этом направлении не ведётся.

{\bf — В 2020 году в интервью RT вы рассказали о разработке модифицированных вирусов, которые вызывают гибель раковых клеток. Мы хотели бы вернуться к этой теме. Готовятся ли клинические испытания или они уже проводятся? }

— Да, эта работа продолжается, у нас есть большая панель онколитических вирусов. Почему нужна целая панель, а не один препарат? Потому что опухоли у людей очень индивидуальны. Допустим, у двух людей рак молочной железы одной и той же гистологической категории. Но на молекулярном уровне это совершенно разные заболевания, там разный набор повреждений. В том числе повреждаются такие гены, которые могут быть нужны какому-то конкретному вирусу для его репликации. И он уже не может лечить данную опухоль. Но может другой — его нужно подобрать из панели.

Конечно, такой сложный препарат трудно быстро внедрить, нужно пройти долгие этапы проверок. К сожалению, регуляторика — узкое место для биотерапии. На самом деле правила нужно очень сильно менять, чтобы создание и внедрение таких препаратов происходило быстрей.

Пока что проведены только доклинические испытания, их итоги подводятся. Возможно, что к концу года будет какое-то заключение. После этого встанет вопрос о том, как организовать клинические испытания. Нужно будет найти инвесторов, собрать пациентов-добровольцев… Всё это затягивается на годы.

Тем не менее по своему опыту мы знаем, что эта терапия действует и абсолютно безопасна.

{\bf — Удалось ли собрать какую-то предварительную статистику? }

— Собрать корректную статистику пока невозможно, потому что все случаи, с которыми мы имели дело, — это случаи четвёртой стадии заболевания.Просто потому, что только на этой стадии возможны такие эксперименты, когда уже исчерпаны все традиционные возможности.Иногда даже не успеваем препарат дать, как человек уже умер.

{\bf — Люди соглашаются принимать экспериментальный препарат, потому что им нечего уже терять? }

— Да. Но всё равно такие вещи находятся в «серой» зоне законодательства. Вообще-то так нельзя делать, но мы всё равно делаем. И какая тут может быть статистика?Мы видим только отдельные случаи, когда действительно это лечение помогает, когда люди долго живут. И есть много случаев, когда улучшается состояние. Это не значит, что человек выздоровел. Потому что после, предположим, 16 курсов химиотерапии ресурсы организма всё равно уже сильно ограничены. А статистику мы получим, когда проведём клинические испытания.

{\bf — Буквально в двух словах напомните, пожалуйста, о принципе действия таких препаратов.}

— Принцип такой: опухолевые клетки высокочувствительны к любым вирусам. Потому что опухолевая клетка — это не часть организма, это уже новый одноклеточный организм внутри организма. Который начинает конкурировать там со своим хозяином, развиваться в виде опухоли. По мере своей эволюции он утрачивает свойства, нужные для поддержания собственно организменных функций, включая механизмы противовирусной защиты клетки. Онколитический вирус — это вирус, не вызывающий заболевание, поэтому его можно использовать для лечения рака. Однако не каждый вирус может конкретную опухоль уничтожить. Поэтому надо иметь много разных вирусов и подбирать их под конкретного пациента.

Внутри опухоли формируется иммуносупрессивное состояние — иммунная система не может туда проникнуть. Но когда в опухоли начинает размножаться вирус, возникает воспаление, которое сопровождается выработкой массы белковых факторов. Они привлекают в опухоль компоненты иммунной системы, которые начинают усиленно атаковать раковые клетки и довершают действие вируса. Это более-менее естественный способ уничтожения раковых клеток, поэтому он практически не даёт побочных эффектов, в отличие от химии.

{\bf — В одной из своих лекций вы рассказывали, что люди начали обращать внимание на позитивное влияние вирусных заболеваний на онкологических больных около ста лет назад. Может быть, с этим явлением могут быть отчасти связаны истории о неожиданном и чудесном излечении от рака? }

— Такие случаи трудно задокументировать, для этого нужно было бы изучить антитела в крови таких пациентов. Хотя, конечно, за такими случаями стоят какие-то механизмы, не исключено, что и какой-то вирус

{\bf — Где-то в мире уже используются такие препараты для лечения пациентов? }

— Сейчас во всём мире наблюдается бум этого направления исследований. Но, ксожалению, сейчас каждый разработчик делает один препарат на основе одного вируса. И в итоге выясняется, что он эффективен только для 15—20\% пациентов. Мы единственные используем целую панель вирусов, в этом наше преимущество. В США есть препарат на основе рекомбинантного вируса герпеса, который применяется сейчас для лечения развитых форм меланомы. Однако прорыва в лечении он не дал — именно по той причине, что одного вируса мало, нужно в каждом случае перебирать варианты. Либо вводить коктейль. Мы считаем, что эффективнее всего использовать коктейль из трёх-пяти разных вирусов.

{\bf — Когда препараты такого типа пройдут все испытания, как будет проводиться лечение? Придётся ли пациентам ездить в какой-то один центр или можно будет внедрить эту терапию по всей стране?}

— Надеюсь, что будут клиники, отделения, которые будут специализироваться на такой терапии. Есть много способов введения препарата — нужно выбирать в зависимости от формы рака. Это станет огромным направлением для исследований, для терапии.

{\bf — А как сейчас настроено сообщество врачей-онкологов, они ждут появления на рынке таких препаратов? }

— Среди специалистов есть понимание, что онкология зашла в тупик в вопросе лечения развитых форм рака. Поэтому люди с большим энтузиазмом воспринимают все новые возможности, ждут, когда препараты пройдут испытания и регистрацию.









\end{document}
