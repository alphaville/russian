\chapter{История}

\section{Море долго выбрасывало тр\'{у}пы}

\textit{70 лет назад цунами разрушило советский город. Как власти скрывали смерть тысяч человек?}

\textit{Источник: \url{https://lenta.ru/articles/2022/11/05/tsunami1952/}}

Р\'{о}вно 70 лет назад, 5 ноября 1952 года, мощное цунами разрушило город Северо-Курильск на острове Парамушир в Сахалинской области, и смыло несколько соседних \ed{посёлков}{посёлок}{settlement}. Точное число погибших до сих пор неизвестно. По официальной версии, в одном только Северо-Курильске пог\'{и}бло 1200 человек из 6000 населения, а \explain{с учётом}{taking into account} соседних населённых пунктов эта цифра превышает 2000 человек. Историки говорят, что погибших могло быть \explain{в раз\'{ы} больше}{many times more}. Трагедию засекр\'{е}тили: п\'{е}рвые данные о ней опубликовали только в начале 1990-х, но некоторые военные архивы недоступны до сих пор. Как произошла самая масштабная забытая катастрофа СССР и как её пытались забыть --- в материале «Ленты.ру».

\textbf{«Накануне вся природа затихла»}

Ночью 5 ноября 1952 года в Тихом океане случилось мощное землетрясение магнитудой 8,3 по шкал\'{е} Рихтера. Его эпицентр находился приблизительно в 200 километрах от Петропавловска-Камчатского и глубок\'{о} под землёй. Из-за землетрясения и возникло цунами, высота волн достиг\'{а}ла 10-18 метров. Сам Петропавловск-Камчатский не пострадал --- город спас узкий проход в \ed{бухту}{бухта}{bay} Ав\'{а}чинская, однако землетрясение местные жители почувствовали.

Первыми об ударе узнали моряки стоявших на \ed{рейде}{рейд}{raid} \ed{рыболовецких судов}{рыбол\'{о}вные суд\'{а}}{fishing boats}.

«5 ноября 1952 года я вместе с другими рыбак\'{а}ми наход\'{и}лся в море на л\'{о}ггере (ловили рыбу), точнее --- наход\'{и}лись в ковше, -- \ed{докладывал}{докладывать/доложить}{to report} в ходе \ed{допроса}{допрос}{interrogation} рыбак Павел Смолин. --- Рано утром на логгере чувствовалось большое \explain{содрогание}{shudder} корабля. Я и другие рыбаки приняли это за землетрясение... В ночь на 5 ноября имелось штормовое предупреждение в 6-7 баллов. После землетрясения наш логгер под командой капитана Лымаря вышел в море первым. Это было около четырех часов утра. Идя по Второму \ed{прол\'{и}ву}{прол\'{и}в}{strait} в районе Банжовского мыса, наш логгер накрыла первая волна высотою несколько метров. Находясь в кубрике, я почувствовал, что корабль как бы опустило в яму, а затем выбросило высоко вверх. Через несколько минут последовала вторая волна, и повторилось то же самое. Затем корабль пошел спокойно, бросков не ощущалось».

Около 18 часов военная радиостанция передала экипажу, на котором был Смолин, что им необходимо вернуться в Северо-Курильск. Он связ\'{а}лся с другим с\'{у}дном и от его рад\'{и}ста узнал, что Северо-Курильска больше нет -- город смыло волной.

\begin{fancyquotes}
    Ещё в Охотском море, не доходя до остров\'{о}в Парамушир и Шумшу, команда логера, в том числе и я, увидели плывущие навстречу крыши домов, бревна, ящики, бочки, кровати, двери. По \ed{распоряжению}{распоряжени}{order, command} капитана команда была выставлена на \ed{палубе}{п\'{а}луба}{deck of ship} по обе стороны борт\'{о}в и на носовой части с целью спасения людей, оказавшихся в море. Но из людей никого обнаружено не было. На протяжении всего пути в пять-шесть миль мы наблюдали все ту же картину: плавающие бочки, ящики и т.п. плотной массой.\\

    \begin{flushright}
        \it
        Павел Смолин,
        \\
        рыбак
    \end{flushright}
\end{fancyquotes}

\explain{Чётко отследить}{to trace clearly}, когда именно начал\'{о}сь цунами, было сложно, рассказывала Валентина Швецова, дочь Михаила Альперина, управляющего Северо-Курильским рыбтрестом.

\begin{quote}\it
    «Где-то накануне, может быть, за сутки абсолютно вся природа вдруг зат\'{и}хла… И старожилы, которые много лет провели на Северных Курилах, поняли, что не к добру --- многие уже тогда ушли в горы. Отец вернулся с работы и сказал, чтобы мы все держались вместе. Все понимали: что-то будет, но никто не знал --- что», --- вспоминала она.
\end{quote}


Альперин погиб во время второй волны --- \explain{предположительно}{presumably}, его ударило чем-то тяжёлым по голове.

\textbf{«Человек висел на мачте крана»}

Мирно спавших жителей Северо-Курильска около 4 часов утра разбудили подземные \ed{толчки}{толчок}{tremour} --- они продолжались около 30 минут. Однако для местных это было привычно, поэтому многие остались в домах. Другие обратили внимание, что море отошло от б\'{е}рега прим\'{е}рно на полкилометра, и часть людей решила отпр\'{а}виться в г\'{о}ры --- в основн\'{о}м это были рыбак\'{и}.

«Кто-то кр\'{и}кнул, что тёмная полос\'{а} у \ed{кромки}{кромка}{edge} воды появилась, значит, она отступает. Действительно, волны стали откатываться \explain{всё быстрее и быстрее}{faster and faster}. Вот уже вода отодвинулась до прежнего уровня, а потом обнажилось дно бухты с камнями, которых прежде никто не видел», --- вспоминал сахалинский литератор Александр Кутелев, которому на момент катастрофы было четыре года.

Затем, через 35-40 минут после землетрясения, на остров обрушилось цунами. Из-за шума, криков и паники многие люди не сразу поняли, что происходит.

\begin{quote}
    \it
    Например, Константин Понедельников, который приехал в город на заработки, сначала услышал слово «война» и лишь потом понял, что бежавшие от залива люди кричали: «Волна! Вода!»
\end{quote}

Одним из тех, кто успел сориентироваться и отправиться на \explain{возвышенность}{elevation, hill, height}, стал единственный на тот момент прокурор Северо-Курильского района Дмитрий Гончаров. Он н\'{а}чал стуч\'{а}ть в сос\'{е}дние дом\'{а}, чтобы отправить людей на \ed{сопки}{сопка}{(from Wikipedia) A sopka is a monumental tumulus resembling a kurgan, referring specifically to those built by the Novgorodian Sopka cultures of the early Middle Ages.}. До сопок в итоге волна не добралась, но успела забрать с собой тех, кто бежал последним.

Высота волны примерно \ed{равнялась}{равняться + \textit{чему}}{is equal to} пятиэтажному дому. Дети очевидцев катастрофы рассказывали, что многие отказывались бежать из города --- они залезали на крыши домов, и их сносило в море.

Ещё через 20 минут город накрыла вторая волна --- ещё большая, достигавшая десятиметровой высоты. Тогда погибли те, кто \explain{спустился}{descended} с возвышенности в город, чтобы осмотреть свои \ed{жил\'{и}ща}{жил\'{и}ще}{dwelling} и забрать вещи.

\begin{quote}
    \it «Она \ed{нанесл\'{а}}{нанесла разрушения}{caused damage/destruction} особо сильные разрушения, смывая все \ed{постройки}{постройка}{building} на пут\'{и}. Позад\'{и} волны на месте оставались лишь цементные фунд\'{а}менты домов», --- писал в своей стать\'{е} «Курильская катастрофа полвека назад» Андрей Никонов, доктор геолого-минералогических наук Института физики Земли им. Г.А. Гамбурцева РАН.
\end{quote}

Другой \explain{очевидец}{eyewitness}, Лев Добмровский, вспоминал о последствиях второй волны так: «Все мы были на взводе... Повсюду на земле были раскиданы мертвые тела... Один человек висел на мачте крана. Неразрушенным был один дом, сделанный из плит. Но уцелела только его основа, а крышу, двери и окна вырвало».

Позднее пришла и третья волна --- высотой 5-8 метров. От нее сильно пострадал поселок Океанский на берегу Парамушира.


Один из живших там капитанов, дом которого был уничтожен, вспоминает: «После сильного сотрясения вновь были колебания. В это время услышал крик людей! Быстро открыл дверь, и в тот же момент вода сшибла меня и подхватила до потолка. Когда движение крыши остановилось, я соскочил наземь, побежал вверх по холму. Позднее обнаружилось, что крыша застряла в полукилометре от берега. На холме пробыли два или три дня, пока не подошли корабли из Петропавловска».

Третья волна уносила с собой остатки того, что разрушили первые две, и смывала людей, которые продолжали бежать в горы. Как писал в отчете замначальника Сахалинской областной милиции Смирнов, многие старались спасти детей, женщин и стариков, несмотря на опасность.

\begin{fancyquotes}
    \it
    Вот две девушки ведут под руку старушку. Преследуемые приближающейся волной, они стараются бежать быстрее к сопке. Старушка, выбившись из сил, в изнеможении опускается на землю. Она умоляет девушек оставить ее и спасаться самим. Но девушки сквозь шум и грохот надвигающейся стихии кричат ей: «Мы тебя все равно не оставим, пусть все вместе утонем». Они подхватывают старушку на руки и пытаются бежать, но в этот момент набежавшая волна подхватывает их и так всех вместе выбрасывает на возвышенность. Они спасены.

    \begin{flushright}\it
        из справки подполковника милиции Смирнова
    \end{flushright}
\end{fancyquotes}

Понедельников писал, что для перехода к сопкам были проложены деревянные мостки. Он вспоминал, что рядом с ним бежала женщина с пятилетним мальчиком: «Я схватил ребенка в охапку --- и вместе с ним перепрыгнул канаву, откуда только силы взялись. А мать уже по досочкам перебралась».

На самих сопках находились армейские блиндажи, где проходили учения, --- там выжившие и провели несколько дней. Когда волны наконец-то отступили, к Северо-Курильску вылетело несколько самолетов-разведчиков из Петропавловска-Камчатского. Они осмотрели местность и сделали снимки.

Затем самолеты начали сбрасывать теплую одежду, палатки и еду для людей, которые жгли костры, чтобы согреться, --- они бежали на сопки в том, в чем спали. Когда на остров обрушилось цунами --- большинство было в одном нижнем белье.

\textbf{«Трупы погибших долго выбрасывало»}

Эвакуация пострадавшего Северо-Курильского района началась 6 ноября. Во Второй Курильский пролив начали прибывать пароходы из Петропавловска и Владивостока, всего около 40 судов. До 11 ноября все население вывезли с острова.

Как вспоминал начальник Камчатской вулканологической станции АН СССР Борис Пийп, который измерял высоту волны, захлестнувшей город, эвакуированные жители в течение трех дней не могли спокойно спать и дергались из-за любого шороха.

«Рассказывают о многих трагических случаях. Например, двое моряков в трусах и тельняшках находились в воде, держась за обломки дома, с 5 часов утра до 5 часов дня. Когда их спасли, один из них, выйдя на берег, упал мертвым, а другой остался живым. Отец, залезший на крышу, не мог спасти дочь, оставшуюся на чердаке под крышей, которая сперва кричала, умоляла отца спасти, но затем замолкла. Трупы погибших море долго выбрасывало, усеивая ими берега», --- писал он.

Пийп оценил число погибших в 4000 человек, что почти в два раза превышает данные властей. Некоторые историки в свою очередь считают, что в официальной версии не учтены военнослужащие и рабочие, приехавшие на заработки из Северной Кореи, --- на многих документах до сих пор гриф «секретно». Некоторые современные историки говорят о 8000 погибших.

В спасательной операции участвовали многие выжившие. Среди них был бывший шкипер Алексей Мезис, который успел забраться на возвышенность.

«Когда к Северо-Курильску пробирались, боялись, как бы не напороться на что-нибудь такое, что может повредить или борт, или винт, --- вспоминал он. --- Увидели кран береговой. Кран упал в море, и вот такая картина: из воды торчит стрела его с гаком, который для подъема груза, и шкентелем --- тросом, и этот трос так согнут, что в нем зажата рука молодого парня; он висел лицом к стреле и, видимо, бился о нее --- лицо разбито, и висел-то в трусах и майке, босой. Хотели мы его вытащить. Не получилось. Сошли на берег, здесь на брекватере тоже... почему не смыло... На самом краю лежала мертвая кореянка, видимо, беременная --- большой живот... Отошли, а дальше, из полузамытой щебенкой и песком ямы торчали рука и ноги. Жуть...»

По словам Мезиса, поиск людей в море и доставка их на суда шли около четырех суток. На берегу к этому моменту убрали трупы.

«Люди были уже организованней, несколько спокойней, некоторые одеты в то, что сбросили с самолетов, иные собрали узелки с кое-какими продуктами. Но это, вероятно, были жители не Северо-Курильска, самого густонаселенного района, который охватило волной примерно на две трети, а его окраин --- наводнение их не тронуло, а только напугало», --- рассуждал он.

Он указывал, что помощь в спасении людей предлагали и американцы, однако советская сторона им отказала: «Во-первых, те бесплатно ничего не делают, а во-вторых, посчитали, что своих судов вполне хватит, чтобы эвакуировать людей».

Катера рыбтреста ушли в открытое море --- туда тоже унесло людей. Некоторые из них смогли выжить и дрейфовали на крышах и обломках зданий, нуждаясь в спасении.

\begin{fancyquotes}
    Мать и малолетняя дочь Лосевы, спасаясь на крыше своего дома, волной были выброшены в пролив. Взывая о помощи, они были замечены находящимися на сопке людьми. Вскоре там же, недалеко от плавающих Лосевых, была замечена на доске маленькая девочка, как потом оказалось, чудом спасшаяся трехлетняя Набережная Светлана, которая то исчезала, то вновь появлялась на гребне волны. Время от времени она заправляла ручонкой развеваемые ветром русые волосы, что указывало, что девочка жива.\\

    \begin{flushright}
        \it
        из справки подполковника милиции Смирнова
    \end{flushright}
\end{fancyquotes}

Таким образом удалось спасти 192 человека. Люди получили еду и теплые вещи с воинских складов. Пострадавших разместили в военном госпитале. По одной из версий, эвакуацию ускорил звонок Иосифа Сталина в Сахалинский обком, однако официального подтверждения этой информации обнаружить не удалось.

\textbf{«Города не было»}

Как рассказывает Швецова, выжившие видели последствия цунами с высоты сопок.

«Города не было. Превратился в черно-белое пятно. Рыбокомбинат, флот, больница, школа... Все снесло. Осталась только электростанция, она и сейчас существует. Тело отца, к счастью, нашли, но многие так и остались без вести пропавшими», --- говорит она.

По словам доктора физико-математических наук Виктора Кайстренко, от Северо-Курильска остались лишь окраины на больших высотах. В самом городе устояли только два бетонных сооружения: арка ворот стадиона и памятник Герою Советского Союза летчику Талалихину.

Писатель Аркадий Стругацкий, служивший тогда военным переводчиком и находившийся в командировке, писал брату о последствиях стихии. Он участвовал в ликвидации последствий.

\begin{fancyquotes}
    Постройки были разрушены, весь берег усеян бревнами, обломками фанеры, кусками изгородей, воротами и дверьми. На пирсе стояли две старые корабельные артиллерийские башни, их поставили японцы чуть ли не в конце русско-японской войны. Цунами отшвырнул их метров на сто. Когда рассвело, с гор спустились те, кому удалось спастись --- мужчины и женщины в белье, дрожащие от холода и ужаса. Большинство жителей либо утонули, либо лежали на берегу вперемежку с бревнами и обломками.\\

    \begin{flushright}
        Аркадий Стругацкий
        \\
        из письма брату
    \end{flushright}
\end{fancyquotes}

В разгар стихии спасать пытались не только людей и личные ценности --- милиционер Смирнов в своей справке отдельно отмечал мужество тех, кто активно участвовал в сборе государственного имущества. Тем не менее отчеты милиции говорят и о случаях мародерства --- например, в одном из разрушенных учреждений взломали сейф и украли 270 тысяч рублей.

О мародерах писал и Мезис: «Например, когда уже в Ворошилове находились, у нас одна с океанского рыбокомбината тоже, как положено, получила помощь и начала скупать в магазинах вещи, да все подороже, и золото с серебром. На нее обратили внимание, проследили, что она скупает. Ну конечно, справки навели: получила три тысячи, а накупила на все тридцать (...) И выяснилось, они с мужем, когда океанский комбинат смыло, увидели на берегу сейф. Взломали его, а там --- зарплата всего коллектива, которую завезли да не успели выдать. Деньги эти они с мужем поделили, и она --- в Ворошилов, а он во Владивостоке остался. Ну его там взяли».

Как писал старший лейтенант Дерябин, мародерством занимались не только простые жители: «Воспользовавшись стихийным бедствием, военнослужащие гарнизона, напившись разбросанного по городу спирта, коньяку и шампанского, начали заниматься мародерством...»

Например, военные с погранпункта забрали по указанию старшего лейтенанта мешок крупы, мешок муки, бочку растительного масла и две подушки, чтобы увезти в часть, но их задержали.

В своей справке Смирнов указывал, что помимо домов на возвышенностях сохранились электростанция, часть флота, склад Северо-Курильского рыболовпотребсоюза и военторга, а также несколько десятков лошадей, коров и свиней, однако их хозяев установить не удалось.

Он также упомянул несколько поселков, которые в разной степени пострадали от цунами. Где-то людей в тот момент вообще не было, но, к примеру, в поселке Океанский погибли 460 человек, выжили 542.

«Необходимо отметить, что, в связи с внезапной эвакуацией пограничников, в первые дни в ряде поселков --- Шелехове, Океанском, Рифовом, Галкине и на острове Алаид --- среди населения имела место паника, вследствие чего в этих пунктах все государственное и общественное имущество было брошено на произвол судьбы», --- вспоминал он.


\textbf{«Местное население на нас косилось»}

Практически все выжившие покинули пострадавший остров с 7 по 11 ноября. С 7 по 8 ноября гражданское население отправляли во Владивосток и Находку, военных с семьями с 9 по 10 ноября --- в Корсаков, авиачасти и погранвойска --- в Петропавловск-Камчатский.

Пострадавшим от цунами в два захода перечислили средства для оказания единовременной помощи. Как вспоминал Мезис, помощь была выдана для того, чтобы люди смогли выехать на материк, однако тем, кто быстро уехал, не выдали последнюю зарплату. Сам он получил зарплату только в середине декабря --- это и удерживало его на острове. Помимо денег пострадавшие получили одежду --- новую и поношенную.

\begin{fancyquotes}
    А помощь пострадавшим по тем временам была существенная --- в пределах 3-3,5 тысячи рублей. Там, на Курилах, некоторые жили в общежитиях, ничего у них не было, кроме той одежонки, что на них. А тут собрались дружки в роли свидетелей и давай говорить комиссии: мол, у него и то было, и это. Один, например, твердил всем, что у него на острове якобы было кожаное пальто, кожаные перчатки, и все, дескать, в море смахнуло. Ну получил 3 тысячи и на самом деле стал расхаживать в кожаном пальто, и перчатки надел кожаные с длиннющими пальцами, и туфли немыслимые. Попугаем его прозвали, но своего-то он добился\\

    \begin{flushright}
        Алексей Мезис,
        \\
        бывший шкипер
    \end{flushright}
\end{fancyquotes}

Мезис писал, что в Ворошилове с завистью относились к людям из Северо-Курильска, потому что они получали бесплатное питание и безвозмездную помощь в виде разных товаров: «Местное население стало на нас коситься. Мол, они ничего купить не могут, а нам все новые товары прут. Нас даже на поездах туда-сюда бесплатно возили. Тем, кто вернулся на Сахалин, было предоставлено и жилье».

Однако, по данным других источников, люди долго не могли добиться компенсаций. В лучшем случае им удавалось подтвердить свой рабочий стаж.

Общая сумма ущерба оценивалась в 285 миллионов советских рублей, однако историки и краеведы считают, что это неполная цифра, так как она не учитывает имущество военных и расходы на эвакуацию.

\textbf{«Население не имело понятия о правилах поведения»}

К 1952 году в СССР не существовало специальной службы, которая оповещала бы население о подобных катаклизмах. Из дневника Пийпа следует, что сигналы бедствия пыталась передать сейсмологическая служба, но на них не отреагировали должным образом: «Радиостанция непрерывно передавала SOS, но как-то бестолково, так что Петропавловск ничего не мог понять».

Мезис отмечает, что на материке не имели понятия о том, что случилось в Северо-Курильске: выжившие писали письма, а в ответ получали вопросы --- что случилось, почему они там оказались. Засекреченность не позволила узнать о трагедии не только простым гражданам, но и ученым. Никонов писал, что сведения о цунами 1952 года приходилось собирать по крупицам. Газеты тоже молчали --- вся страна жила обычной жизнью. Рассуждая о том, что американцы откуда-то узнали о трагедии, раз предлагали свою помощь, Мезис задавался вопросом, от кого же у нас делали секреты.

Как считает Никонов, не последнюю роль в гибели такого большого количества людей сыграло то, что советские власти скрывали информацию о других катастрофах в стране.

\begin{fancyquotes}
    Тогдашнее население островов о правилах поведения во время сильных землетрясений и возможных цунами понятия не имело (если в стране катастрофы не происходят, то чего же к ним готовиться?). Это и сыграло роковую роль. Ибо после прекращения толчков люди возвращались в дома. А дома стояли, и это было другим следствием неведения, на низких террасах, то есть вблизи уреза воды и береговой линии.

    \begin{flushright}
        Андрей Никонов,
        \\
        доктор геолого-минералогических наук
    \end{flushright}
\end{fancyquotes}

Тем не менее как раз после катастрофы 1952 года в СССР было принято решение создать службу предупреждения цунами. С 1956 года сейсмическую часть работы выполняла сейсмическая станция «Южно-Сахалинск», еще через три года к ней присоединили станцию «Петропавловск», а затем еще четыре станции на Курильских островах. В 1958-1959 годах в регионе заработали три цунамистанции и две мареографные

установки. С 1961 года к наблюдениям за волнами цунами были привлечены все метеостанции Курильских островов.

К моменту написания своей статьи в 2005 году Белоусов описывал состояние системы предупреждения о цунами на тихоокеанском побережье как «печальное из-за нищенского финансирования». Он рассуждал, что единственное, что защищает людей от опасности --- это то, что населенных пунктов там практически нет.

С 2006 года началось восстановление системы предупреждения о цунами. Она сработала в 2020 году --- тогда 400 жителей Северо-Курильска эвакуировали из-за угрозы цунами после сильного землетрясения в Тихом океане. Однако на тот момент метеорологические службы Японии не объявляли предупреждений, а власти США отменили его для Гавайев.

Северо-Курильск после катастрофы перестроили, но полностью город оправиться не смог. Город отодвинули от океана, насколько это позволил рельеф, но в результате он оказался в еще более опасном месте --- рядом с вулканом Эбеко, одним из самых активных на Курилах.

Выбросы вулкана Эбеко регулярно продолжаются с 2016 года. Рекордный зафиксировали 31 августа 2018 года. Тогда столб дыма из нового жерла вулкана, образовавшегося в 2017 году, взметнулся на высоту шесть километров. В 2022 году вулкан выбросил пепел 1 сентября, во время школьной линейки, и 30 октября, в День автомобилиста. Местные жители шутят, что так природа поздравляет их с праздниками.

Однако, судя по всему, местные жители привыкли к такому соседству. Например, они катаются по склонам вулкана на сноуборде. За активностью вулкана следит специальная служба, в случае извержения она должна предупредить население.

После цунами 70-летней давности часть предприятий рыбной промышленности решили не восстанавливать. В 1961 году начали приходить в упадок и частично восстановленные поселки, так как в этих водах стало меньше сельди иваси. Люди начали уезжать на материк.

По данным на 2021 год, в Северо-Курильске проживало 2374 человека. В городе есть площадь памяти, там расположена табличка, на которой почти столько же имен --- 2236. Это те, чьи имена удалось установить после цунами.